\begin{center}
\section*{غزل ۲۲۹: پیش رویت دگران صورت بر دیوارند}
\label{sec:229}
\addcontentsline{toc}{section}{\nameref{sec:229}}
\begin{longtable}{l p{0.5cm} r}
پیش رویت دگران صورت بر دیوارند
&&
نه چنین صورت و معنی که تو داری دارند
\\
تا گل روی تو دیدم همه گل‌ها خارند
&&
تا تو را یار گرفتم همه خلق اغیارند
\\
آن که گویند به عمری شب قدری باشد
&&
مگر آنست که با دوست به پایان آرند
\\
دامن دولت جاوید و گریبان امید
&&
حیف باشد که بگیرند و دگر بگذارند
\\
نه من از دست نگارین تو مجروحم و بس
&&
که به شمشیر غمت کشته چو من بسیارند
\\
عجب از چشم تو دارم که شبانش تا روز
&&
خواب می‌گیرد و شهری ز غمت بیدارند
\\
بوالعجب واقعه‌ای باشد و مشکل دردی
&&
که نه پوشیده توان داشت نه گفتن یارند
\\
یعلم الله که خیالی ز تنم بیش نماند
&&
بلکه آن نیز خیالیست که می‌پندارند
\\
سعدی اندازه ندارد که چه شیرین سخنی
&&
باغ طبعت همه مرغان شکرگفتارند
\\
تا به بستان ضمیرت گل معنی بشکفت
&&
بلبلان از تو فرومانده چو بوتیمارند
\\
\end{longtable}
\end{center}
