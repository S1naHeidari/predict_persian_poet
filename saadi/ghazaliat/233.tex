\begin{center}
\section*{غزل ۲۳۳: روندگان مقیم از بلا نپرهیزند}
\label{sec:233}
\addcontentsline{toc}{section}{\nameref{sec:233}}
\begin{longtable}{l p{0.5cm} r}
روندگان مقیم از بلا نپرهیزند
&&
گرفتگان ارادت به جور نگریزند
\\
امیدواران دست طلب ز دامن دوست
&&
اگر فروگسلانند در که آویزند
\\
مگر تو روی بپوشی و گر نه ممکن نیست
&&
که اهل معرفت از تو نظر بپرهیزند
\\
نشان من به سر کوی می‌فروشان ده
&&
من از کجا و کسانی که اهل پرهیزند
\\
بگیر جامه صوفی بیار جام شراب
&&
که نیک نامی و مستی به هم نیامیزند
\\
رضای دوست به دست آر و دیگران بگذار
&&
هزار فتنه چه غم باشد ار برانگیزند
\\
مرا که با تو که مقصودی آشتی افتاد
&&
رواست گر همه عالم به جنگ برخیزند
\\
به خونبهای منت کس مطالبت نکند
&&
حلال باشد خونی که دوستان ریزند
\\
طریق ما سر عجزست و آستان رضا
&&
که از تو صبر نباشد که با تو بستیزند
\\
\end{longtable}
\end{center}
