\begin{center}
\section*{غزل ۴۲۰: ما همه چشمیم و تو نور ای صنم}
\label{sec:420}
\addcontentsline{toc}{section}{\nameref{sec:420}}
\begin{longtable}{l p{0.5cm} r}
ما همه چشمیم و تو نور ای صنم
&&
چشم بد از روی تو دور ای صنم
\\
روی مپوشان که بهشتی بود
&&
هر که ببیند چو تو حور ای صنم
\\
حور خطا گفتم اگر خواندمت
&&
ترک ادب رفت و قصور ای صنم
\\
تا به کرم خرده نگیری که من
&&
غایبم از ذوق حضور ای صنم
\\
روی تو بر پشت زمین خلق را
&&
موجب فتنه‌ست و فتور ای صنم
\\
این همه دلبندی و خوبی تو را
&&
موضع ناز است و غرور ای صنم
\\
سروبنی خاسته چون قامتت
&&
تا ننشینیم صبور ای صنم
\\
این همه طوفان به سرم می‌رود
&&
از جگری همچو تنور ای صنم
\\
سعدی از این چشمه حیوان که خورد
&&
سیر نگردد به مرور ای صنم
\\
\end{longtable}
\end{center}
