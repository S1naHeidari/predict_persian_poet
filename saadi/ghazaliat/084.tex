\begin{center}
\section*{غزل ۸۴: ز من مپرس که در دست او دلت چونست}
\label{sec:084}
\addcontentsline{toc}{section}{\nameref{sec:084}}
\begin{longtable}{l p{0.5cm} r}
ز من مپرس که در دست او دلت چونست
&&
ازو بپرس که انگشت‌هاش در خونست
\\
وگر حدیث کنم تندرست را چه خبر
&&
که اندرون جراحت رسیدگان چونست
\\
به حسن طلعت لیلی نگاه می‌نکند
&&
فتاده در پی بیچاره‌ای که مجنونست
\\
خیال روی کسی در سرست هر کس را
&&
مرا خیال کسی کز خیال بیرونست
\\
خجسته روز کسی کز درش تو بازآیی
&&
که بامداد به روی تو فال میمونست
\\
چنین شمایل موزون و قد خوش که تو راست
&&
به ترک عشق تو گفتن نه طبع موزونست
\\
اگر کسی به ملامت ز عشق برگردد
&&
مرا به هر چه تو گویی ارادت افزونست
\\
نه پادشاه منادی زده‌ست می مخورید
&&
بیا که چشم و دهان تو مست و میگونست
\\
کنار سعدی از آن روز کز تو دور افتاد
&&
از آب دیده تو گویی کنار جیحونست
\\
\end{longtable}
\end{center}
