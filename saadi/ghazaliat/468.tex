\begin{center}
\section*{غزل ۴۶۸: چشم اگر با دوست داری گوش با دشمن مکن}
\label{sec:468}
\addcontentsline{toc}{section}{\nameref{sec:468}}
\begin{longtable}{l p{0.5cm} r}
چشم اگر با دوست داری گوش با دشمن مکن
&&
تیرباران قضا را جز رضا جوشن مکن
\\
هر که ننهاده‌ست چون پروانه دل بر سوختن
&&
گو حریف آتشین را طوف پیرامن مکن
\\
جای پرهیز است در کوی شکرریزان گذشت
&&
یا به ترک دل بگو یا چشم وا روزن مکن
\\
کیست کاو بر ما به بیراهی گواهی می‌دهد
&&
گو ببین آن روی شهرآرا و عیب من مکن
\\
دوستان هرگز نگردانند روی از مهر دوست
&&
نی معاذالله قیاس دوست از دشمن مکن
\\
تا روان دارد روان دارم حدیثش بر زبان
&&
سنگدل گوید که یاد یار سیمین تن مکن
\\
مردن اندر کوی عشق از زندگانی خوشتر است
&&
تا نمیری دست مهرش کوته از دامن مکن
\\
شاهد آیینه‌ست و هر کس را که شکلی خوب نیست
&&
گو نگه بسیار در آیینه روشن مکن
\\
سعدیا با ساعد سیمین نشاید پنجه کرد
&&
گر چه بازو سخت داری زور با آهن مکن
\\
\end{longtable}
\end{center}
