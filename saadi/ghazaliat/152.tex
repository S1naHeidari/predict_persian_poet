\begin{center}
\section*{غزل ۱۵۲: بیا که نوبت صلحست و دوستی و عنایت}
\label{sec:152}
\addcontentsline{toc}{section}{\nameref{sec:152}}
\begin{longtable}{l p{0.5cm} r}
بیا که نوبت صلح است و دوستی و عنایت
&&
به شرط آن که نگوییم از آن چه رفت حکایت
\\
بر این یکی شده بودم که گرد عشق نگردم
&&
قضای عشق درآمد بدوخت چشم درایت
\\
ملامت من مسکین کسی کند که نداند
&&
که عشق تا به چه حد است و حسن تا به چه غایت
\\
ز حرص من چه گشاید تو ره به خویشتنم ده
&&
که چشم سعی ضعیف است بی چراغ هدایت
\\
مرا به دست تو خوشتر هلاک جان گرامی
&&
هزار باره که رفتن به دیگری به حمایت
\\
جنایتی که بکردم اگر درست بباشد
&&
فراق روی تو چندین بس است حد جنایت
\\
به هیچ روی نشاید خلاف رای تو کردن
&&
کجا برم گله از دست پادشاه ولایت
\\
به هیچ صورتی اندر نباشد این همه معنی
&&
به هیچ سورتی اندر نباشد این همه آیت
\\
کمال حسن وجودت به وصف راست نیاید
&&
مگر هم آینه گوید چنان که هست حکایت
\\
مرا سخن به نهایت رسید و فکر به پایان
&&
هنوز وصف جمالت نمی‌رسد به نهایت
\\
فراقنامه سعدی به هیچ گوش نیامد
&&
که دردی از سخنانش در او نکرد سرایت
\\
\end{longtable}
\end{center}
