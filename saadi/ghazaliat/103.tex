\begin{center}
\section*{غزل ۱۰۳: ز حد گذشت جدایی میان ما ای دوست}
\label{sec:103}
\addcontentsline{toc}{section}{\nameref{sec:103}}
\begin{longtable}{l p{0.5cm} r}
ز حد گذشت جدایی میان ما ای دوست
&&
بیا بیا که غلام توام بیا ای دوست
\\
اگر جهان همه دشمن شود ز دامن تو
&&
به تیغ مرگ شود دست من رها ای دوست
\\
سرم فدای قفای ملامتست چه باک
&&
گرم بود سخن دشمن از قفا ای دوست
\\
به ناز اگر بخرامی جهان خراب کنی
&&
به خون خسته اگر تشنه‌ای هلا ای دوست
\\
چنان به داغ تو باشم که گر اجل برسد
&&
به شرعم از تو ستانند خونبها ای دوست
\\
وفای عهد نگه دار و از جفا بگذر
&&
به حق آن که نیم یار بی‌وفا ای دوست
\\
هزار سال پس از مرگ من چو بازآیی
&&
ز خاک نعره برآرم که مرحبا ای دوست
\\
غم تو دست برآورد و خون چشمم ریخت
&&
مکن که دست برآرم به ربنا ای دوست
\\
اگر به خوردن خون آمدی هلا برخیز
&&
و گر به بردن دل آمدی بیا ای دوست
\\
بساز با من رنجور ناتوان ای یار
&&
ببخش بر من مسکین بی‌نوا ای دوست
\\
حدیث سعدی اگر نشنوی چه چاره کند
&&
به دشمنان نتوان گفت ماجرا ای دوست
\\
\end{longtable}
\end{center}
