\begin{center}
\section*{غزل ۵۲۵: اگر مانند رخسارت گلی در بوستانستی}
\label{sec:525}
\addcontentsline{toc}{section}{\nameref{sec:525}}
\begin{longtable}{l p{0.5cm} r}
اگر مانند رخسارت گلی در بوستانستی
&&
زمین را از کمالیت شرف بر آسمانستی
\\
چو سرو بوستانستی وجود مجلس آرایت
&&
اگر در بوستان سروی سخنگوی و روانستی
\\
نگارین روی و شیرین خوی و عنبربوی و سیمین تن
&&
چه خوش بودی در آغوشم اگر یارای آنستی
\\
تو گویی در همه عمرم میسر گردد این دولت
&&
که کام از عمر برگیرم و گر خود یک زمانستی
\\
جز این عیبت نمی‌دانم که بدعهدی و سنگین دل
&&
دلارامی بدین خوبی دریغ ار مهربانستی
\\
شکر در کام من تلخ است بی دیدار شیرینش
&&
و گر حلوا بدان ماند که زهرش در میانستی
\\
دمی در صحبت یاری ملک خوی پری پیکر
&&
گر امید بقا باشد بهشت جاودانستی
\\
نه تا جان در جسد باشد وفاداری کنم با او
&&
که تا تن در لحد باشد و گر خود استخوانستی
\\
چنین گویند سعدی را که دردی هست پنهانی
&&
خبر در مغرب و مشرق نبودی گر نهانستی
\\
هر آن دل را که پنهانی قرینی هست روحانی
&&
به خلوتخانه‌ای ماند که در در بوستانستی
\\
\end{longtable}
\end{center}
