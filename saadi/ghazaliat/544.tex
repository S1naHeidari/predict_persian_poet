\begin{center}
\section*{غزل ۵۴۴: ای که بر دوستان همی‌گذری}
\label{sec:544}
\addcontentsline{toc}{section}{\nameref{sec:544}}
\begin{longtable}{l p{0.5cm} r}
ای که بر دوستان همی‌گذری
&&
تا به هر غمزه‌ای دلی ببری
\\
دردمندی تمام خواهی کشت
&&
یا به رحمت به کشته می‌نگری
\\
ما خود از کوی عشقبازانیم
&&
نه تماشاکنان رهگذری
\\
هیچم اندر نظر نمی‌آید
&&
تا تو خورشیدروی در نظری
\\
گفته بودم که دل به کس ندهم
&&
حذر از عاشقی و بی‌خبری
\\
حلقه‌ای گرد خویشتن بکشم
&&
تا نیاید درون حلقه پری
\\
وین پری پیکران حلقه به گوش
&&
شاهدی می‌کنند و جلوه گری
\\
صبر بلبل شنیده‌ای هرگز
&&
چون بخندد شکوفه سحری
\\
پرده‌داری بر آستانه عشق
&&
می‌کند عقل و گریه پرده‌دری
\\
چو خوری دانی ای پسر غم عشق
&&
تا غم هیچ در جهان نخوری
\\
رایگان است یک نفس با دوست
&&
گر به دنیا و آخرت بخری
\\
قلم است این به دست سعدی در
&&
یا هزار آستین در دری
\\
این نبات از کدام شهر آرند
&&
تو قلم نیستی که نیشکری
\\
\end{longtable}
\end{center}
