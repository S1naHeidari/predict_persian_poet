\begin{center}
\section*{غزل ۱۴۳: این که تو داری قیامتست نه قامت}
\label{sec:143}
\addcontentsline{toc}{section}{\nameref{sec:143}}
\begin{longtable}{l p{0.5cm} r}
این که تو داری قیامتست نه قامت
&&
وین نه تبسم که معجزست و کرامت
\\
هر که تماشای روی چون قمرت کرد
&&
سینه سپر کرد پیش تیر ملامت
\\
هر شب و روزی که بی تو می‌رود از عمر
&&
بر نفسی می‌رود هزار ندامت
\\
عمر نبود آن چه غافل از تو نشستم
&&
باقی عمر ایستاده‌ام به غرامت
\\
سرو خرامان چو قد معتدلت نیست
&&
آن همه وصفش که می‌کنند به قامت
\\
چشم مسافر که بر جمال تو افتاد
&&
عزم رحیلش بدل شود به اقامت
\\
اهل فریقین در تو خیره بمانند
&&
گر بروی در حسابگاه قیامت
\\
این همه سختی و نامرادی سعدی
&&
چون تو پسندی سعادتست و سلامت
\\
\end{longtable}
\end{center}
