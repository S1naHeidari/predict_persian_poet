\begin{center}
\section*{غزل ۵۰۱: تو از هر در که بازآیی بدین خوبی و زیبایی}
\label{sec:501}
\addcontentsline{toc}{section}{\nameref{sec:501}}
\begin{longtable}{l p{0.5cm} r}
تو از هر در که بازآیی بدین خوبی و زیبایی
&&
دری باشد که از رحمت به روی خلق بگشایی
\\
ملامتگوی بی‌حاصل ترنج از دست نشناسد
&&
در آن معرض که چون یوسف جمال از پرده بنمایی
\\
به زیورها بیارایند وقتی خوبرویان را
&&
تو سیمین تن چنان خوبی که زیورها بیارایی
\\
چو بلبل روی گل بیند زبانش در حدیث آید
&&
مرا در رویت از حیرت فروبسته‌ست گویایی
\\
تو با این حسن نتوانی که روی از خلق درپوشی
&&
که همچون آفتاب از جام و حور از جامه پیدایی
\\
تو صاحب منصبی جانا ز مسکینان نیندیشی
&&
تو خواب آلوده‌ای بر چشم بیداران نبخشایی
\\
گرفتم سرو آزادی نه از ماء مهین زادی
&&
مکن بیگانگی با ما چو دانستی که از مایی
\\
دعایی گر نمی‌گویی به دشنامی عزیزم کن
&&
که گر تلخ است شیرین است از آن لب هر چه فرمایی
\\
گمان از تشنگی بردم که دریا تا کمر باشد
&&
چو پایانم  برفت اکنون بدانستم که دریایی
\\
تو خواهی آستین افشان و خواهی روی درهم کش
&&
مگس جایی نخواهد رفتن از دکان حلوایی
\\
قیامت می‌کنی سعدی بدین شیرین سخن گفتن
&&
مسلم نیست طوطی را در ایامت شکرخایی
\\
\end{longtable}
\end{center}
