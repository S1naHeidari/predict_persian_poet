\begin{center}
\section*{غزل ۲۵۳: به بوی آن که شبی در حرم بیاسایند}
\label{sec:253}
\addcontentsline{toc}{section}{\nameref{sec:253}}
\begin{longtable}{l p{0.5cm} r}
به بوی آن که شبی در حرم بیاسایند
&&
هزار بادیه سهلست اگر بپیمایند
\\
طریق عشق جفا بردن است و جانبازی
&&
دگر چه چاره که با زورمند برنایند
\\
اگر به بام برآید ستاره پیشانی
&&
چو ماه عید به انگشت‌هاش بنمایند
\\
در گریز نبسته‌ست لیکن از نظرش
&&
کجا روند اسیران که بند بر پایند
\\
ز خون عزیزترم نیست مایه‌ای در تن
&&
فدای دست عزیزان اگر بیالایند
\\
مگر به خیل تو با دوستان نپیوندند
&&
مگر به شهر تو بر عاشقان نبخشایند
\\
فدای جان تو گر جان من طمع داری
&&
غلام حلقه به گوش آن کند که فرمایند
\\
هزار سرو خرامان به راستی نرسد
&&
به قامت تو و گر سر بر آسمان سایند
\\
حدیث حسن تو و داستان عشق مرا
&&
هزار لیلی و مجنون بر آن نیفزایند
\\
مثال سعدی عود است تا نسوزانی
&&
جماعت از نفسش دم به دم نیاسایند
\\
\end{longtable}
\end{center}
