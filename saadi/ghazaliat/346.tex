\begin{center}
\section*{غزل ۳۴۶: مرا رسد که برآرم هزار ناله چو بلبل}
\label{sec:346}
\addcontentsline{toc}{section}{\nameref{sec:346}}
\begin{longtable}{l p{0.5cm} r}
مرا رسد که برآرم هزار ناله چو بلبل
&&
که احتمال ندارم ز دوستان ورقی گل
\\
خبر برید به بلبل که عهد می‌شکند گل
&&
تو نیز اگر بتوانی ببند بار تحول
\\
اما اخالص ودی الم اراعک جهدی
&&
فکیف تنقض عهدی و فیم تهجرنی قل
\\
اگر چه مالک رقی و پادشاه به حقی
&&
همت حلال نباشد ز خون بنده تغافل
\\
من المبلغ عنی الی معذب قلبی
&&
اذا جرحت فؤادی بسیف لحظک فاقتل
\\
تو آن کمند نداری که من خلاص بیابم
&&
اسیر ماندم و درمان تحمل است و تذلل
\\
لاو ضحن بسری و لو تهتک ستری
&&
اذالاحبه ترضی دع اللوائم تعذل
\\
وفا و عهد مودت میان اهل ارادت
&&
نه چون بقای شکوفه‌ست و عشقبازی بلبل
\\
تمیل بین یدینا و لا تمیل الینا
&&
لقد شددت علینا الام تعقد فاحلل
\\
مرا که چشم ارادت به روی و موی تو باشد
&&
دلیل صدق نباشد نظر به لاله و سنبل
\\
فتات شعرک مسک ان اتخذت عبیرا
&&
و حشو ثوبک ورد و طیب فیک قرنفل
\\
تو خود تأمل سعدی نمی‌کنی که ببینی
&&
که هیچ بار ندیدت که سیر شد ز تأمل
\\
\end{longtable}
\end{center}
