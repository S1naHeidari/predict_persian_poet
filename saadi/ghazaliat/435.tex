\begin{center}
\section*{غزل ۴۳۵: ای سروبالای سهی کز صورت حال آگهی}
\label{sec:435}
\addcontentsline{toc}{section}{\nameref{sec:435}}
\begin{longtable}{l p{0.5cm} r}
ای سروبالای سهی کز صورت حال آگهی
&&
وز هر که در عالم بهی ما نیز هم بد نیستیم
\\
گفتی به رنگ من گلی هرگز نبیند بلبلی
&&
آری نکو گفتی ولی ما نیز هم بد نیستیم
\\
تا چند گویی ما و بس کوته کن ای رعنا و بس
&&
نه خود تویی زیبا و بس ما نیز هم بد نیستیم
\\
ای شاهد هر مجلسی و آرام جان هر کسی
&&
گر دوستان داری بسی ما نیز هم بد نیستیم
\\
گفتی که چون من در زمی دیگر نباشد آدمی
&&
ای جان لطف و مردمی ما نیز هم بد نیستیم
\\
گر گلشن خوش بو تویی ور بلبل خوشگو تویی
&&
ور در جهان نیکو تویی ما نیز هم بد نیستیم
\\
گویی چه شد کان سروبن با ما نمی‌گوید سخن
&&
گو بی‌وفایی پر مکن ما نیز هم بد نیستیم
\\
گر تو به حسن افسانه‌ای یا گوهر یک دانه‌ای
&&
از ما چرا بیگانه‌ای ما نیز هم بد نیستیم
\\
ای در دل ما داغ تو تا کی فریب و لاغ تو
&&
گر به بود در باغ تو ما نیز هم بد نیستیم
\\
باری غرور از سر بنه و انصاف درد من بده
&&
ای باغ شفتالو و به ما نیز هم بد نیستیم
\\
گفتم تو ما را دیده‌ای وز حال ما پرسیده‌ای
&&
پس چون ز ما رنجیده‌ای ما نیز هم بد نیستیم
\\
گفتی به از من در چگل صورت نبندد آب و گل
&&
ای سست مهر سخت دل ما نیز هم بد نیستیم
\\
سعدی گر آن زیباقرین بگزید بر ما همنشین
&&
گو هر که خواهی برگزین ما نیز هم بد نیستیم
\\
\end{longtable}
\end{center}
