\begin{center}
\section*{غزل ۴۵۲: فراق دوستانش باد و یاران}
\label{sec:452}
\addcontentsline{toc}{section}{\nameref{sec:452}}
\begin{longtable}{l p{0.5cm} r}
فراق دوستانش باد و یاران
&&
که ما را دور کرد از دوستداران
\\
دلم در بند تنهایی بفرسود
&&
چو بلبل در قفس روز بهاران
\\
هلاک ما چنان مهمل گرفتند
&&
که قتل مور در پای سواران
\\
به خیل هر که می‌آیم به زنهار
&&
نمی‌بینم به جز زنهارخواران
\\
ندانستم که در پایان صحبت
&&
چنین باشد وفای حق گزاران
\\
به گنج شایگان افتاده بودم
&&
ندانستم که بر گنجند ماران
\\
دلا گر دوستی داری به ناچار
&&
بباید بردنت جور هزاران
\\
خلاف شرط یاران است سعدی
&&
که برگردند روز تیرباران
\\
چه خوش باشد سری در پای یاری
&&
به اخلاص و ارادت جان سپاران
\\
\end{longtable}
\end{center}
