\begin{center}
\section*{غزل ۴۵۵: خفته خبر ندارد سر بر کنار جانان}
\label{sec:455}
\addcontentsline{toc}{section}{\nameref{sec:455}}
\begin{longtable}{l p{0.5cm} r}
خفته خبر ندارد سر بر کنار جانان
&&
کاین شب دراز باشد بر چشم پاسبانان
\\
بر عقل من بخندی گر در غمش بگریم
&&
کاین کارهای مشکل افتد به کاردانان
\\
دلداده را ملامت گفتن چه سود دارد
&&
می‌باید این نصیحت کردن به دلستانان
\\
دامن ز پای برگیر ای خوبروی خوش رو
&&
تا دامنت نگیرد دست خدای خوانان
\\
من ترک مهر اینان در خود نمی‌شناسم
&&
بگذار تا بیاید بر من جفای آنان
\\
روشن روان عاشق از تیره شب ننالد
&&
داند که روز گردد روزی شب شبانان
\\
باور مکن که من دست از دامنت بدارم
&&
شمشیر نگسلاند پیوند مهربانان
\\
چشم از تو برنگیرم ور می‌کشد رقیبم
&&
مشتاق گل بسازد با خوی باغبانان
\\
من اختیار خود را تسلیم عشق کردم
&&
همچون زمام اشتر بر دست ساربانان
\\
شکرفروش مصری حال مگس چه داند
&&
این دست شوق بر سر وان آستین فشانان
\\
شاید که آستینت بر سر زنند سعدی
&&
تا چون مگس نگردی گرد شکردهانان
\\
\end{longtable}
\end{center}
