\begin{center}
\section*{غزل ۳۲۵: کس ندیدست به شیرینی و لطف و نازش}
\label{sec:325}
\addcontentsline{toc}{section}{\nameref{sec:325}}
\begin{longtable}{l p{0.5cm} r}
کس ندیده‌ست به شیرینی و لطف و نازش
&&
کس نبیند که نخواهد که ببیند بازش
\\
مطرب ما را دردیست که خوش می‌نالد
&&
مرغ عاشق طرب انگیز بود آوازش
\\
بارها در دلم آمد که بپوشم غم عشق
&&
آبگینه نتواند که بپوشد رازش
\\
مرغ پرنده اگر در قفسی پیر شود
&&
همچنان طبع فرامش نکند پروازش
\\
تا چه کردیم دگرباره که شیرین لب دوست
&&
به سخن باز نمی‌باشد و چشم از نازش
\\
من دعا گویم اگر تو همه دشنام دهی
&&
بنده خدمت بکند ور نکنند اعزازش
\\
غرق دریای غمت را رمقی بیش نماند
&&
آخر اکنون که بکشتی به کنار اندازش
\\
خون سعدی کم از آن است که دست آلایی
&&
ملخ آن قدر ندارد که بگیرد بازش
\\
\end{longtable}
\end{center}
