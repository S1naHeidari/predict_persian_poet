\begin{center}
\section*{غزل ۳۹۹: خنک آن روز که در پای تو جان اندازم}
\label{sec:399}
\addcontentsline{toc}{section}{\nameref{sec:399}}
\begin{longtable}{l p{0.5cm} r}
خنک آن روز که در پای تو جان اندازم
&&
عقل در دمدمه خلق جهان اندازم
\\
نامه حسن تو بر عالم و جاهل خوانم
&&
نامت اندر دهن پیر و جوان اندازم
\\
تا کی این پرده جان سوز پس پرده زنم
&&
تا کی این ناوک دلدوز نهان اندازم
\\
دردنوشان غمت را چو شود مجلس گرم
&&
خویشتن را به طفیلی به میان اندازم
\\
تا نه هر بی‌خبری وصف جمالت گوید
&&
سنگ تعظیم تو در راه بیان اندازم
\\
گر به میدان محاکای تو جولان یابم
&&
گوی دل در خم چوگان زبان اندازم
\\
گردنان را به سرانگشت قبولت ره نیست
&&
چون قلم هستی خود را سر از آن اندازم
\\
یاد سعدی کن و جان دادن مشتاقان بین
&&
حق علیم است که لبیک زنان اندازم
\\
\end{longtable}
\end{center}
