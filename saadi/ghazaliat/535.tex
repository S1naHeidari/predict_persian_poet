\begin{center}
\section*{غزل ۵۳۵: مپرس از من که هیچم یاد کردی}
\label{sec:535}
\addcontentsline{toc}{section}{\nameref{sec:535}}
\begin{longtable}{l p{0.5cm} r}
مپرس از من که هیچم یاد کردی
&&
که خود هیچم فرامش می‌نگردی
\\
چه نیکوروی و بدعهدی که شهری
&&
غمت خوردند و کس را غم نخوردی
\\
چرا ما با تو ای معشوق طناز
&&
به صلحیم و تو با ما در نبردی
\\
نصیحت می‌کنندم سردگویان
&&
که برگرد از غمش بی روی زردی
\\
نمی‌دانند کز بیمار عشقت
&&
حرارت بازننشیند به سردی
\\
ولیکن با رقیبان چاره‌ای نیست
&&
که ایشان مثل خارند و تو وردی
\\
اگر با خوبرویان می‌نشینی
&&
بساط نیک نامی درنوردی
\\
دگر با من مگوی ای باد گلبوی
&&
که همچون بلبلم دیوانه کردی
\\
چرا دردت نچیند جان سعدی
&&
که هم دردی و هم درمان دردی
\\
\end{longtable}
\end{center}
