\begin{center}
\section*{غزل ۱۶۳: هر که می با تو خورد عربده کرد}
\label{sec:163}
\addcontentsline{toc}{section}{\nameref{sec:163}}
\begin{longtable}{l p{0.5cm} r}
هر که می با تو خورد عربده کرد
&&
هر که روی تو دید عشق آورد
\\
زهر اگر در مذاق من ریزی
&&
با تو همچون شکر بشاید خورد
\\
آفرین خدای بر پدری
&&
که تو فرزند نازنین پرورد
\\
لایق خدمت تو نیست بساط
&&
روی باید در این قدم گسترد
\\
خواستم گفت خاک پای توام
&&
عقلم اندر زمان نصیحت کرد
\\
گفت در راه دوست خاک مباش
&&
نه که بر دامنش نشیند گرد
\\
دشمنان در مخالفت گرمند
&&
و آتش ما بدین نگردد سرد
\\
مرد عشق ار ز پیش تیر بلا
&&
روی درهم کشد مخوانش مرد
\\
هر که را برگ بی مرادی نیست
&&
گو برو گرد کوی عشق مگرد
\\
سعدیا صاف وصل اگر ندهند
&&
ما و دردی کشان مجلس درد
\\
\end{longtable}
\end{center}
