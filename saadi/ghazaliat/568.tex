\begin{center}
\section*{غزل ۵۶۸: کس از این نمک ندارد که تو ای غلام داری}
\label{sec:568}
\addcontentsline{toc}{section}{\nameref{sec:568}}
\begin{longtable}{l p{0.5cm} r}
کس از این نمک ندارد که تو ای غلام داری
&&
دل ریش عاشقان را نمکی تمام داری
\\
نه من اوفتاده تنها به کمند آرزویت
&&
همه کس سر تو دارد تو سر کدام داری
\\
ملکا مها نگارا صنما بتا بهارا
&&
متحیرم ندانم که تو خود چه نام داری
\\
نظری به لشکری کن که هزار خون بریزی
&&
به خلاف تیغ هندی که تو در نیام داری
\\
صفت رخام دارد تن نرم نازنینت
&&
دل سخت نیز با او نه کم از رخام داری
\\
همه دیده‌ها به سویت نگران حسن رویت
&&
منت آن کمینه مرغم که اسیر دام داری
\\
چه مخالفت بدیدی که مخالطت بریدی
&&
مگر آن که ما گداییم و تو احتشام داری
\\
به جز این گنه ندانم که محب و مهربانم
&&
به چه جرم دیگر از من سر انتقام داری
\\
گله از تو حاش لله نکنند و خود نباشد
&&
مگر از وفای عهدی که نه بر دوام داری
\\
نظر از تو برنگیرم همه عمر تا بمیرم
&&
که تو در دلم نشستی و سر مقام داری
\\
سخن لطیف سعدی نه سخن که قند مصری
&&
خجل است از این حلاوت که تو در کلام داری
\\
\end{longtable}
\end{center}
