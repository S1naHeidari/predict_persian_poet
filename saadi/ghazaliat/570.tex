\begin{center}
\section*{غزل ۵۷۰: هرگز نبود سرو به بالا که تو داری}
\label{sec:570}
\addcontentsline{toc}{section}{\nameref{sec:570}}
\begin{longtable}{l p{0.5cm} r}
هرگز نبود سرو به بالا که تو داری
&&
یا مه به صفای رخ زیبا که تو داری
\\
گر شمع نباشد شب دلسوختگان را
&&
روشن کند این غره غرا که تو داری
\\
حوران بهشتی که دل خلق ستانند
&&
هرگز نستانند دل ما که تو داری
\\
بسیار بود سرو روان و گل خندان
&&
لیکن نه بدین صورت و بالا که تو داری
\\
پیداست که سرپنجه ما را چه بود زور
&&
با ساعد سیمین توانا که تو داری
\\
سحر سخنم در همه آفاق ببردند
&&
لیکن چه زند با ید بیضا که تو داری
\\
امثال تو از صحبت ما ننگ ندارند
&&
جای مگس است این همه حلوا که تو داری
\\
این روی به صحرا کند آن میل به بستان
&&
من روی ندارم مگر آن جا که تو داری
\\
سعدی تو نیارامی و کوته نکنی دست
&&
تا سر نرود در سر سودا که تو داری
\\
تا میل نباشد به وصال از طرف دوست
&&
سودی نکند حرص و تمنا که تو داری
\\
\end{longtable}
\end{center}
