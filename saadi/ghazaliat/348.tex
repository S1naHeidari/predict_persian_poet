\begin{center}
\section*{غزل ۳۴۸: چشم خدا بر تو ای بدیع شمایل}
\label{sec:348}
\addcontentsline{toc}{section}{\nameref{sec:348}}
\begin{longtable}{l p{0.5cm} r}
چشم خدا بر تو ای بدیع شمایل
&&
یار من و شمع جمع و شاه قبایل
\\
جلوه کنان می‌روی و باز می‌آیی
&&
سرو ندیدم بدین صفت متمایل
\\
هر صفتی را دلیل معرفتی هست
&&
روی تو بر قدرت خدای دلایل
\\
قصه لیلی مخوان و غصه مجنون
&&
عهد تو منسوخ کرد ذکر اوایل
\\
نام تو می‌رفت و عارفان بشنیدند
&&
هر دو به رقص آمدند سامع و قایل
\\
پرده چه باشد میان عاشق و معشوق
&&
سد سکندر نه مانع است و نه حایل
\\
گو همه شهرم نگه کنند و ببینند
&&
دست در آغوش یار کرده حمایل
\\
دور به آخر رسید و عمر به پایان
&&
شوق تو ساکن نگشت و مهر تو زایل
\\
گر تو برانی کسم شفیع نباشد
&&
ره به تو دانم دگر به هیچ وسایل
\\
با که نگفتم حکایت غم عشقت
&&
این همه گفتیم و حل نگشت مسائل
\\
سعدی از این پس نه عاقلست نه هشیار
&&
عشق بچربید بر فنون فضایل
\\
\end{longtable}
\end{center}
