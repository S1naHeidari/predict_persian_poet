\begin{center}
\section*{غزل ۷۲: پای سرو بوستانی در گلست}
\label{sec:072}
\addcontentsline{toc}{section}{\nameref{sec:072}}
\begin{longtable}{l p{0.5cm} r}
پای سرو بوستانی در گل است
&&
سرو ما را پای معنی در دل است
\\
هر که چشمش بر چنان روی اوفتاد
&&
طالعش میمون و فالش مقبل است
\\
نیکخواهانم نصیحت می‌کنند
&&
خشت بر دریا زدن بی‌حاصل است
\\
ای برادر ما به گرداب اندریم
&&
وان که شنعت می‌زند بر ساحل است
\\
شوق را بر صبر قوت غالب است
&&
عقل را با عشق دعوی باطل است
\\
نسبت عاشق به غفلت می‌کنند
&&
وآن که معشوقی ندارد غافل است
\\
دیده باشی تشنه مستعجل به آب
&&
جان به جانان همچنان مستعجل است
\\
بذل جاه و مال و ترک نام و ننگ
&&
در طریق عشق اول منزل است
\\
گر بمیرد طالبی در بند دوست
&&
سهل باشد زندگانی مشکل است
\\
عاشقی می‌گفت و خوش خوش می‌گریست
&&
جان بیاساید که جانان قاتل است
\\
سعدیا نزدیک رای عاشقان
&&
خلق مجنونند و مجنون عاقل است
\\
\end{longtable}
\end{center}
