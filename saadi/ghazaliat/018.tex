\begin{center}
\section*{غزل ۱۸: ساقی بده آن کوزه یاقوت روان را}
\label{sec:018}
\addcontentsline{toc}{section}{\nameref{sec:018}}
\begin{longtable}{l p{0.5cm} r}
ساقی بده آن کوزه یاقوت روان را
&&
یاقوت چه ارزد بده آن قوت روان را
\\
اول پدر پیر خورد رطل دمادم
&&
تا مدعیان هیچ نگویند جوان را
\\
تا مست نباشی نبری بار غم یار
&&
آری شتر مست کشد بار گران را
\\
ای روی تو آرام دل خلق جهانی
&&
بی روی تو شاید که نبینند جهان را
\\
در صورت و معنی که تو داری چه توان گفت
&&
حسن تو ز تحسین تو بستست زبان را
\\
آنک عسل اندوخته دارد مگس نحل
&&
شهد لب شیرین تو زنبورمیان را
\\
زین دست که دیدار تو دل می‌برد از دست
&&
ترسم نبرم عاقبت از دست تو جان را
\\
یا تیر هلاکم بزنی بر دل مجروح
&&
یا جان بدهم تا بدهی تیر امان را
\\
وان گه که به تیرم زنی اول خبرم ده
&&
تا پیشترت بوسه دهم دست و کمان را
\\
سعدی ز فراق تو نه آن رنج کشیدست
&&
کز شادی وصل تو فرامش کند آن را
\\
ور نیز جراحت به دوا باز هم آید
&&
از جای جراحت نتوان برد نشان را
\\
\end{longtable}
\end{center}
