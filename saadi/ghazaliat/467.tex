\begin{center}
\section*{غزل ۴۶۷: آخر نگهی به سوی ما کن}
\label{sec:467}
\addcontentsline{toc}{section}{\nameref{sec:467}}
\begin{longtable}{l p{0.5cm} r}
آخر نگهی به سوی ما کن
&&
دردی به ارادتی دوا کن
\\
بسیار خلاف عهد کردی
&&
آخر به غلط یکی وفا کن
\\
ما را تو به خاطری همه روز
&&
یک روز تو نیز یاد ما کن
\\
این قاعده خلاف بگذار
&&
وین خوی معاندت رها کن
\\
برخیز و در سرای در بند
&&
بنشین و قبای بسته وا کن
\\
آن را که هلاک می‌پسندی
&&
روزی دو به خدمت آشنا کن
\\
چون انس گرفت و مهر پیوست
&&
بازش به فراق مبتلا کن
\\
سعدی چو حریف ناگزیر است
&&
تن درده و چشم در قضا کن
\\
شمشیر که می‌زند سپر باش
&&
دشنام که می‌دهد دعا کن
\\
زیبا نبود شکایت از دوست
&&
زیبا همه روز گو جفا کن
\\
\end{longtable}
\end{center}
