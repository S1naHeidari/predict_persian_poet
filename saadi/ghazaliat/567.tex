\begin{center}
\section*{غزل ۵۶۷: اگر به تحفه جانان هزار جان آری}
\label{sec:567}
\addcontentsline{toc}{section}{\nameref{sec:567}}
\begin{longtable}{l p{0.5cm} r}
اگر به تحفه جانان هزار جان آری
&&
محقر است نشاید که بر زبان آری
\\
حدیث جان بر جانان همین مثل باشد
&&
که زر به کان بری و گل به بوستان آری
\\
هنوز در دلت ای آفتاب رخ نگذشت
&&
که سایه‌ای به سر یار مهربان آری
\\
تو را چه غم که مرا در غمت نگیرد خواب
&&
تو پادشاه کجا یاد پاسبان آری
\\
ز حسن روی تو بر دین خلق می‌ترسم
&&
که بدعتی که نبوده‌ست در جهان آری
\\
کس از کناری در روی تو نگه نکند
&&
که عاقبت نه به شوخیش در میان آری
\\
ز چشم مست تو واجب کند که هشیاران
&&
حذر کنند ولی تاختن نهان آری
\\
جواب تلخ چه داری بگوی و باک مدار
&&
که شهد محض بود چون تو بر دهان آری
\\
و گر به خنده درآیی چه جای مرهم ریش
&&
که ممکن است که در جسم مرده جان آری
\\
یکی لطیفه ز من بشنو ای که در آفاق
&&
سفر کنی و لطایف ز بحر و کان آری
\\
گرت بدایع سعدی نباشد اندر بار
&&
به پیش اهل و قرابت چه ارمغان آری
\\
\end{longtable}
\end{center}
