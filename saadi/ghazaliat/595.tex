\begin{center}
\section*{غزل ۵۹۵: هرگز حسد نبردم بر منصبی و مالی}
\label{sec:595}
\addcontentsline{toc}{section}{\nameref{sec:595}}
\begin{longtable}{l p{0.5cm} r}
هرگز حسد نبردم بر منصبی و مالی
&&
الا بر آن که دارد با دلبری وصالی
\\
دانی کدام دولت در وصف می‌نیاید
&&
چشمی که باز باشد هر لحظه بر جمالی
\\
خرم تنی که محبوب از در فرازش آید
&&
چون رزق نیکبختان بی محنت سؤالی
\\
همچون دو مغز بادام اندر یکی خزینه
&&
با هم گرفته انسی وز دیگران ملالی
\\
دانی کدام جاهل بر حال ما بخندد
&&
کاو را نبوده باشد در عمر خویش حالی
\\
بعد از حبیب بر من نگذشت جز خیالش
&&
وز پیکر ضعیفم نگذاشت جز خیالی
\\
اول که گوی بردی من بودمی به دانش
&&
گر سودمند بودی بی دولت احتیالی
\\
سال وصال با او یک روز بود گویی
&&
و اکنون در انتظارش روزی به قدر سالی
\\
ایام را به ماهی یک شب هلال باشد
&&
وآن ماه دلستان را هر ابرویی هلالی
\\
صوفی نظر نبازد جز با چنین حریفی
&&
سعدی غزل نگوید جز بر چنین غزالی
\\
\end{longtable}
\end{center}
