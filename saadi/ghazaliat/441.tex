\begin{center}
\section*{غزل ۴۴۱: عهد کردیم که بی دوست به صحرا نرویم}
\label{sec:441}
\addcontentsline{toc}{section}{\nameref{sec:441}}
\begin{longtable}{l p{0.5cm} r}
عهد کردیم که بی دوست به صحرا نرویم
&&
بی تماشاگه رویش به تماشا نرویم
\\
بوستان خانه عیش است و چمن کوی نشاط
&&
تا مهیا نبود عیش مهنا نرویم
\\
دیگران با همه کس دست در آغوش کنند
&&
ما که بر سفره خاصیم به یغما نرویم
\\
نتوان رفت مگر در نظر یار عزیز
&&
ور تحمل نکند زحمت ما تا نرویم
\\
گر به خواری ز در خویش براند ما را
&&
به امیدش بنشینیم و به درها نرویم
\\
گر به شمشیر احبا تن ما پاره کنند
&&
به تظلم به در خانه اعدا نرویم
\\
پای گو بر سر و بر دیده ما نه چو بساط
&&
که اگر نقش بساطت برود ما نرویم
\\
به درشتی و جفا روی مگردان از ما
&&
که به کشتن برویم از نظرت یا نرویم
\\
سعدیا شرط وفاداری لیلی آن است
&&
که اگر مجنون گویند به سودا نرویم
\\
\end{longtable}
\end{center}
