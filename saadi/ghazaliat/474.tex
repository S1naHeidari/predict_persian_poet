\begin{center}
\section*{غزل ۴۷۴: نشان بخت بلندست و طالع میمون}
\label{sec:474}
\addcontentsline{toc}{section}{\nameref{sec:474}}
\begin{longtable}{l p{0.5cm} r}
نشان بخت بلند است و طالع میمون
&&
علی الصباح نظر بر جمال روزافزون
\\
علی الخصوص کسی را که طبع موزون است
&&
چگونه دوست ندارد شمایل موزون
\\
گر آبروی بریزد میان انجمنت
&&
به دست دوست حلال است اگر بریزد خون
\\
مثال عاشق و معشوق شمع و پروانه‌ست
&&
سر هلاک نداری مگرد پیرامون
\\
بسوخت مجنون در عشق صورت لیلی
&&
عجب که لیلی را دل نسوخت بر مجنون
\\
چگونه وصف جمالش کنم که حیران را
&&
مجال نطق نباشد که بازگوید چون
\\
همین تغیر بیرون دلیل عشق بس است
&&
که در حدیث نمی‌گنجد اشتیاق درون
\\
اگر کسی نفسی از زمان صحبت دوست
&&
به ملک روی زمین می‌دهد زهی مغبون
\\
سخن دراز کشیدیم و همچنان باقیست
&&
حدیث دلبر فتان و عاشق مفتون
\\
جفای عشق تو چندان که می‌برد سعدی
&&
خیال وصل تو از سر نمی‌کند بیرون
\\
\end{longtable}
\end{center}
