\begin{center}
\section*{غزل ۵۸۶: به پایان آمد این دفتر حکایت همچنان باقی}
\label{sec:586}
\addcontentsline{toc}{section}{\nameref{sec:586}}
\begin{longtable}{l p{0.5cm} r}
به پایان آمد این دفتر حکایت همچنان باقی
&&
به صد دفتر نشاید گفت حسب الحال مشتاقی
\\
کتاب بالغ منی حبیبا معرضا عنی
&&
ان افعل ما تری انی علی عهدی و میثاقی
\\
نگویم نسبتی دارم به نزدیکان درگاهت
&&
که خود را بر تو می‌بندم به سالوسی و زراقی
\\
اخلایی و احبابی ذروا من حبه مابی
&&
مریض العشق لا یبری و لا یشکو الی الراقی
\\
نشان عاشق آن باشد که شب با روز پیوندد
&&
تو را گر خواب می‌گیرد نه صاحب درد عشاقی
\\
قم املا و اسقنی کأسا و دع ما فیه مسموما
&&
اما انت الذی تسقی فعین السم تریاقی
\\
قدح چون دور ما باشد به هشیاران مجلس ده
&&
مرا بگذار تا حیران بماند چشم در ساقی
\\
سعی فی هتکی الشانی و لما یدر ماشانی
&&
انا المجنون لا اعبا باحراق و اغراق
\\
مگر شمس فلک باشد بدین فرخنده دیداری
&&
مگر نفس ملک باشد بدین پاکیزه اخلاقی
\\
لقیت الاسد فی الغابات لا تقوی علی صیدی
&&
و هذا الظبی فی شیراز یسبینی باحداق
\\
نه حسنت آخری دارد نه سعدی را سخن پایان
&&
بمیرد تشنه مستسقی و دریا همچنان باقی
\\
\end{longtable}
\end{center}
