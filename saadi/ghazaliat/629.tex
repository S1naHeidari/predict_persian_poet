\begin{center}
\section*{غزل ۶۲۹: گلست آن یاسمن یا ماه یا روی}
\label{sec:629}
\addcontentsline{toc}{section}{\nameref{sec:629}}
\begin{longtable}{l p{0.5cm} r}
گل است آن یا سمن یا ماه یا روی
&&
شب است آن یا شبه یا مشک یا موی 
\\
لبت دانم که یاقوت است و تن سیم
&&
نمی‌دانم دلت سنگ است یا روی
\\
نپندارم که در بستان فردوس
&&
بروید چون تو سروی بر لب جوی
\\
چه شیرین لب سخنگویی که عاجز
&&
فرو می‌ماند از وصفت سخنگوی
\\
به بویی الغیاث از ما برآید
&&
که ای باد از کجا آوردی این بوی
\\
الا ای ترک آتشروی ساقی
&&
به آب باده عقل از من فرو شوی
\\
چه شهرآشوبی ای دلبند خودرای
&&
چه بزم آرایی ای گلبرگ خودروی
\\
چو در میدان عشق افتادی ای دل
&&
بباید بودنت سرگشته چون گوی
\\
دلا گر عاشقی می‌سوز و می‌ساز
&&
تنا گر طالبی می‌پرس و می‌پوی
\\
در این ره جان بده یا ترک ما گیر
&&
بر این در سر بنه یا غیر ما جوی
\\
بداندیشان ملامت می‌کنندم
&&
که تا چند احتمال یار بدخوی
\\
محال است این که ترک دوست هرگز
&&
بگوید سعدی ای دشمن تو می‌گوی
\\
\end{longtable}
\end{center}
