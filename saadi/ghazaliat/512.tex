\begin{center}
\section*{غزل ۵۱۲: همه چشمیم تا برون آیی}
\label{sec:512}
\addcontentsline{toc}{section}{\nameref{sec:512}}
\begin{longtable}{l p{0.5cm} r}
همه چشمیم تا برون آیی
&&
همه گوشیم تا چه فرمایی
\\
تو نه آن صورتی که بی رویت
&&
متصور شود شکیبایی
\\
من ز دست تو خویشتن بکشم
&&
تا تو دستم به خون نیالایی
\\
گفته بودی قیامتم بینند
&&
این گروهی محب سودایی
\\
وین چنین روی دلستان که تو راست
&&
خود قیامت بود که بنمایی
\\
ما تماشاکنان کوته دست
&&
تو درخت بلندبالایی
\\
سر ما و آستان خدمت تو
&&
گر برانی و گر ببخشایی
\\
جان به شکرانه دادن از من خواه
&&
گر به انصاف با میان آیی
\\
عقل باید که با صلابت عشق
&&
نکند پنجه توانایی
\\
تو چه دانی که بر تو نگذشته‌ست
&&
شب هجران و روز تنهایی
\\
روشنت گردد این حدیث چو روز
&&
گر چو سعدی شبی بپیمایی
\\
\end{longtable}
\end{center}
