\begin{center}
\section*{غزل ۶۰۲: آسوده خاطرم که تو در خاطر منی}
\label{sec:602}
\addcontentsline{toc}{section}{\nameref{sec:602}}
\begin{longtable}{l p{0.5cm} r}
آسوده خاطرم که تو در خاطر منی
&&
گر تاج می‌فرستی و گر تیغ می‌زنی
\\
ای چشم عقل خیره در اوصاف روی تو
&&
چون مرغ شب که هیچ نبیند به روشنی
\\
شهری به تیغ غمزه خونخوار و لعل لب
&&
مجروح می‌کنی و نمک می‌پراکنی
\\
ما خوشه چین خرمن اصحاب دولتیم
&&
باری نگه کن ای که خداوند خرمنی
\\
گیرم که برکنی دل سنگین ز مهر من
&&
مهر از دلم چگونه توانی که برکنی
\\
حکم آن توست اگر بکشی بی‌گنه ولیک
&&
عهد وفای دوست نشاید که بشکنی
\\
این عشق را زوال نباشد به حکم آنک
&&
ما پاک دیده‌ایم و تو پاکیزه دامنی
\\
از من گمان مبر که بیاید خلاف دوست
&&
ور متفق شوند جهانی به دشمنی
\\
خواهی که دل به کس ندهی دیده‌ها بدوز
&&
پیکان چرخ را سپری باشد آهنی
\\
با مدعی بگوی که ما خود شکسته‌ایم
&&
محتاج نیست پنجه که با ما درافکنی
\\
سعدی چو سروری نتوان کرد لازم است
&&
با سخت بازوان به ضرورت فروتنی
\\
\end{longtable}
\end{center}
