\begin{center}
\section*{غزل ۳۶۶: گو خلق بدانند که من عاشق و مستم}
\label{sec:366}
\addcontentsline{toc}{section}{\nameref{sec:366}}
\begin{longtable}{l p{0.5cm} r}
گو خلق بدانند که من عاشق و مستم
&&
آوازه درست است که من توبه شکستم
\\
گر دشمنم ایذا کند و دوست ملامت
&&
من فارغم از هر چه بگویند که هستم
\\
ای نفس که مطلوب تو ناموس و ریا بود
&&
از بند تو برخاستم و خوش بنشستم
\\
از روی نگارین تو بیزارم اگر من
&&
تا روی تو دیدم به دگر کس نگرستم
\\
زین پیش برآمیختمی با همه مردم
&&
تا یار بدیدم در اغیار ببستم
\\
ای ساقی از آن پیش که مستم کنی از می
&&
من خود ز نظر در قد و بالای تو مستم
\\
شب‌ها گذرد بر من از اندیشه رویت
&&
تا روز نه من خفته نه همسایه ز دستم
\\
حیف است سخن گفتن با هر کس از آن لب
&&
دشنام به من ده که درودت بفرستم
\\
دیریست که سعدی به دل از عشق تو می‌گفت
&&
این بت نه عجب باشد اگر من بپرستم
\\
بند همه غم‌های جهان بر دل من بود
&&
در بند تو افتادم و از جمله برستم
\\
\end{longtable}
\end{center}
