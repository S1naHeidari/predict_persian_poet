\begin{center}
\section*{غزل ۵۸۳: ما سپر انداختیم گر تو کمان می‌کشی}
\label{sec:583}
\addcontentsline{toc}{section}{\nameref{sec:583}}
\begin{longtable}{l p{0.5cm} r}
ما سپر انداختیم گر تو کمان می‌کشی
&&
گو دل ما خوش مباش گر تو بدین دلخوشی
\\
گر بکشی بنده‌ایم ور بنوازی رواست
&&
ما به تو مستأنسیم تو به چه مستوحشی
\\
گفتی اگر درد عشق پای نداری گریز
&&
چون بتوانم گریخت تا تو کمندم کشی
\\
دیده فرودوختیم تا نه به دوزخ برد
&&
باز نگه می‌کنم سخت بهشتی وشی
\\
غایت خوبی که هست قبضه و شمشیر و دست
&&
خلق حسد می‌برند چون تو مرا می‌کشی
\\
موجب فریاد ما خصم نداند که چیست
&&
چاره مجروح عشق نیست به جز خامشی
\\
چند توان ای سلیم آب بر آتش زدن
&&
کآب دیانت برد رنگ رخ آتشی
\\
آدمی هوشمند عیش ندارد ز فکر
&&
ساقی مجلس بیار آن قدح بی هشی
\\
مست می عشق را عیب مکن سعدیا
&&
مست بیفتی تو نیز گر هم از این می چشی
\\
\end{longtable}
\end{center}
