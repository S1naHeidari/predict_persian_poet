\begin{center}
\section*{غزل ۳۵۶: چو بلبل سحری برگرفت نوبت بام}
\label{sec:356}
\addcontentsline{toc}{section}{\nameref{sec:356}}
\begin{longtable}{l p{0.5cm} r}
چو بلبل سحری برگرفت نوبت بام
&&
ز توبه خانه تنهایی آمدم بر بام
\\
نگاه می‌کنم از پیش رایت خورشید
&&
که می‌برد به افق پرچم سپاه ظلام
\\
بیاض روز برآمد چو از دواج سیاه
&&
برهنه بازنشیند یکی سپیداندام
\\
دلم به عشق گرفتار و جان به مهر گرو
&&
درآمد از درم آن دلفریب جان آرام
\\
سرم هنوز چنان مست بوی آن نفس است
&&
که بوی عنبر و گل ره نمی‌برد به مشام
\\
دگر من از شب تاریک هیچ غم نخورم
&&
که هر شبی را روزی مقدر است انجام
\\
تمام فهم نکردم که ارغوان و گل است
&&
در آستینش یا دست و ساعد گلفام
\\
در آبگینه‌اش آبی که گر قیاس کنی
&&
ندانی آب کدام است و آبگینه کدام
\\
بیار ساقی دریای مشرق و مغرب
&&
که دیر مست شود هر که می خورد به دوام
\\
من آن نیم که حلال از حرام نشناسم
&&
شراب با تو حلال است و آب بی تو حرام
\\
به هیچ شهر نباشد چنین شکر که تویی
&&
که طوطیان چو سعدی درآوری به کلام
\\
رها نمی‌کند این نظم چون زره درهم
&&
که خصم تیغ تعنت برآورد ز نیام
\\
\end{longtable}
\end{center}
