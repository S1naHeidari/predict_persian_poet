\begin{center}
\section*{غزل ۳۲۷: چون برآمد ماه روی از مطلع پیراهنش}
\label{sec:327}
\addcontentsline{toc}{section}{\nameref{sec:327}}
\begin{longtable}{l p{0.5cm} r}
چون برآمد ماه روی از مطلع پیراهنش
&&
چشم بد را گفتم الحمدی بدم پیرامنش
\\
تا چه خواهد کرد با من دور گیتی زین دو کار
&&
دست او در گردنم یا خون من در گردنش
\\
هر که معلومش نمی‌گردد که زاهد را که کشت
&&
گو سرانگشتان شاهد بین و رنگ ناخنش
\\
گر چمن گوید مرا همرنگ رویش لاله‌ایست
&&
از قفا باید برون کردن زبان سوسنش
\\
ماه و پروینش نیارم گفت و سرو و آفتاب
&&
لطف جان در جسم دارد جسم در پیراهنش
\\
آستین از چنگ مسکینان گرفتم درکشد
&&
چون تواند رفت و چندین دست دل در دامنش
\\
من سبیل دشمنان کردم نصیب عرض خویش
&&
دشمن آن کس در جهان دارم که دارد دشمنش
\\
گر تنم مویی شود از دست جور روزگار
&&
بر من آسان‌تر بود کآسیب مویی بر تنش
\\
تا چه روی است آن که حیران مانده‌ام در وصف او
&&
صبحی از مشرق همی‌تابد یکی از روزنش
\\
بعد از این ای یار اگر تفصیل هشیاران کنند
&&
گر در آنجا نام من بینی قلم بر سر زنش
\\
لایق سعدی نبود این خرقه تقوا و زهد
&&
ساقیا جامی بده وین جامه از سر برکنش
\\
\end{longtable}
\end{center}
