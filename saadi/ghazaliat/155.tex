\begin{center}
\section*{غزل ۱۵۵: زان گه که بر آن صورت خوبم نظر افتاد}
\label{sec:155}
\addcontentsline{toc}{section}{\nameref{sec:155}}
\begin{longtable}{l p{0.5cm} r}
زان گه که بر آن صورت خوبم نظر افتاد
&&
از صورت بی طاقتیم پرده برافتاد
\\
گفتیم که عقل از همه کاری به درآید
&&
بیچاره فروماند چو عشقش به سر افتاد
\\
شمشیر کشیدست نظر بر سر مردم
&&
چون پای بدارم که ز دستم سپر افتاد
\\
در سوخته پنهان نتوان داشتن آتش
&&
ما هیچ نگفتیم و حکایت به درافتاد
\\
با هر که خبر گفتم از اوصاف جمیلش
&&
مشتاق چنان شد که چو من بی‌خبر افتاد
\\
هان تا لب شیرین نستاند دلت از دست
&&
کان کز غم او کوه گرفت از کمر افتاد
\\
صاحب نظران این نفس گرم چو آتش
&&
دانند که در خرمن من بیشتر افتاد
\\
نیکم نظر افتاد بر آن منظر مطبوع
&&
کاول نظرم هر چه وجود از نظر افتاد
\\
سعدی نه حریف غم او بود ولیکن
&&
با رستم دستان بزند هر که درافتاد
\\
\end{longtable}
\end{center}
