\begin{center}
\section*{غزل ۷۶: یارا بهشت صحبت یاران همدمست}
\label{sec:076}
\addcontentsline{toc}{section}{\nameref{sec:076}}
\begin{longtable}{l p{0.5cm} r}
یارا بهشت صحبت یاران همدمست
&&
دیدار یار نامتناسب جهنمست
\\
هر دم که در حضور عزیزی برآوری
&&
دریاب کز حیات جهان حاصل آن دمست
\\
نه هر که چشم و گوش و دهان دارد آدمیست
&&
بس دیو را که صورت فرزند آدمست
\\
آنست آدمی که در او حسن سیرتی
&&
یا لطف صورتیست دگر حشو عالمست
\\
هرگز حسد نبرده و حسرت نخورده‌ام
&&
جز بر دو روی یار موافق که در همست
\\
آنان که در بهار به صحرا نمی‌روند
&&
بوی خوش ربیع بر ایشان محرمست
\\
وان سنگ دل که دیده بدوزد ز روی خوب
&&
پندش مده که جهل در او نیک محکمست
\\
آرام نیست در همه عالم به اتفاق
&&
ور هست در مجاورت یار محرمست
\\
گر خون تازه می‌رود از ریش اهل دل
&&
دیدار دوستان که ببینند مرهمست
\\
دنیا خوشست و مال عزیزست و تن شریف
&&
لیکن رفیق بر همه چیزی مقدمست
\\
ممسک برای مال همه ساله تنگ دل
&&
سعدی به روی دوست همه روزه خرمست
\\
\end{longtable}
\end{center}
