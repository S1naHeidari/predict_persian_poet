\begin{center}
\section*{غزل ۲۸۵: کاروانی شکر از مصر به شیراز آید}
\label{sec:285}
\addcontentsline{toc}{section}{\nameref{sec:285}}
\begin{longtable}{l p{0.5cm} r}
کاروانی شکر از مصر به شیراز آید
&&
اگر آن یار سفرکرده ما بازآید
\\
گو تو بازآی که گر خون منت در خورد است
&&
پیشت آیم چو کبوتر که به پرواز آید
\\
نام و ننگ و دل و دین گو برود این مقدار
&&
چیست تا در نظر عاشق جانباز آید
\\
من خود این سنگ به جان می‌طلبیدم همه عمر
&&
کاین قفس بشکند و مرغ به پرواز آید
\\
اگر این داغ جگرسوز که بر جان من است
&&
بر دل کوه نهی سنگ به آواز آید
\\
من همان روز که روی تو بدیدم گفتم
&&
هیچ شک نیست که از روی چنین ناز آید
\\
هر چه در صورت عقل آید و در وهم و قیاس
&&
آن که محبوب من است از همه ممتاز آید
\\
گر تو بازآیی و بر ناظر سعدی بروی
&&
هیچ غم نیست که منظور به اعزاز آید
\\
\end{longtable}
\end{center}
