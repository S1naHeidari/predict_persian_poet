\begin{center}
\section*{غزل ۵۹۸: تو کدامی و چه نامی که چنین خوب خرامی}
\label{sec:598}
\addcontentsline{toc}{section}{\nameref{sec:598}}
\begin{longtable}{l p{0.5cm} r}
تو کدامی و چه نامی که چنین خوب خرامی
&&
خون عشاق حلال است زهی شوخ حرامی
\\
بیم آن است دمادم که چو پروانه بسوزم
&&
از تغابن که تو چون شمع چرا شاهد عامی
\\
فتنه انگیزی و خون ریزی و خلقی نگرانت
&&
که چه شیرین حرکاتی و چه مطبوع کلامی
\\
مگر از هیئت شیرین تو می‌رفت حدیثی
&&
نیشکر گفت کمر بسته‌ام اینک به غلامی
\\
کافر ار قامت همچون بت سنگین تو بیند
&&
بار دیگر نکند سجده بت‌های رخامی
\\
بنشین یک نفس ای فتنه که برخاست قیامت
&&
فتنه نادر بنشیند چو تو در حال قیامی
\\
بلعجب باشد از این خلق که رویت چو مه نو
&&
می‌نمایند به انگشت و تو خود بدر تمامی
\\
کس نیارد که کند جور در اقبال اتابک
&&
تو چنین سرکش و بیچاره کش از خیل کدامی
\\
آفت مجلس و میدان و هلاک زن و مردی
&&
فتنه خانه و بازار و بلای در و بامی
\\
در سر کار تو کردم دل و دین با همه دانش
&&
مرغ زیرک به حقیقت منم امروز و تو دامی
\\
طاقتم نیست ز هر بی‌خبری سنگ ملامت
&&
که تو در سینه سعدی چو چراغ از پس جامی
\\
\end{longtable}
\end{center}
