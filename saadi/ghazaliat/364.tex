\begin{center}
\section*{غزل ۳۶۴: من اندر خود نمی‌یابم که روی از دوست برتابم}
\label{sec:364}
\addcontentsline{toc}{section}{\nameref{sec:364}}
\begin{longtable}{l p{0.5cm} r}
من اندر خود نمی‌یابم که روی از دوست برتابم
&&
بدار ای دوست دست از من که طاقت رفت و پایابم
\\
تنم فرسود و عقلم رفت و عشقم همچنان باقی
&&
وگر جانم دریغ آید نه مشتاقم که کذابم
\\
بیار ای لعبت ساقی نگویم چند پیمانه
&&
که گر جیحون بپیمایی نخواهی یافت سیرابم
\\
مرا روی تو محراب است در شهر مسلمانان
&&
وگر جنگ مغول باشد نگردانی ز محرابم
\\
مرا از دنیی و عقبی همینم بود و دیگر نه
&&
که پیش از رفتن از دنیا دمی با دوست دریابم
\\
سر از بیچارگی گفتم نهم شوریده در عالم
&&
دگر ره پای می‌بندد وفای عهد اصحابم
\\
نگفتی بی‌وفا یارا که دلداری کنی ما را
&&
الا ار دست می‌گیری بیا کز سر گذشت آبم
\\
زمستان است و بی برگی بیا ای باد نوروزم
&&
بیابان است و تاریکی بیا ای قرص مهتابم
\\
حیات سعدی آن باشد که بر خاک درت میرد
&&
دری دیگر نمی‌دانم مکن محروم از این بابم
\\
\end{longtable}
\end{center}
