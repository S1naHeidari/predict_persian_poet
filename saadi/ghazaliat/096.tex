\begin{center}
\section*{غزل ۹۶: ز هر چه هست گزیرست و ناگزیر از دوست}
\label{sec:096}
\addcontentsline{toc}{section}{\nameref{sec:096}}
\begin{longtable}{l p{0.5cm} r}
ز هر چه هست گزیرست و ناگزیر از دوست
&&
به قول هر که جهان مهر برمگیر از دوست
\\
به بندگی و صغیری گرت قبول کند
&&
سپاس دار که فضلی بود کبیر از دوست
\\
به جای دوست گرت هر چه در جهان بخشند
&&
رضا مده که متاعی بود حقیر از دوست
\\
جهان و هر چه در او هست با نعیم بهشت
&&
نه نعمتیست که بازآورد فقیر از دوست
\\
نه گر قبول کنندت سپاس داری و بس
&&
که گر هلاک شوی منتی پذیر از دوست
\\
مرا که دیده به دیدار دوست برکردم
&&
حلال نیست که بر هم نهم به تیر از دوست
\\
و گر چنان که مصور شود گزیر از عشق
&&
کجا روم که نمی‌باشدم گزیر از دوست
\\
به هر طریق که باشد اسیر دشمن را
&&
توان خرید و نشاید خرید اسیر از دوست
\\
که در ضمیر من آید ز هر که در عالم
&&
که من هنوز نپرداختم ضمیر از دوست
\\
تو خود نظیر نداری و گر بود به مثل
&&
من آن نیم که بدل گیرم و نظیر از دوست
\\
رضای دوست نگه دار و صبر کن سعدی
&&
که دوستی نبود ناله و نفیر از دوست
\\
\end{longtable}
\end{center}
