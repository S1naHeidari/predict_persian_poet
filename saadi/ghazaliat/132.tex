\begin{center}
\section*{غزل ۱۳۲: چو ابر زلف تو پیرامن قمر می‌گشت}
\label{sec:132}
\addcontentsline{toc}{section}{\nameref{sec:132}}
\begin{longtable}{l p{0.5cm} r}
چو ابر زلف تو پیرامن قمر می‌گشت
&&
ز ابر دیده کنارم به اشک تر می‌گشت
\\
ز شور عشق تو در کام جان خسته من
&&
جواب تلخ تو شیرینتر از شکر می‌گشت
\\
خوی عذار تو بر خاک تیره می‌افتاد
&&
وجود مرده از آن آب جانور می‌گشت
\\
اگر مرا به زر و سیم دسترس بودی
&&
ز سیم سینه تو کار من چو زر می‌گشت
\\
دل از دریچه فکرت به نفس ناطقه داد
&&
نشان حالت زارم که زارتر می‌گشت
\\
ز شوق روی تو اندر سر قلم سودا
&&
فتاد و چون من سودازده به سر می‌گشت
\\
ز خاطرم غزلی سوزناک روی نمود
&&
که در دماغ فراغ من این قدر می‌گشت
\\
\end{longtable}
\end{center}
