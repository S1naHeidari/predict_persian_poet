\begin{center}
\section*{غزل ۶۶: هر کسی را نتوان گفت که صاحب نظرست}
\label{sec:066}
\addcontentsline{toc}{section}{\nameref{sec:066}}
\begin{longtable}{l p{0.5cm} r}
هر کسی را نتوان گفت که صاحب نظر است
&&
عشقبازی دگر و نفس پرستی دگر است
\\
نه هر آن چشم که بینند سیاه است و سپید
&&
یا سپیدی ز سیاهی بشناسد بصر است
\\
هر که در آتش عشقش نبود طاقت سوز
&&
گو به نزدیک مرو کآفت پروانه پر است
\\
گر من از دوست بنالم نفسم صادق نیست
&&
خبر از دوست ندارد که ز خود با خبر است
\\
آدمی صورت اگر دفع کند شهوت نفس
&&
آدمی خوی شود ور نه همان جانور است
\\
شربت از دست دلارام چه شیرین و چه تلخ
&&
بده ای دوست که مستسقی از آن تشنه‌تر است
\\
من خود از عشق لبت فهم سخن می‌نکنم
&&
هرچ از آن تلخترم گر تو بگویی شکر است
\\
ور به تیغم بزنی با تو مرا خصمی نیست
&&
خصم آنم که میان من و تیغت سپر است
\\
من از این بند نخواهم به در آمد همه عمر
&&
بند پایی که به دست تو بود تاج سر است
\\
دست سعدی به جفا نگسلد از دامن دوست
&&
ترک لؤلؤ نتوان گفت که دریا خطر است
\\
\end{longtable}
\end{center}
