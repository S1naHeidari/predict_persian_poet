\begin{center}
\section*{غزل ۳۴۲: یار بیگانه نگیرد هر که دارد یار خویش}
\label{sec:342}
\addcontentsline{toc}{section}{\nameref{sec:342}}
\begin{longtable}{l p{0.5cm} r}
یار بیگانه نگیرد هر که دارد یار خویش
&&
ای که دستی چرب داری پیشتر دیوار خویش
\\
خدمتت را هر که فرمایی کمر بندد به طوع
&&
لیکن آن بهتر که فرمایی به خدمتگار خویش
\\
من هم اول روز گفتم جان فدای روی تو
&&
شرط مردی نیست برگردیدن از گفتار خویش
\\
درد عشق از هر که می‌پرسم جوابم می‌دهد
&&
از که می‌پرسی که من خود عاجزم در کار خویش
\\
صبر چون پروانه باید کردنت بر داغ عشق
&&
ای که صحبت با یکی داری نه در مقدار خویش
\\
یا چو دیدارم نمودی دل نبایستی شکست
&&
یا نبایستی نمود اول مرا دیدار خویش
\\
حد زیبایی ندارند این خداوندان حسن
&&
ای دریغا گر بخوردندی غم غمخوار خویش
\\
عقل را پنداشتم در عشق تدبیری بود
&&
من نخواهم کرد دیگر تکیه بر پندار خویش
\\
هر که خواهد در حق ما هر چه خواهد گو بگوی
&&
ما نمی‌داریم دست از دامن دلدار خویش
\\
روز رستاخیز کان جا کس نپردازد به کس
&&
من نپردازم به هیچ از گفت و گوی یار خویش
\\
سعدیا در کوی عشق از پارسایی دم مزن
&&
هر متاعی را خریداریست در بازار خویش
\\
\end{longtable}
\end{center}
