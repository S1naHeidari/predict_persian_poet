\begin{center}
\section*{غزل ۳۴۱: گرم قبول کنی ور برانی از بر خویش}
\label{sec:341}
\addcontentsline{toc}{section}{\nameref{sec:341}}
\begin{longtable}{l p{0.5cm} r}
گرم قبول کنی ور برانی از بر خویش
&&
نگردم از تو و گر خود فدا کنم سر خویش
\\
تو دانی ار بنوازی و گر بیندازی
&&
چنان که در دلت آید به رای انور خویش
\\
نظر به جانب ما گر چه منت است و ثواب
&&
غلام خویش همی‌پروری و چاکر خویش
\\
اگر برابر خویشم به حکم نگذاری
&&
خیال روی تو نگذارم از برابر خویش
\\
مرا نصیحت بیگانه منفعت نکند
&&
که راضیم که قفا بینم از ستمگر خویش
\\
حدیث صبر من از روی تو همان مثل است
&&
که صبر طفل به شیر از کنار مادر خویش
\\
رواست گر همه خلق از نظر بیندازی
&&
که هیچ خلق نبینی به حسن و منظر خویش
\\
به عشق روی تو گفتم که جان برافشانم
&&
دگر به شرم درافتادم از محقر خویش
\\
تو سر به صحبت سعدی درآوری هیهات
&&
زهی خیال که من کرده‌ام مصور خویش
\\
چه بر سر آید از این شوق غالبم دانی
&&
همانچه مورچه را بر سر آمد از پر خویش
\\
\end{longtable}
\end{center}
