\begin{center}
\section*{غزل ۷۴: شراب از دست خوبان سلسبیلست}
\label{sec:074}
\addcontentsline{toc}{section}{\nameref{sec:074}}
\begin{longtable}{l p{0.5cm} r}
شراب از دست خوبان سلسبیلست
&&
و گر خود خون میخواران سبیلست
\\
نمی‌دانم رطب را چاشنی چیست
&&
همی‌بینم که خرما بر نخیلست
\\
نه وسمست آن به دلبندی خضیبست
&&
نه سرمست آن به جادویی کحیلست
\\
سرانگشتان صاحب دل فریبش
&&
نه در حنا که در خون قتیلست
\\
الا ای کاروان محمل برانید
&&
که ما را بند بر پای رحیلست
\\
هر آن شب در فراق روی لیلی
&&
که بر مجنون رود لیلی طویلست
\\
کمندش می‌دواند پای مشتاق
&&
بیابان را نپرسد چند میلست
\\
چو مور افتان و خیزان رفت باید
&&
و گر خود ره به زیر پای پیلست
\\
حبیب آن جا که دستی برفشاند
&&
محب ار سر نیفشاند بخیلست
\\
ز ما گر طاعت آید شرمساریم
&&
و ز ایشان گر قبیح آید جمیلست
\\
بدیل دوستان گیرند و یاران
&&
ولیکن شاهد ما بی‌بدیلست
\\
سخن بیرون مگوی از عشق سعدی
&&
سخن عشقست و دیگر قال و قیلست
\\
\end{longtable}
\end{center}
