\begin{center}
\section*{غزل ۲۶۶: در من این عیب قدیمست و به در می‌نرود}
\label{sec:266}
\addcontentsline{toc}{section}{\nameref{sec:266}}
\begin{longtable}{l p{0.5cm} r}
در من این عیب قدیمست و به در می‌نرود
&&
که مرا بی می و معشوق به سر می‌نرود
\\
صبرم از دوست مفرمای و تعنت بگذار
&&
کاین بلاییست که از طبع بشر می‌نرود
\\
مرغ مألوف که با خانه خدا انس گرفت
&&
گر به سنگش بزنی جای دگر می‌نرود
\\
عجب از دیده گریان منت می‌آید
&&
عجب آنست کز او خون جگر می‌نرود
\\
من از این بازنیایم که گرفتم در پیش
&&
اگرم می‌رود از پیش اگر می‌نرود
\\
خواستم تا نظری بنگرم و بازآیم
&&
گفت از این کوچه ما راه به در می‌نرود
\\
جور معشوق چنان نیست که الزام رقیب
&&
گویی ابریست که از پیش قمر می‌نرود
\\
تا تو منظور پدید آمدی ای فتنه پارس
&&
هیچ دل نیست که دنبال نظر می‌نرود
\\
زخم شمشیر غمت را به شکیبایی و عقل
&&
چند مرهم بنهادیم و اثر می‌نرود
\\
ترک دنیا و تماشا و تنعم گفتیم
&&
مهر مهریست که چون نقش حجر می‌نرود
\\
موضعی در همه آفاق ندانم امروز
&&
کز حدیث من و حسن تو خبر می‌نرود
\\
ای که گفتی مرو اندر پی خوبان سعدی
&&
چند گویی مگس از پیش شکر می‌نرود
\\
\end{longtable}
\end{center}
