\begin{center}
\section*{غزل ۵۶: ای کاب زندگانی من در دهان توست}
\label{sec:056}
\addcontentsline{toc}{section}{\nameref{sec:056}}
\begin{longtable}{l p{0.5cm} r}
ای کآب زندگانی من در دهان توست
&&
تیر هلاک ظاهر من در کمان توست
\\
گر برقعی فرونگذاری بدین جمال
&&
در شهر هر که کشته شود در ضمان توست
\\
تشبیه روی تو نکنم من به آفتاب
&&
کاین مدح آفتاب نه تعظیم شان توست
\\
گر یک نظر به گوشهٔ چشم ارادتی
&&
با ما کنی و گر نکنی حکم از آن توست
\\
هر روز خلق را سر یاری و صاحبیست
&&
ما را همین سر است که بر آستان توست
\\
بسیار دیده‌ایم درختان میوه‌دار
&&
زین به ندیده‌ایم که در بوستان توست
\\
گر دست دوستان نرسد باغ را چه جرم
&&
منعی که می‌رود گنه از باغبان توست
\\
بسیار در دل آمد از اندیشه‌ها و رفت
&&
نقشی که آن نمی‌رود از دل نشان توست
\\
با من هزار نوبت اگر دشمنی کنی
&&
ای دوست همچنان دل من مهربان توست
\\
سعدی به قدر خویش تمنای وصل کن
&&
سیمرغ ما چه لایق زاغ آشیان توست
\\
\end{longtable}
\end{center}
