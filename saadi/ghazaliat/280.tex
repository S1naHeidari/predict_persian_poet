\begin{center}
\section*{غزل ۲۸۰: مرو به خواب که خوابت ز چشم برباید}
\label{sec:280}
\addcontentsline{toc}{section}{\nameref{sec:280}}
\begin{longtable}{l p{0.5cm} r}
مرو به خواب که خوابت ز چشم برباید
&&
گرت مشاهده خویش در خیال آید
\\
مجال صبر همین بود و منتهای شکیب
&&
دگر مپای که عمر این همه نمی‌پاید
\\
چه ارمغانی از آن به که دوستان بینی
&&
تو خود بیا که دگر هیچ در نمی‌باید
\\
اگر چه صاحب حسنند در جهان بسیار
&&
چو آفتاب برآید ستاره ننماید
\\
ز نقش روی تو مشاطه دست بازکشید
&&
که شرم داشت که خورشید را بیاراید
\\
به لطف دلبر من در جهان نبینی دوست
&&
که دشمنی کند و دوستی بیفزاید
\\
نه زنده را به تو میلست و مهربانی و بس
&&
که مرده را به نسیمت روان بیاساید
\\
دریغ نیست مرا هر چه هست در طلبت
&&
دلی چه باشد و جانی چه در حساب آید
\\
چرا و چون نرسد دردمند عاشق را
&&
مگر مطاوعت دوست تا چه فرماید
\\
گر آه سینه سعدی رسد به حضرت دوست
&&
چه جای دوست که دشمن بر او ببخشاید
\\
\end{longtable}
\end{center}
