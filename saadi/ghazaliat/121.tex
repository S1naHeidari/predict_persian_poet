\begin{center}
\section*{غزل ۱۲۱: با فراقت چند سازم برگ تنهاییم نیست}
\label{sec:121}
\addcontentsline{toc}{section}{\nameref{sec:121}}
\begin{longtable}{l p{0.5cm} r}
با فراقت چند سازم برگ تنهاییم نیست
&&
دستگاه صبر و پایاب شکیباییم نیست
\\
ترسم از تنهایی احوالم به رسوایی کشد
&&
ترس تنهاییست ور نه بیم رسواییم نیست
\\
مرد گستاخی نیم تا جان در آغوشت کشم
&&
بوسه بر پایت دهم چون دست بالاییم نیست
\\
بر گلت آشفته‌ام بگذار تا در باغ وصل
&&
زاغ بانگی می‌کنم چون بلبل آواییم نیست
\\
تا مصور گشت در چشمم خیال روی دوست
&&
چشم خودبینی ندارم روی خودراییم نیست
\\
درد دوری می‌کشم گر چه خراب افتاده‌ام
&&
بار جورت می‌برم گر چه تواناییم نیست
\\
طبع تو سیر آمد از من جای دیگر دل نهاد
&&
من که را جویم که چون تو طبع هرجاییم نیست
\\
سعدی آتش زبانم در غمت سوزان چو شمع
&&
با همه آتش زبانی در تو گیراییم نیست
\\
\end{longtable}
\end{center}
