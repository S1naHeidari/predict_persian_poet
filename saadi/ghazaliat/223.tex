\begin{center}
\section*{غزل ۲۲۳: عیب جویانم حکایت پیش جانان گفته‌اند}
\label{sec:223}
\addcontentsline{toc}{section}{\nameref{sec:223}}
\begin{longtable}{l p{0.5cm} r}
عیب جویانم حکایت پیش جانان گفته‌اند
&&
من خود این پیدا همی‌گویم که پنهان گفته‌اند
\\
پیش از این گویند کز عشقت پریشانست حال
&&
گر بگفتندی که مجموعم پریشان گفته‌اند
\\
پرده بر عیبم نپوشیدند و دامن بر گناه
&&
جرم درویشی چه باشد تا به سلطان گفته‌اند
\\
تا چه مرغم کم حکایت پیش عنقا کرده‌اند
&&
یا چه مورم کم سخن نزد سلیمان گفته‌اند
\\
دشمنی کردند با من لیکن از روی قیاس
&&
دوستی باشد که دردم پیش درمان گفته‌اند
\\
ذکر سودای زلیخا پیش یوسف کرده‌اند
&&
حال سرگردانی آدم به رضوان گفته‌اند
\\
داغ پنهانم نمی‌بینند و مهر سر به مهر
&&
آن چه بر اجزای ظاهر دیده‌اند آن گفته‌اند
\\
ور نگفتندی چه حاجت کآب چشم و رنگ روی
&&
ماجرای عشق از اول تا به پایان گفته‌اند
\\
پیش از این گویند سعدی دوست می‌دارد تو را
&&
پیش از آنت دوست می‌دارم که ایشان گفته‌اند
\\
عاشقان دارند کار و عارفان دانند حال
&&
این سخن در دل فرود آید که از جان گفته‌اند
\\
\end{longtable}
\end{center}
