\begin{center}
\section*{غزل ۵۴۰: خلاف شرط محبت چه مصلحت دیدی}
\label{sec:540}
\addcontentsline{toc}{section}{\nameref{sec:540}}
\begin{longtable}{l p{0.5cm} r}
خلاف شرط محبت چه مصلحت دیدی
&&
که برگذشتی و از دوستان نپرسیدی
\\
گرفتمت که نیامد ز روی خلق آزرم
&&
که بی‌گنه بکشی از خدا نترسیدی
\\
بپوش روی نگارین و موی مشکین را
&&
که حسن طلعت خورشید را بپوشیدی
\\
هزار بیدل مشتاق را به حسرت آن
&&
که لب به لب برسد جان به لب رسانیدی
\\
محل و قیمت خویش آن زمان بدانستم
&&
که برگذشتی و ما را به هیچ نخریدی
\\
هزار بار بگفتیم و هیچ درنگرفت
&&
که گرد عشق مگرد ای فقیر و گردیدی
\\
تو را ملامت رندان و عاشقان سعدی
&&
دگر حلال نباشد که خود بلغزیدی
\\
به تیغ می‌زد و می‌رفت و باز می‌نگریست
&&
که ترک عشق نگفتی سزای خود دیدی
\\
\end{longtable}
\end{center}
