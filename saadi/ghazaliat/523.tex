\begin{center}
\section*{غزل ۵۲۳: همه عمر برندارم سر از این خمار مستی}
\label{sec:523}
\addcontentsline{toc}{section}{\nameref{sec:523}}
\begin{longtable}{l p{0.5cm} r}
همه عمر برندارم سر از این خمار مستی
&&
که هنوز من نبودم که تو در دلم نشستی
\\
تو نه مثل آفتابی که حضور و غیبت افتد
&&
دگران روند و آیند و تو همچنان که هستی
\\
چه حکایت از فراقت که نداشتم ولیکن
&&
تو چو روی باز کردی در ماجرا ببستی
\\
نظری به دوستان کن که هزار بار از آن به
&&
که تحیتی نویسی و هدیتی فرستی
\\
دل دردمند ما را که اسیر توست یارا
&&
به وصال مرهمی نه چو به انتظار خستی
\\
نه عجب که قلب دشمن شکنی به روز هیجا
&&
تو که قلب دوستان را به مفارقت شکستی
\\
برو ای فقیه دانا به خدای بخش ما را
&&
تو و زهد و پارسایی من و عاشقی و مستی
\\
دل هوشمند باید که به دلبری سپاری
&&
که چو قبله‌ایت باشد به از آن که خود پرستی
\\
چو زمام بخت و دولت نه به دست جهد باشد
&&
چه کنند اگر زبونی نکنند و زیردستی
\\
گله از فراق یاران و جفای روزگاران
&&
نه طریق توست سعدی کم خویش گیر و رستی
\\
\end{longtable}
\end{center}
