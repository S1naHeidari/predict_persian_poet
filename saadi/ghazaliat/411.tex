\begin{center}
\section*{غزل ۴۱۱: گر تیغ برکشد که محبان همی‌زنم}
\label{sec:411}
\addcontentsline{toc}{section}{\nameref{sec:411}}
\begin{longtable}{l p{0.5cm} r}
گر تیغ برکشد که محبان همی‌زنم
&&
اول کسی که لاف محبت زند منم
\\
گویند پای دار اگرت سر دریغ نیست
&&
گو سر قبول کن که به پایش درافکنم
\\
امکان دیده بستنم از روی دوست نیست
&&
اولیتر آن که گوش نصیحت بیاکنم
\\
آورده‌اند صحبت خوبان که آتش است
&&
بر من به نیم جو که بسوزند خرمنم
\\
من مرغ زیرکم که چنانم خوش اوفتاد
&&
در قید او که یاد نیاید نشیمنم
\\
دردیست در دلم که گر از پیش آب چشم
&&
برگیرم آستین برود تا به دامنم
\\
گر پیرهن به در کنم از شخص ناتوان
&&
بینی که زیر جامه خیالیست یا تنم
\\
شرط است احتمال جفاهای دشمنان
&&
چون دل نمی‌دهد که دل از دوست برکنم
\\
دردی نبوده را چه تفاوت کند که من
&&
بیچاره درد می‌خورم و نعره می‌زنم
\\
بر تخت جم پدید نیاید شب دراز
&&
من دانم این حدیث که در چاه بیژنم
\\
گویند سعدیا مکن از عشق توبه کن
&&
مشکل توانم و نتوانم که نشکنم
\\
\end{longtable}
\end{center}
