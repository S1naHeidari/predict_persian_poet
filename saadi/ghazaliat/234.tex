\begin{center}
\section*{غزل ۲۳۴: آفتاب از کوه سر بر می‌زند}
\label{sec:234}
\addcontentsline{toc}{section}{\nameref{sec:234}}
\begin{longtable}{l p{0.5cm} r}
آفتاب از کوه سر بر می‌زند
&&
ماهروی انگشت بر در می‌زند
\\
آن کمان ابرو که تیر غمزه‌اش
&&
هر زمانی صید دیگر می‌زند
\\
دست و ساعد می‌کشد درویش را
&&
تا نپنداری که خنجر می‌زند
\\
یاسمین بویی که سرو قامتش
&&
طعنه بر بالای عرعر می‌زند
\\
روی و چشمی دارم اندر مهر او
&&
کاین گهر می‌ریزد آن زر می‌زند
\\
عشق را پیشانیی باید چو میخ
&&
تا حبیبش سنگ بر سر می‌زند
\\
انگبین رویان نترسند از مگس
&&
نوش می‌گیرند و نشتر می‌زند
\\
در به روی دوست بستن شرط نیست
&&
ور ببندی سر به در بر می‌زند
\\
سعدیا دیگر قلم پولاد دار
&&
کاین سخن آتش به نی در می‌زند
\\
\end{longtable}
\end{center}
