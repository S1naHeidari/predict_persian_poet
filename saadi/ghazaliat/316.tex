\begin{center}
\section*{غزل ۳۱۶: ساقی سیمتن چه خسبی خیز}
\label{sec:316}
\addcontentsline{toc}{section}{\nameref{sec:316}}
\begin{longtable}{l p{0.5cm} r}
ساقی سیمتن چه خسبی خیز
&&
آب شادی بر آتش غم ریز
\\
بوسه‌ای بر کنار ساغر نه
&&
پس بگردان شراب شهدآمیز
\\
کابر آذار و باد نوروزی
&&
درفشان می‌کنند و عنبربیز
\\
جهد کردیم تا نیالاید
&&
به خرابات دامن پرهیز
\\
دست بالای عشق زور آورد
&&
معرفت را نماند جای ستیز
\\
گفتم ای عقل زورمند چرا
&&
برگرفتی ز عشق راه گریز
\\
گفت اگر گربه شیر نر گردد
&&
نکند با پلنگ دندان تیز
\\
شاهدان می‌کنند خانه زهد
&&
مطربان می‌زنند راه حجاز
\\
توبه را تلخ می‌کند در حلق
&&
یار شیرین زبان شورانگیز
\\
سعدیا هر دمت که دست دهد
&&
به سر زلف دوستان آویز
\\
دشمنان را به حال خود بگذار
&&
تا قیامت کنند و رستاخیز
\\
\end{longtable}
\end{center}
