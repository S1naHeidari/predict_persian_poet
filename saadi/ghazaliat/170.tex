\begin{center}
\section*{غزل ۱۷۰: غلام آن سبک روحم که با من سر گران دارد}
\label{sec:170}
\addcontentsline{toc}{section}{\nameref{sec:170}}
\begin{longtable}{l p{0.5cm} r}
غلام آن سبک روحم که با من سر گران دارد
&&
جوابش تلخ و پنداری شکر زیر زبان دارد
\\
مرا گر دوستی با او به دوزخ می‌برد شاید
&&
به نقد اندر بهشتست آن که یاری مهربان دارد
\\
کسی را کاختیاری هست و محبوبی و مشروبی
&&
مراد از بخت و حظ از عمر و مقصود از جهان دارد
\\
برون از خوردن و خفتن حیاتی هست مردم را
&&
به جانان زندگانی کن بهایم نیز جان دارد
\\
محبت با کسی دارم کز او باخود نمی‌آیم
&&
چو بلبل کز نشاط گل فراغ از آشیان دارد
\\
نه مردی گر به شمشیر از جفای دوست برگردی
&&
دهل را کاندرون بادست ز انگشتی فغان دارد
\\
به تشویش قیامت در که یار از یار بگریزد
&&
محب از خاک برخیزد محبت همچنان دارد
\\
خوش آمد باد نوروزی به صبح از باغ پیروزی
&&
به بوی دوستان ماند نه بوی بوستان دارد
\\
یکی سر بر کنار یار و خواب صبح مستولی
&&
چه غم دارد ز مسکینی که سر بر آستان دارد
\\
چو سعدی عشق تنها باز و راحت بین و آسایش
&&
به تنها ملک می‌راند که منظوری نهان دارد
\\
\end{longtable}
\end{center}
