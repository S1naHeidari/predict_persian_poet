\begin{center}
\section*{غزل ۴۳۱: امشب آن نیست که در خواب رود چشم ندیم}
\label{sec:431}
\addcontentsline{toc}{section}{\nameref{sec:431}}
\begin{longtable}{l p{0.5cm} r}
امشب آن نیست که در خواب رود چشم ندیم
&&
خواب در روضه رضوان نکند اهل نعیم
\\
خاک را زنده کند تربیت باد بهار
&&
سنگ باشد که دلش زنده نگردد به نسیم
\\
بوی پیراهن گم کرده خود می‌شنوم
&&
گر بگویم همه گویند ضلالیست قدیم
\\
عاشق آن گوش ندارد که نصیحت شنود
&&
درد ما نیک نباشد به مداوای حکیم
\\
توبه گویندم از اندیشه معشوق بکن
&&
هرگز این توبه نباشد که گناهیست عظیم
\\
ای رفیقان سفر دست بدارید از ما
&&
که بخواهیم نشستن به در دوست مقیم
\\
ای برادر غم عشق آتش نمرود انگار
&&
بر من این شعله چنان است که بر ابراهیم
\\
مرده از خاک لحد رقص کنان برخیزد
&&
گر تو بالای عظامش گذری و هی رمیم
\\
طمع وصل تو می‌دارم و اندیشه هجر
&&
دیگر از هر چه جهانم نه امید است و نه بیم
\\
عجب از کشته نباشد به در خیمه دوست
&&
عجب از زنده که چون جان به درآورد سلیم
\\
سعدیا عشق نیامیزد و شهوت با هم
&&
پیش تسبیح ملایک نرود دیو رجیم
\\
\end{longtable}
\end{center}
