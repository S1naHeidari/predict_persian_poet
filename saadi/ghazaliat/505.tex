\begin{center}
\section*{غزل ۵۰۵: خبرت خرابتر کرد جراحت جدایی}
\label{sec:505}
\addcontentsline{toc}{section}{\nameref{sec:505}}
\begin{longtable}{l p{0.5cm} r}
خبرت خرابتر کرد جراحت جدایی
&&
چو خیال آب روشن که به تشنگان نمایی
\\
تو چه ارمغانی آری که به دوستان فرستی
&&
چه از این به ارمغانی که تو خویشتن بیایی
\\
بشدی و دل ببردی و به دست غم سپردی
&&
شب و روز در خیالی و ندانمت کجایی
\\
دل خویش را بگفتم چو تو دوست می‌گرفتم
&&
نه عجب که خوبرویان بکنند بی‌وفایی
\\
تو جفای خود بکردی و نه من نمی‌توانم
&&
که جفا کنم ولیکن نه تو لایق جفایی
\\
چه کنند اگر تحمل نکنند زیردستان
&&
تو هر آن ستم که خواهی بکنی که پادشایی
\\
سخنی که با تو دارم به نسیم صبح گفتم
&&
دگری نمی‌شناسم تو ببر که آشنایی
\\
من از آن گذشتم ای یار که بشنوم نصیحت
&&
برو ای فقیه و با ما مفروش پارسایی
\\
تو که گفته‌ای تأمل نکنم جمال خوبان
&&
بکنی اگر چو سعدی نظری بیازمایی
\\
در چشم بامدادان به بهشت برگشودن
&&
نه چنان لطیف باشد که به دوست برگشایی
\\
\end{longtable}
\end{center}
