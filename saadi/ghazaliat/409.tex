\begin{center}
\section*{غزل ۴۰۹: تا خبر دارم از او بی‌خبر از خویشتنم}
\label{sec:409}
\addcontentsline{toc}{section}{\nameref{sec:409}}
\begin{longtable}{l p{0.5cm} r}
تا خبر دارم از او بی‌خبر از خویشتنم
&&
با وجودش ز من آواز نیاید که منم
\\
پیرهن می‌بدرم دم به دم از غایت شوق
&&
که وجودم همه او گشت و من این پیرهنم
\\
ای رقیب این همه سودا مکن و جنگ مجوی
&&
برکنم دیده که من دیده از او برنکنم
\\
خود گرفتم که نگویم که مرا واقعه‌ایست
&&
دشمن و دوست بدانند قیاس از سخنم
\\
در همه شهر فراهم ننشست انجمنی
&&
که نه من در غمش افسانه آن انجمنم
\\
برشکست از من و از رنج دلم باک نداشت
&&
من نه آنم که توانم که از او برشکنم
\\
گر همین سوز رود با من مسکین در گور
&&
خاک اگر بازکنی سوخته یابی کفنم
\\
گر به خون تشنه‌ای اینک من و سر باکی نیست
&&
که به فتراک تو به زان که بود بر بدنم
\\
مرد و زن گر به جفا کردن من برخیزند
&&
گر بگردم ز وفای تو نه مردم که زنم
\\
شرط عقل است که مردم بگریزند از تیر
&&
من گر از دست تو باشد مژه بر هم نزنم
\\
تا به گفتار درآمد دهن شیرینت
&&
بیم آن است که شوری به جهان درفکنم
\\
لب سعدی و دهانت ز کجا تا به کجا
&&
این قدر بس که رود نام لبت بر دهنم
\\
\end{longtable}
\end{center}
