\begin{center}
\section*{غزل ۵۲: آن ماه دوهفته در نقابست}
\label{sec:052}
\addcontentsline{toc}{section}{\nameref{sec:052}}
\begin{longtable}{l p{0.5cm} r}
آن ماه دوهفته در نقاب است
&&
یا حوری دست در خضاب است
\\
سیلاب ز سر گذشت یارا
&&
ز اندازه به در مبر جفا را
\\
تندی و جفا و زشتخویی
&&
هر چند که می‌کنی نکویی
\\
ای روی تو از بهشت بابی
&&
دل بر نمک لبت کبابی
\\
صبر از تو کسی نیاورد تاب
&&
چشمم ز غمت نمی‌برد خواب
\\
ای شهرهٔ شهر و فتنهٔ خیل
&&
فی منظرک النهار و اللیل
\\
ای داروی دلپذیر دردم
&&
اقرار به بندگیت کردم
\\
گر چه تو امیر و ما اسیریم
&&
گر چه تو بزرگ و ما حقیریم
\\
ای سرو روان و گلبن نو
&&
مه پیکر آفتاب پرتو
\\
امشب شب خلوت است تا روز
&&
ای طالع سعد و بخت فیروز
\\
ساقی قدحی قلندری وار
&&
در ده به معاشران هشیار
\\
باد است غرور زندگانی
&&
برق است لوامع جوانی
\\
این گرسنه گرگ بی ترحم
&&
خود سیر نمی‌شود ز مردم
\\
سعدی تو نه مرد وصل اویی
&&
تا لاف زنی و قرب جویی
\\
\end{longtable}
\end{center}
