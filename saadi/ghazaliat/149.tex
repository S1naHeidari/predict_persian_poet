\begin{center}
\section*{غزل ۱۴۹: چه لطیفست قبا بر تن چون سرو روانت}
\label{sec:149}
\addcontentsline{toc}{section}{\nameref{sec:149}}
\begin{longtable}{l p{0.5cm} r}
چه لطیفست قبا بر تن چون سرو روانت
&&
آه اگر چون کمرم دست رسیدی به میانت
\\
در دلم هیچ نیاید مگر اندیشه وصلت
&&
تو نه آنی که دگر کس بنشیند به مکانت
\\
گر تو خواهی که یکی را سخن تلخ بگویی
&&
سخن تلخ نباشد چو برآید به دهانت
\\
نه من انگشت نمایم به هواداری رویت
&&
که تو انگشت نمایی و خلایق نگرانت
\\
در اندیشه ببستم قلم وهم شکستم
&&
که تو زیباتر از آنی که کنم وصف و بیانت
\\
سرو را قامت خوبست و قمر را رخ زیبا
&&
تو نه آنی و نه اینی که هم اینست و هم آنت
\\
ای رقیب ار نگشایی در دلبند به رویم
&&
این قدر بازنمایی که دعا گفت فلانت
\\
من همه عمر بر آنم که دعاگوی تو باشم
&&
گر تو باشی که نباشم تن من برخی جانت
\\
سعدیا چاره ثباتست و مدارا و تحمل
&&
من که محتاج تو باشم ببرم بار گرانت
\\
\end{longtable}
\end{center}
