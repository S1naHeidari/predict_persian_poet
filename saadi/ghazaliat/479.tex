\begin{center}
\section*{غزل ۴۷۹: من از دست کمانداران ابرو}
\label{sec:479}
\addcontentsline{toc}{section}{\nameref{sec:479}}
\begin{longtable}{l p{0.5cm} r}
من از دست کمانداران ابرو
&&
نمی‌یارم گذر کردن به هر سو
\\
دو چشمم خیره ماند از روشنایی
&&
ندانم قرص خورشید است یا رو
\\
بهشت است این که من دیدم نه رخسار
&&
کمند است آن که وی دارد نه گیسو
\\
لبان لعل چون خون کبوتر
&&
سواد زلف چون پر پرستو
\\
نه آن سرپنجه دارد شوخ عیار
&&
که با او بر توان آمد به بازو
\\
همه جان خواهد از عشاق مشتاق
&&
ندارد سنگ کوچک در ترازو
\\
نفس را بوی خوش چندین نباشد
&&
مگر در جیب دارد ناف آهو
\\
لب خندان شیرین منطقش را
&&
نشاید گفت جز ضحاک جادو
\\
غریبی سخت محبوب اوفتاده‌ست
&&
به ترکستان رویش خال هندو
\\
عجب گر در چمن برپای خیزد
&&
که پیشش سرو ننشیند به زانو
\\
و گر بنشیند اندر محفل عام
&&
دو صد فریاد برخیزد ز هر سو
\\
به یاد روی گلبوی گل اندام
&&
همه شب خار دارم زیر پهلو
\\
تحمل کن جفای یار سعدی
&&
که جور نیکوان ذنبیست معفو
\\
\end{longtable}
\end{center}
