\begin{center}
\section*{غزل ۵۰: عشق ورزیدم و عقلم به ملامت برخاست}
\label{sec:050}
\addcontentsline{toc}{section}{\nameref{sec:050}}
\begin{longtable}{l p{0.5cm} r}
عشق ورزیدم و عقلم به ملامت برخاست
&&
کان که عاشق شد از او حکم سلامت برخاست
\\
هر که با شاهد گلروی به خلوت بنشست
&&
نتواند ز سر راه ملامت برخاست
\\
که شنیدی که برانگیخت سمند غم عشق
&&
که نه اندر عقبش گرد ندامت برخاست
\\
عشق غالب شد و از گوشه نشینان صلاح
&&
نام مستوری و ناموس کرامت برخاست
\\
در گلستانی کان گلبن خندان بنشست
&&
سرو آزاد به یک پای غرامت برخاست
\\
گل صدبرگ ندانم به چه رونق بشکفت
&&
یا صنوبر به کدامین قد و قامت برخاست
\\
دی زمانی به تکلف بر سعدی بنشست
&&
فتنه بنشست چو برخاست قیامت برخاست
\\
\end{longtable}
\end{center}
