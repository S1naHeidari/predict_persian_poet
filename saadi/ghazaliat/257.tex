\begin{center}
\section*{غزل ۲۵۷: از دست دوست هر چه ستانی شکر بود}
\label{sec:257}
\addcontentsline{toc}{section}{\nameref{sec:257}}
\begin{longtable}{l p{0.5cm} r}
از دست دوست هر چه ستانی شکر بود
&&
وز دست غیر دوست تبرزد تبر بود
\\
دشمن گر آستین گل افشاندت به روی
&&
از تیر چرخ و سنگ فلاخن بتر بود
\\
گر خاک پای دوست خداوند شوق را
&&
در دیدگان کشند جلای بصر بود
\\
شرط وفاست آن که چو شمشیر برکشد
&&
یار عزیز جان عزیزش سپر بود
\\
یا رب هلاک من مکن الا به دست دوست
&&
تا وقت جان سپردنم اندر نظر بود
\\
گر جان دهی و گر سر بیچارگی نهی
&&
در پای دوست هر چه کنی مختصر بود
\\
ما سر نهاده‌ایم تو دانی و تیغ و تاج
&&
تیغی که ماهروی زند تاج سر بود
\\
مشتاق را که سر برود در وفای یار
&&
آن روز روز دولت و روز ظفر بود
\\
ما ترک جان از اول این کار گفته‌ایم
&&
آن را که جان عزیز بود در خطر بود
\\
آن کز بلا بترسد و از قتل غم خورد
&&
او عاقلست و شیوه مجنون دگر بود
\\
با نیم پختگان نتوان گفت سوز عشق
&&
خام از عذاب سوختگان بی‌خبر بود
\\
جانا دل شکسته سعدی نگاه دار
&&
دانی که آه سوختگان را اثر بود
\\
\end{longtable}
\end{center}
