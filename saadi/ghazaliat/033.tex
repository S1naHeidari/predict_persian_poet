\begin{center}
\section*{غزل ۳۳: کهن شود همه کس را به روزگار ارادت}
\label{sec:033}
\addcontentsline{toc}{section}{\nameref{sec:033}}
\begin{longtable}{l p{0.5cm} r}
کهن شود همه کس را به روزگار ارادت
&&
مگر مرا که همان عشق اولست و زیادت
\\
گرم جواز نباشد به پیشگاه قبولت
&&
کجا روم که نمیرم بر آستان عبادت
\\
مرا به روز قیامت مگر حساب نباشد
&&
که هجر و وصل تو دیدم چه جای موت و اعادت
\\
شنیدمت که نظر می‌کنی به حال ضعیفان
&&
تبم گرفت و دلم خوش به انتظار عیادت
\\
گرم به گوشه چشمی شکسته وار ببینی
&&
فلک شوم به بزرگی و مشتری به سعادت
\\
بیایمت که ببینم کدام زهره و یارا
&&
روم که بی تو نشینم کدام صبر و جلادت
\\
مرا هرآینه روزی تمام کشته ببینی
&&
گرفته دامن قاتل به هر دو دست ارادت
\\
اگر جنازه سعدی به کوی دوست برآرند
&&
زهی حیات نکونام و رفتنی به شهادت
\\
\end{longtable}
\end{center}
