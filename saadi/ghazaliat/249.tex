\begin{center}
\section*{غزل ۲۴۹: این جا شکری هست که چندین مگسانند}
\label{sec:249}
\addcontentsline{toc}{section}{\nameref{sec:249}}
\begin{longtable}{l p{0.5cm} r}
این جا شکری هست که چندین مگسانند
&&
یا بوالعجبی کاین همه صاحب هوسانند
\\
بس در طلبت سعی نمودیم و نگفتی
&&
کاین هیچ کسان در طلب ما چه کسانند
\\
ای قافله سالار چنین گرم چه رانی
&&
آهسته که در کوه و کمر بازپسانند
\\
صد مشعله افروخته گردد به چراغی
&&
این نور تو داری و دگر مقتبسانند
\\
من قلب و لسانم به وفاداری و صحبت
&&
و اینان همه قلبند که پیش تو لسانند
\\
آنان که شب آرام نگیرند ز فکرت
&&
چون صبح پدیدست که صادق نفسانند
\\
و آنان که به دیدار چنان میل ندارند
&&
سوگند توان خورد که بی عقل و خسانند
\\
دانی چه جفا می‌رود از دست رقیبت
&&
حیفست که طوطی و زغن هم قفسانند
\\
در طالع من نیست که نزدیک تو باشم
&&
می‌گویمت از دور دعا گر برسانند
\\
\end{longtable}
\end{center}
