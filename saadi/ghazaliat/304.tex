\begin{center}
\section*{غزل ۳۰۴: آن کیست که می‌رود به نخجیر}
\label{sec:304}
\addcontentsline{toc}{section}{\nameref{sec:304}}
\begin{longtable}{l p{0.5cm} r}
آن کیست که می‌رود به نخجیر
&&
پای دل دوستان به زنجیر
\\
همشیره جادوان بابل
&&
همسایه لعبتان کشمیر
\\
این است بهشت اگر شنیدی
&&
کز دیدن آن جوان شود پیر
\\
از عشق کمان دست و بازوش
&&
افتاده خبر ندارد از تیر
\\
نقاش که صورتش ببیند
&&
از دست بیفکند تصاویر
\\
ای سخت جفای سست پیوند
&&
رفتی و چنین برفت تقدیر
\\
کوته نظران ملامت از عشق
&&
بی فایده می‌کنند و تحذیر
\\
با جان من از جسد برآید
&&
خونی که فروشده‌ست با شیر
\\
گر جان طلبد حبیب عشاق
&&
نه منع روا بود نه تأخیر
\\
آن را که مراد دوست باید
&&
گو ترک مراد خویشتن گیر
\\
سعدی چو اسیر عشق ماندی
&&
تدبیر تو چیست ترک تدبیر
\\
\end{longtable}
\end{center}
