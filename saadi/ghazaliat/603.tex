\begin{center}
\section*{غزل ۶۰۳: اگر تو میل محبت کنی و گر نکنی}
\label{sec:603}
\addcontentsline{toc}{section}{\nameref{sec:603}}
\begin{longtable}{l p{0.5cm} r}
اگر تو میل محبت کنی و گر نکنی
&&
من از تو روی نپیچم که مستحب منی
\\
چو سرو در چمنی راست در تصور من
&&
چه جای سرو که مانند روح در بدنی
\\
به صید عالمیانت کمند حاجت نیست
&&
همین بس است که برقع ز روی برفکنی
\\
بیاض ساعد سیمین مپوش در صف جنگ
&&
که بی تکلف شمشیر لشکری بزنی
\\
مبارزان جهان قلب دشمنان شکنند
&&
تو را چه شد که همه قلب دوستان شکنی
\\
عجب در آن نه که آفاق در تو حیرانند
&&
تو هم در آینه حیران حسن خویشتنی
\\
تو را که در نظر آمد جمال طلعت خویش
&&
حقیقت است که دیگر نظر به ما نکنی
\\
کسی در آینه شخصی بدین صفت بیند
&&
کند هر آینه جور و جفا و کبر و منی
\\
در آن دهن که تو داری سخن نمی‌گنجد
&&
من آدمی نشنیدم بدین شکردهنی
\\
شنیده‌ای که مقالات سعدی از شیراز
&&
همی‌برند به عالم چو نافه ختنی
\\
مگر که نام خوشت بر دهان من بگذشت
&&
برفت نام من اندر جهان به خوش سخنی
\\
\end{longtable}
\end{center}
