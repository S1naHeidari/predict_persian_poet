\begin{center}
\section*{غزل ۲۰۱: آن به که نظر باشد و گفتار نباشد}
\label{sec:201}
\addcontentsline{toc}{section}{\nameref{sec:201}}
\begin{longtable}{l p{0.5cm} r}
آن به که نظر باشد و گفتار نباشد
&&
تا مدعی اندر پس دیوار نباشد
\\
آن بر سر گنج است که چون نقطه به کنجی
&&
بنشیند و سرگشته چو پرگار نباشد
\\
ای دوست برآور دری از خلق به رویم
&&
تا هیچ کسم واقف اسرار نباشد
\\
می خواهم و معشوق و زمینی و زمانی
&&
کاو باشد و من باشم و اغیار نباشد
\\
پندم مده ای دوست که دیوانه سرمست
&&
هرگز به سخن عاقل و هشیار نباشد
\\
با صاحب شمشیر مبادت سر و کاری
&&
الا به سر خویشتنت کار نباشد
\\
سهل است به خون من اگر دست برآری
&&
جان دادن در پای تو دشوار نباشد
\\
ماهت نتوان خواند بدین صورت و گفتار
&&
مه را لب و دندان شکربار نباشد
\\
وان سرو که گویند به بالای تو باشد
&&
هرگز به چنین قامت و رفتار نباشد
\\
ما توبه شکستیم که در مذهب عشاق
&&
صوفی نپسندند که خمار نباشد
\\
هر پای که در خانه فرو رفت به گنجی
&&
دیگر همه عمرش سر بازار نباشد
\\
عطار که در عین گلاب است عجب نیست
&&
گر وقت بهارش سر گلزار نباشد
\\
مردم همه دانند که در نامه سعدی
&&
مشکیست که در کلبه عطار نباشد
\\
جان در سر کار تو کند سعدی و غم نیست
&&
کان یار نباشد که وفادار نباشد
\\
\end{longtable}
\end{center}
