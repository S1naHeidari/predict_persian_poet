\begin{center}
\section*{غزل ۴۰۲: من بی‌مایه که باشم که خریدار تو باشم}
\label{sec:402}
\addcontentsline{toc}{section}{\nameref{sec:402}}
\begin{longtable}{l p{0.5cm} r}
من بی‌مایه که باشم که خریدار تو باشم
&&
حیف باشد که تو یار من و من یار تو باشم
\\
تو مگر سایه لطفی به سر وقت من آری
&&
که من آن مایه ندارم که به مقدار تو باشم
\\
خویشتن بر تو نبندم که من از خود نپسندم
&&
که تو هرگز گل من باشی و من خار تو باشم
\\
هرگز اندیشه نکردم که کمندت به من افتد
&&
که من آن وقع ندارم که گرفتار تو باشم
\\
هرگز اندر همه عالم نشناسم غم و شادی
&&
مگر آن وقت که شادی خور و غمخوار تو باشم
\\
گذر از دست رقیبان نتوان کرد به کویت
&&
مگر آن وقت که در سایه زنهار تو باشم
\\
گر خداوند تعالی به گناهیت بگیرد
&&
گو بیامرز که من حامل اوزار تو باشم
\\
مردمان عاشق گفتار من ای قبله خوبان
&&
چون نباشند که من عاشق دیدار تو باشم
\\
من چه شایسته آنم که تو را خوانم و دانم
&&
مگرم هم تو ببخشی که سزاوار تو باشم
\\
گر چه دانم که به وصلت نرسم بازنگردم
&&
تا در این راه بمیرم که طلبکار تو باشم
\\
نه در این عالم دنیا که در آن عالم عقبی
&&
همچنان بر سر آنم که وفادار تو باشم
\\
خاک بادا تن سعدی اگرش تو نپسندی
&&
که نشاید که تو فخر من و من عار تو باشم
\\
\end{longtable}
\end{center}
