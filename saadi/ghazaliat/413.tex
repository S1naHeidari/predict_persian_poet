\begin{center}
\section*{غزل ۴۱۳: آن نه رویست که من وصف جمالش دانم}
\label{sec:413}
\addcontentsline{toc}{section}{\nameref{sec:413}}
\begin{longtable}{l p{0.5cm} r}
آن نه روی است که من وصف جمالش دانم
&&
این حدیث از دگری پرس که من حیرانم
\\
همه بینند نه این صنع که من می‌بینم
&&
همه خوانند نه این نقش که من می‌خوانم
\\
آن عجب نیست که سرگشته بود طالب دوست
&&
عجب این است که من واصل و سرگردانم
\\
سرو در باغ نشانند و تو را بر سر و چشم
&&
گر اجازت دهی ای سرو روان بنشانم
\\
عشق من بر گل رخسار تو امروزی نیست
&&
دیر سال است که من بلبل این بستانم
\\
به سرت کز سر پیمان محبت نروم
&&
گر بفرمایی رفتن به سر پیکانم
\\
باش تا جان برود در طلب جانانم
&&
که به کاری به از این بازنیاید جانم
\\
هر نصیحت که کنی بشنوم ای یار عزیز
&&
صبرم از دوست مفرمای که من نتوانم
\\
عجب از طبع هوسناک منت می‌آید
&&
من خود از مردم بی طبع عجب می‌مانم
\\
گفته بودی که بود در همه عالم سعدی
&&
من به خود هیچ نیم هر چه تو گویی آنم
\\
گر به تشریف قبولم بنوازی ملکم
&&
ور به تازانه قهرم بزنی شیطانم
\\
\end{longtable}
\end{center}
