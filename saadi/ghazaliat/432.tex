\begin{center}
\section*{غزل ۴۳۲: ما دگر کس نگرفتیم به جای تو ندیم}
\label{sec:432}
\addcontentsline{toc}{section}{\nameref{sec:432}}
\begin{longtable}{l p{0.5cm} r}
ما دگر کس نگرفتیم به جای تو ندیم
&&
الله الله تو فراموش مکن عهد قدیم
\\
هر یک از دایره جمع به راهی رفتند
&&
ما بماندیم و خیال تو به یک جای مقیم
\\
باغبان گر نگشاید در درویش به باغ
&&
آخر از باغ بیاید بر درویش نسیم
\\
گر نسیم سحر از خلق تو بویی آرد
&&
جان فشانیم به سوغات نسیم تو نه سیم
\\
بوی محبوب که بر خاک احبا گذرد
&&
نه عجب دارم اگر زنده کند عظم رمیم
\\
ای به حسن تو صنم چشم فلک نادیده
&&
وی به مثل تو ولد مادر ایام عقیم
\\
حال درویش چنان است که خال تو سیاه
&&
جسم دل ریش چنان است که چشم تو سقیم
\\
چشم جادوی تو بی واسطه کحل کحیل
&&
طاق ابروی تو بی شائبه وسمه وسیم
\\
ای که دلداری اگر جان منت می‌باید
&&
چاره‌ای نیست در این مسأله الا تسلیم
\\
عشقبازی نه طریق حکما بود ولی
&&
چشم بیمار تو دل می‌برد از دست حکیم
\\
سعدیا عشق نیامیزد و عفت با هم
&&
چند پنهان کنی آواز دهل زیر گلیم
\\
\end{longtable}
\end{center}
