\begin{center}
\section*{غزل ۷: مشتاقی و صبوری از حد گذشت یارا}
\label{sec:007}
\addcontentsline{toc}{section}{\nameref{sec:007}}
\begin{longtable}{l p{0.5cm} r}
مشتاقی و صبوری از حد گذشت یارا
&&
گر تو شکیب داری طاقت نماند ما را
\\
باری به چشم احسان در حال ما نظر کن
&&
کز خوان پادشاهان راحت بود گدا را
\\
سلطان که خشم گیرد بر بندگان حضرت
&&
حکمش رسد ولیکن حدی بود جفا را
\\
من بی تو زندگانی خود را نمی‌پسندم
&&
کاسایشی نباشد بی دوستان بقا را
\\
چون تشنه جان سپردم آن گه چه سود دارد
&&
آب از دو چشم دادن بر خاک من گیا را
\\
حال نیازمندی در وصف می‌نیاید
&&
آن گه که بازگردی گوییم ماجرا را
\\
بازآ و جان شیرین از من ستان به خدمت
&&
دیگر چه برگ باشد درویش بی‌نوا را
\\
یا رب تو آشنا را مهلت ده و سلامت
&&
چندان که بازبیند دیدار آشنا را
\\
نه ملک پادشا را در چشم خوبرویان
&&
وقعیست ای برادر نه زهد پارسا را
\\
ای کاش برفتادی برقع ز روی لیلی
&&
تا مدعی نماندی مجنون مبتلا را
\\
سعدی قلم به سختی رفتست و نیکبختی
&&
پس هر چه پیشت آید گردن بنه قضا را
\\
\end{longtable}
\end{center}
