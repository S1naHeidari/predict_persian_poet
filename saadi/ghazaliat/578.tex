\begin{center}
\section*{غزل ۵۷۸: تو خود به صحبت امثال ما نپردازی}
\label{sec:578}
\addcontentsline{toc}{section}{\nameref{sec:578}}
\begin{longtable}{l p{0.5cm} r}
تو خود به صحبت امثال ما نپردازی
&&
نظر به حال پریشان ما نیندازی
\\
وصال ما و شما دیر متفق گردد
&&
که من اسیر نیازم تو صاحب نازی
\\
کجا به صید ملخ همتت فرو آید
&&
بدین صفت که تو باز بلندپروازی
\\
به راستی که نه همبازی تو بودم من
&&
تو شوخ دیده مگس بین که می‌کند بازی
\\
ز دست ترک ختایی کسی جفا چندان
&&
نمی‌برد که من از دست ترک شیرازی
\\
و گر هلاک منت درخور است باکی نیست
&&
قتیل عشق شهید است و قاتلش غازی
\\
کدام سنگدل است آن که عیب ما گوید
&&
گر آفتاب ببینی چو موم بگدازی
\\
میسرت نشود سر عشق پوشیدن
&&
که عاقبت بکند رنگ روی غمازی
\\
چه جرم رفت که با ما سخن نمی‌گویی
&&
چه دشمنیست که با دوستان نمی‌سازی
\\
من از فراق تو بیچاره سیل می‌رانم
&&
مثال ابر بهار و تو خیل می‌تازی
\\
هنوز با همه بدعهدیت دعاگویم
&&
که گر به قهر برانی به لطف بنوازی
\\
تو همچو صاحب دیوان مکن که سعدی را
&&
به یک ره از نظر خویشتن بیندازی
\\
\end{longtable}
\end{center}
