\begin{center}
\section*{غزل ۳۱۹: هر که بی دوست می‌برد خوابش}
\label{sec:319}
\addcontentsline{toc}{section}{\nameref{sec:319}}
\begin{longtable}{l p{0.5cm} r}
هر که بی دوست می‌برد خوابش
&&
همچنان صبر هست و پایابش
\\
خواب از آن چشم چشم نتوان داشت
&&
که ز سر برگذشت سیلابش
\\
نه به خود می‌رود گرفته عشق
&&
دیگری می‌برد به قلابش
\\
چه کند پای بند مهر کسی
&&
که نبیند جفای اصحابش
\\
هر که حاجت به درگهی دارد
&&
لازمست احتمال بوابش
\\
ناگزیرست تلخ و شیرینش
&&
خار و خرما و زهر و جلابش
\\
سایرست این مثل که مستسقی
&&
نکند رود دجله سیرابش
\\
شب هجران دوست ظلمانیست
&&
ور برآید هزار مهتابش
\\
برود جان مستمند از تن
&&
نرود مهر مهر احبابش
\\
سعدیا گوسفند قربانی
&&
به که نالد ز دست قصابش
\\
\end{longtable}
\end{center}
