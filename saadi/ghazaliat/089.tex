\begin{center}
\section*{غزل ۸۹: بتا هلاک شود دوست در محبت دوست}
\label{sec:089}
\addcontentsline{toc}{section}{\nameref{sec:089}}
\begin{longtable}{l p{0.5cm} r}
بتا هلاک شود دوست در محبت دوست
&&
که زندگانی او در هلاک بودن اوست
\\
مرا جفا و وفای تو پیش یک سان است
&&
که هر چه دوست پسندد به جای دوست نکوست
\\
مرا و عشق تو گیتی به یک شکم زاده‌ست
&&
دو روح در بدنی چون دو مغز در یک پوست
\\
هر آنچه بر سر آزادگان رود زیباست
&&
علی‌الخصوص که از دست یار زیباخوست
\\
دلم ز دست به در برد سروبالایی
&&
خلاف عادت آن سروها که بر لب جوست
\\
به خواب دوش چنان دیدمی که زلفینش
&&
گرفته بودم و دستم هنوز غالیه‌بوست
\\
چو گوی در همه عالم به جان بگردیدم
&&
ز دست عشقش و چوگان هنوز در پی گوست
\\
جماعتی به همین آب چشم بیرونی
&&
نظر کنند و ندانند کآتشم در توست
\\
ز دوست هر که تو بینی مراد خود خواهد
&&
مراد خاطر سعدی مراد خاطر اوست
\\
\end{longtable}
\end{center}
