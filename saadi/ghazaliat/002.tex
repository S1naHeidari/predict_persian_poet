\begin{center}
\section*{غزل ۲: ای نفس خرم باد صبا}
\label{sec:002}
\addcontentsline{toc}{section}{\nameref{sec:002}}
\begin{longtable}{l p{0.5cm} r}
ای نفس خرم باد صبا
&&
از بر یار آمده‌ای مرحبا
\\
قافله شب چه شنیدی ز صبح
&&
مرغ سلیمان چه خبر از سبا
\\
بر سر خشمست هنوز آن حریف
&&
یا سخنی می‌رود اندر رضا
\\
از در صلح آمده‌ای یا خلاف
&&
با قدم خوف روم یا رجا
\\
بار دگر گر به سر کوی دوست
&&
بگذری ای پیک نسیم صبا
\\
گو رمقی بیش نماند از ضعیف
&&
چند کند صورت بی‌جان بقا
\\
آن همه دلداری و پیمان و عهد
&&
نیک نکردی که نکردی وفا
\\
لیکن اگر دور وصالی بود
&&
صلح فراموش کند ماجرا
\\
تا به گریبان نرسد دست مرگ
&&
دست ز دامن نکنیمت رها
\\
دوست نباشد به حقیقت که او
&&
دوست فراموش کند در بلا
\\
خستگی اندر طلبت راحتست
&&
درد کشیدن به امید دوا
\\
سر نتوانم که برآرم چو چنگ
&&
ور چو دفم پوست بدرد قفا
\\
هر سحر از عشق دمی می‌زنم
&&
روز دگر می‌شنوم برملا
\\
قصه دردم همه عالم گرفت
&&
در که نگیرد نفس آشنا
\\
گر برسد ناله سعدی به کوه
&&
کوه بنالد به زبان صدا
\\
\end{longtable}
\end{center}
