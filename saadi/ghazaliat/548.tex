\begin{center}
\section*{غزل ۵۴۸: دانی چه گفت مرا آن بلبل سحری}
\label{sec:548}
\addcontentsline{toc}{section}{\nameref{sec:548}}
\begin{longtable}{l p{0.5cm} r}
دانی چه گفت مرا آن بلبل سحری
&&
تو خود چه آدمیی کز عشق بی‌خبری
\\
اشتر به شعر عرب در حالت است و طرب
&&
گر ذوق نیست تو را کژطبع جانوری
\\
من هرگز از تو نظر با خویشتن نکنم
&&
بیننده تن ندهد هرگز به بی بصری
\\
از بس که در نظرم خوب آمدی صنما
&&
هر جا که می‌نگرم گویی که در نظری
\\
دیگر نگه نکنم بالای سرو چمن
&&
دیگر صفت نکنم رفتار کبک دری
\\
کبک این چنین نرود سرو این چنین نچمد
&&
طاووس را نرسد پیش تو جلوه گری
\\
هر گه که می‌گذری من در تو می‌نگرم
&&
کز حسن قامت خود با کس نمی‌نگری
\\
از بس که فتنه شوم بر رفتنت نه عجب
&&
بر خویشتن تو ز ما صد بار فتنه‌تری
\\
باری به حکم کرم بر حال ما بنگر
&&
کافتد که بار دگر بر خاک ما گذری
\\
سعدی به جور و جفا مهر از تو برنکند
&&
من خاک پای توام ور خون من بخوری
\\
\end{longtable}
\end{center}
