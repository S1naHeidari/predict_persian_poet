\begin{center}
\section*{غزل ۳۳۲: هر که سودای تو دارد چه غم از هر که جهانش}
\label{sec:332}
\addcontentsline{toc}{section}{\nameref{sec:332}}
\begin{longtable}{l p{0.5cm} r}
هر که سودای تو دارد چه غم از هر که جهانش
&&
نگران تو چه اندیشه و بیم از دگرانش
\\
آن پی مهر تو گیرد که نگیرد پی خویشش
&&
وان سر وصل تو دارد که ندارد غم جانش
\\
هر که از یار تحمل نکند یار مگویش
&&
وان که در عشق ملامت نکشد مرد مخوانش
\\
چون دل از دست به درشد مثل کره توسن
&&
نتوان باز گرفتن به همه شهر عنانش
\\
به جفایی و قفایی نرود عاشق صادق
&&
مژه بر هم نزند گر بزنی تیر و سنانش
\\
خفته خاک لحد را که تو ناگه به سر آیی
&&
عجب ار باز نیاید به تن مرده روانش
\\
شرم دارد چمن از قامت زیبای بلندت
&&
که همه عمر نبوده‌ست چنین سرو روانش
\\
گفتم از ورطه عشقت به صبوری به درآیم
&&
باز می‌بینم و دریا نه پدید است کرانش
\\
عهد ما با تو نه عهدی که تغیر بپذیرد
&&
بوستانیست که هرگز نزند باد خزانش
\\
چه گنه کردم و دیدی که تعلق ببریدی
&&
بنده بی جرم و خطایی نه صواب است مرانش
\\
نرسد ناله سعدی به کسی در همه عالم
&&
که نه تصدیق کند کز سر دردیست فغانش
\\
گر فلاطون به حکیمی مرض عشق بپوشد
&&
عاقبت پرده برافتد ز سر راز نهانش
\\
\end{longtable}
\end{center}
