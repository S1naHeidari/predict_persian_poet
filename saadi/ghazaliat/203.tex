\begin{center}
\section*{غزل ۲۰۳: تو را نادیدن ما غم نباشد}
\label{sec:203}
\addcontentsline{toc}{section}{\nameref{sec:203}}
\begin{longtable}{l p{0.5cm} r}
تو را نادیدن ما غم نباشد
&&
که در خیلت به از ما کم نباشد
\\
من از دست تو در عالم نهم روی
&&
ولیکن چون تو در عالم نباشد
\\
عجب گر در چمن برپای خیزی
&&
که سرو راست پیشت خم نباشد
\\
مبادا در جهان دلتنگ رویی
&&
که رویت بیند و خرم نباشد
\\
من اول روز دانستم که این عهد
&&
که با من می‌کنی محکم نباشد
\\
که دانستم که هرگز سازگاری
&&
پری را با بنی آدم نباشد
\\
مکن یارا دلم مجروح مگذار
&&
که هیچم در جهان مرهم نباشد
\\
بیا تا جان شیرین در تو ریزم
&&
که بخل و دوستی با هم نباشد
\\
نخواهم بی تو یک دم زندگانی
&&
که طیب عیش بی همدم نباشد
\\
نظر گویند سعدی با که داری
&&
که غم با یار گفتن غم نباشد
\\
حدیث دوست با دشمن نگویم
&&
که هرگز مدعی محرم نباشد
\\
\end{longtable}
\end{center}
