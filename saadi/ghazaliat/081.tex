\begin{center}
\section*{غزل ۸۱: چه رویست آن که پیش کاروانست}
\label{sec:081}
\addcontentsline{toc}{section}{\nameref{sec:081}}
\begin{longtable}{l p{0.5cm} r}
چه روی است آن که پیش کاروان است
&&
مگر شمعی به دست ساروان است
\\
سلیمان است گویی در عماری
&&
که بر باد صبا تختش روان است
\\
جمال ماه پیکر بر بلندی
&&
بدان ماند که ماه آسمان است
\\
بهشتی صورتی در جوف محمل
&&
چو برجی کآفتابش در میان است
\\
خداوندان عقل این طرفه بینند
&&
که خورشیدی به زیر سایبان است
\\
چو نیلوفر در آب و مهر در میغ
&&
پری رخ در نقاب پرنیان است
\\
ز روی کار من برقع برانداخت
&&
به یک بار آن که در برقع نهان است
\\
شتر پیشی گرفت از من به رفتار
&&
که بر من بیش از او بار گران است
\\
زهی اندک وفای سست پیمان
&&
که آن سنگین دل نامهربان است
\\
تو را گر دوستی با ما همین بود
&&
وفای ما و عهد ما همان است
\\
بدار ای ساربان آخر زمانی
&&
که عهد وصل را آخرزمان است
\\
وفا کردیم و با ما غدر کردند
&&
برو سعدی که این پاداش آن است
\\
ندانستی که در پایان پیری
&&
نه وقت پنجه کردن با جوان است
\\
\end{longtable}
\end{center}
