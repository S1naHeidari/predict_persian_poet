\begin{center}
\section*{غزل ۵۴۶: جور بر من می‌پسندد دلبری}
\label{sec:546}
\addcontentsline{toc}{section}{\nameref{sec:546}}
\begin{longtable}{l p{0.5cm} r}
جور بر من می‌پسندد دلبری
&&
زور با من می‌کند زورآوری
\\
بار خصمی می‌کشم کز جور او
&&
می‌نشاید رفت پیش داوری
\\
عقل بیچاره‌ست در زندان عشق
&&
چون مسلمانی به دست کافری
\\
بارها گفتم بگریم پیش خلق
&&
تا مگر بر من ببخشد خاطری
\\
باز گویم پادشاهی را چه غم
&&
گر به خیلش در بمیرد چاکری
\\
ای که صبر از من طمع داری و هوش
&&
بار سنگین می‌نهی بر لاغری
\\
زآنچه در پای عزیزان افکنند
&&
ما سری داریم اگر داری سری
\\
چشم عادت کرده با دیدار دوست
&&
حیف باشد بعد از او بر دیگری
\\
در سراپای تو حیران مانده‌ام
&&
در نمی‌باید به حسنت زیوری
\\
این سخن سعدی تواند گفت و بس
&&
هر گدایی را نباشد جوهری
\\
\end{longtable}
\end{center}
