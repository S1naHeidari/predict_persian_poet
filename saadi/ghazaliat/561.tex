\begin{center}
\section*{غزل ۵۶۱: خوش بود یاری و یاری بر کنار سبزه زاری}
\label{sec:561}
\addcontentsline{toc}{section}{\nameref{sec:561}}
\begin{longtable}{l p{0.5cm} r}
خوش بود یاری و یاری بر کنار سبزه زاری
&&
مهربانان روی بر هم وز حسودان برکناری
\\
هر که را با دلستانی عیش می‌افتد زمانی
&&
گو غنیمت دان که دیگر دیر دیر افتد شکاری
\\
راحت جان است رفتن با دلارامی به صحرا
&&
عین درمان است گفتن درد دل با غمگساری
\\
هر که منظوری ندارد عمر ضایع می‌گذارد
&&
اختیار این است دریاب ای که داری اختیاری
\\
عیش در عالم نبودی گر نبودی روی زیبا
&&
گر نه گل بودی نخواندی بلبلی بر شاخساری
\\
بار بی اندازه دارم بر دل از سودای جانان
&&
آخر ای بی رحم باری از دلی برگیر باری
\\
دانی از بهر چه معنی خاک پایت می‌نباشم
&&
تا تو را ننشیند از من بر دل نازک غباری
\\
ور تو را با خاکساری سر به صحبت درنیاید
&&
بر سر راهت بیفتم تا کنی بر من گذاری
\\
زندگانی صرف کردن در طلب حیفی نباشد
&&
گر دری خواهد گشودن سهل باشد انتظاری
\\
دوستان معذور دارند از جوانمردی و رحمت
&&
گر بنالد دردمندی یا بگرید بی‌قراری
\\
رفتنش دل می‌رباید گفتنش جان می‌فزاید
&&
با چنین حسن و لطافت چون کند پرهیزگاری
\\
عمر سعدی گر سر آید در حدیث عشق شاید
&&
کاو نخواهد ماند بی شک وین بماند یادگاری
\\
\end{longtable}
\end{center}
