\begin{center}
\section*{غزل ۵۸۹: دل دیوانگیم هست و سر ناباکی}
\label{sec:589}
\addcontentsline{toc}{section}{\nameref{sec:589}}
\begin{longtable}{l p{0.5cm} r}
دل دیوانگیم هست و سر ناباکی
&&
که نه کاریست شکیبایی و اندهناکی
\\
سر به خمخانه تشنیع فرو خواهم برد
&&
خرقه گو در بر من دست بشوی از پاکی
\\
دست در دل کن و هر پرده پندار که هست
&&
بدر ای سینه که از دست ملامت چاکی
\\
تا به نخجیر دل سوختگان کردی میل
&&
هر زمان بسته دلی سوخته بر فتراکی
\\
انت ریان و کم حولک قلب صاد
&&
انت فرحان و کم نحوک طرف باکی
\\
یا رب آن آب حیات است بدان شیرینی
&&
یا رب آن سرو روان است بدان چالاکی
\\
جامه‌ای پهن‌تر از کارگه امکانی
&&
لقمه‌ای بیشتر از حوصله ادراکی
\\
در شکنج سر زلف تو دریغا دل من
&&
که گرفتار دو مار است بدین ضحاکی
\\
آه من باد به گوش تو رساند هرگز
&&
که نه ما بر سر خاکیم و تو بر افلاکی
\\
الغیاث از تو که هم دردی و هم درمانی
&&
زینهار از تو که هم زهری و هم تریاکی
\\
سعدیا آتش سودای تو را آبی بس
&&
باد بی فایده مفروش که مشتی خاکی
\\
\end{longtable}
\end{center}
