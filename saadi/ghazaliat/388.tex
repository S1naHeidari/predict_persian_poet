\begin{center}
\section*{غزل ۳۸۸: منم این بی تو که پروای تماشا دارم}
\label{sec:388}
\addcontentsline{toc}{section}{\nameref{sec:388}}
\begin{longtable}{l p{0.5cm} r}
منم این بی تو که پروای تماشا دارم
&&
کافرم گر دل باغ و سر صحرا دارم
\\
بر گلستان گذرم بی تو و شرمم ناید
&&
در ریاحین نگرم بی تو و یارا دارم
\\
که نه بر ناله مرغان چمن شیفته‌ام
&&
که نه سودای رخ لاله حمرا دارم
\\
بر گل روی تو چون بلبل مستم واله
&&
به رخ لاله و نسرین چه تمنا دارم
\\
گر چه لایق نبود دست من و دامن تو
&&
هر کجا پای نهی فرق سر آن جا دارم
\\
گر به مسجد روم ابروی تو محراب من است
&&
ور به آتشکده زلف تو چلیپا دارم
\\
دلم از پختن سودای وصال تو بسوخت
&&
تو من خام طمع بین که چه سودا دارم
\\
عقل مسکین به چه اندیشه فرا دست کنم
&&
دل شیدا به چه تدبیر شکیبا دارم
\\
سر من دار که چشم از همگان در دوزم
&&
دست من گیر که دست از دو جهان وادارم
\\
با توام یک نفس از هشت بهشت اولیتر
&&
من که امروز چنینم غم فردا دارم
\\
سعدی خویشتنم خوان که به معنی ز توام
&&
که به صورت نسب از آدم و حوا دارم
\\
\end{longtable}
\end{center}
