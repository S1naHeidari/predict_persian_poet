\begin{center}
\section*{غزل ۳۹۴: به خدا اگر بمیرم که دل از تو برنگیرم}
\label{sec:394}
\addcontentsline{toc}{section}{\nameref{sec:394}}
\begin{longtable}{l p{0.5cm} r}
به خدا اگر بمیرم که دل از تو برنگیرم
&&
برو ای طبیبم از سر که دوا نمی‌پذیرم
\\
همه عمر با حریفان بنشستمی و خوبان
&&
تو بخاستی و نقشت بنشست در ضمیرم
\\
مده ای حکیم پندم که به کار در نبندم
&&
که ز خویشتن گزیر است و ز دوست ناگزیرم
\\
برو ای سپر ز پیشم که به جان رسید پیکان
&&
بگذار تا ببینم که که می‌زند به تیرم
\\
نه نشاط دوستانم نه فراغ بوستانم
&&
بروید ای رفیقان به سفر که من اسیرم
\\
تو در آب اگر ببینی حرکات خویشتن را
&&
به زبان خود بگویی که به حسن بی‌نظیرم
\\
تو به خواب خوش بیاسای و به عیش و کامرانی
&&
که نه من غنوده‌ام دوش و نه مردم از نفیرم
\\
نه توانگران ببخشند فقیر ناتوان را
&&
نظری کن ای توانگر که به دیدنت فقیرم
\\
اگرم چو عود سوزی تن من فدای جانت
&&
که خوش است عیش مردم به روایح عبیرم
\\
نه تو گفته‌ای که سعدی نبرد ز دست من جان
&&
نه به خاک پای مردان چو تو می‌کشی نمیرم
\\
\end{longtable}
\end{center}
