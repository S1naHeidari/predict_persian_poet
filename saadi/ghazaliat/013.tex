\begin{center}
\section*{غزل ۱۳: وه که گر من بازبینم روی یار خویش را}
\label{sec:013}
\addcontentsline{toc}{section}{\nameref{sec:013}}
\begin{longtable}{l p{0.5cm} r}
وه که گر من بازبینم روی یار خویش را
&&
تا قیامت شکر گویم کردگار خویش را
\\
یار بارافتاده را در کاروان بگذاشتند
&&
بی‌وفا یاران که بربستند بار خویش را
\\
مردم بیگانه را خاطر نگه دارند خلق
&&
دوستان ما بیازردند یار خویش را
\\
همچنان امید می‌دارم که بعد از داغ هجر
&&
مرهمی بر دل نهد امیدوار خویش را
\\
رای رای توست خواهی جنگ و خواهی آشتی
&&
ما قلم در سر کشیدیم اختیار خویش را
\\
هر که را در خاک غربت پای در گل ماند ماند
&&
گو دگر در خواب خوش بینی دیار خویش را
\\
عافیت خواهی نظر در منظر خوبان مکن
&&
ور کنی بدرود کن خواب و قرار خویش را
\\
گبر و ترسا و مسلمان هر کسی در دین خویش
&&
قبله‌ای دارند و ما زیبا نگار خویش را
\\
خاک پایش خواستم شد بازگفتم زینهار
&&
من بر آن دامن نمی‌خواهم غبار خویش را
\\
دوش حورازاده‌ای دیدم که پنهان از رقیب
&&
در میان یاوران می‌گفت یار خویش را
\\
گر مراد خویش خواهی ترک وصل ما بگوی
&&
ور مرا خواهی رها کن اختیار خویش را
\\
درد دل پوشیده مانی تا جگر پرخون شود
&&
به که با دشمن نمایی حال زار خویش را
\\
گر هزارت غم بود با کس نگویی زینهار
&&
ای برادر تا نبینی غمگسار خویش را
\\
ای سهی سرو روان آخر نگاهی باز کن
&&
تا به خدمت عرضه دارم افتقار خویش را
\\
دوستان گویند سعدی دل چرا دادی به عشق
&&
تا میان خلق کم کردی وقار خویش را
\\
ما صلاح خویشتن در بی‌نوایی دیده‌ایم
&&
هر کسی گو مصلحت بینند کار خویش را
\\
\end{longtable}
\end{center}
