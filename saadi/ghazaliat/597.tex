\begin{center}
\section*{غزل ۵۹۷: بسیار سفر باید تا پخته شود خامی}
\label{sec:597}
\addcontentsline{toc}{section}{\nameref{sec:597}}
\begin{longtable}{l p{0.5cm} r}
بسیار سفر باید تا پخته شود خامی
&&
صوفی نشود صافی تا درنکشد جامی
\\
گر پیر مناجات است ور رند خراباتی
&&
هر کس قلمی رفته‌ست بر وی به سرانجامی
\\
فردا که خلایق را دیوان جزا باشد
&&
هر کس عملی دارد من گوش به انعامی
\\
ای بلبل اگر نالی من با تو هم آوازم
&&
تو عشق گلی داری من عشق گل اندامی
\\
سروی به لب جویی گویند چه خوش باشد
&&
آنان که ندیدستند سروی به لب بامی
\\
روزی تن من بینی قربان سر کویش
&&
وین عید نمی‌باشد الا به هر ایامی
\\
ای در دل ریش من مهرت چو روان در تن
&&
آخر ز دعاگویی یاد آر به دشنامی
\\
باشد که تو خود روزی از ما خبری پرسی
&&
ور نه که برد هیهات از ما به تو پیغامی
\\
گر چه شب مشتاقان تاریک بود اما
&&
نومید نباید بود از روشنی بامی
\\
سعدی به لب دریا دردانه کجا یابی
&&
در کام نهنگان رو گر می‌طلبی کامی
\\
\end{longtable}
\end{center}
