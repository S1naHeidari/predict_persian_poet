\begin{center}
\section*{غزل ۵۲۸: یاد می‌داری که با من جنگ در سر داشتی}
\label{sec:528}
\addcontentsline{toc}{section}{\nameref{sec:528}}
\begin{longtable}{l p{0.5cm} r}
یاد می‌داری که با من جنگ در سر داشتی
&&
رای رای توست خواهی جنگ خواهی آشتی
\\
نیک بد کردی شکستن عهد یار مهربان
&&
این بتر کردی که بد کردی و نیک انگاشتی
\\
دوستان دشمن گرفتن هرگزت عادت نبود
&&
جز در این نوبت که دشمن دوست می‌پنداشتی
\\
خاطرم نگذاشت یک ساعت که بدمهری کنم
&&
گر چه دانستم که پاک از خاطرم بگذاشتی
\\
همچنانت ناخن رنگین گواهی می‌دهد
&&
بر سرانگشتان که در خون عزیزان داشتی
\\
تا تو برگشتی نیامد هیچ خلق اندر نظر
&&
کز خیالت شحنه‌ای بر ناظرم بگماشتی
\\
هر چه خواهی کن که ما را با تو روی جنگ نیست
&&
سر نهادن به در آن موضع که تیغ افراشتی
\\
هر دم از شاخ زبانم میوه‌ای تر می‌رسد
&&
بوستان‌ها رست از آن تخمم که در دل کاشتی
\\
سعدی از عقبی و دنیا روی در دیوار کرد
&&
تا تو در دیوار فکرش نقش خود بنگاشتی
\\
\end{longtable}
\end{center}
