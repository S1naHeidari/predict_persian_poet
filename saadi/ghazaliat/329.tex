\begin{center}
\section*{غزل ۳۲۹: خوشست درد که باشد امید درمانش}
\label{sec:329}
\addcontentsline{toc}{section}{\nameref{sec:329}}
\begin{longtable}{l p{0.5cm} r}
خوش است درد که باشد امید درمانش
&&
دراز نیست بیابان که هست پایانش
\\
نه شرط عشق بود با کمان ابروی دوست
&&
که جان سپر نکنی پیش تیربارانش
\\
عدیم را که تمنای بوستان باشد
&&
ضرورت است تحمل ز بوستانبانش
\\
وصال جان جهان یافتن حرامش باد
&&
که التفات بود بر جهان و بر جانش
\\
ز کعبه روی نشاید به ناامیدی تافت
&&
کمینه آن که بمیریم در بیابانش
\\
اگر چه ناقص و نادانم این قدر دانم
&&
که آبگینه من نیست مرد سندانش
\\
ولیک با همه عیب احتمال یار عزیز
&&
کنند چون نکنند احتمال هجرانش
\\
گر آید از تو به رویم هزار تیر جفا
&&
جفاست گر مژه بر هم زنم ز پیکانش
\\
حریف را که غم جان خویشتن باشد
&&
هنوز لاف دروغ است عشق جانانش
\\
حکیم را که دل از دست رفت و پای از جای
&&
سر صلاح توقع مدار و سامانش
\\
گلی چو روی تو گر ممکن است در آفاق
&&
نه ممکن است چو سعدی هزاردستانش
\\
\end{longtable}
\end{center}
