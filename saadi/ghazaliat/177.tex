\begin{center}
\section*{غزل ۱۷۷: هر گه که بر من آن بت عیار بگذرد}
\label{sec:177}
\addcontentsline{toc}{section}{\nameref{sec:177}}
\begin{longtable}{l p{0.5cm} r}
هر گه که بر من آن بت عیار بگذرد
&&
صد کاروان عالم اسرار بگذرد
\\
مست شراب و خواب و جوانی و شاهدی
&&
هر لحظه پیش مردم هشیار بگذرد
\\
هر گه که بگذرد بکشد دوستان خویش
&&
وین دوست منتظر که دگربار بگذرد
\\
گفتم به گوشه‌ای بنشینم چو عاقلان
&&
دیوانه‌ام کند چو پری وار بگذرد
\\
گفتم دری ز خلق ببندم به روی خویش
&&
دردیست در دلم که ز دیوار بگذرد
\\
بازار حسن جمله خوبان شکسته‌ای
&&
ره نیست کز تو هیچ خریدار بگذرد
\\
غایب مشو که عمر گران مایه ضایعست
&&
الا دمی که در نظر یار بگذرد
\\
آسایشست رنج کشیدن به بوی آنک
&&
روزی طبیب بر سر بیمار بگذرد
\\
ترسم که مست و عاشق و بی‌دل شود چو ما
&&
گر محتسب به خانه خمار بگذرد
\\
سعدی به خویشتن نتوان رفت سوی دوست
&&
کان جا طریق نیست که اغیار بگذرد
\\
\end{longtable}
\end{center}
