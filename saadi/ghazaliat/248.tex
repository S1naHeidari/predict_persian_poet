\begin{center}
\section*{غزل ۲۴۸: شوخی مکن ای یار که صاحب نظرانند}
\label{sec:248}
\addcontentsline{toc}{section}{\nameref{sec:248}}
\begin{longtable}{l p{0.5cm} r}
شوخی مکن ای یار که صاحب نظرانند
&&
بیگانه و خویش از پس و پیشت نگرانند
\\
کس نیست که پنهان نظری با تو ندارد
&&
من نیز بر آنم که همه خلق بر آنند
\\
اهل نظرانند که چشمی به ارادت
&&
با روی تو دارند و دگر بی بصرانند
\\
هر کس غم دین دارد و هر کس غم دنیا
&&
بعد از غم رویت غم بیهوده خورانند
\\
ساقی بده آن کوزهٔ خمخانه به درویش
&&
کانها که بمردند گل کوزه گرانند
\\
چشمی که جمال تو ندیده‌ست چه دیده‌ست؟
&&
افسوس بر اینان که به غفلت گذرانند
\\
تا رای کجا داری و پروای که داری؟
&&
کز هر طرفت طایفه‌ای منتظرانند
\\
اینان که به دیدار تو در رقص می‌آیند
&&
چون می‌روی اندر طلبت جامه درانند
\\
سعدی به جفا ترک محبت نتوان گفت
&&
بر در بنشینم اگر از خانه برانند
\\
\end{longtable}
\end{center}
