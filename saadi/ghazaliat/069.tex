\begin{center}
\section*{غزل ۶۹: عشرت خوشست و بر طرف جوی خوشترست}
\label{sec:069}
\addcontentsline{toc}{section}{\nameref{sec:069}}
\begin{longtable}{l p{0.5cm} r}
عشرت خوش است و بر طرف جوی خوشتر است
&&
می بر سماع بلبل خوشگوی خوشتر است
\\
عیش است بر کنار سمن زار خواب صبح؟
&&
نی، در کنار یار سمن بوی خوشتر است
\\
خواب از خمار بادهٔ نوشین بامداد
&&
بر بستر شقایق خودروی خوشتر است
\\
روی از جمال دوست به صحرا مکن که روی
&&
در روی همنشین وفاجوی خوشتر است
\\
آواز چنگ مطرب خوشگوی گو مباش
&&
ما را حدیث همدم خوشخوی خوشتر است
\\
گر شاهد است سبزه بر اطراف گلستان
&&
بر عارضین شاهد گلروی خوشتر است
\\
آب از نسیم باد زره روی گشته گیر
&&
مفتول زلف یار زره موی خوشتر است
\\
گو چشمه آب کوثر و بستان بهشت باش
&&
ما را مقام بر سر این کوی خوشتر است
\\
سعدی! جفا نبرده چه دانی تو قدر یار؟
&&
تحصیل کام دل به تکاپوی خوشتر است
\\
\end{longtable}
\end{center}
