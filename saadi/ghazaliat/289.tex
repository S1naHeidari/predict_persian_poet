\begin{center}
\section*{غزل ۲۸۹: آن نه عشقست که از دل به دهان می‌آید}
\label{sec:289}
\addcontentsline{toc}{section}{\nameref{sec:289}}
\begin{longtable}{l p{0.5cm} r}
آن نه عشق است که از دل به دهان می‌آید
&&
وان نه عاشق که ز معشوق به جان می‌آید
\\
گو برو در پس زانوی سلامت بنشین
&&
آن که از دست ملامت به فغان می‌آید
\\
کشتی هر که در این ورطه خونخوار افتاد
&&
نشنیدیم که دیگر به کران می‌آید
\\
یا مسافر که در این بادیه سرگردان شد
&&
دیگر از وی خبر و نام و نشان می‌آید
\\
چشم رغبت که به دیدار کسی کردی باز
&&
باز بر هم منه ار تیر و سنان می‌آید
\\
عاشق آن است که بی خویشتن از ذوق سماع
&&
پیش شمشیر بلا رقص‌کنان می‌آید
\\
حاش لله که من از تیر بگردانم روی
&&
گر بدانم که از آن دست و کمان می‌آید
\\
کشته بینند و مقاتل نشناسند که کیست
&&
کاین خدنگ از نظر خلق نهان می‌آید
\\
اندرون با تو چنان انس گرفته‌ست مرا
&&
که ملالم ز همه خلق جهان می‌آید
\\
شرط عشق است که از دوست شکایت نکنند
&&
لیکن از شوق حکایت به زبان می‌آید
\\
سعدیا این همه فریاد تو بی دردی نیست
&&
آتشی هست که دود از سر آن می‌آید
\\
\end{longtable}
\end{center}
