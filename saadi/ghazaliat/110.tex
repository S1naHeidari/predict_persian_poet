\begin{center}
\section*{غزل ۱۱۰: هر چه در روی تو گویند به زیبایی هست}
\label{sec:110}
\addcontentsline{toc}{section}{\nameref{sec:110}}
\begin{longtable}{l p{0.5cm} r}
هر چه در روی تو گویند به زیبایی هست
&&
وان چه در چشم تو از شوخی و رعنایی هست
\\
سروها دیدم در باغ و تأمل کردم
&&
قامتی نیست که چون تو به دلارایی هست
\\
ای که مانند تو بلبل به سخندانی نیست
&&
نتوان گفت که طوطی به شکرخایی هست
\\
نه تو را از من مسکین نه گل خندان را
&&
خبر از مشغله بلبل سودایی هست
\\
راست گفتی که فرج یابی اگر صبر کنی
&&
صبر نیکست کسی را که توانایی هست
\\
هرگز از دوست شنیدی که کسی بشکیبد
&&
دوستی نیست در آن دل که شکیبایی هست
\\
خبر از عشق نبودست و نباشد همه عمر
&&
هر که او را خبر از شنعت و رسوایی هست
\\
آن نه تنهاست که با یاد تو انسی دارد
&&
تا نگویی که مرا طاقت تنهایی هست
\\
همه را دیده به رویت نگرانست ولیک
&&
همه کس را نتوان گفت که بینایی هست
\\
گفته بودی همه زرقند و فریبند و فسوس
&&
سعدی آن نیست ولیکن چو تو فرمایی هست
\\
\end{longtable}
\end{center}
