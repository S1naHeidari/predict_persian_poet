\begin{center}
\section*{غزل ۲۴۶: دلبرا پیش وجودت همه خوبان عدمند}
\label{sec:246}
\addcontentsline{toc}{section}{\nameref{sec:246}}
\begin{longtable}{l p{0.5cm} r}
دلبرا پیش وجودت همه خوبان عدمند
&&
سروران بر در سودای تو خاک قدمند
\\
شهری اندر هوست سوخته در آتش عشق
&&
خلقی اندر طلبت غرقه دریای غمند
\\
خون صاحب نظران ریختی ای کعبه حسن
&&
قتل اینان که روا داشت که صید حرمند
\\
صنم اندر بلد کفر پرستند و صلیب
&&
زلف و روی تو در اسلام صلیب و صنمند
\\
گاه گاهی بگذر در صف دلسوختگان
&&
تا ثناییت بگویند و دعایی بدمند
\\
هر خم از جعد پریشان تو زندان دلیست
&&
تا نگویی که اسیران کمند تو کمند
\\
حرف‌های خط موزون تو پیرامن روی
&&
گویی از مشک سیه بر گل سوری رقمند
\\
در چمن سرو ستادست و صنوبر خاموش
&&
که اگر قامت زیبا ننمایی بچمند
\\
زین امیران ملاحت که تو بینی بر کس
&&
به شکایت نتوان رفت که خصم و حکمند
\\
بندگان را نه گزیرست ز حکمت نه گریز
&&
چه کنند ار بکشی ور بنوازی خدمند
\\
جور دشمن چه کند گر نکشد طالب دوست
&&
گنج و مار و گل و خار و غم و شادی به همند
\\
غم دل با تو نگویم که تو در راحت نفس
&&
نشناسی که جگرسوختگان در المند
\\
تو سبکبار قوی حال کجا دریابی
&&
که ضعیفان غمت بارکشان ستمند
\\
سعدیا عاشق صادق ز بلا نگریزد
&&
سست عهدان ارادت ز ملامت برمند
\\
\end{longtable}
\end{center}
