\begin{center}
\section*{غزل ۴۹: خرم آن بقعه که آرامگه یار آن جاست}
\label{sec:049}
\addcontentsline{toc}{section}{\nameref{sec:049}}
\begin{longtable}{l p{0.5cm} r}
خرم آن بقعه که آرامگه یار آنجاست
&&
راحت جان و شفای دل بیمار آنجاست
\\
من در این جای همین صورت بی‌جانم و بس
&&
دلم آنجاست که آن دلبر عیار آنجاست
\\
تنم اینجاست سقیم و دلم آنجاست مقیم
&&
فلک اینجاست ولی کوکب سیار آنجاست
\\
آخر ای باد صبا بویی اگر می‌آری
&&
سوی شیراز گذر کن که مرا یار آنجاست
\\
درد دل پیش که گویم غم دل با که خورم
&&
روم آنجا که مرا محرم اسرار آنجاست
\\
نکند میل دل من به تماشای چمن
&&
که تماشای دل آنجاست که دلدار آنجاست
\\
سعدی این منزل ویران چه کنی جای تو نیست
&&
رخت بربند که منزلگه احرار آنجاست
\\
\end{longtable}
\end{center}
