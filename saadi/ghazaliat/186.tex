\begin{center}
\section*{غزل ۱۸۶: بگذشت و باز آتش در خرمن سکون زد}
\label{sec:186}
\addcontentsline{toc}{section}{\nameref{sec:186}}
\begin{longtable}{l p{0.5cm} r}
بگذشت و بازم آتش در خرمن سکون زد
&&
دریای آتشینم در دیده موج خون زد
\\
خود کرده بود غارت عشقش حوالی دل
&&
بازم به یک شبیخون بر ملک اندرون زد
\\
دیدار دلفروزش در پایم ارغوان ریخت
&&
گفتار جان فزایش در گوشم ارغنون زد
\\
دیوانگان خود را می‌بست در سلاسل
&&
هر جا که عاقلی بود اینجا دم از جنون زد
\\
یا رب دلی که در وی پروای خود نگنجد
&&
دست محبت آنجا خرگاه عشق چون زد
\\
غلغل فکند روحم در گلشن ملایک
&&
هر گه که سنگ آهی بر طاق آبگون زد
\\
سعدی ز خود برون شو گر مرد راه عشقی
&&
کان کس رسید در وی کز خود قدم برون زد
\\
\end{longtable}
\end{center}
