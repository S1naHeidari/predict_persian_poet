\begin{center}
\section*{غزل ۴۳۹: ما گدایان خیل سلطانیم}
\label{sec:439}
\addcontentsline{toc}{section}{\nameref{sec:439}}
\begin{longtable}{l p{0.5cm} r}
ما گدایان خیل سلطانیم
&&
شهربند هوای جانانیم
\\
بنده را نام خویشتن نبود
&&
هر چه ما را لقب دهند آنیم
\\
گر برانند و گر ببخشایند
&&
ره به جای دگر نمی‌دانیم
\\
چون دلارام می‌زند شمشیر
&&
سر ببازیم و رخ نگردانیم
\\
دوستان در هوای صحبت یار
&&
زر فشانند و ما سر افشانیم
\\
مر خداوند عقل و دانش را
&&
عیب ما گو مکن که نادانیم
\\
هر گلی نو که در جهان آید
&&
ما به عشقش هزاردستانیم
\\
تنگ چشمان نظر به میوه کنند
&&
ما تماشاکنان بستانیم
\\
تو به سیمای شخص می‌نگری
&&
ما در آثار صنع حیرانیم
\\
هر چه گفتیم جز حکایت دوست
&&
در همه عمر از آن پشیمانیم
\\
سعدیا بی وجود صحبت یار
&&
همه عالم به هیچ نستانیم
\\
ترک جان عزیز بتوان گفت
&&
ترک یار عزیز نتوانیم
\\
\end{longtable}
\end{center}
