\begin{center}
\section*{غزل ۳۲۴: هر که نامهربان بود یارش}
\label{sec:324}
\addcontentsline{toc}{section}{\nameref{sec:324}}
\begin{longtable}{l p{0.5cm} r}
هر که نامهربان بود یارش
&&
واجب است احتمال آزارش
\\
طاقت رفتنم نمی‌ماند
&&
چون نظر می‌کنم به رفتارش
\\
وز سخن گفتنش چنان مستم
&&
که ندانم جواب گفتارش
\\
کشته تیر عشق زنده کند
&&
گر به سر بگذرد دگربارش
\\
هر چه زان تلخ‌تر بخواهد گفت
&&
گو بگو از لب شکربارش
\\
عشق پوشیده بود و صبر نماند
&&
پرده برداشتم ز اسرارش
\\
وه که گر من به خدمتش برسم
&&
خود چه خدمت کنم به مقدارش
\\
بیم دیوانگیست مردم را
&&
ز آمدن رفتن پری وارش
\\
کاش بیرون نیامدی سلطان
&&
تا ندیدی گدای بازارش
\\
سعدیا روی دوست نادیدن
&&
به که دیدن میان اغیارش
\\
\end{longtable}
\end{center}
