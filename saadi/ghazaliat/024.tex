\begin{center}
\section*{غزل ۲۴: وقتی دل سودایی می‌رفت به بستان‌ها}
\label{sec:024}
\addcontentsline{toc}{section}{\nameref{sec:024}}
\begin{longtable}{l p{0.5cm} r}
وقتی دل سودایی می‌رفت به بستان‌ها
&&
بی خویشتنم کردی بوی گل و ریحان‌ها
\\
گه نعره زدی بلبل گه جامه دریدی گل
&&
با یاد تو افتادم از یاد برفت آن‌ها
\\
ای مهر تو در دل‌ها وی مهر تو بر لب‌ها
&&
وی شور تو در سرها وی سر تو در جان‌ها
\\
تا عهد تو دربستم عهد همه بشکستم
&&
بعد از تو روا باشد نقض همه پیمان‌ها
\\
تا خار غم عشقت آویخته در دامن
&&
کوته نظری باشد رفتن به گلستان‌ها
\\
آن را که چنین دردی از پای دراندازد
&&
باید که فروشوید دست از همه درمان‌ها
\\
گر در طلبت رنجی ما را برسد شاید
&&
چون عشق حرم باشد سهل است بیابان‌ها
\\
هر تیر که در کیش است گر بر دل ریش آید
&&
ما نیز یکی باشیم از جمله قربان‌ها
\\
هر کاو نظری دارد با یار کمان ابرو
&&
باید که سپر باشد پیش همه پیکان‌ها
\\
گویند مگو سعدی چندین سخن از عشقش
&&
می‌گویم و بعد از من گویند به دوران‌ها
\\
\end{longtable}
\end{center}
