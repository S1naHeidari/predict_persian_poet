\begin{center}
\section*{غزل ۱۱۳: مرا از آن چه که بیرون شهر صحراییست}
\label{sec:113}
\addcontentsline{toc}{section}{\nameref{sec:113}}
\begin{longtable}{l p{0.5cm} r}
مرا از آن چه که بیرون شهر صحراییست
&&
قرین دوست به هر جا که هست خوش جاییست
\\
کسی که روی تو دیدست از او عجب دارم
&&
که باز در همه عمرش سر تماشاییست
\\
امید وصل مدار و خیال دوست مبند
&&
گرت به خویشتن از ذکر دوست پرواییست
\\
چو بر ولایت دل دست یافت لشکر عشق
&&
به دست باش که هر بامداد یغماییست
\\
به بوی زلف تو با باد عیش‌ها دارم
&&
اگر چه عیب کنندم که بادپیماییست
\\
فراغ صحبت دیوانگان کجا باشد
&&
تو را که هر خم مویی کمند داناییست
\\
ز دست عشق تو هر جا که می‌روم دستی
&&
نهاده بر سر و خاری شکسته در پاییست
\\
هزار سرو به معنی به قامتت نرسد
&&
و گر چه سرو به صورت بلندبالاییست
\\
تو را که گفت که حلوا دهم به دست رقیب
&&
به دست خویشتنم زهر ده که حلواییست
\\
نه خاص در سر من عشق در جهان آمد
&&
که هر سری که تو بینی رهین سوداییست
\\
تو را ملامت سعدی حلال کی باشد
&&
که بر کناری و او در میان دریاییست
\\
\end{longtable}
\end{center}
