\begin{center}
\section*{غزل ۴۶۴: دست با سرو روان چون نرسد در گردن}
\label{sec:464}
\addcontentsline{toc}{section}{\nameref{sec:464}}
\begin{longtable}{l p{0.5cm} r}
دست با سرو روان چون نرسد در گردن
&&
چاره‌ای نیست به جز دیدن و حسرت خوردن
\\
آدمی را که طلب هست و توانایی نیست
&&
صبر اگر هست و گر نیست بباید کردن
\\
بند بر پای توقف چه کند گر نکند
&&
شرط عشق است بلا دیدن و پای افشردن
\\
روی در خاک در دوست بباید مالید
&&
چون میسر نشود روی به روی آوردن
\\
نیم جانی چه بود تا ندهد دوست به دوست
&&
که به صد جان دل جانان نتوان آزردن
\\
سهل باشد سخن سخت که خوبان گویند
&&
جور شیرین دهنان تلخ نباشد بردن
\\
هیچ شک می‌نکنم کآهوی مشکین تتار
&&
شرم دارد ز تو مشکین خط آهو گردن
\\
روزی اندر سر کار تو کنم جان عزیز
&&
پیش بالای تو باری چو بباید مردن
\\
سعدیا دیده نگه داشتن از صورت خوب
&&
نه چنان است که دل دادن و جان پروردن
\\
\end{longtable}
\end{center}
