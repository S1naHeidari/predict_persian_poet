\begin{center}
\section*{غزل ۲۴۵: هر که بی او زندگانی می‌کند}
\label{sec:245}
\addcontentsline{toc}{section}{\nameref{sec:245}}
\begin{longtable}{l p{0.5cm} r}
هر که بی او زندگانی می‌کند
&&
گر نمی‌میرد گرانی می‌کند
\\
من بر آن بودم که ندهم دل به عشق
&&
سروبالا دلستانی می‌کند
\\
مهربانی می‌نمایم بر قدش
&&
سنگدل نامهربانی می‌کند
\\
برف پیری می‌نشیند بر سرم
&&
همچنان طبعم جوانی می‌کند
\\
ماجرای دل نمی‌گفتم به خلق
&&
آب چشمم ترجمانی می‌کند
\\
آهن افسرده می‌کوبد که جهد
&&
با قضای آسمانی می‌کند
\\
عقل را با عشق زور پنجه نیست
&&
احتمال از ناتوانی می‌کند
\\
چشم سعدی در امید روی یار
&&
چون دهانش درفشانی می‌کند
\\
هم بود شوری در این سر بی خلاف
&&
کاین همه شیرین زبانی می‌کند
\\
\end{longtable}
\end{center}
