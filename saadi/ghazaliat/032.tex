\begin{center}
\section*{غزل ۳۲: معلمت همه شوخی و دلبری آموخت}
\label{sec:032}
\addcontentsline{toc}{section}{\nameref{sec:032}}
\begin{longtable}{l p{0.5cm} r}
معلمت همه شوخی و دلبری آموخت
&&
جفا و ناز و عتاب و ستمگری آموخت
\\
غلام آن لب ضحاک و چشم فتانم
&&
که کید سحر به ضحاک و سامری آموخت
\\
تو بت چرا به معلم روی که بتگر چین
&&
به چین زلف تو آید به بتگری آموخت
\\
هزار بلبل دستان سرای عاشق را
&&
بباید از تو سخن گفتن دری آموخت
\\
برفت رونق بازار آفتاب و قمر
&&
از آن که ره به دکان تو مشتری آموخت
\\
همه قبیله من عالمان دین بودند
&&
مرا معلم عشق تو شاعری آموخت
\\
مرا به شاعری آموخت روزگار آن گه
&&
که چشم مست تو دیدم که ساحری آموخت
\\
مگر دهان تو آموخت تنگی از دل من
&&
وجود من ز میان تو لاغری آموخت
\\
بلای عشق تو بنیاد زهد و بیخ ورع
&&
چنان بکند که صوفی قلندری آموخت
\\
دگر نه عزم سیاحت کند نه یاد وطن
&&
کسی که بر سر کویت مجاوری آموخت
\\
من آدمی به چنین شکل و قد و خوی و روش
&&
ندیده‌ام مگر این شیوه از پری آموخت
\\
به خون خلق فروبرده پنجه کاین حناست
&&
ندانمش که به قتل که شاطری آموخت
\\
چنین بگریم از این پس که مرد بتواند
&&
در آب دیده سعدی شناوری آموخت
\\
\end{longtable}
\end{center}
