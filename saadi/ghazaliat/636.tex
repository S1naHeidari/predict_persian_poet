\begin{center}
\section*{غزل ۶۳۶: نشنیده‌ام که ماهی بر سر نهد کلاهی}
\label{sec:636}
\addcontentsline{toc}{section}{\nameref{sec:636}}
\begin{longtable}{l p{0.5cm} r}
نشنیده‌ام که ماهی بر سر نهد کلاهی
&&
یا سرو با جوانان هرگز رود به راهی
\\
سرو بلند بستان با این همه لطافت
&&
هر روزش از گریبان سر برنکرد ماهی
\\
گر من سخن نگویم در حسن اعتدالت
&&
بالات خود بگوید زین راست‌تر گواهی
\\
روزی چو پادشاهان خواهم که برنشینی
&&
تا بشنوی ز هر سو فریاد دادخواهی
\\
با لشکرت چه حاجت رفتن به جنگ دشمن
&&
تو خود به چشم و ابرو بر هم زنی سپاهی
\\
خیلی نیازمندان بر راهت ایستاده
&&
گر می‌کنی به رحمت در کشتگان نگاهی
\\
ایمن مشو که رویت آیینه‌ایست روشن
&&
تا کی چنین بماند وز هر کناره آهی
\\
گویی چه جرم دیدی تا دشمنم گرفتی
&&
خود را نمی‌شناسم جز دوستی گناهی
\\
ای ماه سرو قامت شکرانه سلامت
&&
از حال زیردستان می‌پرس گاه گاهی
\\
شیری در این قضیت کهتر شده ز موری
&&
کوهی در این ترازو کمتر شده ز کاهی
\\
ترسم چو بازگردی از دست رفته باشم
&&
وز رستنی نبینی بر گور من گیاهی
\\
سعدی به هر چه آید گردن بنه که شاید
&&
پیش که داد خواهی از دست پادشاهی
\\
\end{longtable}
\end{center}
