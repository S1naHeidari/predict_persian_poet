\begin{center}
\section*{غزل ۱۳۸: عشق در دل ماند و یار از دست رفت}
\label{sec:138}
\addcontentsline{toc}{section}{\nameref{sec:138}}
\begin{longtable}{l p{0.5cm} r}
عشق در دل ماند و یار از دست رفت
&&
دوستان دستی که کار از دست رفت
\\
ای عجب گر من رسم در کام دل
&&
کی رسم چون روزگار از دست رفت
\\
بخت و رای و زور و زر بودم دریغ
&&
کاندر این غم هر چهار از دست رفت
\\
عشق و سودا و هوس در سر بماند
&&
صبر و آرام و قرار از دست رفت
\\
گر من از پای اندرآیم گو درآی
&&
بهتر از من صد هزار از دست رفت
\\
بیم جان کاین بار خونم می‌خورد
&&
ور نه این دل چند بار از دست رفت
\\
مرکب سودا جهانیدن چه سود
&&
چون زمام اختیار از دست رفت
\\
سعدیا با یار عشق آسان بود
&&
عشق باز اکنون که یار از دست رفت
\\
\end{longtable}
\end{center}
