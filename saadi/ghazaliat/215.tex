\begin{center}
\section*{غزل ۲۱۵: ساعتی کز درم آن سرو روان بازآمد}
\label{sec:215}
\addcontentsline{toc}{section}{\nameref{sec:215}}
\begin{longtable}{l p{0.5cm} r}
ساعتی کز درم آن سرو روان بازآمد
&&
راست گویی به تن مرده روان بازآمد
\\
بخت پیروز که با ما به خصومت می‌بود
&&
بامداد از در من صلح کنان بازآمد
\\
پیر بودم ز جفای فلک و جور زمان
&&
باز پیرانه سرم عشق جوان بازآمد
\\
دوست بازآمد و دشمن به مصیبت بنشست
&&
باد نوروز علی رغم خزان بازآمد
\\
مژدگانی بده ای نفس که سختی بگذشت
&&
دل گرانی مکن ای جسم که جان بازآمد
\\
باور از بخت ندارم که به صلح از در من
&&
آن بت سنگ دل سخت کمان بازآمد
\\
تا تو بازآمدی ای مونس جان از در غیب
&&
هر که در سر هوسی داشت از آن بازآمد
\\
عشق روی تو حرامست مگر سعدی را
&&
که به سودای تو از هر که جهان بازآمد
\\
دوستان عیب مگیرید و ملامت مکنید
&&
کاین حدیثیست که از وی نتوان بازآمد
\\
\end{longtable}
\end{center}
