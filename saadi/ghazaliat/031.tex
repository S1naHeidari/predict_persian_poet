\begin{center}
\section*{غزل ۳۱: چه فتنه بود که حسن تو در جهان انداخت}
\label{sec:031}
\addcontentsline{toc}{section}{\nameref{sec:031}}
\begin{longtable}{l p{0.5cm} r}
چه فتنه بود که حسن تو در جهان انداخت
&&
که یک دم از تو نظر بر نمی‌توان انداخت
\\
بلای غمزه نامهربان خون خوارت
&&
چه خون که در دل یاران مهربان انداخت
\\
ز عقل و عافیت آن روز بر کران ماندم
&&
که روزگار حدیث تو در میان انداخت
\\
نه باغ ماند و نه بستان که سرو قامت تو
&&
برست و ولوله در باغ و بوستان انداخت
\\
تو دوستی کن و از دیده مفکنم زنهار
&&
که دشمنم ز برای تو در زبان انداخت
\\
به چشم‌های تو کان چشم کز تو برگیرند
&&
دریغ باشد بر ماه آسمان انداخت
\\
همین حکایت روزی به دوستان برسد
&&
که سعدی از پی جانان برفت و جان انداخت
\\
\end{longtable}
\end{center}
