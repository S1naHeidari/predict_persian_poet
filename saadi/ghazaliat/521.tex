\begin{center}
\section*{غزل ۵۲۱: سل المصانع رکبا تهیم فی الفلوات}
\label{sec:521}
\addcontentsline{toc}{section}{\nameref{sec:521}}
\begin{longtable}{l p{0.5cm} r}
سل المصانع رکبا تهیم فی الفلوات
&&
تو قدر آب چه دانی که در کنار فراتی
\\
شبم به روی تو روز است و دیده‌ها به تو روشن
&&
و ان هجرت سواء عشیتی و غداتی
\\
اگر چه دیر بماندم امید برنگرفتم
&&
مضی الزمان و قلبی یقول انک آتی
\\
من آدمی به جمالت نه دیدم و نه شنیدم
&&
اگر گلی به حقیقت عجین آب حیاتی
\\
شبان تیره امیدم به صبح روی تو باشد
&&
و قد تفتش عین الحیوة فی الظلمات
\\
فکم تمرر عیشی و انت حامل شهد
&&
جواب تلخ بدیع است از آن دهان نباتی
\\
نه پنج روزهٔ عمر است عشق روی تو ما را
&&
وجدت رائحة الود ان شممت رفاتی
\\
وصفت کل ملیح کما یحب و یرضی
&&
محامد تو چه گویم که ماورای صفاتی
\\
اخاف منک و ارجوا و استغیث و ادنو
&&
که هم کمند بلایی و هم کلید نجاتی
\\
ز چشم دوست فتادم به کامهٔ دل دشمن
&&
احبتی هجرونی کما تشاء عداتی
\\
فراقنامهٔ سعدی عجب که در تو نگیرد
&&
و ان شکوت الی الطیر نحن فی الوکنات
\\
\end{longtable}
\end{center}
