\begin{center}
\section*{غزل ۱۹۶: سر جانان ندارد هر که او را خوف جان باشد}
\label{sec:196}
\addcontentsline{toc}{section}{\nameref{sec:196}}
\begin{longtable}{l p{0.5cm} r}
سر جانان ندارد هر که او را خوف جان باشد
&&
به جان گر صحبت جانان برآید رایگان باشد
\\
مغیلان چیست تا حاجی عنان از کعبه برپیچد
&&
خسک در راه مشتاقان بساط پرنیان باشد
\\
ندارد با تو بازاری مگر شوریده اسراری
&&
که مهرش در میان جان و مهرش بر دهان باشد
\\
پری رویا چرا پنهان شوی از مردم چشمم
&&
پری را خاصیت آنست کز مردم نهان باشد
\\
نخواهم رفتن از دنیا مگر در پای دیوارت
&&
که تا در وقت جان دادن سرم بر آستان باشد
\\
گر از رای تو برگردم بخیل و ناجوانمردم
&&
روان از من تمنا کن که فرمانت روان باشد
\\
به دریای غمت غرقم گریزان از همه خلقم
&&
گریزد دشمن از دشمن که تیرش در کمان باشد
\\
خلایق در تو حیرانند و جای حیرتست الحق
&&
که مه را بر زمین بینند و مه بر آسمان باشد
\\
میانت را و مویت را اگر صد ره بپیمایی
&&
میانت کمتر از مویی و مویت تا میان باشد
\\
به شمشیر از تو نتوانم که روی دل بگردانم
&&
و گر میلم کشی در چشم میلم همچنان باشد
\\
چو فرهاد از جهان بیرون به تلخی می‌رود سعدی
&&
ولیکن شور شیرینش بماند تا جهان باشد
\\
\end{longtable}
\end{center}
