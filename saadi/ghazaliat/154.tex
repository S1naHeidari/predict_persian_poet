\begin{center}
\section*{غزل ۱۵۴: جان من جان من فدای تو باد}
\label{sec:154}
\addcontentsline{toc}{section}{\nameref{sec:154}}
\begin{longtable}{l p{0.5cm} r}
جان من! جان من فدای تو باد
&&
هیچت از دوستان نیاید یاد
\\
می روی و التفات می‌نکنی
&&
سرو هرگز چنین نرفت آزاد
\\
آفرین خدای بر پدری
&&
که تو پرورد و مادری که تو زاد
\\
بخت نیکت به منتهای امید
&&
برساناد و چشم بد مرساد
\\
تا چه کرد آن که نقش روی تو بست
&&
که در فتنه بر جهان بگشاد
\\
من بگیرم عنان شه روزی
&&
گویم از دست خوبرویان داد
\\
تو بدین چشم مست و پیشانی
&&
دل ما بازپس نخواهی داد
\\
عقل با عشق بر نمی‌آید
&&
جور مزدور می‌برد استاد
\\
آن که هرگز بر آستانه عشق
&&
پای ننهاده بود سر بنهاد
\\
روی در خاک رفت و سر نه عجب
&&
که رود هم در این هوس بر باد
\\
مرغ وحشی که می‌رمید از قید
&&
با همه زیرکی به دام افتاد
\\
همه از دست غیر ناله کنند
&&
سعدی از دست خویشتن فریاد
\\
روی گفتم که در جهان بنهم
&&
گردم از قید بندگی آزاد
\\
که نه بیرون پارس منزل هست
&&
شام و رومست و بصره و بغداد
\\
دست از دامنم نمی‌دارد
&&
خاک شیراز و آب رکن آباد
\\
\end{longtable}
\end{center}
