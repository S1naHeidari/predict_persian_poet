\begin{center}
\section*{غزل ۶۰۰: صاحب نظر نباشد دربند نیک نامی}
\label{sec:600}
\addcontentsline{toc}{section}{\nameref{sec:600}}
\begin{longtable}{l p{0.5cm} r}
صاحب نظر نباشد در بند نیک نامی
&&
خاصان خبر ندارند از گفت و گوی عامی
\\
ای نقطه سیاهی بالای خط سبزش
&&
خوش دانه‌ای ولیکن بس بر کنار دامی
\\
حور از بهشت بیرون ناید تو از کجایی
&&
مه بر زمین نباشد تو ماه رخ کدامی
\\
دیگر کسش نبیند در بوستان خرامان
&&
گر سرو بوستانت بیند که می‌خرامی
\\
بدر تمام روزی در آفتاب رویت
&&
گر بنگرد بیارد اقرار ناتمامی
\\
طوطی شکر شکستن دیگر روا ندارد
&&
گر پسته‌ات ببیند وقتی که در کلامی
\\
در حسن بی‌نظیری در لطف بی نهایت
&&
در مهر بی ثباتی در عهد بی دوامی
\\
لایقتر از امیری در خدمتت اسیری
&&
خوشتر ز پادشاهی در حضرتت غلامی
\\
ترک عمل بگفتم ایمن شدم ز عزلت
&&
بی چیز را نباشد اندیشه از حرامی
\\
فردا به داغ دوزخ ناپخته‌ای بسوزد
&&
کامروز آتش عشق از وی نبرد خامی
\\
هر لحظه سر به جایی بر می‌کند خیالم
&&
تا خود چه بر من آید زین منقطع لگامی
\\
سعدی چو ترک هستی گفتی ز خلق رستی
&&
از سنگ غم نباشد بعد از شکسته جامی
\\
\end{longtable}
\end{center}
