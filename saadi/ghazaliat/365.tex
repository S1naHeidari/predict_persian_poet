\begin{center}
\section*{غزل ۳۶۵: به خاک پای عزیزت که عهد نشکستم}
\label{sec:365}
\addcontentsline{toc}{section}{\nameref{sec:365}}
\begin{longtable}{l p{0.5cm} r}
به خاک پای عزیزت که عهد نشکستم
&&
ز من بریدی و با هیچ کس نپیوستم
\\
کجا روم که بمیرم بر آستان امید
&&
اگر به دامن وصلت نمی‌رسد دستم
\\
شگفت مانده‌ام از بامداد روز وداع
&&
که برنخاست قیامت چو بی تو بنشستم
\\
بلای عشق تو نگذاشت پارسا در پارس
&&
یکی منم که ندانم نماز چون بستم
\\
نماز کردم و از بیخودی ندانستم
&&
که در خیال تو عقد نماز چون بستم
\\
نماز مست شریعت روا نمی‌دارد
&&
نماز من که پذیرد که روز و شب مستم
\\
چنین که دست خیالت گرفت دامن من
&&
چه بودی ار برسیدی به دامنت دستم
\\
من از کجا و تمنای وصل تو ز کجا
&&
اگر چه آب حیاتی هلاک خود جستم
\\
اگر خلاف تو بوده‌ست در دلم همه عمر
&&
نه نیک رفت خطا کردم و ندانستم
\\
بکش چنان که توانی که سعدی آن کس نیست
&&
که با وجود تو دعوی کند که من هستم
\\
\end{longtable}
\end{center}
