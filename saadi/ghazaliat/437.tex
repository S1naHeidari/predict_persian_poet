\begin{center}
\section*{غزل ۴۳۷: بگذار تا مقابل روی تو بگذریم}
\label{sec:437}
\addcontentsline{toc}{section}{\nameref{sec:437}}
\begin{longtable}{l p{0.5cm} r}
بگذار تا مقابل روی تو بگذریم
&&
دزدیده در شمایل خوب تو بنگریم
\\
شوق است در جدایی و جور است در نظر
&&
هم جور به که طاقت شوقت نیاوریم
\\
روی ار به روی ما نکنی حکم از آن توست
&&
بازآ که روی در قدمانت بگستریم
\\
ما را سریست با تو که گر خلق روزگار
&&
دشمن شوند و سر برود هم بر آن سریم
\\
گفتی ز خاک بیشترند اهل عشق من
&&
از خاک بیشتر نه که از خاک کمتریم
\\
ما با توایم و با تو نه‌ایم اینت بلعجب
&&
در حلقه‌ایم با تو و چون حلقه بر دریم
\\
نه بوی مهر می‌شنویم از تو ای عجب
&&
نه روی آن که مهر دگر کس بپروریم
\\
از دشمنان برند شکایت به دوستان
&&
چون دوست دشمن است شکایت کجا بریم
\\
ما خود نمی‌رویم دوان در قفای کس
&&
آن می‌برد که ما به کمند وی اندریم
\\
سعدی تو کیستی که در این حلقه کمند
&&
چندان فتاده‌اند که ما صید لاغریم
\\
\end{longtable}
\end{center}
