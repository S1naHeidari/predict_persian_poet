\begin{center}
\section*{غزل ۴۹۸: قیمت گل برود چون تو به گلزار آیی}
\label{sec:498}
\addcontentsline{toc}{section}{\nameref{sec:498}}
\begin{longtable}{l p{0.5cm} r}
قیمت گل برود چون تو به گلزار آیی
&&
و آب شیرین چو تو در خنده و گفتار آیی
\\
این همه جلوه طاووس و خرامیدن او
&&
بار دیگر نکند گر تو به رفتار آیی
\\
چند بار آخرت ای دل به نصیحت گفتم
&&
دیده بردوز نباید که گرفتار آیی
\\
مه چنین خوب نباشد تو مگر خورشیدی
&&
دل چنین سخت نباشد تو مگر خارایی
\\
گر تو صد بار بیایی به سر کشته عشق
&&
چشم باشد مترصد که دگربار آیی
\\
سپر از تیغ تو در روی کشیدن نهی است
&&
من خصومت نکنم گر تو به پیکار آیی
\\
کس نماند که به دیدار تو واله نشود
&&
چون تو لعبت ز پس پرده پدیدار آیی
\\
دیگر ای باد حدیث گل و سنبل نکنی
&&
گر بر آن سنبل زلف و گل رخسار آیی
\\
دوست دارم که کست دوست ندارد جز من
&&
حیف باشد که تو در خاطر اغیار آیی
\\
سعدیا دختر انفاس تو بس دل ببرد
&&
به چنین صورت و معنی که تو می‌آرایی
\\
\end{longtable}
\end{center}
