\begin{center}
\section*{غزل ۷۳: دیده از دیدار خوبان برگرفتن مشکلست}
\label{sec:073}
\addcontentsline{toc}{section}{\nameref{sec:073}}
\begin{longtable}{l p{0.5cm} r}
دیده از دیدار خوبان برگرفتن مشکلست
&&
هر که ما را این نصیحت می‌کند بی‌حاصلست
\\
یار زیبا گر هزارت وحشت از وی در دلست
&&
بامدادان روی او دیدن صباح مقبلست
\\
آن که در چاه زنخدانش دل بیچارگان
&&
چون ملک محبوس در زندان چاه بابلست
\\
پیش از این من دعوی پرهیزگاری کردمی
&&
باز می‌گویم که هر دعوی که کردم باطلست
\\
زهر نزدیک خردمندان اگر چه قاتلست
&&
چون ز دست دوست می‌گیری شفای عاجلست
\\
من قدم بیرون نمی‌یارم نهاد از کوی دوست
&&
دوستان معذور داریدم که پایم در گلست
\\
باش تا دیوانه گویندم همه فرزانگان
&&
ترک جان نتوان گرفتن تا تو گویی عاقلست
\\
آن که می‌گوید نظر در صورت خوبان خطاست
&&
او همین صورت همی‌بیند ز معنی غافلست
\\
ساربان آهسته ران کآرام جان در محملست
&&
چارپایان بار بر پشتند و ما را بر دلست
\\
گر به صد منزل فراق افتد میان ما و دوست
&&
همچنانش در میان جان شیرین منزلست
\\
سعدی آسانست با هر کس گرفتن دوستی
&&
لیک چون پیوند شد خو باز کردن مشکلست
\\
\end{longtable}
\end{center}
