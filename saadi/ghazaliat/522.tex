\begin{center}
\section*{غزل ۵۲۲: تو هیچ عهد نبستی که عاقبت نشکستی}
\label{sec:522}
\addcontentsline{toc}{section}{\nameref{sec:522}}
\begin{longtable}{l p{0.5cm} r}
تو هیچ عهد نبستی که عاقبت نشکستی
&&
مرا بر آتش سوزان نشاندی و ننشستی
\\
بنای مهر نمودی که پایدار نماند
&&
مرا به بند ببستی خود از کمند بجستی
\\
دلم شکستی و رفتی خلاف شرط مودت
&&
به احتیاط رو اکنون که آبگینه شکستی
\\
چراغ چون تو نباشد به هیچ خانه ولیکن
&&
کس این سرای نبندد در این چنین که تو بستی
\\
گرم عذاب نمایی به داغ و درد جدایی
&&
شکنجه صبر ندارم بریز خونم و رستی
\\
بیا که ما سر هستی و کبریا و رعونت
&&
به زیر پای نهادیم و پای بر سر هستی
\\
گرت به گوشه چشمی نظر بود به اسیران
&&
دوای درد من اول که بی‌گناه بخستی
\\
هر آن کست که ببیند روا بود که بگوید
&&
که من بهشت بدیدم به راستی و درستی
\\
گرت کسی بپرستد ملامتش نکنم من
&&
تو هم در آینه بنگر که خویشتن بپرستی
\\
عجب مدار که سعدی به یاد دوست بنالد
&&
که عشق موجب شوق است و خمر علت مستی
\\
\end{longtable}
\end{center}
