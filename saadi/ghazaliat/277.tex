\begin{center}
\section*{غزل ۲۷۷: بخت بازآید از آن در که یکی چون درآید}
\label{sec:277}
\addcontentsline{toc}{section}{\nameref{sec:277}}
\begin{longtable}{l p{0.5cm} r}
بخت بازآید از آن در که یکی چون تو درآید
&&
روی میمون تو دیدن در دولت بگشاید
\\
صبر بسیار بباید پدر پیر فلک را
&&
تا دگر مادر گیتی چو تو فرزند بزاید
\\
این لطافت که تو داری همه دل‌ها بفریبد
&&
وین بشاشت که تو داری همه غم‌ها بزداید
\\
رشکم از پیرهن آید که در آغوش تو خسبد
&&
زهرم از غالیه آید که بر اندام تو ساید
\\
نیشکر با همه شیرینی اگر لب بگشایی
&&
پیش نطق شکرینت چو نی انگشت بخاید
\\
گر مرا هیچ نباشد نه به دنیا نه به عقبی
&&
چون تو دارم همه دارم دگرم هیچ نباید
\\
دل به سختی بنهادم پس از آن دل به تو دادم
&&
هر که از دوست تحمل نکند عهد نپاید
\\
با همه خلق نمودم خم ابرو که تو داری
&&
ماه نو هر که ببیند به همه کس بنماید
\\
گر حلالست که خون همه عالم تو بریزی
&&
آن که روی از همه عالم به تو آورد نشاید
\\
چشم عاشق نتوان دوخت که معشوق نبیند
&&
پای بلبل نتوان بست که بر گل نسراید
\\
سعدیا دیدن زیبا نه حرامست ولیکن
&&
نظری گر بربایی دلت از کف برباید
\\
\end{longtable}
\end{center}
