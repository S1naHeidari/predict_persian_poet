\begin{center}
\section*{غزل ۲۷۶: به حسن دلبر من هیچ در نمی‌باید}
\label{sec:276}
\addcontentsline{toc}{section}{\nameref{sec:276}}
\begin{longtable}{l p{0.5cm} r}
به حسن دلبر من هیچ در نمی‌باید
&&
جز این دقیقه که با دوستان نمی‌پاید
\\
حلاوتیست لب لعل آبدارش را
&&
که در حدیث نیاید چو در حدیث آید
\\
ز چشم غمزده خون می‌رود به حسرت آن
&&
که او به گوشه چشم التفات فرماید
\\
بیا که دم به دمت یاد می‌رود هر چند
&&
که یاد آب به جز تشنگی نیفزاید
\\
امیدوار تو جمعی که روی بنمایی
&&
اگر چه فتنه نشاید که روی بنماید
\\
نخست خونم اگر می‌روی به قتل بریز
&&
که گر نریزی از دیده‌ام بپالاید
\\
به انتظار تو آبی که می‌رود از چشم
&&
به آب چشم نماند که چشمه می‌زاید
\\
کنند هر کسی از حضرتت تمنایی
&&
خلاف همت من کز توام تو می‌باید
\\
شکر به دست ترش روی خادمم مفرست
&&
و گر به دست خودم زهر می‌دهی شاید
\\
تو همچو کعبه عزیز اوفتاده‌ای در اصل
&&
که هر که وصل تو خواهد جهان بپیماید
\\
من آن قیاس نکردم که زور بازوی عشق
&&
عنان عقل ز دست حکیم برباید
\\
نگفتمت که به ترکان نظر مکن سعدی
&&
چو ترک ترک نگفتی تحملت باید
\\
در سرای در این شهر اگر کسی خواهد
&&
که روی خوب نبیند به گل برانداید
\\
\end{longtable}
\end{center}
