\begin{center}
\section*{غزل ۶۳۲: سرو سیمینا به صحرا می‌روی}
\label{sec:632}
\addcontentsline{toc}{section}{\nameref{sec:632}}
\begin{longtable}{l p{0.5cm} r}
سرو سیمینا به صحرا می‌روی
&&
نیک بدعهدی که بی ما می‌روی
\\
کس بدین شوخی و رعنایی نرفت
&&
خود چنینی یا به عمدا می‌روی
\\
روی پنهان دارد از مردم پری
&&
تو پری روی آشکارا می‌روی
\\
گر تماشا می‌کنی در خود نگر
&&
یا به خوشتر زین تماشا می‌روی
\\
می‌نوازی بنده را یا می‌کشی
&&
می‌نشینی یک نفس یا می‌روی
\\
اندرونم با تو می‌آید ولیک
&&
خائفم گر دست غوغا می‌روی
\\
ما خود اندر قید فرمان توایم
&&
تا کجا دیگر به یغما می‌روی
\\
جان نخواهد بردن از تو هیچ دل
&&
شهر بگرفتی به صحرا می‌روی
\\
گر قدم بر چشم من خواهی نهاد
&&
دیده بر ره می‌نهم تا می‌روی
\\
ما به دشنام از تو راضی گشته‌ایم
&&
وز دعای ما به سودا می‌روی
\\
گر چه آرام از دل ما می‌رود
&&
همچنین می‌رو که زیبا می‌روی
\\
دیده سعدی و دل همراه توست
&&
تا نپنداری که تنها می‌روی
\\
\end{longtable}
\end{center}
