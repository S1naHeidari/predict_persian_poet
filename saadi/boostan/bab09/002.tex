\begin{center}
\section*{بخش ۲ - حکایت پیرمرد و تحسر او بر روزگار جوانی: شبی در جوانی و طیب نعم}
\label{sec:002}
\addcontentsline{toc}{section}{\nameref{sec:002}}
\begin{longtable}{l p{0.5cm} r}
شبی در جوانی و طیب نعم
&&
جوانان نشستیم چندی بهم
\\
چو بلبل، سرایان چو گل تازه روی
&&
ز شوخی در افکنده غلغل به کوی
\\
جهاندیده پیری ز ما بر کنار
&&
ز دور فلک لیل مویش نهار
\\
چو فندق دهان از سخن بسته بود
&&
نه چون ما لب از خنده چون پسته بود
\\
جوانی فرا رفت کای پیرمرد
&&
چه در کنج حسرت نشینی به درد؟
\\
یکی سر برآر از گریبان غم
&&
به آرام دل با جوانان بچم
\\
برآورد سر سالخورد از نهفت
&&
جوابش نگر تا چه پیرانه گفت
\\
چو باد صبا بر گلستان وزد
&&
چمیدن درخت جوان را سزد
\\
چمد تا جوان است و سرسبز خوید
&&
شکسته شود چون به زردی رسید
\\
بهاران که بید آورد بید مشک
&&
بریزد درخت گشن برگ خشک
\\
نزیبد مرا با جوانان چمید
&&
که بر عارضم صبح پیری دمید
\\
به قید اندرم جره بازی که بود
&&
دمادم سر رشته خواهد ربود
\\
شما راست نوبت بر این خوان نشست
&&
که ما از تنعم بشستیم دست
\\
چو بر سر نشست از بزرگی غبار
&&
دگر چشم عیش جوانی مدار
\\
مرا برف باریده بر پر زاغ
&&
نشاید چو بلبل تماشای باغ
\\
کند جلوه طاووس صاحب جمال
&&
چه می‌خواهی از باز برکنده بال؟
\\
مرا غله تنگ اندر آمد درو
&&
شما را کنون می‌دمد سبزه نو
\\
گلستان ما را طراوت گذشت
&&
که گل دسته بندد چو پژمرده گشت؟
\\
مرا تکیه جان پدر بر عصاست
&&
دگر تکیه بر زندگانی خطاست
\\
مسلم جوان راست بر پای جست
&&
که پیران برند استعانت به دست
\\
گل سرخ رویم نگر زر ناب
&&
فرو رفت، چون زرد شد آفتاب
\\
هوس پختن از کودک ناتمام
&&
چنان زشت نبود که از پیر خام
\\
مرا می‌بباید چو طفلان گریست
&&
ز شرم گناهان، نه طفلانه زیست
\\
نکو گفت لقمان که نازیستن
&&
به از سالها بر خطا زیستن
\\
هم از بامدادان در کلبه بست
&&
به از سود و سرمایه دادن ز دست
\\
جوان تا رساند سیاهی به نور
&&
برد پیر مسکین سپیدی به گور
\\
\end{longtable}
\end{center}
