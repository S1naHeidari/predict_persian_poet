\begin{center}
\section*{بخش ۱۴ - حکایت: یکی مال مردم به تلبیس خورد}
\label{sec:014}
\addcontentsline{toc}{section}{\nameref{sec:014}}
\begin{longtable}{l p{0.5cm} r}
یکی مال مردم به تلبیس خورد
&&
چو برخاست لعنت بر ابلیس کرد
\\
چنین گفتش ابلیس اندر رهی
&&
که هرگز ندیدم چنین ابلهی
\\
تو را با من است ای فلان، آشتی
&&
به جنگم چرا گردن افراشتی؟
\\
دریغ است فرمودهٔ دیو زشت
&&
که دست ملک بر تو خواهد نبشت
\\
روا داری از جهل و ناباکیت
&&
که پاکان نویسند ناپاکیت
\\
طریقی به دست آر و صلحی بجوی
&&
شفیعی برانگیز و عذری بگوی
\\
که یک لحظه صورت نبندد امان
&&
چو پیمانه پر شد به دور زمان
\\
وگر دست قدرت نداری به کار
&&
چو بیچارگان دست زاری بر آر
\\
گرت رفت از اندازه بیرون بدی
&&
چو گفتی که بد رفت نیک آمدی
\\
فرا شو چو بینی ره صلح باز
&&
که ناگه در توبه گردد فراز
\\
مرو زیر بار گنه ای پسر
&&
که حمال عاجز بود در سفر
\\
پی نیک‌مردان بباید شتافت
&&
که هر کاین سعادت طلب کرد یافت
\\
ولیکن تو دنبال دیو خسی
&&
ندانم که در صالحان چون رسی؟
\\
پیمبر کسی را شفاعتگر است
&&
که بر جادهٔ شرع پیغمبر است
\\
ره راست رو تا به منزل رسی
&&
تو بر ره نه ای زین قبل واپسی
\\
چو گاوی که عصار چشمش ببست
&&
دوان تا به شب، شب همانجا که هست
\\
گل آلوده‌ای راه مسجد گرفت
&&
ز بخت نگون بود اندر شگفت
\\
یکی زجر کردش که تبت یداک
&&
مرو دامن آلوده بر جای پاک
\\
مرا رقتی در دل آمد بر این
&&
که پاک است و خرم بهشت برین
\\
در آن جای پاکان امیدوار
&&
گل آلودهٔ معصیت را چه کار؟
\\
بهشت آن ستاند که طاعت برد
&&
کرا نقد باید بضاعت برد
\\
مکن، دامن از گرد زلت بشوی
&&
که ناگه ز بالا ببندند جوی
\\
مگو مرغ دولت ز قیدم بجست
&&
هنوزش سر رشته داری به دست
\\
وگر دیر شد گرم رو باش و چست
&&
ز دیر آمدن غم ندارد درست
\\
هنوزت اجل دست خواهش نبست
&&
بر آور به درگاه دادار دست
\\
مخسب ای گنه کار خوش خفته، خیز
&&
به عذر گناه آب چشمی بریز
\\
چو حکم ضرورت بود کآبروی
&&
بریزند باری بر این خاک کوی
\\
ور آبت نماند شفیع آر پیش
&&
کسی را که هست آبروی از تو بیش
\\
به قهر ار براند خدای از درم
&&
روان بزرگان شفیع آورم
\\
\end{longtable}
\end{center}
