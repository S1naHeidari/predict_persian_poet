\begin{center}
\section*{بخش ۳ - حکایت: کهن سالی آمد به نزد طبیب}
\label{sec:003}
\addcontentsline{toc}{section}{\nameref{sec:003}}
\begin{longtable}{l p{0.5cm} r}
کهنسالی آمد به نزد طبیب
&&
ز نالیدنش تا به مردن قریب
\\
که دستم به رگ بر نه، ای نیک رای
&&
که پایم همی بر نیاید ز جای
\\
بدین ماند این قامت خفته‌ام
&&
که گویی به گل در فرو رفته‌ام
\\
برو، گفت دست از جهان در گسل
&&
که پایت قیامت برآید ز گل
\\
نشاط جوانی ز پیران مجوی
&&
که آب روان باز ناید به جوی
\\
اگر در جوانی زدی دست و پای
&&
در ایام پیری به هش باش و رای
\\
چو دوران عمر از چهل در گذشت
&&
مزن دست و پا کآبت از سر گذشت
\\
نشاط از من آن گه رمیدن گرفت
&&
که شامم سپیده دمیدن گرفت
\\
بباید هوس کردن از سر به در
&&
که دور هوسبازی آمد به سر
\\
به سبزه کجا تازه گردد دلم
&&
که سبزه بخواهد دمید از گلم؟
\\
تفرج کنان در هوا و هوس
&&
گذشتیم بر خاک بسیار کس
\\
کسانی که دیگر به غیب اندرند
&&
بیایند و بر خاک ما بگذرند
\\
دریغا که فصل جوانی برفت
&&
به لهو و لعب زندگانی برفت
\\
دریغا چنان روح پرور زمان
&&
که بگذشت بر ما چو برق یمان
\\
ز سودای آن پوشم و این خورم
&&
نپرداختم تا غم دین خورم
\\
دریغا که مشغول باطل شدیم
&&
ز حق دور ماندیم و غافل شدیم
\\
چه خوش گفت با کودک آموزگار
&&
که کاری نکردیم و شد روزگار
\\
\end{longtable}
\end{center}
