\begin{center}
\section*{بخش ۲۱ - حکایت: یکی را به چوگان مه دامغان}
\label{sec:021}
\addcontentsline{toc}{section}{\nameref{sec:021}}
\begin{longtable}{l p{0.5cm} r}
یکی را به چوگان مه دامغان
&&
بزد تا چو طبلش بر آمد فغان
\\
شب از بی قراری نیارست خفت
&&
بر او پارسایی گذر کرد و گفت
\\
به شب گر ببردی بر شحنه، سوز
&&
گناه آبرویش نبردی به روز
\\
کسی روز محشر نگردد خجل
&&
که شبها به درگه برد سوز دل
\\
هنوز ار سر صلح داری چه بیم؟
&&
در عذرخواهان نبندد کریم
\\
ز یزدان دادار داور بخواه
&&
شب توبه تقصیر روز گناه
\\
کریمی که آوردت از نیست هست
&&
عجب گر بیفتی نگیردت دست
\\
اگر بنده‌ای دست حاجت بر آر
&&
و گر شرمسار آب حسرت ببار
\\
نیامد بر این در کسی عذر خواه
&&
که سیل ندامت نشستش گناه
\\
نریزد خدای آبروی کسی
&&
که ریزد گناه آب چشمش بسی
\\
\end{longtable}
\end{center}
