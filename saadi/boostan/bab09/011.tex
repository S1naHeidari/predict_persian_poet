\begin{center}
\section*{بخش ۱۱ - موعظه و تنبیه: خبر داری ای استخوانی قفس}
\label{sec:011}
\addcontentsline{toc}{section}{\nameref{sec:011}}
\begin{longtable}{l p{0.5cm} r}
خبر داری ای استخوانی قفس
&&
که جان تو مرغی است نامش نفس؟
\\
چو مرغ از قفس رفت و بگسست قید
&&
دگر ره نگردد به سعی تو صید
\\
نگه دار فرصت که عالم دمی است
&&
دمی پیش دانا به از عالمی است
\\
سکندر که بر عالمی حکم داشت
&&
در آن دم که بگذشت و عالم گذاشت
\\
میسر نبودش کز او عالمی
&&
ستانند و مهلت دهندش دمی
\\
برفتند و هر کس درود آنچه کشت
&&
نماند به جز نام نیکو و زشت
\\
چرا دل بر این کاروانگه نهیم؟
&&
که یاران برفتند و ما بر رهیم
\\
پس از ما همین گل دمد بوستان
&&
نشینند با یکدگر دوستان
\\
دل اندر دلارام دنیا مبند
&&
که ننشست با کس که دل بر نکند
\\
چو در خاکدان لحد خفت مرد
&&
قیامت بیفشاند از موی گرد
\\
سر از جیب غفلت برآور کنون
&&
که فردا نماند به حسرت نگون
\\
نه چون خواهی آمد به شیراز در
&&
سر و تن بشویی ز گرد سفر
\\
پس ای خاکسار گنه عن قریب
&&
سفر کرد خواهی به شهری غریب
\\
بران از دو سرچشمهٔ دیده جوی
&&
ور آلایشی داری از خود بشوی
\\
\end{longtable}
\end{center}
