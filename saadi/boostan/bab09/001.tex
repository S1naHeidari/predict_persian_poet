\begin{center}
\section*{بخش ۱ - سر آغاز: بیا ای که عمرت به هفتاد رفت}
\label{sec:001}
\addcontentsline{toc}{section}{\nameref{sec:001}}
\begin{longtable}{l p{0.5cm} r}
بیا ای که عمرت به هفتاد رفت
&&
مگر خفته بودی که بر باد رفت؟
\\
همه برگ بودن همی ساختی
&&
به تدبیر رفتن نپرداختی
\\
قیامت که بازار مینو نهند
&&
منازل به اعمال نیکو دهند
\\
بضاعت به چندان که آری بری
&&
وگر مفلسی شرمساری بری
\\
که بازار چندان که آکنده‌تر
&&
تهیدست را دل پراکنده‌تر
\\
ز پنجه درم پنج اگر کم شود
&&
دلت ریش سرپنجهٔ غم شود
\\
چو پنجاه سالت برون شد ز دست
&&
غنیمت شمر پنج روزی که هست
\\
اگر مرده مسکین زبان داشتی
&&
به فریاد و زاری فغان داشتی
\\
که ای زنده چون هست امکان گفت
&&
لب از ذکر چون مرده بر هم مخفت
\\
چو ما را به غفلت بشد روزگار
&&
تو باری دمی چند فرصت شمار
\\
\end{longtable}
\end{center}
