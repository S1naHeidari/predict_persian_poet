\begin{center}
\section*{بخش ۱۰ - حکایت: شبی خفته بودم به عزم سفر}
\label{sec:010}
\addcontentsline{toc}{section}{\nameref{sec:010}}
\begin{longtable}{l p{0.5cm} r}
شبی خفته بودم به عزم سفر
&&
پی کاروانی گرفتم سحر
\\
که آمد یکی سهمگین باد و گرد
&&
که بر چشم مردم جهان تیره کرد
\\
به ره در یکی دختر خانه بود
&&
به معجر غبار از پدر می‌زدود
\\
پدر گفتش ای نازنین چهر من
&&
که داری دل آشفتهٔ مهر من
\\
نه چندان نشیند در این دیده خاک
&&
که بازش به معجر توان کرد پاک
\\
بر این خاک چندان صبا بگذرد
&&
که هر ذره از ما به جایی برد
\\
تو را نفس رعنا چو سرکش ستور
&&
دوان می‌برد تا سر شیب گور
\\
اجل ناگهت بگسلاند رکیب
&&
عنان باز نتوان گرفت از نشیب
\\
\end{longtable}
\end{center}
