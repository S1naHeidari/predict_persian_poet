\begin{center}
\section*{بخش ۶ - حکایت: قضا زنده‌ای رگ جان برید}
\label{sec:006}
\addcontentsline{toc}{section}{\nameref{sec:006}}
\begin{longtable}{l p{0.5cm} r}
قضا زنده‌ای را رگ جان برید
&&
دگر کس به مرگش گریبان درید
\\
چنین گفت بیننده‌ای تیز هوش
&&
چو فریاد و زاری رسیدش به گوش
\\
ز دست شما مرده بر خویشتن
&&
گرش دست بودی دریدی کفن
\\
که چندین ز تیمار و دردم مپیچ
&&
که روزی دو پیش از تو کردم بسیچ
\\
فراموش کردی مگر مرگ خویش
&&
که مرگ منت ناتوان کرد و ریش
\\
محقق که بر مرده ریزد گلش
&&
نه بر وی که بر خود بسوزد دلش
\\
ز هجران طفلی که در خاک رفت
&&
چه نالی؟ که پاک آمد و پاک رفت
\\
تو پاک آمدی بر حذر باش و پاک
&&
که ننگ است ناپاک رفتن به خاک
\\
کنون باید این مرغ را پای بست
&&
نه آنگه که سررشته بردت ز دست
\\
نشستی به جای دگر کس بسی
&&
نشیند به جای تو دیگر کسی
\\
اگر پهلوانی و گر تیغزن
&&
نخواهی به در بردن الا کفن
\\
خر وحش اگر بگسلاند کمند
&&
چو در ریگ ماند شود پای بند
\\
تو را نیز چندان بود دست زور
&&
که پایت نرفته‌ست در ریگ گور
\\
منه دل بر این سالخورده مکان
&&
که گنبد نپاید بر او گردکان
\\
چو دی رفت و فردا نیامد به دست
&&
حساب از همین یک نفس کن که هست
\\
\end{longtable}
\end{center}
