\begin{center}
\section*{بخش ۱۹ - حکایت دهقان در لشکر سلطان: رئیس دهی با پسر در رهی}
\label{sec:019}
\addcontentsline{toc}{section}{\nameref{sec:019}}
\begin{longtable}{l p{0.5cm} r}
رئیس دهی با پسر در رهی
&&
گذشتند بر قلب شاهنشهی
\\
پسر چاوشان دید و تیغ و تبر
&&
قباهای اطلس، کمرهای زر
\\
یلان کماندار نخجیر زن
&&
غلامان ترکش کش تیرزن
\\
یکی در برش پرنیانی قباه
&&
یکی بر سرش خسروانی کلاه
\\
پسر کان همه شوکت و پایه دید
&&
پدر را به غایت فرومایه دید
\\
که حالش بگردید و رنگش بریخت
&&
ز هیبت به بیغوله‌ای در گریخت
\\
پسر گفتش آخر بزرگ دهی
&&
به سرداری از سر بزرگان مهی
\\
چه بودت که ببریدی از جان امید
&&
بلرزیدی از باد هیبت چو بید؟
\\
بلی، گفت سالار و فرماندهم
&&
ولی عزتم هست تا در دهم
\\
بزرگان از آن دهشت آلوده‌اند
&&
که در بارگاه ملک بوده‌اند
\\
تو، ای بی خبر، همچنان در دهی
&&
که بر خویشتن منصبی می‌نهی
\\
نگفتند حرفی زبان آوران
&&
که سعدی نگوید مثالی بر آن
\\
مگر دیده باشی که در باغ و راغ
&&
بتابد به شب کرمکی چون چراغ
\\
یکی گفتش ای کرمک شب فروز
&&
چه بودت که بیرون نیایی به روز؟
\\
ببین کآتشی کرمک خاکزاد
&&
جواب از سر روشنایی چه داد
\\
که من روز و شب جز به صحرا نیم
&&
ولی پیش خورشید پیدا نیم
\\
\end{longtable}
\end{center}
