\begin{center}
\section*{بخش ۳ - در محبت روحانی: چو عشقی که بنیاد آن بر هواست}
\label{sec:003}
\addcontentsline{toc}{section}{\nameref{sec:003}}
\begin{longtable}{l p{0.5cm} r}
چو عشقی که بنیاد آن بر هواست
&&
چنین فتنه‌انگیز و فرمانرواست
\\
عجب داری از سالکان طریق
&&
که باشند در بحر معنی غریق؟
\\
به سودای جانان ز جان مشتعل
&&
به ذکر حبیب از جهان مشتغل
\\
به یاد حق از خلق بگریخته
&&
چنان مست ساقی که می ریخته
\\
نشاید به دارو دوا کردشان
&&
که کس مطلع نیست بر دردشان
\\
الست از ازل همچنانشان به گوش
&&
به فریاد قالوا بلی در خروش
\\
گروهی عمل دار عزلت نشین
&&
قدمهای خاکی، دم آتشین
\\
به یک نعره کوهی ز جا بر کنند
&&
به یک ناله شهری به هم بر زنند
\\
چو بادند پنهان و چالاک پوی
&&
چو سنگند خاموش و تسبیح گوی
\\
سحرها بگریند چندان که آب
&&
فرو شوید از دیده‌شان کحل خواب
\\
فرس کشته از بس که شب رانده‌اند
&&
سحرگه خروشان که وامانده‌اند
\\
شب و روز در بحر سودا و سوز
&&
ندانند ز آشفتگی شب ز روز
\\
چنان فتنه بر حسن صورت نگار
&&
که با حسن صورت ندارند کار
\\
ندادند صاحبدلان دل به پوست
&&
وگر ابلهی داد بی مغز کاوست
\\
می صرف وحدت کسی نوش کرد
&&
که دنیا و عقبی فراموش کرد
\\
\end{longtable}
\end{center}
