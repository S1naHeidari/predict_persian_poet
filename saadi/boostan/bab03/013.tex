\begin{center}
\section*{بخش ۱۳ - حکایت در معنی استیلای عشق بر عقل: یکی پنجهٔ آهنین راست کرد}
\label{sec:013}
\addcontentsline{toc}{section}{\nameref{sec:013}}
\begin{longtable}{l p{0.5cm} r}
یکی پنجهٔ آهنین راست کرد
&&
که با شیر زورآوری خواست کرد
\\
چو شیرش به سرپنجه در خود کشید
&&
دگر زور در پنجه در خود ندید
\\
یکی گفتش آخر چه خسبی چو زن؟
&&
به سرپنجه آهنینش بزن
\\
شنیدم که مسکین در آن زیر گفت
&&
نشاید بدین پنجه با شیر گفت
\\
چو بر عقل دانا شود عشق چیر
&&
همان پنجه آهنین است و شیر
\\
تو در پنجه شیر مرد اوژنی
&&
چه سودت کند پنجهٔ آهنی؟
\\
چو عشق آمد از عقل دیگر مگوی
&&
که در دست چوگان اسیر است گوی
\\
\end{longtable}
\end{center}
