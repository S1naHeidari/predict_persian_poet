\begin{center}
\section*{بخش ۸ - حکایت صبر و ثبات روندگان: چنین نقل دارم ز مردان راه}
\label{sec:008}
\addcontentsline{toc}{section}{\nameref{sec:008}}
\begin{longtable}{l p{0.5cm} r}
چنین نقل دارم ز مردان راه
&&
فقیران منعم، گدایان شاه
\\
که پیری به در یوزه شد بامداد
&&
در مسجدی دید و آواز داد
\\
یکی گفتش این خانهٔ خلق نیست
&&
که چیزی دهندت، بشوخی مایست
\\
بدو گفت کاین خانه کیست پس
&&
که بخشایشش نیست بر حال کس؟
\\
بگفتا خموش، این چه لفظ خطاست
&&
خداوند خانه خداوند ماست
\\
نگه کرد و قندیل و محراب دید
&&
به سوز از جگر نعره‌ای بر کشید
\\
که حیف است از این جا فراتر شدن
&&
دریغ است محروم از این در شدن
\\
نرفتم به محرومی از هیچ کوی
&&
چرا از در حق شوم زردروی؟
\\
هم این جا کنم دست خواهش دراز
&&
که دانم نگردم تهیدست باز
\\
شنیدم که سالی مجاور نشست
&&
چو فریاد خواهان برآورده دست
\\
شبی پای عمرش فرو شد به گل
&&
تپیدن گرفت از ضعیفیش دل
\\
سحر برد شخصی چراغش به سر
&&
رمق دید از او چون چراغ سحر
\\
همی‌گفت غلغل کنان از فرح
&&
و من دق باب الکریم انفتح
\\
طلبکار باید صبور و حمول
&&
که نشنیده‌ام کیمیاگر ملول
\\
چه زرها به خاک سیه در کنند
&&
که باشد که روزی مسی زر کنند
\\
زر از بهر چیزی خریدن نکوست
&&
نخواهی خریدن به از یاد دوست
\\
گر از دلبری دل به تنگ آیدت
&&
دگر غمگساری به چنگ آیدت
\\
مبر تلخ عیشی ز روی ترش
&&
به آب دگر آتشش باز کش
\\
ولی گر به خوبی ندارد نظیر
&&
به اندک دل آزار ترکش مگیر
\\
توان از کسی دل بپرداختن
&&
که دانی که بی او توان ساختن
\\
\end{longtable}
\end{center}
