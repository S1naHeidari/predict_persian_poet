\begin{center}
\section*{بخش ۱۷ - حکایت: قضا را من و پیری از فاریاب}
\label{sec:017}
\addcontentsline{toc}{section}{\nameref{sec:017}}
\begin{longtable}{l p{0.5cm} r}
قضا را من و پیری از فاریاب
&&
رسیدیم در خاک مغرب به آب
\\
مرا یک درم بود برداشتند
&&
به کشتی و درویش بگذاشتند
\\
سیاهان براندند کشتی چو دود
&&
که آن ناخدا نا خدا ترس بود
\\
مرا گریه آمد ز تیمار جفت
&&
بر آن گریه قهقه بخندید و گفت
\\
مخور غم برای من ای پر خرد
&&
مرا آن کس آرد که کشتی برد
\\
بگسترد سجاده بر روی آب
&&
خیال است پنداشتم یا به خواب
\\
ز مدهوشیم دیده آن شب نخفت
&&
نگه بامدادان به من کرد و گفت
\\
تو لنگی به چوب آمدی من به پای
&&
تو را کشتی آورد و ما را خدای
\\
چرا اهل معنی بدین نگروند
&&
که ابدال در آب و آتش روند؟
\\
نه طفلی کز آتش ندارد خبر
&&
نگه داردش مادر مهرور؟
\\
پس آنان که در وجد مستغرقند
&&
شب و روز در عین حفظ حقند
\\
نگه دارد از تاب آتش خلیل
&&
چو تابوت موسی ز غرقاب نیل
\\
چو کودک به دست شناور برست
&&
نترسد وگر دجله پهناورست
\\
تو بر روی دریا قدم چون زنی
&&
چو مردان که بر خشک تردامنی؟
\\
\end{longtable}
\end{center}
