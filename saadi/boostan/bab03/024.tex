\begin{center}
\section*{بخش ۲۴ - حکایت پروانه و صدق محبت او: کسی گفت پروانه را کای حقیر}
\label{sec:024}
\addcontentsline{toc}{section}{\nameref{sec:024}}
\begin{longtable}{l p{0.5cm} r}
کسی گفت پروانه را کای حقیر
&&
برو دوستی در خور خویش گیر
\\
رهی رو که بینی طریق رجا
&&
تو و مهر شمع از کجا تا کجا؟
\\
سمندر نه‌ای گرد آتش مگرد
&&
که مردانگی باید آنگه نبرد
\\
ز خورشید پنهان شود موش کور
&&
که جهل است با آهنین پنجه زور
\\
کسی را که دانی که خصم تو اوست
&&
نه از عقل باشد گرفتن به دوست
\\
تو را کس نگوید نکو می‌کنی
&&
که جان در سر کار او می‌کنی
\\
گدایی که از پادشه خواست دخت
&&
قفا خورد و سودای بیهوده پخت
\\
کجا در حساب آرد او چون تو دوست
&&
که روی ملوک و سلاطین در اوست؟
\\
مپندار کاو در چنان مجلسی
&&
مدارا کند با چو تو مفلسی
\\
وگر با همه خلق نرمی کند
&&
تو بیچاره‌ای با تو گرمی کند
\\
نگه کن که پروانهٔ سوزناک
&&
چه گفت، ای عجب گر بسوزم چه باک؟
\\
مرا چون خلیل آتشی در دل است
&&
که پنداری این شعله بر من گل است
\\
نه دل دامن دلستان می‌کشد
&&
که مهرش گریبان جان می‌کشد
\\
نه خود را بر آتش به خود می‌زنم
&&
که زنجیر شوق است در گردنم
\\
مرا همچنان دور بودم که سوخت
&&
نه این دم که آتش به من در فروخت
\\
نه آن می‌کند یار در شاهدی
&&
که با او توان گفتن از زاهدی
\\
که عیبم کند بر تولای دوست؟
&&
که من راضیم کشته در پای دوست
\\
مرا بر تلف حرص دانی چراست؟
&&
چو او هست اگر من نباشم رواست
\\
بسوزم که یار پسندیده اوست
&&
که در وی سرایت کند سوز دوست
\\
مرا چند گویی که در خورد خویش
&&
حریفی به دست آر همدرد خویش
\\
بدان ماند اندرز شوریده حال
&&
که گویی به کژدم گزیده منال
\\
کسی را نصیحت مگو ای شگفت
&&
که دانی که در وی نخواهد گرفت
\\
ز کف رفته بیچاره‌ای را لگام
&&
نگویند کآهسته را ای غلام
\\
چه نغز آمد این نکته در سندباد
&&
که عشق آتش است ای پسر پند، باد
\\
به باد آتش تیز برتر شود
&&
پلنگ از زدن کینه ور تر شود
\\
چو نیکت بدیدم بدی می‌کنی
&&
که رویم فرا چون خودی می‌کنی
\\
ز خود بهتری جوی و فرصت شمار
&&
که با چون خودی گم کنی روزگار
\\
پی چون خودی خودپرستان روند
&&
به کوی خطرناک مستان روند
\\
من اول که این کار سر داشتم
&&
دل از سر به یک بار برداشتم
\\
سر انداز در عاشقی صادق است
&&
که بدزهره بر خویشتن عاشق است
\\
اجل ناگهی در کمینم کشد
&&
همان به که آن نازنینم کشد
\\
چو بی شک نبشته‌ست بر سر هلاک
&&
به دست دلارام خوشتر هلاک
\\
نه روزی به بیچارگی جان دهی؟
&&
همان به که در پای جانان دهی
\\
\end{longtable}
\end{center}
