\begin{center}
\section*{بخش ۶ - حکایت در معنی غلبه وجد و سلطنت عشق: یکی شاهدی در سمرقند داشت}
\label{sec:006}
\addcontentsline{toc}{section}{\nameref{sec:006}}
\begin{longtable}{l p{0.5cm} r}
یکی شاهدی در سمرقند داشت
&&
که گفتی به جای سمر قند داشت
\\
جمالی گرو برده از آفتاب
&&
ز شوخیش بنیاد تقوی خراب
\\
تعالی الله از حسن تا غایتی
&&
که پنداری از رحمت است آیتی
\\
همی رفتی و دیده‌ها در پیش
&&
دل دوستان کرده جان برخیش
\\
نظر کردی این دوست در وی نهفت
&&
نگه کرد باری به تندی و گفت
\\
که ای خیره سر چند پویی پیم
&&
ندانی که من مرغ دامت نیم؟
\\
گرت بار دیگر ببینم به تیغ
&&
چو دشمن ببرم سرت بی دریغ
\\
کسی گفتش اکنون سر خویش گیر
&&
از این سهل تر مطلبی پیش گیر
\\
نپندارم این کام حاصل کنی
&&
مبادا که جان در سر دل کنی
\\
چو مفتون صادق ملامت شنید
&&
بدرد از درون ناله‌ای برکشید
\\
که بگذار تا زخم تیغ هلاک
&&
بغلطاندم لاشه در خون و خاک
\\
مگر پیش دشمن بگویند و دوست
&&
که این کشته دست و شمشیر اوست
\\
نمی‌بینم از خاک کویش گریز
&&
به بیداد گو آبرویم بریز
\\
مرا توبه فرمایی ای خودپرست
&&
تو را توبه زین گفت اولی ترست
\\
ببخشای بر من که هرچ او کند
&&
وگر قصد خون است نیکو کند
\\
بسوزاندم هر شبی آتشش
&&
سحر زنده گردم به بوی خوشش
\\
اگر میرم امروز در کوی دوست
&&
قیامت زنم خیمه پهلوی دوست
\\
مده تا توانی در این جنگ پشت
&&
که زنده‌ست سعدی که عشقش بکشت
\\
\end{longtable}
\end{center}
