\begin{center}
\section*{بخش ۵ - گفتار اندر گزاردن شکر نعمتها: شب از بهر آسایش تست و روز}
\label{sec:005}
\addcontentsline{toc}{section}{\nameref{sec:005}}
\begin{longtable}{l p{0.5cm} r}
شب از بهر آسایش توست و روز
&&
مه روشن و مهر گیتی فروز
\\
سپهر از برای تو فراش وار
&&
همی گستراند بساط بهار
\\
اگر باد و برف است و باران و میغ
&&
وگر رعد چوگان زند، برق تیغ
\\
همه کارداران فرمانبرند
&&
که تخم تو در خاک می‌پرورند
\\
اگر تشنه مانی ز سختی مجوش
&&
که سقای ابر آبت آرد به دوش
\\
ز خاک آورد رنگ و بوی و طعام
&&
تماشاگه دیده و مغز و کام
\\
عسل دادت از نحل و من از هوا
&&
رطب دادت از نخل و نخل از نوی
\\
همه نخلبندان بخایند دست
&&
ز حیرت که نخلی چنین کس نبست
\\
خور و ماه و پروین برای تواند
&&
قنادیل سقف سرای تواند
\\
ز خارت گل آورد و از نافه مشک
&&
زر از کان و برگ تر از چوب خشک
\\
به دست خودت چشم و ابرو نگاشت
&&
که محرم به اغیار نتوان گذاشت
\\
توانا که او نازنین پرورد
&&
به الوان نعمت چنین پرورد
\\
به جان گفت باید نفس بر نفس
&&
که شکرش نه کار زبان است و بس
\\
خدایا دلم خون شد و دیده ریش
&&
که می‌بینم انعامت از گفت بیش
\\
نگویم دد و دام و مور و سمک
&&
که فوج ملائک بر اوج فلک
\\
هنوزت سپاس اندکی گفته‌اند
&&
ز بیور هزاران یکی گفته‌اند
\\
برو سعدیا دست و دفتر بشوی
&&
به راهی که پایان ندارد مپوی
\\
\end{longtable}
\end{center}
