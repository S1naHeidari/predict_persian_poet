\begin{center}
\section*{بخش ۱۵ - حکایت سفر هندوستان و ضلالت بت پرستان: بتی دیدم از عاج در سومنات}
\label{sec:015}
\addcontentsline{toc}{section}{\nameref{sec:015}}
\begin{longtable}{l p{0.5cm} r}
بتی دیدم از عاج در سومنات
&&
مرصع چو در جاهلیت منات
\\
چنان صورتش بسته تمثالگر
&&
که صورت نبندد از آن خوبتر
\\
ز هر ناحیت کاروانها روان
&&
به دیدار آن صورت بی روان
\\
طمع کرده رایان چین و چگل
&&
چو سعدی وفا ز آن بت سخت دل
\\
زبان آوران رفته از هر مکان
&&
تضرع کنان پیش آن بی زبان
\\
فرو ماندم از کشف آن ماجرا
&&
که حیی جمادی پرستد چرا؟
\\
مغی را که با من سر و کار بود
&&
نکوگوی و هم حجره و یار بود
\\
به نرمی بپرسیدم ای برهمن
&&
عجب دارم از کار این بقعه من
\\
که مدهوش این ناتوان پیکرند
&&
مقید به چاه ضلال اندرند
\\
نه نیروی دستش، نه رفتار پای
&&
ورش بفکنی بر نخیزد ز جای
\\
نبینی که چشمانش از کهرباست؟
&&
وفا جستن از سنگ چشمان خطاست
\\
بر این گفتم آن دوست دشمن گرفت
&&
چو آتش شد از خشم و در من گرفت
\\
مغان را خبر کرد و پیران دیر
&&
ندیدم در آن انجمن روی خیر
\\
فتادند گبران پازند خوان
&&
چو سگ در من از بهر آن استخوان
\\
چو آن راه کژ پیششان راست بود
&&
ره راست در چشمشان کژ نمود
\\
که مرد ار چه دانا و صاحبدل است
&&
به نزدیک بی‌دانشان جاهل است
\\
فرو ماندم از چاره همچون غریق
&&
برون از مدارا ندیدم طریق
\\
چو بینی که جاهل به کین اندر است
&&
سلامت به تسلیم و لین اندر است
\\
مهین برهمن را ستودم بلند
&&
که ای پیر تفسیر استا و زند
\\
مرا نیز با نقش این بت خوش است
&&
که شکلی خوش و قامتی دلکش است
\\
بدیع آیدم صورتش در نظر
&&
ولیکن ز معنی ندارم خبر
\\
که سالوک این منزلم عن قریب
&&
بد از نیک کمتر شناسد غریب
\\
تو دانی که فرزین این رقعه‌ای
&&
نصیحتگر شاه این بقعه‌ای
\\
چه معنی است در صورت این صنم
&&
که اول پرستندگانش منم
\\
عبادت به تقلید گمراهی است
&&
خنک رهروی را که آگاهی است
\\
برهمن ز شادی بر افروخت روی
&&
پسندید و گفت ای پسندیده گوی
\\
سؤالت صواب است و فعلت جمیل
&&
به منزل رسد هر که جوید دلیل
\\
بسی چون تو گردیدم اندر سفر
&&
بتان دیدم از خویشتن بی خبر
\\
جز این بت که هر صبح از اینجا که هست
&&
برآرد به یزدان دادار دست
\\
وگر خواهی امشب همینجا بباش
&&
که فردا شود سر این بر تو فاش
\\
شب آنجا ببودم به فرمان پیر
&&
چو بیژن به چاه بلا در اسیر
\\
شبی همچو روز قیامت دراز
&&
مغان گرد من بی وضو در نماز
\\
کشیشان هرگز نیازرده آب
&&
بغلها چو مردار در آفتاب
\\
مگر کرده بودم گناهی عظیم
&&
که بردم در آن شب عذابی الیم
\\
همه شب در این قید غم مبتلا
&&
یکم دست بر دل، یکی بر دعا
\\
که ناگه دهل زن فرو کوفت کوس
&&
بخواند از فضای برهمن خروس
\\
خطیب سیه پوش شب بی خلاف
&&
بر آهخت شمشیر روز از غلاف
\\
فتاد آتش صبح در سوخته
&&
به یک دم جهانی شد افروخته
\\
تو گفتی که در خطهٔ زنگبار
&&
ز یک گوشه ناگه در آمد تتار
\\
مغان تبه رای ناشسته روی
&&
به دیر آمدند از در و دشت و کوی
\\
کس از مرد در شهر و از زن نماند
&&
در آن بتکده جای درزن نماند
\\
من از غصه رنجور و از خواب مست
&&
که ناگاه تمثال برداشت دست
\\
به یک بار از ایشان برآمد خروش
&&
تو گفتی که دریا بر آمد به جوش
\\
چو بتخانه خالی شد از انجمن
&&
برهمن نگه کرد خندان به من
\\
که دانم تو را بیش مشکل نماند
&&
حقیقت عیان گشت و باطل نماند
\\
چو دیدم که جهل اندر او محکم است
&&
خیال محال اندر او مدغم است
\\
نیارستم از حق دگر هیچ گفت
&&
که حق ز اهل باطل بباید نهفت
\\
چو بینی زبر دست را زور دست
&&
نه مردی بود پنجهٔ خود شکست
\\
زمانی به سالوس گریان شدم
&&
که من زآنچه گفتم پشیمان شدم
\\
به گریه دل کافران کرد میل
&&
عجب نیست سنگ ار بگردد به سیل
\\
دویدند خدمت کنان سوی من
&&
به عزت گرفتند بازوی من
\\
شدم عذرگویان بر شخص عاج
&&
به کرسی زر کوفت بر تخت ساج
\\
بتک را یکی بوسه دادم به دست
&&
که لعنت بر او باد و بر بت پرست
\\
به تقلید کافر شدم روز چند
&&
برهمن شدم در مقالات زند
\\
چو دیدم که در دیر گشتم امین
&&
نگنجیدم از خرمی در زمین
\\
در دیر محکم ببستم شبی
&&
دویدم چپ و راست چون عقربی
\\
نگه کردم از زیر تخت و زبر
&&
یکی پرده دیدم مکلل به زر
\\
پس پرده مطرانی آذرپرست
&&
مجاور سر ریسمانی به دست
\\
به فورم در آن حال معلوم شد
&&
چو داود کآهن بر او موم شد
\\
که ناچار چون در کشد ریسمان
&&
بر آرد صنم دست، فریادخوان
\\
برهمن شد از روی من شرمسار
&&
که شنعت بود بخیه بر روی کار
\\
بتازید و من در پیش تاختم
&&
نگونش به چاهی در انداختم
\\
که دانستم ار زنده آن برهمن
&&
بماند، کند سعی در خون من
\\
پسندد که از من بر آید دمار
&&
مبادا که سرش کنم آشکار
\\
چو از کار مفسد خبر یافتی
&&
ز دستش برآور چو دریافتی
\\
که گر زنده‌اش مانی، آن بی هنر
&&
نخواهد تو را زندگانی دگر
\\
وگر سر به خدمت نهد بر درت
&&
اگر دست یابد ببرد سرت
\\
فریبنده را پای در پی منه
&&
چو رفتی و دیدی امانش مده
\\
تمامش بکشتم به سنگ آن خبیث
&&
که از مرده دیگر نیاید حدیث
\\
چو دیدم که غوغایی انگیختم
&&
رها کردم آن بوم و بگریختم
\\
چو اندر نیستانی آتش زدی
&&
ز شیران بپرهیز اگر بخردی
\\
مکش بچهٔ مار مردم گزای
&&
چو کشتی در آن خانه دیگر مپای
\\
چو زنبور خانه بیاشوفتی
&&
گریز از محلت که گرم اوفتی
\\
به چابک‌تر از خود مینداز تیر
&&
چو افتاد، دامن به دندان بگیر
\\
در اوراق سعدی چنین پند نیست
&&
که چون پای دیوار کندی مایست
\\
به هند آمدم بعد از آن رستخیز
&&
وز آنجا به راه یمن تا حجیز
\\
از آن جمله سختی که بر من گذشت
&&
دهانم جز امروز شیرین نگشت
\\
در اقبال و تأیید بوبکر سعد
&&
که مادر نزاید چنو قبل و بعد
\\
ز جور فلک دادخواه آمدم
&&
در این سایه‌گستر پناه آمدم
\\
دعاگوی این دولتم بنده‌وار
&&
خدایا تو این سایه پاینده دار
\\
که مرهم نهادم نه در خورد ریش
&&
که در خورد انعام و اکرام خویش
\\
کی این شکر نعمت به جای آورم
&&
و گر پای گردد به خدمت سرم؟
\\
فرج یافتم بعد از آن بندها
&&
هنوزم به گوش است از آن پندها
\\
یکی آن که هر گه که دست نیاز
&&
برآرم به درگاه دانای راز
\\
به یاد آید آن لعبت چینیم
&&
کند خاک در چشم خودبینیم
\\
بدانم که دستی که برداشتم
&&
به نیروی خود بر نیفراشتم
\\
نه صاحبدلان دست بر می‌کشند
&&
که سررشته از غیب در می‌کشند
\\
در خیر باز است و طاعت ولیک
&&
نه هر کس تواناست بر فعل نیک
\\
همین است مانع که در بارگاه
&&
نشاید شدن جز به فرمان شاه
\\
کلید قدر نیست در دست کس
&&
توانای مطلق خدای است و بس
\\
پس ای مرد پوینده بر راه راست
&&
تو را نیست منت، خداوند راست
\\
چو در غیب نیکو نهادت سرشت
&&
نیاید ز خوی تو کردار زشت
\\
ز زنبور کرد این حلاوت پدید
&&
همان کس که در مار زهر آفرید
\\
چو خواهد که ملک تو ویران کند
&&
نخست از تو خلقی پریشان کند
\\
وگر باشدش بر تو بخشایشی
&&
رساند به خلق از تو آسایشی
\\
تکبر مکن بر ره راستی
&&
که دستت گرفتند و برخاستی
\\
سخن سودمند است اگر بشنوی
&&
به مردان رسی گر طریقت روی
\\
مقامی بیابی گرت ره دهند
&&
که بر خوان عزت سماطت نهند
\\
ولیکن نباید که تنها خوری
&&
ز درویش درمنده یاد آوری
\\
فرستی مگر رحمتی در پیم
&&
که بر کردهٔ خویش واثق نیم
\\
\end{longtable}
\end{center}
