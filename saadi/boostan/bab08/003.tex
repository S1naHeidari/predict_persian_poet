\begin{center}
\section*{بخش ۳ - گفتار اندر صنع باری عز اسمه در  ترکیب خلقت انسان: ببین تا یک انگشت از چند بند}
\label{sec:003}
\addcontentsline{toc}{section}{\nameref{sec:003}}
\begin{longtable}{l p{0.5cm} r}
ببین تا یک انگشت از چند بند
&&
به صنع الهی به هم در فگند
\\
پس آشفتگی باشد و ابلهی
&&
که انگشت بر حرف صنعش نهی
\\
تأمل کن از بهر رفتار مرد
&&
که چند استخوان پی زد و وصل کرد
\\
که بی گردش کعب و زانو و پای
&&
نشاید قدم بر گرفتن ز جای
\\
از آن سجده بر آدمی سخت نیست
&&
که در صلب او مهره یک لخت نیست
\\
دو صد مهره بر یکدگر ساخته‌ست
&&
که گل مهره‌ای چون تو پرداخته‌ست
\\
رگت بر تن است ای پسندیده خوی
&&
زمینی در او سیصد و شصت جوی
\\
بصر در سر و فکر و رای و تمیز
&&
جوارح به دل، دل به دانش عزیز
\\
بهایم به روی اندر افتاده خوار
&&
تو همچون الف بر قدمها سوار
\\
نگون کرده ایشان سر از بهر خور
&&
تو آری به عزت خورش پیش سر
\\
نزیبد تو را با چنین سروری
&&
که سر جز به طاعت فرود آوری
\\
به انعام خود دانه دادت نه کاه
&&
نکردت چو انعام سر در گیاه
\\
ولیکن بدین صورت دلپذیر
&&
فرفته مشو، سیرت خوب گیر
\\
ره راست باید نه بالای راست
&&
که کافر هم از روی صورت چو ماست
\\
تو را آن که چشم و دهان داد و گوش
&&
اگر عاقلی در خلافش مکوش
\\
گرفتم که دشمن بکوبی به سنگ
&&
مکن باری از جهل با دوست جنگ
\\
خردمند طبعان منت شناس
&&
بدوزند نعمت به میخ سپاس
\\
\end{longtable}
\end{center}
