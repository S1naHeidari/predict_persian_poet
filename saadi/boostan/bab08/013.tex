\begin{center}
\section*{بخش ۱۳ - نظر در اسباب وجود عالم: نهاده‌ست باری شفا در عسل}
\label{sec:013}
\addcontentsline{toc}{section}{\nameref{sec:013}}
\begin{longtable}{l p{0.5cm} r}
سرشته‌ست باری شفا در عسل
&&
نه چندان که زور آورد با اجل
\\
عسل خوش کند زندگان را مزاج
&&
ولی درد مردن ندارد علاج
\\
رمق مانده‌ای را که جان از بدن
&&
برآمد، چه سود انگبین در دهن؟
\\
یکی گرز پولاد بر مغز خورد
&&
کسی گفت صندل بمالش به درد
\\
ز پیش خطر تا توانی گریز
&&
ولیکن مکن با قضا پنجه تیز
\\
درون تا بود قابل شرب و اکل
&&
بدن تازه روی است و پاکیزه شکل
\\
خراب آنگه این خانه گردد تمام
&&
که با هم نسازند طبع و طعام
\\
طبایع تر و خشک و گرم است و سرد
&&
مرکب از این چار طبع است مرد
\\
یکی زین چو بر دیگری یافت دست
&&
ترازوی عدل طبیعت شکست
\\
اگر باد سرد نفس نگذرد
&&
تف معده جان در خروش آورد
\\
وگر دیگ معده نجوشد طعام
&&
تن نازنین را شود کار خام
\\
در اینان نبندد دل، اهل شناخت
&&
که پیوسته با هم نخواهند ساخت
\\
توانایی تن مدان از خورش
&&
که لطف حقت می‌دهد پرورش
\\
به حقش که گر دیده بر تیغ و کارد
&&
نهی، حق شکرش نخواهی گزارد
\\
چو رویی به طاعت نهی بر زمین
&&
خدا را ثناگوی و خود را مبین
\\
گدایی است تسبیح و ذکر و حضور
&&
گدا را نباید که باشد غرور
\\
گرفتم که خود خدمتی کرده‌ای
&&
نه پیوسته اقطاع او خورده‌ای؟
\\
\end{longtable}
\end{center}
