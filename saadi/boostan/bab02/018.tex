\begin{center}
\section*{بخش ۱۸ - حکایت: شنیدم که مردی است پاکیزه بوم}
\label{sec:018}
\addcontentsline{toc}{section}{\nameref{sec:018}}
\begin{longtable}{l p{0.5cm} r}
شنیدم که مردی است پاکیزه بوم
&&
شناسا و رهرو در اقصای روم
\\
من و چند صیاد صحرانورد
&&
برفتیم قاصد به دیدار مرد
\\
سر و چشم هر یک ببوسید و دست
&&
به تمکین و عزت نشاند و نشست
\\
زرش دیدم و زرع و شاگرد و رخت
&&
ولی بی مروت چو بی بر درخت
\\
به لطف و سخن گرم رو مرد بود
&&
ولی دیگدانش عجب سرد بود
\\
همه شب نبودش قرار و هجوع
&&
ز تسبیح و تهلیل و ما را ز جوع
\\
سحرگه میان بست و در باز کرد
&&
همان لطف و پرسیدن آغاز کرد
\\
یکی بد که شیرین و خوش طبع بود
&&
که با ما مسافر در آن ربع بود
\\
مرا بوسه گفتا به تصحیف ده
&&
که درویش را توشه از بوسه به
\\
به خدمت منه دست بر کفش من
&&
مرا نان ده و کفش بر سر بزن
\\
به ایثار مردان سبق برده‌اند
&&
نه شب زنده داران دل مرده‌اند
\\
همین دیدم از پاسبان تتار
&&
دل مرده و چشم شب زنده‌دار
\\
کرامت جوانمردی و نان دهی است
&&
مقالات بیهوده طبل تهی است
\\
قیامت کسی بینی اندر بهشت
&&
که معنی طلب کرد و دعوی بهشت
\\
به معنی توان کرد دعوی درست
&&
دم بی قدم تکیه گاهی است سست
\\
\end{longtable}
\end{center}
