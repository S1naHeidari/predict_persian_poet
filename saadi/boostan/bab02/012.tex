\begin{center}
\section*{بخش ۱۲ - گفتار اندر گردش روزگار: تو با خلق سهلی کن ای نیکبخت}
\label{sec:012}
\addcontentsline{toc}{section}{\nameref{sec:012}}
\begin{longtable}{l p{0.5cm} r}
تو با خلق سهلی کن ای نیکبخت
&&
که فردا نگیرد خدا با تو سخت
\\
گر از پا در آید، نماند اسیر
&&
که افتادگان را بود دستگیر
\\
به آزار فرمان مده بر رهی
&&
که باشد که افتد به فرماندهی
\\
چو تمکین و جاهت بود بر دوام
&&
مکن زور بر ضعف درویش عام
\\
که افتد که با جاه و تمکین شود
&&
چو بیدق که ناگاه فرزین شود
\\
نصیحت شنو مردم دوربین
&&
نپاشند در هیچ دل تخم کین
\\
خداوند خرمن زیان می‌کند
&&
که بر خوشه‌چین سر گران می‌کند
\\
نترسد که نعمت به مسکین دهند
&&
وزآن بار غم بر دل این نهند؟
\\
بسا زرومندا که افتاد سخت
&&
بس افتاده را یاوری کرد بخت
\\
دل زیر دستان نباید شکست
&&
مبادا که روزی شوی زیردست
\\
\end{longtable}
\end{center}
