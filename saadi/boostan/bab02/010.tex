\begin{center}
\section*{بخش ۱۰ - حکایت کرم مردان صاحبدل: یکی را کرم بود و قوت نبود}
\label{sec:010}
\addcontentsline{toc}{section}{\nameref{sec:010}}
\begin{longtable}{l p{0.5cm} r}
یکی را کرم بود و قوت نبود
&&
کفافش به قدر مروت نبود
\\
که سفله خداوند هستی مباد
&&
جوانمرد را تنگدستی مباد
\\
کسی را که همت بلند اوفتد
&&
مرادش کم اندر کمند اوفتد
\\
چو سیلاب ریزان که در کوهسار
&&
نگیرد همی بر بلندی قرار
\\
نه در خورد سرمایه کردی کرم
&&
تنک مایه بودی از این لاجرم
\\
برش تنگدستی دو حرفی نبشت
&&
که ای خوب فرجام نیکو سرشت
\\
یکی دست گیرم به چندین درم
&&
که چندی است تا من به زندان درم
\\
به چشم اندرش قدر چیزی نبود
&&
ولیکن به دستش پشیزی نبود
\\
به خصمان بندی فرستاد مرد
&&
که ای نیکنامان آزاد مرد
\\
بدارید چندی کف از دامنش
&&
و گر می‌گریزد ضمان بر منش
\\
وز آنجا به زندانی آمد که خیز
&&
وز این شهر تا پای داری گریز
\\
چو گنجشک در باز دید از قفس
&&
قرارش نماند اندر آن یک نفس
\\
چو باد صبا زآن میان سیر کرد
&&
نه سیری که بادش رسیدی به گرد
\\
گرفتند حالی جوانمرد را
&&
که حاصل کن این سیم یا مرد را
\\
به بیچارگی راه زندان گرفت
&&
که مرغ از قفس رفته نتوان گرفت
\\
شنیدم که در حبس چندی بماند
&&
نه شکوت نوشت و نه فریاد خواند
\\
زمانها نیاسود و شبها نخفت
&&
بر او پارسایی گذر کرد و گفت:
\\
نپندارمت مال مردم خوری
&&
چه پیش آمدت تا به زندان دری؟
\\
بگفت ای جلیس مبارک نفس
&&
نخوردم به حیلتگری مال کس
\\
یکی ناتوان دیدم از بند ریش
&&
خلاصش ندیدم به جز بند خویش
\\
ندیدم به نزدیک رایم پسند
&&
من آسوده و دیگری پایبند
\\
بمرد آخر و نیکنامی ببرد
&&
زهی زندگانی که نامش نمرد
\\
تنی زنده دل، خفته در زیر گل
&&
به از عالمی زندهٔ مرده دل
\\
دل زنده هرگز نگردد هلاک
&&
تن زنده دل گر بمیرد چه باک؟
\\
\end{longtable}
\end{center}
