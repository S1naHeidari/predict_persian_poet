\begin{center}
\section*{بخش ۳۰ - حکایت: شنیدم که مردی غم خانه خورد}
\label{sec:030}
\addcontentsline{toc}{section}{\nameref{sec:030}}
\begin{longtable}{l p{0.5cm} r}
شنیدم که مردی غم خانه خورد
&&
که زنبور بر سقف او لانه کرد
\\
زنش گفت از اینان چه خواهی؟ مکن
&&
که مسکین پریشان شوند از وطن
\\
بشد مرد نادان پس کار خویش
&&
گرفتند یک روز زن را به نیش
\\
زن بی خرد بر در و بام و کوی
&&
همی کرد فریاد و می‌گفت شوی:
\\
مکن روی بر مردم ای زن ترش
&&
تو گفتی که زنبور مسکین مکش
\\
کسی با بدان نیکویی چون کند؟
&&
بدان را تحمل، بد افزون کند
\\
چو اندر سری بینی آزار خلق
&&
به شمشیر تیزش بیازار حلق
\\
سگ آخر که باشد که خوانش نهند؟
&&
بفرمای تا استخوانش دهند
\\
چه نیکو زده‌ست این مثل پیر ده
&&
ستور لگدزن گران بار به
\\
اگر نیکمردی نماید عسس
&&
نیارد به شب خفتن از دزد، کس
\\
نی نیزه در حلقهٔ کارزار
&&
بقیمت تر از نیشکر صد هزار
\\
نه هر کس سزاوار باشد به مال
&&
یکی مال خواهد، یکی گوشمال
\\
چو گربه نوازی کبوتر برد
&&
چو فربه کنی گرگ، یوسف درد
\\
بنایی که محکم ندارد اساس
&&
بلندش مکن ور کنی زو هراس
\\
چه خوش گفت بهرام صحرانشین
&&
چو یکران توسن زدش بر زمین
\\
دگر اسبی از گله باید گرفت
&&
که گر سر کشد باز شاید گرفت
\\
ببند ای پسر دجله در آب کاست
&&
که سودی ندارد چو سیلاب خاست
\\
چو گرگ خبیث آمدت در کمند
&&
بکش ور نه دل بر کن از گوسفند
\\
از ابلیس هرگز نیاید سجود
&&
نه از بد گهر نیکویی در وجود
\\
بد اندیش را جاه و فرصت مده
&&
عدو در چه و دیو در شیشه به
\\
مگو شاید این مار کشتن به چوب
&&
چو سر زیر سنگ تو دارد بکوب
\\
قلم زن که بد کرد با زیردست
&&
قلم بهتر او را به شمشیر دست
\\
مدبر که قانون بد می‌نهد
&&
تو را می‌برد تا به دوزخ دهد
\\
مگو ملک را این مدبر بس است
&&
مدبر مخوانش که مدبر کس است
\\
سعید آورد قول سعدی به جای
&&
که ترتیب ملک است و تدبیر رای
\\
\end{longtable}
\end{center}
