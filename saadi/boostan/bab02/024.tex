\begin{center}
\section*{بخش ۲۴ - حکایت: شنیدم که مغروری از کبر مست}
\label{sec:024}
\addcontentsline{toc}{section}{\nameref{sec:024}}
\begin{longtable}{l p{0.5cm} r}
شنیدم که مغروری از کبر مست
&&
در خانه بر روی سائل ببست
\\
به کنجی فرو ماند و بنشست مرد
&&
جگر گرم و آه از تف سینه سرد
\\
شنیدش یکی مرد پوشیده چشم
&&
بپرسیدش از موجب کین و خشم
\\
فرو گفت و بگریست بر خاک کوی
&&
جفایی کز آن شخصش آمد به روی
\\
بگفت ای فلان ترک آزار کن
&&
یک امشب به نزد من افطار کن
\\
به خلق و فریبش گریبان کشید
&&
به خانه در آوردش و خوان کشید
\\
بر آسود درویش روشن نهاد
&&
بگفت ایزدت روشنایی دهاد
\\
شب از نرگسش قطره چندی چکید
&&
سحر دیده بر کرد و دنیا بدید
\\
حکایت به شهر اندر افتاد و جوش
&&
که آن بی بصر دیده بر کرد دوش
\\
شنید این سخن خواجه سنگدل
&&
که برگشت درویش از او تنگدل
\\
بگفتا حکایت کن ای نیکبخت
&&
که چون سهل شد بر تو این کار سخت؟
\\
که بر کردت این شمع گیتی فروز؟
&&
بگفت ای ستمکار آشفته روز
\\
تو کوته نظر بودی و سست رای
&&
که مشغول گشتی به جغد از همای
\\
به روی من این در کسی کرد باز
&&
که کردی تو بر روی وی در، فراز
\\
اگر بوسه بر خاک مردان زنی
&&
به مردی که پیش آیدت روشنی
\\
کسانی که پوشیده چشم دلند
&&
همانا کز این توتیا غافلند
\\
چو برگشته دولت ملامت شنید
&&
سر انگشت حیرت به دندان گزید
\\
که شهباز من صید دام تو شد
&&
مرا بود دولت به نام تو شد
\\
کسی چون به دست آورد جره باز
&&
فرو برده چون موش دندان آز؟
\\
الا گر طلبکار اهل دلی
&&
ز خدمت مکن یک زمان غافلی
\\
خورش ده به گنجشک و کبک و حمام
&&
که یک روزت افتد همایی به دام
\\
چو هر گوشه تیر نیاز افکنی
&&
امید است ناگه که صیدی زنی
\\
دری هم بر آید ز چندین صدف
&&
ز صد چوبه آید یکی بر هدف
\\
\end{longtable}
\end{center}
