\begin{center}
\section*{بخش ۲۰ - حکایت در آزمودن پادشاه یمن حاتم را به آزادمردی: ندانم که گفت این حکایت به من}
\label{sec:020}
\addcontentsline{toc}{section}{\nameref{sec:020}}
\begin{longtable}{l p{0.5cm} r}
ندانم که گفت این حکایت به من
&&
که بوده‌ست فرماندهی در یمن
\\
ز نام آوران گوی دولت ربود
&&
که در گنج بخشی نظیرش نبود
\\
توان گفت او را سحاب کرم
&&
که دستش چو باران فشاندی درم
\\
کسی نام حاتم نبردی برش
&&
که سودا نرفتی از او بر سرش
\\
که چند از مقالات آن بادسنج
&&
که نه ملک دارد نه فرمان نه گنج
\\
شنیدم که جشنی ملوکانه ساخت
&&
چو چنگ اندر آن بزم خلقی نواخت
\\
در ذکر حاتم کسی باز کرد
&&
دگر کس ثنا گفتن آغاز کرد
\\
حسد مرد را بر سر کینه داشت
&&
یکی را به خون خوردنش بر گماشت
\\
که تا هست حاتم در ایام من
&&
نخواهد به نیکی شدن نام من
\\
بلا جوی راه بنی طی گرفت
&&
به کشتن جوانمرد را پی گرفت
\\
جوانی به ره پیشباز آمدش
&&
کز او بوی انسی فراز آمدش
\\
نکو روی و دانا و شیرین زبان
&&
بر خویش برد آن شبش میهمان
\\
کرم کرد و غم خورد و پوزش نمود
&&
بد اندیش را دل به نیکی ربود
\\
نهادش سحر بوسه بر دست و پای
&&
که نزدیک ما چند روزی بپای
\\
بگفتا نیارم شد اینجا مقیم
&&
که در پیش دارم مهمی عظیم
\\
بگفت ار نهی با من اندر میان
&&
چو یاران یکدل بکوشم به جان
\\
به من دار گفت، ای جوانمرد، گوش
&&
که دانم جوانمرد را پرده پوش
\\
در این بوم حاتم شناسی مگر
&&
که فرخنده رای است و نیکو سیر؟
\\
سرش پادشاه یمن خواسته‌ست
&&
ندانم چه کین در میان خاسته‌ست!
\\
گرم ره نمایی بدان جا که اوست
&&
همین چشم دارم ز لطف تو دوست
\\
بخندید برنا که حاتم منم
&&
سر اینک جدا کن به تیغ از تنم
\\
نباید که چون صبح گردد سفید
&&
گزندت رسد یا شوی ناامید
\\
چو حاتم به آزادگی سر نهاد
&&
جوان را برآمد خروش از نهاد
\\
به خاک اندر افتاد و بر پای جست
&&
گهش خاک بوسید و گه پای و دست
\\
بینداخت شمشیر و ترکش نهاد
&&
چو بیچارگان دست بر کش نهاد
\\
که من گر گلی بر وجودت زنم
&&
به نزدیک مردان نه مردم، زنم
\\
دو چشمش ببوسید و در بر گرفت
&&
وز آنجا طریق یمن بر گرفت
\\
ملک در میان دو ابروی مرد
&&
بدانست حالی که کاری نکرد
\\
بگفتا بیا تا چه داری خبر
&&
چرا سر نبستی به فتراک بر؟
\\
مگر بر تو نام‌آوری حمله کرد
&&
نیاوردی از ضعف تاب نبرد؟
\\
جوانمرد شاطر زمین بوسه داد
&&
ملک را ثنا گفت و تمکین نهاد
\\
که دریافتم حاتم نامجوی
&&
هنرمند و خوش منظر و خوبروی
\\
جوانمرد و صاحب خرد دیدمش
&&
به مردانگی فوق خود دیدمش
\\
مرا بار لطفش دو تا کرد پشت
&&
به شمشیر احسان و فضلم بکشت
\\
بگفت آنچه دید از کرمهای وی
&&
شهنشه ثنا گفت بر آل طی
\\
فرستاده را داد مهری درم
&&
که مهر است بر نام حاتم کرم
\\
مر او را سزد گر گواهی دهند
&&
که معنی و آوازه‌اش همرهند
\\
\end{longtable}
\end{center}
