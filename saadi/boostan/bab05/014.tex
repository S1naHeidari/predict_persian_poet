\begin{center}
\section*{بخش ۱۴ - حکایت: شنیدم که نابالغی روزه داشت}
\label{sec:014}
\addcontentsline{toc}{section}{\nameref{sec:014}}
\begin{longtable}{l p{0.5cm} r}
شنیدم که نابالغی روزه داشت
&&
به صد محنت آورد روزی به چاشت
\\
به کتابش آن روز سائق نبرد
&&
بزرگ آمدش طاعت از طفل خرد
\\
پدر دیده بوسید و مادر سرش
&&
فشاندند بادام و زر بر سرش
\\
چو بر وی گذر کرد یک نیمه روز
&&
فتاد اندر او ز آتش معده سوز
\\
به دل گفت اگر لقمه چندی خورم
&&
چه داند پدر غیب یا مادرم؟
\\
چو روی پسر در پدر بود و قوم
&&
نهان خورد و پیدا به سر برد صوم
\\
که داند چو در بند حق نیستی
&&
اگر بی وضو در نماز ایستی؟
\\
پس این پیر از آن طفل نادان تر است
&&
که از بهر مردم به طاعت در است
\\
کلید در دوزخ است آن نماز
&&
که در چشم مردم گزاری دراز
\\
اگر جز به حق می‌رود جاده‌ات
&&
در آتش فشانند سجاده‌ات
\\
\end{longtable}
\end{center}
