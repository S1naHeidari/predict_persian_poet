\begin{center}
\section*{بخش ۳ - حکایت تیرانداز اردبیلی: یکی آهنین پنجه در اردبیل}
\label{sec:003}
\addcontentsline{toc}{section}{\nameref{sec:003}}
\begin{longtable}{l p{0.5cm} r}
یکی آهنین پنجه در اردبیل
&&
همی بگذرانید پیلک ز پیل
\\
نمد پوشی آمد به جنگش فراز
&&
جوانی جهان سوز پیکار ساز
\\
به پرخاش جستن چو بهرام گور
&&
کمندی به کتفش بر از خام گور
\\
چو دید اردبیلی نمد پاره پوش
&&
کمان در زه آورد و زه را به گوش
\\
به پنجاه تیر خدنگش بزد
&&
که یک چوبه بیرون نرفت از نمد
\\
درآمد نمدپوش چون سام گرد
&&
به خم کمندش درآورد و برد
\\
به لشکرگهش برد و در خیمه دست
&&
چو دزدان خونی به گردن ببست
\\
شب از غیرت و شرمساری نخفت
&&
سحرگه پرستاری از خیمه گفت
\\
تو کآهن به ناوک بدوزی و تیر
&&
نمدپوش را چون فتادی اسیر؟
\\
شنیدم که می‌گفت و خون می‌گریست
&&
ندانی که روز اجل کس نزیست؟
\\
من آنم که در شیوهٔ طعن و ضرب
&&
به رستم در آموزم آداب حرب
\\
چو بازوی بختم قوی حال بود
&&
ستبری پیلم نمد می‌نمود
\\
کنونم که در پنجه اقبیل نیست
&&
نمد پیش تیرم کم از پیل نیست
\\
به روز اجل نیزه جوشن درد
&&
ز پیراهن بی اجل نگذرد
\\
کرا تیغ قهر اجل در قفاست
&&
برهنه‌ست اگر جوشنش چند لاست
\\
ورش بخت یاور بود، دهر پشت
&&
برهنه نشاید به ساطور کشت
\\
نه دانا به سعی از اجل جان ببرد
&&
نه نادان به ناساز خوردن بمرد
\\
\end{longtable}
\end{center}
