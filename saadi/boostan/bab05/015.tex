\begin{center}
\section*{بخش ۱۵ - حکایت: سیهکاری از نردبانی فتاد}
\label{sec:015}
\addcontentsline{toc}{section}{\nameref{sec:015}}
\begin{longtable}{l p{0.5cm} r}
سیهکاری از نردبانی فتاد
&&
شنیدم که هم در نفس جان بداد
\\
پسر چند روزی گرستن گرفت
&&
دگر با حریفان نشستن گرفت
\\
به خواب اندرش دید و پرسید حال
&&
که چون رستی از حشر و نشر و سؤال؟
\\
بگفت ای پسر قصه بر من مخوان
&&
به دوزخ در افتادم از نردبان
\\
نکو سیرتی بی تکلف برون
&&
به از نیکنامی خراب اندرون
\\
به نزدیک من شبرو راهزن
&&
به از فاسق پارسا پیرهن
\\
یکی بر در خلق رنج آزمای
&&
چه مزدش دهد در قیامت خدای؟
\\
ز عمرو ای پسر چشم اجرت مدار
&&
چو در خانهٔ زید باشی به کار
\\
نگویم تواند رسیدن به دوست
&&
در این ره جز آن کس که رویش در اوست
\\
ره راست رو تا به منزل رسی
&&
تو در ره نه‌ای، زین قبل واپسی
\\
چو گاوی که عصار چشمش ببست
&&
دوان تا به شب، شب همان جا که هست
\\
کسی گر بتابد ز محراب روی
&&
به کفرش گواهی دهند اهل کوی
\\
تو هم پشت بر قبله‌ای در نماز
&&
گرت در خدا نیست روی نیاز
\\
درختی که بیخش بود برقرار
&&
بپرور، که روزی دهد میوه بار
\\
گرت بیخ اخلاص در بوم نیست
&&
از این بر کسی چون تو محروم نیست
\\
هر آن کافکند تخم بر روی سنگ
&&
جوی وقت دخلش نیاید به چنگ
\\
منه آبروی ریا را محل
&&
که این آب در زیر دارد وحل
\\
چو در خفیه بد باشم و خاکسار
&&
چه سود آب ناموس بر روی کار؟
\\
به روی و ریا خرقه سهل است دوخت
&&
گرش با خدا در توانی فروخت
\\
چه دانند مردم که در جامه کیست؟
&&
نویسنده داند که در نامه چیست
\\
چه وزن آورد جایی انبان باد
&&
که میزان عدل است و دیوان داد؟
\\
مرائی که چندین ورع می‌نمود
&&
بدیدند و هیچش در انبان نبود
\\
کنند ابره پاکیزه‌تر ز آستر
&&
که آن در حجاب است و این در نظر
\\
بزرگان فراغ از نظر داشتند
&&
از آن پرنیان آستر داشتند
\\
ور آوازه خواهی در اقلیم فاش
&&
برون حله کن گو درون حشو باش
\\
به بازی نگفت این سخن بایزید
&&
که از منکر ایمن‌ترم کز مرید
\\
کسانی که سلطان و شاهنشهند
&&
سراسر گدایان این درگهند
\\
طمع در گدا، مرد معنی نبست
&&
نشاید گرفتن در افتاده دست
\\
همان به گر آبستن گوهری
&&
که همچون صدف سر به خود در بری
\\
چو روی پرستیدنت در خداست
&&
اگر جبرئیلت نبیند رواست
\\
تو را پند سعدی بس است ای پسر
&&
اگر گوش گیری چو پند پدر
\\
گر امروز گفتار ما نشنوی
&&
مبادا که فردا پشیمان شوی
\\
از این به نصیحتگری بایدت
&&
ندانم پس از من چه پیش آیدت!
\\
\end{longtable}
\end{center}
