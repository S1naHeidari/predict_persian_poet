\begin{center}
\section*{بخش ۲ - حکایت: مرا در سپاهان یکی یار بود}
\label{sec:002}
\addcontentsline{toc}{section}{\nameref{sec:002}}
\begin{longtable}{l p{0.5cm} r}
مرا در سپاهان یکی یار بود
&&
که جنگاور و شوخ و عیار بود
\\
مدامش به خون دست و خنجر خضاب
&&
بر آتش دل خصم از او چون کباب
\\
ندیدمش روزی که ترکش نبست
&&
ز پولاد پیکانش آتش نجست
\\
دلاور به سرپنجهٔ گاوزور
&&
ز هولش به شیران در افتاده شور
\\
به دعوی چنان ناوک انداختی
&&
که عذرا به هر یک یک انداختی
\\
چنان خار در گل ندیدم که رفت
&&
که پیکان او در سپرهای جفت
\\
نزد تارک جنگجویی به خشت
&&
که خود و سرش را نه در هم سرشت
\\
چو گنجشک روز ملخ در نبرد
&&
به کشتن چه گنجشک پیشش چه مرد
\\
گرش بر فریدون بدی تاختن
&&
امانش ندادی به تیغ آختن
\\
پلنگانش از زور سرپنجه زیر
&&
فرو برده چنگال در مغز شیر
\\
گرفتی کمربند جنگ آزمای
&&
وگر کوه بودی بکندی ز جای
\\
زره پوش را چون تبرزین زدی
&&
گذر کردی از مرد و بر زین زدی
\\
نه در مردی او را نه در مردمی
&&
دوم در جهان کس شنید آدمی
\\
مرا یک دم از دست نگذاشتی
&&
که با راست طبعان سری داشتی
\\
سفر ناگهم زان زمین در ربود
&&
که بیشم در آن بقعه روزی نبود
\\
قضا نقل کرد از عراقم به شام
&&
خوش آمد در آن خاک پاکم مقام
\\
مع القصه چندی ببودم مقیم
&&
به رنج و به راحت، به امید و بیم
\\
دگر پر شد از شام پیمانه‌ام
&&
کشید آرزومندی خانه‌ام
\\
قضا را چنان اتفاق اوفتاد
&&
که بازم گذر بر عراق اوفتاد
\\
شبی سر فرو شد به اندیشه‌ام
&&
به دل برگذشت آن هنر پیشه‌ام
\\
نمک ریش دیرینه‌ام تازه کرد
&&
که بودم نمک خورده از دست مرد
\\
به دیدار وی در سپاهان شدم
&&
به مهرش طلبکار و خواهان شدم
\\
جوان دیدم از گردش دهر، پیر
&&
خدنگش کمان، ارغوانش زریر
\\
چو کوه سپیدش سر از برف موی
&&
دوان آبش از برف پیری به روی
\\
فلک دست قوت بر او یافته
&&
سر دست مردیش بر تافته
\\
بدر کرده گیتی غرور از سرش
&&
سر ناتوانی به زانو برش
\\
بدو گفتم ای سرور شیر گیر
&&
چه فرسوده کردت چو روباه پیر؟
\\
بخندید کز روز جنگ تتر
&&
بدر کردم آن جنگجویی ز سر
\\
زمین دیدم از نیزه چو نیستان
&&
گرفته علمها چو آتش در آن
\\
بر انگیختم گرد هیجا چو دود
&&
چو دولت نباشد تهور چه سود؟
\\
من آنم که چون حمله آوردمی
&&
به رمح از کف انگشتری بردمی
\\
ولی چون نکرد اخترم یاوری
&&
گرفتند گردم چو انگشتری
\\
غنیمت شمردم طریق گریز
&&
که نادان کند با قضا پنجه تیز
\\
چه یاری کند مغفر و جوشنم
&&
چو یاری نکرد اختر روشنم؟
\\
کلید ظفر چون نباشد به دست
&&
به بازو در فتح نتوان شکست
\\
گروهی پلنگ افکن پیل زور
&&
در آهن سر مرد و سم ستور
\\
همان دم که دیدیم گرد سپاه
&&
زره جامه کردیم و مغفر کلاه
\\
چو ابر اسب تازی برانگیختیم
&&
چو باران بلارک فرو ریختیم
\\
دو لشکر به هم بر زدند از کمین
&&
تو گفتی زدند آسمان بر زمین
\\
ز باریدن تیر همچو تگرگ
&&
به هر گوشه برخاست طوفان مرگ
\\
به صید هژبران پرخاش ساز
&&
کمند اژدهای دهن کرده باز
\\
زمین آسمان شد ز گرد کبود
&&
چو انجم در او برق شمشیر و خود
\\
سواران دشمن چو دریافتیم
&&
پیاده سپر در سپر بافتیم
\\
به تیر و سنان موی بشکافتیم
&&
چو دولت نبد روی بر تافتیم
\\
چه زور آورد پنجهٔ جهد مرد
&&
چو بازوی توفیق یاری نکرد؟
\\
نه شمشیر گندآوران کند بود
&&
که کین آوری ز اختر تند بود
\\
کس از لشکر ما ز هیجا برون
&&
نیامد جز آغشته خفتان به خون
\\
چو صد دانه مجموع در خوشه‌ای
&&
فتادیم هر دانه‌ای گوشه‌ای
\\
به نامردی از هم بدادیم دست
&&
چو ماهی که با جوشن افتد به شست
\\
کسان را نشد ناوک اندر حریر
&&
که گفتم بدوزند سندان به تیر
\\
چو طالع ز ما روی بر پیچ بود
&&
سپر پیش تیر قضا هیچ بود
\\
از این بوالعجب‌تر حدیثی شنو
&&
که بی بخت کوشش نیرزد دو جو
\\
\end{longtable}
\end{center}
