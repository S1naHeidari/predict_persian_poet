\begin{center}
\section*{بخش ۷ - حکایت توبه کردن ملک زادهٔ گنجه: یکی پادشه‌زاده در گنجه بود}
\label{sec:007}
\addcontentsline{toc}{section}{\nameref{sec:007}}
\begin{longtable}{l p{0.5cm} r}
یکی پادشه‌زاده در گنجه بود
&&
که دور از تو ناپاک و سرپنجه بود
\\
به مسجد در آمد سرایان و مست
&&
می اندر سر و ساتکینی به دست
\\
به مقصوره در پارسایی مقیم
&&
زبانی دلاویز و قلبی سلیم
\\
تنی چند بر گفت او مجتمع
&&
چو عالم نباشی کم از مستمع
\\
چو بی عزتی پیشه کرد آن حرون
&&
شدند آن عزیزان خراب اندرون
\\
چو منکر بود پادشه را قدم
&&
که یارد زد از امر معروف دم؟
\\
تحکم کند سیر بر بوی گل
&&
فرو ماند آواز چنگ از دهل
\\
گرت نهی منکر بر آید ز دست
&&
نشاید چو بی دست و پایان نشست
\\
وگر دست قدرت نداری، بگوی
&&
که پاکیزه گردد به اندرز خوی
\\
چو دست و زبان را نماند مجال
&&
به همت نمایند مردی رجال
\\
یکی پیش دانای خلوت نشین
&&
بنالید و بگریست سر بر زمین
\\
که باری بر این رند ناپاک و مست
&&
دعا کن که ما بی زبانیم و دست
\\
دمی سوزناک از دلی با خبر
&&
قوی تر که هفتاد تیغ و تبر
\\
بر آورد مرد جهاندیده دست
&&
چه گفت ای خداوند بالا و پست
\\
خوش است این پسر وقتش از روزگار
&&
خدایا همه وقت او خوش بدار
\\
کسی گفتش ای قدوهٔ راستی
&&
بر این بد چرا نیکویی خواستی؟
\\
چو بد عهد را نیک خواهی ز بهر
&&
چه بد خواستی بر سر خلق شهر؟
\\
چنین گفت بینندهٔ تیز هوش
&&
چو سر سخن در نیابی مجوش
\\
به طامات مجلس نیاراستم
&&
ز داد آفرین توبه‌اش خواستم
\\
که هر گه که باز آید از خوی زشت
&&
به عیشی رسد جاودان در بهشت
\\
همین پنج روز است عیش مدام
&&
به ترک اندرش عیشهای مدام
\\
حدیثی که مرد سخن ساز گفت
&&
کسی ز آن میان با ملک باز گفت
\\
ز وجد آب در چشمش آمد چو میغ
&&
ببارید بر چهره سیل دریغ
\\
به نیران شوق اندرونش بسوخت
&&
حیا دیده بر پشت پایش بدوخت
\\
بر نیک محضر فرستاد کس
&&
در توبه کوبان که فریاد رس
\\
قدم رنجه فرمای تا سر نهم
&&
سر جهل و ناراستی بر نهم
\\
دو رویه ستادند بر در سپاه
&&
سخن پرور آمد در ایوان شاه
\\
شکر دید و عناب و شمع و شراب
&&
ده از نعمت آباد و مردم خراب
\\
یکی غایب از خود، یکی نیم مست
&&
یکی شعر گویان صراحی به دست
\\
ز سویی بر آورده مطرب خروش
&&
ز دیگر سو آواز ساقی که نوش
\\
حریفان خراب از می لعل رنگ
&&
سر چنگی از خواب در بر چو چنگ
\\
نبود از ندیمان گردن فراز
&&
به جز نرگس آن جا کسی دیده باز
\\
دف و چنگ با یکدگر سازگار
&&
بر آورده زیر از میان ناله زار
\\
بفرمود و در هم شکستند خرد
&&
مبدل شد آن عیش صافی به درد
\\
شکستند چنگ و گسستند رود
&&
به در کرد گوینده از سر سرود
\\
به میخانه در سنگ بر دن زدند
&&
کدو را نشاندند و گردن زدند
\\
می لاله گون از بط سرنگون
&&
روان همچنان کز بط کشته خون
\\
خم آبستن خمر نه ماهه بود
&&
در آن فتنه دختر بینداخت زود
\\
شکم تا به نافش دریدند مشک
&&
قدح را بر او چشم خونی پر اشک
\\
بفرمود تا سنگ صحن سرای
&&
بکندند و کردند نو باز جای
\\
که گلگونه خمر یاقوت فام
&&
به شستن نمی‌شد ز روی رخام
\\
عجب نیست بالوعه گر شد خراب
&&
که خورد اندر آن روز چندان شراب
\\
دگر هر که بربط گرفتی به کف
&&
قفا خوردی از دست مردم چو دف
\\
وگر فاسقی چنگ بردی به دوش
&&
بمالیدی او را چو طنبور گوش
\\
جوان سر از کبر و پندار مست
&&
چو پیران به کنج عبادت نشست
\\
پدر بارها گفته بودش به هول
&&
که شایسته رو باش و پاکیزه قول
\\
جفای پدر برد و زندان و بند
&&
چنان سودمندش نیامد که پند
\\
گرش سخت گفتی سخنگوی سهل
&&
که بیرون کن از سر جوانی و جهل
\\
خیال و غرورش بر آن داشتی
&&
که درویش را زنده نگذاشتی
\\
سپر نفکند شیر غران ز جنگ
&&
نیندیشد از تیغ بران پلنگ
\\
به نرمی ز دشمن توان کرد دوست
&&
چو با دوست سختی کنی دشمن اوست
\\
چو سندان کسی سخت رویی نکرد
&&
که خایسک تأدیب بر سر نخورد
\\
به گفتن درشتی مکن با امیر
&&
چو بینی که سختی کند، سست گیر
\\
به اخلاق با هر که بینی بساز
&&
اگر زیردست است اگر سرفراز
\\
که این گردن از نازکی بر کشد
&&
به گفتار خوش، و آن سر اندر کشد
\\
به شیرین زبانی توان برد گوی
&&
که پیوسته تلخی برد تندخوی
\\
تو شیرین زبانی ز سعدی بگیر
&&
ترش روی را گو به تلخی بمیر
\\
\end{longtable}
\end{center}
