\begin{center}
\section*{بخش ۶ - حکایت دانشمند: فقیهی کهن جامهٔ تنگدست}
\label{sec:006}
\addcontentsline{toc}{section}{\nameref{sec:006}}
\begin{longtable}{l p{0.5cm} r}
فقیهی کهن جامهٔ تنگدست
&&
در ایوان قاضی به صف بر نشست
\\
نگه کرد قاضی در او تیز تیز
&&
معرف گرفت آستینش که خیز
\\
ندانی که برتر مقام تو نیست
&&
فروتر نشین، یا برو، یا بایست
\\
نه هر کس سزاوار باشد به صدر
&&
کرامت به جاه است و منزل به قدر
\\
دگر ره چه حاجت ببیند کست؟
&&
همین شرمساری عقوبت بست
\\
به عزت هر آن کاو فروتر نشست
&&
به خواری نیفتد ز بالا به پست
\\
به جای بزرگان دلیری مکن
&&
چو سر پنجه‌ات نیست شیری مکن
\\
چو دید آن خردمند درویش رنگ
&&
که بنشست و برخاست بختش به جنگ
\\
چو آتش برآورد بیچاره دود
&&
فروتر نشست از مقامی که بود
\\
فقیهان طریق جدل ساختند
&&
لم و لا اسلم درانداختند
\\
گشادند بر هم در فتنه باز
&&
به لا و نعم کرده گردن دراز
\\
تو گفتی خروسان شاطر به جنگ
&&
فتادند در هم به منقار و چنگ
\\
یکی بی خود از خشمناکی چو مست
&&
یکی بر زمین می‌زند هر دو دست
\\
فتادند در عقدهٔ پیچ پیچ
&&
که در حل آن ره نبردند هیچ
\\
کهن جامه در صف آخرترین
&&
به غرش در آمد چو شیر عرین
\\
بگفت ای صنادید شرع رسول
&&
به ابلاغ تنزیل و فقه و اصول
\\
دلایل قوی باید و معنوی
&&
نه رگهای گردن به حجت قوی
\\
مرا نیز چوگان لعب است و گوی
&&
بگفتند اگر نیک دانی بگوی
\\
به کلک فصاحت بیانی که داشت
&&
به دلها چو نقش نگین بر نگاشت
\\
سر از کوی صورت به معنی کشید
&&
قلم بر سر حرف دعوی کشید
\\
بگفتندش از هر کنار آفرین
&&
که بر عقل و طبعت هزار آفرین
\\
سمند سخن تا به جایی براند
&&
که قاضی چو خر در وحل باز ماند
\\
برون آمد از طاق و دستار خویش
&&
به اکرام و لطفش فرستاد پیش
\\
که هیهات قدر تو نشناختم
&&
به شکر قدومت نپرداختم
\\
دریغ آیدم با چنین مایه‌ای
&&
که بینم تو را در چنین پایه‌ای
\\
معرف به دلداری آمد برش
&&
که دستار قاضی نهد بر سرش
\\
به دست و زبان منع کردش که دور
&&
منه بر سرم پایبند غرور
\\
که فردا شود بر کهن میزران
&&
به دستار پنجه گزم سر گران
\\
چو مولام خوانند و صدر کبیر
&&
نمایند مردم به چشمم حقیر
\\
تفاوت کند هرگز آب زلال
&&
گرش کوزه زرین بود یا سفال؟
\\
خرد باید اندر سر مرد و مغز
&&
نباید مرا چون تو دستار نغز
\\
کس از سر بزرگی نباشد به چیز
&&
کدو سر بزرگ است و بی مغز نیز
\\
میفراز گردن به دستار و ریش
&&
که دستار پنبه‌ست و سبلت حشیش
\\
به صورت کسانی که مردم وشند
&&
چو صورت همان به که دم در کشند
\\
به قدر هنر جست باید محل
&&
بلندی و نحسی مکن چون زحل
\\
نی بوریا را بلندی نکوست
&&
که خاصیت نیشکر خود در اوست
\\
بدین عقل و همت نخوانم کست
&&
و گر می‌رود صد غلام از پست
\\
چه خوش گفت خرمهره‌ای در گلی
&&
چو برداشتش پر طمع جاهلی
\\
مرا کس نخواهد خریدن به هیچ
&&
به دیوانگی در حریرم مپیچ
\\
خبزدو همان قدر دارد که هست
&&
وگر در میان شقایق نشست
\\
نه منعم به مال از کسی بهتر است
&&
خر ار جل اطلس بپوشد خر است
\\
بدین شیوه مرد سخنگوی چست
&&
به آب سخن کینه از دل بشست
\\
دل آزرده را سخت باشد سخن
&&
چو خصمت بیفتاد سستی مکن
\\
چو دستت رسد مغز دشمن بر آر
&&
که فرصت فرو شوید از دل غبار
\\
چنان ماند قاضی به جورش اسیر
&&
که گفت ان هذا لیوم عسیر
\\
به دندان گزید از تعجب یدین
&&
بماندش در او دیده چون فرقدین
\\
وز آنجا جوان روی همت بتافت
&&
برون رفت و بازش نشان کس نیافت
\\
غریو از بزرگان مجلس بخاست
&&
که گویی چنین شوخ چشم از کجاست؟
\\
نقیب از پیش رفت و هر سو دوید
&&
که مردی بدین نعت و صورت که دید؟
\\
یکی گفت از این نوع شیرین نفس
&&
در این شهر سعدی شناسیم و بس
\\
بر آن صد هزار آفرین کاین بگفت
&&
حق تلخ بین تا چه شیرین بگفت
\\
\end{longtable}
\end{center}
