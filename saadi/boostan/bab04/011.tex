\begin{center}
\section*{بخش ۱۱ - حکایت خواجه نیکوکار و بنده نافرمان: بزرگی هنرمند آفاق بود}
\label{sec:011}
\addcontentsline{toc}{section}{\nameref{sec:011}}
\begin{longtable}{l p{0.5cm} r}
بزرگی هنرمند آفاق بود
&&
غلامش نکوهیده اخلاق بود
\\
از این خفرگی موی کالیده‌ای
&&
بدی، سرکه در روی مالیده‌ای
\\
چو ثعبانش آلوده دندان به زهر
&&
گرو برده از زشت رویان شهر
\\
مدامش به روی آب چشم سبل
&&
دویدی ز بوی پیاز بغل
\\
گره وقت پختن بر ابرو زدی
&&
چو پختند با خواجه زانو زدی
\\
دمادم به نان خوردنش هم نشست
&&
و گر مردی آبش ندادی به دست
\\
نه گفت اندر او کار کردی نه چوب
&&
شب و روز از او خانه در کند و کوب
\\
گهی خار و خس در ره انداختی
&&
گهی ماکیان در چه انداختی
\\
ز سیماش وحشت فراز آمدی
&&
نرفتی به کاری که باز آمدی
\\
کسی گفت از این بندهٔ بد خصال
&&
چه خواهی؟ ادب ، یا هنر، یا جمال؟
\\
نیرزد وجودی بدین ناخوشی
&&
که جورش پسندی و بارش کشی
\\
منت بندهٔ خوب و نیکو سیر
&&
به دست آرم، این را به نخاس بر
\\
و گر یک پشیز آورد سر مپیچ
&&
گران است اگر راست خواهی به هیچ
\\
شنید این سخن مرد نیکو نهاد
&&
بخندید کای یار فرخ نژاد
\\
بد است این پسر طبع و خویش ولیک
&&
مرا زاو طبیعت شود خوی نیک
\\
چو زاو کرده باشم تحمل بسی
&&
توانم جفا بردن از هر کسی
\\
تحمل چو زهرت نماید نخست
&&
ولی شهد گردد چو در طبع رست
\\
\end{longtable}
\end{center}
