\begin{center}
\section*{بخش ۸ - حکایت: شکر خنده‌ای انگبین می‌فروخت}
\label{sec:008}
\addcontentsline{toc}{section}{\nameref{sec:008}}
\begin{longtable}{l p{0.5cm} r}
شکر خنده‌ای انگبین می‌فروخت
&&
که دلها ز شیرینیش می‌بسوخت
\\
نباتی میان بسته چون نیشکر
&&
بر او مشتری از مگس بیشتر
\\
گر او زهر برداشتی فی‌المثل
&&
بخوردندی از دست او چون عسل
\\
گرانی نظر کرد در کار او
&&
حسد برد بر گرم بازار او
\\
دگر روز شد گرد گیتی دوان
&&
عسل بر سر و سرکه بر ابروان
\\
بسی گشت فریادخوان پیش و پس
&&
که ننشست بر انگبینش مگس
\\
شبانگه چو نقدش نیامد به دست
&&
به دلتنگ رویی به کنجی نشست
\\
چو عاصی ترش کرده روی از وعید
&&
چو ابروی زندانیان روز عید
\\
زنی گفت بازی کنان شوی را
&&
عسل تلخ باشد ترش روی را
\\
به دوزخ برد مرد را خوی زشت
&&
که اخلاق نیک آمده‌ست از بهشت
\\
برو آب گرم از لب جوی خور
&&
نه جلاب سرد ترش روی خور
\\
حرامت بود نان آن کس چشید
&&
که چون سفره ابرو به هم در کشید
\\
مکن خواجه بر خویشتن کار سخت
&&
که بدخوی باشد نگون‌سار بخت
\\
گرفتم که سیم و زرت چیز نیست
&&
چو سعدی زبان خوشت نیز نیست؟
\\
\end{longtable}
\end{center}
