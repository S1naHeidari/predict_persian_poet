\begin{center}
\section*{بخش ۱۵ - حکایت در محرومی خویشتن بینان: یکی در نجوم اندکی دست داشت}
\label{sec:015}
\addcontentsline{toc}{section}{\nameref{sec:015}}
\begin{longtable}{l p{0.5cm} r}
یکی در نجوم اندکی دست داشت
&&
ولی از تکبر سری مست داشت
\\
بر کوشیار آمد از راه دور
&&
دلی پر ارادت، سری پر غرور
\\
خردمند از او دیده بردوختی
&&
یکی حرف در وی نیاموختی
\\
چو بی بهره عزم سفر کرد باز
&&
بدو گفت دانای گردن فراز
\\
تو خود را گمان برده‌ای پر خرد
&&
انائی که پر شد دگر چون برد؟
\\
ز دعوی پری زان تهی می‌روی
&&
تهی آی تا پر معانی شوی
\\
ز هستی در آفاق سعدی صفت
&&
تهی گرد و باز آی پر معرفت
\\
\end{longtable}
\end{center}
