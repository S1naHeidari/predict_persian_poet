\begin{center}
\section*{بخش ۱۸ - حکایت حاتم اصم: گروهی برآنند از اهل سخن}
\label{sec:018}
\addcontentsline{toc}{section}{\nameref{sec:018}}
\begin{longtable}{l p{0.5cm} r}
گروهی برآنند از اهل سخن
&&
که حاتم اصم بود، باور مکن
\\
برآمد طنین مگس بامداد
&&
که در چنبر عنکبوتی فتاد
\\
همه ضعف و خاموشیش کید بود
&&
مگس قند پنداشتش قید بود
\\
نگه کرد شیخ از سر اعتبار
&&
که ای پایبند طمع پای دار
\\
نه هر جا شکر باشد و شهد و قند
&&
که در گوشه‌ها دامیار است و بند
\\
یکی گفت از آن حلقهٔ اهل رای
&&
عجب دارم ای مرد راه خدای
\\
مگس را تو چون فهم کردی خروش
&&
که ما را به دشواری آمد به گوش؟
\\
تو آگاه گشتی به بانگ مگس
&&
نشاید اصم خواندنت زین سپس
\\
تبسم کنان گفت ای تیز هوش
&&
اصم به که گفتار باطل نیوش
\\
کسانی که با ما به خلوت درند
&&
مرا عیب پوش و ثنا گسترند
\\
چو پوشیده دارند اخلاق دون
&&
کند هستیم زیر، طبع زبون
\\
فرا می‌نمایم که می‌نشنوم
&&
مگر کز تکلف مبرا شوم
\\
چو کالیو دانندم اهل نشست
&&
بگویند نیک و بدم هر چه هست
\\
اگر بد شنیدن نیاید خوشم
&&
ز کردار بد دامن اندر کشم
\\
به حبل ستایش فرا چه مشو
&&
چو حاتم اصم باش و عیبت شنو
\\
\end{longtable}
\end{center}
