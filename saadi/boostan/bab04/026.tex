\begin{center}
\section*{بخش ۲۶ - حکایت: گدایی شنیدم که در تنگ جای}
\label{sec:026}
\addcontentsline{toc}{section}{\nameref{sec:026}}
\begin{longtable}{l p{0.5cm} r}
گدایی شنیدم که در تنگ جای
&&
نهادش عمر پای بر پشت پای
\\
ندانست درویش بیچاره کاوست
&&
که رنجیده دشمن نداند ز دوست
\\
برآشفت بر وی که کوری مگر؟
&&
بدو گفت سالار عادل عمر
\\
نه کورم ولیکن خطا رفت کار
&&
ندانستم از من گنه در گذار
\\
چه منصف بزرگان دین بوده‌اند
&&
که با زیردستان چنین بوده‌اند
\\
فروتن بود هوشمند گزین
&&
نهد شاخ پر میوه سر بر زمین
\\
بنازند فردا تواضع کنان
&&
نگون از خجالت سر گردنان
\\
اگر می‌بترسی ز روز شمار
&&
از آن کز تو ترسد خطا در گذار
\\
مکن خیره بر زیر دستان ستم
&&
که دستی است بالای دست تو هم
\\
\end{longtable}
\end{center}
