\begin{center}
\section*{بخش ۱۶ - حکایت: به خشم از ملک بنده‌ای سربتافت}
\label{sec:016}
\addcontentsline{toc}{section}{\nameref{sec:016}}
\begin{longtable}{l p{0.5cm} r}
به خشم از ملک بنده‌ای سربتافت
&&
بفرمود جستن کسش در نیافت
\\
چو بازآمد از راه خشم و ستیز
&&
به شمشیر زن گفت خونش بریز
\\
به خون تشنه جلاد نامهربان
&&
برون کرد دشنه چو تشنه زبان
\\
شنیدم که گفت از دل تنگ ریش
&&
خدایا بحل کردمش خون خویش
\\
که پیوسته در نعمت و ناز و نام
&&
در اقبال او بوده‌ام دوستکام
\\
مبادا که فردا به خون منش
&&
بگیرند و خرم شود دشمنش
\\
ملک را چو گفت وی آمد به گوش
&&
دگر دیگ خشمش نیاورد جوش
\\
بسی بر سرش داد و بر دیده بوس
&&
خداوند رایت شد و طبل و کوس
\\
به رفق از چنان سهمگن جایگاه
&&
رسانید دهرش بدان پایگاه
\\
غرض زین حدیث آن که گفتار نرم
&&
چو آب است بر آتش مرد گرم
\\
تواضع کن ای دوست با خصم تند
&&
که نرمی کند تیغ برنده کند
\\
نبینی که در معرض تیغ و تیر
&&
بپوشند خفتان صد تو حریر
\\
\end{longtable}
\end{center}
