\begin{center}
\section*{بخش ۲۰ - حکایت در معنی احتمال از دشمن از بهر دوست: یکی را چو سعدی دلی ساده بود}
\label{sec:020}
\addcontentsline{toc}{section}{\nameref{sec:020}}
\begin{longtable}{l p{0.5cm} r}
یکی را چو سعدی دلی ساده بود
&&
که با ساده رویی در افتاده بود
\\
جفا بردی از دشمن سختگوی
&&
ز چوگان سختی بخستی چو گوی
\\
ز کس چین بر ابرو نینداختی
&&
ز یاری به تندی نپرداختی
\\
یکی گفتش آخر تو را ننگ نیست؟
&&
خبر زین همه سیلی و سنگ نیست؟
\\
تن خویشتن سغبه دونان کنند
&&
ز دشمن تحمل زبونان کنند
\\
نشاید ز دشمن خطا درگذاشت
&&
که گویند یارا و مردی نداشت
\\
بدو گفت شیدای شوریده سر
&&
جوابی که شاید نبشتن به زر
\\
دلم خانهٔ مهر یار است و بس
&&
از آن می‌نگنجد در او کین کس
\\
چه خوش گفت بهلول فرخنده خوی
&&
چو بگذشت بر عارفی جنگجوی
\\
گر این مدعی دوست بشناختی
&&
به پیکار دشمن نپرداختی
\\
گر از هستی حق خبر داشتی
&&
همه خلق را نیست پنداشتی
\\
\end{longtable}
\end{center}
