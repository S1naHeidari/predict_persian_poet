\begin{center}
\section*{بخش ۳۰ - حکایت درویش صادق و پادشاه بیدادگر: شنیدم که از نیکمردی فقیر}
\label{sec:030}
\addcontentsline{toc}{section}{\nameref{sec:030}}
\begin{longtable}{l p{0.5cm} r}
شنیدم که از نیکمردی فقیر
&&
دل آزرده شد پادشاهی کبیر
\\
مگر بر زبانش حقی رفته بود
&&
ز گردن‌کشی بر وی آشفته بود
\\
به زندان فرستادش از بارگاه
&&
که زورآزمای است بازوی جاه
\\
ز یاران کسی گفتش اندر نهفت
&&
مصالح نبود این سخن گفت، گفت
\\
رسانیدن امر حق طاعت است
&&
ز زندان نترسم که یک ساعت است
\\
همان دم که در خفیه این راز رفت
&&
حکایت به گوش ملک باز رفت
\\
بخندید کاو ظن بیهوده برد
&&
نداند که خواهد در این حبس مرد
\\
غلامی به درویش برد این پیام
&&
بگفتا به خسرو بگو ای غلام
\\
مرا بار غم بر دل ریش نیست
&&
که دنیا همین ساعتی بیش نیست
\\
نه گر دستگیری کنی خرمم
&&
نه گر سر بری بر دل آید غمم
\\
تو گر کامرانی به فرمان و گنج
&&
دگر کس فرومانده در ضعف و رنج
\\
به دروازهٔ مرگ چون در شویم
&&
به یک هفته با هم برابر شویم
\\
منه دل بر این دولت پنج روز
&&
به دود دل خلق، خود را مسوز
\\
نه پیش از تو بیش از تو اندوختند
&&
به بیداد کردن جهان سوختند؟
\\
چنان زی که ذکرت به تحسین کنند
&&
چو مردی، نه بر گور نفرین کنند
\\
نباید به رسم بد آیین نهاد
&&
که گویند لعنت بر آن، کاین نهاد
\\
وگر بر سرآید خداوند زور
&&
نه زیرش کند عاقبت خاک گور؟
\\
بفرمود دلتنگ روی از جفا
&&
که بیرون کنندش زبان از قفا
\\
چنین گفت مرد حقایق شناس
&&
کز این هم که گفتی ندارم هراس
\\
من از بی زبانی ندارم غمی
&&
که دانم که ناگفته داند همی
\\
اگر بینوایی برم ور ستم
&&
گرم عاقبت خیر باشد چه غم؟
\\
عروسی بود نوبت ماتمت
&&
گرت نیکروزی بود خاتمت
\\
\end{longtable}
\end{center}
