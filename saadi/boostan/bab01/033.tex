\begin{center}
\section*{بخش ۳۳ - گفتار اندر رای و تدبیر ملک و لشکرکشی: همی تا برآید به تدبیر کار}
\label{sec:033}
\addcontentsline{toc}{section}{\nameref{sec:033}}
\begin{longtable}{l p{0.5cm} r}
همی تا برآید به تدبیر کار
&&
مدارای دشمن به از کارزار
\\
چو نتوان عدو را به قوت شکست
&&
به نعمت بباید در فتنه بست
\\
گر اندیشه باشد ز خصمت گزند
&&
به تعویذ احسان زبانش ببند
\\
عدو را به جای خسک زر بریز
&&
که احسان کند کند، دندان تیز
\\
چو دستی نشاید گزیدن، ببوس
&&
که با غالبان چاره زرق است و لوس
\\
به تدبیر رستم در آید به بند
&&
که اسفندیارش نجست از کمند
\\
عدو را به فرصت توان کند پوست
&&
پس او را مدارا چنان کن که دوست
\\
حذر کن ز پیکار کمتر کسی
&&
که از قطره سیلاب دیدم بسی
\\
مزن تا توانی بر ابرو گره
&&
که دشمن اگرچه زبون، دوست به
\\
بود دشمنش تازه و دوست ریش
&&
کسی کش بود دشمن از دوست بیش
\\
مزن با سپاهی ز خود بیشتر
&&
که نتوان زد انگشت بر نیشتر
\\
وگر زو تواناتری در نبرد
&&
نه مردی است بر ناتوان زور کرد
\\
اگر پیل زوری وگر شیر چنگ
&&
به نزدیک من صلح بهتر که جنگ
\\
چو دست از همه حیلتی در گسست
&&
حلال است بردن به شمشیر دست
\\
اگر صلح خواهد عدو سر مپیچ
&&
وگر جنگ جوید عنان بر مپیچ
\\
که گر وی ببندد در کارزار
&&
تو را قدر و هیبت شود یک، هزار
\\
ور او پای جنگ آورد در رکاب
&&
نخواهد به حشر از تو داور حساب
\\
تو هم جنگ را باش چون کینه خواست
&&
که با کینه ور مهربانی خطاست
\\
چو با سفله گویی به لطف و خوشی
&&
فزون گرددش کبر و گردن کشی
\\
به اسبان تازی و مردان مرد
&&
بر آر از نهاد بداندیش گرد
\\
و گر می بر آید به نرمی و هوش
&&
به تندی و خشم و درشتی مکوش
\\
چو دشمن به عجز اندر آمد ز در
&&
نباید که پرخاش جویی دگر
\\
چو زنهار خواهد کرم پیشه کن
&&
ببخشای و از مکرش اندیشه کن
\\
ز تدبیر پیر کهن بر مگرد
&&
که کارآزموده بود سالخورد
\\
در آرند بنیاد رویین ز پای
&&
جوانان به نیروی و پیران به رای
\\
بیندیش در قلب هیجا مفر
&&
چه دانی که زان که باشد ظفر؟
\\
چو بینی که لشکر ز هم دست داد
&&
به تنها مده جان شیرین به باد
\\
اگر بر کناری به رفتن بکوش
&&
وگر در میان لبس دشمن بپوش
\\
وگر خود هزاری و دشمن دویست
&&
چو شب شد در اقلیم دشمن مایست
\\
شب تیره پنجه سوار از کمین
&&
چو پانصد به هیبت بدرد زمین
\\
چو خواهی بریدن به شب راهها
&&
حذر کن نخست از کمینگاهها
\\
میان دو لشکر چو یک روز راه
&&
بماند، بزن خیمه بر جایگاه
\\
گر او پیشدستی کند غم مدار
&&
ور افراسیاب است مغزش بر آر
\\
ندانی که لشکر چو یک روزه راند
&&
سر پنجهٔ زورمندش نماند
\\
تو آسوده بر لشکر مانده زن
&&
که نادان ستم کرد بر خویشتن
\\
چو دشمن شکستی بیفکن علم
&&
که بازش نیاید جراحت به هم
\\
بسی در قفای هزیمت مران
&&
نباید که دور افتی از یاوران
\\
هوا بینی از گرد هیجا چو میغ
&&
بگیرند گردت به زوبین و تیغ
\\
به دنبال غارت نراند سپاه
&&
که خالی بماند پس پشت شاه
\\
سپه را نگهبانی شهریار
&&
به از جنگ در حلقهٔ کارزار
\\
\end{longtable}
\end{center}
