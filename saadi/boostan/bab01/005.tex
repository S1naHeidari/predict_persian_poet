\begin{center}
\section*{بخش ۵ - حکایت در شناختن دوست و دشمن را: شنیدم که دارای فرخ تبار}
\label{sec:005}
\addcontentsline{toc}{section}{\nameref{sec:005}}
\begin{longtable}{l p{0.5cm} r}
شنیدم که دارای فرخ تبار
&&
ز لشکر جدا ماند روز شکار
\\
دوان آمدش گله‌بانی به پیش
&&
به دل گفت دارای فرخنده کیش
\\
مگر دشمن است این که آمد به جنگ
&&
ز دورش بدوزم به تیر خدنگ
\\
کمان کیانی به زه راست کرد
&&
به یک دم وجودش عدم خواست کرد
\\
بگفت ای خداوند ایران و تور
&&
که چشم بد از روزگار تو دور
\\
من آنم که اسبان شه پرورم
&&
به خدمت بدین مرغزار اندرم
\\
ملک را دل رفته آمد به جای
&&
بخندید و گفت: ای نکوهیده رای
\\
تو را یاوری کرد فرخ سروش
&&
وگر نه زه آورده بودم به گوش
\\
نگهبان مرعی بخندید و گفت:
&&
نصیحت ز منعم نباید نهفت
\\
نه تدبیر محمود و رای نکوست
&&
که دشمن نداند شهنشه ز دوست
\\
چنان است در مهتری شرط زیست
&&
که هر کهتری را بدانی که کیست
\\
مرا بارها در حضر دیده‌ای
&&
ز خیل و چراگاه پرسیده‌ای
\\
کنونت به مهر آمدم پیشباز
&&
نمی‌دانیم از بداندیش باز
\\
توانم من، ای نامور شهریار
&&
که اسبی برون آرم از صد هزار
\\
مرا گله‌بانی به عقل است و رای
&&
تو هم گلهٔ خویش باری، بپای
\\
در آن تخت و ملک از خلل غم بود
&&
که تدبیر شاه از شبان کم بود
\\
\end{longtable}
\end{center}
