\begin{center}
\section*{بخش ۱۵ - اندر معنی عدل و ظلم و ثمرهٔ آن: خبرداری از خسروان عجم}
\label{sec:015}
\addcontentsline{toc}{section}{\nameref{sec:015}}
\begin{longtable}{l p{0.5cm} r}
خبرداری از خسروان عجم
&&
که کردند بر زیردستان ستم؟
\\
نه آن شوکت و پادشایی بماند
&&
نه آن ظلم بر روستایی بماند
\\
خطا بین که بر دست ظالم برفت
&&
جهان ماند و با او مظالم برفت
\\
خنک روز محشر تن دادگر
&&
که در سایهٔ عرش دارد مقر
\\
به قومی که نیکی پسندد خدای
&&
دهد خسروی عادل و نیک رای
\\
چو خواهد که ویران شود عالمی
&&
کند ملک در پنجهٔ ظالمی
\\
سگالند از او نیکمردان حذر
&&
که خشم خدای است بیدادگر
\\
بزرگی از او دان و منت شناس
&&
که زایل شود نعمت ناسپاس
\\
اگر شکر کردی بر این ملک و مال
&&
به مالی و ملکی رسی بی زوال
\\
وگر جور در پادشایی کنی
&&
پس از پادشایی گدایی کنی
\\
حرام است بر پادشه خواب خوش
&&
چو باشد ضعیف از قوی بارکش
\\
میازار عامی به یک خردله
&&
که سلطان شبان است و عامی گله
\\
چو پرخاش بینند و بیداد از او
&&
شبان نیست، گرگ است، فریاد از او
\\
بد انجام رفت و بد اندیشه کرد
&&
که با زیردستان جفا، پیشه کرد
\\
به سختی و سستی بر این بگذرد
&&
بماند بر او سالها نام بد
\\
نخواهی که نفرین کنند از پست
&&
نکو باش تا بد نگوید کست
\\
\end{longtable}
\end{center}
