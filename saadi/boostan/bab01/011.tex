\begin{center}
\section*{بخش ۱۱ - حکایت مرزبان ستمگار با زاهد: خردمند مردی در اقصای شام}
\label{sec:011}
\addcontentsline{toc}{section}{\nameref{sec:011}}
\begin{longtable}{l p{0.5cm} r}
خردمند مردی در اقصای شام
&&
گرفت از جهان کنج غاری مقام
\\
به صبرش در آن کنج تاریک جای
&&
به گنج قناعت فرو رفته پای
\\
شنیدم که نامش خدادوست بود
&&
ملک سیرتی، آدمی پوست بود
\\
بزرگان نهادند سر بر درش
&&
که در می‌نیامد به درها سرش
\\
تمنا کند عارف پاکباز
&&
به دریوزه از خویشتن ترک آز
\\
چو هر ساعتش نفس گوید بده
&&
به خواری بگرداندش ده به ده
\\
در آن مرز کاین پیر هشیار بود
&&
یکی مرزبان ستمکار بود
\\
که هر ناتوان را که دریافتی
&&
به سرپنجگی پنجه برتافتی
\\
جهان سوز و بی‌رحمت و خیره‌کش
&&
ز تلخیش روی جهانی ترش
\\
گروهی برفتند از آن ظلم و عار
&&
ببردند نام بدش در دیار
\\
گروهی بماندند مسکین و ریش
&&
پس چرخه نفرین گرفتند پیش
\\
ید ظلم جایی که گردد دراز
&&
نبینی لب مردم از خنده باز
\\
به دیدار شیخ آمدی گاه گاه
&&
خدادوست در وی نکردی نگاه
\\
ملک نوبتی گفتش: ای نیکبخت
&&
به نفرت ز من در مکش روی سخت
\\
مرا با تو دانی سر دوستی است
&&
تو را دشمنی با من از بهر چیست؟
\\
گرفتم که سالار کشور نیم
&&
به عزت ز درویش کمتر نیم
\\
نگویم فضیلت نهم بر کسی
&&
چنان باش با من که با هر کسی
\\
شنید این سخن عابد هوشیار
&&
بر آشفت و گفت: ای ملک، هوش دار
\\
وجودت پریشانی خلق از اوست
&&
ندارم پریشانی خلق دوست
\\
تو با آن که من دوستم، دشمنی
&&
نپندارمت دوستدار منی
\\
چرا دوست دارم به باطل منت
&&
چو دانم که دارد خدا دشمنت؟
\\
مده بوسه بر دست من دوستوار
&&
برو دوستداران من دوست دار
\\
خدادوست را گر بدرند پوست
&&
نخواهد شدن دشمن دوست، دوست
\\
عجب دارم از خواب آن سنگدل
&&
که خلقی بخسبند از او تنگدل
\\
\end{longtable}
\end{center}
