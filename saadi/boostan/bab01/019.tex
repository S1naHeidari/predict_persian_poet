\begin{center}
\section*{بخش ۱۹ - گفتار اندر نکوکاری و بدکاری و عاقبت آنها: نکوکار مردم نباشد بدش}
\label{sec:019}
\addcontentsline{toc}{section}{\nameref{sec:019}}
\begin{longtable}{l p{0.5cm} r}
نکوکار مردم نباشد بدش
&&
نورزد کسی بد که نیک افتدش
\\
شر انگیز هم بر سر شر شود
&&
چو کژدم که با خانه کمتر شود
\\
اگر نفع کس در نهاد تو نیست
&&
چنین گوهر و سنگ خارا یکی است
\\
غلط گفتم ای یار شایسته خوی
&&
که نفع است در آهن و سنگ و روی
\\
چنین آدمی مرده به ننگ را
&&
که بر وی فضیلت بود سنگ را
\\
نه هر آدمی زاده از دد به است
&&
که دد ز آدمی زادهٔ بد به است
\\
به است از دد انسان صاحب خرد
&&
نه انسان که در مردم افتد چو دد
\\
چو انسان نداند به جز خورد و خواب
&&
کدامش فضیلت بود بر دواب؟
\\
سوار نگون بخت بی راهرو
&&
پیاده برد ز او به رفتن گرو
\\
کسی دانهٔ نیکمردی نکاشت
&&
کز او خرمن کام دل برنداشت
\\
نه هرگز شنیدیم در عمر خویش
&&
که بدمرد را نیکی آمد به پیش
\\
\end{longtable}
\end{center}
