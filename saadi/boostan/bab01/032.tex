\begin{center}
\section*{بخش ۳۲ - حکایت در معنی خاموشی از نصیحت کسی که پند نپذیرد: حکایت کنند از جفاگستری}
\label{sec:032}
\addcontentsline{toc}{section}{\nameref{sec:032}}
\begin{longtable}{l p{0.5cm} r}
حکایت کنند از جفاگستری
&&
که فرماندهی داشت بر کشوری
\\
در ایام او روز مردم چو شام
&&
شب از بیم او خواب مردم حرام
\\
همه روز نیکان از او در بلا
&&
به شب دست پاکان از او بر دعا
\\
گروهی بر شیخ آن روزگار
&&
ز دست ستمگر گرستند زار
\\
که ای پیر دانای فرخنده رای
&&
بگوی این جوان را بترس از خدای
\\
بگفتا دریغ آیدم نام دوست
&&
که هر کس نه در خورد پیغام اوست
\\
کسی را که بینی ز حق بر کران
&&
منه با وی، ای خواجه، حق در میان
\\
دریغ است با سفله گفت از علوم
&&
که ضایع شود تخم در شوره بوم
\\
چو در وی نگیرد عدو داندت
&&
برنجد به جان و برنجاندت
\\
تو را عادت، ای پادشه، حق روی است
&&
دل مرد حق گوی از این جا قوی است
\\
نگین خصلتی دارد ای نیکبخت
&&
که در موم گیرد نه در سنگ سخت
\\
عجب نیست گر ظالم از من به جان
&&
برنجد که دزد است و من پاسبان
\\
تو هم پاسبانی به انصاف و داد
&&
که حفظ خدا پاسبان تو باد
\\
تو را نیست منت ز روی قیاس
&&
خداوند را من و فضل و سپاس
\\
که در کار خیرت به خدمت بداشت
&&
نه چون دیگرانت معطل گذاشت
\\
همه کس به میدان کوشش درند
&&
ولی گوی بخشش نه هر کس برند
\\
تو حاصل نکردی به کوشش بهشت
&&
خدا در تو خوی بهشتی بهشت
\\
دلت روشن و وقت مجموع باد
&&
قدم ثابت و پایه مرفوع باد
\\
حیاتت خوش و رفتنت بر صواب
&&
عبادت قبول و دعا مستجاب
\\
\end{longtable}
\end{center}
