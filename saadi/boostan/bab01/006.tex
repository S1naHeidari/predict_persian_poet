\begin{center}
\section*{بخش ۶ - گفتار اندر نظر در حق رعیت مظلوم: تو کی بشنوی نالهٔ دادخواه}
\label{sec:006}
\addcontentsline{toc}{section}{\nameref{sec:006}}
\begin{longtable}{l p{0.5cm} r}
تو کی بشنوی نالهٔ دادخواه
&&
به کیوان برت کلهٔ خوابگاه؟
\\
چنان خسب کآید فغانت به گوش
&&
اگر دادخواهی برآرد خروش
\\
که نالد ز ظالم که در دور توست
&&
که هر جور کاو می‌کند جور توست
\\
نه سگ دامن کاروانی درید
&&
که دهقان نادان که سگ پرورید
\\
دلیر آمدی سعدیا در سخن
&&
چو تیغت به دست است فتحی بکن
\\
بگو آنچه دانی که حق گفته به
&&
نه رشوت ستانی و نه عشوه ده
\\
طمع بند و دفتر ز حکمت بشوی
&&
طمع بگسل و هرچه دانی بگوی
\\
\end{longtable}
\end{center}
