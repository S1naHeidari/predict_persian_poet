\begin{center}
\section*{بخش ۴۰ - گفتار اندر حذر از دشمنی که در طاعت آید: گرت خویش دشمن شود دوستدار}
\label{sec:040}
\addcontentsline{toc}{section}{\nameref{sec:040}}
\begin{longtable}{l p{0.5cm} r}
گرت خویش دشمن شود دوستدار
&&
ز تلبیسش ایمن مشو زینهار
\\
که گردد درونش به کین تو ریش
&&
چو یاد آیدش مهر پیوند خویش
\\
بد اندیش را لفظ شیرین مبین
&&
که ممکن بود زهر در انگبین
\\
کسی جان از آسیب دشمن ببرد
&&
که مر دوستان را به دشمن شمرد
\\
نگه دارد آن شوخ در کیسه در
&&
که بیند همه خلق را کیسه بر
\\
سپاهی که عاصی شود در امیر
&&
ورا تا توانی به خدمت مگیر
\\
ندانست سالار خود را سپاس
&&
تو را هم ندارد، ز غدرش هراس
\\
به سوگند و عهد استوارش مدار
&&
نگهبان پنهان بر او بر گمار
\\
نو آموز را ریسمان کن دراز
&&
نه بگسل که دیگر نبینیش باز
\\
چو اقلیم دشمن به جنگ و حصار
&&
گرفتی، به زندانیانش سپار
\\
که بندی چو دندان به خون در برد
&&
ز حلقوم بیدادگر خون خورد
\\
چو بر کندی از دست دشمن دیار
&&
رعیت به سامان تر از وی بدار
\\
که گر باز کوبد در کارزار
&&
بر آرند عام از دماغش دمار
\\
وگر شهریان را رسانی گزند
&&
در شهر بر روی دشمن مبند
\\
مگو دشمن تیغ زن بر در است
&&
که انباز دشمن به شهر اندر است
\\
\end{longtable}
\end{center}
