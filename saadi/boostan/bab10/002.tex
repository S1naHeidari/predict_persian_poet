\begin{center}
\section*{بخش ۲ - حکایت: سیه چرده‌ای را کسی زشت خواند}
\label{sec:002}
\addcontentsline{toc}{section}{\nameref{sec:002}}
\begin{longtable}{l p{0.5cm} r}
سیه چرده‌ای را کسی زشت خواند
&&
جوابی بگفتش که حیران بماند
\\
نه من صورت خویش خود کرده‌ام
&&
که عیبم شماری که بد کرده‌ام
\\
تو را با من ار زشت رویم چه کار؟
&&
نه آخر منم زشت و زیبا نگار
\\
از آنم که بر سر نبشتی ز پیش
&&
نه کم کردم ای بنده پرور نه بیش
\\
تو دانایی آخر که قادر نیم
&&
توانای مطلق تویی، من کیم؟
\\
گرم ره نمایی رسیدم به خیر
&&
وگر گم کنی باز ماندم ز سیر
\\
جهان آفرین گر نه یاری کند
&&
کجا بنده پرهیزگاری کند؟
\\
چه خوش گفت درویش کوتاه دست
&&
که شب توبه کرد و سحرگه شکست
\\
گر او توبه بخشد بماند درست
&&
که پیمان ما بی ثبات است و سست
\\
به حقت که چشمم ز باطل بدوز
&&
به نورت که فردا به نارم مسوز
\\
ز مسکینیم روی در خاک رفت
&&
غبار گناهم بر افلاک رفت
\\
تو یک نوبت ای ابر رحمت ببار
&&
که در پیش باران نپاید غبار
\\
ز جرمم در این مملکت جاه نیست
&&
ولیکن به ملکی دگر راه نیست
\\
تو دانی ضمیر زبان بستگان
&&
تو مرهم نهی بر دل خستگان
\\
\end{longtable}
\end{center}
