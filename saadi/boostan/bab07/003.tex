\begin{center}
\section*{بخش ۳ - حکایت سلطان تکش و حفظ اسرار: تکش با غلامان یکی راز گفت}
\label{sec:003}
\addcontentsline{toc}{section}{\nameref{sec:003}}
\begin{longtable}{l p{0.5cm} r}
تکش با غلامان یکی راز گفت
&&
که این را نباید به کس باز گفت
\\
به یک سالش آمد ز دل بر دهان
&&
به یک روز شد منتشر در جهان
\\
بفرمود جلاد را بی دریغ
&&
که بر دار سرهای اینان به تیغ
\\
یکی زآن میان گفت و زنهار خواست
&&
مکش بندگان کاین گناه از تو خاست
\\
تو اول نبستی که سرچشمه بود
&&
چو سیلاب شد پیش بستن چه سود؟
\\
تو پیدا مکن راز دل بر کسی
&&
که او خود نگوید بر هر کسی
\\
جواهر به گنجینه داران سپار
&&
ولی راز را خویشتن پاس دار
\\
سخن تا نگویی بر او دست هست
&&
چو گفته شود یابد او بر تو دست
\\
سخن دیو بندی است در چاه دل
&&
به بالای کام و زبانش مهل
\\
توان باز دادن ره نره دیو
&&
ولی باز نتوان گرفتن به ریو
\\
تو دانی که چون دیو رفت از قفس
&&
نیاید به لا حول کس باز پس
\\
یکی طفل بر گیرد از رخش بند
&&
نیاید به صد رستم اندر کمند
\\
مگوی آن که گر بر ملا اوفتد
&&
وجودی از آن در بلا اوفتد
\\
به دهقان نادان چه خوش گفت زن:
&&
به دانش سخن گوی یا دم مزن
\\
مگوی آنچه طاقت نداری شنود
&&
که جو کشته گندم نخواهی درود
\\
چه نیکو زده‌ست این مثل برهمن
&&
بود حرمت هر کس از خویشتن
\\
چو دشنام گویی دعا نشنوی
&&
به جز کشتهٔ خویشتن ندروی
\\
مگوی و منه تا توانی قدم
&&
از اندازه بیرون وز اندازه کم
\\
نباید که بسیار بازی کنی
&&
که مر قیمت خویش را بشکنی
\\
وگر تند باشی به یک بار و تیز
&&
جهان از تو گیرند راه گریز
\\
نه کوتاه دستی و بیچارگی
&&
نه زجر و تطاول به یک‌بارگی
\\
\end{longtable}
\end{center}
