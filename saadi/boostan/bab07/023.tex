\begin{center}
\section*{بخش ۲۳ - گفتار اندر پروردن فرزندان: پسر چون زده بر گذشتش سنین}
\label{sec:023}
\addcontentsline{toc}{section}{\nameref{sec:023}}
\begin{longtable}{l p{0.5cm} r}
پسر چون ز ده بر گذشتش سنین
&&
ز نامحرمان گو فراتر نشین
\\
بر پنبه آتش نشاید فروخت
&&
که تا چشم بر هم زنی خانه سوخت
\\
چو خواهی که نامت بماند به جای
&&
پسر را خردمندی آموز و رای
\\
که گر عقل و طبعش نباشد بسی
&&
بمیری و از تو نماند کسی
\\
بسا روزگارا که سختی برد
&&
پسر چون پدر نازکش پرورد
\\
خردمند و پرهیزگارش بر آر
&&
گرش دوست داری به نازش مدار
\\
به خردی درش زجر و تعلیم کن
&&
به نیک و بدش وعده و بیم کن
\\
نوآموز را ذکر و تحسین و زه
&&
ز توبیخ و تهدید استاد به
\\
بیاموز پرورده را دسترنج
&&
وگر دست داری چو قارون به گنج
\\
مکن تکیه بر دستگاهی که هست
&&
که باشد که نعمت نماند به دست
\\
به پایان رسد کیسهٔ سیم و زر
&&
نگردد تهی کیسهٔ پیشه‌ور
\\
چه دانی که گردیدن روزگار
&&
به غربت بگرداندش در دیار
\\
چو بر پیشه‌ای باشدش دسترس
&&
کجا دست حاجت برد پیش کس؟
\\
ندانی که سعدی مراد از چه یافت؟
&&
نه هامون نوشت و نه دریا شکافت
\\
به خردی بخورد از بزرگان قفا
&&
خدا دادش اندر بزرگی صفا
\\
هر آن کس که گردن به فرمان نهد
&&
بسی بر نیاید که فرمان دهد
\\
هر آن طفل کاو جور آموزگار
&&
نبیند، جفا بیند از روزگار
\\
پسر را نکو دار و راحت رسان
&&
که چشمش نماند به دست کسان
\\
هر آن کس که فرزند را غم نخورد
&&
دگر کس غمش خورد و بدنام کرد
\\
نگه‌دار از آمیزگار بدش
&&
که بدبخت و بی ره کند چون خودش
\\
\end{longtable}
\end{center}
