\begin{center}
\section*{بخش ۱۷ - گفتار اندر کسانی که غیبت ایشان روا باشد: سه کس را شنیدم که غیبت رواست}
\label{sec:017}
\addcontentsline{toc}{section}{\nameref{sec:017}}
\begin{longtable}{l p{0.5cm} r}
سه کس را شنیدم که غیبت رواست
&&
وز این درگذشتی چهارم خطاست
\\
یکی پادشاهی ملامت پسند
&&
کز او بر دل خلق بینی گزند
\\
حلال است از او نقل کردن خبر
&&
مگر خلق باشند از او بر حذر
\\
دوم پرده بر بی حیایی متن
&&
که خود می‌درد پرده بر خویشتن
\\
ز حوضش مدار ای برادر نگاه
&&
که او می‌درافتد به گردن به چاه
\\
سوم کژ ترازوی ناراست خوی
&&
ز فعل بدش هرچه دانی بگوی
\\
\end{longtable}
\end{center}
