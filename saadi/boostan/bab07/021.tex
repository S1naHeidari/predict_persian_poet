\begin{center}
\section*{بخش ۲۱ - گفتار اندر پرورش زنان و ذکر صلاح و فساد ایشان: زن خوب فرمانبر پارسا}
\label{sec:021}
\addcontentsline{toc}{section}{\nameref{sec:021}}
\begin{longtable}{l p{0.5cm} r}
زن خوب فرمانبر پارسا
&&
کند مرد درویش را پادشا
\\
برو پنج نوبت بزن بر درت
&&
چو یاری موافق بود در برت
\\
همه روز اگر غم خوری غم مدار
&&
چو شب غمگسارت بود در کنار
\\
کرا خانه آباد و همخوابه دوست
&&
خدا را به رحمت نظر سوی اوست
\\
چو مستور باشد زن و خوبروی
&&
به دیدار او در بهشت است شوی
\\
کسی بر گرفت از جهان کام دل
&&
که یکدل بود با وی آرام دل
\\
اگر پارسا باشد و خوش سخن
&&
نگه در نکویی و زشتی مکن
\\
زن خوش منش دل نشان تر که خوب
&&
که آمیزگاری بپوشد عیوب
\\
ببرد از پری چهرهٔ زشت خوی
&&
زن دیو سیمای خوش طبع، گوی
\\
چو حلوا خورد سرکه از دست شوی
&&
نه حلوا خورد سرکه اندوده روی
\\
دلارام باشد زن نیک خواه
&&
ولیکن زن بد، خدایا پناه!
\\
چو طوطی کلاغش بود هم نفس
&&
غنیمت شمارد خلاص از قفس
\\
سر اندر جهان نه به آوارگی
&&
وگرنه بنه دل به بیچارگی
\\
تهی پای رفتن به از کفش تنگ
&&
بلای سفر به که در خانه جنگ
\\
به زندان قاضی گرفتار به
&&
که در خانه دیدن بر ابرو گره
\\
سفر عید باشد بر آن کدخدای
&&
که بانوی زشتش بود در سرای
\\
در خرمی بر سرایی ببند
&&
که بانگ زن از وی برآید بلند
\\
چو زن راه بازار گیرد بزن
&&
وگرنه تو در خانه بنشین چو زن
\\
اگر زن ندارد سوی مرد گوش
&&
سراویل کحلیش در مرد پوش
\\
زنی را که جهل است و ناراستی
&&
بلا بر سر خود نه زن خواستی
\\
چو در کیله یک جو امانت شکست
&&
از انبار گندم فرو شوی دست
\\
بر آن بنده حق نیکویی خواسته است
&&
که با او دل و دست زن راست است
\\
چو در روی بیگانه خندید زن
&&
دگر مرد گو لاف مردی مزن
\\
زن شوخ چون دست در قلیه کرد
&&
برو گو بنه پنجه بر روی مرد
\\
ز بیگانگان چشم زن کور باد
&&
چو بیرون شد از خانه در گور باد
\\
چو بینی که زن پای بر جای نیست
&&
ثبات از خردمندی و رای نیست
\\
گریز از کفش در دهان نهنگ
&&
که مردن به از زندگانی به ننگ
\\
بپوشانش از چشم بیگانه روی
&&
وگر نشنود چه زن آنگه چه شوی
\\
زن خوب خوش طبع رنج است و بار
&&
رها کن زن زشت ناسازگار
\\
چه نغز آمد این یک سخن زآن دو تن
&&
که بودند سرگشته از دست زن
\\
یکی گفت کس را زن بد مباد
&&
دگر گفت زن در جهان خود مباد
\\
زن نو کن ای دوست هر نوبهار
&&
که تقویم پاری نیاید بکار
\\
کسی را که بینی گرفتار زن
&&
مکن سعدیا طعنه بر وی مزن
\\
تو هم جور بینی و بارش کشی
&&
اگر یک سحر در کنارش کشی
\\
\end{longtable}
\end{center}
