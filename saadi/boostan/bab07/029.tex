\begin{center}
\section*{بخش ۲۹ - حکایت: چوانی هنرمند فرزانه بود}
\label{sec:029}
\addcontentsline{toc}{section}{\nameref{sec:029}}
\begin{longtable}{l p{0.5cm} r}
جوانی هنرمند فرزانه بود
&&
که در وعظ چالاک و مردانه بود
\\
نکونام و صاحبدل و حق پرست
&&
خط عارضش خوشتر از خط دست
\\
قوی در بلاغات و در نحو چست
&&
ولی حرف ابجد نگفتی درست
\\
یکی را بگفتم ز صاحبدلان
&&
که دندان پیشین ندارد فلان
\\
برآمد ز سودای من سرخ روی
&&
کز این جنس بیهوده دیگر مگوی
\\
تو در وی همان عیب دیدی که هست
&&
ز چندان هنر چشم عقلت ببست
\\
یقین بشنو از من که روز یقین
&&
نبینند بد، مردم نیک بین
\\
یکی را که فضل است و فرهنگ و رای
&&
گرش پای عصمت بخیزد ز جای
\\
به یک خرده مپسند بر وی جفا
&&
بزرگان چه گفتند؟ خذ ما صفا
\\
بود خار و گل با هم ای هوشمند
&&
چه در بند خاری؟ تو گل دسته بند
\\
که را زشت خویی بود در سرشت
&&
نبیند ز طاووس جز پای زشت
\\
صفایی به دست آور ای خیره روی
&&
که ننماید آیینهٔ تیره، روی
\\
طریقی طلب کز عقوبت رهی
&&
نه حرفی که انگشت بر وی نهی
\\
منه عیب خلق ای خردمند پیش
&&
که چشمت فرو دوزد از عیب خویش
\\
چرا دامن آلوده را حد زنم
&&
چو در خود شناسم که تردامنم؟
\\
نشاید که بر کس درشتی کنی
&&
چو خود را به تأویل پشتی کنی
\\
چو بد ناپسند آیدت خود مکن
&&
پس آنگه به همسایه گو بد مکن
\\
من ار حق شناسم وگر خود نمای
&&
برون با تو دارم، درون با خدای
\\
چو ظاهر به عفت بیاراستم
&&
تصرف مکن در کژ و راستم
\\
اگر سیرتم خوب و گر منکر است
&&
خدایم به سر از تو داناتر است
\\
تو خاموش اگر من بهم یا بدم
&&
که حمال سود و زیان خودم
\\
کسی را به کردار بد کن عذاب
&&
که چشم از تو دارد به نیکی ثواب
\\
نکو کاری از مردم نیک رای
&&
یکی را به ده می‌نویسد خدای
\\
تو نیز ای عجب هر که را یک هنر
&&
ببینی، ز ده عیبش اندر گذر
\\
نه یک عیب او را بر انگشت پیچ
&&
جهانی فضیلت بر آور به هیچ
\\
چو دشمن که در شعر سعدی، نگاه
&&
به نفرت کند ز اندرون تباه
\\
ندارد به صد نکتهٔ نغز گوش
&&
چو زحفی ببیند بر آرد خروش
\\
جز این علتش نیست کان بد پسند
&&
حسد دیده نیک بینش بکند
\\
نه مر خلق را صنع باری سرشت؟
&&
سیاه و سپید آمد و خوب و زشت
\\
نه هر چشم و ابرو که بینی نکوست
&&
بخور پسته مغز و بینداز پوست
\\
\end{longtable}
\end{center}
