\begin{center}
\section*{بخش ۲۴ - حکایت: شبی دعوتی بود در کوی من}
\label{sec:024}
\addcontentsline{toc}{section}{\nameref{sec:024}}
\begin{longtable}{l p{0.5cm} r}
شبی دعوتی بود در کوی من
&&
ز هر جنس مردم در او انجمن
\\
چو آواز مطرب در آمد ز کوی
&&
به گردون شد از عاشقان های و هوی
\\
پریچهره‌ای بود محبوب من
&&
بدو گفتم ای لعبت خوب من
\\
چرا با رفیقان نیایی به جمع
&&
که روشن کنی بزم ما را چو شمع؟
\\
شنیدم سهی قامت سیم‌تن
&&
که می‌رفت و می‌گفت با خویشتن
\\
محاسن چو مردان ندارم به دست
&&
نه مردی بود پیش مردان نشست
\\
سیه نامه تر زآن مخنث مخواه
&&
که پیش از خطش روی گردد سیاه
\\
از آن بی حمیت بباید گریخت
&&
که نامردیش آب مردان بریخت
\\
پسر کاو میان قلندر نشست
&&
پدر گو ز خیرش فرو شوی دست
\\
دریغش مخور بر هلاک و تلف
&&
که پیش از پدر مرده به ناخلف
\\
\end{longtable}
\end{center}
