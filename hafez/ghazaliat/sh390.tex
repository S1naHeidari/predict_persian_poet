\begin{center}
\section*{غزل شماره ۳۹۰: افسر سلطان گل پیدا شد از طرف چمن}
\label{sec:sh390}
\addcontentsline{toc}{section}{\nameref{sec:sh390}}
\begin{longtable}{l p{0.5cm} r}
افسر سلطان گل پیدا شد از طرف چمن
&&
مقدمش یا رب مبارک باد بر سرو و سمن
\\
خوش به جای خویشتن بود این نشست خسروی
&&
تا نشیند هر کسی اکنون به جای خویشتن
\\
خاتم جم را بشارت ده به حسن خاتمت
&&
کاسم اعظم کرد از او کوتاه دست اهرمن
\\
تا ابد معمور باد این خانه کز خاک درش
&&
هر نفس با بوی رحمان می‌وزد باد یمن
\\
شوکت پور پشنگ و تیغ عالمگیر او
&&
در همه شهنامه‌ها شد داستان انجمن
\\
خنگ چوگانی چرخت رام شد در زیر زین
&&
شهسوارا چون به میدان آمدی گویی بزن
\\
جویبار ملک را آب روان شمشیر توست
&&
تو درخت عدل بنشان بیخ بدخواهان بکن
\\
بعد از این نشگفت اگر با نکهت خلق خوشت
&&
خیزد از صحرای ایذج نافه مشک ختن
\\
گوشه گیران انتظار جلوه خوش می‌کنند
&&
برشکن طرف کلاه و برقع از رخ برفکن
\\
مشورت با عقل کردم گفت حافظ می بنوش
&&
ساقیا می ده به قول مستشار مؤتمن
\\
ای صبا بر ساقی بزم اتابک عرضه دار
&&
تا از آن جام زرافشان جرعه‌ای بخشد به من
\\
\end{longtable}
\end{center}
