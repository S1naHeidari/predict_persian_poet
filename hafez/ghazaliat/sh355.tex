\begin{center}
\section*{غزل شماره ۳۵۵: حالیا مصلحت وقت در آن می‌بینم}
\label{sec:sh355}
\addcontentsline{toc}{section}{\nameref{sec:sh355}}
\begin{longtable}{l p{0.5cm} r}
حالیا مصلحت وقت در آن می‌بینم
&&
که کشم رخت به میخانه و خوش بنشینم
\\
جام می گیرم و از اهل ریا دور شوم
&&
یعنی از اهل جهان پاکدلی بگزینم
\\
جز صراحی و کتابم نبود یار و ندیم
&&
تا حریفان دغا را به جهان کم بینم
\\
سر به آزادگی از خلق برآرم چون سرو
&&
گر دهد دست که دامن ز جهان درچینم
\\
بس که در خرقه آلوده زدم لاف صلاح
&&
شرمسار از رخ ساقی و می رنگینم
\\
سینه تنگ من و بار غم او هیهات
&&
مرد این بار گران نیست دل مسکینم
\\
من اگر رند خراباتم و گر زاهد شهر
&&
این متاعم که همی‌بینی و کمتر زینم
\\
بنده آصف عهدم دلم از راه مبر
&&
که اگر دم زنم از چرخ بخواهد کینم
\\
بر دلم گرد ستم‌هاست خدایا مپسند
&&
که مکدر شود آیینه مهرآیینم
\\
\end{longtable}
\end{center}
