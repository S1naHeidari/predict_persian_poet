\begin{center}
\section*{غزل شماره ۴۱۴: گلبن عیش می‌دمد ساقی گلعذار کو}
\label{sec:sh414}
\addcontentsline{toc}{section}{\nameref{sec:sh414}}
\begin{longtable}{l p{0.5cm} r}
گلبن عیش می‌دمد ساقی گلعذار کو
&&
باد بهار می‌وزد باده خوشگوار کو
\\
هر گل نو ز گلرخی یاد همی‌کند ولی
&&
گوش سخن شنو کجا دیده اعتبار کو
\\
مجلس بزم عیش را غالیه مراد نیست
&&
ای دم صبح خوش نفس نافه زلف یار کو
\\
حسن فروشی گلم نیست تحمل ای صبا
&&
دست زدم به خون دل بهر خدا نگار کو
\\
شمع سحرگهی اگر لاف ز عارض تو زد
&&
خصم زبان دراز شد خنجر آبدار کو
\\
گفت مگر ز لعل من بوسه نداری آرزو
&&
مردم از این هوس ولی قدرت و اختیار کو
\\
حافظ اگر چه در سخن خازن گنج حکمت است
&&
از غم روزگار دون طبع سخن گزار کو
\\
\end{longtable}
\end{center}
