\begin{center}
\section*{غزل شماره ۸۸: شنیده‌ام سخنی خوش که پیر کنعان گفت}
\label{sec:sh088}
\addcontentsline{toc}{section}{\nameref{sec:sh088}}
\begin{longtable}{l p{0.5cm} r}
شنیده‌ام سخنی خوش که پیر کنعان گفت
&&
فراق یار نه آن می‌کند که بتوان گفت
\\
حدیث هول قیامت که گفت واعظ شهر
&&
کنایتیست که از روزگار هجران گفت
\\
نشان یار سفرکرده از که پرسم باز
&&
که هر چه گفت برید صبا پریشان گفت
\\
فغان که آن مه نامهربان مهرگسل
&&
به ترک صحبت یاران خود چه آسان گفت
\\
من و مقام رضا بعد از این و شکر رقیب
&&
که دل به درد تو خو کرد و ترک درمان گفت
\\
غم کهن به می سالخورده دفع کنید
&&
که تخم خوشدلی این است پیر دهقان گفت
\\
گره به باد مزن گر چه بر مراد رود
&&
که این سخن به مثل باد با سلیمان گفت
\\
به مهلتی که سپهرت دهد ز راه مرو
&&
تو را که گفت که این زال ترک دستان گفت
\\
مزن ز چون و چرا دم که بنده مقبل
&&
قبول کرد به جان هر سخن که جانان گفت
\\
که گفت حافظ از اندیشه تو آمد باز
&&
من این نگفته‌ام آن کس که گفت بهتان گفت
\\
\end{longtable}
\end{center}
