\begin{center}
\section*{غزل شماره ۴۸۱: بشنو این نکته که خود را ز غم آزاده کنی}
\label{sec:sh481}
\addcontentsline{toc}{section}{\nameref{sec:sh481}}
\begin{longtable}{l p{0.5cm} r}
بشنو این نکته که خود را ز غم آزاده کنی
&&
خون خوری گر طلب روزی ننهاده کنی
\\
آخرالامر گل کوزه گران خواهی شد
&&
حالیا فکر سبو کن که پر از باده کنی
\\
گر از آن آدمیانی که بهشتت هوس است
&&
عیش با آدمی ای چند پری زاده کنی
\\
تکیه بر جای بزرگان نتوان زد به گزاف
&&
مگر اسباب بزرگی همه آماده کنی
\\
اجرها باشدت ای خسرو شیرین دهنان
&&
گر نگاهی سوی فرهاد دل افتاده کنی
\\
خاطرت کی رقم فیض پذیرد هیهات
&&
مگر از نقش پراگنده ورق ساده کنی
\\
کار خود گر به کرم بازگذاری حافظ
&&
ای بسا عیش که با بخت خداداده کنی
\\
ای صبا بندگی خواجه جلال الدین کن
&&
که جهان پرسمن و سوسن آزاده کنی
\\
\end{longtable}
\end{center}
