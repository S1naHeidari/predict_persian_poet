\begin{center}
\section*{غزل شماره ۲۱۱: دوش می‌آمد و رخساره برافروخته بود}
\label{sec:sh211}
\addcontentsline{toc}{section}{\nameref{sec:sh211}}
\begin{longtable}{l p{0.5cm} r}
دوش می‌آمد و رخساره برافروخته بود
&&
تا کجا باز دل غمزده‌ای سوخته بود
\\
رسم عاشق کشی و شیوه شهرآشوبی
&&
جامه‌ای بود که بر قامت او دوخته بود
\\
جان عشاق سپند رخ خود می‌دانست
&&
و آتش چهره بدین کار برافروخته بود
\\
گر چه می‌گفت که زارت بکشم می‌دیدم
&&
که نهانش نظری با من دلسوخته بود
\\
کفر زلفش ره دین می‌زد و آن سنگین دل
&&
در پی اش مشعلی از چهره برافروخته بود
\\
دل بسی خون به کف آورد ولی دیده بریخت
&&
الله الله که تلف کرد و که اندوخته بود
\\
یار مفروش به دنیا که بسی سود نکرد
&&
آن که یوسف به زر ناسره بفروخته بود
\\
گفت و خوش گفت برو خرقه بسوزان حافظ
&&
یا رب این قلب شناسی ز که آموخته بود
\\
\end{longtable}
\end{center}
