\begin{center}
\section*{غزل شماره ۴۰۶: گفتا برون شدی به تماشای ماه نو}
\label{sec:sh406}
\addcontentsline{toc}{section}{\nameref{sec:sh406}}
\begin{longtable}{l p{0.5cm} r}
گفتا برون شدی به تماشای ماه نو
&&
از ماه ابروان منت شرم باد رو
\\
عمریست تا دلت ز اسیران زلف ماست
&&
غافل ز حفظ جانب یاران خود مشو
\\
مفروش عطر عقل به هندوی زلف ما
&&
کان جا هزار نافه مشکین به نیم جو
\\
تخم وفا و مهر در این کهنه کشته زار
&&
آن گه عیان شود که بود موسم درو
\\
ساقی بیار باده که رمزی بگویمت
&&
از سر اختران کهن سیر و ماه نو
\\
شکل هلال هر سر مه می‌دهد نشان
&&
از افسر سیامک و ترک کلاه زو
\\
حافظ جناب پیر مغان مامن وفاست
&&
درس حدیث عشق بر او خوان و ز او شنو
\\
\end{longtable}
\end{center}
