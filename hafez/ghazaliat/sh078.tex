\begin{center}
\section*{غزل شماره ۷۸: دیدی که یار جز سر جور و ستم نداشت}
\label{sec:sh078}
\addcontentsline{toc}{section}{\nameref{sec:sh078}}
\begin{longtable}{l p{0.5cm} r}
دیدی که یار جز سر جور و ستم نداشت
&&
بشکست عهد وز غم ما هیچ غم نداشت
\\
یا رب مگیرش ار چه دل چون کبوترم
&&
افکند و کشت و عزت صید حرم نداشت
\\
بر من جفا ز بخت من آمد وگرنه یار
&&
حاشا که رسم لطف و طریق کرم نداشت
\\
با این همه هر آن که نه خواری کشید از او
&&
هر جا که رفت هیچ کسش محترم نداشت
\\
ساقی بیار باده و با محتسب بگو
&&
انکار ما مکن که چنین جام جم نداشت
\\
هر راهرو که ره به حریم درش نبرد
&&
مسکین برید وادی و ره در حرم نداشت
\\
حافظ ببر تو گوی فصاحت که مدعی
&&
هیچش هنر نبود و خبر نیز هم نداشت
\\
\end{longtable}
\end{center}
