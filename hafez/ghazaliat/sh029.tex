\begin{center}
\section*{غزل شماره ۲۹: ما را ز خیال تو چه پروای شراب است}
\label{sec:sh029}
\addcontentsline{toc}{section}{\nameref{sec:sh029}}
\begin{longtable}{l p{0.5cm} r}
ما را ز خیال تو چه پروای شراب است
&&
خم گو سر خود گیر که خمخانه خراب است
\\
گر خمر بهشت است بریزید که بی دوست
&&
هر شربت عذبم که دهی عین عذاب است
\\
افسوس که شد دلبر و در دیده گریان
&&
تحریر خیال خط او نقش بر آب است
\\
بیدار شو ای دیده که ایمن نتوان بود
&&
زین سیل دمادم که در این منزل خواب است
\\
معشوق عیان می‌گذرد بر تو ولیکن
&&
اغیار همی‌بیند از آن بسته نقاب است
\\
گل بر رخ رنگین تو تا لطف عرق دید
&&
در آتش شوق از غم دل غرق گلاب است
\\
سبز است در و دشت بیا تا نگذاریم
&&
دست از سر آبی که جهان جمله سراب است
\\
در کنج دماغم مطلب جای نصیحت
&&
کاین گوشه پر از زمزمه چنگ و رباب است
\\
حافظ چه شد ار عاشق و رند است و نظرباز
&&
بس طور عجب لازم ایام شباب است
\\
\end{longtable}
\end{center}
