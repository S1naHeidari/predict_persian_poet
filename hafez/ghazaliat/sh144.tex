\begin{center}
\section*{غزل شماره ۱۴۴: به سر جام جم آن گه نظر توانی کرد}
\label{sec:sh144}
\addcontentsline{toc}{section}{\nameref{sec:sh144}}
\begin{longtable}{l p{0.5cm} r}
به سر جام جم آن گه نظر توانی کرد
&&
که خاک میکده کحل بصر توانی کرد
\\
مباش بی می و مطرب که زیر طاق سپهر
&&
بدین ترانه غم از دل به در توانی کرد
\\
گل مراد تو آن گه نقاب بگشاید
&&
که خدمتش چو نسیم سحر توانی کرد
\\
گدایی در میخانه طرفه اکسیریست
&&
گر این عمل بکنی خاک زر توانی کرد
\\
به عزم مرحله عشق پیش نه قدمی
&&
که سودها کنی ار این سفر توانی کرد
\\
تو کز سرای طبیعت نمی‌روی بیرون
&&
کجا به کوی طریقت گذر توانی کرد
\\
جمال یار ندارد نقاب و پرده ولی
&&
غبار ره بنشان تا نظر توانی کرد
\\
بیا که چاره ذوق حضور و نظم امور
&&
به فیض بخشی اهل نظر توانی کرد
\\
ولی تو تا لب معشوق و جام می خواهی
&&
طمع مدار که کار دگر توانی کرد
\\
دلا ز نور هدایت گر آگهی یابی
&&
چو شمع خنده زنان ترک سر توانی کرد
\\
گر این نصیحت شاهانه بشنوی حافظ
&&
به شاهراه حقیقت گذر توانی کرد
\\
\end{longtable}
\end{center}
