\begin{center}
\section*{غزل شماره ۸۶: ساقی بیا که یار ز رخ پرده برگرفت}
\label{sec:sh086}
\addcontentsline{toc}{section}{\nameref{sec:sh086}}
\begin{longtable}{l p{0.5cm} r}
ساقی بیا که یار ز رخ پرده برگرفت
&&
کار چراغ خلوتیان باز درگرفت
\\
آن شمع سرگرفته دگر چهره برفروخت
&&
وین پیر سالخورده جوانی ز سر گرفت
\\
آن عشوه داد عشق که مفتی ز ره برفت
&&
وان لطف کرد دوست که دشمن حذر گرفت
\\
زنهار از آن عبارت شیرین دلفریب
&&
گویی که پسته تو سخن در شکر گرفت
\\
بار غمی که خاطر ما خسته کرده بود
&&
عیسی دمی خدا بفرستاد و برگرفت
\\
هر سروقد که بر مه و خور حسن می‌فروخت
&&
چون تو درآمدی پی کاری دگر گرفت
\\
زین قصه هفت گنبد افلاک پرصداست
&&
کوته نظر ببین که سخن مختصر گرفت
\\
حافظ تو این سخن ز که آموختی که بخت
&&
تعویذ کرد شعر تو را و به زر گرفت
\\
\end{longtable}
\end{center}
