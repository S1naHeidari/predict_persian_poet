\begin{center}
\section*{غزل شماره ۱۶۸: گداخت جان که شود کار دل تمام و نشد}
\label{sec:sh168}
\addcontentsline{toc}{section}{\nameref{sec:sh168}}
\begin{longtable}{l p{0.5cm} r}
گداخت جان که شود کار دل تمام و نشد
&&
بسوختیم در این آرزوی خام و نشد
\\
به لابه گفت شبی میر مجلس تو شوم
&&
شدم به رغبت خویشش کمین غلام و نشد
\\
پیام داد که خواهم نشست با رندان
&&
بشد به رندی و دردی کشیم نام و نشد
\\
رواست در بر اگر می‌تپد کبوتر دل
&&
که دید در ره خود تاب و پیچ دام و نشد
\\
بدان هوس که به مستی ببوسم آن لب لعل
&&
چه خون که در دلم افتاد همچو جام و نشد
\\
به کوی عشق منه بی‌دلیل راه قدم
&&
که من به خویش نمودم صد اهتمام و نشد
\\
فغان که در طلب گنج نامه مقصود
&&
شدم خراب جهانی ز غم تمام و نشد
\\
دریغ و درد که در جست و جوی گنج حضور
&&
بسی شدم به گدایی بر کرام و نشد
\\
هزار حیله برانگیخت حافظ از سر فکر
&&
در آن هوس که شود آن نگار رام و نشد
\\
\end{longtable}
\end{center}
