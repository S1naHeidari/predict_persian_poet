\begin{center}
\section*{غزل شماره ۱۹۸: گفتم کی ام دهان و لبت کامران کنند}
\label{sec:sh198}
\addcontentsline{toc}{section}{\nameref{sec:sh198}}
\begin{longtable}{l p{0.5cm} r}
گفتم کی ام دهان و لبت کامران کنند
&&
گفتا به چشم هر چه تو گویی چنان کنند
\\
گفتم خراج مصر طلب می‌کند لبت
&&
گفتا در این معامله کمتر زیان کنند
\\
گفتم به نقطه دهنت خود که برد راه
&&
گفت این حکایتیست که با نکته دان کنند
\\
گفتم صنم پرست مشو با صمد نشین
&&
گفتا به کوی عشق هم این و هم آن کنند
\\
گفتم هوای میکده غم می‌برد ز دل
&&
گفتا خوش آن کسان که دلی شادمان کنند
\\
گفتم شراب و خرقه نه آیین مذهب است
&&
گفت این عمل به مذهب پیر مغان کنند
\\
گفتم ز لعل نوش لبان پیر را چه سود
&&
گفتا به بوسه شکرینش جوان کنند
\\
گفتم که خواجه کی به سر حجله می‌رود
&&
گفت آن زمان که مشتری و مه قران کنند
\\
گفتم دعای دولت او ورد حافظ است
&&
گفت این دعا ملایک هفت آسمان کنند
\\
\end{longtable}
\end{center}
