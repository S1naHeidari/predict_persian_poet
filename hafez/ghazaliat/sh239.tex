\begin{center}
\section*{غزل شماره ۲۳۹: رسید مژده که آمد بهار و سبزه دمید}
\label{sec:sh239}
\addcontentsline{toc}{section}{\nameref{sec:sh239}}
\begin{longtable}{l p{0.5cm} r}
رسید مژده که آمد بهار و سبزه دمید
&&
وظیفه گر برسد مصرفش گل است و نبید
\\
صفیر مرغ برآمد بط شراب کجاست
&&
فغان فتاد به بلبل نقاب گل که کشید
\\
ز میوه‌های بهشتی چه ذوق دریابد
&&
هر آن که سیب زنخدان شاهدی نگزید
\\
مکن ز غصه شکایت که در طریق طلب
&&
به راحتی نرسید آن که زحمتی نکشید
\\
ز روی ساقی مه وش گلی بچین امروز
&&
که گرد عارض بستان خط بنفشه دمید
\\
چنان کرشمه ساقی دلم ز دست ببرد
&&
که با کسی دگرم نیست برگ گفت و شنید
\\
من این مرقع رنگین چو گل بخواهم سوخت
&&
که پیر باده فروشش به جرعه‌ای نخرید
\\
بهار می‌گذرد دادگسترا دریاب
&&
که رفت موسم و حافظ هنوز می نچشید
\\
\end{longtable}
\end{center}
