\begin{center}
\section*{غزل شماره ۱۵۰: ساقی ار باده از این دست به جام اندازد}
\label{sec:sh150}
\addcontentsline{toc}{section}{\nameref{sec:sh150}}
\begin{longtable}{l p{0.5cm} r}
ساقی ار باده از این دست به جام اندازد
&&
عارفان را همه در شرب مدام اندازد
\\
ور چنین زیر خم زلف نهد دانه خال
&&
ای بسا مرغ خرد را که به دام اندازد
\\
ای خوشا دولت آن مست که در پای حریف
&&
سر و دستار نداند که کدام اندازد
\\
زاهد خام که انکار می و جام کند
&&
پخته گردد چو نظر بر می خام اندازد
\\
روز در کسب هنر کوش که می خوردن روز
&&
دل چون آینه در زنگ ظلام اندازد
\\
آن زمان وقت می صبح فروغ است که شب
&&
گرد خرگاه افق پرده شام اندازد
\\
باده با محتسب شهر ننوشی زنهار
&&
بخورد باده‌ات و سنگ به جام اندازد
\\
حافظا سر ز کله گوشه خورشید برآر
&&
بختت ار قرعه بدان ماه تمام اندازد
\\
\end{longtable}
\end{center}
