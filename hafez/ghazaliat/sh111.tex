\begin{center}
\section*{غزل شماره ۱۱۱: عکس روی تو چو در آینه جام افتاد}
\label{sec:sh111}
\addcontentsline{toc}{section}{\nameref{sec:sh111}}
\begin{longtable}{l p{0.5cm} r}
عکس روی تو چو در آینه جام افتاد
&&
عارف از خنده می در طمع خام افتاد
\\
حسن روی تو به یک جلوه که در آینه کرد
&&
این همه نقش در آیینه اوهام افتاد
\\
این همه عکس می و نقش نگارین که نمود
&&
یک فروغ رخ ساقیست که در جام افتاد
\\
غیرت عشق زبان همه خاصان ببرید
&&
کز کجا سر غمش در دهن عام افتاد
\\
من ز مسجد به خرابات نه خود افتادم
&&
اینم از عهد ازل حاصل فرجام افتاد
\\
چه کند کز پی دوران نرود چون پرگار
&&
هر که در دایره گردش ایام افتاد
\\
در خم زلف تو آویخت دل از چاه زنخ
&&
آه کز چاه برون آمد و در دام افتاد
\\
آن شد ای خواجه که در صومعه بازم بینی
&&
کار ما با رخ ساقی و لب جام افتاد
\\
زیر شمشیر غمش رقص کنان باید رفت
&&
کان که شد کشته او نیک سرانجام افتاد
\\
هر دمش با من دلسوخته لطفی دگر است
&&
این گدا بین که چه شایسته انعام افتاد
\\
صوفیان جمله حریفند و نظرباز ولی
&&
زین میان حافظ دلسوخته بدنام افتاد
\\
\end{longtable}
\end{center}
