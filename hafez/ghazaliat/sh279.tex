\begin{center}
\section*{غزل شماره ۲۷۹: خوشا شیراز و وضع بی‌مثالش}
\label{sec:sh279}
\addcontentsline{toc}{section}{\nameref{sec:sh279}}
\begin{longtable}{l p{0.5cm} r}
خوشا شیراز و وضع بی‌مثالش
&&
خداوندا نگه دار از زوالش
\\
ز رکن آباد ما صد لوحش الله
&&
که عمر خضر می‌بخشد زلالش
\\
میان جعفرآباد و مصلا
&&
عبیرآمیز می‌آید شمالش
\\
به شیراز آی و فیض روح قدسی
&&
بجوی از مردم صاحب کمالش
\\
که نام قند مصری برد آنجا
&&
که شیرینان ندادند انفعالش
\\
صبا زان لولی شنگول سرمست
&&
چه داری آگهی چون است حالش
\\
گر آن شیرین پسر خونم بریزد
&&
دلا چون شیر مادر کن حلالش
\\
مکن از خواب بیدارم خدا را
&&
که دارم خلوتی خوش با خیالش
\\
چرا حافظ چو می‌ترسیدی از هجر
&&
نکردی شکر ایام وصالش
\\
\end{longtable}
\end{center}
