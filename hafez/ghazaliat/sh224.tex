\begin{center}
\section*{غزل شماره ۲۲۴: خوشا دلی که مدام از پی نظر نرود}
\label{sec:sh224}
\addcontentsline{toc}{section}{\nameref{sec:sh224}}
\begin{longtable}{l p{0.5cm} r}
خوشا دلی که مدام از پی نظر نرود
&&
به هر درش که بخوانند بی‌خبر نرود
\\
طمع در آن لب شیرین نکردنم اولی
&&
ولی چگونه مگس از پی شکر نرود
\\
سواد دیده غمدیده‌ام به اشک مشوی
&&
که نقش خال توام هرگز از نظر نرود
\\
ز من چو باد صبا بوی خود دریغ مدار
&&
چرا که بی سر زلف توام به سر نرود
\\
دلا مباش چنین هرزه گرد و هرجایی
&&
که هیچ کار ز پیشت بدین هنر نرود
\\
مکن به چشم حقارت نگاه در من مست
&&
که آبروی شریعت بدین قدر نرود
\\
من گدا هوس سروقامتی دارم
&&
که دست در کمرش جز به سیم و زر نرود
\\
تو کز مکارم اخلاق عالمی دگری
&&
وفای عهد من از خاطرت به در نرود
\\
سیاه نامه‌تر از خود کسی نمی‌بینم
&&
چگونه چون قلمم دود دل به سر نرود
\\
به تاج هدهدم از ره مبر که باز سفید
&&
چو باشه در پی هر صید مختصر نرود
\\
بیار باده و اول به دست حافظ ده
&&
به شرط آن که ز مجلس سخن به در نرود
\\
\end{longtable}
\end{center}
