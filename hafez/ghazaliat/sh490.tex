\begin{center}
\section*{غزل شماره ۴۹۰: در همه دیر مغان نیست چو من شیدایی}
\label{sec:sh490}
\addcontentsline{toc}{section}{\nameref{sec:sh490}}
\begin{longtable}{l p{0.5cm} r}
در همه دیر مغان نیست چو من شیدایی
&&
خرقه جایی گرو باده و دفتر جایی
\\
دل که آیینه شاهیست غباری دارد
&&
از خدا می‌طلبم صحبت روشن رایی
\\
کرده‌ام توبه به دست صنم باده فروش
&&
که دگر می نخورم بی رخ بزم آرایی
\\
نرگس ار لاف زد از شیوه چشم تو مرنج
&&
نروند اهل نظر از پی نابینایی
\\
شرح این قصه مگر شمع برآرد به زبان
&&
ور نه پروانه ندارد به سخن پروایی
\\
جوی‌ها بسته‌ام از دیده به دامان که مگر
&&
در کنارم بنشانند سهی بالایی
\\
کشتی باده بیاور که مرا بی رخ دوست
&&
گشت هر گوشه چشم از غم دل دریایی
\\
سخن غیر مگو با من معشوقه پرست
&&
کز وی و جام می‌ام نیست به کس پروایی
\\
این حدیثم چه خوش آمد که سحرگه می‌گفت
&&
بر در میکده‌ای با دف و نی ترسایی
\\
گر مسلمانی از این است که حافظ دارد
&&
آه اگر از پی امروز بود فردایی
\\
\end{longtable}
\end{center}
