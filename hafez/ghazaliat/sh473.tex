\begin{center}
\section*{غزل شماره ۴۷۳: وقت را غنیمت دان آن قدر که بتوانی}
\label{sec:sh473}
\addcontentsline{toc}{section}{\nameref{sec:sh473}}
\begin{longtable}{l p{0.5cm} r}
وقت را غنیمت دان آن قدر که بتوانی
&&
حاصل از حیات ای جان این دم است تا دانی
\\
کام بخشی گردون عمر در عوض دارد
&&
جهد کن که از دولت داد عیش بستانی
\\
باغبان چو من زین جا بگذرم حرامت باد
&&
گر به جای من سروی غیر دوست بنشانی
\\
زاهد پشیمان را ذوق باده خواهد کشت
&&
عاقلا مکن کاری کآورد پشیمانی
\\
محتسب نمی‌داند این قدر که صوفی را
&&
جنس خانگی باشد همچو لعل رمانی
\\
با دعای شبخیزان ای شکردهان مستیز
&&
در پناه یک اسم است خاتم سلیمانی
\\
پند عاشقان بشنو و از در طرب بازآ
&&
کاین همه نمی‌ارزد شغل عالم فانی
\\
یوسف عزیزم رفت ای برادران رحمی
&&
کز غمش عجب بینم حال پیر کنعانی
\\
پیش زاهد از رندی دم مزن که نتوان گفت
&&
با طبیب نامحرم حال درد پنهانی
\\
می‌روی و مژگانت خون خلق می‌ریزد
&&
تیز می‌روی جانا ترسمت فرومانی
\\
دل ز ناوک چشمت گوش داشتم لیکن
&&
ابروی کماندارت می‌برد به پیشانی
\\
جمع کن به احسانی حافظ پریشان را
&&
ای شکنج گیسویت مجمع پریشانی
\\
گر تو فارغی از ما ای نگار سنگین دل
&&
حال خود بخواهم گفت پیش آصف ثانی
\\
\end{longtable}
\end{center}
