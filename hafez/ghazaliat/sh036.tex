\begin{center}
\section*{غزل شماره ۳۶: تا سر زلف تو در دست نسیم افتادست}
\label{sec:sh036}
\addcontentsline{toc}{section}{\nameref{sec:sh036}}
\begin{longtable}{l p{0.5cm} r}
تا سر زلف تو در دست نسیم افتادست
&&
دل سودازده از غصه دو نیم افتادست
\\
چشم جادوی تو خود عین سواد سحر است
&&
لیکن این هست که این نسخه سقیم افتادست
\\
در خم زلف تو آن خال سیه دانی چیست
&&
نقطه دوده که در حلقه جیم افتادست
\\
زلف مشکین تو در گلشن فردوس عذار
&&
چیست طاووس که در باغ نعیم افتادست
\\
دل من در هوس روی تو ای مونس جان
&&
خاک راهیست که در دست نسیم افتادست
\\
همچو گرد این تن خاکی نتواند برخاست
&&
از سر کوی تو زان رو که عظیم افتادست
\\
سایه قد تو بر قالبم ای عیسی دم
&&
عکس روحیست که بر عظم رمیم افتادست
\\
آن که جز کعبه مقامش نبد از یاد لبت
&&
بر در میکده دیدم که مقیم افتادست
\\
حافظ گمشده را با غمت ای یار عزیز
&&
اتحادیست که در عهد قدیم افتادست
\\
\end{longtable}
\end{center}
