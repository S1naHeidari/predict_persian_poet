\begin{center}
\section*{غزل شماره ۴۲۹: ساقی بیا که شد قدح لاله پر ز می}
\label{sec:sh429}
\addcontentsline{toc}{section}{\nameref{sec:sh429}}
\begin{longtable}{l p{0.5cm} r}
ساقی بیا که شد قدح لاله پر ز می
&&
طامات تا به چند و خرافات تا به کی
\\
بگذر ز کبر و ناز که دیده‌ست روزگار
&&
چین قبای قیصر و طرف کلاه کی
\\
هشیار شو که مرغ چمن مست گشت هان
&&
بیدار شو که خواب عدم در پی است هی
\\
خوش نازکانه می‌چمی ای شاخ نوبهار
&&
کآشفتگی مبادت از آشوب باد دی
\\
بر مهر چرخ و شیوه او اعتماد نیست
&&
ای وای بر کسی که شد ایمن ز مکر وی
\\
فردا شراب کوثر و حور از برای ماست
&&
و امروز نیز ساقی مه روی و جام می
\\
باد صبا ز عهد صبی یاد می‌دهد
&&
جان دارویی که غم ببرد درده ای صبی
\\
حشمت مبین و سلطنت گل که بسپرد
&&
فراش باد هر ورقش را به زیر پی
\\
درده به یاد حاتم طی جام یک منی
&&
تا نامه سیاه بخیلان کنیم طی
\\
زان می که داد حسن و لطافت به ارغوان
&&
بیرون فکند لطف مزاج از رخش به خوی
\\
مسند به باغ بر که به خدمت چو بندگان
&&
استاده است سرو و کمر بسته است نی
\\
حافظ حدیث سحرفریب خوشت رسید
&&
تا حد مصر و چین و به اطراف روم و ری
\\
\end{longtable}
\end{center}
