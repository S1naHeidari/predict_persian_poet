\begin{center}
\section*{غزل شماره ۴۲۵: دامن کشان همی‌شد در شرب زرکشیده}
\label{sec:sh425}
\addcontentsline{toc}{section}{\nameref{sec:sh425}}
\begin{longtable}{l p{0.5cm} r}
دامن کشان همی‌شد در شرب زرکشیده
&&
صد ماه رو ز رشکش جیب قصب دریده
\\
از تاب آتش می بر گرد عارضش خوی
&&
چون قطره‌های شبنم بر برگ گل چکیده
\\
لفظی فصیح شیرین قدی بلند چابک
&&
رویی لطیف زیبا چشمی خوش کشیده
\\
یاقوت جان فزایش از آب لطف زاده
&&
شمشاد خوش خرامش در ناز پروریده
\\
آن لعل دلکشش بین وان خنده دل آشوب
&&
وان رفتن خوشش بین وان گام آرمیده
\\
آن آهوی سیه چشم از دام ما برون شد
&&
یاران چه چاره سازم با این دل رمیده
\\
زنهار تا توانی اهل نظر میازار
&&
دنیا وفا ندارد ای نور هر دو دیده
\\
تا کی کشم عتیبت از چشم دلفریبت
&&
روزی کرشمه‌ای کن ای یار برگزیده
\\
گر خاطر شریفت رنجیده شد ز حافظ
&&
بازآ که توبه کردیم از گفته و شنیده
\\
بس شکر بازگویم در بندگی خواجه
&&
گر اوفتد به دستم آن میوه رسیده
\\
\end{longtable}
\end{center}
