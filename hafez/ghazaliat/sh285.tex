\begin{center}
\section*{غزل شماره ۲۸۵: در عهد پادشاه خطابخش جرم پوش}
\label{sec:sh285}
\addcontentsline{toc}{section}{\nameref{sec:sh285}}
\begin{longtable}{l p{0.5cm} r}
در عهد پادشاه خطابخش جرم پوش
&&
حافظ قرابه کش شد و مفتی پیاله نوش
\\
صوفی ز کنج صومعه با پای خم نشست
&&
تا دید محتسب که سبو می‌کشد به دوش
\\
احوال شیخ و قاضی و شرب الیهودشان
&&
کردم سؤال صبحدم از پیر می فروش
\\
گفتا نه گفتنیست سخن گر چه محرمی
&&
درکش زبان و پرده نگه دار و می بنوش
\\
ساقی بهار می‌رسد و وجه می نماند
&&
فکری بکن که خون دل آمد ز غم به جوش
\\
عشق است و مفلسی و جوانی و نوبهار
&&
عذرم پذیر و جرم به ذیل کرم بپوش
\\
تا چند همچو شمع زبان آوری کنی
&&
پروانه مراد رسید ای محب خموش
\\
ای پادشاه صورت و معنی که مثل تو
&&
نادیده هیچ دیده و نشنیده هیچ گوش
\\
چندان بمان که خرقه ازرق کند قبول
&&
بخت جوانت از فلک پیر ژنده پوش
\\
\end{longtable}
\end{center}
