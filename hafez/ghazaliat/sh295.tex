\begin{center}
\section*{غزل شماره ۲۹۵: سحر به بوی گلستان دمی شدم در باغ}
\label{sec:sh295}
\addcontentsline{toc}{section}{\nameref{sec:sh295}}
\begin{longtable}{l p{0.5cm} r}
سحر به بوی گلستان دمی شدم در باغ
&&
که تا چو بلبل بیدل کنم علاج دماغ
\\
به جلوه گل سوری نگاه می‌کردم
&&
که بود در شب تیره به روشنی چو چراغ
\\
چنان به حسن و جوانی خویشتن مغرور
&&
که داشت از دل بلبل هزار گونه فراغ
\\
گشاده نرگس رعنا ز حسرت آب از چشم
&&
نهاده لاله ز سودا به جان و دل صد داغ
\\
زبان کشیده چو تیغی به سرزنش سوسن
&&
دهان گشاده شقایق چو مردم ایغاغ
\\
یکی چو باده پرستان صراحی اندر دست
&&
یکی چو ساقی مستان به کف گرفته ایاغ
\\
نشاط و عیش و جوانی چو گل غنیمت دان
&&
که حافظا نبود بر رسول غیر بلاغ
\\
\end{longtable}
\end{center}
