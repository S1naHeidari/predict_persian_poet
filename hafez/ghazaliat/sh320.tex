\begin{center}
\section*{غزل شماره ۳۲۰: دیشب به سیل اشک ره خواب می‌زدم}
\label{sec:sh320}
\addcontentsline{toc}{section}{\nameref{sec:sh320}}
\begin{longtable}{l p{0.5cm} r}
دیشب به سیل اشک ره خواب می‌زدم
&&
نقشی به یاد خط تو بر آب می‌زدم
\\
ابروی یار در نظر و خرقه سوخته
&&
جامی به یاد گوشه محراب می‌زدم
\\
هر مرغ فکر کز سر شاخ سخن بجست
&&
بازش ز طره تو به مضراب می‌زدم
\\
روی نگار در نظرم جلوه می‌نمود
&&
وز دور بوسه بر رخ مهتاب می‌زدم
\\
چشمم به روی ساقی و گوشم به قول چنگ
&&
فالی به چشم و گوش در این باب می‌زدم
\\
نقش خیال روی تو تا وقت صبحدم
&&
بر کارگاه دیده بی‌خواب می‌زدم
\\
ساقی به صوت این غزلم کاسه می‌گرفت
&&
می‌گفتم این سرود و می ناب می‌زدم
\\
خوش بود وقت حافظ و فال مراد و کام
&&
بر نام عمر و دولت احباب می‌زدم
\\
\end{longtable}
\end{center}
