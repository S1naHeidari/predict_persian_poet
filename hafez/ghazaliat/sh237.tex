\begin{center}
\section*{غزل شماره ۲۳۷: نفس برآمد و کام از تو بر نمی‌آید}
\label{sec:sh237}
\addcontentsline{toc}{section}{\nameref{sec:sh237}}
\begin{longtable}{l p{0.5cm} r}
نفس برآمد و کام از تو بر نمی‌آید
&&
فغان که بخت من از خواب در نمی‌آید
\\
صبا به چشم من انداخت خاکی از کویش
&&
که آب زندگیم در نظر نمی‌آید
\\
قد بلند تو را تا به بر نمی‌گیرم
&&
درخت کام و مرادم به بر نمی‌آید
\\
مگر به روی دلارای یار ما ور نی
&&
به هیچ وجه دگر کار بر نمی‌آید
\\
مقیم زلف تو شد دل که خوش سوادی دید
&&
وز آن غریب بلاکش خبر نمی‌آید
\\
ز شست صدق گشادم هزار تیر دعا
&&
ولی چه سود یکی کارگر نمی‌آید
\\
بسم حکایت دل هست با نسیم سحر
&&
ولی به بخت من امشب سحر نمی‌آید
\\
در این خیال به سر شد زمان عمر و هنوز
&&
بلای زلف سیاهت به سر نمی‌آید
\\
ز بس که شد دل حافظ رمیده از همه کس
&&
کنون ز حلقه زلفت به در نمی‌آید
\\
\end{longtable}
\end{center}
