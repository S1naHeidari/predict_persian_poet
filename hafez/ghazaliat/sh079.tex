\begin{center}
\section*{غزل شماره ۷۹: کنون که می‌دمد از بوستان نسیم بهشت}
\label{sec:sh079}
\addcontentsline{toc}{section}{\nameref{sec:sh079}}
\begin{longtable}{l p{0.5cm} r}
کنون که می‌دمد از بوستان نسیم بهشت
&&
من و شراب فرح بخش و یار حورسرشت
\\
گدا چرا نزند لاف سلطنت امروز
&&
که خیمه سایه ابر است و بزمگه لب کشت
\\
چمن حکایت اردیبهشت می‌گوید
&&
نه عاقل است که نسیه خرید و نقد بهشت
\\
به می عمارت دل کن که این جهان خراب
&&
بر آن سر است که از خاک ما بسازد خشت
\\
وفا مجوی ز دشمن که پرتوی ندهد
&&
چو شمع صومعه افروزی از چراغ کنشت
\\
مکن به نامه سیاهی ملامت من مست
&&
که آگه است که تقدیر بر سرش چه نوشت
\\
قدم دریغ مدار از جنازه حافظ
&&
که گر چه غرق گناه است می‌رود به بهشت
\\
\end{longtable}
\end{center}
