\begin{center}
\section*{غزل شماره ۳۶۷: فتوی پیر مغان دارم و قولیست قدیم}
\label{sec:sh367}
\addcontentsline{toc}{section}{\nameref{sec:sh367}}
\begin{longtable}{l p{0.5cm} r}
فتوی پیر مغان دارم و قولیست قدیم
&&
که حرام است می آن جا که نه یار است ندیم
\\
چاک خواهم زدن این دلق ریایی چه کنم
&&
روح را صحبت ناجنس عذابیست الیم
\\
تا مگر جرعه فشاند لب جانان بر من
&&
سال‌ها شد که منم بر در میخانه مقیم
\\
مگرش خدمت دیرین من از یاد برفت
&&
ای نسیم سحری یاد دهش عهد قدیم
\\
بعد صد سال اگر بر سر خاکم گذری
&&
سر برآرد ز گلم رقص کنان عظم رمیم
\\
دلبر از ما به صد امید ستد اول دل
&&
ظاهرا عهد فرامش نکند خلق کریم
\\
غنچه گو تنگ دل از کار فروبسته مباش
&&
کز دم صبح مدد یابی و انفاس نسیم
\\
فکر بهبود خود ای دل ز دری دیگر کن
&&
درد عاشق نشود به به مداوای حکیم
\\
گوهر معرفت آموز که با خود ببری
&&
که نصیب دگران است نصاب زر و سیم
\\
دام سخت است مگر یار شود لطف خدا
&&
ور نه آدم نبرد صرفه ز شیطان رجیم
\\
حافظ ار سیم و زرت نیست چه شد شاکر باش
&&
چه به از دولت لطف سخن و طبع سلیم
\\
\end{longtable}
\end{center}
