\begin{center}
\section*{غزل شماره ۴۳۵: با مدعی مگویید اسرار عشق و مستی}
\label{sec:sh435}
\addcontentsline{toc}{section}{\nameref{sec:sh435}}
\begin{longtable}{l p{0.5cm} r}
با مدعی مگویید اسرار عشق و مستی
&&
تا بی‌خبر بمیرد در درد خودپرستی
\\
عاشق شو ار نه روزی کار جهان سر آید
&&
ناخوانده نقش مقصود از کارگاه هستی
\\
دوش آن صنم چه خوش گفت در مجلس مغانم
&&
با کافران چه کارت گر بت نمی‌پرستی
\\
سلطان من خدا را زلفت شکست ما را
&&
تا کی کند سیاهی چندین درازدستی
\\
در گوشه سلامت مستور چون توان بود
&&
تا نرگس تو با ما گوید رموز مستی
\\
آن روز دیده بودم این فتنه‌ها که برخاست
&&
کز سرکشی زمانی با ما نمی‌نشستی
\\
عشقت به دست طوفان خواهد سپرد حافظ
&&
چون برق از این کشاکش پنداشتی که جستی
\\
\end{longtable}
\end{center}
