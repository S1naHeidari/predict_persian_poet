\begin{center}
\section*{غزل شماره ۴۴: کنون که بر کف گل جام باده صاف است}
\label{sec:sh044}
\addcontentsline{toc}{section}{\nameref{sec:sh044}}
\begin{longtable}{l p{0.5cm} r}
کنون که بر کف گل جام باده صاف است
&&
به صد هزار زبان بلبلش در اوصاف است
\\
بخواه دفتر اشعار و راه صحرا گیر
&&
چه وقت مدرسه و بحث کشف کشاف است
\\
فقیه مدرسه دی مست بود و فتوی داد
&&
که می حرام ولی به ز مال اوقاف است
\\
به درد و صاف تو را حکم نیست خوش درکش
&&
که هر چه ساقی ما کرد عین الطاف است
\\
ببر ز خلق و چو عنقا قیاس کار بگیر
&&
که صیت گوشه نشینان ز قاف تا قاف است
\\
حدیث مدعیان و خیال همکاران
&&
همان حکایت زردوز و بوریاباف است
\\
خموش حافظ و این نکته‌های چون زر سرخ
&&
نگاه دار که قلاب شهر صراف است
\\
\end{longtable}
\end{center}
