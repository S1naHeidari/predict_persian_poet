\begin{center}
\section*{غزل شماره ۸: ساقیا برخیز و درده جام را}
\label{sec:sh008}
\addcontentsline{toc}{section}{\nameref{sec:sh008}}
\begin{longtable}{l p{0.5cm} r}
ساقیا برخیز و درده جام را
&&
خاک بر سر کن غم ایام را
\\
ساغر می بر کفم نه تا ز بر
&&
برکشم این دلق ازرق فام را
\\
گر چه بدنامیست نزد عاقلان
&&
ما نمی‌خواهیم ننگ و نام را
\\
باده درده چند از این باد غرور
&&
خاک بر سر نفس نافرجام را
\\
دود آه سینهٔ نالان من
&&
سوخت این افسردگان خام را
\\
محرم راز دل شیدای خود
&&
کس نمی‌بینم ز خاص و عام را
\\
با دلارامی مرا خاطر خوش است
&&
کز دلم یک باره برد آرام را
\\
ننگرد دیگر به سرو اندر چمن
&&
هر که دید آن سرو سیم اندام را
\\
صبر کن حافظ به سختی روز و شب
&&
عاقبت روزی بیابی کام را
\\
\end{longtable}
\end{center}
