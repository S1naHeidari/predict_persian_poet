\begin{center}
\section*{غزل شماره ۳۵۴: به مژگان سیه کردی هزاران رخنه در دینم}
\label{sec:sh354}
\addcontentsline{toc}{section}{\nameref{sec:sh354}}
\begin{longtable}{l p{0.5cm} r}
به مژگان سیه کردی هزاران رخنه در دینم
&&
بیا کز چشم بیمارت هزاران درد برچینم
\\
الا ای همنشین دل که یارانت برفت از یاد
&&
مرا روزی مباد آن دم که بی یاد تو بنشینم
\\
جهان پیر است و بی‌بنیاد از این فرهادکش فریاد
&&
که کرد افسون و نیرنگش ملول از جان شیرینم
\\
ز تاب آتش دوری شدم غرق عرق چون گل
&&
بیار ای باد شبگیری نسیمی زان عرق چینم
\\
جهان فانی و باقی فدای شاهد و ساقی
&&
که سلطانی عالم را طفیل عشق می‌بینم
\\
اگر بر جای من غیری گزیند دوست حاکم اوست
&&
حرامم باد اگر من جان به جای دوست بگزینم
\\
صباح الخیر زد بلبل کجایی ساقیا برخیز
&&
که غوغا می‌کند در سر خیال خواب دوشینم
\\
شب رحلت هم از بستر روم در قصر حورالعین
&&
اگر در وقت جان دادن تو باشی شمع بالینم
\\
حدیث آرزومندی که در این نامه ثبت افتاد
&&
همانا بی‌غلط باشد که حافظ داد تلقینم
\\
\end{longtable}
\end{center}
