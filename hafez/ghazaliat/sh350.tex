\begin{center}
\section*{غزل شماره ۳۵۰: به عزم توبه سحر گفتم استخاره کنم}
\label{sec:sh350}
\addcontentsline{toc}{section}{\nameref{sec:sh350}}
\begin{longtable}{l p{0.5cm} r}
به عزم توبه سحر گفتم استخاره کنم
&&
بهار توبه شکن می‌رسد چه چاره کنم
\\
سخن درست بگویم نمی‌توانم دید
&&
که می خورند حریفان و من نظاره کنم
\\
چو غنچه با لب خندان به یاد مجلس شاه
&&
پیاله گیرم و از شوق جامه پاره کنم
\\
به دور لاله دماغ مرا علاج کنید
&&
گر از میانه بزم طرب کناره کنم
\\
ز روی دوست مرا چون گل مراد شکفت
&&
حواله سر دشمن به سنگ خاره کنم
\\
گدای میکده‌ام لیک وقت مستی بین
&&
که ناز بر فلک و حکم بر ستاره کنم
\\
مرا که نیست ره و رسم لقمه پرهیزی
&&
چرا ملامت رند شرابخواره کنم
\\
به تخت گل بنشانم بتی چو سلطانی
&&
ز سنبل و سمنش ساز طوق و یاره کنم
\\
ز باده خوردن پنهان ملول شد حافظ
&&
به بانگ بربط و نی رازش آشکاره کنم
\\
\end{longtable}
\end{center}
