\begin{center}
\section*{غزل شماره ۱۸۵: نقدها را بود آیا که عیاری گیرند}
\label{sec:sh185}
\addcontentsline{toc}{section}{\nameref{sec:sh185}}
\begin{longtable}{l p{0.5cm} r}
نقدها را بود آیا که عیاری گیرند
&&
تا همه صومعه داران پی کاری گیرند
\\
مصلحت دید من آن است که یاران همه کار
&&
بگذارند و خم طره یاری گیرند
\\
خوش گرفتند حریفان سر زلف ساقی
&&
گر فلکشان بگذارد که قراری گیرند
\\
قوت بازوی پرهیز به خوبان مفروش
&&
که در این خیل حصاری به سواری گیرند
\\
یا رب این بچه ترکان چه دلیرند به خون
&&
که به تیر مژه هر لحظه شکاری گیرند
\\
رقص بر شعر تر و ناله نی خوش باشد
&&
خاصه رقصی که در آن دست نگاری گیرند
\\
حافظ ابنای زمان را غم مسکینان نیست
&&
زین میان گر بتوان به که کناری گیرند
\\
\end{longtable}
\end{center}
