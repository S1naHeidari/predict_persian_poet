\begin{center}
\section*{غزل شماره ۸۳: گر ز دست زلف مشکینت خطایی رفت رفت}
\label{sec:sh083}
\addcontentsline{toc}{section}{\nameref{sec:sh083}}
\begin{longtable}{l p{0.5cm} r}
گر ز دست زلف مشکینت خطایی رفت رفت
&&
ور ز هندوی شما بر ما جفایی رفت رفت
\\
برق عشق ار خرمن پشمینه پوشی سوخت سوخت
&&
جور شاه کامران گر بر گدایی رفت رفت
\\
در طریقت رنجش خاطر نباشد می بیار
&&
هر کدورت را که بینی چون صفایی رفت رفت
\\
عشقبازی را تحمل باید ای دل پای دار
&&
گر ملالی بود بود و گر خطایی رفت رفت
\\
گر دلی از غمزه دلدار باری برد برد
&&
ور میان جان و جانان ماجرایی رفت رفت
\\
از سخن چینان ملالت‌ها پدید آمد ولی
&&
گر میان همنشینان ناسزایی رفت رفت
\\
عیب حافظ گو مکن واعظ که رفت از خانقاه
&&
پای آزادی چه بندی گر به جایی رفت رفت
\\
\end{longtable}
\end{center}
