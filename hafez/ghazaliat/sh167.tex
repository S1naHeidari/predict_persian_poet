\begin{center}
\section*{غزل شماره ۱۶۷: ستاره‌ای بدرخشید و ماه مجلس شد}
\label{sec:sh167}
\addcontentsline{toc}{section}{\nameref{sec:sh167}}
\begin{longtable}{l p{0.5cm} r}
ستاره‌ای بدرخشید و ماه مجلس شد
&&
دل رمیده ما را رفیق و مونس شد
\\
نگار من که به مکتب نرفت و خط ننوشت
&&
به غمزه مسئله آموز صد مدرس شد
\\
به بوی او دل بیمار عاشقان چو صبا
&&
فدای عارض نسرین و چشم نرگس شد
\\
به صدر مصطبه‌ام می‌نشاند اکنون دوست
&&
گدای شهر نگه کن که میر مجلس شد
\\
خیال آب خضر بست و جام اسکندر
&&
به جرعه نوشی سلطان ابوالفوارس شد
\\
طربسرای محبت کنون شود معمور
&&
که طاق ابروی یار منش مهندس شد
\\
لب از ترشح می پاک کن برای خدا
&&
که خاطرم به هزاران گنه موسوس شد
\\
کرشمه تو شرابی به عاشقان پیمود
&&
که علم بی‌خبر افتاد و عقل بی‌حس شد
\\
چو زر عزیز وجود است نظم من آری
&&
قبول دولتیان کیمیای این مس شد
\\
ز راه میکده یاران عنان بگردانید
&&
چرا که حافظ از این راه رفت و مفلس شد
\\
\end{longtable}
\end{center}
