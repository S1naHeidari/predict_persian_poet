\begin{center}
\section*{غزل شماره ۹۴: زان یار دلنوازم شکریست با شکایت}
\label{sec:sh094}
\addcontentsline{toc}{section}{\nameref{sec:sh094}}
\begin{longtable}{l p{0.5cm} r}
زان یار دلنوازم شکریست با شکایت
&&
گر نکته دان عشقی بشنو تو این حکایت
\\
بی مزد بود و منت هر خدمتی که کردم
&&
یا رب مباد کس را مخدوم بی عنایت
\\
رندان تشنه لب را آبی نمی‌دهد کس
&&
گویی ولی شناسان رفتند از این ولایت
\\
در زلف چون کمندش ای دل مپیچ کانجا
&&
سرها بریده بینی بی جرم و بی جنایت
\\
چشمت به غمزه ما را خون خورد و می‌پسندی
&&
جانا روا نباشد خونریز را حمایت
\\
در این شب سیاهم گم گشت راه مقصود
&&
از گوشه‌ای برون آی ای کوکب هدایت
\\
از هر طرف که رفتم جز وحشتم نیفزود
&&
زنهار از این بیابان وین راه بی‌نهایت
\\
ای آفتاب خوبان می‌جوشد اندرونم
&&
یک ساعتم بگنجان در سایه عنایت
\\
این راه را نهایت صورت کجا توان بست
&&
کش صد هزار منزل بیش است در بدایت
\\
هر چند بردی آبم روی از درت نتابم
&&
جور از حبیب خوشتر کز مدعی رعایت
\\
عشقت رسد به فریاد ار خود به سان حافظ
&&
قرآن ز بر بخوانی در چارده روایت
\\
\end{longtable}
\end{center}
