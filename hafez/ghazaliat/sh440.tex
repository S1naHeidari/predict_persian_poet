\begin{center}
\section*{غزل شماره ۴۴۰: سحر با باد می‌گفتم حدیث آرزومندی}
\label{sec:sh440}
\addcontentsline{toc}{section}{\nameref{sec:sh440}}
\begin{longtable}{l p{0.5cm} r}
سحر با باد می‌گفتم حدیث آرزومندی
&&
خطاب آمد که واثق شو به الطاف خداوندی
\\
دعای صبح و آه شب کلید گنج مقصود است
&&
بدین راه و روش می‌رو که با دلدار پیوندی
\\
قلم را آن زبان نبود که سر عشق گوید باز
&&
ورای حد تقریر است شرح آرزومندی
\\
الا ای یوسف مصری که کردت سلطنت مغرور
&&
پدر را بازپرس آخر کجا شد مهر فرزندی
\\
جهان پیر رعنا را ترحم در جبلت نیست
&&
ز مهر او چه می‌پرسی در او همت چه می‌بندی
\\
همایی چون تو عالی قدر حرص استخوان تا کی
&&
دریغ آن سایه همت که بر نااهل افکندی
\\
در این بازار اگر سودیست با درویش خرسند است
&&
خدایا منعمم گردان به درویشی و خرسندی
\\
به شعر حافظ شیراز می‌رقصند و می‌نازند
&&
سیه چشمان کشمیری و ترکان سمرقندی
\\
\end{longtable}
\end{center}
