\begin{center}
\section*{غزل شماره ۴۵۱: خوش کرد یاوری فلکت روز داوری}
\label{sec:sh451}
\addcontentsline{toc}{section}{\nameref{sec:sh451}}
\begin{longtable}{l p{0.5cm} r}
خوش کرد یاوری فلکت روز داوری
&&
تا شکر چون کنی و چه شکرانه آوری
\\
آن کس که اوفتاد خدایش گرفت دست
&&
گو بر تو باد تا غم افتادگان خوری
\\
در کوی عشق شوکت شاهی نمی‌خرند
&&
اقرار بندگی کن و اظهار چاکری
\\
ساقی به مژدگانی عیش از درم درآی
&&
تا یک دم از دلم غم دنیا به دربری
\\
در شاهراه جاه و بزرگی خطر بسیست
&&
آن به کز این گریوه سبکبار بگذری
\\
سلطان و فکر لشکر و سودای تاج و گنج
&&
درویش و امن خاطر و کنج قلندری
\\
یک حرف صوفیانه بگویم اجازت است
&&
ای نور دیده صلح به از جنگ و داوری
\\
نیل مراد بر حسب فکر و همت است
&&
از شاه نذر خیر و ز توفیق یاوری
\\
حافظ غبار فقر و قناعت ز رخ مشوی
&&
کاین خاک بهتر از عمل کیمیاگری
\\
\end{longtable}
\end{center}
