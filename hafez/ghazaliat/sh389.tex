\begin{center}
\section*{غزل شماره ۳۸۹: چو گل هر دم به بویت جامه در تن}
\label{sec:sh389}
\addcontentsline{toc}{section}{\nameref{sec:sh389}}
\begin{longtable}{l p{0.5cm} r}
چو گل هر دم به بویت جامه در تن
&&
کنم چاک از گریبان تا به دامن
\\
تنت را دید گل گویی که در باغ
&&
چو مستان جامه را بدرید بر تن
\\
من از دست غمت مشکل برم جان
&&
ولی دل را تو آسان بردی از من
\\
به قول دشمنان برگشتی از دوست
&&
نگردد هیچ کس با دوست دشمن
\\
تنت در جامه چون در جام باده
&&
دلت در سینه چون در سیم آهن
\\
ببار ای شمع اشک از چشم خونین
&&
که شد سوز دلت بر خلق روشن
\\
مکن کز سینه‌ام آه جگرسوز
&&
برآید همچو دود از راه روزن
\\
دلم را مشکن و در پا مینداز
&&
که دارد در سر زلف تو مسکن
\\
چو دل در زلف تو بسته‌ست حافظ
&&
بدین سان کار او در پا میفکن
\\
\end{longtable}
\end{center}
