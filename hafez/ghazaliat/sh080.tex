\begin{center}
\section*{غزل شماره ۸۰: عیب رندان مکن ای زاهد پاکیزه سرشت}
\label{sec:sh080}
\addcontentsline{toc}{section}{\nameref{sec:sh080}}
\begin{longtable}{l p{0.5cm} r}
عیب رندان مکن ای زاهد پاکیزه سرشت
&&
که گناه دگران بر تو نخواهند نوشت
\\
من اگر نیکم و گر بد تو برو خود را باش
&&
هر کسی آن درود عاقبت کار که کشت
\\
همه کس طالب یارند چه هشیار و چه مست
&&
همه جا خانه عشق است چه مسجد چه کنشت
\\
سر تسلیم من و خشت در میکده‌ها
&&
مدعی گر نکند فهم سخن گو سر و خشت
\\
ناامیدم مکن از سابقه لطف ازل
&&
تو پس پرده چه دانی که که خوب است و که زشت
\\
نه من از پرده تقوا به درافتادم و بس
&&
پدرم نیز بهشت ابد از دست بهشت
\\
حافظا روز اجل گر به کف آری جامی
&&
یک سر از کوی خرابات برندت به بهشت
\\
\end{longtable}
\end{center}
