\begin{center}
\section*{غزل شماره ۱۸۰: ای پسته تو خنده زده بر حدیث قند}
\label{sec:sh180}
\addcontentsline{toc}{section}{\nameref{sec:sh180}}
\begin{longtable}{l p{0.5cm} r}
ای پسته تو خنده زده بر حدیث قند
&&
مشتاقم از برای خدا یک شکر بخند
\\
طوبی ز قامت تو نیارد که دم زند
&&
زین قصه بگذرم که سخن می‌شود بلند
\\
خواهی که برنخیزدت از دیده رود خون
&&
دل در وفای صحبت رود کسان مبند
\\
گر جلوه می‌نمایی و گر طعنه می‌زنی
&&
ما نیستیم معتقد شیخ خودپسند
\\
ز آشفتگی حال من آگاه کی شود
&&
آن را که دل نگشت گرفتار این کمند
\\
بازار شوق گرم شد آن سروقد کجاست
&&
تا جان خود بر آتش رویش کنم سپند
\\
جایی که یار ما به شکرخنده دم زند
&&
ای پسته کیستی تو خدا را به خود مخند
\\
حافظ چو ترک غمزه ترکان نمی‌کنی
&&
دانی کجاست جای تو خوارزم یا خجند
\\
\end{longtable}
\end{center}
