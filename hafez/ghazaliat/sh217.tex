\begin{center}
\section*{غزل شماره ۲۱۷: مسلمانان مرا وقتی دلی بود}
\label{sec:sh217}
\addcontentsline{toc}{section}{\nameref{sec:sh217}}
\begin{longtable}{l p{0.5cm} r}
مسلمانان مرا وقتی دلی بود
&&
که با وی گفتمی گر مشکلی بود
\\
به گردابی چو می‌افتادم از غم
&&
به تدبیرش امید ساحلی بود
\\
دلی همدرد و یاری مصلحت بین
&&
که استظهار هر اهل دلی بود
\\
ز من ضایع شد اندر کوی جانان
&&
چه دامنگیر یا رب منزلی بود
\\
هنر بی‌عیب حرمان نیست لیکن
&&
ز من محروم‌تر کی سائلی بود
\\
بر این جان پریشان رحمت آرید
&&
که وقتی کاردانی کاملی بود
\\
مرا تا عشق تعلیم سخن کرد
&&
حدیثم نکته هر محفلی بود
\\
مگو دیگر که حافظ نکته‌دان است
&&
که ما دیدیم و محکم جاهلی بود
\\
\end{longtable}
\end{center}
