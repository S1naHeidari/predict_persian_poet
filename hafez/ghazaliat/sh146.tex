\begin{center}
\section*{غزل شماره ۱۴۶: صبا وقت سحر بویی ز زلف یار می‌آورد}
\label{sec:sh146}
\addcontentsline{toc}{section}{\nameref{sec:sh146}}
\begin{longtable}{l p{0.5cm} r}
صبا وقت سحر بویی ز زلف یار می‌آورد
&&
دل شوریده ما را به بو در کار می‌آورد
\\
من آن شکل صنوبر را ز باغ دیده برکندم
&&
که هر گل کز غمش بشکفت محنت بار می‌آورد
\\
فروغ ماه می‌دیدم ز بام قصر او روشن
&&
که رو از شرم آن خورشید در دیوار می‌آورد
\\
ز بیم غارت عشقش دل پرخون رها کردم
&&
ولی می‌ریخت خون و ره بدان هنجار می‌آورد
\\
به قول مطرب و ساقی برون رفتم گه و بی‌گه
&&
کز آن راه گران قاصد خبر دشوار می‌آورد
\\
سراسر بخشش جانان طریق لطف و احسان بود
&&
اگر تسبیح می‌فرمود اگر زنار می‌آورد
\\
عفاالله چین ابرویش اگر چه ناتوانم کرد
&&
به عشوه هم پیامی بر سر بیمار می‌آورد
\\
عجب می‌داشتم دیشب ز حافظ جام و پیمانه
&&
ولی منعش نمی‌کردم که صوفی وار می‌آورد
\\
\end{longtable}
\end{center}
