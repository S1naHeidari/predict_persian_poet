\begin{center}
\section*{غزل شماره ۲۸۳۴: به مبارکی و شادی بستان ز عشق جامی}
\label{sec:2834}
\addcontentsline{toc}{section}{\nameref{sec:2834}}
\begin{longtable}{l p{0.5cm} r}
به مبارکی و شادی بستان ز عشق جامی
&&
که ندا کند شرابش که کجاست تلخکامی
\\
چه بود حیات بی‌او هوسی و چارمیخی
&&
چه بود به پیش او جان دغلی کمین غلامی
\\
قدحی دو چون بخوردی خوش و شیرگیر گردی
&&
به دماغ تو فرستد شه و شیر ما پیامی
\\
خنک آن دلی که در وی بنهاد بخت تختی
&&
خنک آن سری که در وی می ما نهاد کامی
\\
ز سلام پادشاهان به خدا ملول گردد
&&
چو شنید نیکبختی ز تو سرسری سلامی
\\
به میان دلق مستی به قمارخانه جان
&&
بر خلق نام او بد سوی عرش نیک نامی
\\
خنک آن دمی که مالد کف شاه پر و بالش
&&
که سپیدباز مایی به چنین گزیده دامی
\\
ز شراب خوش بخورش نه شکوفه و نه شورش
&&
نه به دوستان نیازی نه ز دشمن انتقامی
\\
همه خلق در کشاکش تو خراب و مست و دلخوش
&&
همه را نظاره می‌کن هله از کنار بامی
\\
ز تو یک سؤال دارم بکنم دگر نگویم
&&
ز چه گشت زر پخته دل و جان ما ز خامی
\\
\end{longtable}
\end{center}
