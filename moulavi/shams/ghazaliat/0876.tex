\begin{center}
\section*{غزل شماره ۸۷۶: تا چند خرقه بردرم از بیم و از امید}
\label{sec:0876}
\addcontentsline{toc}{section}{\nameref{sec:0876}}
\begin{longtable}{l p{0.5cm} r}
تا چند خرقه بردرم از بیم و از امید
&&
درده شراب و واخرام از بیم و از امید
\\
پیش آر جام آتش اندیشه سوز را
&&
کاندیشه‌هاست در سرم از بیم و از امید
\\
کشتی نوح را که ز طوفان امان ماست
&&
بنما که زیر لنگرم از بیم و از امید
\\
آن زر سرخ و نقد طرب را بده که من
&&
رخسارزرد چون زرم از بیم و از امید
\\
در حلقه ز آنچ دادی در حلق من بریز
&&
کآخر چو حلقه بر درم از بیم و از امید
\\
بار دگر به آب ده این رنگ و بوی را
&&
کاین دم به رنگ دیگرم از بیم و از امید
\\
ز آبی که آب کوثر اندر هوای اوست
&&
کاندر هوای کوثرم از بیم و از امید
\\
در عین آتشم چو خلیلم فرست آب
&&
کزر مثال بتگرم از بیم و از امید
\\
کوری چشم بد تو ز چشمم نهان مشو
&&
کز چشم‌ها نهانترم از بیم و از امید
\\
در آفتاب روی خودم دار زانک من
&&
مانند این غزل ترم از بیم و از امید
\\
\end{longtable}
\end{center}
