\begin{center}
\section*{غزل شماره ۱۷۱۶: امشب جان را ببر از تن چاکر تمام}
\label{sec:1716}
\addcontentsline{toc}{section}{\nameref{sec:1716}}
\begin{longtable}{l p{0.5cm} r}
امشب جان را ببر از تن چاکر تمام
&&
تا نبود در جهان بیش مرا نقش و نام
\\
این دم مست توام رطل دگر دردهم
&&
تا بشوم محو تو از دو جهان والسلام
\\
چون ز تو فانی شدم و آنچ تو دانی شدم
&&
گیرم جام عدم می کشمش جام جام
\\
جان چو فروزد ز تو شمع بروزد ز تو
&&
گر بنسوزد ز تو جمله بود خام خام
\\
این نفسم دم به دم درده باده عدم
&&
چون به عدم درشدم خانه ندانم ز بام
\\
چون عدمت می فزود جان کندت صد سجود
&&
ای که هزاران وجود مر عدمت را غلام
\\
باده دهم طاس طاس ده ز وجودم خلاص
&&
باده شد انعام خاص عقل شد انعام عام
\\
موج برآر از عدم تا برباید مرا
&&
بر لب دریا به ترس چند روم گام گام
\\
دام شهم شمس دین صید به تبریز کرد
&&
من چو به دام اندرم نیست مرا ترس دام
\\
\end{longtable}
\end{center}
