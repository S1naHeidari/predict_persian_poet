\begin{center}
\section*{غزل شماره ۲۴۳۸: دزدید جمله رخت ما لولی و لولی زاده‌ای}
\label{sec:2438}
\addcontentsline{toc}{section}{\nameref{sec:2438}}
\begin{longtable}{l p{0.5cm} r}
دزدید جمله رخت ما لولی و لولی زاده‌ای
&&
در هیچ مسجد مکر او نگذاشته سجاده‌ای
\\
خرقه فلک ده شاخ از او برج قمر سوراخ از او
&&
وای ار بیفتد در کفش چون من سلیمی ساده‌ای
\\
زد آتش اندر عود ما بر آسمان شد دود ما
&&
بشکست باد و بود ما ساقی به نادر باده‌ای
\\
در کار مشکل می‌کند در بحر منزل می‌کند
&&
جان قصه دل می‌کند کو عاشقی دل داده‌ای
\\
دل داده آن باشد که او در صبر باشد سخت رو
&&
نی چون تو گوشه گشته‌ای در گوشه‌ای افتاده‌ای
\\
در غصه‌ای افتاده‌ای تا خود کجا دل داده‌ای
&&
در آرزوی قحبه یا وسوسه قواده‌ای
\\
شرمی بدار از ریش خود از ریش پرتشویش خود
&&
بسته دو چشم از عاقبت در هرزه لب گشاده‌ای
\\
خوب است عقل آن سری در عاقبت بینی جری
&&
از حرص وز شهوت بری در عاشقی آماده‌ای
\\
خامش که مرغ گفت من پرد سبک سوی چمن
&&
نبود گرو در دفتری در حجره‌ای بنهاده‌ای
\\
\end{longtable}
\end{center}
