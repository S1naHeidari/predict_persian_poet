\begin{center}
\section*{غزل شماره ۵۱۰: باز به بط گفت که صحرا خوشست}
\label{sec:0510}
\addcontentsline{toc}{section}{\nameref{sec:0510}}
\begin{longtable}{l p{0.5cm} r}
باز به بط گفت که صحرا خوشست
&&
گفت شبت خوش که مرا جا خوشست
\\
سر بنهم من که مرا سر خوشست
&&
راه تو پیما که سرت ناخوشست
\\
گر چه که تاریک بود مسکنم
&&
در نظر یوسف زیبا خوشست
\\
دوست چو در چاه بود چه خوشست
&&
دوست چو بالاست به بالا خوشست
\\
در بن دریا به تک آب تلخ
&&
در طلب گوهر رعنا خوشست
\\
بلبل نالنده به گلشن به دشت
&&
طوطی گوینده شکرخا خوشست
\\
تابش تسبیح فرشته‌ست و روح
&&
کاین فلک نادره مینا خوشست
\\
چونک خدا روفت دلت را ز حرص
&&
رو به دل آور دل یکتا خوشست
\\
از تو چو انداخت خدا رنج کار
&&
رو به تماشا که تماشا خوشست
\\
گفت تماشای جهان عکس ماست
&&
هم بر ما باش که با ما خوشست
\\
عکس در آیینه اگر چه نکوست
&&
لیک خود آن صورت احیا خوشست
\\
زردی رو عکس رخ احمرست
&&
بگذر از این عکس که حمرا خوشست
\\
نور خدایی‌ست که ذرات را
&&
رقص کنان بی‌سر و بی‌پا خوشست
\\
رقص در این نور خرد کن کز او
&&
تحت ثری تا به ثریا خوشست
\\
ذره شدی بازمرو که مشو
&&
صبر و وفا کن که وفاها خوشست
\\
بس کن چون دیده ببین و مگو
&&
دیده مجو دیده بینا خوشست
\\
مفخر تبریز شهم شمس دین
&&
با همه فرخنده و تنها خوشست
\\
\end{longtable}
\end{center}
