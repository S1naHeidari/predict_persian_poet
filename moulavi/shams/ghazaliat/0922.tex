\begin{center}
\section*{غزل شماره ۹۲۲: چو عشق را هوس بوسه و کنار بود}
\label{sec:0922}
\addcontentsline{toc}{section}{\nameref{sec:0922}}
\begin{longtable}{l p{0.5cm} r}
چو عشق را هوس بوسه و کنار بود
&&
که را قرار بود جان که را قرار بود
\\
شکارگاه بخندد چو شه شکار رود
&&
ولی چه گویی آن دم که شه شکار بود
\\
هزار ساغر می‌نشکند خمار مرا
&&
دلم چو مست چنان چشم پرخمار بود
\\
گهی که خاک شوم خاک ذره ذره شود
&&
نه ذره ذره من عاشق نگار بود
\\
ز هر غبار که آوازهای و هو شنوی
&&
بدانک ذره من اندر آن غبار بود
\\
دلم ز آه شود ساکن و ازو خجلم
&&
اگر چه آه ز ماه تو شرمسار بود
\\
به از صبوری اندر زمانه چیزی نیست
&&
ولی نه از تو که صبر از تو سخت عار بود
\\
ایا به خویش فرورفته در غم کاری
&&
تو تا برون نروی از میان چه کار بود
\\
چو عنکبوت زدود لعاب اندیشه
&&
دگر مباف که پوسیده پود و تار بود
\\
برو تو بازده اندیشه را بدو که بداد
&&
به شه نگر نه به اندیشه کان نثار بود
\\
چو تو نگویی گفت تو گفت او باشد
&&
چو تو نبافی بافنده کردگار بود
\\
\end{longtable}
\end{center}
