\begin{center}
\section*{غزل شماره ۱۲۶۵: آن مه که هست گردون گردان و بی‌قرارش}
\label{sec:1265}
\addcontentsline{toc}{section}{\nameref{sec:1265}}
\begin{longtable}{l p{0.5cm} r}
آن مه که هست گردون گردان و بی‌قرارش
&&
وان جان که هست این جان وین عقل مستعارش
\\
هر لحظه اختیاری نو نو دهد به جان‌ها
&&
وین اختیارها را بشکسته اختیارش
\\
من جسم و جان ندانم من این و آن ندانم
&&
من در جهان ندانم جز چشم پرخمارش
\\
آن روی همچو روزش وان رنگ دلفروزش
&&
وان لطف توبه سوزش وان خلق چون بهارش
\\
عشقش بلای توبه داده سزای توبه
&&
آخر چه جای توبه با عشق توبه خوارش
\\
چون دوست و دشمن او هستند رهزن او
&&
ماییم و دامن او بگرفته استوارش
\\
از عشق جام و دورش شاید کشید جورش
&&
چون گوش دوست داری می‌بوس گوشوارش
\\
من حلقه‌های زلفش از عشق می‌شمارم
&&
ور نه کجا رسد کس در حد و در شمارش
\\
لطفش همی‌شمارم دل با دم شمرده
&&
جانیش بخش آخر ای کشته زار زارش
\\
\end{longtable}
\end{center}
