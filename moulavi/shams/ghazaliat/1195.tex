\begin{center}
\section*{غزل شماره ۱۱۹۵: سوی خانه خویش آمد عشق آن عاشق نواز}
\label{sec:1195}
\addcontentsline{toc}{section}{\nameref{sec:1195}}
\begin{longtable}{l p{0.5cm} r}
سوی خانه خویش آمد عشق آن عاشق نواز
&&
عشق دارد در تصور صورتی صورت گداز
\\
خانه خویش آمدی خوش اندرآ شاد آمدی
&&
از در دل اندرآ تا پیشگاه جان بتاز
\\
ذره ذره از وجودم عاشق خورشید توست
&&
هین که با خورشید دارد ذره‌ها کار دراز
\\
پیش روزن ذره‌ها بین خوش معلق می‌زنند
&&
هر که را خورشید شد قبله چنین باشد نماز
\\
در سماع آفتاب این ذره‌ها چون صوفیان
&&
کس نداند بر چه قولی بر چه ضربی بر چه ساز
\\
اندرون هر دلی خود نغمه و ضربی دگر
&&
پای کوبان آشکار و مطربان پنهان چو راز
\\
برتر از جمله سماع ما بود در اندرون
&&
جزوهای ما در او رقصان به صد گون عز و ناز
\\
شمس تبریزی تویی سلطان سلطانان جان
&&
چون تو محمودی نیامد همچو من دیگر ایاز
\\
\end{longtable}
\end{center}
