\begin{center}
\section*{غزل شماره ۴۶۴: نوبت وصل و لقاست نوبت حشر و بقاست}
\label{sec:0464}
\addcontentsline{toc}{section}{\nameref{sec:0464}}
\begin{longtable}{l p{0.5cm} r}
نوبت وصل و لقاست نوبت حشر و بقاست
&&
نوبت لطف و عطاست بحر صفا در صفاست
\\
درج عطا شد پدید غره دریا رسید
&&
صبح سعادت دمید صبح چه نور خداست
\\
صورت و تصویر کیست این شه و این میر کیست
&&
این خرد پیر کیست این همه روپوش‌هاست
\\
چاره روپوش‌ها هست چنین جوش‌ها
&&
چشمه این نوش‌ها در سر و چشم شماست
\\
در سر خود پیچ لیک هست شما را دو سر
&&
این سر خاک از زمین وان سر پاک از سماست
\\
ای بس سرهای پاک ریخته در پای خاک
&&
تا تو بدانی که سر زان سر دیگر به پاست
\\
آن سر اصلی نهان وان سر فرعی عیان
&&
دانک پس این جهان عالم بی‌منتهاست
\\
مشک ببند ای سقا می‌نبرد خنب ما
&&
کوزه ادراک‌ها تنگ از این تنگناست
\\
از سوی تبریز تافت شمس حق و گفتمش
&&
نور تو هم متصل با همه و هم جداست
\\
\end{longtable}
\end{center}
