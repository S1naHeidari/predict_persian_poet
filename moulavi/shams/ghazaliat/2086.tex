\begin{center}
\section*{غزل شماره ۲۰۸۶: من کجا بودم عجب بی‌تو این چندین زمان}
\label{sec:2086}
\addcontentsline{toc}{section}{\nameref{sec:2086}}
\begin{longtable}{l p{0.5cm} r}
من کجا بودم عجب بی‌تو این چندین زمان
&&
در پی تو همچو تیر در کف تو چون کمان
\\
تو مرا دستور ده تا بگویم حال ده
&&
گر چه ازرق پوش شد شیخ ما چون آسمان
\\
برگشا این پرده را تازه کن پژمرده را
&&
تا رود خاکی به خاک تا روان گردد روان
\\
من کجا بودم عجب غایب از سلطان خویش
&&
ساعتی ترسان چو دزد ساعتی چون پاسبان
\\
گه اسیر چار و پنج گه میان گنج و رنج
&&
سود من بی‌روی تو بد زیان اندر زیان
\\
ور تو ای استاسرا متهم داری مرا
&&
روی زرد و چشم تر می‌دهد از دل نشان
\\
رحم را سیلاب برد یا نکوکاری بمرد
&&
ای زده تیر جفا ای کمان کرده نهان
\\
ای همه کردی ولی برنگشت از تو دلی
&&
ای جفا و جور تو به ز لطف دیگران
\\
باری این دم رسته‌ام با تو درپیوسته‌ام
&&
ای سبک روح جهان درده آن رطل گران
\\
واخرم یک بارگی از غم و بیچارگی
&&
سیرم از غمخوارگی منت غمخوارگان
\\
مست جام حق شوم فانی مطلق شوم
&&
پر برآرم در عدم برپرم در لامکان
\\
جان بر جانان رود گوش و هوشم نشنود
&&
بینی هر قلتبوز و چربک هر قلتبان
\\
همچو ذره مر مرا رقص باره کرده‌ای
&&
پای کوبان پای کوب جان دهم ای جان جان
\\
ای عجب گویم دگر باقیات این خبر
&&
نی خمش کردم تو گوی مطرب شیرین زبان
\\
اقتلونی یا ثقات ان فی قتلی حیات
&&
و الحیات فی الممات فی صبابات الحسان
\\
قد هدانا ربنا من سقام طبنا
&&
قد قضی ما فاتنا نعم هذا المستعان
\\
اقچلر در گزلری خوش نسا اول قشلری
&&
الدر ریز سواری کمدر اول الپ ارسلان
\\
نورکم فی ناظری حسنکم فی خاطری
&&
ان ربی ناصری رب زد هذا القرآن
\\
دب طیف فی الحشا نعم ماش قد مشا
&&
قد سقانا ما یشا فی کأس کالجفان
\\
ارفضوا هذا الفراق و اکرموا بالاعتناق
&&
و ارغبوا فی الاتفاق و افتحوا باب الجنان
\\
وقت عشرت هر کسی گوشه خلوت رود
&&
عشرت و شرب مرا می‌نباید شد نهان
\\
از کف این نیکبخت می‌خورم همچون درخت
&&
ور نه من سرسبز چون می‌روم مست و جوان
\\
چون سنان است این غزل در دل و جان دغل
&&
بیشتر شد عیب نیست این درازی در سنان
\\
فاعلاتن فاعلات فاعلاتن فاعلات
&&
شمس تبریزی تویی هم شه و هم ترجمان
\\
\end{longtable}
\end{center}
