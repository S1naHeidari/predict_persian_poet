\begin{center}
\section*{غزل شماره ۲۷۴۶: آن را که به لطف سر بخاری}
\label{sec:2746}
\addcontentsline{toc}{section}{\nameref{sec:2746}}
\begin{longtable}{l p{0.5cm} r}
آن را که به لطف سر بخاری
&&
از عقل و معامله برآری
\\
از یک نظرت قیامتی خاست
&&
یا رب تو در آن نظر چه داری
\\
از لعل تو دل دری بدزدید
&&
دزد است از آنش می‌فشاری
\\
بفشار به غم تو دزد خود را
&&
غم نیست چو هم تو غمگساری
\\
بفشار که رخت مؤمنان را
&&
پنهان کرده است از عیاری
\\
یا من نعش العبید فضلا
&&
من کل مواقع العثار
\\
بالفضل اعاد ما فقدنا
&&
بعد الحولان و التواری
\\
فجرت من الهوا عیونا
&&
فی مرج قلوبنا جواری
\\
تخضر بمائها غصون
&&
فی الروح لذیذه الثمار
\\
یا من غصب القلوب جهرا
&&
ثم اکرمهن فی السرار
\\
دی رفت و پریر رفت و امروز
&&
جان منتظر است تا چه آری
\\
هر روز ز تو وظیفه دارد
&&
این باز هزار گون شکاری
\\
برگیر کلاه از سر باز
&&
تا پر بزند در این صحاری
\\
زان پیش که می‌دهد مرا دوست
&&
آن لطف نمود و بردباری
\\
که مست شدم ز باده ماندم
&&
اندر بر لطف و حق گزاری
\\
آید از باغ لطف و سبزی
&&
آید ز بهار هم بهاری
\\
ای باد بهار عشق و سودا
&&
بر خسته دلان چه سازگاری
\\
اسکت و افتح جناح عشق
&&
حان الجولان فی المطار
\\
خاموش که غیر حرف و آواز
&&
بی صد لغت دگر سواری
\\
\end{longtable}
\end{center}
