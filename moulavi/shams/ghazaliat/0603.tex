\begin{center}
\section*{غزل شماره ۶۰۳: گویند به بلا ساقون ترکی دو کمان دارد}
\label{sec:0603}
\addcontentsline{toc}{section}{\nameref{sec:0603}}
\begin{longtable}{l p{0.5cm} r}
گویند به بلا ساقون ترکی دو کمان دارد
&&
ور زان دو یکی کم شد ما را چه زیان دارد
\\
ای در غم بیهوده از بوده و نابوده
&&
کاین کیسه زر دارد وان کاسه و خوان دارد
\\
در شام اگر میری زینی به کسی بخشد
&&
جانت ز حسد این جا رنج خفقان دارد
\\
جز غمزه چشم شه جز غصه خشم شه
&&
والله که نیندیشد هر زنده که جان دارد
\\
دیوانه کنم خود را تا هرزه نیندیشم
&&
دیوانه من از اصلم ای آنک عیان دارد
\\
چون عقل ندارم من پیش آ که تویی عقلم
&&
تو عقل بسی آن را کو چون تو شبان دارد
\\
گر طاعت کم دارم تو طاعت و خیر من
&&
آن را که تویی طاعت از خوف امان دارد
\\
ای کوزه گر صورت مفروش مرا کوزه
&&
کوزه چه کند آن کس کو جوی روان دارد
\\
تو وقف کنی خود را بر وقف یکی مرده
&&
من وقف کسی باشم کو جان و جهان دارد
\\
تو نیز بیا یارا تا یار شوی ما را
&&
زیرا که ز جان ما جان تو نشان دارد
\\
شمس الحق تبریزی خورشید وجود آمد
&&
کان چرخ چه چرخست آن کان جا سیران دارد
\\
\end{longtable}
\end{center}
