\begin{center}
\section*{غزل شماره ۱۳۲۴: برخیز ز خواب و ساز کن چنگ}
\label{sec:1324}
\addcontentsline{toc}{section}{\nameref{sec:1324}}
\begin{longtable}{l p{0.5cm} r}
برخیز ز خواب و ساز کن چنگ
&&
کان فتنه مه عذار گلرنگ
\\
نی خواب گذاشت خواجه نی صبر
&&
نی نام گذاشت خواجه نی ننگ
\\
بدرید خرد هزار خرقه
&&
بگریخت ادب هزار فرسنگ
\\
اندیشه و دل به خشم با هم
&&
استاره و مه ز رشک در جنگ
\\
استاره به جنگ کز فراقش
&&
این عرصه چرخ تنگ شد تنگ
\\
مه گوید بی ز آفتابش
&&
تا کی باشم ز چرخ آونگ
\\
بازار وجود بی‌عقیقش
&&
گو باش خراب سنگ بر سنگ
\\
ای عشق هزارنام خوش جام
&&
فرهنگ ده هزار فرهنگ
\\
بی‌صورت با هزار صورت
&&
صورت ده ترک و رومی و زنگ
\\
درده ز رحیق خویش یک جام
&&
یا از رز خویش یک کفی بنگ
\\
بگشا سر خنب را دگربار
&&
تا سر بنهد هزار سرهنگ
\\
تا حلقه مطربان گردون
&&
مستانه برآورند آهنگ
\\
مخمور رهد ز قیل و از قال
&&
تا حشر چو حشریان بود دنگ
\\
\end{longtable}
\end{center}
