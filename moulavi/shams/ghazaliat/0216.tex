\begin{center}
\section*{غزل شماره ۲۱۶: روم به حجره خیاط عاشقان فردا}
\label{sec:0216}
\addcontentsline{toc}{section}{\nameref{sec:0216}}
\begin{longtable}{l p{0.5cm} r}
روم به حجره خیاط عاشقان فردا
&&
من درازقبا با هزار گز سودا
\\
ببردت ز یزید و بدوزدت بر زید
&&
بدین یکی کندت جفت و زان دگر عذرا
\\
بدان یکیت بدوزد که دل نهی همه عمر
&&
زهی بریشم و بخیه زهی ید بیضا
\\
چو دل تمام نهادی ز هجر بشکافد
&&
به زخم نادره مقراض اهبطوا منها
\\
ز جمع کردن و تفریق او شدم حیران
&&
به ثبت و محو چو تلوین خاطر شیدا
\\
دل‌ست تخته پرخاک او مهندس دل
&&
زهی رسوم و رقوم و حقایق و اسما
\\
تو را چو در دگری ضرب کرد همچو عدد
&&
ز ضرب خود چه نتیجه همی‌کند پیدا
\\
چو ضرب دیدی اکنون بیا و قسمت بین
&&
که قطره‌ای را چون بخش کرد در دریا
\\
به جبر جمله اضداد را مقابله کرد
&&
خمش که فکر دراشکست زین عجایب‌ها
\\
\end{longtable}
\end{center}
