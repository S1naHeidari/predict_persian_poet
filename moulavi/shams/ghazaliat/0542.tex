\begin{center}
\section*{غزل شماره ۵۴۲: بی گاه شد بی‌گاه شد خورشید اندر چاه شد}
\label{sec:0542}
\addcontentsline{toc}{section}{\nameref{sec:0542}}
\begin{longtable}{l p{0.5cm} r}
بی گاه شد بی‌گاه شد خورشید اندر چاه شد
&&
خورشید جان عاشقان در خلوت الله شد
\\
روزیست اندر شب نهان ترکی میان هندوان
&&
هین ترک تازیی بکن کان ترک در خرگاه شد
\\
گر بو بری زان روشنی آتش به خواب اندرزنی
&&
کز شب روی و بندگی زهره حریف ماه شد
\\
گردیم ما آن شب روان اندر پی ما هندوان
&&
زیرا که ما بردیم زر تا پاسبان آگاه شد
\\
ما شب روی آموخته صد پاسبان را سوخته
&&
رخ‌ها چو گل افروخته کان بیذق ما شاه شد
\\
بشکست بازار زمین بازار انجم را ببین
&&
کز انجم و در ثمین آفاق خرمنگاه شد
\\
تا چند از این استور تن کو کاه و جو خواهد ز من
&&
بر چرخ راه کهکشان از بهر او پرکاه شد
\\
استور را اشکال نه رخ بر رخ اقبال نه
&&
اقبال آن جانی که او بی‌مثل و بی‌اشباه شد
\\
تن را بدیدی جان نگر گوهر بدیدی کان نگر
&&
این نادره ایمان نگر کایمان در او گمراه شد
\\
معنی همی‌گوید مکن ما را در این دلق کهن
&&
دلق کهن باشد سخن کو سخره افواه شد
\\
من گویم ای معنی بیا چون روح در صورت درآ
&&
تا خرقه‌ها و کهنه‌ها از فر جان دیباه شد
\\
بس کن رها کن گازری تا نشنود گوش پری
&&
کان روح از کروبیان هم سیر و خلوت خواه شد
\\
\end{longtable}
\end{center}
