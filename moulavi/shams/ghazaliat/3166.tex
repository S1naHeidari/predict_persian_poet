\begin{center}
\section*{غزل شماره ۳۱۶۶: ای دل سرمست، کجا می‌پری}
\label{sec:3166}
\addcontentsline{toc}{section}{\nameref{sec:3166}}
\begin{longtable}{l p{0.5cm} r}
ای دل سرمست، کجا می‌پری؟
&&
بزم تو کو؟ باده کجا می‌خوری؟
\\
مایهٔ هر نقش و ترا نقش نی
&&
دایهٔ هر جان و تو از جان بری
\\
صد مثل و نام و لقب گفتمت
&&
برتری از نام ولقب، برتری
\\
چونک ترا در دو جهان خانه نیست
&&
هر نفسی رخت کجا می‌بری؟
\\
نقد ترا بردم من پیش عقل
&&
گفتم: « قیمت کنش ای جوهری
\\
صیر فی نقد معانی توی
&&
سرمه کش دیدهٔ هر ناظری »
\\
گفت: « چه دانم ببرش پیش عشق
&&
عشق بود نقد ترا مشتری
\\
چون به سر کوچهٔ عشق آمدیم
&&
دل بشد و من بشدم بر سری
\\
\end{longtable}
\end{center}
