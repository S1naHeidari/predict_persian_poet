\begin{center}
\section*{غزل شماره ۱۹۲۰: دیر آمده‌ای مرو شتابان}
\label{sec:1920}
\addcontentsline{toc}{section}{\nameref{sec:1920}}
\begin{longtable}{l p{0.5cm} r}
دیر آمده‌ای مرو شتابان
&&
ای رفتن تو چو رفتن جان
\\
دیر آمدن و شتاب رفتن
&&
آیین گل است در گلستان
\\
گفتی چونی چنانک ماهی
&&
افتاده میان ریگ سوزان
\\
چون باشد شهر شهریارا
&&
بی دولت داد و عدل سلطان
\\
من بی‌تو نیم ولیک خواهم
&&
آن باتویی که هست پنهان
\\
شب پرتو آفتاب هم هست
&&
خاصه به تموز گرم و تفسان
\\
قانع نشود به گرمی او
&&
جز خفاشی ز بیم مرغان
\\
گرمی خواهند و روشنی هم
&&
مرغان که معودند با آن
\\
ما وصف دو جنس مرغ گفتیم
&&
بنگر ز کدامی ای غزل خوان
\\
\end{longtable}
\end{center}
