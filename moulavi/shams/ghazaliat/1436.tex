\begin{center}
\section*{غزل شماره ۱۴۳۶: تو خورشیدی و یا زهره و یا ماهی نمی‌دانم}
\label{sec:1436}
\addcontentsline{toc}{section}{\nameref{sec:1436}}
\begin{longtable}{l p{0.5cm} r}
تو خورشیدی و یا زهره و یا ماهی نمی‌دانم
&&
وزین سرگشته مجنون چه می خواهی نمی‌دانم
\\
در این درگاه بی‌چونی همه لطف است و موزونی
&&
چه صحرایی چه خضرایی چه درگاهی نمی‌دانم
\\
به خرمنگاه گردونی که راه کهکشان دارد
&&
چو ترکان گرد تو اختر چه خرگاهی نمی‌دانم
\\
ز رویت جان ما گلشن بنفشه و نرگس و سوسن
&&
ز ماهت ماه ما روشن چه همراهی نمی‌دانم
\\
زهی دریای بی‌ساحل پر از ماهی درون دل
&&
چنین دریا ندیدستم چنین ماهی نمی‌دانم
\\
شهی خلق افسانه محقر همچو شه دانه
&&
بجز آن شاه باقی را شهنشاهی نمی‌دانم
\\
زهی خورشید بی‌پایان که ذراتت سخن گویان
&&
تو نور ذات اللهی تو اللهی نمی‌دانم
\\
هزاران جان یعقوبی همی‌سوزد از این خوبی
&&
چرا ای یوسف خوبان در این چاهی نمی‌دانم
\\
خمش کن کز سخن چینی همیشه غرق تلوینی
&&
دمی هویی دمی‌هایی دمی آهی نمی‌دانم
\\
خمش کردم که سرمستم از آن افسون که خوردستم
&&
که بی‌خویشی و مستی را ز آگاهی نمی‌دانم
\\
\end{longtable}
\end{center}
