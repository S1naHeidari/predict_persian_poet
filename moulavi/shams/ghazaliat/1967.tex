\begin{center}
\section*{غزل شماره ۱۹۶۷: ای تو را گردن زده آن تسخرت بر گرد نان}
\label{sec:1967}
\addcontentsline{toc}{section}{\nameref{sec:1967}}
\begin{longtable}{l p{0.5cm} r}
ای تو را گردن زده آن تسخرت بر گرد نان
&&
ای سیاهی بر سیاهی جان تو از گرد نان
\\
ای تو در آیینه دیده روی خود کور و کبود
&&
تسخر و خنده زده بر آینه چون ابلهان
\\
تسخرت بر آینه نبود به روی خود بود
&&
زانک رویت هست تسخرگاه هر روشن روان
\\
آن منافق روی ظلمت جان تسخرکن که خود
&&
جمله سر تا پای تسخر بوده‌ست آن قلتبان
\\
هر کی در خون خود آید دست من چه گو درآ
&&
هر کی او دزدی کند حق است دار و نردبان
\\
هر کی استهزا کند بر خاصگان عشق حق
&&
تیغ قهرش بر سر آید از جلاد قهرمان
\\
ندهدش قهر خدا مهلت که تا یک دم زند
&&
گر چه دارد طاعت اهل زمین و آسمان
\\
عبرت از ابلیس گیرد آنک نسل آدم است
&&
کو به استهزای آدم شد سیه روی قران
\\
تا که بهتان‌ها نهد آن مظلم تاریک دل
&&
خنبک و مسخرگی و افسوس بر صاحب دلان
\\
احمد مرسل به طعن و سخره بوجهل بود
&&
موسی عمران به تسخرهای فرعونی چنان
\\
صبرها کردند تا قهر خدا اندررسید
&&
دود قهر حق برآمدشان ز سقف دودمان
\\
از ملامت‌های حسادان جگرها خون شود
&&
درد استهزای ایشان داغ‌ها آرد به جان
\\
گر از ایشان درگریزی در مغاره خلوتی
&&
عشق چون چوگانت آرد همچو گوی اندر میان
\\
تا چشاند مر تو را زهری ز هر افسرده‌ای
&&
تا کشاند نزد تو از هر حسودی ارمغان
\\
تا بده است این گوشمال عاشقان بوده‌ست از آنک
&&
در همه وقتی چنین بوده‌ست کار عاشقان
\\
گر تو اندر دین عشقی بر ملامت دل بنه
&&
وز فسوس و تسخر دشمن مکن رو را گران
\\
عاشقی چون روگری دان یا مثل آهنگری
&&
پس سیه باشد هماره چهره‌های روگران
\\
بر رخ روگر سیاهی از پی قزغان بود
&&
و آنگهی جمله سیاهی گرد شد بر قازغان
\\
همچنان در عاقبت این روسیاهی عاشقان
&&
جمع گردد بر رخ تسخرکن خنبک زنان
\\
عشق نقشی را حسودان دشمنی‌ها می کنند
&&
خاصه عشق پادشاه نقش ساز کامران
\\
نقش ساز نقش سوز ملک بخش بی‌نظیر
&&
جان فزایی دلربایی خوش پناه دو جهان
\\
خاص خاص سر حق و شمس دین بی‌نظیر
&&
فخر تبریز و خلاصه هستی و نور روان
\\
\end{longtable}
\end{center}
