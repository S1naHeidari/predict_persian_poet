\begin{center}
\section*{غزل شماره ۱۸۳۹: واقعه‌ای بدیده‌ام لایق لطف و آفرین}
\label{sec:1839}
\addcontentsline{toc}{section}{\nameref{sec:1839}}
\begin{longtable}{l p{0.5cm} r}
واقعه‌ای بدیده‌ام لایق لطف و آفرین
&&
خیز معبرالزمان صورت خواب من ببین
\\
خواب بدیده‌ام قمر چیست قمر به خواب در
&&
زانک به خواب حل شود آخر کار و اولین
\\
آن قمری که نور دل زو است گه حضور دل
&&
تا ز فروغ و ذوق دل روشنی است بر جبین
\\
یومئذ مسفره ضاحکه بود چنان
&&
ناعمه لسعیها راضیه بود چنین
\\
دور کن این وحوش را تا نکشند هوش را
&&
پنبه نهیم گوش را از هذیان آن و این
\\
ماند یکی دو سه نفس چند خیال بوالهوس
&&
نیست به خانه هیچ کس خانه مساز بر زمین
\\
شب بگذشت و شد سحر خیز مخسب بی‌خبر
&&
بی خبرت کجا هلد شعله آفتاب دین
\\
جوق تتار و سویرق حامله شد ز کین افق
&&
گو شکم فلک بدر بوک بزاید این جنین
\\
رو به میان روشنی چند تتار و ارمنی
&&
تیغ و کفن بپوش و رو چند ز جیب و آستین
\\
در شب شنبهی که شد پنجم ماه قعده را
&&
ششصد و پنجه‌ست و هم هست چهار از سنین
\\
هست به شهر ولوله این که شده‌ست زلزله
&&
شهر مدینه را کنون نقل کژ است یا یقین
\\
رو ز مدینه درگذر زلزله جهان نگر
&&
جنبش آسمان نگر بر نمطی عجبترین
\\
بحر نگر نهنگ بین بحر کبودرنگ بین
&&
موج نگر که اندر او هست نهنگ آتشین
\\
شکل نهنگ خفته بین یونس جان گرفته بین
&&
یونس جان که پیش از این کان من المسبحین
\\
بحر که می صفت کنم خارج شش جهت کنم
&&
بحر معلق از صور صاف بده‌ست پیش از این
\\
تیره نگشت آن صفا خیره شده‌ست چشم ما
&&
از قطرات آب و گل وز حرکات نقش طین
\\
گردن آنک دست او دست حدث پرست او
&&
تیره کند شراب ما تا بزنیم هین و هین
\\
چون نکنیم یاد او هست سزا و داد او
&&
کینه چو از خبر بود بی‌خبری است دفع کین
\\
خواست یکی نوشته‌ای عاشقی از معزمی
&&
گفت بگیر رقعه را زیر زمین بکن دفین
\\
لیک به وقت دفن این یاد مکن تو بوزنه
&&
زانک ز یاد بوزنه دور بمانی از قرین
\\
هر طرفی که رفت او تا بنهد دفینه را
&&
صورت بوزنه ز دل می بنمود از کمین
\\
گفت که آه اگر تو خود بوزنه را نگفتیی
&&
یاد نبد ز بوزنه در دل هیچ مستعین
\\
گفت بنه تو نیش را تازه مکن تو ریش را
&&
خواب بکن تو خویش را خواب مرو حسام دین
\\
\end{longtable}
\end{center}
