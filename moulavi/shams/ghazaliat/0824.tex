\begin{center}
\section*{غزل شماره ۸۲۴: عاشقان پیدا و دلبر ناپدید}
\label{sec:0824}
\addcontentsline{toc}{section}{\nameref{sec:0824}}
\begin{longtable}{l p{0.5cm} r}
عاشقان پیدا و دلبر ناپدید
&&
در همه عالم چنین عشقی که دید
\\
نارسیده یک لبی بر نقش جان
&&
صد هزاران جان‌ها تا لب رسید
\\
قاب قوسین از علی تیری فکند
&&
تا سپرهای فلک‌ها را درید
\\
ناکشیده دامن معشوق غیب
&&
دل هزاران محنت و ضربت کشید
\\
ناگزیده او لب شیرین لبی
&&
چند پشت دست در هجران گزید
\\
ناچریده از لبش شاخ شکر
&&
دل هزاران عشوه او را چرید
\\
ناشکفته از گلستانش گلی
&&
صد هزاران خار در سینه خلید
\\
گر چه جان از وی ندید الا جفا
&&
از وفاها بر امید او رمید
\\
آن الم را بر کرم‌ها فضل داد
&&
وان جفا را از وفاها برگزید
\\
خار او از جمله گل‌ها دست برد
&&
قفل او دلکشترست از صد کلید
\\
جور او از دور دولت گوی برد
&&
قندها از زهر قهرش بردمید
\\
رد او به از قبول دیگران
&&
لعل و مروارید سنگش را مرید
\\
این سعادت‌های دنیا هیچ نیست
&&
آن سعادت جو که دارد بوسعید
\\
این زیادت‌های این عالم کمیست
&&
آن زیادت جو که دارد بایزید
\\
آن زیادت دست شش انگشت تست
&&
قیمت او کم به ظاهر مستزید
\\
آن سناجو کش سنایی شرح کرد
&&
یافت فردیت ز عطار آن فرید
\\
چرب و شیرین می‌نماید پاک و خوش
&&
یک شبی بگذشت با تو شد پلید
\\
چرب و شیرین از غذای عشق خور
&&
تا پرت برروید و دانی پرید
\\
آخر اندر غار در طفلی خلیل
&&
از سر انگشت شیری می‌مکید
\\
آن رها کن آن جنین اندر شکم
&&
آب حیوانی ز خونی می‌مزید
\\
قد و بالایی که چرخش کرد راست
&&
عاقبت چون چرخ کژقامت خمید
\\
قد و بالایی که عشقش برفراشت
&&
برگذشت آن قدش از عرش مجید
\\
نی خمش کن عالم السر حاضرست
&&
نحن اقرب گفت من حبل الورید
\\
\end{longtable}
\end{center}
