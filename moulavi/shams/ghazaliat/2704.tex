\begin{center}
\section*{غزل شماره ۲۷۰۴: برون کن سر که جان سرخوشانی}
\label{sec:2704}
\addcontentsline{toc}{section}{\nameref{sec:2704}}
\begin{longtable}{l p{0.5cm} r}
برون کن سر که جان سرخوشانی
&&
فروکن سر ز بام بی‌نشانی
\\
به هر دم رخت مشتاقان خود را
&&
بدان سو کش که بس خوش می‌کشانی
\\
که عاشق همچو سیل و تو چو بحری
&&
که عاشق چون قراضه‌ست و تو کانی
\\
سقط‌های چو شکر باز می‌گوی
&&
که تو از لعل‌ها در می‌فشانی
\\
زهی آرامگاه جمله جان‌ها
&&
عجب افتاد حسن و مهربانی
\\
ز خوبی روی مه را خیره کردی
&&
به رحمت خود چنانتر از چنانی
\\
به هر تیری هزار آهو بگیری
&&
زهی شیری که بس سخته کمانی
\\
به هر بحری که تازی همچو موسی
&&
شکافد بحر تا در وی برانی
\\
همه جان در شکر دارند از وصل
&&
که هر یک گفت ما را نیست ثانی
\\
به کوه طور تو بسیار موسی
&&
ز غیرت گفته نی نی لن ترانی
\\
ز شمس الدین بپرس اسرار لن را
&&
که تبریز است دریای معانی
\\
\end{longtable}
\end{center}
