\begin{center}
\section*{غزل شماره ۹۸۱: شعر من نان مصر را ماند}
\label{sec:0981}
\addcontentsline{toc}{section}{\nameref{sec:0981}}
\begin{longtable}{l p{0.5cm} r}
شعر من نان مصر را ماند
&&
شب بر او بگذرد نتانی خورد
\\
آن زمانش بخور که تازه بود
&&
پیش از آنک بر او نشیند گرد
\\
گرمسیر ضمیر جای ویست
&&
می‌بمیرد در این جهان از برد
\\
همچو ماهی دمی به خشک طپید
&&
ساعتی دیگرش ببینی سرد
\\
ور خوری بر خیال تازگیش
&&
بس خیالات نقش باید کرد
\\
آنچ نوشی خیال تو باشد
&&
نبود گفتن کهن ای مرد
\\
\end{longtable}
\end{center}
