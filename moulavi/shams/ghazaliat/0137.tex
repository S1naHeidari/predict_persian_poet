\begin{center}
\section*{غزل شماره ۱۳۷: با چنین شمشیر دولت تو زبون مانی چرا}
\label{sec:0137}
\addcontentsline{toc}{section}{\nameref{sec:0137}}
\begin{longtable}{l p{0.5cm} r}
با چنین شمشیر دولت تو زبون مانی چرا
&&
گوهری باشی و از سنگی فرومانی چرا
\\
می‌کشد هر کرکسی اجزات را هر جانبی
&&
چون نه مرداری تو بلک باز جانانی چرا
\\
دیده‌ات را چون نظر از دیده باقی رسید
&&
دیده‌ات شرمین شود از دیده فانی چرا
\\
آن که او را کس به نسیه و نقد نستاند به خاک
&&
این چنین بیشی کند بر نقده کانی چرا
\\
آن سیه جانی که کفر از جان تلخش ننگ داشت
&&
زهر ریزد بر تو و تو شهد ایمانی چرا
\\
تو چنین لرزان او باشی و او سایه توست
&&
آخر او نقشیست جسمانی و تو جانی چرا
\\
او همه عیب تو گیرد تا بپوشد عیب خود
&&
تو بر او از غیب جان ریزی و می‌دانی چرا
\\
چون در او هستی به بینی گویی آن من نیستم
&&
دعوی او چون نبینی گوییش آنی چرا
\\
خشم یاران فرع باشد اصلشان عشق نوست
&&
از برای خشم فرعی اصل را رانی چرا
\\
شه به حق چون شمس تبریزیست ثانی نیستش
&&
ناحقی را اصل گویی شاه را ثانی چرا
\\
\end{longtable}
\end{center}
