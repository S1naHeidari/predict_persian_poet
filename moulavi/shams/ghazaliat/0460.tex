\begin{center}
\section*{غزل شماره ۴۶۰: عاشق آن قند تو جان شکرخای ماست}
\label{sec:0460}
\addcontentsline{toc}{section}{\nameref{sec:0460}}
\begin{longtable}{l p{0.5cm} r}
عاشق آن قند تو جان شکرخای ماست
&&
سایه زلفین تو در دو جهان جای ماست
\\
از قد و بالای اوست عشق که بالا گرفت
&&
و آنک بشد غرق عشق قامت و بالای ماست
\\
هر گل سرخی که هست از مدد خون ماست
&&
هر گل زردی که رست رسته ز صفرای ماست
\\
هر چه تصور کنی خواجه که همتاش نیست
&&
عاشق و مسکین آن بی‌ضد و همتای ماست
\\
از سبب هجر اوست شب که سیه پوش گشت
&&
توی به تو دود شب ز آتش سودای ماست
\\
نیست ز من باورت این سخن از شب بپرس
&&
تا بدهد شرح آنک فتنه فردای ماست
\\
شب چه بود روز نیز شهره و رسوای اوست
&&
کاهش مه از غم ماه دل افزای ماست
\\
آه که از هر دو کون تا چه نهان بوده‌ای
&&
خه که نهانی چنین شهره و پیدای ماست
\\
زان سوی لوح وجود مکتب عشاق بود
&&
و آنچ ز لوحش نمود آن همه اسمای ماست
\\
اول و پایان راه از اثر پای ماست
&&
ناطقه و نفس کل ناله سرنای ماست
\\
گر نه کژی همچو چنگ واسطه نای چیست
&&
در هوس آن سری اوست که هم پای ماست
\\
گر چه که ما هم کژیم در صفت جسم خویش
&&
بر سر منشور عشق جسم چو طغرای ماست
\\
رخت به تبریز برد مفخر جان شمس دین
&&
بازبیاریم زود کان همه کالای ماست
\\
\end{longtable}
\end{center}
