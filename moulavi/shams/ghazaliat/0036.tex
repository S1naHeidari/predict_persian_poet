\begin{center}
\section*{غزل شماره ۳۶: خواجه بیا خواجه بیا خواجه دگربار بیا}
\label{sec:0036}
\addcontentsline{toc}{section}{\nameref{sec:0036}}
\begin{longtable}{l p{0.5cm} r}
خواجه بیا خواجه بیا خواجه دگربار بیا
&&
دفع مده دفع مده ای مه عیار بیا
\\
عاشق مهجور نگر عالم پرشور نگر
&&
تشنه مخمور نگر ای شه خمار بیا
\\
پای تویی دست تویی هستی هر هست تویی
&&
بلبل سرمست تویی جانب گلزار بیا
\\
گوش تویی دیده تویی وز همه بگزیده تویی
&&
یوسف دزدیده تویی بر سر بازار بیا
\\
از نظر گشته نهان ای همه را جان و جهان
&&
بار دگر رقص کنان بی‌دل و دستار بیا
\\
روشنی روز تویی شادی غم سوز تویی
&&
ماه شب افروز تویی ابر شکربار بیا
\\
ای علم عالم نو پیش تو هر عقل گرو
&&
گاه میا گاه مرو خیز به یک بار بیا
\\
ای دل آغشته به خون چند بود شور و جنون
&&
پخته شد انگور کنون غوره میفشار بیا
\\
ای شب آشفته برو وی غم ناگفته برو
&&
ای خرد خفته برو دولت بیدار بیا
\\
ای دل آواره بیا وی جگر پاره بیا
&&
ور ره در بسته بود از ره دیوار بیا
\\
ای نفس نوح بیا وی هوس روح بیا
&&
مرهم مجروح بیا صحت بیمار بیا
\\
ای مه افروخته رو آب روان در دل جو
&&
شادی عشاق بجو کوری اغیار بیا
\\
بس بود ای ناطق جان چند از این گفت زبان
&&
چند زنی طبل بیان بی‌دم و گفتار بیا
\\
\end{longtable}
\end{center}
