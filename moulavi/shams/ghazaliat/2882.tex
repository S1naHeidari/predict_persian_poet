\begin{center}
\section*{غزل شماره ۲۸۸۲: تیغ را گر تو چو خورشید دمی رنده زنی}
\label{sec:2882}
\addcontentsline{toc}{section}{\nameref{sec:2882}}
\begin{longtable}{l p{0.5cm} r}
تیغ را گر تو چو خورشید دمی رنده زنی
&&
بر سر و سبلت این خنده زنان خنده زنی
\\
ژنده پوشیدی و جامه ملکی برکندی
&&
پاره پاره دل ما را تو بر آن ژنده زنی
\\
هر کی بندی است از این آب و از این گل برهد
&&
گر تو یک بند از آن طره بر این بنده زنی
\\
ساقیا عقل کجا ماند یا شرم و ادب
&&
زان می لعل چو بر مردم شرمنده زنی
\\
ماه فربه شود آن سان که نگنجد در چرخ
&&
گر تو تابی ز رخت بر مه تابنده زنی
\\
ماه می‌گوید با زهره که گر مست شوی
&&
ز آنچ من مست شدم ضرب پراکنده زنی
\\
ماه تا ماهی از این ساقی جان سرمستند
&&
نقد بستان تو چرا لاف ز آینده زنی
\\
خیز کامروز همایون و خوش و فرخنده‌ست
&&
خاصه که چشم بر آن چهره فرخنده زنی
\\
سر باز از کله و پاش از این کنده غمی است
&&
برهد پاش اگر تیشه بر این کنده زنی
\\
هله ای باز کله بازده و پر بگشا
&&
وقت آن شد که بر آن دولت پاینده زنی
\\
همچو منصور تو بر دار کن این ناطقه را
&&
چو زنان چند بر این پنبه و پاغنده زنی
\\
\end{longtable}
\end{center}
