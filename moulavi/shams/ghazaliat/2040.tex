\begin{center}
\section*{غزل شماره ۲۰۴۰: روز است ای دو دیده در روزنم نظر کن}
\label{sec:2040}
\addcontentsline{toc}{section}{\nameref{sec:2040}}
\begin{longtable}{l p{0.5cm} r}
روز است ای دو دیده در روزنم نظر کن
&&
تو اصل آفتابی چون آمدی سحر کن
\\
بردار طالبان را وز هفت بحر بگذر
&&
منگر به گاو و ماهی وز صد چنین گذر کن
\\
پیدا بکن که پاکی از کون و پست و بالا
&&
وین خانه کهن را بی‌زیر و بی‌زبر کن
\\
عالم فناست جمله در یک دمش بقا کن
&&
ماری است زهر دارد تو زهر او شکر کن
\\
هر سو که خشک بینی تو چشمه‌ای روان کن
&&
هر جا که سنگ بینی از عکس خود گهر کن
\\
اندر قفای عاشق هر سو که خصم بینی
&&
او را به زخم سیلی اندر زمان به درکن
\\
تا چند عذر گویی کورند و می‌نبینند
&&
گر کورشان نخواهی در دیده شان نظر کن
\\
خواهی که پرده‌هاشان در دیده‌ها نباشد
&&
فرما تو پردگی را کز پرده‌ها عبر کن
\\
فرمان تو راست مطلق با جمع در میان نه
&&
بستم قبای عطلت هم چاره کمر کن
\\
ای آفتاب عرشی ای شمس حق تبریز
&&
چون ماه نو نزارم رویم تو در قمر کن
\\
\end{longtable}
\end{center}
