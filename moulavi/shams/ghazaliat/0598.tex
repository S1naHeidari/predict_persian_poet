\begin{center}
\section*{غزل شماره ۵۹۸: یاران سحر خیزان تا صبح کی دریابد}
\label{sec:0598}
\addcontentsline{toc}{section}{\nameref{sec:0598}}
\begin{longtable}{l p{0.5cm} r}
یاران سحر خیزان تا صبح کی دریابد
&&
تا ذره صفت ما را کی زیر و زبر یابد
\\
آن بخت که را باشد کید به لب جویی
&&
تا آب خورد از جو خود عکس قمر یابد
\\
یعقوب صفت کی بود کز پیرهن یوسف
&&
او بوی پسر جوید خود نور بصر یابد
\\
یا تشنه چو اعرابی در چه فکند دلوی
&&
در دلو نگارینی چون تنگ شکر یابد
\\
یا موسی آتش جو کرد به درختی رو
&&
آید که برد آتش صد صبح و سحر یابد
\\
در خانه جهد عیسی تا وارهد از دشمن
&&
از خانه سوی گردون ناگاه گذر یابد
\\
یا همچو سلیمانی بشکافد ماهی را
&&
اندر شکم ماهی آن خاتم زر یابد
\\
شمشیر به کف عمر در قصد رسول آید
&&
در دام خدا افتد وز بخت نظر یابد
\\
یا چون پسر ادهم راند به سوی آهو
&&
تا صید کند آهو خود صید دگر یابد
\\
یا چون صدف تشنه بگشاده دهان آید
&&
تا قطره به خود گیرد در خویش گهر یابد
\\
یا مرد علف کش کو گردد سوی ویران‌ها
&&
ناگاه به ویرانی از گنج خبر یابد
\\
ره رو بهل افسانه تا محرم و بیگانه
&&
از نور الم نشرح بی‌شرح تو دریابد
\\
هر کو سوی شمس الدین از صدق نهد گامی
&&
گر پاش فروماند از عشق دو پر یابد
\\
\end{longtable}
\end{center}
