\begin{center}
\section*{غزل شماره ۳۰۱۳: یار در آخرزمان کرد طرب سازیی}
\label{sec:3013}
\addcontentsline{toc}{section}{\nameref{sec:3013}}
\begin{longtable}{l p{0.5cm} r}
یار در آخرزمان کرد طرب سازیی
&&
باطن او جد جد ظاهر او بازیی
\\
جمله عشاق را یار بدین علم کشت
&&
تا نکند هان و هان جهل تو طنازیی
\\
در حرکت باش ازانک آب روان نفسرد
&&
کز حرکت یافت عشق سر سراندازیی
\\
جنبش جان کی کند صورت گرمابه‌ای
&&
صف شکنی کی کند اسب گدا غازیی
\\
طبل غزا کوفتند این دم پیدا شود
&&
جنبش پالانیی از فرس تازیی
\\
می‌زن و می‌خور چو شیر تا به شهادت رسی
&&
تا بزنی گردن کافر ابخازیی
\\
بازی شیران مصاف بازی روبه گریز
&&
روبه با شیر حق کی کند انبازیی
\\
گرم روان از کجا تیره دلان از کجا
&&
مروزیی اوفتاد در ره با رازیی
\\
عشق عجب غازییست زنده شود زو شهید
&&
سر بنه ای جان پاک پیش چنین غازیی
\\
چرخ تن دل سیاه پر شود از نور ماه
&&
گر بکند قلب تو قالب پردازیی
\\
مطرب و سرنا و دف باده برآورده کف
&&
هر نفسی زان لطف آرد غمازیی
\\
ای خنک آن جان پاک کز سر میدان خاک
&&
گیرد زین قلبگاه قالب پردازیی
\\
\end{longtable}
\end{center}
