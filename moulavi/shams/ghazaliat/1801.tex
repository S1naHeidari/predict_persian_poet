\begin{center}
\section*{غزل شماره ۱۸۰۱: آن شاخ خشک است و سیه‌هان ای صبا بر وی مزن}
\label{sec:1801}
\addcontentsline{toc}{section}{\nameref{sec:1801}}
\begin{longtable}{l p{0.5cm} r}
آن شاخ خشک است و سیه‌هان ای صبا بر وی مزن
&&
ای زندگی باغ‌ها وی رنگ بخش مرد و زن
\\
هان ای صبای خوب خد اندر رکابت می رود
&&
آب روان و سبزه‌ها وز هر طرف وجه الحسن
\\
دریادلی و روشنی بر خشک و بر تر می زنی
&&
او سخت خشک است و سیه بر وی مزن از بهر من
\\
من خیره روتر آمدم بر جود تو راهی زدم
&&
این کی تواند گفت گل با لاله یا سرو و سمن
\\
ای باغ ساز و دست نی چون عقل فوق و پست نی
&&
هستی چو نحل خانه کن یا جان معمار بدن
\\
خواهی که معنی کش شوم رو صبر کن تا خوش شوم
&&
رنجور بسته فن بود خاصه در این باریک فن
\\
\end{longtable}
\end{center}
