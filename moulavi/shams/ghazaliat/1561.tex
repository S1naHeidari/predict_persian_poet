\begin{center}
\section*{غزل شماره ۱۵۶۱: تا چهره آن یگانه دیدم}
\label{sec:1561}
\addcontentsline{toc}{section}{\nameref{sec:1561}}
\begin{longtable}{l p{0.5cm} r}
تا چهره آن یگانه دیدم
&&
دل در غم بی‌کرانه دیدم
\\
گفتی فرداست روز بازار
&&
بازار تو را بهانه دیدم
\\
دل را چو انار ترش و شیرین
&&
خون بسته و دانه دانه دیدم
\\
زهر عالم همه عسل شد
&&
تا شهد تو در میانه دیدم
\\
جان را چو وثاق و جای زنبور
&&
از شهد تو خانه خانه دیدم
\\
بر آتشم و هنوز در عشق
&&
زان دوزخ یک زبانه دیدم
\\
شطرنج که صد هزار خانه‌ست
&&
از جمله آن دو خانه دیدم
\\
یک خانه پر از خمار دیدم
&&
یک خانه می مغانه دیدم
\\
چون عشق چنین دو روی دارد
&&
سرگشتگی زمانه دیدم
\\
وانگه زین سر به سوی آن سر
&&
دزدیده ره و دهانه دیدم
\\
زان ره خرد دقیقه بین را
&&
اندیشه ابلهانه دیدم
\\
او بر سر گنج بی‌نشانی
&&
سرگشته که من نشانه دیدم
\\
او زیر پر همای دولت
&&
گوید که به خواب لانه دیدم
\\
جانی که ز غم ز پا درآمد
&&
در عالم دل روانه دیدم
\\
جانی که فسانه داند این را
&&
او را همگی فسانه دیدم
\\
نالنده و بی‌خبر ز نالش
&&
چون بربط و چون چغانه دیدم
\\
بس شانه مکن که طره عشق
&&
بیرون ز حدود شانه دیدم
\\
صد شب بر او ترانه گویی
&&
روزت گوید تو را ندیدم
\\
هر درد که آن دوا ندارد
&&
سوی دل خود دوانه دیدم
\\
\end{longtable}
\end{center}
