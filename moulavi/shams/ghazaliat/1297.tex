\begin{center}
\section*{غزل شماره ۱۲۹۷: مدارم یک زمان از کار فارغ}
\label{sec:1297}
\addcontentsline{toc}{section}{\nameref{sec:1297}}
\begin{longtable}{l p{0.5cm} r}
مدارم یک زمان از کار فارغ
&&
که گردد آدمی غمخوار فارغ
\\
چو فارغ شد غم او را سخره گیرد
&&
مبادا هیچ کس ای یار فارغ
\\
قلندر گر چه فارغ می‌نماید
&&
ولیکن نیست در اسرار فارغ
\\
ز اول می‌کشد او خار بسیار
&&
همه گل گشت و گشت از خار فارغ
\\
چو موری دانه‌ها انبار می‌کرد
&&
سلیمان شد شد از انبار فارغ
\\
چو دریاییست او پرکار و بی‌کار
&&
از او گیرند و او ز ایثار فارغ
\\
قلندر هست در کشتی نشسته
&&
روان در را و از رفتار فارغ
\\
در این حیرت بسی بینی در این راه
&&
ز کشتی و ز دریابار فارغ
\\
به یاد بحر مست از وهم کشتی
&&
نشسته احمقی بسیار فارغ
\\
\end{longtable}
\end{center}
