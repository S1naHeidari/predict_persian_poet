\begin{center}
\section*{غزل شماره ۱۰۲۵: مرا آن اصل بیداری دگرباره به خواب اندر}
\label{sec:1025}
\addcontentsline{toc}{section}{\nameref{sec:1025}}
\begin{longtable}{l p{0.5cm} r}
مرا آن اصل بیداری دگرباره به خواب اندر
&&
بداد افیون شور و شر ببرد از سر ببرد از سر
\\
به صد حیله کنم غافل از او خود را کنم جاهل
&&
بیاید آن مه کامل به دست او چنین ساغر
\\
مرا گوید نمی‌گویی که تا چند از گدارویی
&&
چو هر عوری و ادباری گدایی می‌کنی هر در
\\
بدین زاری و خفریقی غلام دلق و ابریقی
&&
اگر حقی و تحقیقی چرایی این جوال اندر
\\
از این‌ها کز تو می‌زاید شهان را ننگ می‌آید
&&
ملک بودی چرا باید که باشی دیو را تسخر
\\
که داند گفت گفت او که عالم نیست جفت او
&&
ز پیدا و نهفت او جهان کورست و هستی کر
\\
مرا گر آن زبان بودی که راز یار بگشودی
&&
هر آن جانی که بشنودی برون جستی از این معبر
\\
از آن دلدار دریادل مرا حالیست بس مشکل
&&
که ویران می‌شود سینه از آن جولان و کر و فر
\\
اگر با مؤمنان گویم همه کافر شوند آن دم
&&
وگر با کافران گویم نماند در جهان کافر
\\
چو دوش آمد خیال او به خواب اندر تفضل جو
&&
مرا پرسید چونی تو بگفتم بی‌تو بس مضطر
\\
اگر صد جان بود ما را شود خون از غمت یارا
&&
دلت سنگست یا خارا و یا کوهیست از مرمر
\\
\end{longtable}
\end{center}
