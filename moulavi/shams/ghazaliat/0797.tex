\begin{center}
\section*{غزل شماره ۷۹۷: ز اول روز که مخموری مستان باشد}
\label{sec:0797}
\addcontentsline{toc}{section}{\nameref{sec:0797}}
\begin{longtable}{l p{0.5cm} r}
ز اول روز که مخموری مستان باشد
&&
شیخ را ساغر جان در کف دستان باشد
\\
پیش او ذره صفت هر سحری رقص کنیم
&&
این چنین عادت خورشیدپرستان باشد
\\
تا ابد این رخ خورشید سحر در سحرست
&&
تا دل سنگ از او لعل بدخشان باشد
\\
ای صلاح دل و دین تو ز برون جهتی
&&
تا چنین شش جهت از نور تو رخشان باشد
\\
بنده عشق تو در عشق کجا سرد شود
&&
چون صلاح دل و دین آتش سوزان باشد
\\
تو رضای دل او جو اگرت دل باید
&&
دل او چون طلبد آنک گران جان باشد
\\
ای بس ایمان که شود کفر چو با او نبود
&&
ای بسی کفر که از دولتش ایمان باشد
\\
گلخنی را چو ببینی به دل و روی سیاه
&&
هر چه از کان گهر گوید بهتان باشد
\\
شمس تبریز تو سلطان همه خوبانی
&&
هم جمال تو مگر یوسف کنعان باشد
\\
\end{longtable}
\end{center}
