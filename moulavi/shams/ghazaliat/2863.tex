\begin{center}
\section*{غزل شماره ۲۸۶۳: وقت آن شد که بدان روح فزا آمیزی}
\label{sec:2863}
\addcontentsline{toc}{section}{\nameref{sec:2863}}
\begin{longtable}{l p{0.5cm} r}
وقت آن شد که بدان روح فزا آمیزی
&&
مرغ زیرک شوی و خوش به دو پا آویزی
\\
سینه بگشا چو درختان به سوی باد بهار
&&
ز آنک زهر است تو را باد روی پاییزی
\\
به شکرخنده معنی تو شکر شو همگی
&&
در صفات ترشی خواجه چرا بستیزی
\\
زیر دیوار وجود تو تویی گنج گهر
&&
گنج ظاهر شود ار تو ز میان برخیزی
\\
آن قراضه ازلی ریخته در خاک تن است
&&
کو قراضه تک غلبیر تو گر می‌بیزی
\\
تیغ جانی تو برآور ز نیام بدنت
&&
که دو نیمه کند او قرص قمر از تیزی
\\
تیغ در دست درآ در سر میدان ابد
&&
از شب و روز برون تاز چو بر شبدیزی
\\
آب حیوان بکش از چشمه به سوی دل خود
&&
ز آنک در خلقت جان بر مثل کاریزی
\\
ور نتانی بگریز آ بر شه شمس الدین
&&
کو به جان هست ز عرش و به بدن تبریزی
\\
\end{longtable}
\end{center}
