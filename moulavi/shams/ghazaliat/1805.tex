\begin{center}
\section*{غزل شماره ۱۸۰۵: پوشیده چون جان می روی اندر میان جان من}
\label{sec:1805}
\addcontentsline{toc}{section}{\nameref{sec:1805}}
\begin{longtable}{l p{0.5cm} r}
پوشیده چون جان می روی اندر میان جان من
&&
سرو خرامان منی ای رونق بستان من
\\
چون می روی بی‌من مرو ای جان جان بی‌تن مرو
&&
وز چشم من بیرون مشو ای مشعله تابان من
\\
هفت آسمان را بردرم وز هفت دریا بگذرم
&&
چون دلبرانه بنگری در جان سرگردان من
\\
تا آمدی اندر برم شد کفر و ایمان چاکرم
&&
ای دیدن تو دین من وی روی تو ایمان من
\\
بی پا و سر کردی مرا بی‌خواب و خور کردی مرا
&&
در پیش یعقوب اندرآ ای یوسف کنعان من
\\
از لطف تو چون جان شدم وز خویشتن پنهان شدم
&&
ای هست تو پنهان شده در هستی پنهان من
\\
گل جامه در از دست تو وی چشم نرگس مست تو
&&
ای شاخه‌ها آبست تو وی باغ بی‌پایان من
\\
یک لحظه داغم می کشی یک دم به باغم می کشی
&&
پیش چراغم می کشی تا وا شود چشمان من
\\
ای جان پیش از جان‌ها وی کان پیش از کان‌ها
&&
ای آن بیش از آن‌ها ای آن من ای آن من
\\
چون منزل ما خاک نیست گر تن بریزد باک نیست
&&
اندیشه‌ام افلاک نیست ای وصل تو کیوان من
\\
بر یاد روی ماه من باشد فغان و آه من
&&
بر بوی شاهنشاه من هر لحظه‌ای حیران من
\\
ای جان چو ذره در هوا تا شد ز خورشیدت جدا
&&
بی تو چرا باشد چرا ای اصل چارارکان من
\\
ای شه صلاح الدین من ره دان من ره بین من
&&
ای فارغ از تمکین من ای برتر از امکان من
\\
\end{longtable}
\end{center}
