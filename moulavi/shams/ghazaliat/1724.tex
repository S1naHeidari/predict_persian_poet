\begin{center}
\section*{غزل شماره ۱۷۲۴: همه جمال تو بینم چو چشم باز کنم}
\label{sec:1724}
\addcontentsline{toc}{section}{\nameref{sec:1724}}
\begin{longtable}{l p{0.5cm} r}
همه جمال تو بینم چو چشم باز کنم
&&
همه شراب تو نوشم چو لب فراز کنم
\\
حرام دارم با مردمان سخن گفتن
&&
و چون حدیث تو آید سخن دراز کنم
\\
هزار گونه بلنگم به هر رهم که برند
&&
رهی که آن به سوی تو است ترک تاز کنم
\\
اگر به دست من آید چو خضر آب حیات
&&
ز خاک کوی تو آن آب را طراز کنم
\\
ز خارخار غم تو چو خارچین گردم
&&
ز نرگس و گل صدبرگ احتراز کنم
\\
ز آفتاب و ز مهتاب بگذرد نورم
&&
چو روی خود به شهنشاه دلنواز کنم
\\
چو پر و بال برآرم ز شوق چون بهرام
&&
به مسجد فلک هفتمین نماز کنم
\\
همه سعادت بینم چو سوی نحس روم
&&
همه حقیقت گردد اگر مجاز کنم
\\
مرا و قوم مرا عاقبت شود محمود
&&
چو خویش را پی محمود خود ایاز کنم
\\
چو آفتاب شوم آتش و ز گرمی دل
&&
چو ذره‌ها همه را مست و عشقباز کنم
\\
پریر عشق مرا گفت من همه نازم
&&
همه نیاز شو آن لحظه‌ای که ناز کنم
\\
چو ناز را بگذاری همه نیاز شوی
&&
من از برای تو خود را همه نیاز کنم
\\
خموش باش زمانی بساز با خمشی
&&
که تا برای سماع تو چنگ ساز کنم
\\
\end{longtable}
\end{center}
