\begin{center}
\section*{غزل شماره ۱۰۰: بیا ای جان نو داده جهان را}
\label{sec:0100}
\addcontentsline{toc}{section}{\nameref{sec:0100}}
\begin{longtable}{l p{0.5cm} r}
بیا ای جان نو داده جهان را
&&
ببر از کار عقل کاردان را
\\
چو تیرم تا نپرانی نپرم
&&
بیا بار دگر پر کن کمان را
\\
ز عشقت باز طشت از بام افتاد
&&
فرست از بام باز آن نردبان را
\\
مرا گویند بامش از چه سویست
&&
از آن سویی که آوردند جان را
\\
از آن سویی که هر شب جان روانست
&&
به وقت صبح بازآرد روان را
\\
از آن سو که بهار آید زمین را
&&
چراغ نو دهد صبح آسمان را
\\
از آن سو که عصایی اژدها شد
&&
به دوزخ برد او فرعونیان را
\\
از آن سو که تو را این جست و جو خاست
&&
نشان خود اوست می‌جوید نشان را
\\
تو آن مردی که او بر خر نشسته است
&&
همی‌پرسد ز خر این را و آن را
\\
خمش کن کو نمی‌خواهد ز غیرت
&&
که در دریا درآرد همگنان را
\\
\end{longtable}
\end{center}
