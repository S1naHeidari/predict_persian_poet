\begin{center}
\section*{غزل شماره ۱۴۵۷: من خفته وشم اما بس آگه و بیدارم}
\label{sec:1457}
\addcontentsline{toc}{section}{\nameref{sec:1457}}
\begin{longtable}{l p{0.5cm} r}
من خفته وشم اما بس آگه و بیدارم
&&
هر چند که بی‌هوشم در کار تو هشیارم
\\
با شیره فشارانت اندر چرش عشقم
&&
پای از پی آن کوبم کانگور تو افشارم
\\
تو پای همی‌بینی و انگور نمی‌بینی
&&
بستان قدحی شیره دریاب که عصارم
\\
اندر چرش جان آ گر پای همی‌کوبی
&&
تا غوطه خورم یک دم در شیره بسیارم
\\
زین باده نگردد سر زین شیره نشورد دل
&&
هین چاشنیی بستان زین باده که من دارم
\\
زین باده که داری تو پیوسته خماری تو
&&
دانم که چه داری تو در روت نمی‌آرم
\\
دامی که درافتادی بنگر سوی دام افکن
&&
تا ناظر حق باشی ای مرغ گرفتارم
\\
دام ار تک چه باشد فردوس کند حقش
&&
ور خار خسک باشد حق سازد گلزارم
\\
آن دم که به چاه آمد یوسف خبرش آمد
&&
که کار تو می سازد ای خسته بیمارم
\\
داروی تو می کوبم خرگاه تو می روبم
&&
از ضد ضدش انگیزم من قادر و قهارم
\\
گویم به حجر حی شو گویم به عدم شیء شو
&&
گویم به چمن دی شو داری عجب اقرارم
\\
شمس الحق تبریزی تو روشنی روزی
&&
و اندر پی روز تو من چون شب سیارم
\\
\end{longtable}
\end{center}
