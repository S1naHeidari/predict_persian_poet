\begin{center}
\section*{غزل شماره ۴۷۵: بیا که عاشق ماهست وز اختران پیداست}
\label{sec:0475}
\addcontentsline{toc}{section}{\nameref{sec:0475}}
\begin{longtable}{l p{0.5cm} r}
بیا که عاشق ماهست وز اختران پیداست
&&
بدانک مست تجلی به ماه راه نماست
\\
میان روز شتر بر سر مناره رود
&&
هر آنک گوید کو کو بدانک نابیناست
\\
بگرد عاشق اگر صد هزار خام بود
&&
مرا دو چشم ببندی بگویمت که کجاست
\\
بیا به پیش من آ تا به گوش تو گویم
&&
که از دهان و لب من پری رخی گویاست
\\
کسی که عاشق روی پری من باشد
&&
نزاده است ز آدم نه مادرش حواست
\\
عجب مدار از آن کس که ماه ما را دید
&&
چو آفتاب در آتش چو چرخ بی‌سر و پاست
\\
سر بریده نگر در میان خون غلطان
&&
دمی قرار ندارد مگر سر یحیاست
\\
او آفتاب و چو ماهست آن سر بی‌تن
&&
که روز و شب متقلب در این نشیب و علاست
\\
بر این بساط خرد را اگر خرد بودی
&&
بیامدی و بگفتی که این چه کارافزاست
\\
کسی که چهره دل دید اوست اهل خرد
&&
کسی که قامت جان یافت اوست کاهل صلاست
\\
در این چمن نظری کن به زعفران رویان
&&
که روی زرد و دل درد داغ آن سیماست
\\
خموش باش مگو راز اگر خرد داری
&&
ز ما خرد مطلب تا پری ما با ماست
\\
که برد مفخر تبریز شمس تبریزی
&&
خرد ز حلقه مغزم که سخت حلقه رباست
\\
\end{longtable}
\end{center}
