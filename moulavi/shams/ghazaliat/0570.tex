\begin{center}
\section*{غزل شماره ۵۷۰: بهار آمد بهار آمد بهار خوش عذار آمد}
\label{sec:0570}
\addcontentsline{toc}{section}{\nameref{sec:0570}}
\begin{longtable}{l p{0.5cm} r}
بهار آمد بهار آمد بهار خوش عذار آمد
&&
خوش و سرسبز شد عالم اوان لاله زار آمد
\\
ز سوسن بشنو ای ریحان که سوسن صد زبان دارد
&&
به دشت آب و گل بنگر که پرنقش و نگار آمد
\\
گل از نسرین همی‌پرسد که چون بودی در این غربت
&&
همی‌گوید خوشم زیرا خوشی‌ها زان دیار آمد
\\
سمن با سرو می‌گوید که مستانه همی‌رقصی
&&
به گوشش سرو می‌گوید که یار بردبار آمد
\\
بنفشه پیش نیلوفر درآمد که مبارک باد
&&
که زردی رفت و خشکی رفت و عمر پایدار آمد
\\
همی‌زد چشمک آن نرگس به سوی گل که خندانی
&&
بدو گفتا که خندانم که یار اندر کنار آمد
\\
صنوبر گفت راه سخت آسان شد به فضل حق
&&
که هر برگی به ره بری چو تیغ آبدار آمد
\\
ز ترکستان آن دنیا بنه ترکان زیبارو
&&
به هندستان آب و گل به امر شهریار آمد
\\
ببین کان لکلک گویا برآمد بر سر منبر
&&
که ای یاران آن کاره صلا که وقت کار آمد
\\
\end{longtable}
\end{center}
