\begin{center}
\section*{غزل شماره ۳۱۸۲: بغداد همانست که دیدی و شنیدی}
\label{sec:3182}
\addcontentsline{toc}{section}{\nameref{sec:3182}}
\begin{longtable}{l p{0.5cm} r}
بغداد همانست که دیدی و شنیدی
&&
رو دلبر نوجوی، چو دربند قدیدی؟!
\\
زین دیک جهان یک دو سه کفگیر بخوردی
&&
باقی، همه دیک آن مزه دارد که چشیدی
\\
الله مراد لی والله مریدی
&&
فرقت علی الله عتیقی و جدیدی
\\
من فرش شدم زیر قدمهای قضاهاش
&&
خود را نکشد فرش ز پاکی و پلیدی
\\
لا خیر ولا میر، سوی الله تعالی
&&
فالغیبة عنه نفسا غیر سدید
\\
از راحت و دردش نکشم خویش، و ندزدم
&&
قفلی دهدم حکم حق، و گاه کلیدی
\\
لا ارفع عنه بصری طرفة عین
&&
لا امنع عن رب ظریقی و تلیدی
\\
مرا هو العین و بالعین تطری
&&
روحی، و عمادی، و عتادی، و عتیدی
\\
رو خویش درانداز چو گوی، ارچه زنندت
&&
شه را تو به میدان نه که بازیچهٔ عیدی؟!
\\
این خلق چو چوگان و، زننده ملک و بس
&&
فاعل همه او دان، به قریبی و بعیدی
\\
از ناز برون آی، کزین ناز به ارزی
&&
تو روشنی چشم حسینی، نه یزیدی
\\
صالحت و بایعت مع‌العشق علی ان
&&
یاتینی محیاه نصیری و شهیدی
\\
لا اقسم بالوعد و بالصادق فیه
&&
ان قد ملاء العشق مرادی بمریدی
\\
هرجای که خشکیست درین بحر در آرید
&&
تا تر شود و تازه و غرقاب مجیدی
\\
الغصة والصحو جزاء لشحیح
&&
والقهوة والسکر وفاق لسعید
\\
العزةلله تعالی، فتعالوا
&&
فالعز من الله نثار لعبید
\\
یا خامد یا جامد یا منکر سکری
&&
یا قایم فی‌الصورة، یا شر حسیدی
\\
ارواح درین گلشن چون سرو روانند
&&
تو همچو بنفشه به جوانی چه خمیدی؟!
\\
لا حول ولا قوة الا بملیک
&&
یجعلک ملیکا وسنا کل ولید
\\
ای آهوی خوش ناف بران ناف عبر، باف
&&
کز سوسن و از سنبل آن پار چریدی
\\
\end{longtable}
\end{center}
