\begin{center}
\section*{غزل شماره ۲۹۷۰: با صد هزار دستان آمد خیال یاری}
\label{sec:2970}
\addcontentsline{toc}{section}{\nameref{sec:2970}}
\begin{longtable}{l p{0.5cm} r}
با صد هزار دستان آمد خیال یاری
&&
در پای او بمیرا هر جا بود نگاری
\\
خوبان بسی بدیدی حوران صفت شنیدی
&&
این جا بیا که بینی حسن و جمال یاری
\\
تا یافت جانم او را من گم شدم ز هستی
&&
تا پای او گرفتم دستم نشد به کاری
\\
ای مطرب الله الله از بهر عشق آن شه
&&
آن چنگ را در این ره خوش برنواز تاری
\\
زان چهره‌های شیرین در دل عجیب شوری
&&
این روی همچو زر را از مهر او عیاری
\\
گویند زاریت چیست زین ناله در دو عالم
&&
گفتم همین بسستم در هر دو عالم آری
\\
رفتم نظاره کردن سوی شکار آن شه
&&
می‌تاخت شاد و خندان آن ماه در غباری
\\
تیری ز غمزه خود انداخت بر من آمد
&&
تیری بدان شگرفی در لاغری شکاری
\\
از گلستان عشقش خاری در این جگر شد
&&
صد گلستان غلام خارش چگونه خاری
\\
در پیش ذوق عشقش در نور آفتابش
&&
تن چیست چون غباری جان چیست چون بخاری
\\
در باغ عشق رویش خصمت خدای بادا
&&
گر تو ز گل بگویی یا قامت چناری
\\
از چشم ساحر تو گشتیم شاعر تو
&&
عذر عظیم دارم در عشق خوش عذاری
\\
یا رب ببینم آن را کان شاه می‌خرامد
&&
داده به کون نوری زان چهره‌ای چو ناری
\\
بینم که جان تلخم شیرین شده ز شهدش
&&
بینم که اندرافتد شوری نو از شراری
\\
از عشق شمس دین شد تبریز بهر این دم
&&
مر گوش را سماعی مر چشم را نظاری
\\
\end{longtable}
\end{center}
