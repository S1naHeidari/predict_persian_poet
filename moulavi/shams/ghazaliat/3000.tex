\begin{center}
\section*{غزل شماره ۳۰۰۰: ساقی بیار باده سغراق ده منی}
\label{sec:3000}
\addcontentsline{toc}{section}{\nameref{sec:3000}}
\begin{longtable}{l p{0.5cm} r}
ساقی بیار باده سغراق ده منی
&&
اندیشه را رها کن کاری است کردنی
\\
ای نقد جان مگوی که ایام بیننا
&&
گردن مخار خواجه که وامی است گردنی
\\
ای آب زندگانی در تشنگان نگر
&&
بر دوست رحم آر به کوری دشمنی
\\
هوشی است بند ما و به پیش تو هوش چیست
&&
گر برج خیبر است بخواهیش برکنی
\\
اندر مقام هوش همه خوف و زلزله‌ست
&&
در بی‌هشی است عیش و مقامات ایمنی
\\
در بزم بی‌هشی همه جان‌ها مجردند
&&
رقصان چو ذره‌ها خورشان نور و روشنی
\\
ای آفتاب جان در و دیوار تن بسوز
&&
قانع نمی‌شویم بدین نور روزنی
\\
این قصه را رها کن ما سخت تشنه‌ایم
&&
تو ساقی کریمی و بی‌صرفه و غنی
\\
هیهای عاشقان همه از بوی گلشنی است
&&
آگاه نیست کس که چه باغ و چه گلشنی
\\
خشک آر و می‌نگر ز چپ و راست اشک خون
&&
ای سنگ دل بگوی که تا چند تن زنی
\\
بیهوده چند گویی خاموش کن بس است
&&
فرمان گفت نیست همان گیر که الکنی
\\
تا شمس حق تبریز آرد گشایشی
&&
کاین ناطقه نماند در حرف معتنی
\\
\end{longtable}
\end{center}
