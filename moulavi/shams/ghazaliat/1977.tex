\begin{center}
\section*{غزل شماره ۱۹۷۷: در ستایش‌های شمس الدین نباشم مفتتن}
\label{sec:1977}
\addcontentsline{toc}{section}{\nameref{sec:1977}}
\begin{longtable}{l p{0.5cm} r}
در ستایش‌های شمس الدین نباشم مفتتن
&&
تا تو گویی کاین غرض نفی من است از لا و لن
\\
چونک هست او کل کل صافی صافی کمال
&&
وصف او چون نوبهار و وصف اجزا یاسمن
\\
هر یکی نوعی گلی و هر یکی نوعی ثمر
&&
او چو سرمجموع باغ و جان جان صد چمن
\\
چون ستودی باغ را پس جمله را بستوده‌ای
&&
چون ستودی حق را داخل شود نقش وثن
\\
ور وثن را مدح گویی نیست داخل حسن حق
&&
گر چه هم می بازگردد آن به خالق فاعلمن
\\
لیک باقی وصف‌ها بستوده باشی جزو در
&&
شمس حق و دین چو دریا کی شود داخل بدن
\\
حق همی‌گوید منم هش دار ای کوته نظر
&&
شمس حق و دین بهانه‌ست اندر این برداشتن
\\
هر چه تو با فخر تبریز آوری بی‌خردگی
&&
آن به عین ذات من تو کرده‌ای ای ممتحن
\\
\end{longtable}
\end{center}
