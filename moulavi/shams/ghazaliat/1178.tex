\begin{center}
\section*{غزل شماره ۱۱۷۸: جاء الربیع و البطر زال الشتاء و الخطر}
\label{sec:1178}
\addcontentsline{toc}{section}{\nameref{sec:1178}}
\begin{longtable}{l p{0.5cm} r}
جاء الربیع و البطر زال الشتاء و الخطر
&&
من فضل رب عنده کل الخطایا تغتفر
\\
اوحی الیکم ربکم انا غفرنا ذنبکم
&&
فارضوا بما یقضی لکم ان الرضا خیر السیر
\\
کم قایلین فی الخفا انا علمنا بره
&&
فاجرک لدینا سره لا تشتغل فیما اشتهر
\\
السر فیک یا فتی لا تلتمس ممن اتی
&&
من لیس سر عنده لم ینتفع مما ظهر
\\
انظر الی اهل الردی کم عاینوا نور الهدی
&&
لم ترتفع استارهم من بعد ما انشق القمر
\\
یا ربنا رب المنن ان انت لم ترحم فمن
&&
منک الهدی منک الردی ما غیر ذا الا غرر
\\
یا شوق این العافیه کی اضطفر بالقافیه
&&
عندی صفات صافیه فی جنبها نطقی کدر
\\
ان کان نطقی مدرسی قد ظل عشقی مخرسی
&&
و العشق قرن غالب فینا و سلطان الظفر
\\
سر کتیم لفظه سیف جسیم لحظه
&&
شمس الضحی لا تختفی الا بسحار سحر
\\
یا ساحراء ابصارنا بالغت فی اسحارنا
&&
فارفق بنا اودارنا انا حضرنا فی السفر
\\
یا قوم موسی اننا فی التیه تهنا مثلکم
&&
کیف اهتدیتم فاخبرو الا تکتموا عنا الخبر
\\
ان عوقوا ترحالنا فالمن و السلوی لنا
&&
اصلحت ربی بالنا طاب السفر طاب الحضر
\\
ان الهوی قد غرنا من بعد ما قد سرنا
&&
فاکشف به لطف ضرنا قال النبی لا ضرر
\\
قالوا ندبر شأنکم نفتح لکم آذانکم
&&
نرفع لکم ارکانکم انتم مصابیح البشر
\\
هاکم معاریج اللقا فیها تداریج البقا
&&
انعم به من مستقی اکرم به من مستقر
\\
العیش حقاء عیشکم و الموت حقاء موتکم
&&
و الدین و الدنیا لکم هذا جزاء من شکر
\\
اسکت فلا تکثر اخی ان طلت تکثر ترتخی
&&
الحیل فی ریح الهوی فاحفظه کلا لا وزر
\\
\end{longtable}
\end{center}
