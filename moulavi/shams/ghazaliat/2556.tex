\begin{center}
\section*{غزل شماره ۲۵۵۶: زهی چشم مرا حاصل شده آیین خون ریزی}
\label{sec:2556}
\addcontentsline{toc}{section}{\nameref{sec:2556}}
\begin{longtable}{l p{0.5cm} r}
زهی چشم مرا حاصل شده آیین خون ریزی
&&
ز هجران خداوندی شمس الدین تبریزی
\\
ایا خورشید رخشنده متاب از امر او سر را
&&
که تاریک ابد گردی اگر با او تو بستیزی
\\
ایا ای ابر گر تو یک نظر از نرگسش یابی
&&
به جای آب آب زندگانی و گهربیزی
\\
اگر آتش شبی در خواب لطف و حلم او دیدی
&&
گلستان‌ها شدی آتش نکردی ذره‌ای تیزی
\\
به هنگامی که هر جانی به جانی جفت می‌گردند
&&
بفرمودند گر جانی به جان او نیامیزی
\\
که جان او چنان صاف و لطیف آمد که جان‌ها را
&&
ز روی شرم و لطف او فریضه گشت پرهیزی
\\
هر آنچ از روح او آید به وهم روح‌ها ناید
&&
که خشتک کی تواند کرد اندر جامه تیریزی
\\
کسی کاندر جهان از بوش انا لا غیر می گفته‌ست
&&
گر از جاهش ببردی بو ز حسرت کرده خون ریزی
\\
بیا ای عقل کل با من که بردابرد او بینی
&&
ورای بحر روحانی بدان شرطی که نگریزی
\\
از آن بحری گذشته‌ست او که دل‌ها دل از او یابند
&&
و جان‌ها جان از او گیرند و هر چیزی از او چیزی
\\
اگر انکار خواهی کرد از عجزی است اندر تو
&&
چه داند قوت حیدر مزاج حیز از حیزی
\\
علی الله خانه کعبه و فی الله بیت معمورا
&&
گهی که بشنوی تبریز از تعظیم برخیزی
\\
ایا ای عقل و تمییزی که لاف دیدنش داری
&&
وآنگه باخودی بالله که بی‌الهام و تمییزی
\\
\end{longtable}
\end{center}
