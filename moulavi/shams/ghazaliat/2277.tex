\begin{center}
\section*{غزل شماره ۲۲۷۷: یک چند رندند این طرف در ظل دل پنهان شده}
\label{sec:2277}
\addcontentsline{toc}{section}{\nameref{sec:2277}}
\begin{longtable}{l p{0.5cm} r}
یک چند رندند این طرف در ظل دل پنهان شده
&&
و آن آفتاب از سقف دل بر جانشان تابان شده
\\
هر نجم ناهیدی شده هر ذره خورشیدی شده
&&
خورشید و اختر پیششان چون ذره سرگردان شده
\\
آن عقل و دل گم کردگان جان سوی کیوان بردگان
&&
بی‌چتر و سنجق هر یکی کیخسرو و سلطان شده
\\
بسیار مرکب کشته‌ای گرد جهان برگشته‌ای
&&
در جان سفر کن درنگر قومی سراسر جان شده
\\
با این عطای ایزدی با این جمال و شاهدی
&&
فرمان پرستان را نگر مستغرق فرمان شده
\\
چون آینه آن سینه شان آن سینه بی‌کینه شان
&&
دلشان چو میدان فلک سلطان سوی میدان شده
\\
از هیهی و هیهایشان وز لعل شکرخایشان
&&
نقل و شراب و آن دگر در شهر ما ارزان شده
\\
چون دوش اگر بی‌خویشمی از فتنه من نندیشمی
&&
باقی این را بودمی بی‌خویشتن گویان شده
\\
این دم فروبندم دهن زیرا به خویشم مرتهن
&&
تا آن زمانی که دلم باشد از او سکران شده
\\
سلطان سلطانان جان شمس الحق تبریزیان
&&
هر جان از او دریا شده هر جسم از او مرجان شده
\\
\end{longtable}
\end{center}
