\begin{center}
\section*{غزل شماره ۲۵۸۸: ای یار غلط کردی با یار دگر رفتی}
\label{sec:2588}
\addcontentsline{toc}{section}{\nameref{sec:2588}}
\begin{longtable}{l p{0.5cm} r}
ای یار غلط کردی با یار دگر رفتی
&&
از کار خود افتادی در کار دگر رفتی
\\
صد بار ببخشودم بر تو به تو بنمودم
&&
ای خویش پسندیده هین بار دگر رفتی
\\
صد بار فسون کردم خار از تو برون کردم
&&
گلزار ندانستی در خار دگر رفتی
\\
گفتم که تویی ماهی با مار چه همراهی
&&
ای حال غلط کرده با مار دگر رفتی
\\
مانند مکوک کژ اندر کف جولاهه
&&
صد تار بریدی تو در تار دگر رفتی
\\
گفتی که تو را یارا در غار نمی‌بینم
&&
آن یار در آن غار است تو غار دگر رفتی
\\
چون کم نشود سنگت چون بد نشود رنگت
&&
بازار مرا دیده بازار دگر رفتی
\\
\end{longtable}
\end{center}
