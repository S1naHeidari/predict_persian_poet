\begin{center}
\section*{غزل شماره ۹۶۶: دیده خون گشت و خون نمی‌خسبد}
\label{sec:0966}
\addcontentsline{toc}{section}{\nameref{sec:0966}}
\begin{longtable}{l p{0.5cm} r}
دیده خون گشت و خون نمی‌خسبد
&&
دل من از جنون نمی‌خسبد
\\
مرغ و ماهی ز من شده خیره
&&
کاین شب و روز چون نمی‌خسبد
\\
پیش از این در عجب همی‌بودم
&&
کآسمان نگون نمی‌خسبد
\\
آسمان خود کنون ز من خیره است
&&
که چرا این زبون نمی‌خسبد
\\
عشق بر من فسون اعظم خواند
&&
جان شنید آن فسون نمی‌خسبد
\\
این یقینم شدست پیش از مرگ
&&
کز بدن جان برون نمی‌خسبد
\\
هین خمش کن به اصل راجع شو
&&
دیده راجعون نمی‌خسبد
\\
\end{longtable}
\end{center}
