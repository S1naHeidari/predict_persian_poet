\begin{center}
\section*{غزل شماره ۲۸۴۴: تو کیی در این ضمیرم که فزونتر از جهانی}
\label{sec:2844}
\addcontentsline{toc}{section}{\nameref{sec:2844}}
\begin{longtable}{l p{0.5cm} r}
تو کیی در این ضمیرم که فزونتر از جهانی
&&
تو که نکته جهانی ز چه نکته می‌جهانی
\\
تو کدام و من کدامم تو چه نام و من چه نامم
&&
تو چه دانه من چه دامم که نه اینی و نه آنی
\\
تو قلم به دست داری و جهان چو نقش پیشت
&&
صفتیش می‌نگاری صفتیش می‌ستانی
\\
چو قلم ز دست بنهی بدهیش بی‌قلم تو
&&
صفتی که نور گیرد ز خطاب لن ترانی
\\
تن اگر چه در دوادو اثر نشان جان است
&&
بنماید از لطافت رخ جان بدین نشانی
\\
سخن و زبان اگر چه که نشان و فیض حق است
&&
به چه ماند این زبانه به فسانه زبانی
\\
گل و خار و باغ اگر چه اثری است ز آسمان‌ها
&&
به چه ماند این حشیشی به جمال آسمانی
\\
وگر آسمان و اختر دهدت نشان جانان
&&
به چه ماند این دو فانی به جلالت معانی
\\
بفروز آتشی را که در او نشان بسوزد
&&
به نشان رسی تو آن دم که تو بی‌نشان بمانی
\\
هجر الحبیب روحی و هما بلامکان
&&
حجبا عن المدارک لنهایه التدانی
\\
و هوائه ربیع نضرت به جنان
&&
و جنانه محیط و جنانه جنانی
\\
\end{longtable}
\end{center}
