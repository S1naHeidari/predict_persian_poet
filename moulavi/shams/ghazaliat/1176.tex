\begin{center}
\section*{غزل شماره ۱۱۷۶: انتم الشمس و القمر منکم السمع و البصر}
\label{sec:1176}
\addcontentsline{toc}{section}{\nameref{sec:1176}}
\begin{longtable}{l p{0.5cm} r}
انتم الشمس و القمر منکم السمع و البصر
&&
نظر القلب فیکم بکم ینجلی النظر
\\
قلتم الصبر اجمل صبر العبد ما انصبر
&&
نحن ابناء وقتنا رحم الله من غبر
\\
قدموا ساده الهوی قلت یا قوم ما الخبر
&&
خوفونی بفتنه و اشاروا الی الحذر
\\
قلت القتل فی الهوی برکات بلا ضرر
&&
جرد العشق سیفه بادروا امه الفکر
\\
ان من عاش بعد ذا ضیع الوقت و احتکر
&&
نفخوا فی شبابه حمل الریح بالشرر
\\
مزج النار بالهوی لیس یبقی و لا یذر
&&
شببوا لی بنفخه یسکر نفخه السحر
\\
بر آن یار خوش نظر تو مگو هیچ از خبر
&&
چو خبر نیست محرمش بر او باش بی‌خبر
\\
دل من شد حجاب دل نظرم پرده نظر
&&
گفتم ای دوست غیر تو اگرم هست جان و سر
\\
بزن از عشق گردنم بجوی مر مرا مخر
&&
گفت من چیز دیگرم به جز این صورت بشر
\\
گفتمش روح خود تویی عجبا چیست آن دگر
&&
هله ای نای خوش نوا هله ای باد پرده در
\\
برو از گوش سوی دل بنگر کیست مستتر
&&
بدر این کیسه‌های ما تو به کوری کیسه گر
\\
چه غمست ار زرم بشد که میی هست همچو زر
&&
عربی گر چه خوش بود عجمی گو تو ای پسر
\\
\end{longtable}
\end{center}
