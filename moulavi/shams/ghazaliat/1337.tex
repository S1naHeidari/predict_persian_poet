\begin{center}
\section*{غزل شماره ۱۳۳۷: الا ای رو ترش کرده که تا نبود مرا مدخل}
\label{sec:1337}
\addcontentsline{toc}{section}{\nameref{sec:1337}}
\begin{longtable}{l p{0.5cm} r}
الا ای رو ترش کرده که تا نبود مرا مدخل
&&
نبشته گرد روی خود صلا نعم الادام الخل
\\
دو سه گام ار ز حرص و کین به حلم آیی عسل جوشی
&&
که عالم‌ها کنی شیرین نمی‌آیی زهی کاهل
\\
غلط دیدم غلط گفتم همیشه با غلط جفتم
&&
که گر من دیدمی رویت نماندی چشم من احول
\\
دلا خود را در آیینه چو کژ بینی هرآیینه
&&
تو کژ باشی نه آیینه تو خود را راست کن اول
\\
یکی می‌رفت در چاهی چو در چه دید او ماهی
&&
مه از گردون ندا کردش من این سویم تو لاتعجل
\\
مجو مه را در این پستی که نبود در عدم هستی
&&
نروید نیشکر هرگز چو کارد آدمی حنظل
\\
خوشی در نفی تست ای جان تو در اثبات می‌جویی
&&
از آن جا جو که می‌آید نگردد مشکل این جا حل
\\
تو آن بطی کز اشتابی ستاره جست در آبی
&&
تو آنی کز برای پا همی‌زد او رگ اکحل
\\
در این پایان در این ساران چو گم گشتند هشیاران
&&
چه سازم من که من در ره چنان مستم که لاتسأل
\\
خدایا دست مست خود بگیر ار نی در این مقصد
&&
ز مستی آن کند با خود که در مستی کند منبل
\\
گرم زیر و زبر کردی به خود نزدیکتر کردی
&&
که صحت آید از دردی چو افشرده شود دنبل
\\
ز بعد این می و مستی چو کار من تو کردستی
&&
توکل کرده‌ام بر تو صلا ای کاهلان تنبل
\\
تویی ای شمس تبریزی نه زین مشرق نه زین مغرب
&&
نه آن شمسی که هر باری کسوف آید شود مختل
\\
\end{longtable}
\end{center}
