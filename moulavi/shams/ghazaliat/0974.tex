\begin{center}
\section*{غزل شماره ۹۷۴: گر تو را بخت یار خواهد بود}
\label{sec:0974}
\addcontentsline{toc}{section}{\nameref{sec:0974}}
\begin{longtable}{l p{0.5cm} r}
گر تو را بخت یار خواهد بود
&&
عشق را با تو کار خواهد بود
\\
عمر بی‌عاشقی مدان به حساب
&&
کان برون از شمار خواهد بود
\\
هر زمانی که می‌رود بی‌عشق
&&
پیش حق شرمسار خواهد بود
\\
هر چه اندر وطن تو را سبکست
&&
ساعت کوچ بار خواهد بود
\\
بر تو این دم که در غم عشقی
&&
چون پدر بردبار خواهد بود
\\
فقر کز وی تو ننگ می‌داری
&&
آن جهان افتخار خواهد بود
\\
تلخی صبر اگر گلوگیر است
&&
عاقبت خوشگوار خواهد بود
\\
چون رهد شیر روح از این صندوق
&&
اندر آن مرغزار خواهد بود
\\
چون از این لاشه خر فرود آید
&&
شاه دل شهسوار خواهد بود
\\
دامن جهد و جد را بگشا
&&
کز فلک زر نثار خواهد بود
\\
تو نهان بودی و شدی پیدا
&&
هر نهان آشکار خواهد بود
\\
هر کی خود را نکرد خوار امروز
&&
همچو فرعون خوار خواهد بود
\\
هر که چون گل ز آتش آب نشد
&&
اندر آتش چو خار خواهد بود
\\
چون شکار خدا نشد نمرود
&&
پشه‌ای را شکار خواهد بود
\\
هر که از نقد وقت بست نظر
&&
سخره‌ای انتظار خواهد بود
\\
هر که را اختیار کردش عشق
&&
مست و بی‌اختیار خواهد بود
\\
هر که او پست و مست عشق نشد
&&
تا ابد در خمار خواهد بود
\\
هر که را مهر و مهر این دم نیست
&&
اشتری بی‌مهار خواهد بود
\\
در سر هر که چشم عبرت نیست
&&
خوار و بی‌اعتبار خواهد بود
\\
بس کن ار چه سخن نشاند غبار
&&
آخر از وی غبار خواهد بود
\\
شمس تبریز چون قرار گرفت
&&
دل از او بی‌قرار خواهد بود
\\
\end{longtable}
\end{center}
