\begin{center}
\section*{غزل شماره ۳۰۷۹: بیامدیم دگربار سوی مولایی}
\label{sec:3079}
\addcontentsline{toc}{section}{\nameref{sec:3079}}
\begin{longtable}{l p{0.5cm} r}
بیامدیم دگربار سوی مولایی
&&
که تا به زانوی او نیست هیچ دریایی
\\
هزار عقل ببندی به هم بدو نرسد
&&
کجا رسد به مه چرخ دست یا پایی
\\
فلک به طمع گلو را دراز کرد بدو
&&
نیافت بوسه ولیکن چشید حلوایی
\\
هزار حلق و گلو شد دراز سوی لبش
&&
که ریز بر سر ما نیز من و سلوایی
\\
بیامدیم دگربار سوی معشوقی
&&
که می‌رسید به گوش از هواش هیهایی
\\
بیامدیم دگربار سوی آن حرمی
&&
که فرق سجده کنش هست آسمان سایی
\\
بیامدیم دگربار سوی آن چمنی
&&
که هست بلبل او را غلام عنقایی
\\
بیامدیم بدو کو جدا نبود از ما
&&
که مشک پر نشود بی‌وجود سقایی
\\
همیشه مشک بچفسیده بر تن سقا
&&
که نیست بی‌تو مرا دست و دانش و رایی
\\
بیامدیم دگربار سوی آن بزمی
&&
که شد ز نقل خوشش کام نیشکرخایی
\\
بیامدیم دگربار سوی آن چرخی
&&
که جان چو رعد زند در خمش علالایی
\\
بیامدیم دگربار سوی آن عشقی
&&
که دیو گشت ز آسیب او پری زایی
\\
خموش زیر زبان ختم کن تو باقی را
&&
که هست بر تو موکل غیور لالایی
\\
حدیث مفخر تبریز شمس دین کم گو
&&
که نیست درخور آن گفت عقل گویایی
\\
\end{longtable}
\end{center}
