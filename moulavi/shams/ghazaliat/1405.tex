\begin{center}
\section*{غزل شماره ۱۴۰۵: میل هواش می کنم طال بقاش می زنم}
\label{sec:1405}
\addcontentsline{toc}{section}{\nameref{sec:1405}}
\begin{longtable}{l p{0.5cm} r}
میل هواش می کنم طال بقاش می زنم
&&
حلقه به گوش و عاشقم طبل وفاش می زنم
\\
از دل و جان شکسته‌ام بر سر ره نشسته‌ام
&&
قافله خیال را بهر لقاش می زنم
\\
غیر طواشی غمش یا یلواج مرهمش
&&
هر چه سری برون کند بر سر و پاش می زنم
\\
این دل همچو چنگ را مست خراب دنگ را
&&
زخمه به کف گرفته‌ام همچو سه تاش می زنم
\\
دل که خرید جوهری از تک حوض کوثری
&&
خفت و بها نمی‌دهد بهر بهاش می زنم
\\
شب چو به خواب می رود گوش کشانش می کشم
&&
چون به سحر دعا کند وقت دعاش می زنم
\\
لذت تازیانه‌ام کی برسد به لاشه‌اش
&&
چون که گمان برد که من بهر فناش می زنم
\\
گر قمر و فلک بود ور خرد و ملک بود
&&
چونک حجاب دل شود زود قفاش می زنم
\\
گفتم شیشه مرا بر سر سنگ می زنی
&&
گفت چو لاف عشق زد تیغ بلاش می زنم
\\
هر رگ این رباب را ناله نو نوای نو
&&
تا ز نواش پی برد دل که کجاش می زنم
\\
در دل هر فغان او چاشنی سرشته‌ام
&&
تا نبری گمان که من سهو و خطاش می زنم
\\
خشم شهان گه عطا خنجر و گرز می زند
&&
من به سخاش می کشم من به عطاش می زنم
\\
سخت لطیف می زنم دیده بدان نمی‌رسد
&&
دل که هوای ما کند همچو هواش می زنم
\\
خامش باش زین حنین پرده راست نیست این
&&
راه شماست این نوا پیش شماش می زنم
\\
\end{longtable}
\end{center}
