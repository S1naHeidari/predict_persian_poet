\begin{center}
\section*{غزل شماره ۴۵: با لب او چه خوش بود گفت و شنید و ماجرا}
\label{sec:0045}
\addcontentsline{toc}{section}{\nameref{sec:0045}}
\begin{longtable}{l p{0.5cm} r}
با لب او چه خوش بود گفت و شنید و ماجرا
&&
خاصه که در گشاید و گوید خواجه اندرآ
\\
با لب خشک گوید او قصه چشمه خضر
&&
بر قد مرد می‌برد درزی عشق او قبا
\\
مست شوند چشم‌ها از سکرات چشم او
&&
رقص کنان درخت‌ها پیش لطافت صبا
\\
بلبل با درخت گل گوید چیست در دلت
&&
این دم در میان بنه نیست کسی تویی و ما
\\
گوید تا تو با تویی هیچ مدار این طمع
&&
جهد نمای تا بری رخت توی از این سرا
\\
چشمه سوزن هوس تنگ بود یقین بدان
&&
ره ندهد به ریسمان چونک ببیندش دوتا
\\
بنگر آفتاب را تا به گلو در آتشی
&&
تا که ز روی او شود روی زمین پر از ضیا
\\
چونک کلیم حق بشد سوی درخت آتشین
&&
گفت من آب کوثرم کفش برون کن و بیا
\\
هیچ مترس ز آتشم زانک من آبم و خوشم
&&
جانب دولت آمدی صدر تراست مرحبا
\\
جوهریی و لعل کان جان مکان و لامکان
&&
نادره زمانه‌ای خلق کجا و تو کجا
\\
بارگه عطا شود از کف عشق هر کفی
&&
کارگه وفا شود از تو جهان بی‌وفا
\\
ز اول روز آمدی ساغر خسروی به کف
&&
جانب بزم می‌کشی جان مرا که الصلا
\\
دل چه شود چو دست دل گیرد دست دلبری
&&
مس چه شود چو بشنود بانگ و صلای کیمیا
\\
آمد دلبری عجب نیزه به دست چون عرب
&&
گفتم هست خدمتی گفت تعال عندنا
\\
جست دلم که من دوم گفت خرد که من روم
&&
کرد اشارت از کرم گفت بلی کلا کما
\\
خوان چو رسید از آسمان دست بشوی و هم دهان
&&
تا که نیاید از کفت بوی پیاز و گندنا
\\
کان نمک رسید هین گر تو ملیح و عاشقی
&&
کاس ستان و کاسه ده شور گزین نه شوربا
\\
بسته کنم من این دو لب تا که چراغ روز و شب
&&
هم به زبانه زبان گوید قصه با شما
\\
\end{longtable}
\end{center}
