\begin{center}
\section*{غزل شماره ۶۸۰: نگارا مردگان از جان چه دانند}
\label{sec:0680}
\addcontentsline{toc}{section}{\nameref{sec:0680}}
\begin{longtable}{l p{0.5cm} r}
نگارا مردگان از جان چه دانند
&&
کلاغان قدر تابستان چه دانند
\\
بر بیگانگان تا چند باشی
&&
بیا جان قدر تو ایشان چه دانند
\\
بپوشان قد خوبت را از ایشان
&&
که کوران سرو در بستان چه دانند
\\
خرامان جانب میدان خویش آ
&&
مباش آن جا خران میدان چه دانند
\\
بزن چوگان خود را بر در ما
&&
که خامان لطف آن چوگان چه دانند
\\
بهل ویرانه بر جغدان منکر
&&
که جغدان شهر آبادان چه دانند
\\
چه دانند ملک دل را تن پرستان
&&
گدایان طبع سلطانان چه دانند
\\
یکی مشتی از این بی‌دست و بی‌پا
&&
حدیث رستم دستان چه دانند
\\
\end{longtable}
\end{center}
