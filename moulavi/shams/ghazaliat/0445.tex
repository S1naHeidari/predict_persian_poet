\begin{center}
\section*{غزل شماره ۴۴۵: این طرفه آتشی که دمی برقرار نیست}
\label{sec:0445}
\addcontentsline{toc}{section}{\nameref{sec:0445}}
\begin{longtable}{l p{0.5cm} r}
این طرفه آتشی که دمی برقرار نیست
&&
گر نزد یار باشد وگر نزد یار نیست
\\
صورت چه پای دارد کو را ثبات نیست
&&
معنی چه دست گیرد چون آشکار نیست
\\
عالم شکارگاه و خلایق همه شکار
&&
غیر نشانه‌ای ز امیر شکار نیست
\\
هر سوی کار و بار که ما میر و مهتریم
&&
وان سو که بارگاه امیرست بار نیست
\\
ای روح دست برکن و بنمای رنگ خوش
&&
کاین‌ها همه به جز کف و نقش و نگار نیست
\\
هر جا غبار خیزد آن جای لشکرست
&&
کآتش همیشه بی‌تف و دود و بخار نیست
\\
تو مرد را ز گرد ندانی چه مردیست
&&
در گرد مرد جوی که با گرد کار نیست
\\
ای نیکبخت اگر تو نجویی بجویدت
&&
جوینده‌ای که رحمت وی را شمار نیست
\\
سیلت چو دررباید دانی که در رهش
&&
هست اختیار خلق ولیک اختیار نیست
\\
در فقر عهد کردم تا حرف کم کنم
&&
اما گلی که دید که پهلویش خار نیست
\\
ما خار این گلیم برادر گواه باش
&&
این جنس خار بودن فخرست عار نیست
\\
\end{longtable}
\end{center}
