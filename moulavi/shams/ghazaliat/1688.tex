\begin{center}
\section*{غزل شماره ۱۶۸۸: رفتم ز دست خود من در بیخودی فتادم}
\label{sec:1688}
\addcontentsline{toc}{section}{\nameref{sec:1688}}
\begin{longtable}{l p{0.5cm} r}
رفتم ز دست خود من در بیخودی فتادم
&&
در بیخودی مطلق با خود چه نیک شادم
\\
چشمم بدوخت دلبر تا غیر او نبینم
&&
تا چشم‌ها به ناگه در روی او گشادم
\\
با من به جنگ شد جان گفتا مرا مرنجان
&&
گفتم طلاق بستان گفتا بده بدادم
\\
مادر چو داغ عشقت می دید در رخ من
&&
نافم بر آن برید او آن دم که من بزادم
\\
گر بر فلک روانم ور لوح غیب خوانم
&&
ای تو صلاح جانم بی‌تو چه در فسادم
\\
ای پرده برفکنده تا مرده گشته زنده
&&
وز نور رویت آمد عهد الست یادم
\\
از عشق شاه پریان چون یاوه گشتم ای جان
&&
از خویش و خلق پنهان گویی پری نژادم
\\
تبریز شمس دین را گفتم تنا کی باشی
&&
تن گفت خاک و جان گفت سرگشته همچو بادم
\\
\end{longtable}
\end{center}
