\begin{center}
\section*{غزل شماره ۲۴۹۹: مسلمانان مسلمانان مرا ترکی است یغمایی}
\label{sec:2499}
\addcontentsline{toc}{section}{\nameref{sec:2499}}
\begin{longtable}{l p{0.5cm} r}
مسلمانان مسلمانان مرا ترکی است یغمایی
&&
که او صف‌های شیران را بدراند به تنهایی
\\
کمان را چون بجنباند بلرزد آسمان را دل
&&
فروافتد ز بیم او مه و زهره ز بالایی
\\
به پیش خلق نامش عشق و پیش من بلای جان
&&
بلا و محنتی شیرین که جز با وی نیاسایی
\\
چو او رخسار بنماید نماند کفر و تاریکی
&&
چو جعد خویش بگشاید نه دین ماند نه ترسایی
\\
مرا غیرت همی‌گوید خموش ار جانت می‌باید
&&
ز جان خویش بیزارم اگر دارد شکیبایی
\\
ندارد چاره دیوانه به جز زنجیر خاییدن
&&
حلالستت حلالستت اگر زنجیر می‌خایی
\\
بگو اسرار ای مجنون ز هشیاران چه می‌ترسی
&&
قبا بشکاف ای گردون قیامت را چه می‌پایی
\\
وگر پرواز عشق تو در این عالم نمی‌گنجد
&&
به سوی قاف قربت پر که سیمرغی و عنقایی
\\
اگر خواهی که حق گویم به من ده ساغر مردی
&&
وگر خواهی که ره بینم درآ ای چشم و بینایی
\\
در آتش بایدت بودن همه تن همچو خورشیدی
&&
اگر خواهی که عالم را ضیا و نور افزایی
\\
گدازان بایدت بودن چو قرص ماه اگر خواهی
&&
که از خورشید خورشیدان تو را باشد پذیرایی
\\
اگر دلگیر شد خانه نه پاگیر است برجه رو
&&
وگر نازک دلی منشین بر گیجان سودایی
\\
گهی سودای فاسد بین زمانی فاسد سودا
&&
گهی گم شو از این هر دو اگر همخرقه مایی
\\
به ترک ترک اولیتر سیه رویان هندو را
&&
که ترکان راست جانبازی و هندو راست لالایی
\\
منم باری بحمدالله غلام ترک همچون مه
&&
که مه رویان گردونی از او دارند زیبایی
\\
دهان عشق می‌خندد که نامش ترک گفتم من
&&
خود این او می‌دمد در ما که ما ناییم و او نایی
\\
چه نالد نای بیچاره جز آنک دردمد نایی
&&
ببین نی‌های اشکسته به گورستان چو می‌آیی
\\
بمانده از دم نایی نه جان مانده نه گویایی
&&
زبان حالشان گوید که رفت از ما من و مایی
\\
هلا بس کن هلا بس کن منه هیزم بر این آتش
&&
که می‌ترسم که این آتش بگیرد راه بالایی
\\
\end{longtable}
\end{center}
