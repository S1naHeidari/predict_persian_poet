\begin{center}
\section*{غزل شماره ۱۸۵۱: نشانی‌هاست در چشمش نشانش کن نشانش کن}
\label{sec:1851}
\addcontentsline{toc}{section}{\nameref{sec:1851}}
\begin{longtable}{l p{0.5cm} r}
نشانی‌هاست در چشمش نشانش کن نشانش کن
&&
ز من بشنو که وقت آمد کشانش کن کشانش کن
\\
برآمد آفتاب جان فزون از مشرق و مغرب
&&
بیا ای حاسد ار مردی نهانش کن نهانش کن
\\
از این نکته منم در خون خدا داند که چونم چون
&&
بیا ای جان روزافزون بیانش کن بیانش کن
\\
بیانش کرده گیر ای جان نه آن دریاست وان مرجان
&&
نیارامد به شرحش جان عیانش کن عیانش کن
\\
عیانش بود ما آمد زیانش سود ما آمد
&&
اگر تو سود جان خواهی زیانش کن زیانش کن
\\
یکی جان خواهد آن دریا همه آتش نهنگ آسا
&&
اگر داری چنین جانی روانش کن روانش کن
\\
هر آن کو بحربین باشد فلک پیشش زمین باشد
&&
هر آن کو نی چنین باشد چنانش کن چنانش کن
\\
برون جه از جهان زوتر درآ در بحر پرگوهر
&&
جهنده‌ست این جهان بنگر جهانش کن جهانش کن
\\
اگر خواهی که بگریزی ز شاه شمس تبریزی
&&
مپران تیر دعوی را کمانش کن کمانش کن
\\
\end{longtable}
\end{center}
