\begin{center}
\section*{غزل شماره ۴۰۶: چند گویی که چه چاره‌ست و مرا درمان چیست}
\label{sec:0406}
\addcontentsline{toc}{section}{\nameref{sec:0406}}
\begin{longtable}{l p{0.5cm} r}
چند گویی که چه چاره‌ست و مرا درمان چیست
&&
چاره جوینده که کرده‌ست تو را خود آن چیست
\\
چند باشد غم آنت که ز غم جان ببرم
&&
خود نباشد هوس آنک بدانی جان چیست
\\
بوی نانی که رسیده‌ست بر آن بوی برو
&&
تا همان بوی دهد شرح تو را کاین نان چیست
\\
گر تو عاشق شده‌ای عشق تو برهان تو بس
&&
ور تو عاشق نشدی پس طلب برهان چیست
\\
این قدر عقل نداری که ببینی آخر
&&
گر نه شاهیست پس این بارگه سلطان چیست
\\
گر نه اندر تتق ازرق زیباروییست
&&
در کف روح چنین مشعله تابان چیست
\\
چونک از دور دلت همچو زنان می‌لرزد
&&
تو چه دانی که در آن جنگ دل مردان چیست
\\
آتش دیده مردان حجب غیب بسوخت
&&
تو پس پرده نشسته که به غیب ایمان چیست
\\
شمس تبریز اگر نیست مقیم اندر چشم
&&
چشمه شهد از او در بن هر دندان چیست
\\
\end{longtable}
\end{center}
