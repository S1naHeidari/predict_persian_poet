\begin{center}
\section*{غزل شماره ۵۵: شب قدر است جسم تو کز او یابند دولت‌ها}
\label{sec:0055}
\addcontentsline{toc}{section}{\nameref{sec:0055}}
\begin{longtable}{l p{0.5cm} r}
شب قدر است جسم تو کز او یابند دولت‌ها
&&
مه بدرست روح تو کز او بشکافت ظلمت‌ها
\\
مگر تقویم یزدانی که طالع‌ها در او باشد
&&
مگر دریای غفرانی کز او شویند زلت‌ها
\\
مگر تو لوح محفوظی که درس غیب از او گیرند
&&
و یا گنجینه رحمت کز او پوشند خلعت‌ها
\\
عجب تو بیت معموری که طوافانش املاکند
&&
عجب تو رق منشوری کز او نوشند شربت‌ها
\\
و یا آن روح بی‌چونی کز این‌ها جمله بیرونی
&&
که در وی سرنگون آمد تأمل‌ها و فکرت‌ها
\\
ولی برتافت بر چون‌ها مشارق‌های بی‌چونی
&&
بر آثار لطیف تو غلط گشتند الفت‌ها
\\
عجایب یوسفی چون مه که عکس اوست در صد چه
&&
از او افتاده یعقوبان به دام و جاه ملت‌ها
\\
چو زلف خود رسن سازد ز چه‌هاشان براندازد
&&
کشدشان در بر رحمت رهاندشان ز حیرت‌ها
\\
چو از حیرت گذر یابد صفات آن را که دریابد
&&
خمش که بس شکسته شد عبارت‌ها و عبرت‌ها
\\
\end{longtable}
\end{center}
