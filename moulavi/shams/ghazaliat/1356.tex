\begin{center}
\section*{غزل شماره ۱۳۵۶: اگر درآید ناگه صنم زهی اقبال}
\label{sec:1356}
\addcontentsline{toc}{section}{\nameref{sec:1356}}
\begin{longtable}{l p{0.5cm} r}
اگر درآید ناگه صنم زهی اقبال
&&
چو در بتان زند آتش بتم زهی اقبال
\\
چنانک دی ز جمالش هزار توبه شکست
&&
اگر رسد عجب امروز هم زهی اقبال
\\
نشسته‌اند در اومید او قطار قطار
&&
اگر ز لطف نماید کرم زهی اقبال
\\
میان لشکر هجران که تیغ در تیغست
&&
سپاه وصل برآرد علم زهی اقبال
\\
هزار گل بنماید که خار مست شود
&&
هزار خنده برآرد ز غم زهی اقبال
\\
به رغم حرص شکم خوار خوان نهد با دل
&&
هزار کاسه کشد بی‌شکم زهی اقبال
\\
چو عشق دست برآرد سبک شود قالب
&&
دود بگرد فلک بی‌قدم زهی اقبال
\\
چو صبحدم برسد شاه شمس تبریزی
&&
چو آفتاب جهان بی‌حشم زهی اقبال
\\
\end{longtable}
\end{center}
