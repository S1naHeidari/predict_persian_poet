\begin{center}
\section*{غزل شماره ۱۴۳۴: چو آمد روی مه رویم که باشم من که من باشم}
\label{sec:1434}
\addcontentsline{toc}{section}{\nameref{sec:1434}}
\begin{longtable}{l p{0.5cm} r}
چو آمد روی مه رویم که باشم من که من باشم
&&
چو هر خاری از او گل شد چرا من یاسمن باشم
\\
چو هر سنگی عسل گردد چرا مومی کند مومی
&&
همه اجسام چون جان شد چرا استیزه تن باشم
\\
یقین هر چشم جو گردد چو آن آب روان آمد
&&
چو در جلوه‌ست حسن او چه بند بوالحسن باشم
\\
اگر چه در لگن بودم مثال شمع تا اکنون
&&
چو شمعم جمله گشت آتش چرا اندر لگن باشم
\\
چو از نحس زحل رستم چه زیر آسمان باشم
&&
چو محنت جمله دولت گشت از چه ممتحن باشم
\\
حسد بر من حسد دارد مرا بر کی حسد باشد
&&
ز جوی خمر چون مستم چرا تشنه لبن باشم
\\
\end{longtable}
\end{center}
