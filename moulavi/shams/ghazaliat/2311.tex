\begin{center}
\section*{غزل شماره ۲۳۱۱: از انبهی ماهی دریا به نهان گشته}
\label{sec:2311}
\addcontentsline{toc}{section}{\nameref{sec:2311}}
\begin{longtable}{l p{0.5cm} r}
از انبهی ماهی دریا به نهان گشته
&&
انبه شده قالب‌ها تا پرده جان گشته
\\
از فرقت آن دریا چون زهر شده شکر
&&
زهر از هوس دریا آب حیوان گشته
\\
در عشرت آن دریا نی این و نه آن بوده
&&
بر ساحل این خشکی این گشته و آن گشته
\\
اندر هوس دریا ای جان چو مرغابی
&&
چندان تو چنین گفته کز عشق چنان گشته
\\
دوش از شکم دریا برخاست یکی صورت
&&
و آن غمزه‌اش از دریا بس سخته کمان گشته
\\
دل گفت به زیر لب من جان نبرم از وی
&&
سوگند به جان دل کان کار چنان گشته
\\
از غمزه غمازی وز طرفه بغدادی
&&
دل گشته چنان شادی جانم همدان گشته
\\
در بیشه درافتاده در نیم شبی آتش
&&
در پختن این شیران تا مغز پزان گشته
\\
از شعله آن بیشه تابان شده اندیشه
&&
تا قالب جان پیشه بی‌جا و مکان گشته
\\
گرمابه روحانی آوخ چه پری خوان است
&&
وین عالم گورستان چون جامه کنان گشته
\\
از بهر چنین سری در سوسن‌ها بنگر
&&
دستوری گفتن نی سر جمله زبان گشته
\\
شمس الحق تبریزی درتافته از روزن
&&
تا آنچ نیارم گفت چون ماه عیان گشته
\\
\end{longtable}
\end{center}
