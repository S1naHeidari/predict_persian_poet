\begin{center}
\section*{غزل شماره ۱۸۶۹: رو مذهب عاشق را برعکس روش‌ها دان}
\label{sec:1869}
\addcontentsline{toc}{section}{\nameref{sec:1869}}
\begin{longtable}{l p{0.5cm} r}
رو مذهب عاشق را برعکس روش‌ها دان
&&
کز یار دروغی‌ها از صدق به و احسان
\\
حال است محال او مزد است وبال او
&&
عدل است همه ظلمش داد است از او بهتان
\\
نرم است درشت او کعبه‌ست کنشت او
&&
خاری که خلد دلبر خوشتر ز گل و ریحان
\\
آن دم که ترش باشد بهتر ز شکرخانه
&&
وان دل که ملول آید خوش بوس و کنار است آن
\\
وان دم که تو را گوید والله ز تو بیزارم
&&
آن آب خضر باشد از چشمه گه حیوان
\\
وان دم که بگوید نی در نیش هزار آری
&&
بیگانگیش خویشی در مذهب بی‌خویشان
\\
کفرش همه ایمان شد سنگش همه مرجان شد
&&
بخلش همه احسان شد جرمش همگی غفران
\\
گر طعنه زنی گویی تو مذهب کژ داری
&&
من مذهب ابرویش بخریدم و دادم جان
\\
زین مذهب کژ مستم بس کردم و لب بستم
&&
بردار دل روشن باقیش فرو می خوان
\\
شمس الحق تبریزی یا رب چه شکرریزی
&&
گویی ز دهان من صد حجت و صد برهان
\\
\end{longtable}
\end{center}
