\begin{center}
\section*{غزل شماره ۱۰۶۶: مرحبا ای جان باقی پادشاه کامیار}
\label{sec:1066}
\addcontentsline{toc}{section}{\nameref{sec:1066}}
\begin{longtable}{l p{0.5cm} r}
مرحبا ای جان باقی پادشاه کامیار
&&
روح بخش هر قران و آفتاب هر دیار
\\
این جهان و آن جهان هر دو غلام امر تو
&&
گر نخواهی برهمش زن ور همی‌خواهی بدار
\\
تابشی از آفتاب فقر بر هستی بتاب
&&
فارغ آور جملگان را از بهشت و خوف نار
\\
وارهان مر فاخران فقر را از ننگ جان
&&
در ره نقاش بشکن جمله این نقش و نگار
\\
قهرمانی را که خون صد هزاران ریخته‌ست
&&
ز آتش اقبال سرمد دود از جانش برآر
\\
آن کسی دریابد این اسرار لطفت را که او
&&
بی‌وجود خود برآید محو فقر از عین کار
\\
بی‌کراهت محو گردد جان اگر بیند که او
&&
چون زر سرخست خندان دل درون آن شرار
\\
ای که تو از اصل کان زر و گوهر بوده‌ای
&&
پس تو را از کیمیاهای جهان ننگست و عار
\\
جسم خاک از شمس تبریزی چو کلی کیمیاست
&&
تابش آن کیمیا را بر مس ایشان گمار
\\
\end{longtable}
\end{center}
