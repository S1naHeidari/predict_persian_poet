\begin{center}
\section*{غزل شماره ۱۶۰: مفروشید کمان و زره و تیغ زنان را}
\label{sec:0160}
\addcontentsline{toc}{section}{\nameref{sec:0160}}
\begin{longtable}{l p{0.5cm} r}
مفروشید کمان و زره و تیغ زنان را
&&
که سزا نیست سلح‌ها به جز از تیغ زنان را
\\
چه کند بنده صورت کمر عشق خدا را
&&
چه کند عورت مسکین سپر و گرز و سنان را
\\
چو میان نیست کمر را به کجا بندد آخر
&&
که وی از سنگ کشیدن بشکستست میان را
\\
زر و سیم و در و گوهر نه که سنگیست مزور
&&
ز پی سنگ کشیدن چو خری ساخته جان را
\\
منشین با دو سه ابله که بمانی ز چنین ره
&&
تو ز مردان خدا جو صفت جان و جهان را
\\
سوی آن چشم نظر کن که بود مست تجلی
&&
که در آن چشم بیابی گهر عین و عیان را
\\
تو در آن سایه بنه سر که شجر را کند اخضر
&&
که بدان جاست مجاری همگی امن و امان را
\\
گذر از خواب برادر به شب تیره چو اختر
&&
که به شب باید جستن وطن یار نهان را
\\
به نظربخش نظر کن ز میش بلبله تر کن
&&
سوی آن دور سفر کن چه کنی دور زمان را
\\
بپران تیر نظر را به مؤثر ده اثر را
&&
تبع تیر نظر دان تن مانند کمان را
\\
چو عدواید تو گردد چو کرم قید تو گردد
&&
چو یقین صید تو گردد بدران دام گمان را
\\
سوی حق چون بشتابی تو چو خورشید بتابی
&&
چو چنان سود بیابی چه کنی سود و زیان را
\\
هله ای ترش چو آلو بشنو بانگ تعالوا
&&
که گشادست به دعوت مه جاوید دهان را
\\
من از این فاتحه بستم لب خود باقی از او جو
&&
که درآکند به گوهر دهن فاتحه خوان را
\\
\end{longtable}
\end{center}
