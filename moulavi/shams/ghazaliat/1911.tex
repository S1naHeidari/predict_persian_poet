\begin{center}
\section*{غزل شماره ۱۹۱۱: تو را پندی دهم ای طالب دین}
\label{sec:1911}
\addcontentsline{toc}{section}{\nameref{sec:1911}}
\begin{longtable}{l p{0.5cm} r}
تو را پندی دهم ای طالب دین
&&
یکی پندی دلاویزی خوش آیین
\\
مشین غافل به پهلوی حریصان
&&
که جان گرگین شود از جان گرگین
\\
ز خارش‌های دل ار پاک گردی
&&
ز دل یابی حلاوت‌های والتین
\\
بجوشند از درون دل عروسان
&&
چو مرد حق شوی ای مرد عنین
\\
ز چشمه چشم پریان سر برآرند
&&
چو ماه و زهره و خورشید و پروین
\\
بنوش این را که تلقین‌های عشق است
&&
که سودت کم کند در گور تلقین
\\
به احسان زر به خوبان آن چنان ده
&&
که نفریبند زشتانت به تحسین
\\
نمی‌خواهند خوبان جز ممیز
&&
بمفریبان تو ایشان را به کابین
\\
ز تو آن گلرخان را ننگ آید
&&
چو بفروشی تو سرگی را به سرگین
\\
ز سنگ آسیا زیرین حمول است
&&
نه قیمت بیش دارد سنگ زیرین
\\
میان سنگ‌ها آن بیش ارزد
&&
که افزون خورده باشد زخم میتین
\\
ز اشکست تجلی فضل دارد
&&
میان کوه‌ها آن طور سینین
\\
خمش کن صبر کن تمکین تو کو
&&
که را ماند ز دست عشق تمکین
\\
\end{longtable}
\end{center}
