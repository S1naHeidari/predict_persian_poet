\begin{center}
\section*{غزل شماره ۱۹۴۷: سوی بیماران خود شد شاه مه رویان من}
\label{sec:1947}
\addcontentsline{toc}{section}{\nameref{sec:1947}}
\begin{longtable}{l p{0.5cm} r}
سوی بیماران خود شد شاه مه رویان من
&&
گفت ای رخ‌های زرد و زعفرانستان من
\\
زعفرانستان خود را آب خواهم داد آب
&&
زعفران را گل کنم از چشمه حیوان من
\\
زرد و سرخ و خار و گل در حکم و در فرمان ماست
&&
سر منه جز بر خط فرمان من فرمان من
\\
ماه رویان جهان از حسن ما دزدند حسن
&&
ذره‌ای دزدیده‌اند از حسن و از احسان من
\\
عاقبت آن ماه رویان کاه رویان می شوند
&&
حال دزدان این بود در حضرت سلطان من
\\
روز شد ای خاکیان دزدیده‌ها را رد کنید
&&
خاک را ملک از کجا حسن از کجا ای جان من
\\
شب چو شد خورشید غایب اختران لافی زنند
&&
زهره گوید آن من دان ماه گوید آن من
\\
مشتری از کیسه زر جعفری بیرون کند
&&
با زحل مریخ گوید خنجر بران من
\\
وان عطارد صدر گیرد که منم صدرالصدور
&&
چرخ‌ها ملک من است و برج‌ها ارکان من
\\
آفتاب از سوی مشرق صبحدم لشکر کشد
&&
گوید ای دزدان کجا رفتید اینک آن من
\\
زهره زهره درید و ماه را گردن شکست
&&
شد عطارد خشک و بارد با رخ رخشان من
\\
کار مریخ و زحل از نور ماهم درشکست
&&
مشتری مفلس برآمد کاه شد همیان من
\\
چون یکی میدان دوانید آفتاب آمد ندا
&&
هان و هان ای بی‌ادب بیرون شو از میدان من
\\
آفتاب آفتابم آفتابا تو برو
&&
در چه مغرب فرورو باش در زندان من
\\
وقت صبح از گور مشرق سر برآر و زنده شو
&&
منکران حشر را آگه کن از برهان من
\\
عید هر کس آن مهی باشد که او قربان او است
&&
عید تو ماه من آمد ای شده قربان من
\\
شمس تبریزی چو تافت از برج لاشرقیه
&&
تاب ذات او برون شد از حد و امکان من
\\
\end{longtable}
\end{center}
