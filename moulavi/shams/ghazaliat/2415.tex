\begin{center}
\section*{غزل شماره ۲۴۱۵: تو دیده گشته و ما را بکرده نادیده}
\label{sec:2415}
\addcontentsline{toc}{section}{\nameref{sec:2415}}
\begin{longtable}{l p{0.5cm} r}
تو دیده گشته و ما را بکرده نادیده
&&
بدیده گریه ما را بدین بخندیده
\\
بخند جان و جهان چون مقام خنده تو راست
&&
بکن که هر چه کنی هست بس پسندیده
\\
ز درد و حسرت تو جان لاله‌ها سیه است
&&
گل از جمال رخ توست جامه بدریده
\\
ز خلق عالم جان‌های پاک بگزیدند
&&
و آنگهان ز میانشان تو بوده بگزیده
\\
بدانک عشق نبات و درخت او خشک است
&&
به گرد گرد درخت من است پیچیده
\\
چو خشک گشت درختم بسی بلندی یافت
&&
چو زرد گشت رخم شد چو زر بنازیده
\\
خزینه‌های جواهر که این دلم را بود
&&
قمارخانه درون جمله را ببازیده
\\
هزار ساغر هستی شکسته این دل من
&&
خمار نرگس مخمور تو نسازیده
\\
ز خام و پخته تهی گشت جان من باری
&&
مدد مدد تو چنین آتشی فروزیده
\\
مرا چو نی بنوازید شمس تبریزی
&&
بهانه بر نی و مطرب ز غم خروشیده
\\
\end{longtable}
\end{center}
