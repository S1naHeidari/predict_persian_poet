\begin{center}
\section*{غزل شماره ۵۳۳: رندان سلامت می‌کنند جان را غلامت می‌کنند}
\label{sec:0533}
\addcontentsline{toc}{section}{\nameref{sec:0533}}
\begin{longtable}{l p{0.5cm} r}
رندان سلامت می‌کنند جان را غلامت می‌کنند
&&
مستی ز جامت می‌کنند مستان سلامت می‌کنند
\\
در عشق گشتم فاشتر وز همگنان قلاشتر
&&
وز دلبران خوش باشتر مستان سلامت می‌کنند
\\
غوغای روحانی نگر سیلاب طوفانی نگر
&&
خورشید ربانی نگر مستان سلامت می‌کنند
\\
افسون مرا گوید کسی توبه ز من جوید کسی
&&
بی پا چو من پوید کسی مستان سلامت می‌کنند
\\
ای آرزوی آرزو آن پرده را بردار زو
&&
من کس نمی‌دانم جز او مستان سلامت می‌کنند
\\
ای ابر خوش باران بیا وی مستی یاران بیا
&&
وی شاه طراران بیا مستان سلامت می‌کنند
\\
حیران کن و بی‌رنج کن ویران کن و پرگنج کن
&&
نقد ابد را سنج کن مستان سلامت می‌کنند
\\
شهری ز تو زیر و زبر هم بی‌خبر هم باخبر
&&
وی از تو دل صاحب نظر مستان سلامت می‌کنند
\\
آن میر مه رو را بگو وان چشم جادو را بگو
&&
وان شاه خوش خو را بگو مستان سلامت می‌کنند
\\
آن میر غوغا را بگو وان شور و سودا را بگو
&&
وان سرو خضرا را بگو مستان سلامت می‌کنند
\\
آن جا که یک باخویش نیست یک مست آن جا بیش نیست
&&
آن جا طریق و کیش نیست مستان سلامت می‌کنند
\\
آن جان بی‌چون را بگو وان دام مجنون را بگو
&&
وان در مکنون را بگو مستان سلامت می‌کنند
\\
آن دام آدم را بگو وان جان عالم را بگو
&&
وان یار و همدم را بگو مستان سلامت می‌کنند
\\
آن بحر مینا را بگو وان چشم بینا را بگو
&&
وان طور سینا را بگو مستان سلامت می‌کنند
\\
آن توبه سوزم را بگو وان خرقه دوزم را بگو
&&
وان نور روزم را بگو مستان سلامت می‌کنند
\\
آن عید قربان را بگو وان شمع قرآن را بگو
&&
وان فخر رضوان را بگو مستان سلامت می‌کنند
\\
ای شه حسام الدین ما ای فخر جمله اولیا
&&
ای از تو جان‌ها آشنا مستان سلامت می‌کنند
\\
\end{longtable}
\end{center}
