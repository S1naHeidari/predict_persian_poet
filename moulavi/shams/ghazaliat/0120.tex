\begin{center}
\section*{غزل شماره ۱۲۰: تا چند تو پس روی به پیش آ}
\label{sec:0120}
\addcontentsline{toc}{section}{\nameref{sec:0120}}
\begin{longtable}{l p{0.5cm} r}
تا چند تو پس روی به پیش آ
&&
در کفر مرو به سوی کیش آ
\\
در نیش تو نوش بین به نیش آ
&&
آخر تو به اصل اصل خویش آ
\\
هر چند به صورت از زمینی
&&
پس رشته گوهر یقینی
\\
بر مخزن نور حق امینی
&&
آخر تو به اصل اصل خویش آ
\\
خود را چو به بیخودی ببستی
&&
می‌دانک تو از خودی برستی
\\
وز بند هزار دام جستی
&&
آخر تو به اصل اصل خویش آ
\\
از پشت خلیفه‌ای بزادی
&&
چشمی به جهان دون گشادی
\\
آوه که بدین قدر تو شادی
&&
آخر تو به اصل اصل خویش آ
\\
هر چند طلسم این جهانی
&&
در باطن خویشتن تو کانی
\\
بگشای دو دیده نهانی
&&
آخر تو به اصل اصل خویش آ
\\
چون زاده پرتو جلالی
&&
وز طالع سعد نیک فالی
\\
از هر عدمی تو چند نالی
&&
آخر تو به اصل اصل خویش آ
\\
لعلی به میان سنگ خارا
&&
تا چند غلط دهی تو ما را
\\
در چشم تو ظاهرست یارا
&&
آخر تو به اصل اصل خویش آ
\\
چون از بر یار سرکش آیی
&&
سرمست و لطیف و دلکش آیی
\\
با چشم خوش و پرآتش آیی
&&
آخر تو به اصل اصل خویش آ
\\
در پیش تو داشت جام باقی
&&
شمس تبریز شاه و ساقی
\\
سبحان الله زهی رواقی
&&
آخر تو به اصل اصل خویش آ
\\
\end{longtable}
\end{center}
