\begin{center}
\section*{غزل شماره ۲۶۷۸: اگر خورشید جاویدان نگشتی}
\label{sec:2678}
\addcontentsline{toc}{section}{\nameref{sec:2678}}
\begin{longtable}{l p{0.5cm} r}
اگر خورشید جاویدان نگشتی
&&
درخت و رخت بازرگان نگشتی
\\
دو دست کفشگر گر ساکنستی
&&
همیشه گربه در انبان نگشتی
\\
اگر نه عشوه‌های باد بودی
&&
سر شاخ گل خندان نگشتی
\\
چه گویم گر نبودی آن که دانی
&&
به هر دم این نگشتی آن نگشتی
\\
فلک چتر است و سلطان عقل کلی
&&
نگشتی چتر اگر سلطان نگشتی
\\
اگر آواز سرهنگان نبودی
&&
نگشتی اختر و کیوان نگشتی
\\
کریمی گر ندادی ابر و باران
&&
یکی جرعه به گرد خوان نگشتی
\\
درونت گر نبودی کیمیاگر
&&
به هر دم خون و بلغم جان نگشتی
\\
نهان از عالم ار نی عالمستی
&&
دل تاریک تو میدان نگشتی
\\
نهان دار این سخن را ز آنک زرها
&&
اگر پنهان نبودی کان نگشتی
\\
\end{longtable}
\end{center}
