\begin{center}
\section*{غزل شماره ۲۵۹۷: گر روی بگردانی تو پشت قوی داری}
\label{sec:2597}
\addcontentsline{toc}{section}{\nameref{sec:2597}}
\begin{longtable}{l p{0.5cm} r}
گر روی بگردانی تو پشت قوی داری
&&
کان روی چو خورشیدت صد گون کندت یاری
\\
من بی‌رخ چون ماهت گر روی به ماه آرم
&&
مه بی‌تو ز من گیرد صد دوری و بیزاری
\\
جان بی‌تو یتیم آمد مه بی‌تو دو نیم آمد
&&
گلزار جفا گردد چون تخم جفا کاری
\\
چون سرکشی آغازی یا اسب جفا تازی
&&
دست کی رسد در تو گر پای نیفشاری
\\
مهمان توام ای جان ای شادی هر مهمان
&&
شاید که ز بخشایش این دم سر من خاری
\\
رو ای دل بیچاره با تیغ و کفن پیشش
&&
کی پیش رود با او بدفعلی و طراری
\\
ای جان نه ز باغ تو رسته‌ست درخت من
&&
پرورده و خو کرده با عشرت و خماری
\\
اجزای وجود من مستان تواند ای جان
&&
مستان مرا مفکن در نوحه و در زاری
\\
آن ساغر پنهانی خواهم که بگردانی
&&
مستانه به پیش آیی بی‌نخوت و جباری
\\
ای ساغر پنهانی تو جامی و یا جانی
&&
یا چشمه حیوانی یا صحت بیماری
\\
یا آب حیاتی تو یا خط نجاتی تو
&&
یا کان نباتی تو یا ابر شکرباری
\\
آن ساغر و آن کوزه کو نشکندم روزه
&&
اما نهلد در سر نی عقل نی هشیاری
\\
هم عقلی و هم جانی هم اینی و هم آنی
&&
هم آبی و هم نانی هم یاری و هم غاری
\\
خاموش شدم حاصل تا برنپرد این دل
&&
نی زان که سخن کم شد از غایت بسیاری
\\
\end{longtable}
\end{center}
