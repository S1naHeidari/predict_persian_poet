\begin{center}
\section*{غزل شماره ۹۴۸: کسی خراب خرابات و مست می‌باشد}
\label{sec:0948}
\addcontentsline{toc}{section}{\nameref{sec:0948}}
\begin{longtable}{l p{0.5cm} r}
کسی خراب خرابات و مست می‌باشد
&&
از او عمارت ایمان و خیر کی باشد
\\
یکی وجود چو آتش بود نباشد آب
&&
محال باشد یک مه بهار و دی باشد
\\
منم خراب خرابات و مست طاعت حق
&&
درون شهر معظم ز نیک و بی‌باشد
\\
عمارتیست خراباتیان شهر مرا
&&
که خانه‌هاش نهان در زمین چو ری باشد
\\
شکوفه‌هاست درختان زهد را ز شراب
&&
نه آن شراب که اشکوفه‌هاش قی باشد
\\
چو هست و نیست مرا دید چشم معتزلی
&&
بگفت دیدم معدوم را که شیء باشد
\\
به سایه‌ها و به خورشید شمس تبریزی
&&
که بی‌مکان و زمان آفتاب و فی باشد
\\
\end{longtable}
\end{center}
