\begin{center}
\section*{غزل شماره ۱۹۹۳: هیچ باشد که رسد آن شکر و پسته من}
\label{sec:1993}
\addcontentsline{toc}{section}{\nameref{sec:1993}}
\begin{longtable}{l p{0.5cm} r}
هیچ باشد که رسد آن شکر و پسته من
&&
نقل سازد جهت این جگر خسته من
\\
دست خود بر سر من مالد از روی کرم
&&
که تو چونی هله ای بی‌دل و پابسته من
\\
سر گران گشته از آن باده بی‌ساغر من
&&
زعفران کشته بدین لاله بررسته من
\\
زخم بر تار تو اندرخور خود چون رانم
&&
ای گسسته رگت از زخمه آهسته من
\\
چون تنم جان نشود زان ابدی آب حیات
&&
چون دلم برنجهد زان بت برجسته من
\\
هله ای طیف خیالش بنشین و بشنو
&&
یک زمانی سخن پخته به نبشته من
\\
چون مه چارده شب را تو برآرای به حسن
&&
ای به شب‌ها و سحرها به دعا جسته من
\\
چند صف‌ها بشکستی و بدیدی همه را
&&
هیچ دیدی تو صفی چون صف اشکسته من
\\
لاله زار و چمن ار چه که همه ملک وی است
&&
هوس و رغبت او بین تو به گلدسته من
\\
لب ببند و قصص عشق به گوش او گوی
&&
که حریص آمد بر گفتن پیوسته من
\\
\end{longtable}
\end{center}
