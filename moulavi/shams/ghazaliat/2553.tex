\begin{center}
\section*{غزل شماره ۲۵۵۳: کجا شد عهد و پیمانی که می‌کردی نمی‌گویی}
\label{sec:2553}
\addcontentsline{toc}{section}{\nameref{sec:2553}}
\begin{longtable}{l p{0.5cm} r}
کجا شد عهد و پیمانی که می‌کردی نمی‌گویی
&&
کسی را کو به جان و دل تو را جوید نمی‌جویی
\\
دل افکاری که روی خود به خون دیده می‌شوید
&&
چرا از وی نمی‌داری دو دست خود نمی‌شویی
\\
مثال تیر مژگانت شدم من راست یک سانت
&&
چرا ای چشم بخت من تو با من کژ چو ابرویی
\\
چه با لذت جفاکاری که می‌بکشی بدین زاری
&&
پس آنگه عاشق کشته تو را گوید چو خوش خویی
\\
ز شیران جمله آهویان گریزان دیدم و پویان
&&
دلا جویای آن شیری خدا داند چه آهویی
\\
دلا گر چه نزاری تو مقیم کوی یاری تو
&&
مرا بس شد ز جان و تن تو را مژده کز آن کویی
\\
به پیش شاه خوش می‌دو گهی بالا و گه در گو
&&
از او ضربت ز تو خدمت که او چوگان و تو گویی
\\
دلا جستیم سرتاسر ندیدم در تو جز دلبر
&&
مخوان ای دل مرا کافر اگر گویم که تو اویی
\\
غلام بیخودی ز آنم که اندر بیخودی آنم
&&
چو بازآیم به سوی خود من این سویم تو آن سویی
\\
خمش کن کز ملامت او بدان ماند که می‌گوید
&&
زبان تو نمی‌دانم که من ترکم تو هندویی
\\
\end{longtable}
\end{center}
