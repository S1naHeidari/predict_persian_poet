\begin{center}
\section*{غزل شماره ۱۳۱۱: باز از آن کوه قاف آمد عنقای عشق}
\label{sec:1311}
\addcontentsline{toc}{section}{\nameref{sec:1311}}
\begin{longtable}{l p{0.5cm} r}
باز از آن کوه قاف آمد عنقای عشق
&&
باز برآمد ز جان نعره و هیهای عشق
\\
باز برآورد عشق سر به مثال نهنگ
&&
تا شکند زورق عقل به دریای عشق
\\
سینه گشادست فقر جانب دل‌های پاک
&&
در شکم طور بین سینه سینای عشق
\\
مرغ دل عاشقان باز پر نو گشاد
&&
کز قفس سینه یافت عالم پهنای عشق
\\
هر نفس آید نثار بر سر یاران کار
&&
از بر جانان که اوست جان و دل افزای عشق
\\
فتنه نشان عقل بود رفت و به یک سو نشست
&&
هر طرف اکنون ببین فتنه دروای عشق
\\
عقل بدید آتشی گفت که عشقست و نی
&&
عشق ببیند مگر دیده بینای عشق
\\
عشق ندای بلند کرد به آواز پست
&&
کای دل بالا بپر بنگر بالای عشق
\\
بنگر در شمس دین خسرو تبریزیان
&&
شادی جان‌های پاک دیده دل‌های عشق
\\
\end{longtable}
\end{center}
