\begin{center}
\section*{غزل شماره ۲۱۹۶: از حلاوت‌ها که هست از خشم و از دشنام او}
\label{sec:2196}
\addcontentsline{toc}{section}{\nameref{sec:2196}}
\begin{longtable}{l p{0.5cm} r}
از حلاوت‌ها که هست از خشم و از دشنام او
&&
می‌ستیزم هر شبی با چشم خون آشام او
\\
دام‌های عشق او گر پر و بالم بسکلد
&&
طوطی جان نسکلد از شکر و بادام او
\\
چند پرسی مر مرا از وحشت و شب‌های هجر
&&
شب کجا ماند بگو در دولت ایام او
\\
خون ما را رنگ خون و فعل می‌آمد از آنک
&&
خون‌ها می می‌شود چون می‌رود در جام او
\\
وعده‌های خام او در مغز جان جوشان شده
&&
عاشقان پخته بین از وعده‌های خام او
\\
خسروان بر تخت دولت بین که حسرت می‌خورند
&&
در لقای عاشقان کشته بدنام او
\\
آن سگان کوی او شاهان شیران گشته‌اند
&&
کان چنان آهوی فتنه دیده شد بر بام او
\\
الله الله تو مپرس از باخودان اوصاف می
&&
تو ببین در چشم مستان لطف‌های عام او
\\
دست بر رگ‌های مستان نه دلا تا پی بری
&&
از دهان آلودگان زان باده خودکام او
\\
شمس تبریزی که گامش بر سر ارواح بود
&&
پا منه تو سر بنه بر جایگاه گام او
\\
\end{longtable}
\end{center}
