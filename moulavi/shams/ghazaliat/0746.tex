\begin{center}
\section*{غزل شماره ۷۴۶: برنشست آن شاه عشق و دام ظلمت بردرید}
\label{sec:0746}
\addcontentsline{toc}{section}{\nameref{sec:0746}}
\begin{longtable}{l p{0.5cm} r}
برنشست آن شاه عشق و دام ظلمت بردرید
&&
همچو ماه هفت و هشت و آفتاب روز عید
\\
اختران در خدمت او صد هزار اندر هزار
&&
هر یکی از نور روی او مزید اندر مزید
\\
چون در آن دور مبارک برج‌ها را می‌گذشت
&&
سوی برج آتشین عاشقان خود رسید
\\
در دلش یاد من آمد هر طرف کرد التفات
&&
مر مرا در هیچ صفی آن زمان آن جا ندید
\\
موج دریاهای رحمت از دلش در جوش شد
&&
هم نظر می‌کرد هر سو هم عنان را می‌کشید
\\
گفت نزدیکان خود را کان فلان غایت چراست
&&
آن خراب عاشق حاضرمثال ناپدید
\\
آنک دیده هر شبش در سوختن مانند شمع
&&
آنک هر صبحی که آمد ناله‌های او شنید
\\
آنک آتش‌های عالم ز آتش او کاغ کرد
&&
تا فسون می‌خواند عشق و بر دل او می‌دمید
\\
آن یکی خاکی که چون مهتاب بر وی تافتیم
&&
همچو مهتاب از ثری سوی ثریا می‌دوید
\\
آنک چون جرجیس اندر امتحان عشق ما
&&
گشت او صد بار زنده کشته شد صد ره شهید
\\
آنک حامل شد عدم از آفرینش بخت نیک
&&
ناف او بر عشق شمس الدین تبریزی برید
\\
\end{longtable}
\end{center}
