\begin{center}
\section*{غزل شماره ۷۷۸: از دلم صورت آن خوب ختن می‌نرود}
\label{sec:0778}
\addcontentsline{toc}{section}{\nameref{sec:0778}}
\begin{longtable}{l p{0.5cm} r}
از دلم صورت آن خوب ختن می‌نرود
&&
چاشنی شکر او ز دهن می‌نرود
\\
بالله ار شور کنم هر نفسی عیب مکن
&&
گر برفت از دل تو از دل من می‌نرود
\\
بوالحسن گفت حسن را که از این خانه برو
&&
بوالحسن نیز درافتاد و حسن می‌نرود
\\
جان پروانه مسکین ز پی شعله شمع
&&
تا نسوزد پر و بالش ز لگن می‌نرود
\\
همه مرغان چمن هر طرفی می‌پرند
&&
بلبل از واسطه گل ز چمن می‌نرود
\\
مرغ جان هر نفسی بال گشاید که پرد
&&
وز امید نظر دوست ز تن می‌نرود
\\
زن ز شوهر ببرد چون به تو آسیب زند
&&
مرد چون روی تو بیند سوی زن می‌نرود
\\
جان منصور چو در عشق توش دار زدند
&&
در رسن کرد سر خود ز رسن می‌نرود
\\
جان ادیم و تو سهیلی و هوای تو یمن
&&
از پی تربیت تو ز یمن می‌نرود
\\
چون خیال شکن زلف تو در دل دارم
&&
این شکسته دلم از عشق شکن می‌نرود
\\
گر سبو بشکند آن آب سبو کی شکند
&&
جان عاشق به سوی گور و کفن می‌نرود
\\
حیله‌ها دانم و تلبیسک و کژبازی‌ها
&&
جان ز شرم تو به تلبیس و به فن می‌نرود
\\
\end{longtable}
\end{center}
