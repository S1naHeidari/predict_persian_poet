\begin{center}
\section*{غزل شماره ۱۶: ای یوسف آخر سوی این یعقوب نابینا بیا}
\label{sec:0016}
\addcontentsline{toc}{section}{\nameref{sec:0016}}
\begin{longtable}{l p{0.5cm} r}
ای یوسف آخر سوی این یعقوب نابینا بیا
&&
ای عیسی پنهان شده بر طارم مینا بیا
\\
از هجر روزم قیر شد دل چون کمان بد تیر شد
&&
یعقوب مسکین پیر شد ای یوسف برنا بیا
\\
ای موسی عمران که در سینه چه سیناهاستت
&&
گاوی خدایی می‌کند از سینه سینا بیا
\\
رخ زعفران رنگ آمدم خم داده چون چنگ آمدم
&&
در گور تن تنگ آمدم ای جان باپهنا بیا
\\
چشم محمد با نمت واشوق گفته در غمت
&&
زان طره‌ای اندرهمت ای سر ارسلنا بیا
\\
خورشید پیشت چون شفق ای برده از شاهان سبق
&&
ای دیده بینا به حق وی سینه دانا بیا
\\
ای جان تو و جان‌ها چو تن بی‌جان چه ارزد خود بدن
&&
دل داده‌ام دیر است من تا جان دهم جانا بیا
\\
تا برده‌ای دل را گرو شد کشت جانم در درو
&&
اول تو ای دردا برو و آخر تو درمانا بیا
\\
ای تو دوا و چاره‌ام نور دل صدپاره‌ام
&&
اندر دل بیچاره‌ام چون غیر تو شد لا بیا
\\
نشناختم قدر تو من تا چرخ می‌گوید ز فن
&&
دی بر دلش تیری بزن دی بر سرش خارا بیا
\\
ای قاب قوس مرتبت وان دولت بامکرمت
&&
کس نیست شاها محرمت در قرب او ادنی بیا
\\
ای خسرو مه وش بیا ای خوشتر از صد خوش بیا
&&
ای آب و ای آتش بیا ای در و ای دریا بیا
\\
مخدوم جانم شمس دین از جاهت ای روح الامین
&&
تبریز چون عرش مکین از مسجد اقصی بیا
\\
\end{longtable}
\end{center}
