\begin{center}
\section*{غزل شماره ۱۹۶۴: شمس دین بر یوسفان و نازنینان نازنین}
\label{sec:1964}
\addcontentsline{toc}{section}{\nameref{sec:1964}}
\begin{longtable}{l p{0.5cm} r}
شمس دین بر یوسفان و نازنینان نازنین
&&
بر سر جمله شهان و سرفرازان نازنین
\\
بر سران و سروران صد سر زیاده جاه او
&&
در میان واصلان لطف رحمان نازنین
\\
او به اوصاف الهی گشته موصوف کمال
&&
بر سریر و بر سران تخت سلطان نازنین
\\
بزم را از وی جمال و رزم را از وی جلال
&&
هم به بزم و هم به رزم لطف کیهان نازنین
\\
پیش او بنهاد مفتاح خزاین‌های خاص
&&
کرده از عشق و محبت‌هاش یزدان نازنین
\\
در میان صد هزاران ماه او تابان چو خور
&&
وصف او اندر میان وصف شاهان نازنین
\\
آنک خاک پاش شد او بر سران شد سرفراز
&&
مست او اندر میان جمله مستان نازنین
\\
اندر آن موجی که خاصان بر حذر باشند از آن
&&
اندر آن موج خطر او خفته استان نازنین
\\
\end{longtable}
\end{center}
