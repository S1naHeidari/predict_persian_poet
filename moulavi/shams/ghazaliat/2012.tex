\begin{center}
\section*{غزل شماره ۲۰۱۲: ای دلارام من و ای دل شکن}
\label{sec:2012}
\addcontentsline{toc}{section}{\nameref{sec:2012}}
\begin{longtable}{l p{0.5cm} r}
ای دلارام من و ای دل شکن
&&
وی کشیده خویش بی‌جرمی ز من
\\
از نظر رفتی ز دل بیرون نه‌ای
&&
ز آنک تو شمعی و جان و دل لگن
\\
جان من جان تو جانت جان من
&&
هیچ کس دیده‌ست یک جان در دو تن
\\
زندگی‌ام وصل تو مرگم فراق
&&
بی‌نظیرم کرده‌ای اندر دو فن
\\
بس بجستم آب حیوان خضر گفت
&&
بی‌وصالش جان نیابی جان مکن
\\
غم نیارد گرد غمگین تو گشت
&&
ور بگردد بایدش گردن زدن
\\
جان‌ها زان گرد تو گرددهمی
&&
جان ادیم و تو سهیل اندر یمن
\\
بهر تو گفته‌ست منصور حلاج
&&
یا صغیر السن یا رطب البدن
\\
شیر مست شهد تو گشت و بگفت
&&
یا قریب العهد من شرب اللبن
\\
پیش مستان تو غم را راه نیست
&&
فکرت و غم هست کار بوالحسن
\\
هر کی در چاه طبیعت مانده است
&&
چاره‌اش نبود ز فکر چون رسن
\\
چونک برپرید کاسد گشت حبل
&&
چون یقینی یافت کاسد گشت ظن
\\
همزبان بی‌زبانان شو دلا
&&
تا به گفت و گو نباشی مرتهن
\\
\end{longtable}
\end{center}
