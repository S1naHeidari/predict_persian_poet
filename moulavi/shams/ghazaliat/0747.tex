\begin{center}
\section*{غزل شماره ۷۴۷: ای طربناکان ز مطرب التماس می‌کنید}
\label{sec:0747}
\addcontentsline{toc}{section}{\nameref{sec:0747}}
\begin{longtable}{l p{0.5cm} r}
ای طربناکان ز مطرب التماس می‌کنید
&&
سوی عشرت‌ها روید و میل بانگ نی کنید
\\
شهسوار اسب شادی‌ها شوید ای مقبلان
&&
اسب غم را در قدم‌های طرب‌ها پی کنید
\\
زان می صافی ز خم وحدتش ای باخودان
&&
عقل و هوش و عاقبت بینی همه لاشیء کنید
\\
نوبهاری هست با صد رنگ گلزار و چمن
&&
ترک سرد و خشک و ادباری ماه دی کنید
\\
کشتگان خواهید دیدن سربریده جوق جوق
&&
ایها العشاق مرتدید اگر هی هی کنید
\\
سوی چینست آن بت چینی که طالب گشته‌اید
&&
این چه عقلست این که هر دم قصد راه ری کنید
\\
در خرابات بقا اندر سماع گوش جان
&&
ترک تکرار حروف ابجد و حطی کنید
\\
از شراب صرف باقی کاسه سر پر کنید
&&
فرش عقل و عاقلی از بهر لله طی کنید
\\
از صفات باخودی بیرون شوید ای عاشقان
&&
خویشتن را محو دیدار جمال حی کنید
\\
با شه تبریز شمس الدین خداوند شهان
&&
جان فدا دارید و تن قربان ز بهر وی کنید
\\
\end{longtable}
\end{center}
