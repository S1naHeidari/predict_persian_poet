\begin{center}
\section*{غزل شماره ۲۳۰۹: من بیخود و تو بیخود ما را کی برد خانه}
\label{sec:2309}
\addcontentsline{toc}{section}{\nameref{sec:2309}}
\begin{longtable}{l p{0.5cm} r}
من بیخود و تو بیخود  ما را کی برد خانه
&&
من چند تو را  گفتم کم خور دو سه پیمانه
\\
در شهر یکی کس را هشیار نمی‌بینم
&&
هر یک بتر از دیگر شوریده و دیوانه
\\
جانا به خرابات آ تا لذت جان بینی
&&
جان را چه خوشی باشد بی‌صحبت جانانه
\\
هر گوشه یکی مستی دستی ز بر دستی
&&
و آن ساقی هر هستی با ساغر شاهانه
\\
تو وقف خراباتی دخلت می و خرجت می
&&
زین وقف به هشیاران مسپار یکی دانه
\\
ای لولی بربط زن تو مستتری یا من
&&
ای پیش چو تو مستی افسون من افسانه
\\
از خانه برون رفتم مستیم به پیش آمد
&&
در هر نظرش مضمر صد گلشن و کاشانه
\\
چون کشتی بی‌لنگر کژ می‌شد و مژ می‌شد
&&
وز حسرت او مرده صد عاقل و فرزانه
\\
گفتم ز کجایی تو تسخر زد و گفت ای جان
&&
نیمیم ز ترکستان نیمیم ز فرغانه
\\
نیمیم ز آب و گل نیمیم ز جان و دل
&&
نیمیم لب دریا نیمی همه دردانه
\\
گفتم که رفیقی کن با من که منم خویشت
&&
گفتا که بنشناسم من خویش ز بیگانه
\\
من بی‌دل و دستارم در خانه خمارم
&&
یک سینه سخن دارم هین شرح دهم یا نه
\\
در حلقه لنگانی می‌باید لنگیدن
&&
این پند ننوشیدی از خواجه علیانه
\\
سرمست چنان خوبی کی کم بود از چوبی
&&
برخاست فغان آخر از استن حنانه
\\
شمس الحق تبریزی از خلق چه پرهیزی
&&
اکنون که درافکندی صد فتنه فتانه
\\
\end{longtable}
\end{center}
