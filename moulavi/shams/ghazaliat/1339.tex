\begin{center}
\section*{غزل شماره ۱۳۳۹: مهم را لطف در لطفست از آنم بی‌قرار ای دل}
\label{sec:1339}
\addcontentsline{toc}{section}{\nameref{sec:1339}}
\begin{longtable}{l p{0.5cm} r}
مهم را لطف در لطفست از آنم بی‌قرار ای دل
&&
دلم پرچشمه حیوان تنم در لاله زار ای دل
\\
به زیر هر درختی بین نشسته بهر روی شه
&&
ملیحی یوسفی مه رو لطیفی گلعذار ای دل
\\
فکنده در دل خوبان روحانی و جسمانی
&&
ز عشق روح و جسم خود ز سوداها شرار ای دل
\\
درآکنده ز شادی‌ها درون چاکران خود
&&
مثال دانه‌های در که باشد در انار ای دل
\\
به بزم او چو مستان را کنار و لطف‌ها باشد
&&
بگیرد آب با آتش ز عشقش هم کنار ای دل
\\
در آن خلوت که خوبان را به جام خاص بنوازد
&&
بود روح الامین حارس و خضرش پرده دار ای دل
\\
چو از بزمش برون آید کمینه چاکرش سکران
&&
ز ملک و ملک و تخت و بخت دارد ننگ و عار ای دل
\\
جهان بستان او را دان و این عالم چو غاری دان
&&
برون آرد تو را لطفش از این تاریک غار ای دل
\\
گلستان‌ها و ریحان‌ها شقایق‌های گوناگون
&&
بنفشه زارها بر خاک و باد و آب و نار ای دل
\\
که این گل‌های خاکی هم ز عکس آن همی‌روید
&&
تو خاکی می‌خوری این جا تو را آن جا چه کار ای دل
\\
بزن دستی و رقصی کن ز عشق آن خداوندان
&&
که چون بوسی از او یابی کند آفت کنار ای دل
\\
به جان پاک شمس الدین خداوند خداوندان
&&
که پرها هم از او یابی اگر خواهی فرار ای دل
\\
به خاک پای تبریزی که اکسیرست خاک او
&&
که جان‌ها یابی ار بر وی کنی جانی نثار ای دل
\\
کنون از هجر بر پایم چنین بندیست از آتش
&&
ز یادش مست و مخمورم اگر چندم نزار ای دل
\\
مثال چنگ می‌باشم هزاران نغمه‌ها دارد
&&
به لحن عشق انگیزش وگر نالید زار ای دل
\\
به سودای چنان بختی که معشوق از سر دستی
&&
به دستم داده بود از لطف دنبال مهار ای دل
\\
بگرد مرکبم بودی به زیر سایه آن شاه
&&
هزاران شاه در خدمت به صف‌ها در قطار ای دل
\\
از این سو نه از آن سوی جهان روح تا دانی
&&
که آن جا که نه امسالست و آن سالست پار ای دل
\\
چو دیدم من عنایت‌ها ز صدر غیب شمس الدین
&&
شدم مغرور خاصه مست و مجنون خمار ای دل
\\
چنان حلمی و تمکینی چنان صبر خداوندی
&&
که اندر صبر ایوبش نتاند بود یار ای دل
\\
عنان از من چنان برتافت جایی شد که وهم آن جا
&&
به جسم او نیابد راه و نی چشمش غبار ای دل
\\
به درگاه خدا نالم که سایه آفتابی را
&&
به ما آرد که دل را نیست بی او پود و تار ای دل
\\
امیدست ای دل غمگین که ناگاهان درآید او
&&
تو این جان را به صد حیله همی‌کن داردار ای دل
\\
\end{longtable}
\end{center}
