\begin{center}
\section*{غزل شماره ۸۹۵: وسوسه تن گذشت غلغله جان رسید}
\label{sec:0895}
\addcontentsline{toc}{section}{\nameref{sec:0895}}
\begin{longtable}{l p{0.5cm} r}
وسوسه تن گذشت غلغله جان رسید
&&
مور فروشد به گور چتر سلیمان رسید
\\
این فلک آتشی چند کند سرکشی
&&
نوح به کشتی نشست جوشش طوفان رسید
\\
چند مخنث نژاد دعوی مردی کند
&&
رستم خنجر کشید سام و نریمان رسید
\\
جادوکانی ز فن چند عصا و رسن
&&
مار کنند از فریب موسی و ثعبان رسید
\\
درد به پستی نشست صاف ز دردی برست
&&
گردن گرگان شکست یوسف کنعان رسید
\\
صبح دروغین گذشت صبح سعادت رسید
&&
جان شد و جان بقا از بر جانان رسید
\\
محنت ایوب را فاقه یعقوب را
&&
چاره دیگر نبود رحمت رحمان رسید
\\
دزد کی باشد چو رفت شحنه ایمان به شهر
&&
شحنه کی باشد بگو چون شه و سلطان رسید
\\
صدق نگر بی‌نفاق وصل نگر بی‌فراق
&&
طاق طرنبین و طاق طاق شوم کان رسید
\\
مفتعلن فاعلات جان مرا کرد مات
&&
جان خداخوان بمرد جان خدادان رسید
\\
میوه دل می‌پزید روح از او می‌مزید
&&
باد کرم بروزید حرف پریشان رسید
\\
\end{longtable}
\end{center}
