\begin{center}
\section*{غزل شماره ۲۰۵۴: بشنیده‌ام که عزم سفر می‌کنی مکن}
\label{sec:2054}
\addcontentsline{toc}{section}{\nameref{sec:2054}}
\begin{longtable}{l p{0.5cm} r}
بشنیده‌ام که عزم سفر می‌کنی مکن
&&
مهر حریف و یار دگر می‌کنی مکن
\\
تو در جهان غریبی غربت چه می‌کنی
&&
قصد کدام خسته جگر می‌کنی مکن
\\
از ما مدزد خویش به بیگانگان مرو
&&
دزدیده سوی غیر نظر می‌کنی مکن
\\
ای مه که چرخ زیر و زبر از برای توست
&&
ما را خراب و زیر و زبر می‌کنی مکن
\\
چه وعده می‌دهی و چه سوگند می‌خوری
&&
سوگند و عشوه را تو سپر می‌کنی مکن
\\
کو عهد و کو وثیقه که با بنده کرده‌ای
&&
از عهد و قول خویش عبر می‌کنی مکن
\\
ای برتر از وجود و عدم بارگاه تو
&&
از خطه وجود گذر می‌کنی مکن
\\
ای دوزخ و بهشت غلامان امر تو
&&
بر ما بهشت را چو سقر می‌کنی مکن
\\
اندر شکرستان تو از زهر ایمنیم
&&
آن زهر را حریف شکر می‌کنی مکن
\\
جانم چو کوره‌ای است پرآتش بست نکرد
&&
روی من از فراق چو زر می‌کنی مکن
\\
چون روی درکشی تو شود مه سیه ز غم
&&
قصد خسوف قرص قمر می‌کنی مکن
\\
ما خشک لب شویم چو تو خشک آوری
&&
چشم مرا به اشک چه تر می‌کنی مکن
\\
چون طاقت عقیله عشاق نیستت
&&
پس عقل را چه خیره نگر می‌کنی مکن
\\
حلوا نمی‌دهی تو به رنجور ز احتما
&&
رنجور خویش را تو بتر می‌کنی مکن
\\
چشم حرام خواره من دزد حسن توست
&&
ای جان سزای دزد بصر می‌کنی مکن
\\
سر درکش ای رفیق که هنگام گفت نیست
&&
در بی‌سری عشق چه سر می‌کنی مکن
\\
\end{longtable}
\end{center}
