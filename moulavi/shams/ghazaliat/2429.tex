\begin{center}
\section*{غزل شماره ۲۴۲۹: ای در طواف ماه تو ماه و سپهر مشتری}
\label{sec:2429}
\addcontentsline{toc}{section}{\nameref{sec:2429}}
\begin{longtable}{l p{0.5cm} r}
ای در طواف ماه تو ماه و سپهر مشتری
&&
ای آمده در چرخ تو خورشید و چرخ چنبری
\\
یا رب منم جویان تو یا خود تویی جویان من
&&
ای ننگ من تا من منم من دیگرم تو دیگری
\\
ای ما و من آویخته وی خون هر دو ریخته
&&
چیزی دگر انگیخته نی آدمی و نی پری
\\
تا پا نباشد ز آنک پا ما را به خارستان برد
&&
تا سر نباشد ز آنک سر کافر شود از دوسری
\\
آبی میان جو روان آبی لب جو بسته یخ
&&
آن تیزرو این سست رو هین تیز رو تا نفسری
\\
خورشید گوید سنگ را زان تافتم بر سنگ تو
&&
تا تو ز سنگی وارهی پا درنهی در گوهری
\\
خورشید عشق لم یزل زان تافته‌ست اندر دلت
&&
کاول فزایی بندگی و آخر نمایی مهتری
\\
خورشید گوید غوره را زان آمدم در مطبخت
&&
تا سرکه نفروشی دگر پیشه کنی حلواگری
\\
شه باز را گوید که من زان بسته‌ام دو چشم تو
&&
تا بگسلی از جنس خود جز روی ما را ننگری
\\
گوید بلی فرمان برم جز در جمالت ننگرم
&&
جز بر خیالت نگذرم وز جان نمایم چاکری
\\
گل باغ را گوید که من زان عرضه کردم رخت خود
&&
تا جمله رخت خویش را بفروشی و با ما خوری
\\
آن کس کز این جا زر برد با دلبری دیگر خورد
&&
تو کژ نشین و راست گو آن از چه باشد از خری
\\
آن آدمی باشد که او خر بدهد و عیسی خرد
&&
وین از خری باشد که تو عیسی دهی و خر خری
\\
عیسی مست را زر کند ور زر بود گوهر کند
&&
گوهر بود بهتر کند بهتر ز ماه و مشتری
\\
نی مشتری بی‌نوا بل نور الله اشتری
&&
گر یوسفی باشد تو را زین پیرهن بویی بری
\\
ما را چو مریم بی‌سبب از شاخ خشک آید رطب
&&
ما را چو عیسی بی‌طلب در مهد آید سروری
\\
بی‌باغ و رز انگور بین بی‌روز و بی‌شب نور بین
&&
وین دولت منصور بین از داد حق بی‌داوری
\\
از روی همچون آتشم حمام عالم گرم شد
&&
بر صورت گرمابه‌ای چون کودکان کمتر گری
\\
فردا ببینی روش را شد طعمه مار و موش را
&&
دروازه موران شده آن چشم‌های عبهری
\\
مهتاب تا مه رانده دیوار تیره مانده
&&
اناالیه آمده کان سو نگر گر مبصری
\\
یا جانب تبریز رو از شمس دین محفوظ شو
&&
یا از زبان واصفان از صدق بنما باوری
\\
\end{longtable}
\end{center}
