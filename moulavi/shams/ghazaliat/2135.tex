\begin{center}
\section*{غزل شماره ۲۱۳۵: ای شعشعه نور فلق در قبه مینای تو}
\label{sec:2135}
\addcontentsline{toc}{section}{\nameref{sec:2135}}
\begin{longtable}{l p{0.5cm} r}
ای شعشعه نور فلق در قبه مینای تو
&&
پیمانه خون شفق پنگان خون پیمای تو
\\
ای میل‌ها در میل‌ها وی سیل‌ها در سیل‌ها
&&
رقصان و غلطان آمده تا ساحل دریای تو
\\
با رفعت و آهنگ مه مه را فتد از سر کله
&&
چون ماه رو بالا کند تا بنگرد بالای تو
\\
در هر صبوحی بلبلان افغان کنان چون بی‌دلان
&&
بر پرده‌های واصلان در روضه خضرای تو
\\
ای جان‌ها دیدارجو دل‌ها همه دلدارجو
&&
ای برگشاده چارجو در باغ باپهنای تو
\\
یک جو روان ماء معین یک جوی دیگر انگبین
&&
یک جوی شیر تازه بین یک جو می حمرای تو
\\
تو مهلتم کی می‌دهی می بر سر می می‌دهی
&&
کو سر که تا شرحی کنم از سرده صهبای تو
\\
من خود کی باشم آسمان در دور این رطل گران
&&
یک دم نمی‌یابد امان از عشق و استسقای تو
\\
ای ماه سیمین منطقه با عشق داری سابقه
&&
وی آسمان هم عاشقی پیداست در سیمای تو
\\
عشقی که آمد جفت دل شد بس ملول از گفت دل
&&
ای دل خمش تا کی بود این جهد و استقصای تو
\\
دل گفت من نای ویم نالان ز دم‌های ویم
&&
گفتم که نالان شو کنون جان بنده سودای تو
\\
انا فتحنا بابکم لا تهجروا اصحابکم
&&
حمدا لعشق شامل بگرفته سر تا پای تو
\\
\end{longtable}
\end{center}
