\begin{center}
\section*{غزل شماره ۲۹۲۲: هم تو شمعی هم تو شاهد هم تو می}
\label{sec:2922}
\addcontentsline{toc}{section}{\nameref{sec:2922}}
\begin{longtable}{l p{0.5cm} r}
هم تو شمعی هم تو شاهد هم تو می
&&
هم بهاری در میان ماه دی
\\
هر طرف از عشق تو پر سوخته
&&
آفتاب و صد هزاران همچو دی
\\
چون همیشه آتشت در نی فتد
&&
رفت شکر زین هوس در جان نی
\\
سر بریدی صد هزاران را به عشق
&&
زهره نی جان را که گوید های و هی
\\
عاشقان سازیده‌اند از چشم بد
&&
خانه‌ها زیر زمین چون شهر ری
\\
نیست از دانش بتر اشکنجه‌ای
&&
وای آنک ماند اندر نیک و بی
\\
آن زنان مصر اندر بیخودی
&&
زخم‌ها خورده نکرده وای وی
\\
در شب معراج شاه از بیخودی
&&
صد هزاران ساله ره را کرده طی
\\
برشکن از باده‌های بیخودان
&&
تخته بندی ز استخوان و عرق و پی
\\
شمس تبریزی تو ما را محو کن
&&
ز آنک تو چون آفتابی ما چو فی
\\
\end{longtable}
\end{center}
