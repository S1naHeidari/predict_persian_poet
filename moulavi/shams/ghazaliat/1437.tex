\begin{center}
\section*{غزل شماره ۱۴۳۷: چو رعد و برق می خندد ثنا و حمد می خوانم}
\label{sec:1437}
\addcontentsline{toc}{section}{\nameref{sec:1437}}
\begin{longtable}{l p{0.5cm} r}
چو رعد و برق می خندد ثنا و حمد می خوانم
&&
چو چرخ صاف پرنورم به گرد ماه گردانم
\\
زبانم عقده‌ای دارد چو موسی من ز فرعونان
&&
ز رشک آنک فرعونی خبر یابد ز برهانم
\\
فروبندید دستم را چو دریابید هستم را
&&
به لشکرگاه فرعونی که من جاسوس سلطانم
\\
نه جاسوسم نه ناموسم من از اسرار قدوسم
&&
رها کن چونک سرمستم که تا لافی بپرانم
\\
ز باده باد می خیزد که باده باد انگیزد
&&
خصوصا این چنین باده که من از وی پریشانم
\\
همه زهاد عالم را اگر بویی رسد زین می
&&
چه ویرانی پدید آید چه گویم من نمی‌دانم
\\
چه جای می که گر بویی از آن انفاس سرمستان
&&
رسد در سنگ و در مرمر بلافد کآب حیوانم
\\
وجود من عزبخانه‌ست و آن مستان در او جمعند
&&
دلم حیران کز ایشانم عجب یا خود من ایشانم
\\
اگر من جنس ایشانم وگر من غیر ایشانم
&&
نمی‌دانم همین دانم که من در روح و ریحانم
\\
\end{longtable}
\end{center}
