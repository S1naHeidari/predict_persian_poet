\begin{center}
\section*{غزل شماره ۲۵۰۵: چو بی گه آمدی باری درآ مردانه‌ای ساقی}
\label{sec:2505}
\addcontentsline{toc}{section}{\nameref{sec:2505}}
\begin{longtable}{l p{0.5cm} r}
چو بی گه آمدی باری درآ مردانه‌ای ساقی
&&
بپیما پنج پیمانه به یک پیمانه‌ای ساقی
\\
ز جام باده عرشی حصار فرش ویران کن
&&
پس آنگه گنج باقی بین در این ویرانه‌ای ساقی
\\
اگر من بشکنم جامی و یا مجلس بشورانم
&&
مگیر از من منم بی‌دل تویی فرزانه‌ای ساقی
\\
چو باشد شیشه روحانی ببین باده چه سان باشد
&&
بگویم از کی می‌ترسم تویی در خانه‌ای ساقی
\\
در آب و گل بنه پایی که جان آب است و تن چون گل
&&
جدا کن آب را از گل چو کاه از دانه‌ای ساقی
\\
ز آب و گل بود این جا عمارت‌های کاشانه
&&
خلل از آب و گل باشد در این کاشانه‌ای ساقی
\\
زهی شمشیر پرگوهر که نامش باده و ساغر
&&
تویی حیدر ببر زوتر سر بیگانه‌ای ساقی
\\
یکی سر نیست عاشق را که ببریدی و آسودی
&&
ببر هر دم سر این شمع فراشانه‌ای ساقی
\\
نمی‌تانم سخن گفتن به هشیاری خرابم کن
&&
از آن جام سخن بخش لطیف افسانه‌ای ساقی
\\
سقاهم ربهم گاهی کند دیوانه را عاقل
&&
گهی باشد که عاقل را کند دیوانه‌ای ساقی
\\
\end{longtable}
\end{center}
