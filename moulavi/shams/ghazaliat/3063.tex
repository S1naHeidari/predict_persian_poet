\begin{center}
\section*{غزل شماره ۳۰۶۳: به هر دلی که درآیی چو عشق بنشینی}
\label{sec:3063}
\addcontentsline{toc}{section}{\nameref{sec:3063}}
\begin{longtable}{l p{0.5cm} r}
به هر دلی که درآیی چو عشق بنشینی
&&
بجوشد از تک دل چشمه چشمه شیرینی
\\
کلید حاجت خلقان بدان شده‌ست دعا
&&
که جان جان دعایی و نور آمینی
\\
دلا به کوی خرابات ناز تو نخرند
&&
مکن تو بینی و ناموس تا جهان بینی
\\
در آن الست و بلی جان بی‌بدن بودی
&&
تو را نمود که آنی چه در غم اینی
\\
تو را یکی پر و بالیست آسمان پیما
&&
چه در پی خر و اسپی چه در غم زینی
\\
بگو بگو تو چه جستی که آنت پیش نرفت
&&
بیا بیا که تو سلطان این سلاطینی
\\
تو تاج شاه جهان را عزیزتر گهری
&&
عروس جان نهان را هزار کابینی
\\
چه چنگ درزده‌ای در جهان و قانونش
&&
که از ورای فلک زهره قوانینی
\\
به روز جلوه ملایک تو را سجود کنند
&&
بنشنوند ز ابلیسیان که تو طینی
\\
میان ببستی و کردی به صدق خدمت دین
&&
کنند خدمت تو بعد از این که تو دینی
\\
ستاره وار به انگشت‌ها نمودندت
&&
چو آفتاب کنون نامشار تعیینی
\\
اگر چه درخور نازی نیاز را مگذار
&&
برای رشک ز ویسه خوشست رامینی
\\
خمش به سوره کنون اقرا بسی عمل کردی
&&
ز قشر حرف گذر کن کنون که والتینی
\\
\end{longtable}
\end{center}
