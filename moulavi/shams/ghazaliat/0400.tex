\begin{center}
\section*{غزل شماره ۴۰۰: چون نظر کردن همه اوصاف خوب اندر دلست}
\label{sec:0400}
\addcontentsline{toc}{section}{\nameref{sec:0400}}
\begin{longtable}{l p{0.5cm} r}
چون نظر کردن همه اوصاف خوب اندر دلست
&&
وین همه اوصاف رسوا معدنش آب و گلست
\\
از هوا و شهوت ای جان آب و گل می صد شود
&&
مشکل این ترک هوا و کاشف هر مشکلست
\\
وین تعلل بهر ترکش دافع صد علتست
&&
چون بشد علت ز تو پس نقل منزل منزلست
\\
لیک شرطی کن تو با خود تا که شرطی نشکنی
&&
ور نه علت باقی و درمانت محو و زایلست
\\
چونک طبعت خو کند با شرط تندش بعد از آن
&&
صد هزاران حاصل جان از درونت حاصلست
\\
پس تو را آیینه گردد این دل آهن چنانک
&&
هر دمی رویی نماید روی آن کو کاهلست
\\
پس تو را مطرب شود در عیش و هم ساقی شود
&&
آن امانت چونک شد محمول جان را حاملست
\\
فارغ آیی بعد از آن از شغل و هم از فارغی
&&
شهره گردد از تو آن گنجی که آن بس خاملست
\\
گر چه حلواها خوری شیرین نگردد جان تو
&&
ذوق آن برقی بود تا در دهان آکلست
\\
این طبیعت کور و کر گر نیست پس چون آزمود
&&
کاین حجاب و حائل‌ست آن سوی آن چون مایلست
\\
لیک طبع از اصل رنج و غصه‌ها بررسته‌ست
&&
در پی رنج و بلاها عاشق بی‌طایلست
\\
در تواضع‌های طبعت سر نخوت را نگر
&&
و اندر آن کبرش تواضع‌های بی‌حد شاکلست
\\
هر حدیث طبع را تو پرورش‌هایی بدش
&&
شرح و تأویلی بکن وادانک این بی‌حائلست
\\
هر یکی بیتی جمال بیت دیگر دانک هست
&&
با مؤید این طریقت ره روان را شاغلست
\\
ور تو را خوف مطالب باشد از اشهادها
&&
از خدا می‌خواه شیرینی اجل کان آجلست
\\
هر طرف رنجی دگرگون فرض کن آن گاه برو
&&
جز به سوی بی‌سوی‌ها کان دگر بی‌حاصلست
\\
تو وثاق مار آیی از پی ماری دگر
&&
غصه ماران ببینی زانک این چون سلسله‌ست
\\
تا نگویی مار را از خویش عذری زهرناک
&&
وان گهت او متهم دارد که این هم باطلست
\\
از حدیث شمس دین آن فخر تبریز صفا
&&
آن مزاجش گرم باید کاین نه کار پلپلست
\\
\end{longtable}
\end{center}
