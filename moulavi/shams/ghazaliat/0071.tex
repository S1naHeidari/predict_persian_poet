\begin{center}
\section*{غزل شماره ۷۱: اگر نه عشق شمس الدین بدی در روز و شب ما را}
\label{sec:0071}
\addcontentsline{toc}{section}{\nameref{sec:0071}}
\begin{longtable}{l p{0.5cm} r}
اگر نه عشق شمس الدین بدی در روز و شب ما را
&&
فراغت‌ها کجا بودی ز دام و از سبب ما را
\\
بت شهوت برآوردی دمار از ما ز تاب خود
&&
اگر از تابش عشقش نبودی تاب و تب ما را
\\
نوازش‌های عشق او لطافت‌های مهر او
&&
رهانید و فراغت داد از رنج و نصب ما را
\\
زهی این کیمیای حق که هست از مهر جان او
&&
که عین ذوق و راحت شد همه رنج و تعب ما را
\\
عنایت‌های ربانی ز بهر خدمت آن شه
&&
برویانید و هستی داد از عین ادب ما را
\\
بهار حسن آن مهتر به ما بنمود ناگاهان
&&
شقایق‌ها و ریحان‌ها و گل‌های عجب ما را
\\
زهی دولت زهی رفعت زهی بخت و زهی اختر
&&
که مطلوب همه جان‌ها کند از جان طلب ما را
\\
گزید او لب گه مستی که رو پیدا مکن مستی
&&
چو جام جان لبالب شد از آن می‌های لب ما را
\\
عجب بختی که رو بنمود ناگاهان هزاران شکر
&&
ز معشوق لطیف اوصاف خوب بوالعجب ما را
\\
در آن مجلس که گردان کرد از لطف او صراحی‌ها
&&
گران قدر و سبک دل شد دل و جان از طرب ما را
\\
به سوی خطه تبریز چه چشمه آب حیوانست
&&
کشاند دل بدان جانب به عشق چون کنب ما را
\\
\end{longtable}
\end{center}
