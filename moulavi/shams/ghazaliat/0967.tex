\begin{center}
\section*{غزل شماره ۹۶۷: رسم نو بین که شهریار نهاد}
\label{sec:0967}
\addcontentsline{toc}{section}{\nameref{sec:0967}}
\begin{longtable}{l p{0.5cm} r}
رسم نو بین که شهریار نهاد
&&
قبله مان سوی شهر یار نهاد
\\
نقد عشاق را عیار نبود
&&
او ز کان کرم عیار نهاد
\\
گل صدبرگ برگ عیش بساخت
&&
روی سوی بنفشه زار نهاد
\\
هر که را چون بنفشه دید دوتا
&&
کرد یکتا و در شمار نهاد
\\
بی دلان را چو دل گرفت به بر
&&
سرکشان را چو سر خمار نهاد
\\
منتظر باش و چشم بر در دار
&&
کو نظر را در انتظار نهاد
\\
غم او را کنار گیر که غم
&&
روی بر روی غمگسار نهاد
\\
کس چه داند که گلشن رخ او
&&
بر دل بی‌دلم چه خار نهاد
\\
از دل بی‌دلم قرار مجوی
&&
کاندر او درد بی‌قرار نهاد
\\
آهوان صید چشم او گشتند
&&
چونک رو جانب شکار نهاد
\\
آن زره موی در کمان ز کمین
&&
تیرهای زره گذار نهاد
\\
خویشتن را چو در کنار گرفت
&&
خلق را دور و برکنار نهاد
\\
رحمتش آه عاشقان بشنید
&&
آهشان را بس اعتبار نهاد
\\
در عنایات خویششان بکشید
&&
جرمشان را به جای کار نهاد
\\
نور عشاق شمس تبریزی
&&
نور در دیده شمس وار نهاد
\\
\end{longtable}
\end{center}
