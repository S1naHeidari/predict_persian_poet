\begin{center}
\section*{غزل شماره ۲۳۲۱: ای دل به کجایی تو آگاه هیی یا نه}
\label{sec:2321}
\addcontentsline{toc}{section}{\nameref{sec:2321}}
\begin{longtable}{l p{0.5cm} r}
ای دل به کجایی تو آگاه هیی یا نه
&&
از سر تو برون کن هی سودای گدایانه
\\
در بزم چنان شاهی در نور چنان ماهی
&&
خط در دو جهان درکش چه جای یکی خانه
\\
در دولت سلطانی گر یاوه شود جانی
&&
یک جان چه محل دارد در خدمت جانانه
\\
گر جان بداندیشت گوید بد شه پیشت
&&
ده بر دهن او زن تا کم کند افسانه
\\
یک دانه به یک بستان بیع است بده بستان
&&
و آن گاه چو سرمستان می‌گو که زهی دانه
\\
شاهی نگری خندان چون ماه و دو صد چندان
&&
بی‌ناز خوشاوندان بی‌زحمت بیگانه
\\
شمس الحق تبریزی آن کو به تو بازآید
&&
آن باز بود عرشی بر عرش کند لانه
\\
\end{longtable}
\end{center}
