\begin{center}
\section*{غزل شماره ۱۶۰۸: چه کسم من چه کسم من که بسی وسوسه مندم}
\label{sec:1608}
\addcontentsline{toc}{section}{\nameref{sec:1608}}
\begin{longtable}{l p{0.5cm} r}
چه کسم من چه کسم من که بسی وسوسه مندم
&&
گه از آن سوی کشندم گه از این سوی کشندم
\\
ز کشاکش چو کمانم به کف گوش کشانم
&&
قدر از بام درافتد چو در خانه ببندم
\\
مگر استاره چرخم که ز برجی سوی برجی
&&
به نحوسیش بگریم به سعودیش بخندم
\\
به سما و به بروجش به هبوط و به عروجش
&&
نفسی همتک بادم نفسی من هلپندم
\\
نفسی آتش سوزان نفسی سیل گریزان
&&
ز چه اصلم ز چه فصلم به چه بازار خرندم
\\
نفسی فوق طباقم نفسی شام و عراقم
&&
نفسی غرق فراقم نفسی راز تو رندم
\\
نفسی همره ماهم نفسی مست الهم
&&
نفسی یوسف چاهم نفسی جمله گزندم
\\
نفسی رهزن و غولم نفسی تند و ملولم
&&
نفسی زین دو برونم که بر آن بام بلندم
\\
بزن ای مطرب قانون هوس لیلی و مجنون
&&
که من از سلسله جستم وتد هوش بکندم
\\
به خدا که نگریزی قدح مهر نریزی
&&
چه شود ای شه خوبان که کنی گوش به پندم
\\
هله ای اول و آخر بده آن باده فاخر
&&
که شد این بزم منور به تو ای عشق پسندم
\\
بده آن باده جانی ز خرابات معانی
&&
که بدان ارزد چاکر که از آن باده دهندم
\\
بپران ناطق جان را تو از این منطق رسمی
&&
که نمی‌یابد میدان بگو حرف سمندم
\\
\end{longtable}
\end{center}
