\begin{center}
\section*{غزل شماره ۱۲۵۲: آن که مه غاشیه زین چو غلامان کشدش}
\label{sec:1252}
\addcontentsline{toc}{section}{\nameref{sec:1252}}
\begin{longtable}{l p{0.5cm} r}
آن که مه غاشیه زین چو غلامان کشدش
&&
بوک این همت ما جانب بستان کشدش
\\
گر چه جان را نبود قوت این گستاخی
&&
آنک جان از مدد رحمت جانان کشدش
\\
هر دم از یاد لبش جان لب خود می‌لیسد
&&
ور سقط می‌شنود از بن دندان کشدش
\\
جانب محو و فنا رخت کشیدند مهان
&&
تا بقا لطف کند جانب ایشان کشدش
\\
ای بسا جان که چو یعقوب همی زهر چشد
&&
تا که آن یوسف جان در شکرستان کشدش
\\
هر کسی کو بتر از وی خرد فخر کند
&&
گر چه چون ماه بود چرخ به میزان کشدش
\\
هر که در دیده عشاق شود مردمکی
&&
آن نظر زود سوی گوهر انسان کشدش
\\
کافر زلف وی آن را که ز راهش ببرد
&&
کفر آید بر او جانب ایمان کشدش
\\
شمس تبریز مرا عشق تو سرمست کند
&&
هر کی او باده کشد باده بدین سان کشدش
\\
\end{longtable}
\end{center}
