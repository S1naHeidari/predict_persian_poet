\begin{center}
\section*{غزل شماره ۲۸۸۷: بده ای کف تو را قاعده لطف افزایی}
\label{sec:2887}
\addcontentsline{toc}{section}{\nameref{sec:2887}}
\begin{longtable}{l p{0.5cm} r}
بده ای کف تو را قاعده لطف افزایی
&&
کف دریا چه کند خواجه به جز دریایی
\\
چون تو خواهی که شکرخایی غلط اندازی
&&
ز پی خشم رهی ساعد و کف می‌خایی
\\
صنما مغلطه بگذار و مگو تا فردا
&&
چون تویی پای علم نقد که را می‌پایی
\\
ترشم گفتی و پیش شکر بی‌حد تو
&&
عسل و قند چه دارند به جز سرکایی
\\
گر چه من روترشم لیک خم سرکه نیم
&&
ور چه هر جا بروم لیک نیم هرجایی
\\
گر تو خوبی و منم آینه روی خوشت
&&
پیش رو دار مرا چونک جهان آرایی
\\
نی غلط گفتم سرمست بدم زفت زدم
&&
کی بود آینه را با رخ تو گنجایی
\\
نو فسونی است مرا سخت عجب پیشتر آ
&&
تا به گوش تو فروخوانم ای بینایی
\\
\end{longtable}
\end{center}
