\begin{center}
\section*{غزل شماره ۲۳۳۴: ای طبل رحیل از طرف چرخ شنیده}
\label{sec:2334}
\addcontentsline{toc}{section}{\nameref{sec:2334}}
\begin{longtable}{l p{0.5cm} r}
ای طبل رحیل از طرف چرخ شنیده
&&
وی رخت از این جای بدان جای کشیده
\\
ای نرگس چشم و رخ چون لاله کجایی
&&
از گور تو آن نرگس و آن لاله دمیده
\\
اندر لحد بی‌در و بی‌بام مقیمی
&&
ای بر در و بر بام به صد ناز دویده
\\
کو شیوه ابروی تو کو غمزه چشمت
&&
ای چشم بد مرگ بدان هر دو رسیده
\\
ای دست تو بوسه گه لب‌های عزیزان
&&
در دست فنا مانده تو با دست بریده
\\
این‌ها همه سهل است اگر مرغ ضمیرت
&&
بر چرخ پریده بود و دام دریده
\\
صورت چه کم آید چه برد جان به سلامت
&&
موزه چه کم آید چو بود پای رهیده
\\
صد شکر کند جان چو رهد از تن و صورت
&&
ای بی‌خبر از چاشنی جان جریده
\\
کو لذت آب و گل و کو آب حیاتی
&&
کو قبه گردونی و کو بام خمیده
\\
یا رب چه طلسم است کز آن خلد نفوریم
&&
ما در تک این دوزخ امشاج خزیده
\\
محسود فلک بوده و مسجود ملایک
&&
وز همت ناپاک ز ما دیو رمیده
\\
باغ آی و ز باران سخن نرگس و گل چین
&&
نرگس ندهد قطره ای از بام چکیده
\\
بربند دهان از سخن و باده لب نوش
&&
تا قصه کند چشم خمار از ره دیده
\\
\end{longtable}
\end{center}
