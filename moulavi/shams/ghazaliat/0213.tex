\begin{center}
\section*{غزل شماره ۲۱۳: اگر تو عاشق عشقی و عشق را جویا}
\label{sec:0213}
\addcontentsline{toc}{section}{\nameref{sec:0213}}
\begin{longtable}{l p{0.5cm} r}
اگر تو عاشق عشقی و عشق را جویا
&&
بگیر خنجر تیز و ببر گلوی حیا
\\
بدانک سد عظیم است در روش ناموس
&&
حدیث بی‌غرض است این قبول کن به صفا
\\
هزار گونه جنون از چه کرد آن مجنون
&&
هزار شید برآورد آن گزین شیدا
\\
گهی قباش درید و گهی به کوه دوید
&&
گهی ز زهر چشید و گهی گزید فنا
\\
چو عنکبوت چنان صیدهای زفت گرفت
&&
ببین چه صید کند دام ربی الاعلی
\\
چو عشق چهره لیلی بدان همه ارزید
&&
چگونه باشد اسری بعبده لیلا
\\
ندیده‌ای تو دواوین ویسه و رامین
&&
نخوانده‌ای تو حکایات وامق و عذرا
\\
تو جامه گرد کنی تا ز آب تر نشود
&&
هزار غوطه تو را خوردنی‌ست در دریا
\\
طریق عشق همه مستی آمد و پستی
&&
که سیل پست رود کی رود سوی بالا
\\
میان حلقه عشاق چون نگین باشی
&&
اگر تو حلقه به گوش تکینی ای مولا
\\
چنانک حلقه به گوش است چرخ را این خاک
&&
چنانک حلقه به گوش است روح را اعضا
\\
بیا بگو چه زیان کرد خاک از این پیوند
&&
چه لطف‌ها که نکرده‌ست عقل با اجزا
\\
دهل به زیر گلیم ای پسر نشاید زد
&&
علم بزن چو دلیران میانه صحرا
\\
به گوش جان بشنو از غریو مشتاقان
&&
هزار غلغله در جو گنبد خضرا
\\
چو برگشاید بند قبا ز مستی عشق
&&
توهای و هوی ملک بین و حیرت حورا
\\
چه اضطراب که بالا و زیر عالم راست
&&
ز عشق کوست منزه ز زیر و از بالا
\\
چو آفتاب برآمد کجا بماند شب
&&
رسید جیش عنایت کجا بماند عنا
\\
خموش کردم ای جان جان جان تو بگو
&&
که ذره ذره ز عشق رخ تو شد گویا
\\
\end{longtable}
\end{center}
