\begin{center}
\section*{غزل شماره ۳۱۱۰: جان جان مایی، خوشتر از حلوایی}
\label{sec:3110}
\addcontentsline{toc}{section}{\nameref{sec:3110}}
\begin{longtable}{l p{0.5cm} r}
جان جان مایی، خوشتر از حلوایی
&&
چرخ را پر کردزینت و زیبایی
\\
دایهٔ هستیها، چشمهٔ مستیها
&&
سرده مستانی، و افت سرهایی
\\
باغ و گنج خاکی، مشعلهٔ افلاکی
&&
از طوافت کیوان یافته بالایی
\\
وعده کردی کایم، وعده را می‌پایم
&&
ای قمر سیمایم، تو کرا می‌پایی؟
\\
وقت بخشش جانا، کانی و دریایی
&&
وقت گفتن مانا، که شکر می‌خایی
\\
بی‌توم پروانی، جای تو پیدا نی
&&
در پی تو دلها، خیره و هر جایی
\\
هوش را برباید، عمر را افزاید
&&
چشم را بگشاید، هرچه تو فرمایی
\\
اندران مجلسها، که تو باشی شاها
&&
جان نگنجد، تا تو ندهیش گنجایی
\\
تلختر جام ای جان، صعبتر دام ای جان
&&
آن بود که مانم، تا تو ندهیش گنجایی
\\
تلختر جام ای جان، صعبتر دام ای جان
&&
آن بود که مانم، بی‌تو در تنهایی
\\
خوشترین مقصودی، با نوا ترسودی
&&
آن بود که گویی:« چونی ای سودایی؟»
\\
پختگان را خمری، بهر خامان شیری
&&
بهر شیره و شیرت، بین تو خون پالایی
\\
عشق تو خوش خیزی، در جگر آمیزی
&&
دست تو خون‌ریزی، دست را نالایی
\\
گر شود هر دستی دستگیر مستی
&&
نیست چاره پیدا، تا تو ناپیدایی
\\
روحها دریادان، جسمها کفها دان
&&
تو بیا، ای آنک گوهر دریایی
\\
سیدی مولایی، مسکنی مشوایی
&&
مبدع الاشیاء مسکرالاجزاء
\\
فالق‌الصباح، خالق‌الرواح
&&
یا کریم الراح، ساعة السقاء
\\
من نهادم دستم، بر دهان مستم
&&
تا تو گویی که تو دادهٔ گویایی
\\
\end{longtable}
\end{center}
