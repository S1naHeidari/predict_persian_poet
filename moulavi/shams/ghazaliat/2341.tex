\begin{center}
\section*{غزل شماره ۲۳۴۱: سماع آمد هلا ای یار برجه}
\label{sec:2341}
\addcontentsline{toc}{section}{\nameref{sec:2341}}
\begin{longtable}{l p{0.5cm} r}
سماع آمد هلا ای یار برجه
&&
مسابق باش و وقت کار برجه
\\
هزاران بار خفتی همچو لنگر
&&
مثال بادبان این بار برجه
\\
بسی خفتی تو مست از سرگرانی
&&
چو کردندت کنون بیدار برجه
\\
هلا ای فکرت طیار برپر
&&
تو نیز ای قالب سیار برجه
\\
هلا صوفی چو ابن الوقت باشد
&&
گذر از پار و از پیرار برجه
\\
به عشق اندرنگنجد شرم و ناموس
&&
رها کن شرم و استکبار برجه
\\
وگر کاهل بود قوال عارف
&&
بدو ده خرقه و دستار برجه
\\
سماح آمد رباح از قول یزدان
&&
که عشقی به ز صد قنطار برجه
\\
به عشق آنک فرشت گوهر آمد
&&
چو موج قلزم زخار برجه
\\
چو زلفین ار فروسو می‌کشندت
&&
تو همچون جعد آن دلدار برجه
\\
صلایی از خیال یار آمد
&&
خیالانه تو هم ز اسرار برجه
\\
بسی در غدر و حیلت برجهیدی
&&
یکی از عالم غدار برجه
\\
بسی بهر قوافی برجهیدی
&&
خموشی گیر و بی‌گفتار برجه
\\
\end{longtable}
\end{center}
