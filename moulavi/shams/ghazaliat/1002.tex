\begin{center}
\section*{غزل شماره ۱۰۰۲: چونک کمند تو دلم را کشید}
\label{sec:1002}
\addcontentsline{toc}{section}{\nameref{sec:1002}}
\begin{longtable}{l p{0.5cm} r}
چونک کمند تو دلم را کشید
&&
یوسفم از چاه به صحرا دوید
\\
آنک چو یوسف به چهم درفکند
&&
باز به فریادم هم او رسید
\\
چون رسن لطف در این چه فکند
&&
چنبره دل گل و نسرین دمید
\\
قیصر از آن قصر به چه میل کرد
&&
چه چو بهشتی شد و قصر مشید
\\
گفتم ای چه چه شد آن ظلمتت
&&
گفت که خورشید به من بنگرید
\\
هر که فسردست کنون گرم شد
&&
جمره عشقت بگدازد جلید
\\
قیصر رومست که بر زنگ زد
&&
اوست که ترسابچه خواندش فرید
\\
پرتو دل بود که زد بر سعیر
&&
پر شد و بشکافت که هل من مزید
\\
دوزخ گفتش که مرا جان ببخش
&&
تا بخورم هرک ز یزدان برید
\\
برگذر از آتش ای بحر لطف
&&
ور نه بمردم تبشم بفسرید
\\
گفت که ای آتش قوم مرا
&&
زود به من ده که خداشان گزید
\\
جمله یکایک به کف او سپرد
&&
گفت که نار تو ز نورم رهید
\\
تافت ز تبریز رخ شمس دین
&&
شمس بود نور جهان را کلید
\\
\end{longtable}
\end{center}
