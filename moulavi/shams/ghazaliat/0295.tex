\begin{center}
\section*{غزل شماره ۲۹۵: الا ای روی تو صد ماه و مهتاب}
\label{sec:0295}
\addcontentsline{toc}{section}{\nameref{sec:0295}}
\begin{longtable}{l p{0.5cm} r}
الا ای روی تو صد ماه و مهتاب
&&
مگو شب گشت و بی‌گه گشت بشتاب
\\
مرا در سایه‌ات ای کعبه جان
&&
به هر مسجد ز خورشیدست محراب
\\
غلط گفتم که اندر مسجد ما
&&
برون در بود خورشید بواب
\\
از این هفت آسیا ما نان نجوییم
&&
ننوشیم آب ما زین سبز دولاب
\\
مسبب اوست اسباب جهان را
&&
چه باشد تار و پود لاف اسباب
\\
ز مستی در هزاران چه فتادیم
&&
برون مان می‌کشد عشقش به قلاب
\\
چه رونق دارد از مجلس جان
&&
زهی چشم و چراغ جان اصحاب
\\
بخندد باغ دل زان سرو مقبل
&&
بجوشد خون ما زین شاخ عناب
\\
فتوح اندر فتوح اندر فتوحی
&&
توی مفتاح و حق مفتاح ابواب
\\
ز نفط انداز عشق آتشینت
&&
زمین و آسمان لرزان چو سیماب
\\
بر مستانش آید می به دعوی
&&
خلق گردد برانندش به مضراب
\\
خمش کن ختم کن ای دل چو دیدی
&&
که آن خوبی نمی‌گنجد در القاب
\\
\end{longtable}
\end{center}
