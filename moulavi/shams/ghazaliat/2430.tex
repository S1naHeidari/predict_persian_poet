\begin{center}
\section*{غزل شماره ۲۴۳۰: ای آن که بر اسب بقا از دیر فانی می‌روی}
\label{sec:2430}
\addcontentsline{toc}{section}{\nameref{sec:2430}}
\begin{longtable}{l p{0.5cm} r}
ای آن که بر اسب بقا از دیر فانی می‌روی
&&
دانا و بینای رهی آن سو که دانی می‌روی
\\
بی‌همره جسم و عرض بی‌دام و دانه و بی‌غرض
&&
از تلخکامی می‌رهی در کامرانی می‌روی
\\
نی همچو عقل دانه چین نی همچو نفس پر ز کین
&&
نی روح حیوان زمین تو جان جانی می‌روی
\\
ای چون فلک دربافته‌ای همچو مه درتافته
&&
از ره نشانی یافته در بی‌نشانی می‌روی
\\
ای غرقه سودای او ای بیخود از صهبای او
&&
از مدرسه اسمای او اندر معانی می‌روی
\\
ای خوی تو چون آب جو داده زمین را رنگ و بو
&&
تا کس نپندارد که تو بی‌ارمغانی می‌روی
\\
کو سایه منصور حق تا فاش فرماید سبق
&&
کز مستعینی می‌رهی در مستعانی می‌روی
\\
شب کاروان‌ها زین جهان بر می‌رود تا آسمان
&&
تو خود به تنهایی خود صد کاروانی می‌روی
\\
ای آفتاب آن جهان در ذره‌ای چونی نهان
&&
وی پادشاه شه نشان در پاسبانی می‌روی
\\
ای بس طلسمات عجب بستی برون از روز و شب
&&
تا چشم پندارد که تو اندر مکانی می‌روی
\\
ای لطف غیبی چند تو شکل بهاری می‌شوی
&&
وی عدل مطلق چند تو اندر خزانی می‌روی
\\
آخر برون آ زین صور چادر برون افکن ز سر
&&
تا چند در رنگ بشر در گله بانی می‌روی
\\
ای ظاهر و پنهان چو جان وی چاکر و سلطان چو جان
&&
کی بینمت پنهان چو جان در بی‌زبانی می‌روی
\\
\end{longtable}
\end{center}
