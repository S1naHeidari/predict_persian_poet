\begin{center}
\section*{غزل شماره ۲۵۹۸: ای جان و جهان آخر از روی نکوکاری}
\label{sec:2598}
\addcontentsline{toc}{section}{\nameref{sec:2598}}
\begin{longtable}{l p{0.5cm} r}
ای جان و جهان آخر از روی نکوکاری
&&
یک دم چه زیان دارد گر روی به ما آری
\\
ای روی تو چون آتش وی بوی تو چون گل خوش
&&
یا رب که چه رو داری یا رب که چه بو داری
\\
در پیش دو چشم من پیوسته خیال تو
&&
خوش خواب که می‌بینم در حالت بیداری
\\
دل را چو خیال تو بنوازد مسکین دل
&&
در پوست نمی‌گنجد از لذت دلداری
\\
قرص قمرت گویم نور بصرت گویم
&&
جان دگرت گویم یا صحت بیماری
\\
از شرم تو شاخ گل سر پیش درافکنده
&&
وز زاری من بلبل وامانده شد از زاری
\\
از جمله ببر زیرا آن جا که تویی و او
&&
تو نیز نمی‌گنجی جز او که دهد یاری
\\
اندر شکم ماهی دم با کی زند یونس
&&
جز او کی بود مونس در نیم شب تاری
\\
در چشمه سوزن تو خواهی که رود اشتر
&&
ای بسته تو بر اشتر شش تنگ به سرباری
\\
با این همه ای دیده نومید مباش از وی
&&
چون ابر بهاری کن در عشق گهرباری
\\
\end{longtable}
\end{center}
