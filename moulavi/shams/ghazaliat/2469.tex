\begin{center}
\section*{غزل شماره ۲۴۶۹: آه خجسته ساعتی که صنما به من رسی}
\label{sec:2469}
\addcontentsline{toc}{section}{\nameref{sec:2469}}
\begin{longtable}{l p{0.5cm} r}
آه خجسته ساعتی که صنما به من رسی
&&
پاک و لطیف همچو جان صبحدمی به تن رسی
\\
آن سر زلف سرکشت گفته مرا که شب خوشت
&&
زین سفر چو آتشت کی تو بدین وطن رسی
\\
کی بود آفتاب تو در دل چون حمل رسد
&&
تا تو چو آب زندگی بر گل و بر سمن رسی
\\
همچو حسن ز دست غم جرعه زهر می‌کشم
&&
ای تریاق احمدی کی تو به بوالحسن رسی
\\
گر چه غمت به خون من چابک و تیز می‌رود
&&
هست امید جان که تو در غم دل شکن رسی
\\
جمله تو باشی آن زمان دل شده باشد از میان
&&
پاک شود بدن چو جان چون تو بدین بدن رسی
\\
چرخ فروسکل تو خوش ننگ فلک دگر مکش
&&
بوک به بوی طره‌اش بر سر آن رسن رسی
\\
زن ز زنی برون شود مرد میان خون شود
&&
چون تو به حسن لم یزل بر سر مرد و زن رسی
\\
حسن تو پای درنهد یوسف مصر سر نهد
&&
مرده ز گور برجهد چون به سر کفن رسی
\\
لطف خیال شمس دین از تبریز در کمین
&&
طالب جان شوی چو دین تا به چه شکل و فن رسی
\\
\end{longtable}
\end{center}
