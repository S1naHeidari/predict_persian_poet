\begin{center}
\section*{غزل شماره ۲۷۹۶: ای که جان‌ها خاک پایت صورت اندیش آمدی}
\label{sec:2796}
\addcontentsline{toc}{section}{\nameref{sec:2796}}
\begin{longtable}{l p{0.5cm} r}
ای که جان‌ها خاک پایت صورت اندیش آمدی
&&
دست بر در نه درآ در خانه خویش آمدی
\\
نیست بر هستی شکستی گرد چون انگیختی
&&
چون تو پس کردی جهان چونی چو واپیش آمدی
\\
در دو عالم قاعده نیش است وآنگه ذوق نوش
&&
تو ورای هر دو عالم نوش بی‌نیش آمدی
\\
خویش را ذوقی بود بیگانه را ذوق نوی
&&
هم قدیمی هم نوی بیگانه و خویش آمدی
\\
بر دل و جان قلندر ریش و مرهم هر دو تو
&&
فقر را ای نور مطلق مرهم و ریش آمدی
\\
کیش هفتاد و دو ملت جمله قربان تواند
&&
تا تو شاهنشاه باقربان و باکیش آمدی
\\
ای که بر خوان فلک با ماه همکاسه شدی
&&
ماه را یک لقمه کردی کآفتابیش آمدی
\\
عقل و حس مهتاب را کی گز تواند کرد لیک
&&
داندی خورشید بی‌گز کز مهان بیش آمدی
\\
عشق شمس الدین تبریزی که عید اکبر است
&&
کی تو را قربان کند چون لاغری میش آمدی
\\
\end{longtable}
\end{center}
