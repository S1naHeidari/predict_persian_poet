\begin{center}
\section*{غزل شماره ۱۳۰۲: ما دو سه مست خلوتی جمع شدیم این طرف}
\label{sec:1302}
\addcontentsline{toc}{section}{\nameref{sec:1302}}
\begin{longtable}{l p{0.5cm} r}
ما دو سه مست خلوتی جمع شدیم این طرف
&&
چون شتران رو به رو پوز نهاده در علف
\\
هر طرفی همی‌رسد مست و خراب جوق جوق
&&
چون شتران مست لب سست فکنده کرده کف
\\
خوش بخورید کاشتران ره نبرند سوی ما
&&
زانک بوادی اندرند ما سر کوه بر شرف
\\
گر چه درازگردن‌اند تا سر کوه کی رسند
&&
ور چه که عف عفی کنند غم نخوریم ما ز عف
\\
بحر اگر شود جهان کشتی نوح اندریم
&&
کشتی نوح کی بود سخره آفت و تلف
\\
جمله جهان پرست غم در پی منصب و درم
&&
ما خوش و نوش و محترم مست خرف در این کنف
\\
کان زمردیم ما آفت چشم مار غم
&&
آنک اسیر غم بود حصه اوست وااسف
\\
مطرب عارفان بیا مست شدند عارفان
&&
زود بگو رباعیی پیش درآ بگیر دف
\\
باد به بیشه درفکن بر سر هر درخت زن
&&
تا که شوند سرفشان شاخ درخت صف به صف
\\
ابله اگر زنخ زند تو ره عشق گم مکن
&&
عشق حیات جان بود مرده بود دگر حرف
\\
چون غزلی به سر بری مدحت شمس دین بگو
&&
از تبریز یاد کن کوری خصم ناخلف
\\
\end{longtable}
\end{center}
