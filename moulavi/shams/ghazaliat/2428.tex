\begin{center}
\section*{غزل شماره ۲۴۲۸: فصل بهاران شد ببین بستان پر از حور و پری}
\label{sec:2428}
\addcontentsline{toc}{section}{\nameref{sec:2428}}
\begin{longtable}{l p{0.5cm} r}
فصل بهاران شد ببین بستان پر از حور و پری
&&
گویی سلیمان بر سپه عرضه نمود انگشتری
\\
رومی رخان ماه وش زاییده از خاک حبش
&&
چون تو مسلمانان خوش بیرون شده از کافری
\\
گلزار بین گلزار بین در آب نقش یار بین
&&
و آن نرگس خمار بین و آن غنچه‌های احمری
\\
گلبرگ‌ها بر همدگر افتاده بین چون سیم و زر
&&
آویزها و حلقه‌ها بی‌دستگاه زرگری
\\
در جان بلبل گل نگر وز گل به عقل کل نگر
&&
وز رنگ در بی‌رنگ پر تا بوک آن جا ره بری
\\
گل عقل غارت می‌کند نسرین اشارت می‌کند
&&
کاینک پس پرده است آن کو می‌کند صورتگری
\\
ای صلح داده جنگ را وی آب داده سنگ را
&&
چون این گل بدرنگ را در رنگ‌ها می‌آوری
\\
گر شاخه‌ها دارد تری ور سرو دارد سروری
&&
ور گل کند صد دلبری ای جان تو چیزی دیگری
\\
چه جای باغ و راغ و گل چه جای نقل و جام مل
&&
چه جای روح و عقل کل کز جان جان هم خوشتری
\\
\end{longtable}
\end{center}
