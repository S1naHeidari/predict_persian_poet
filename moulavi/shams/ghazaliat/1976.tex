\begin{center}
\section*{غزل شماره ۱۹۷۶: عشق شمس حق و دین کان گوهر کانی است آن}
\label{sec:1976}
\addcontentsline{toc}{section}{\nameref{sec:1976}}
\begin{longtable}{l p{0.5cm} r}
عشق شمس حق و دین کان گوهر کانی است آن
&&
در دو عالم جان و دل را دولت معنی است آن
\\
گر به ظاهر لشکر و اقبال و مخزن نیستش
&&
رو به چشم جان نگر کان دولت جانی است آن
\\
کله سر را تهی کن از هوا بهر میش
&&
کله سر جام سازش کان می جامی است آن
\\
پختگان عشق را باشد ز خام خمر جان
&&
پخته نی و خام جستن مایه خامی است آن
\\
تا کتاب جان او اندر غلاف تن بود
&&
گر چه خاص خاص باشد در هنر عامی است آن
\\
آنک بالایی گزیند پست باشد عشق در
&&
آنک پستی را گزید او مجلس سامی است آن
\\
هرک جان پاک او زان می درآشامد ابد
&&
گر چه هندو باشد آن و مکی و شامی است آن
\\
مر تن معمور را ویران کند هجران می
&&
هرک کرد این تن خراب می میش بانی است آن
\\
آن می باقی بود اول که جان زاید از او
&&
پس دروغ است آنک می جان است کان ثانی است آن
\\
جان فانی را همیشه مست دار از جام او
&&
رنگ باقی گیرد از می روح کان فانی است آن
\\
در می باقی نشان پیوسته جان مردنی
&&
کز جوار کیمیا آن مس زر کانی است آن
\\
چون میان عقل و تن افتاد از می سه طلاق
&&
هر تنی کو با خرد جفت است آن زانی است آن
\\
در دل تنگ هوس باده بقا ساکن نگشت
&&
هر دلی کاین می در او بنشست میدانی است آن
\\
آنک جام او بگیرد یک نشانش این بود
&&
در بیان سر حکمت جان او منشی است آن
\\
در شعاع می بقا بیند ابد پس بعد از آن
&&
مال چه بود کو ز عین جان خود معطی است آن
\\
آنک وصف می بگوید باخود است و هوشیار
&&
اهل قرآن نبود آن کس لیک او مقری است آن
\\
حق و صاحب حق را از عاشقان مست پرس
&&
زانک جام مست اندر عاشقان قاضی است آن
\\
زانک حکم مست فعل می بود پس روشن است
&&
حق و صاحب حق هم با حکم او راضی است آن
\\
مطرب مستور بی‌پرده یکی چنگی بزن
&&
وارهان از نام و ننگم گر چه بدنامی است آن
\\
وانما رخسار را تا بشکنی بازار بت
&&
زان رخی کو حسرت صد آزر و مانی است آن
\\
ای صبا تبریز رو سجده ببر کان خاک پاک
&&
خاک درگاه حیات انگیز ربانی است آن
\\
\end{longtable}
\end{center}
