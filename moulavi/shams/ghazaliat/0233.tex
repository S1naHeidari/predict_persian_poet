\begin{center}
\section*{غزل شماره ۲۳۳: کجاست ساقی جان تا به هم زند ما را}
\label{sec:0233}
\addcontentsline{toc}{section}{\nameref{sec:0233}}
\begin{longtable}{l p{0.5cm} r}
کجاست ساقی جان تا به هم زند ما را
&&
بروبد از دل ما فکر دی و فردا را
\\
چنو درخت کم افتد پناه مرغان را
&&
چنو امیر بباید سپاه سودا را
\\
روان شود ز ره سینه صد هزار پری
&&
چو بر قنینه بخواند فسون احیا را
\\
کجاست شیر شکاری و حمله‌های خوشش
&&
که پر کنند ز آهوی مشک صحرا را
\\
ز مشرقست و ز خورشید نور عالم را
&&
ز آدمست در و نسل و بچه حوا را
\\
کجاست بحر حقایق کجاست ابر کرم
&&
که چشم‌های روان داده است خارا را
\\
کجاست کان شه ما نیست لیک آن باشد
&&
که چشم بند کند سحرهاش بینا را
\\
چنان ببندد چشمت که ذره را بینی
&&
میان روز و نبینی تو شمس کبری را
\\
ز چشم بند ویست آنک زورقی بینی
&&
میان بحر و نبینی تو موج دریا را
\\
تو را طپیدن زورق ز بحر غمز کند
&&
چنانک جنبش مردم به روز اعمی را
\\
نخوانده‌ای ختم الله خدای مهر نهد
&&
همو گشاید مهر و برد غطاها را
\\
دو چشم بسته تو در خواب نقش‌ها بینی
&&
دو چشم باز شود پرده آن تماشا را
\\
عجب مدار اگر جان حجاب جانانست
&&
ریاضتی کن و بگذار نفس غوغا را
\\
عجبتر اینک خلایق مثال پروانه
&&
همی‌پرند و نبینی تو شمع دل‌ها را
\\
چه جرم کردی ای چشم ما که بندت کرد
&&
بزار و توبه کن و ترک کن خطاها را
\\
سزاست جسم بفرسودن این چنین جان را
&&
سزاست مشی علی الراس آن تقاضا را
\\
خموش باش که تا وحی‌های حق شنوی
&&
که صد هزار حیاتست وحی گویا را
\\
\end{longtable}
\end{center}
