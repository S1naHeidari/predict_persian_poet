\begin{center}
\section*{غزل شماره ۱۵۱۸: من آن ماهم که اندر لامکانم}
\label{sec:1518}
\addcontentsline{toc}{section}{\nameref{sec:1518}}
\begin{longtable}{l p{0.5cm} r}
من آن ماهم که اندر لامکانم
&&
مجو بیرون مرا در عین جانم
\\
تو را هر کس به سوی خویش خواند
&&
تو را من جز به سوی تو نخوانم
\\
مرا هم تو به هر رنگی که خوانی
&&
اگر رنگین اگر ننگین ندانم
\\
گهی گویی خلاف و بی‌وفایی
&&
بلی تا تو چنینی من چنانم
\\
به پیش کور هیچم من چنانم
&&
به پیش گوش کر من بی‌زبانم
\\
گلابه چند ریزی بر سر چشم
&&
فروشو چشم از گل من عیانم
\\
لباس و لقمه‌ات گل‌های رنگین
&&
تو گل خواری نشایی میهمانم
\\
گل است این گل در او لطفی است بنگر
&&
چو لطف عاریت را واستانم
\\
من آب آب و باغ باغم ای جان
&&
هزاران ارغوان را ارغوانم
\\
سخن کشتی و معنی همچو دریا
&&
درآ زوتر که تا کشتی برانم
\\
\end{longtable}
\end{center}
