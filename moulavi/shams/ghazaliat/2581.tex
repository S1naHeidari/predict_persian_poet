\begin{center}
\section*{غزل شماره ۲۵۸۱: ای سوخته یوسف در آتش یعقوبی}
\label{sec:2581}
\addcontentsline{toc}{section}{\nameref{sec:2581}}
\begin{longtable}{l p{0.5cm} r}
ای سوخته یوسف در آتش یعقوبی
&&
گه بیت و غزل گویی گه پای عمل کوبی
\\
گه دور بگردانی گاهی شکر افشانی
&&
گه غوطه خوری عریان در چشمه ایوبی
\\
خلقان همه مرد و زن لب بسته و در شیون
&&
وز دولت و داد او ما غرقه این خوبی
\\
بر عشق چو می‌چسبد عاشق ز چه رو خسپد
&&
چون دوست نمی‌خسپد با آن همه مطلوبی
\\
آن دوست که می‌باید چون سوی تو می‌آید
&&
از بهر چنان مهمان چون خانه نمی‌روبی
\\
چون رزم نمی‌سازی چون چست نمی‌تازی
&&
چون سر تو نیندازی از غصه محجوبی
\\
ای نعل تو در آتش آن سوی ز پنج و شش
&&
از جذبه آن است این کاندر غم و آشوبی
\\
کی باشد و کی باشد کو گل ز تو بتراشد
&&
بی‌عیب خرد جان را از جمله معیوبی
\\
اجزای درختان را چون میوه کند دارا
&&
بنگر که چه مبدل شد آن چوب از آن چوبی
\\
زین به بتوان گفتن اما بمگو تن زن
&&
منگر ز حساب ای جان در عالم محسوبی
\\
\end{longtable}
\end{center}
