\begin{center}
\section*{غزل شماره ۶۳۴: نک ماه رجب آمد تا ماه عجب بیند}
\label{sec:0634}
\addcontentsline{toc}{section}{\nameref{sec:0634}}
\begin{longtable}{l p{0.5cm} r}
نک ماه رجب آمد تا ماه عجب بیند
&&
وز سوختگان ره گرمی و طلب بیند
\\
گر سجده کنان آید در امن و امان آید
&&
ور بی‌ادبی آرد سیلی و ادب بیند
\\
حکمی که کند یزدان راضی بود و شادان
&&
ور سر کشد از سلطان در حلق کنب بیند
\\
گر درخور عشق آید خرم چو دمشق آید
&&
ور دل ندهد دل را ویران چو حلب بیند
\\
گوید چه سبب باشد آن خرم و این ویران
&&
جان خضری باید تا جان سبب بیند
\\
آمد شعبان عمدا از بهر برات ما
&&
تا روزی و بی‌روزی از بخشش رب بیند
\\
ماه رمضان آمد آن بند دهان آمد
&&
زد بر دهن بسته تا لذت لب بیند
\\
آمد قدح روزه بشکست قدح‌ها را
&&
تا منکر این عشرت بی‌باده طرب بیند
\\
سغراق معانی را بر معده خالی زن
&&
معشوقه خلوت را هم چشم عزب بیند
\\
با غره دولت گو هم بگذرد این نوبت
&&
چون بگذرد این نوبت هم نوبت تب بیند
\\
نوبت بگذار و رو نوبت زن احمد شو
&&
تا برف وجود تو خورشید عرب بیند
\\
خامش کن و کمتر گو بسیار کسی گوید
&&
کو جاه و هوا جوید تا نام و لقب بیند
\\
\end{longtable}
\end{center}
