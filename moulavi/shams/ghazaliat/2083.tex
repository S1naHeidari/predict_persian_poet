\begin{center}
\section*{غزل شماره ۲۰۸۳: اگر سزای لب تو نبود گفته من}
\label{sec:2083}
\addcontentsline{toc}{section}{\nameref{sec:2083}}
\begin{longtable}{l p{0.5cm} r}
اگر سزای لب تو نبود گفته من
&&
برآر سنگ گران و دهان من بشکن
\\
چو طفل بیهده گوید نه مادر مشفق
&&
پی ادب لب او را فروبرد سوزن
\\
دو صد دهان و جهان از برای عز لبت
&&
بسوز و پاره کن و بردران و برهم زن
\\
چو تشنه‌ای دود استاخ بر لب دریا
&&
نه موج تیغ برآرد ببردش گردن
\\
غلام سوسنم ایرا که دید گلشن تو
&&
ز شرم نرگس تو ده زبانش شد الکن
\\
ولیک من چو دفم چون زنی تو کف بر من
&&
فغان کنم که رخم را بکوب چون هاون
\\
مرا ز دست منه تا سماع گرم بود
&&
بکش تو دامن خود از جهان تردامن
\\
بلی ز گلشن معنی است چشم‌ها مخمور
&&
ولیک نغمه بلبل خوش است در گلشن
\\
اگر تجلی یوسف برهنه خوبتر است
&&
دو چشم باز نگردد مگر به پیراهن
\\
اگر چه شعشعه آفتاب جان اصل است
&&
بر آن فلک نرسیده‌ست آدمی بی‌تن
\\
خمش که گر دهنم مرده شوی بربندد
&&
ز گور من شنوی این نوا پس مردن
\\
\end{longtable}
\end{center}
