\begin{center}
\section*{غزل شماره ۵۶۹: بهار آمد بهار آمد بهار مشکبار آمد}
\label{sec:0569}
\addcontentsline{toc}{section}{\nameref{sec:0569}}
\begin{longtable}{l p{0.5cm} r}
بهار آمد بهار آمد بهار مشکبار آمد
&&
نگار آمد نگار آمد نگار بردبار آمد
\\
صبوح آمد صبوح آمد صبوح راح و روح آمد
&&
خرامان ساقی مه رو به ایثار عقار آمد
\\
صفا آمد صفا آمد که سنگ و ریگ روشن شد
&&
شفا آمد شفا آمد شفای هر نزار آمد
\\
حبیب آمد حبیب آمد به دلداری مشتاقان
&&
طبیب آمد طبیب آمد طبیب هوشیار آمد
\\
سماع آمد سماع آمد سماع بی‌صداع آمد
&&
وصال آمد وصال آمد وصال پایدار آمد
\\
ربیع آمد ربیع آمد ربیع بس بدیع آمد
&&
شقایق‌ها و ریحان‌ها و لاله خوش عذار آمد
\\
کسی آمد کسی آمد که ناکس زو کسی گردد
&&
مهی آمد مهی آمد که دفع هر غبار آمد
\\
دلی آمد دلی آمد که دل‌ها را بخنداند
&&
میی آمد میی آمد که دفع هر خمار آمد
\\
کفی آمد کفی آمد که دریا در از او یابد
&&
شهی آمد شهی آمد که جان هر دیار آمد
\\
کجا آمد کجا آمد کز این جا خود نرفتست او
&&
ولیکن چشم گه آگاه و گه بی‌اعتبار آمد
\\
ببندم چشم و گویم شد گشایم گویم او آمد
&&
و او در خواب و بیداری قرین و یار غار آمد
\\
کنون ناطق خمش گردد کنون خامش به نطق آید
&&
رها کن حرف بشمرده که حرف بی‌شمار آمد
\\
\end{longtable}
\end{center}
