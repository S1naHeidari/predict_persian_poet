\begin{center}
\section*{غزل شماره ۲۳۰۴: هر روز پری زادی از سوی سراپرده}
\label{sec:2304}
\addcontentsline{toc}{section}{\nameref{sec:2304}}
\begin{longtable}{l p{0.5cm} r}
هر روز پری زادی از سوی سراپرده
&&
ما را و حریفان را در چرخ درآورده
\\
صوفی ز هوای او پشمینه شکافیده
&&
عالم ز بلای او دستار کشان کرده
\\
سالوس نتان کردن مستور نتان بودن
&&
از دست چنین رندی سغراق رضا خورده
\\
دی رفت سوی گوری در مرده زد او شوری
&&
معذورم آخر من کمتر نیم از مرده
\\
هر روز برون آید ساغر به کف و گوید
&&
والله که بنگذارم در شهر یک افسرده
\\
ای مونس و ای جانم چندانت بپیچانم
&&
تا شهد و شکر گردی ای سرکه پرورده
\\
خستم جگرت را من بستان جگری دیگر
&&
همچون جگر شیران ای گربه پژمرده
\\
همرنگ دل من شو زیرا که نمی‌شاید
&&
من سرخ و سپید ای جان تو زرد و سیه چرده
\\
خامش کن و خامش کن دررو به حریم دل
&&
کاندر حرمین دل نبود دل آزرده
\\
شمس الحق تبریزی بادا دل بدخواهت
&&
بر گرد جهان گردان در طمع یکی گرده
\\
\end{longtable}
\end{center}
