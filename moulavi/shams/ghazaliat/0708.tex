\begin{center}
\section*{غزل شماره ۷۰۸: برخیز که ساقی اندرآمد}
\label{sec:0708}
\addcontentsline{toc}{section}{\nameref{sec:0708}}
\begin{longtable}{l p{0.5cm} r}
برخیز که ساقی اندرآمد
&&
وان جان هزار دلبر آمد
\\
آمد می ناب وز پی نقل
&&
بادام و نبات و شکر آمد
\\
آن جان و جهان رسید و از وی
&&
صد جان جهان مصور آمد
\\
مشک آمد پیش طره او
&&
کان طره ز حسن بر سر آمد
\\
زد حلقه مشک فام و می‌گفت
&&
بگشای که بنده عنبر آمد
\\
از تابش لعل او چه گویم
&&
کز لعل و عقیق برتر آمد
\\
زان سنبل ابروش حیاتم
&&
با برگ و لطیف و اخضر آمد
\\
درده می خام و بین که ما را
&&
در مجلس خام دیگر آمد
\\
آن رایت سرخ کز نهیبش
&&
اسپاه فرج مظفر آمد
\\
هر کار که بسته گشت و مشکل
&&
آن کار بدو میسر آمد
\\
می ده که سر سخن ندارم
&&
زیرا که سخن چو لنگر آمد
\\
\end{longtable}
\end{center}
