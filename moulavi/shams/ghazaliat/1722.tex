\begin{center}
\section*{غزل شماره ۱۷۲۲: بدار دست ز ریشم که باده‌ای خوردم}
\label{sec:1722}
\addcontentsline{toc}{section}{\nameref{sec:1722}}
\begin{longtable}{l p{0.5cm} r}
بدار دست ز ریشم که باده‌ای خوردم
&&
ز بیخودی سر و ریش و سبال گم کردم
\\
ز پیشگاه و ز درگاه نیستم آگاه
&&
به پیشگاه خرابات روی آوردم
\\
خرد که گرد برآورد از تک دریا
&&
هزار سال دود درنیابد او گردم
\\
فراختر ز فلک گشت سینه تنگم
&&
لطیفتر ز قمر گشت چهره زردم
\\
دکان جمله طبیبان خراب خواهم کرد
&&
که من سعادت بیمار و داروی دردم
\\
شرابخانه عالم شده‌ست سینه من
&&
هزار رحمت بر سینه جوامردم
\\
هزار حمد و ثنا مر خدای عالم را
&&
که دنگ عشقم و از ننگ خویشتن فردم
\\
چو خاک شاه شدم ارغوان ز من رویید
&&
چو مات شاه شدم جمله لعب را بردم
\\
چو دانه‌ای که بمیرد هزار خوشه شود
&&
شدم به فضل خدا صد هزار چون مردم
\\
منم بهشت خدا لیک نام من عشق است
&&
که از فشار رهد هر دلی کش افشردم
\\
رهد ز تیر فلک وز سنان مریخش
&&
هر آن مرید که او را به عشق پروردم
\\
چو آفتاب سعادت رسید سوی حمل
&&
دو صد تموز بجوشید از دی سردم
\\
خموش باش که گر نی ز خوف فتنه بدی
&&
هزار پرده دریدی زبان من هر دم
\\
\end{longtable}
\end{center}
