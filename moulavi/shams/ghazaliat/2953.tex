\begin{center}
\section*{غزل شماره ۲۹۵۳: بازآمدی که ما را درهم زنی به شوری}
\label{sec:2953}
\addcontentsline{toc}{section}{\nameref{sec:2953}}
\begin{longtable}{l p{0.5cm} r}
بازآمدی که ما را درهم زنی به شوری
&&
داوود روزگاری با نغمه زبوری
\\
یا مصر پرنباتی یا یوسف حیاتی
&&
یعقوب را نپرسی چونی از این صبوری
\\
بازآمد آن قیامت با فتنه و ملامت
&&
گفتم که آفتابی یا نور نور نوری
\\
ای آسمان برین دم گردان و بی‌قراری
&&
وی خاک هم در این غم خاموش و در حضوری
\\
ای دلبر پریرین وی فتنه تو شیرین
&&
دل نام تو نگوید از غایت غیوری
\\
خورشید چون برآید خود را چرا نماید
&&
با آفتاب رویت از جاهلی و کوری
\\
بازآمد آن سلیمان بر تخت پادشاهی
&&
جان را نثار او کن آخر نه کم ز موری
\\
در پرده چون نشستی رسوا چرا نگشتی
&&
این نیست از ستیری این نیست از ستوری
\\
تره فروش کویش این عقل را نگیرد
&&
تو بر سرش نهادی بنگر چه دور دوری
\\
بازآمده‌ست بازی صیاد هر نیازی
&&
ای بوم اگر نه شومی از وی چرا نفوری
\\
بازآمد آن تجلی از بارگاه اعلا
&&
ای روح نعره می زن موسی و کوه طوری
\\
بازآمدی به خانه‌ای قبله زمانه
&&
والله صلاح دینی پیوسته در ظهوری
\\
\end{longtable}
\end{center}
