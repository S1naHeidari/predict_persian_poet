\begin{center}
\section*{غزل شماره ۱۰۶۰: عاشقی در خشم شد از یار خود معشوق وار}
\label{sec:1060}
\addcontentsline{toc}{section}{\nameref{sec:1060}}
\begin{longtable}{l p{0.5cm} r}
عاشقی در خشم شد از یار خود معشوق وار
&&
گازری در خشم گشت از آفتاب نامدار
\\
وانگهان چون گازری از گازران درویشتر
&&
وانگهان چون آفتابی آفتاب هر دیار
\\
ناز گازر چون بدید آن آفتاب از لطف خود
&&
ابر پیش آورد اینک گازری باکار و بار
\\
گفت تا گازر نخندد من برون نایم ز ابر
&&
تا دل او خوش نگردد من نباشم برقرار
\\
دسته دسته جامه‌های گازران از کار ماند
&&
تا پدید آید که گازر اختیارست اختیار
\\
هر کی باشد عاشق آن آفتاب از جان و دل
&&
سر ز خاک پای گازر برندارد زینهار
\\
گویم آن گازر که باشد شمس تبریزی و بس
&&
کز برای او برآید آفتاب از هر کنار
\\
\end{longtable}
\end{center}
