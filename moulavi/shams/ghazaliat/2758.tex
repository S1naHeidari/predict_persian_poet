\begin{center}
\section*{غزل شماره ۲۷۵۸: مست می عشق را حیا نی}
\label{sec:2758}
\addcontentsline{toc}{section}{\nameref{sec:2758}}
\begin{longtable}{l p{0.5cm} r}
مست می عشق را حیا نی
&&
وین باده عشق را بها نی
\\
آن عشق چو بزم و باده جان را
&&
می نوشد و ممکن صلا نی
\\
با عقل بگفت ماجراها
&&
جان گفت که وقت ماجرا نی
\\
از روح بجستم آن صفا گفت
&&
آن هست صفا ولی ز ما نی
\\
گفتم که مکن نهان از این مس
&&
ای کفو تو زر و کیمیا نی
\\
کاین برق حدیث تو از آن است
&&
جز جان افزا و دلربا نی
\\
گفتا غلطی که آن نیم من
&&
ما بوالحسنیم و بوالعلا نی
\\
گفتم که به حق نرگسانت
&&
دفعم بمده به شیوه‌ها نی
\\
کاین غمزه مست خونی تو
&&
کشته‌ست هزار و خونبها نی
\\
بالله که تویی که بی‌تویی تو
&&
ای کبر تو غیر کبریا نی
\\
گر ز آنک تویی و گر نه‌ای تو
&&
از تو گذری دو دیده را نی
\\
گر فرمایی که نیست هست است
&&
کو زهره که گویمت چرا نی
\\
مغناطیسی و جان چو آهن
&&
می‌آید مست و دست و پا نی
\\
چون گرم شوم ز جام اول
&&
غیر تسلیم در قضا نی
\\
چون شد به سرم میم سراسر
&&
می را تسلیم یا رضا نی
\\
از بهر نسیم زلف جعدت
&&
یکتا زلفی که جز دو تا نی
\\
ای باد صبا به انتظارت
&&
از بهر صبا و خود صبا نی
\\
پس ما چه زنیم ای قلندر
&&
اندر گره و گره گشا نی
\\
گر ز آنک نه هر دمی خداوند
&&
کو جز سر و خاصه خدا نی
\\
مخدومی شمس دین تبریز
&&
چون خورشیدش در این سما نی
\\
\end{longtable}
\end{center}
