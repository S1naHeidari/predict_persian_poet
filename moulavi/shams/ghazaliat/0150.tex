\begin{center}
\section*{غزل شماره ۱۵۰: درد شمس الدین بود سرمایه درمان ما}
\label{sec:0150}
\addcontentsline{toc}{section}{\nameref{sec:0150}}
\begin{longtable}{l p{0.5cm} r}
درد شمس الدین بود سرمایه درمان ما
&&
بی سر و سامان عشقش بود سامان ما
\\
آن خیال جان فزای بخت ساز بی‌نظیر
&&
هم امیر مجلس و هم ساقی گردان ما
\\
در رخ جان بخش او بخشیدن جان هر زمان
&&
گشته در مستی جان هم سهل و هم آسان ما
\\
صد هزاران همچو ما در حسن او حیران شود
&&
کاندر آن جا گم شود جان و دل حیران ما
\\
خوش خوش اندر بحر بی‌پایان او غوطی خورد
&&
تا ابدهای ابد خود این سر و پایان ما
\\
شکر ایزد را که جمله چشمه حیوان‌ها
&&
تیره باشد پیش لطف چشمه حیوان ما
\\
شرم آرد جان و دل تا سجده آرد هوشیار
&&
پیش چشم مست مخمور خوش جانان ما
\\
دیو گیرد عشق را از غصه هم این عقل را
&&
ناگهان گیرد گلوی عقل آدم سان ما
\\
پس برآرد نیش خونی کز سرش خون می‌چکد
&&
پس ز جان عقل بگشاید رگ شیران ما
\\
در دهان عقل ریزد خون او را بردوام
&&
تا رهاند روح را از دام و از دستان ما
\\
تا بشاید خدمت مخدوم جان‌ها شمس دین
&&
آن قباد و سنجر و اسکندر و خاقان ما
\\
تا ز خاک پاش بگشاید دو چشم سر به غیب
&&
تا ببیند حال اولیان و آخریان ما
\\
شکر آن را سوی تبریز معظم رو نهد
&&
کز زمینش می‌بروید نرگس و ریحان ما
\\
\end{longtable}
\end{center}
