\begin{center}
\section*{غزل شماره ۲۷۴۷: خضری به میان سینه داری}
\label{sec:2747}
\addcontentsline{toc}{section}{\nameref{sec:2747}}
\begin{longtable}{l p{0.5cm} r}
خضری به میان سینه داری
&&
در آب حیات و سبزه زاری
\\
خضر آب حیات را نپاید
&&
گر بوی برد که تو چه داری
\\
در کشتی نوح همچو روحی
&&
در گلشن روح نوبهاری
\\
گر طبل وجودها بدرد
&&
از کتم عدم علم برآری
\\
این چار طبیعت ار بسوزد
&&
غم نیست تو جان هر چهاری
\\
صیاد بدایت وجودی
&&
اجزای جهان همه شکاری
\\
گه بند کند گهی گشاید
&&
ای کارافزا تو بر چه کاری
\\
او سرو بلند و تو چو سایه
&&
او باد شمال و تو غباری
\\
در چشم تو ریخت کحل پندار
&&
می‌پنداری به اختیاری
\\
این چرخ به اختیار خود نیست
&&
آخر تو کیی بدین نزاری
\\
از نیست تو خویش هست کردی
&&
وین گردن خود تو می‌فشاری
\\
زین ترس تو حجت است بر تو
&&
کز غیر تو است ترسگاری
\\
از خویش دل کسی نترسد
&&
از خویش کسی نجست یاری
\\
پس خوف و رجای تو گواهند
&&
بر ملکت شاه و کامکاری
\\
وز خوف و رجا چو برتر آیی
&&
ایمن چو صفات کردگاری
\\
کشتی ترسد ز بحر نی بحر
&&
تو کشتی بحر بی‌کناری
\\
کشتی توی تو چو بشکست
&&
خاموش کن از سخن گزاری
\\
کشتی شکسته را کی راند
&&
جز آب به موج بی‌قراری
\\
کشتیبان شکستگان است
&&
آن بحر کرم به بردباری
\\
خامش که زبان عقل مهر است
&&
بنشین بر جا که گشت تاری
\\
\end{longtable}
\end{center}
