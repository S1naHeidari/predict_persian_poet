\begin{center}
\section*{غزل شماره ۷۸۸: آه کان طوطی دل بی‌شکرستان چه کند}
\label{sec:0788}
\addcontentsline{toc}{section}{\nameref{sec:0788}}
\begin{longtable}{l p{0.5cm} r}
آه کان طوطی دل بی‌شکرستان چه کند
&&
آه کان بلبل جان بی‌گل و بستان چه کند
\\
آنک از نقد وصال تو به یک جو نرسید
&&
چو گه عرض بود بر سر میزان چه کند
\\
آنک بحر تو چو خاشاک به یک سوش افکند
&&
چو بجویند از او گوهر ایمان چه کند
\\
نقش گرمابه ز گرمابه چه لذت یابد
&&
در تماشاگه جان صورت بی‌جان چه کند
\\
با بد و نیک بد و نیک مرا کاری نیست
&&
دل تشنه لب من در شب هجران چه کند
\\
دست و پا و پر و بال دل من منتظرند
&&
تا که عشقش چه کند عشق جز احسان چه کند
\\
آنک او دست ندارد چه برد روز نثار
&&
و آنک او پای ندارد گه خیزان چه کند
\\
آنک بر پرده عشاق دلش زنگله نیست
&&
پرده زیر و عراقی و سپاهان چه کند
\\
آنک از باده جان گوش و سرش گرم نشد
&&
سرد و افسرده میان صف مستان چه کند
\\
آنک چون شیر نجست از صفت گرگی خویش
&&
چشم آهوفکن یوسف کنعان چه کند
\\
گر چه فرعون به در ریش مرصع دارد
&&
او حدیث چو در موسی عمران چه کند
\\
آنک او لقمه حرص است به طمع خامی
&&
او دم عیسی و یا حکمت لقمان چه کند
\\
بس کن و جمع شو و بیش پراکنده مگو
&&
بی دل جمع دو سه حرف پریشان چه کند
\\
شمس تبریز تویی صبح شکرریز تویی
&&
عاشق روز به شب قبله پنهان چه کند
\\
\end{longtable}
\end{center}
