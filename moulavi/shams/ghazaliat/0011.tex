\begin{center}
\section*{غزل شماره ۱۱: ای طوطی عیسی نفس وی بلبل شیرین نوا}
\label{sec:0011}
\addcontentsline{toc}{section}{\nameref{sec:0011}}
\begin{longtable}{l p{0.5cm} r}
ای طوطی عیسی نفس وی بلبل شیرین نوا
&&
هین زهره را کالیوه کن زان نغمه‌های جان فزا
\\
دعوی خوبی کن بیا تا صد عدو و آشنا
&&
با چهره‌ای چون زعفران با چشم تر آید گوا
\\
غم جمله را نالان کند تا مرد و زن افغان کند
&&
که داد ده ما را ز غم کو گشت در ظلم اژدها
\\
غم را بدرانی شکم با دورباش زیر و بم
&&
تا غلغل افتد در عدم از عدل تو ای خوش صدا
\\
ساقی تو ما را یاد کن صد خیک را پرباد کن
&&
ارواح را فرهاد کن در عشق آن شیرین لقا
\\
چون تو سرافیل دلی زنده کن آب و گلی
&&
دردم ز راه مقبلی در گوش ما نفخه خدا
\\
ما همچو خرمن ریخته گندم به کاه آمیخته
&&
هین از نسیم باد جان که را ز گندم کن جدا
\\
تا غم به سوی غم رود خرم سوی خرم رود
&&
تا گل به سوی گل رود تا دل برآید بر سما
\\
این دانه‌های نازنین محبوس مانده در زمین
&&
در گوش یک باران خوش موقوف یک باد صبا
\\
تا کار جان چون زر شود با دلبران هم‌بر شود
&&
پا بود اکنون سر شود که بود اکنون کهربا
\\
خاموش کن آخر دمی دستور بودی گفتمی
&&
سری که نفکندست کس در گوش اخوان صفا
\\
\end{longtable}
\end{center}
