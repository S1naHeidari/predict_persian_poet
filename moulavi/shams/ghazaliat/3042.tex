\begin{center}
\section*{غزل شماره ۳۰۴۲: چو مهر عشق سلیمان به هر دو کون تو داری}
\label{sec:3042}
\addcontentsline{toc}{section}{\nameref{sec:3042}}
\begin{longtable}{l p{0.5cm} r}
چو مهر عشق سلیمان به هر دو کون تو داری
&&
مکش تو دامن خود را که شرط نیست بیاری
\\
نه بند گردد بندی نه دل پذیرد پندی
&&
چو تنگ شکرقندی توام درون کناری
\\
طراوت سمنی تو چه رونق چمنی تو
&&
مگر تو عین منی تو مگر تو آینه واری
\\
چه نور پنج و ششی تو که آفت حبشی تو
&&
چو خوان عشق کشی تو ز سنگ آب برآری
\\
چه کیمیای زری تو چه رونق قمری تو
&&
چو دل ز سینه بری تو هزار سینه بیاری
\\
ز خلق جمله گسستم که عشق دوست بسستم
&&
چو در فنا بنشستم مرا چه کار به زاری
\\
بسوخت عشق تو خرمن نه جان بماند نه این تن
&&
جوی نیابی تو از من اگر هزار فشاری
\\
برون ز دور زمانی مثال گوهر کانی
&&
نشسته‌ایم چو جانی اگر کشی و بداری
\\
ز جام شربت شافی شدم به عشق تو لافی
&&
بیامدم زر صافی اگر تو کوره ناری
\\
کف از بهشت بشوید چو باغ عشق تو گوید
&&
کز او جواهر روید اگر چه سنگ بکاری
\\
دلی که عشق نوازد در این جهان بنسازد
&&
ازانک می‌نگذارد که یک زمانش بخاری
\\
تو شمس خسرو تبریز شراب باقی برریز
&&
براق عشق بکن تیز که بس لطیف سواری
\\
\end{longtable}
\end{center}
