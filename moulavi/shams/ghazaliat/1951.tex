\begin{center}
\section*{غزل شماره ۱۹۵۱: ای برادر تو چه مرغی خویشتن را بازبین}
\label{sec:1951}
\addcontentsline{toc}{section}{\nameref{sec:1951}}
\begin{longtable}{l p{0.5cm} r}
ای برادر تو چه مرغی خویشتن را بازبین
&&
گر تو دست آموز شاهی خویشتن را باز بین
\\
هر کی انبازی برید از خویش آن بازی مدان
&&
در جهان او را چو حق بی‌مثل و بی‌انباز بین
\\
ز آفتابی کآفتاب آسمان یک جام او است
&&
ذره‌ها و قطره‌ها را مست و دست انداز بین
\\
چونک قبله شاه یابی قبله اقبال شو
&&
چون دو دم خوردی ز جامش بخت را دمساز بین
\\
گفتم ای اکسیر بنما مس را چون زر کنی
&&
رو به صرافان دل آورد گفتا گاز بین
\\
گفتمش چون زنده کردی مرغ ابراهیم را
&&
گفت پر و بال برکن هم کنون پرواز بین
\\
گفتم از آغاز مرغ روح ما بی‌پر بده‌ست
&&
گفت هین بشکن قفس آغاز بی‌آغاز بین
\\
زان فروبسته دمی کت همدم و همراز نیست
&&
چشم بگشا هر دمی همراز بین همراز بین
\\
این دمی چندی که زد جان تو در سوز و نیاز
&&
چون دم عیسی به حضرت زنده و باساز بین
\\
خاک خواری را بمان چون خاک خواری پیشه گیر
&&
خاک را از بعد خواری در چمن اعزاز بین
\\
\end{longtable}
\end{center}
