\begin{center}
\section*{غزل شماره ۲۲۴۹: به وقت خواب بگیری مرا که هین برگو}
\label{sec:2249}
\addcontentsline{toc}{section}{\nameref{sec:2249}}
\begin{longtable}{l p{0.5cm} r}
به وقت خواب بگیری مرا که هین برگو
&&
چو اشتهای سماعت بود بگه‌تر گو
\\
چو من ز خواب سر و پای خویش گم کردم
&&
تو گوش من بگشایی که قصه از سر گو
\\
چو روی روز نهان شد به زیر طره شب
&&
بگیریم که از آن طره معنبر گو
\\
فتاده آتش خواب اندر این نیستان‌ها
&&
تو آمده که حدیث لب چو شکر گو
\\
و آنگهی به یکی بار کی شوی قانع
&&
غزل تمام کنم گوییم مکرر گو
\\
بیا بگو چه کنی گر ز خوابناکی خویش
&&
به تو بگوید لالا برو به عنبر گو
\\
از آنچ خورده‌ای و در نشاط آمده‌ای
&&
مرا از آن بخوران و حدیث درخور گو
\\
ز من چو می‌طلبی مطربی مستانه
&&
تو نیز با من بی‌دل ز جام و ساغر گو
\\
من این به طیبت گفتم وگر نه خاک توام
&&
مرا مبارک و قیماز خوان و سنجر گو
\\
\end{longtable}
\end{center}
