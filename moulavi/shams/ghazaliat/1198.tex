\begin{center}
\section*{غزل شماره ۱۱۹۸: سیمرغ کوه قاف رسیدن گرفت باز}
\label{sec:1198}
\addcontentsline{toc}{section}{\nameref{sec:1198}}
\begin{longtable}{l p{0.5cm} r}
سیمرغ کوه قاف رسیدن گرفت باز
&&
مرغ دلم ز سینه پریدن گرفت باز
\\
مرغی که تا کنون ز پی دانه مست بود
&&
درسوخت دانه را و طپیدن گرفت باز
\\
چشمی که غرقه بود به خون در شب فراق
&&
آن چشم روی صبح به دیدن گرفت باز
\\
صدیق و مصطفی به حریفی درون غار
&&
بر غار عنکبوت تنیدن گرفت باز
\\
دندان عیش کند شد از هجر ترش روی
&&
امروز قند وصل گزیدن گرفت باز
\\
پیراهن سیاه که پوشید روز فصل
&&
تا جایگاه ناف دریدن گرفت باز
\\
مستورگان مصر ز دیدار یوسفی
&&
هر یک ترنج و دست بریدن گرفت باز
\\
افغان ز یوسفی که زلیخاش در مزاد
&&
با تنگ‌های لعل خریدن گرفت باز
\\
آهوی چشم خونی آن شیر یوسفان
&&
در خون عاشقان بچریدن گرفت باز
\\
خاتون روح خانه نشین از سرای تن
&&
چادرکشان ز عشق دویدن گرفت باز
\\
دیگ خیال عشق دلارام خام پز
&&
سه پایه دماغ پزیدن گرفت باز
\\
نظاره خلیل کن آخر که شهد و شیر
&&
از اصبعین خویش مزیدن گرفت باز
\\
آن دل که توبه کرد ز عشقش ستیز شد
&&
افسون و مکر دوست شنیدن گرفت باز
\\
بر بام فکر خفته ستان دل به عشق ما
&&
یک یک ستاره را شمریدن گرفت باز
\\
سودای عشق لولی دزد سیاه کار
&&
بر زلف چون رسن بخزیدن گرفت باز
\\
صراف ناز ناقد نقد ضمیر عشق
&&
بر کف قراضه‌ها بگزیدن گرفت باز
\\
تبریز را کرامت شمس حقست و او
&&
گوش مرا به خویش کشیدن گرفت باز
\\
\end{longtable}
\end{center}
