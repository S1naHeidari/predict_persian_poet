\begin{center}
\section*{غزل شماره ۱۹۱۸: فرود آ تو ز مرکب بار می بین}
\label{sec:1918}
\addcontentsline{toc}{section}{\nameref{sec:1918}}
\begin{longtable}{l p{0.5cm} r}
فرود آ تو ز مرکب بار می بین
&&
وجودت را تو پود و تار می بین
\\
هر آن گلزار کاندر هجر مانده‌ست
&&
سراسر جان او پرخار می بین
\\
چو جمله راه‌های وصل را بست
&&
رخان عاشقان را زار می بین
\\
چو سررشته اشارت‌هاش دیدی
&&
بر آن رشته برو گلزار می بین
\\
ز جان‌ها جوق جوق از آتش او
&&
فغان لابه کنان مکثار می بین
\\
بزن تو چنگ در قانون شرطش
&&
سماع دلکش اوتار می بین
\\
به پیش ماجرای صدق آن شه
&&
سرافکنده همه اخیار می بین
\\
میان کودکان مکتب او
&&
چه کوه و بحر از احبار می بین
\\
چو بی‌میلی کند آن خدمت مه
&&
چو مه سرگشته و دوار می بین
\\
چو روی از منبرش برتافت جانی
&&
درآویزان ورا بر دار می بین
\\
اگر چه کار و باری بینی او را
&&
ولی نسبت به شه بی‌کار می بین
\\
خیالش دید جانم گفت آخر
&&
به هجرت می خورم من نار می بین
\\
بگفتا که عنایت بر فزون است
&&
ولیکن دیدن ناچار می بین
\\
اگر تو عاقلی گندم چو دیدی
&&
ز سنبل‌ها نه از انبار می بین
\\
دلت انبار و لطفم اصل سنبل
&&
اشارت بشنو و بسیار می بین
\\
خداوند شمس دین را گر ببینی
&&
به غیب اندر رو و ازهار می بین
\\
شود دیده گذاره سوی بی‌سو
&&
در او انوار در انوار می بین
\\
\end{longtable}
\end{center}
