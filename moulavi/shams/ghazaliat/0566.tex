\begin{center}
\section*{غزل شماره ۵۶۶: بتی کو زهره و مه را همه شب شیوه آموزد}
\label{sec:0566}
\addcontentsline{toc}{section}{\nameref{sec:0566}}
\begin{longtable}{l p{0.5cm} r}
بتی کو زهره و مه را همه شب شیوه آموزد
&&
دو چشم او به جادویی دو چشم چرخ بردوزد
\\
شما دل‌ها نگه دارید مسلمانان که من باری
&&
چنان آمیختم با او که دل با من نیامیزد
\\
نخست از عشق او زادم به آخر دل بدو دادم
&&
چو میوه زاید از شاخی از آن شاخ اندرآویزد
\\
ز سایه خود گریزانم که نور از سایه پنهانست
&&
قرارش از کجا باشد کسی کز سایه بگریزد
\\
سر زلفش همی‌گوید صلا زوتر رسن بازی
&&
رخ شمعش همی‌گوید کجا پروانه تا سوزد
\\
برای این رسن بازی دلاور باش و چنبر شو
&&
درافکن خویش در آتش چو شمع او برافروزد
\\
چو ذوق سوختن دیدی دگر نشکیبی از آتش
&&
اگر آب حیات آید تو را ز آتش نینگیزد
\\
\end{longtable}
\end{center}
