\begin{center}
\section*{غزل شماره ۲۲۴۲: تا که درآمد به باغ چهره گلنار تو}
\label{sec:2242}
\addcontentsline{toc}{section}{\nameref{sec:2242}}
\begin{longtable}{l p{0.5cm} r}
تا که درآمد به باغ چهره گلنار تو
&&
اه که چه سوز افکند در دل گل نار تو
\\
دود دل لاله‌ها ز آتش جان رنگ تو
&&
پشت بنفشه به خم از کشش بار تو
\\
غنچه گلزار جان روی تو را یاد کرد
&&
چشم چه خوش برگشاد بر هوس خار تو
\\
سوسن تیغی کشید خون سمن را بریخت
&&
تیغ به سوسن کی داد نرگس خون خوار تو
\\
بر مثل زاهدان جمله چمن خشک بود
&&
مستک و سرسبز شد از لب خمار تو
\\
از سر مستی عشق گفتم یار منی
&&
ور نه جز احول کی دید در دو جهان یار تو
\\
بر دل من خط توست مهر الست و بلی
&&
منکر آن خط مشو نک خط و اقرار تو
\\
گوشت کجا ماند و پوست در تن آن کس که او
&&
رفت نمکسودوار سوی نمکسار تو
\\
دامن تو دل گرفت دامن دل تن گرفت
&&
های از این کش مکش‌های از این کار تو
\\
خسرو جان شمس دین مفخر تبریزیان
&&
در دل تن عشق دل در دل دلدار تو
\\
\end{longtable}
\end{center}
