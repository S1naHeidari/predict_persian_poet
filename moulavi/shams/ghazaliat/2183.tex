\begin{center}
\section*{غزل شماره ۲۱۸۳: به پیشت نام جان گویم زهی رو}
\label{sec:2183}
\addcontentsline{toc}{section}{\nameref{sec:2183}}
\begin{longtable}{l p{0.5cm} r}
به پیشت نام جان گویم زهی رو
&&
حدیث گلستان گویم زهی رو
\\
تو این جا حاضر و شرمم نباشد
&&
که از حسن بتان گویم زهی رو
\\
چو شاه بی‌نشان عالم بیاراست
&&
من از شکل و نشان گویم زهی رو
\\
چو نور لامکان آفاق بگرفت
&&
من از جا و مکان گویم زهی رو
\\
به پیش این دکان که کان شادی است
&&
من از سود و زیان گویم زهی رو
\\
به پیش این چنین دانای اسرار
&&
کژی در دل نهان گویم زهی رو
\\
چو استاره و جهان شد محو خورشید
&&
فسانه این جهان گویم زهی رو
\\
اوان قاب قوسین است و ادنی
&&
حدیث خرکمان گویم زهی رو
\\
از آن جان که روان شد سوی جانان
&&
بر هر بی‌روان گویم زهی رو
\\
حدیثی را که جان هم نیست محرم
&&
من از راه دهان گویم زهی رو
\\
چو شاهنشاه صد جان و جهانی
&&
من از جان و جهان گویم زهی رو
\\
\end{longtable}
\end{center}
