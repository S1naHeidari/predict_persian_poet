\begin{center}
\section*{غزل شماره ۲۱۸۱: دل و جان را طربگاه و مقام او}
\label{sec:2181}
\addcontentsline{toc}{section}{\nameref{sec:2181}}
\begin{longtable}{l p{0.5cm} r}
دل و جان را طربگاه و مقام او
&&
شراب خم بی‌چون را قوام او
\\
همه عالم دهان خشکند و تشنه
&&
غذای جمله را داده تمام او
\\
غذاها هم غذا جویند از وی
&&
که گندم را دهد آب از غمام او
\\
عدم چون اژدهای فتنه جویان
&&
ببسته فتنه را حلق و مسام او
\\
سزای صد عتاب و صد عذابیم
&&
کشیده از سزای ما لگام او
\\
ز حلم او جهان گستاخ گشته
&&
که گویی ما شهانیم و غلام او
\\
برای مغز مخموران عشقش
&&
بجوشیده به دست خود مدام او
\\
کشیده گوش هشیاران به مستی
&&
زهی اقبال و بخت مستدام او
\\
پیمبر را چو پرده کرده در پیش
&&
پس آن پرده می‌گوید پیام او
\\
نکرده بندگان او را سلامی
&&
بر ایشان کرده از اول سلام او
\\
چه باشد گر شبی را زنده داری
&&
به عشق او که آرد صبح و شام او
\\
وگر خامی‌کنی غافل بخسپی
&&
بنگذارد تو را ای دوست خام او
\\
ز خردی تا کنون بس جا بخفتی
&&
کشانیدت ز پستی تا به بام او
\\
ز خاکی تا به چالاکی کشیدت
&&
بدادت دانش و ناموس و نام او
\\
مقامات نوت خواهد نمودن
&&
که تا خاصت کند ز انعام عام او
\\
به خردی هم ز مکتب می‌جهیدی
&&
چه نرمت کرد و پابرجا و رام او
\\
به خاکی و نباتی و به نطفه
&&
ستیزیدی درآوردت به دام او
\\
ز چندین ره به مهمانیت آورد
&&
نیاوردت برای انتقام او
\\
به وقت درد می‌دانی که او او است
&&
به خاکی می‌دهد اویی به وام او
\\
همه اویان چو خاشاکی نمایند
&&
چو بوی خود فرستد در مشام او
\\
سخن‌ها بانگ زنبوران نماید
&&
چو اندر گوش ما گوید کلام او
\\
نماید چرخ بیت العنکبوتی
&&
چو بنماید مقام بی‌مقام او
\\
همه عالم گرفته‌ست آفتابی
&&
زهی کوری که می‌گوید کدام او
\\
چو درماند نگوید او جز او را
&&
چو بجهد هر خسی را کرده نام او
\\
شکنجه بایدش زیرا که دزد است
&&
مقر ناید به نرمی‌و به کام او
\\
تو باری دزد خود را سیخ می‌زن
&&
چو می‌دانی که دزدیده‌ست جام او
\\
به یاری‌های شمس الدین تبریز
&&
شود بس مستخف و مستهام او
\\
خمش از پارسی تازی بگویم
&&
فؤاد ما تسلیه المدام
\\
\end{longtable}
\end{center}
