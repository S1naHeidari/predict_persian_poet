\begin{center}
\section*{غزل شماره ۲۶۱۰: چون بسته کنی راهی آخر بشنو آهی}
\label{sec:2610}
\addcontentsline{toc}{section}{\nameref{sec:2610}}
\begin{longtable}{l p{0.5cm} r}
چون بسته کنی راهی آخر بشنو آهی
&&
از بهر خدا بشنو فریاد و علی اللهی
\\
در روح نظر کردم بی‌رنگ چو آبی بود
&&
ناگاه پدید آمد در آب چنان ماهی
\\
آن آب به جوش آمد هستی به خروش آمد
&&
تا واشد و دریا شد این عالم چون چاهی
\\
دیدم که فراز آمد دریا و بشد قطره
&&
من قطره و او قطره گشتیم چو همراهی
\\
چون پیشترک رفتم دریا شد و بگرفتم
&&
او قطره شده دریا من قطره شده گاهی
\\
پیش آی تو دریا را نظاره بکن ما را
&&
باشد که تو هم افتی در مکر شهنشاهی
\\
آبی است به زیرش مه آبی است به زیرش که
&&
او چشم چنین بندد چون جادو دلخواهی
\\
با لعل تو کی جویم من ملک بدخشان را
&&
چاه و رسن زلفت والله که به از جاهی
\\
از غمزه جادواش شمس الحق تبریزی
&&
در سحر نمی‌بندد جز سینه آگاهی
\\
\end{longtable}
\end{center}
