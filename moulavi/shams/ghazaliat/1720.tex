\begin{center}
\section*{غزل شماره ۱۷۲۰: بار دگر ذره وار رقص کنان آمدیم}
\label{sec:1720}
\addcontentsline{toc}{section}{\nameref{sec:1720}}
\begin{longtable}{l p{0.5cm} r}
بار دگر ذره وار رقص کنان آمدیم
&&
زان سوی گردون عشق چرخ زنان آمدیم
\\
بر سر میدان عشق چونک یکی گو شدیم
&&
گه به کران تاختیم گه به میان آمدیم
\\
عشق نیاز آورد گر تو چنانی رواست
&&
ما چو از آن سوتریم ما نه چنان آمدیم
\\
خواجه مجلس تویی مجلسیان حاضرند
&&
آب چو آتش بیار ما نه بنان آمدیم
\\
شکر که ناداشت وار از سبب زخم تو
&&
چون که به جان آمدیم زود به جان آمدیم
\\
شمس حق این عشق تو تشنه خون من است
&&
تیغ و کفن در بغل بهر همان آمدیم
\\
جز نمکت نشکند شورش تبریز را
&&
فخر زمین در غمت شور زمان آمدیم
\\
\end{longtable}
\end{center}
