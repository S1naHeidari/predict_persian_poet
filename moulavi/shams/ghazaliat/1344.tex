\begin{center}
\section*{غزل شماره ۱۳۴۴: شتران مست شدستند ببین رقص جمل}
\label{sec:1344}
\addcontentsline{toc}{section}{\nameref{sec:1344}}
\begin{longtable}{l p{0.5cm} r}
شتران مست شدستند ببین رقص جمل
&&
ز اشتر مست که جوید ادب و علم و عمل
\\
علم ما داده او و ره ما جاده او
&&
گرمی ما دم گرمش نه ز خورشید حمل
\\
دم او جان دهدت روز نفخت بپذیر
&&
کار او کن فیکون‌ست نه موقوف علل
\\
ما در این ره همه نسرین و قرنفل کوبیم
&&
ما نه زان اشتر عامیم که کوبیم وحل
\\
شتران وحلی بسته این آب و گلند
&&
پیش جان و دل ما آب و گلی را چه محل
\\
ناقه الله بزاده به دعای صالح
&&
جهت معجزه دین ز کمرگاه جبل
\\
هان و هان ناقه حقیم تعرض مکنید
&&
تا نبرد سرتان را سر شمشیر اجل
\\
سوی مشرق نرویم و سوی مغرب نرویم
&&
تا ابد گام زنان جانب خورشید ازل
\\
هله بنشین تو بجنبان سر و می‌گوی بلی
&&
شمس تبریز نماید به تو اسرار غزل
\\
\end{longtable}
\end{center}
