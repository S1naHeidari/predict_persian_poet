\begin{center}
\section*{غزل شماره ۲۷۱۴: چو اسم شمس دین اسما تو دیدی}
\label{sec:2714}
\addcontentsline{toc}{section}{\nameref{sec:2714}}
\begin{longtable}{l p{0.5cm} r}
چو اسم شمس دین اسما تو دیدی
&&
خلاصه او است در اشیاء تو دیدی
\\
چه دارد عقل‌ها پیشش ز دانش
&&
برابر با سری کش پا تو دیدی
\\
منورتر به هر دو کون ای دل
&&
ز حلقه خاص او هیجا تو دیدی
\\
به مانندش ز اول تا به آخر
&&
بگو آخر کی دیده‌ست یا تو دیدی
\\
در آن گوهر نبوده‌ست هیچ نقصان
&&
اگر هستت خیال آن‌ها تو دیدی
\\
به پیش خدمتش اندر سجودند
&&
از آن سوی حجاب لا تو دیدی
\\
خدیو سینه پهن و سروبالا
&&
نه بالا است و نی پهنا تو دیدی
\\
شهی کش جن و انس اندر سجودند
&&
همه رویش در آن رعنا تو دیدی
\\
ورا حلمی که خاک آن برنتابد
&&
چنان حلمی در استغنا تو دیدی
\\
ز وصف تلخ خود زهرا یکی وصف
&&
به لعل شکر و زهرا تو دیدی
\\
ز فرمان کردنش سوی سماوات
&&
نهاده نردبان بالا تو دیدی
\\
چنان لؤلؤ به تابانی و خوبی
&&
که او را هست جان لالا تو دیدی
\\
کسی خود این شبه فانی دون را
&&
از او خواهد چنین کالا تو دیدی
\\
به نرمی در هوای هرزه آبی
&&
و یا آن عشق چون خارا تو دیدی
\\
برونم جمله رنج و اندرون گنج
&&
بدین وصف عجب ما را تو دیدی
\\
خداوند شمس دین را در دو عالم
&&
به ملک و بخت او همتا تو دیدی
\\
ز بهر آتش ای باد صبا تا
&&
رسانی خدمتی از ما تو دیدی
\\
چو خاک سنب اسب جبرئیل است
&&
همه تبریزیان احیا تو دیدی
\\
\end{longtable}
\end{center}
