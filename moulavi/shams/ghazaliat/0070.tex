\begin{center}
\section*{غزل شماره ۷۰: برات آمد برات آمد بنه شمع براتی را}
\label{sec:0070}
\addcontentsline{toc}{section}{\nameref{sec:0070}}
\begin{longtable}{l p{0.5cm} r}
برات آمد برات آمد بنه شمع براتی را
&&
خضر آمد خضر آمد بیار آب حیاتی را
\\
عمر آمد عمر آمد ببین سرزیر شیطان را
&&
سحر آمد سحر آمد بهل خواب سباتی را
\\
بهار آمد بهار آمد رهیده بین اسیران را
&&
به بستان آ به بستان آ ببین خلق نجاتی را
\\
چو خورشید حمل آمد شعاعش در عمل آمد
&&
ببین لعل بدخشان را و یاقوت زکاتی را
\\
همان سلطان همان سلطان که خاکی را نبات آرد
&&
ببخشد جان ببخشد جان نگاران نباتی را
\\
درختان بین درختان بین همه صایم همه قایم
&&
قبول آمد قبول آمد مناجات صلاتی را
\\
ز نورافشان ز نورافشان نتانی دید ذاتش را
&&
ببین باری ببین باری تجلی صفاتی را
\\
گلستان را گلستان را خماری بد ز جور دی
&&
فرستاد او فرستاد او شرابات نباتی را
\\
بشارت ده بشارت ده به محبوسان جسمانی
&&
که حشر آمد که حشر آمد شهیدان رفاتی را
\\
شقایق را شقایق را تو شاکر بین و گفتی نی
&&
تو هم نو شو تو هم نو شو بهل نطق بیاتی را
\\
شکوفه و میوه بستان برات هر درخت آمد
&&
که بیخم نیست پوسیده ببین وصل سماتی را
\\
زبان صدق و برق رو برات مؤمنان آمد
&&
که جانم واصل وصلست و هشته بی‌ثباتی را
\\
\end{longtable}
\end{center}
