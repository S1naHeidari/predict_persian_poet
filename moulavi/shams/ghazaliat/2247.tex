\begin{center}
\section*{غزل شماره ۲۲۴۷: مرا اگر تو نیابی به پیش یار بجو}
\label{sec:2247}
\addcontentsline{toc}{section}{\nameref{sec:2247}}
\begin{longtable}{l p{0.5cm} r}
مرا اگر تو نیابی به پیش یار بجو
&&
در آن بهشت و گلستان و سبزه زار بجو
\\
چو سایه خسپم و کاهل مرا اگر جویی
&&
به زیر سایه آن سرو پایدار بجو
\\
چو خواهیم که ببینی خراب و غرق شراب
&&
بیا حوالی آن چشم پرخمار بجو
\\
اگر ز روز شمردن ملول و سیر شدی
&&
درآ به دور و قدح‌های بی‌شمار بجو
\\
در آن دو دیده مخمور و قلزم پرنور
&&
درآ جواهر اسرار کردگار بجو
\\
دلی که هیچ نگرید به پیش دلبر جو
&&
گلی که هیچ نریزد در آن بهار بجو
\\
زهی فسرده کسی کو قرار می‌جوید
&&
تو جان عاشق سرمست بی‌قرار بجو
\\
اگر چراغ نداری از او چراغ بخواه
&&
وگر عقار نداری از او عقار بجو
\\
به مجلس تو اگر دوش بیخودی کردم
&&
تو عذر عقل زبونم از آن عذار بجو
\\
تو هر چه را که بجویی ز اصل و کانش جوی
&&
ز مشک و گل نفس خوش خلش ز خار بجو
\\
خیال یار سواره همی‌رسد ای دل
&&
پیام‌های غریب از چنین سوار بجو
\\
به نزد او همه جان‌های رفتگان جمعند
&&
کنار پرگلشان را در آن کنار بجو
\\
چو صبح پیش تو آید از او صبوح بخواه
&&
چو شب به پیش تو آید در او نهار بجو
\\
چو مردمک تو خمش کن مقام تو چشم است
&&
وگر نه آن نظرستت در انتظار بجو
\\
چو شمس مفخر تبریز دیده فقر است
&&
فقیروار مر او را در افتقار بجو
\\
\end{longtable}
\end{center}
