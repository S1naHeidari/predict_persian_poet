\begin{center}
\section*{غزل شماره ۲۳۰۳: ناگاه درافتادم زان قصر و سراپرده}
\label{sec:2303}
\addcontentsline{toc}{section}{\nameref{sec:2303}}
\begin{longtable}{l p{0.5cm} r}
ناگاه درافتادم زان قصر و سراپرده
&&
در قعر چنین چاهی ناخورده و نابرده
\\
دنیا نبود عیدم من زشتی او دیدم
&&
گلگونه نهد بر رو آن روسپی زرده
\\
گلگونه چه آراید آن خاربن بد را
&&
آن خار فرورفته در هر جگر و گرده
\\
با تارک گل آمد موبند فروهشته
&&
ابروی خود از وسمه آن کور سیه کرده
\\
منگر تو به خلخالش ساق سیهش را بین
&&
خوش آید شب بازی لیک از سپس پرده
\\
رو دست بشو از وی ای صوفی روشسته
&&
دل را بستر از وی ای مرد سراسترده
\\
بدبخت و گران جانی کو بخت از او جوید
&&
دربند بزرگی شد می‌سوزد چون خرده
\\
فریاد رس ای جانان ما را ز گران جانان
&&
ای از عدمی ما را در چرخ درآورده
\\
خاموش سخن می‌ران زان خوش دم بی‌پایان
&&
تا چند سخن سازی تو زین دم بشمرده
\\
\end{longtable}
\end{center}
