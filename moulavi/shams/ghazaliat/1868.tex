\begin{center}
\section*{غزل شماره ۱۸۶۸: دروازه هستی را جز ذوق مدان ای جان}
\label{sec:1868}
\addcontentsline{toc}{section}{\nameref{sec:1868}}
\begin{longtable}{l p{0.5cm} r}
دروازه هستی را جز ذوق مدان ای جان
&&
این نکته شیرین را در جان بنشان ای جان
\\
زیرا عرض و جوهر از ذوق برآرد سر
&&
ذوق پدر و مادر کردت مهمان ای جان
\\
هر جا که بود ذوقی ز آسیب دو جفت آید
&&
زان یک شدن دو تن ذوق است نشان ای جان
\\
هر حس به محسوسی جفت است یکی گشته
&&
هر عقلی به معقولی جفت و نگران ای جان
\\
گر جفت شوی ای حس با آنک حست کرد او
&&
وز غیر بپرهیزی باشی سلطان ای جان
\\
ذوقی که ز خلق آید زو هستی تن زاید
&&
ذوقی که ز حق آید زاید دل و جان ای جان
\\
کو چشم که تا بیند هر گوشه تتق بسته
&&
هر ذره بپیوسته با جفت نهان ای جان
\\
آمیخته با شاهد هم عاشق و هم زاهد
&&
وز ذوق نمی‌گنجد در کون و مکان ای جان
\\
پنهان ز همه عالم گرمابه زده هر دم
&&
هم پیر خردپیشه هم جان جوان ای جان
\\
پنهان مکن ای رستم پنهان تو را جستم
&&
احوال تو دانستم تو عشوه مخوان ای جان
\\
گر روی ترش داری دانیم که طراری
&&
ز احداث همی‌ترسی وز مکر عوان ای جان
\\
در کنج عزبخانه حوری چو دردانه
&&
دور از لب بیگانه خفته‌ست ستان ای جان
\\
صد عشق همی‌بازد صد شیوه همی‌سازد
&&
آن لحظه که می یازد بوسه بستان ای جان
\\
بر ظاهر دریا کی بینی خورش ماهی
&&
کان آب تتق آمد بر عیش کنان ای جان
\\
چندان حیوان آن سو می خاید و می زاید
&&
چون گرگ گرو برده پنهان ز شبان ای جان
\\
خنبک زده هر ذره بر معجب بی‌بهره
&&
کب حیوان را کی داند حیوان ای جان
\\
اندر دل هر ذره تابان شده خورشیدی
&&
در باطن هر قطره صد جوی روان ای جان
\\
خاموش که آن لقمه هر بسته دهان خاید
&&
تا لقمه نیندازی بربند دهان ای جان
\\
\end{longtable}
\end{center}
