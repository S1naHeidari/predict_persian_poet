\begin{center}
\section*{غزل شماره ۶۰۴: هرک آتش من دارد او خرقه ز من دارد}
\label{sec:0604}
\addcontentsline{toc}{section}{\nameref{sec:0604}}
\begin{longtable}{l p{0.5cm} r}
هرک آتش من دارد او خرقه ز من دارد
&&
زخمی چو حسینستش جامی چو حسن دارد
\\
غم نیست اگر ماهش افتاد در این چاهش
&&
زیرا رسن زلفش در دست رسن دارد
\\
نفس ار چه که زاهد شد او راست نخواهد شد
&&
گر راستیی خواهی آن سرو چمن دارد
\\
صد مه اگر افزاید در چشم خوشش ناید
&&
با تنگی چشم او کان خوب ختن دارد
\\
از عکس ویست ای جان گر چرخ ضیا دارد
&&
یا باغ گل خندان یا سرو و سمن دارد
\\
گر صورت شمع او اندر لگن غیرست
&&
بر سقف زند نورش گر شمع لگن دارد
\\
گر با دگرانی تو در ما نگرانی تو
&&
ما روح صفا داریم گر غیر بدن دارد
\\
بس مست شدست این دل وز دست شدست این دل
&&
گر خرد شدست این دل زان زلف شکن دارد
\\
شمس الحق تبریزی شاه همه شیرانست
&&
در بیشه جان ما آن شیر وطن دارد
\\
\end{longtable}
\end{center}
