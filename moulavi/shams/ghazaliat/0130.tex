\begin{center}
\section*{غزل شماره ۱۳۰: بیدار کنید مستیان را}
\label{sec:0130}
\addcontentsline{toc}{section}{\nameref{sec:0130}}
\begin{longtable}{l p{0.5cm} r}
بیدار کنید مستیان را
&&
از بهر نبیذ همچو جان را
\\
ای ساقی باده بقایی
&&
از خم قدیم گیر آن را
\\
بر راه گلو گذر ندارد
&&
لیکن بگشاید او زبان را
\\
جان را تو چو مشک ساز ساقی
&&
آن جان شریف غیب دان را
\\
پس جانب آن صبوحیان کش
&&
آن مشک سبک دل گران را
\\
وز ساغرهای چشم مستت
&&
درده تو فلان بن فلان را
\\
از دیده به دیده باده‌ای ده
&&
تا خود نشود خبر دهان را
\\
زیرا ساقی چنان گذارد
&&
اندر مجلس می نهان را
\\
بشتاب که چشم ذره ذره
&&
جویا گشتست آن عیان را
\\
آن نافه مشک را به دست آر
&&
بشکاف تو ناف آسمان را
\\
زیرا غلبات بوی آن مشک
&&
صبری بنهشت یوسفان را
\\
چون نامه رسید سجده‌ای کن
&&
شمس تبریز درفشان را
\\
\end{longtable}
\end{center}
