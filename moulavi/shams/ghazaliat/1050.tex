\begin{center}
\section*{غزل شماره ۱۰۵۰: کی باشد اختری در اقطار}
\label{sec:1050}
\addcontentsline{toc}{section}{\nameref{sec:1050}}
\begin{longtable}{l p{0.5cm} r}
کی باشد اختری در اقطار
&&
در برج چنین مهی گرفتار
\\
آواره شده ز کفر و ایمان
&&
اقرار به پیش او چو انکار
\\
کس دید دلی که دل ندارد
&&
با جان فنا به تیغ جان دار
\\
من دیدم اگر کسی ندیدست
&&
زیرا که مرا نمود دیدار
\\
علم و عملم قبول او بس
&&
ای من ز جز این قبول بیزار
\\
گر خواب شبم ببست آن شه
&&
بخشید وصال و بخت بیدار
\\
این وصل به از هزار خوابست
&&
از خواب مکن تو یاد زنهار
\\
از گریه خود چه داند آن طفل
&&
کاندر دل‌ها چه دارد آثار
\\
می‌گرید بی‌خبر ولیکن
&&
صد چشمه شیر از او در اسرار
\\
بگری تو اگر اثر ندانی
&&
کز گریه تست خلد و انهار
\\
امشب کر و فر شهریاریش
&&
اندر ده ماست شاه و سالار
\\
نی خواب رها کند نه آرام
&&
آن صبح صفا و شیر کرار
\\
\end{longtable}
\end{center}
