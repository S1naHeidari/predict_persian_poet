\begin{center}
\section*{غزل شماره ۴۰۹: تا نلغزی که ز خون راه پس و پیش‌ترست}
\label{sec:0409}
\addcontentsline{toc}{section}{\nameref{sec:0409}}
\begin{longtable}{l p{0.5cm} r}
تا نلغزی که ز خون راه پس و پیش‌ترست
&&
آدمی دزد ز زردزد کنون بیشترست
\\
گربزانند که از عقل و خبر می‌دزدند
&&
خود چه دارند کسی را که ز خود بی‌خبرست
\\
خود خود را تو چنین کاسد و بی‌خصم مدان
&&
که جهان طالب زر و خود تو کان زرست
\\
که رسول حق الناس معادن گفته‌ست
&&
معدن نقره و زرست و یقین پرگهرست
\\
گنج یابی و در او عمر نیابی تو به گنج
&&
خویش دریاب که این گنج ز تو بر گذرست
\\
خویش دریاب و حذر کن تو ولیکن چه کنی
&&
که یکی دزد سبک دست در این ره حذرست
\\
سحر ار چند که تاریست حساب روزست
&&
هر که را روی سوی شمس بود چون سحرست
\\
روح‌ها مست شود از دم صبح از پی آنک
&&
صبح را روی به شمس است و حریف نظرست
\\
چند بر بوک و مگر مهره فروگردانی
&&
که تو بس مفلسی و چرخ فلک پاک برست
\\
مغز پالوده و بر هیچ نه در خواب شدی
&&
گوییا لقمه هر روزه تو مغز خرست
\\
بیشتر جان کن و زر جمع کن و خوشدل باش
&&
که همه سیم و زر و مال تو مار سقرست
\\
یک شب از بهر خدا بی‌خور و بی‌خواب بزی
&&
صد شب از بهر هوا نفس تو بی‌خواب و خورست
\\
از سر درد و دریغ از پس هر ذره خاک
&&
آه و فریاد همی‌آید گوش تو کرست
\\
خون دل بر رخت افشان به سحرگاه از آنک
&&
توشه راه تو خون دل و آه سحرست
\\
دل پرامید کن و صیقلیش ده به صفا
&&
که دل پاک تو آیینه خورشید فرست
\\
مونس احمد مرسل به جهان کیست بگو
&&
شمس تبریز شهنشاه که احدی الکبرست
\\
\end{longtable}
\end{center}
