\begin{center}
\section*{غزل شماره ۳۸۳: هین که گردن سست کردی کو کبابت کو شرابت}
\label{sec:0383}
\addcontentsline{toc}{section}{\nameref{sec:0383}}
\begin{longtable}{l p{0.5cm} r}
هین که گردن سست کردی کو کبابت کو شرابت
&&
هین که بس تاریک رویی ای گرفته آفتابت
\\
یاد داری که ز مستی با خرد استیزه بستی
&&
چون کلیدش را شکستی از کی باشد فتح بابت
\\
در غم شیرین نجوشی لاجرم سرکه فروشی
&&
آب حیوان را ببستی لاجرم رفتست آبت
\\
بوالمعالی گشته بودی فضل و حجت می‌نمودی
&&
نک محک عشق آمد کو سؤالت کو جوابت
\\
مهتر تجار بودی خویش قارون می‌نمودی
&&
خواب بود و آن فنا شد چونک از سر رفت خوابت
\\
بس زدی تو لاف زفتی عاقبت در دوغ رفتی
&&
می‌خور اکنون آنچ داری دوغ آمد خمر نابت
\\
مخلص و معنی این‌ها گر چه دانی هم نهان کن
&&
اندر الواح ضمیری تا نیاید در کتابت
\\
\end{longtable}
\end{center}
