\begin{center}
\section*{غزل شماره ۷۳: آمد بت میخانه تا خانه برد ما را}
\label{sec:0073}
\addcontentsline{toc}{section}{\nameref{sec:0073}}
\begin{longtable}{l p{0.5cm} r}
آمد بت میخانه تا خانه برد ما را
&&
بنمود بهار نو تا تازه کند ما را
\\
بگشاد نشان خود بربست میان خود
&&
پر کرد کمان خود تا راه زند ما را
\\
صد نکته دراندازد صد دام و دغل سازد
&&
صد نرد عجب بازد تا خوش بخورد ما را
\\
رو سایه سروش شو پیش و پس او می‌دو
&&
گر چه چو درخت نو از بن بکند ما را
\\
گر هست دلش خارا مگریز و مرو یارا
&&
کاول بکشد ما را و آخر بکشد ما را
\\
چون ناز کند جانان اندر دل ما پنهان
&&
بر جمله سلطانان صد ناز رسد ما را
\\
بازآمد و بازآمد آن عمر دراز آمد
&&
آن خوبی و ناز آمد تا داغ نهد ما را
\\
آن جان و جهان آمد وان گنج نهان آمد
&&
وان فخر شهان آمد تا پرده درد ما را
\\
می‌آید و می‌آید آن کس که همی‌باید
&&
وز آمدنش شاید گر دل بجهد ما را
\\
شمس الحق تبریزی در برج حمل آمد
&&
تا بر شجر فطرت خوش خوش بپزد ما را
\\
\end{longtable}
\end{center}
