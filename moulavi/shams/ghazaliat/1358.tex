\begin{center}
\section*{غزل شماره ۱۳۵۸: به گوش دل پنهانی بگفت رحمت کل}
\label{sec:1358}
\addcontentsline{toc}{section}{\nameref{sec:1358}}
\begin{longtable}{l p{0.5cm} r}
به گوش دل پنهانی بگفت رحمت کل
&&
که هر چه خواهی می‌کن ولی ز ما مسکل
\\
تو آن ما و من آن تو همچو دیده و روز
&&
چرا روی ز بر من به هر غلیظ و عتل
\\
بگفت دل که سکستن ز تو چگونه بود
&&
چگونه بی ز دهلزن کند غریو دهل
\\
همه جهان دهلند و تویی دهلزن و بس
&&
کجا روند ز تو چونک بسته است سبل
\\
جواب داد که خود را دهل شناس و مباش
&&
گهی دهلزن و گاهی دهل که آرد ذل
\\
نجنبد این تن بیچاره تا نجنبد جان
&&
که تا فرس بنجنبد بر او نجنبد جل
\\
دل تو شیر خدایست و نفس تو فرس است
&&
چنان که مرکب شیر خدای شد دلدل
\\
چو درخور تک دلدل نبود عرصه عقل
&&
ز تنگنای خرد تاخت سوی عرصه قل
\\
تو را و عقل تو را عشق و خارخار چراست
&&
که وقت شد که بروید ز خار تو آن گل
\\
از این غم ار چه ترش روست مژده‌ها بشنو
&&
که گر شبی سحر آمد وگر خماری مل
\\
ز آه آه تو جوشید بحر فضل اله
&&
مسافر امل تو رسید تا آمل
\\
دمی رسید که هر شوق از او رسد به مشوق
&&
شهی رسید کز او طوق می شود هر غل
\\
حطام داد از این جیفه دایه تبدیل
&&
در آفتاب فکنده‌ست ظل حق غلغل
\\
از این همه بگذر بی‌گه آمدست حبیب
&&
شبم یقین شب قدرست قل للیلی طل
\\
چو وحی سر کند از غیب گوش آن سر باش
&&
از آنک اذن من الراس گفت صدر رسل
\\
تو بلبل چمنی لیک می توانی شد
&&
به فضل حق چمن و باغ با دو صد بلبل
\\
خدای را بنگر در سیاست عالم
&&
عقول را بنگر در صناعت انمل
\\
چو مست باشد عاشق طمع مکن خمشی
&&
چو نان رسد به گرسنه مگو که لاتأکل
\\
ز حرف بگذر و چون آب نقش‌ها مپذیر
&&
که حرف و صوت ز دنیاست و هست دنیا پل
\\
\end{longtable}
\end{center}
