\begin{center}
\section*{غزل شماره ۳۱۶۸: یا ملک المغرب والمشرق}
\label{sec:3168}
\addcontentsline{toc}{section}{\nameref{sec:3168}}
\begin{longtable}{l p{0.5cm} r}
یا ملک المغرب والمشرق
&&
مثلک فی االعالم یخلق
\\
باده ده ای ساقی هر متقی
&&
بادهٔ شاهنشهی راوقی
\\
جان سخن بخش که از تف او
&&
گردد هر گنگ خرف منطقی
\\
بر در حیرت، بکش اندیشه را
&&
حاکم ارواح و شه مطلقی
\\
جنت حسنت جو تجلی کند
&&
باغ شود دورخ بر هر شقی
\\
چون بگریزی نرسد در تو کس
&&
ور بگریزیم ز تو، سابقی
\\
ظلمت و نور از تو تحیر درند
&&
تا تو حقی یا که تو نور حقی
\\
گشت شب و روز کنون غرق نور
&&
نیست مهت مغربی و مشرقی
\\
لابه کنی، باده دهی رایگان
&&
ساقی دریا صفت مشفقی
\\
مرده همی‌باید و قلب سلیم
&&
زیرکی از خواجه بود احمقی
\\
فکرت اگر راحت جانها بدی
&&
باده نجستی خرد و موسقی
\\
فرد چرایی تو ز من؟! اگر منی
&&
از چه تو عذرایی اگر وامقی؟!
\\
غنچه صفت چشم ببستی ز گل
&&
رو، بهمان خار کشی لایقی
\\
خار کشانند همه، گر شهند
&&
جز که تو بر گلشن جان عاشقی
\\
خامش باش و بنگر فتح باب
&&
چند پی هر سخن مغلقی؟!
\\
\end{longtable}
\end{center}
