\begin{center}
\section*{غزل شماره ۱۳۶۰: باده ده ای ساقی جان باده بی‌درد و دغل}
\label{sec:1360}
\addcontentsline{toc}{section}{\nameref{sec:1360}}
\begin{longtable}{l p{0.5cm} r}
باده ده ای ساقی جان باده بی‌درد و دغل
&&
کار ندارم جز از این گر بزیم تا به اجل
\\
هات حبیبی سکرا لا بفتور و کسل
&&
یقطع عن شاربه کل ملال و فشل
\\
باده چو زر ده که زرم ساغر پر ده که نرم
&&
غرقه مقصود شدی تا چه کنی علم و عمل
\\
اصبح قلبی سهرا من سکر مفتخرا
&&
ان کذب الیوم صدق ان ظلم الیوم عدل
\\
ای قدح امروز تو را طاق و طرنبیست بیا
&&
باده خنب ملکی داده حق عز و جل
\\
طفت به معتمرا فزت به مفتخرا
&&
من سقی الیوم کذی جمله ما دام حصل
\\
مست و خوشی خواجه حسن نی نی چنان مست که من
&&
کیسه زر مست کند لیک نه چون جام ازل
\\
لواء نا مرتفع و شملنا مجتمع
&&
و روحنا کما تری فی درجات و دول
\\
توبه ما جان عمو توبه ماهیست ز جو
&&
از دل و جان توبه کند هیچ تن ای شیخ اجل
\\
عشقک قد جادلنا ثم عدا جادلنا
&&
من سکر مفتضح شاربه حیث دخل
\\
بحر که مسجور بود تلخ بود شور بود
&&
در دل ماهی روشش به بود از قند و عسل
\\
یا اسدا عن لنا فنعم ما سن لنا
&&
حبک قد حببنا فاعف لنا کل زلل
\\
بس بود ای مست خمش جان ز بدن رست خمش
&&
باده ستان که دگران عربده دارند و جدل
\\
اسکت یا صاح کفی واعف عفا الله عفا
&&
هات رحیقا به صفا قد وصل الوصل وصل
\\
\end{longtable}
\end{center}
