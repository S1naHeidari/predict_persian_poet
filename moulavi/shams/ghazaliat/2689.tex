\begin{center}
\section*{غزل شماره ۲۶۸۹: منم غرقه درون جوی باری}
\label{sec:2689}
\addcontentsline{toc}{section}{\nameref{sec:2689}}
\begin{longtable}{l p{0.5cm} r}
منم غرقه درون جوی باری
&&
نهانم می‌خلد در آب خاری
\\
اگر چه خار را من می‌نبینم
&&
نیم خالی ز زخم خار باری
\\
ندانم تا چه خار است اندر این جوی
&&
که خالی نیست جان از خارخاری
\\
تنم را بین که صورتگر ز سوزن
&&
بر او بنگاشت هر سویی نگاری
\\
چو پیراهن برون افکندم از سر
&&
به دریا درشدم مرغاب واری
\\
که غسل آرم برون آیم به پاکی
&&
به خنده گفت موج بحر کاری
\\
مثال کاسه چوبین بگشتم
&&
بر آن آبی که دارد سهم ناری
\\
نمی‌دانم که آن ساحل کجا شد
&&
که پیدا نیست دریا را کناری
\\
تو شمس الدین تبریز ار ملولی
&&
به هر لحظه چه افروزی شراری
\\
\end{longtable}
\end{center}
