\begin{center}
\section*{غزل شماره ۱۵۸۵: ای جهان آب و گل تا من تو را بشناختم}
\label{sec:1585}
\addcontentsline{toc}{section}{\nameref{sec:1585}}
\begin{longtable}{l p{0.5cm} r}
ای جهان آب و گل تا من تو را بشناختم
&&
صد هزاران محنت و رنج و بلا بشناختم
\\
تو چراگاه خرانی نی مقام عیسیی
&&
این چراگاه خران را من چرا بشناختم
\\
آب شیرینم ندادی تا که خوان گسترده‌ای
&&
دست و پایم بسته‌ای تا دست و پا بشناختم
\\
دست و پا را چون نبندی گاهواره ت خواند حق
&&
دست و پا را برگشایم پاگشا بشناختم
\\
چون درخت از زیر خاکی دست‌ها بالا کنم
&&
در هوای آن کسی کز وی هوا بشناختم
\\
ای شکوفه تو به طفلی چون شدی پیر تمام
&&
گفت رستم از صبا تا من صبا بشناختم
\\
شاخ بالا زان رود زیرا ز بالا آمده‌ست
&&
سوی اصل خویش یازم کاصل را بشناختم
\\
زیر و بالا چند گویم لامکان اصل من است
&&
من نه از جایم کجا را از کجا بشناختم
\\
نی خمش کن در عدم رو در عدم ناچیز شو
&&
چیزها را بین که از ناچیزها بشناختم
\\
\end{longtable}
\end{center}
