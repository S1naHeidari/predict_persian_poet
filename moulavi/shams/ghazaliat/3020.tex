\begin{center}
\section*{غزل شماره ۳۰۲۰: پیشتر آ پیشتر چند از این رهزنی}
\label{sec:3020}
\addcontentsline{toc}{section}{\nameref{sec:3020}}
\begin{longtable}{l p{0.5cm} r}
پیشتر آ پیشتر چند از این رهزنی
&&
چون تو منی من توام چند تویی و منی
\\
نور حقیم و زجاج با خود چندین لجاج
&&
از چه گریزد چنین روشنی از روشنی
\\
ما همه یک کاملیم از چه چنین احولیم
&&
خوار چرا بنگرد سوی فقیران غنی
\\
راست چرا بنگرد سوی چپ خویش خوار
&&
هر دو چو دست تواند چه یمنی چه دنی
\\
ما همه یک گوهریم یک خرد و یک سریم
&&
لیک دوبین گشته‌ایم زین فلک منحنی
\\
رخت از این پنج و شش جانب توحید کش
&&
عرعر توحید را چند کنی منثنی
\\
هین ز منی خیز کن با همه آمیز کن
&&
با خود خود حبه‌ای با همه چون معدنی
\\
هر چه کند شیر نر سگ بکند هم سگی
&&
هر چه کند روح پاک تن بکند هم تنی
\\
روح یکی دان و تن گشته عدد صد هزار
&&
همچو که بادام‌ها در صفت روغنی
\\
چند لغت در جهان جمله به معنی یکی
&&
آب یکی گشت چون خابیه‌ها بشکنی
\\
جان بفرستد خبر جانب هر بانظر
&&
چون که به توحید تو دل ز سخن برکنی
\\
\end{longtable}
\end{center}
