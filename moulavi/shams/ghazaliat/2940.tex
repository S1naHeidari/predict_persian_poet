\begin{center}
\section*{غزل شماره ۲۹۴۰: چون زخمه رجا را بر تار می‌کشانی}
\label{sec:2940}
\addcontentsline{toc}{section}{\nameref{sec:2940}}
\begin{longtable}{l p{0.5cm} r}
چون زخمه رجا را بر تار می‌کشانی
&&
کاهل روان ره را در کار می‌کشانی
\\
ای عشق چون درآیی در لطف و دلربایی
&&
دامان جان بگیری تا یار می‌کشانی
\\
ایمن کنی تو جان را کوری رهزنان را
&&
دزدان نقد دل را بر دار می‌کشانی
\\
سوداییان جان را از خود دهی مفرح
&&
صفراییان زر را بس زار می‌کشانی
\\
مهجور خارکش را گلزار می‌نمایی
&&
گلروی خارخو را در خار می‌کشانی
\\
موسی خاک رو را بر بحر می‌نشانی
&&
فرعون بوش جو را در عار می‌کشانی
\\
موسی عصا بگیرد تا یار خویش سازد
&&
ماری کنی عصا را چون مار می‌کشانی
\\
چون مار را بگیرد یابد عصای خود را
&&
این نعل بازگونه هموار می‌کشانی
\\
آن کو در آتش افتد راهش دهی به آبی
&&
و آن کو در آب آید در نار می‌کشانی
\\
ای دل چه خوش ز پرده سرمست و باده خورده
&&
سر را برهنه کرده دستار می‌کشانی
\\
ما را مده به غیری تا سوی خود کشاند
&&
ما را تو کش ازیرا شهوار می‌کشانی
\\
تا یار زنده باشد کوهی کنی تو سدش
&&
چون در غمش بکشتی در غار می‌کشانی
\\
خاموش و درکش این سر خوش خامشانه می‌خور
&&
زیرا که چون خموشی اسرار می‌کشانی
\\
\end{longtable}
\end{center}
