\begin{center}
\section*{غزل شماره ۲۹۴۵: ای حیله‌هات شیرین تا کی مرا فریبی}
\label{sec:2945}
\addcontentsline{toc}{section}{\nameref{sec:2945}}
\begin{longtable}{l p{0.5cm} r}
ای حیله‌هات شیرین تا کی مرا فریبی
&&
آن را که ملک کردی دیگر چرا فریبی
\\
اما چو جمله عالم ملک تو است کلی
&&
بیرون ز ملکت خود دیگر که را فریبی
\\
داوود را فریبی در دام ملک و دولت
&&
و ایوب را دگرگون اندر بلا فریبی
\\
آن را به دانه بردی وین را به دام بردی
&&
آن دام دانه شد چون تو خوش لقا فریبی
\\
فرعون عالمی را بفریبد و نداند
&&
کان خاین دغا را هم در دغا فریبی
\\
ای کمترین فریبت صد خونبهای صیدان
&&
ای پربها که او را تو بی‌بها فریبی
\\
ای دل خدا کسی را دانی چه سان فریبد
&&
آخر تو جملگان را خود از خدا فریبی
\\
\end{longtable}
\end{center}
