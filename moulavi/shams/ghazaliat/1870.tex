\begin{center}
\section*{غزل شماره ۱۸۷۰: ای نفس چو سگ آخر تا چند زنی دندان}
\label{sec:1870}
\addcontentsline{toc}{section}{\nameref{sec:1870}}
\begin{longtable}{l p{0.5cm} r}
ای نفس چو سگ آخر تا چند زنی دندان
&&
وز کبر کسان رنجی و اندر تو دو صد چندان
\\
گریانی و پرزهری با خلق چه باقهری
&&
مانند سر بریان گشته که منم خندان
\\
من صوفی باصوفم من آمر معروفم
&&
چون شحنه بود آن کس کو باشد در زندان
\\
معذوری خود دیده در خویش ترنجیده
&&
عذر دگران خواهد از باب هنرمندان
\\
بر دانش و حال خود تأویل کنی قرآن
&&
وان گاه هم از قرآن در خلق زنی سندان
\\
آب حیوان یابی گر خاک شوی ره را
&&
وز باد و بروت آیی در نار تو دربندان
\\
بگریز از این دربند بر جمله تو در دربند
&&
جز شمس حق تبریز سلطان شکرقندان
\\
\end{longtable}
\end{center}
