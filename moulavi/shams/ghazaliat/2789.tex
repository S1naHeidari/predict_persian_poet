\begin{center}
\section*{غزل شماره ۲۷۸۹: پیش شمع نور جان دل هست چون پروانه‌ای}
\label{sec:2789}
\addcontentsline{toc}{section}{\nameref{sec:2789}}
\begin{longtable}{l p{0.5cm} r}
پیش شمع نور جان دل هست چون پروانه‌ای
&&
در شعاع شمع جانان دل گرفته خانه‌ای
\\
سرفرازی شیرگیری مست عشقی فتنه‌ای
&&
نزد جانان هوشیاری نزد خود دیوانه‌ای
\\
خشم شکلی صلح جانی تلخ رویی شکری
&&
من بدین خویشی ندیدم در جهان بیگانه‌ای
\\
با هزاران عقل بینا چون ببیند روی شمع
&&
پر او در پای پیچد درفتد مستانه‌ای
\\
خرمن آتش گرفته صحن صحراهای عشق
&&
گندم او آتشین و جان او پیمانه‌ای
\\
نور گیرد جمله عالم بر مثال کوه طور
&&
گر بگویم بی‌حجاب از حال دل افسانه‌ای
\\
شمع گویم یا نگاری دلبری جان پروری
&&
محض روحی سروقدی کافری جانانه‌ای
\\
پیش تختش پیرمردی پای کوبان مست وار
&&
لیک او دریای علمی حاکمی فرزانه‌ای
\\
دامن دانش گرفته زیر دندان‌ها ولیک
&&
کلبتین عشق نامانده در او دندانه‌ای
\\
من ز نور پیر واله پیر در معشوق محو
&&
او چو آیینه یکی رو من دوسر چون شانه‌ای
\\
پیر گشتم در جمال و فر آن پیر لطیف
&&
من چو پروانه در او او را به من پروانه‌ای
\\
گفتم آخر ای به دانش اوستاد کائنات
&&
در هنر اقلیم‌هایی لطف کن کاشانه‌ای
\\
گفت گویم من تو را ای دوربین بسته چشم
&&
بشنو از من پند جانی محکمی پیرانه‌ای
\\
دانش و دانا حکیم و حکمت و فرهنگ ما
&&
غرقه بین تو در جمال گلرخی دردانه‌ای
\\
چون نگه کردم چه دیدم آفت جان و دلی
&&
ای مسلمانان ز رحمت یاریی یارانه‌ای
\\
این همه پوشیده گفتی آخر این را برگشا
&&
از حسودان غم مخور تو شرح ده مردانه‌ای
\\
شمس حق و دین تبریزی خداوندی کز او
&&
گشت این پس مانده اندر عشق او پیشانه‌ای
\\
\end{longtable}
\end{center}
