\begin{center}
\section*{غزل شماره ۹۲۸: هزار جان مقدس فدای روی تو باد}
\label{sec:0928}
\addcontentsline{toc}{section}{\nameref{sec:0928}}
\begin{longtable}{l p{0.5cm} r}
هزار جان مقدس فدای روی تو باد
&&
که در جهان چو تو خوبی کسی ندید و نزاد
\\
هزار رحمت دیگر نثار آن عاشق
&&
که او به دام هوای چو تو شهی افتاد
\\
ز صورت تو حکایت کنند یا ز صفت
&&
که هر یکی ز یکی خوبتر زهی بنیاد
\\
دلم هزار گره داشت همچو رشته سحر
&&
ز سحر چشم خوشت آن همه گره بگشاد
\\
بلندبین ز تو گشتست هر دو دیده عشق
&&
ببین تو قوت شاگرد و حکمت استاد
\\
نشسته‌ایم دل و عشق و کالبد پیشت
&&
یکی خراب و یکی مست وان دگر دلشاد
\\
به حکم تست بگریانی و بخندانی
&&
همه چو شاخ درختیم و عشق تو چون باد
\\
به باد عشق تو زردیم هم بدان سبزیم
&&
تو راست جمله ولایت تو راست جمله مراد
\\
کلوخ و سنگ چه داند بهار را چه اثر
&&
بهار را ز چمن پرس و سنبل و شمشاد
\\
درخت را ز برون سوی باد گرداند
&&
درخت دل را باد اندرونست یعنی یاد
\\
به زیر سایه زلفت دلم چه خوش خفته‌ست
&&
خراب و مست و لطیف و خوش و کش و آزاد
\\
چو غیرت تو دلم را ز خواب بجهانید
&&
خمار خیزد و فریاد دردهد فریاد
\\
ولی چو مست کنی مر مرا غلط گردم
&&
گمان برم که امیرم چرا شوم منقاد
\\
به وقت درد بگوییم کای تو و همه تو
&&
چو درد رفت حجابی میان ما بنهاد
\\
در آن زمان که کند عقل عاقبت بینی
&&
ندا ز عشق برآید که هرچ بادا باد
\\
\end{longtable}
\end{center}
