\begin{center}
\section*{غزل شماره ۲۴۵۹: عارف گوینده اگر تا سحر صبر کنی}
\label{sec:2459}
\addcontentsline{toc}{section}{\nameref{sec:2459}}
\begin{longtable}{l p{0.5cm} r}
عارف گوینده اگر تا سحر صبر کنی
&&
از جهت خسته دلان جان و نگهبان منی
\\
همچو علی در صف خود سر نبری از کف خود
&&
بولهب وسوسه را تا نکنی راه زنی
\\
راه زنان را بزنی تا که حقت نام نهد
&&
غازی من حاجی من گر چه به تن در وطنی
\\
ساقی جام ازلی مایه قند و عسلی
&&
بارگه جان و دلی گنجگه بوالحسنی
\\
جنبش پر ملکی مطلع بام فلکی
&&
جمع صفا را نمکی شمع خدا را لگنی
\\
باده دهی مست کنی جمله حریفان مرا
&&
عربده شان یاد دهی یا منشان درفکنی
\\
از یک سوراخ تو را مار دوباره نگزد
&&
گر نری و پاکدلی مؤمنی و مؤتمنی
\\
خامش باش ای دل من نام مرا هیچ مگو
&&
نام کسی گو که از او چون گل تر خوش دهنی
\\
\end{longtable}
\end{center}
