\begin{center}
\section*{غزل شماره ۲۳۷۰: این چه باد صرصر است از آسمان پویان شده}
\label{sec:2370}
\addcontentsline{toc}{section}{\nameref{sec:2370}}
\begin{longtable}{l p{0.5cm} r}
این چه باد صرصر است از آسمان پویان شده
&&
صد هزاران کشتی از وی مست و سرگردان شده
\\
مخلص کشتی ز باد و غرقه کشتی ز باد
&&
هم بدو زنده شده‌ست و هم بدو بی‌جان شده
\\
باد اندر امر یزدان چون نفس در امر تو
&&
ز امر تو دشنام گشته وز تو مدحت خوان شده
\\
بادها را مختلف از مروحه تقدیر دان
&&
از صبا معمور عالم با وبا ویران شده
\\
باد را یا رب نمودی مروحه پنهان مدار
&&
مروحه دیدن چراغ سینه پاکان شده
\\
هر که بیند او سبب باشد یقین صورت پرست
&&
و آنک بیند او مسبب نور معنی دان شده
\\
اهل صورت جان دهند از آرزوی شبه‌ای
&&
پیش اهل بحر معنی درها ارزان شده
\\
شد مقلد خاک مردان نقل‌ها ز ایشان کند
&&
و آن دگر خاموش کرده زیر زیر ایشان شده
\\
چشم بر ره داشت پوینده قراضه می‌بچید
&&
آن قراضه چین ره را بین کنون در کان شده
\\
همچو مادر بر بچه لرزیم بر ایمان خویش
&&
از چه لرزد آن ظریف سر به سر ایمان شده
\\
همچو ماهی می‌گدازی در غم سرلشکری
&&
بینمت چون آفتابی بی‌حشم سلطان شده
\\
چند گویی دود برهان است بر آتش خمش
&&
بینمت بی‌دود آتش گشته و برهان شده
\\
چند گشت و چند گردد بر سرت کیوان بگو
&&
بینمت همچون مسیحا بر سر کیوان شده
\\
ای نصیبه جو ز من که این بیار و آن بیار
&&
بینمت رسته از این و آن و آن و آن شده
\\
بس کن ای مست معربد ناطق بسیارگو
&&
بینمت خاموش گویان چون کفه میزان شده
\\
\end{longtable}
\end{center}
