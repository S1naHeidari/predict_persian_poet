\begin{center}
\section*{غزل شماره ۱۸۹: آمد بهار جان‌ها ای شاخ تر به رقص آ}
\label{sec:0189}
\addcontentsline{toc}{section}{\nameref{sec:0189}}
\begin{longtable}{l p{0.5cm} r}
آمد بهار ِ جان‌ها ای شاخ ِ تر به رقص آ
&&
چون یوسف اندر آمد، مصر و شکر! به رقص آ
\\
ای شاه ِ عشق‌پرور مانند ِ شیر ِ مادر
&&
ای شیر! جوش‌در رو. جان ِ پدر به رقص آ
\\
چوگان ِ زلف دیدی، چون گوی دررسیدی
&&
از پا و سر بریدی. بی‌پا و سر به رقص آ
\\
تیغی به دست، خونی. آمد مرا که: چونی؟
&&
گفتم بیا که خیر است! گفتا: نه! شر! به رقص آ
\\
از عشق، تاج‌داران در چرخ ِ او چو باران
&&
آن جا قبا چه باشد؟ ای خوش کمر به رقص آ
\\
ای مست ِ هست گشته! بر تو فنا نبشته.
&&
رقعه‌ی فنا رسیده. بهر ِ سفر به رقص آ
\\
در دست، جام ِ باده آمد بُت‌ام پیاده
&&
گر نیستی تو ماده، ز آن شاه ِ نر به رقص آ
\\
پایان ِ جنگ آمد. آواز ِ چنگ آمد
&&
یوسف زِ چاه آمد. ای بی‌هنر! به رقص آ
\\
تا چند وعده باشد؟ و این سَر به سجده باشد؟
&&
هجر اَم ببُرده باشد رنگ و اثر؟ به رقص آ
\\
کی باشد آن زمانی، گوید مرا فلانی:
&&
ک‌ای بی‌خبر! فنا شو! ای باخبر! به رقص آ
\\
طاووس ِ ما درآید و آن رنگ‌ها برآید
&&
با مرغ ِ جان سراید: بی‌بال و پر به رقص آ
\\
کور و کران ِ عالَم، دید از مسیح، مرهم
&&
گفته مسیح ِ مریم: ک‌ای کور و کر! به رقص آ
\\
مخدوم، شمس ِ دین است. تبریز رشک ِ چین است
&&
اندر بهار حسنش شاخ و شجر به رقص آ
\\
\end{longtable}
\end{center}
