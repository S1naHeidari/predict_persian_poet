\begin{center}
\section*{غزل شماره ۶۳: چه چیزست آنک عکس او حلاوت داد صورت را}
\label{sec:0063}
\addcontentsline{toc}{section}{\nameref{sec:0063}}
\begin{longtable}{l p{0.5cm} r}
چه چیزست آنک عکس او حلاوت داد صورت را
&&
چو آن پنهان شود گویی که دیوی زاد صورت را
\\
چو بر صورت زند یک دم ز عشق آید جهان برهم
&&
چو پنهان شد درآید غم نبینی شاد صورت را
\\
اگر آن خود همین جانست چرا بعضی گران جانست
&&
بسی جانی که چون آتش دهد بر باد صورت را
\\
وگر عقلست آن پرفن چرا عقلی بود دشمن
&&
که مکر عقل بد در تن کند بنیاد صورت را
\\
چه داند عقل کژخوانش مپرس از وی مرنجانش
&&
همان لطف و همان دانش کند استاد صورت را
\\
زهی لطف و زهی نوری زهی حاضر زهی دوری
&&
چنین پیدا و مستوری کند منقاد صورت را
\\
جهانی را کشان کرده بدن‌هاشان چو جان کرده
&&
برای امتحان کرده ز عشق استاد صورت را
\\
چو با تبریز گردیدم ز شمس الدین بپرسیدم
&&
از آن سری کز او دیدم همه ایجاد صورت را
\\
\end{longtable}
\end{center}
