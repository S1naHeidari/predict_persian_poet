\begin{center}
\section*{غزل شماره ۲۸۱۴: خنک آن دم که به رحمت سر عشاق بخاری}
\label{sec:2814}
\addcontentsline{toc}{section}{\nameref{sec:2814}}
\begin{longtable}{l p{0.5cm} r}
خنک آن دم که به رحمت سر عشاق بخاری
&&
خنک آن دم که برآید ز خزان باد بهاری
\\
خنک آن دم که بگویی که بیا عاشق مسکین
&&
که تو آشفته مایی سر اغیار نداری
\\
خنک آن دم که درآویزد در دامن لطفت
&&
تو بگویی که چه خواهی ز من ای مست نزاری
\\
خنک آن دم که صلا دردهد آن ساقی مجلس
&&
که کند بر کف ساقی قدح باده سواری
\\
شود اجزای تن ما خوش از آن باده باقی
&&
برهد این تن طامع ز غم مایده خواری
\\
خنک آن دم که ز مستان طلبد دوست عوارض
&&
بستاند گرو از ما بکش و خوب عذاری
\\
خنک آن دم که ز مستی سر زلف تو بشورد
&&
دل بیچاره بگیرد به هوس حلقه شماری
\\
خنک آن دم که بگوید به تو دل کشت ندارم
&&
تو بگویی که بروید پی تو آنچ بکاری
\\
خنک آن دم که شب هجر بگوید که شبت خوش
&&
خنک آن دم که سلامی کند آن نور بهاری
\\
خنک آن دم که برآید به هوا ابر عنایت
&&
تو از آن ابر به صحرا گهر لطف بباری
\\
خورد این خاک که تشنه‌تر از آن ریگ سیاه است
&&
به تمام آب حیات و نکند هیچ غباری
\\
دخل العشق علینا بکأوس و عقار
&&
ظهر السکر علینا لحبیب متوار
\\
سخنی موج همی‌زد که گهرها بفشاند
&&
خمشش باید کردن چو در اینش نگذاری
\\
\end{longtable}
\end{center}
