\begin{center}
\section*{غزل شماره ۲۱۷۵: در خشکی ما بنگر و آن پرده تر برگو}
\label{sec:2175}
\addcontentsline{toc}{section}{\nameref{sec:2175}}
\begin{longtable}{l p{0.5cm} r}
در خشکی ما بنگر وآن پرده تر برگو
&&
چشم تر ما را بین ای نور بصر برگو
\\
جمع شکران را بین در ما نگران را بین
&&
شیرین نظران را بین هین شرح شکر برگو
\\
امروز چنان مستی کز جوی جهان جستی
&&
امروز اگر خواهی آن چیز دگر برگو
\\
هر چند که استادی داد دو جهان دادی
&&
در دست کی افتادی زان طرفه خبر برگو
\\
از جای نجنبیده لیک از دل و از دیده
&&
بسیار بگردیده احوال سفر برگو
\\
در کشتی و دریایی خوش موج و مصفایی
&&
زیری گه و بالایی ای زیر و زبر برگو
\\
با صبر تویی محرم روسخت تویی در غم
&&
شمشیر زبان برکش وز صبر و سپر برگو
\\
مستی جماعت بین کرده ز قدح بالین
&&
یا رب بفزا آمین این قصه ز سر برگو
\\
بر هر که زد این برهان جان یابد و سیصد جان
&&
باور نکنی این را بر چوب و حجر برگو
\\
گفت ار سر او باشم رخسار تو بخراشم
&&
ای عارف این را هم با او به سحر برگو
\\
آمد دگری از ده هین دیگ دگر برنه
&&
گر تاج گرو کردی از رهن کمر برگو
\\
گر رافضیی باشد از داد علی در ده
&&
ور زآنک بود سنی از عدل عمر برگو
\\
موری چه قدر گوید از تخت سلیمانی
&&
بگشا لب و شرحش کن اسباب ظفر برگو
\\
\end{longtable}
\end{center}
