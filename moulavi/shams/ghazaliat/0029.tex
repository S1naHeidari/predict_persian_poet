\begin{center}
\section*{غزل شماره ۲۹: ای از ورای پرده‌ها تاب تو تابستان ما}
\label{sec:0029}
\addcontentsline{toc}{section}{\nameref{sec:0029}}
\begin{longtable}{l p{0.5cm} r}
ای از ورای پرده‌ها تاب تو تابستان ما
&&
ما را چو تابستان ببر دل گرم تا بستان ما
\\
ای چشم جان را توتیا آخر کجا رفتی بیا
&&
تا آب رحمت برزند از صحن آتشدان ما
\\
تا سبزه گردد شوره‌ها تا روضه گردد گورها
&&
انگور گردد غوره‌ها تا پخته گردد نان ما
\\
ای آفتاب جان و دل ای آفتاب از تو خجل
&&
آخر ببین کاین آب و گل چون بست گرد جان ما
\\
شد خارها گلزارها از عشق رویت بارها
&&
تا صد هزار اقرارها افکند در ایمان ما
\\
ای صورت عشق ابد خوش رو نمودی در جسد
&&
تا ره بری سوی احد جان را از این زندان ما
\\
در دود غم بگشا طرب روزی نما از عین شب
&&
روزی غریب و بوالعجب ای صبح نورافشان ما
\\
گوهر کنی خرمهره را زهره بدری زهره را
&&
سلطان کنی بی‌بهره را شاباش ای سلطان ما
\\
کو دیده‌ها درخورد تو تا دررسد در گرد تو
&&
کو گوش هوش آورد تو تا بشنود برهان ما
\\
چون دل شود احسان شمر در شکر آن شاخ شکر
&&
نعره برآرد چاشنی از بیخ هر دندان ما
\\
آمد ز جان بانگ دهل تا جزوها آید به کل
&&
ریحان به ریحان گل به گل از حبس خارستان ما
\\
\end{longtable}
\end{center}
