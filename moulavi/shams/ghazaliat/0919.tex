\begin{center}
\section*{غزل شماره ۹۱۹: ببرد خواب مرا عشق و عشق خواب برد}
\label{sec:0919}
\addcontentsline{toc}{section}{\nameref{sec:0919}}
\begin{longtable}{l p{0.5cm} r}
ببرد خواب مرا عشق و عشق خواب برد
&&
که عشق جان و خرد را به نیم جو نخرد
\\
که عشق شیر سیاه‌ست تشنه و خون خوار
&&
به غیر خون دل عاشقان همی‌نچرد
\\
به مهر بر تو بچفسد به سوی دام آرد
&&
چو درفتادی از آن پس ز دور می‌نگرد
\\
امیر دست درازست و شحنه بی‌باک
&&
شکنجه می‌کند و بی‌گناه می‌فشرد
\\
هر آنک در کفش آید چو ابر می‌گرید
&&
هر آنک دور شد از وی چو برف می‌فسرد
\\
هزار جام به هر لحظه خرد درشکند
&&
هزار جامه به یک دم بدوزد و بدرد
\\
هزار چشم بگریاند و فروخندد
&&
هزار کس بکشد زار زار و یک شمرد
\\
به کوه قاف اگر چه که خوش پرد سیمرغ
&&
چو دام عشق ببیند فتد دگر نپرد
\\
ز بند او نرهد کس به شید یا به جنون
&&
ز دام او نرهد هیچ عاقلی به خرد
\\
مخبط‌ست سخن‌های من از او گر نی
&&
نمودمی به تو آن راه‌ها که می‌سپرد
\\
نمودمی به تو کو شیر را چه سان گیرد
&&
نمودمی که چگونه شکار را شکرد
\\
\end{longtable}
\end{center}
