\begin{center}
\section*{غزل شماره ۱۱۷۱: در بگشا کمد خامی دگر}
\label{sec:1171}
\addcontentsline{toc}{section}{\nameref{sec:1171}}
\begin{longtable}{l p{0.5cm} r}
در بگشا کآمد خامی دگر
&&
پیشکشی کن دو سه جامی دگر
\\
هین که رسیدیم به نزدیک ده
&&
همره ما شو دو سه گامی دگر
\\
هین هله چونی تو ز راه دراز
&&
هر قدمی غصه و دامی دگر
\\
غصه کجا دارد کان عسل
&&
ای که تو را سیصد نامی دگر
\\
بسته بدی تو در و بام سرا
&&
آمدت آن حکم ز بامی دگر
\\
گر به سنام سر گردون روی
&&
بر تو قضا راست سنامی دگر
\\
ای ز تو صد کام دلم یافته
&&
می‌طلبد دل ز تو کامی دگر
\\
ای رخ و رخسار تو رومی دگر
&&
ای سر زلفین تو شامی دگر
\\
سوی چنان روم و چنان شام رو
&&
تا ببری دولت را می دگر
\\
لطف تو عام آمد چون آفتاب
&&
گیر مرا نیز تو عامی دگر
\\
هر سحری سر نهدت آفتاب
&&
گوید بپذیر غلامی دگر
\\
بر تو و برگرد تو هر کس که هست
&&
دم به دم از عرش سلامی دگر
\\
بی‌سخنی ره رو راه تو را
&&
در غم و شادیست پیامی دگر
\\
این غم و شادی چو زمام دلند
&&
ناقه حق راست زمانی دگر
\\
شاد زمانی که ببندم دهن
&&
بشنوم از روح کلامی دگر
\\
رخت از این سوی بدان سو کشم
&&
بنگرم آن سوی نظامی دگر
\\
عیش جهان گردد بر من حرام
&&
بینم من بیت حرامی دگر
\\
طرفه که چون خنب تنم بشکند
&&
یابد این باده قوامی دگر
\\
توبه مکن زین که شدم ناتمام
&&
بعد شدن هست تمامی دگر
\\
بس کنم ای دوست تو خود گفته گیر
&&
یک دو سه میم و دو سه لامی دگر
\\
\end{longtable}
\end{center}
