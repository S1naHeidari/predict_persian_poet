\begin{center}
\section*{غزل شماره ۴۰۵: به خدا کت نگذارم که روی راه سلامت}
\label{sec:0405}
\addcontentsline{toc}{section}{\nameref{sec:0405}}
\begin{longtable}{l p{0.5cm} r}
به خدا کت نگذارم که روی راه سلامت
&&
که سر و پا و سلامت نبود روز قیامت
\\
حشم عشق درآمد ربض شهر برآمد
&&
هله ای یار قلندر بشنو طبل ملامت
\\
دل و جان فانی لا کن تن خود همچو قبا کن
&&
نه اثر گو نه خبر گو نه نشانی نه علامت
\\
چو من از خویش برستم ره اندیشه ببستم
&&
هله ای سرده مستم برهانم به تمامت
\\
هله برجه هله برجه قدمی بر سر خود نه
&&
هله برپر هله برپر چو من از شکر و غرامت
\\
ببر ای عشق چو موسی سر فرعون تکبر
&&
هله فرعون به پیش آ که گرفتم در و بامت
\\
چو من از غیب رسیدم سپه غیب کشیدم
&&
برو ای ظالم سرکش که فتادی ز زعامت
\\
هله پالیز تو باقی سر خر عالم فانی
&&
همه دیدار کریمست در این عشق کرامت
\\
نکند رحمت مطلق به بلا جان تو ویران
&&
نکند والده ما را ز پی کینه حجامت
\\
نبود جان و دلم را ز تو سیری و ملولی
&&
نبود هیچ کسی را ز دل و دیده سمت
\\
بجز از عشق مجرد به هر آن نقش که رفتم
&&
بنه ارزید خوشی‌هاش به تلخی ندامت
\\
هله تا یاوه نگردی چو در این حوض رسیدی
&&
که تکش آب حیاتست و لبش جای اقامت
\\
چو در این حوض درافتی همه خویش بدو ده
&&
به مزن دستک و پایک تو به چستی و شهامت
\\
همه تسلیم و خمش کن نه امامی تو ز جمعی
&&
نرسد هیچ کسی را به جز این عشق امامت
\\
\end{longtable}
\end{center}
