\begin{center}
\section*{غزل شماره ۲۳۳۳: این کیست چنین مست ز خمار رسیده}
\label{sec:2333}
\addcontentsline{toc}{section}{\nameref{sec:2333}}
\begin{longtable}{l p{0.5cm} r}
این کیست چنین مست ز خمار رسیده
&&
یا یار بود یا ز بر یار رسیده
\\
یا شاهد جان باشد روبند گشاده
&&
یا یوسف مصری است ز بازار رسیده
\\
یا زهره و ماه است درآمیخته با هم
&&
یا سرو روان است ز گلزار رسیده
\\
یا چشمه خضر است روان گشته بدین سو
&&
یا ترک خوش ماست ز بلغار رسیده
\\
یا برق کله گوشه خاقان شکاری است
&&
اندر طلب آهوی تاتار رسیده
\\
یا ساقی دریادل ما بزم نهاده‌ست
&&
یا نقل و شکرهاست به قنطار رسیده
\\
یا صورت غیب است که جان همه جان‌هاست
&&
یا مشعله از عالم انوار رسیده
\\
شاه پریان بین ز سلیمان پیمبر
&&
اندر طلب هدهد طیار رسیده
\\
خوبان جهان از پی او جیب دریده
&&
قاضی خرد بی‌دل و دستار رسیده
\\
از هیبت خون ریزی آن چشم چو مریخ
&&
مریخ ز گردون پی زنهار رسیده
\\
وز بهر دیت دادن هر زنده که او کشت
&&
همیان زر آورده به ایثار رسیده
\\
اول دیت خون تو جامی است به دستش
&&
درکش که رحیق است ز اسرار رسیده
\\
خاموش کن ای خاسر انسان لفی خسر
&&
از گلشن دیدار به گفتار رسیده
\\
\end{longtable}
\end{center}
