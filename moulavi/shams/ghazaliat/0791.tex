\begin{center}
\section*{غزل شماره ۷۹۱: این کبوتربچه هم عزم هوا کرد و پرید}
\label{sec:0791}
\addcontentsline{toc}{section}{\nameref{sec:0791}}
\begin{longtable}{l p{0.5cm} r}
این کبوتربچه هم عزم هوا کرد و پرید
&&
چون صفیری و ندایی ز سوی غیب شنید
\\
آن مراد همه عالم چه فرستاد رسول
&&
که بیا جانب ما چون نپرد جان مرید
\\
بپرد جانب بالا چو چنان بال بیافت
&&
بدرد جامه تن را چو چنان نامه رسید
\\
چه کمندست که پر می‌کشد این جان‌ها را
&&
چه ره است آن ره پنهان که از آن راه کشید
\\
رحمتش نامه فرستاد که این جا بازآ
&&
که در آن تنگ قفس جان تو بسیار طپید
\\
لیک در خانه بی‌در تو چو مرغی بی‌پر
&&
این کند مرغ هوا چونک به چستی افتید
\\
بی قراریش گشاید در رحمت آخر
&&
بر در و سقف همی‌کوب پر اینست کلید
\\
تا نخوانیم ندانی تو ره واگشتن
&&
که ره از دعوت ما گردد بر عقل بدید
\\
هر چه بالا رود ار کهنه بود نو گردد
&&
هر نوی کید این جا شود از دهر قدید
\\
هین خرامان رو در غیب سوی پس منگر
&&
فی امان الله کان جا همه سودست و مزید
\\
هله خاموش برو جانب ساقی وجود
&&
که می پاک ویت داد در این جام پلید
\\
\end{longtable}
\end{center}
