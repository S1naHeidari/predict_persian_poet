\begin{center}
\section*{غزل شماره ۴۶: دی بنواخت یار من بنده غم رسیده را}
\label{sec:0046}
\addcontentsline{toc}{section}{\nameref{sec:0046}}
\begin{longtable}{l p{0.5cm} r}
دی بنواخت یار من بنده غم رسیده را
&&
داد ز خویش چاشنی جان ستم چشیده را
\\
هوش فزود هوش را حلقه نمود گوش را
&&
جوش نمود نوش را نور فزود دیده را
\\
گفت که ای نزار من خسته و ترسگار من
&&
من نفروشم از کرم بنده خودخریده را
\\
بین که چه داد می‌کند بین چه گشاد می‌کند
&&
یوسف یاد می‌کند عاشق کف بریده را
\\
داشت مرا چو جان خود رفت ز من گمان بد
&&
بر کتفم نهاد او خلعت نورسیده را
\\
عاجز و بی‌کسم مبین اشک چو اطلسم مبین
&&
در تن من کشیده بین اطلس زرکشیده را
\\
هر که بود در این طلب بس عجبست و بوالعجب
&&
صد طربست در طرب جان ز خود رهیده را
\\
چاشنی جنون او خوشتر یا فسون او
&&
چونک نهفته لب گزد خسته غم گزیده را
\\
وعده دهد به یار خود گل دهد از کنار خود
&&
پر کند از خمار خود دیده خون چکیده را
\\
کحل نظر در او نهد دست کرم بر او زند
&&
سینه بسوزد از حسد این فلک خمیده را
\\
جام می الست خود خویش دهد به سمت خود
&&
طبل زند به دست خود باز دل پریده را
\\
بهر خدای را خمش خوی سکوت را مکش
&&
چون که عصیده می‌رسد کوته کن قصیده را
\\
مفتعلن مفاعلن مفتعلن مفاعلن
&&
در مگشا و کم نما گلشن نورسیده را
\\
\end{longtable}
\end{center}
