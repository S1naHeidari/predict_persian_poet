\begin{center}
\section*{غزل شماره ۱۱۲۲: اندیشه را رها کن اندر دلش مگیر}
\label{sec:1122}
\addcontentsline{toc}{section}{\nameref{sec:1122}}
\begin{longtable}{l p{0.5cm} r}
اندیشه را رها کن اندر دلش مگیر
&&
زیرا برهنه‌ای تو و اندیشه زمهریر
\\
اندیشه می‌کنی که رهی از زحیر و رنج
&&
اندیشه کردن آمد سرچشمه زحیر
\\
ز اندیشه‌ها برون دان بازار صنع را
&&
آثار را نظاره کن ای سخره اثیر
\\
آن کوی را نگر که پرد زو مصورات
&&
وان جوی را کز او شد گردنده چرخ پیر
\\
گلگونه‌ای کز اوست رخ دلبران چو گل
&&
سرفتنه‌ای کز اوست رخ عاشقان زریر
\\
خوش از عدم همی‌پرد این صد هزار مرغ
&&
از یک کمان همی‌جهد این صد هزار تیر
\\
بی‌چون و بی‌چگونه برون از رسوم و فهم
&&
بی‌دست می‌سریشد در غیب صد خمیر
\\
بی‌آتشی تنور دل و معده‌ها فروخت
&&
نان بر دکان نهاده و خباز ما ستیر
\\
از لوح خاک ساده دهد صد هزار نقش
&&
وز جوش خون ماده دهد صد هزار شیر
\\
شییء اللهی بگفتی و آمد ز چرخ بانگ
&&
زنبیل برگشا که عطا آمد ای فقیر
\\
زفت آمد آن نواله و زنبیل را درید
&&
از مطبخ خدای نیاید صله حقیر
\\
آن کس که من و سلوی بفرستد از هوا
&&
و آنک از شکاف کوه برون می‌کشد بعیر
\\
وان کو ز آب نطفه برآرد تهمتنی
&&
وان کو ز خواب خفته گشاید ره مطیر
\\
اندر عدم نماید هر لحظه صورتی
&&
تا این خیالیان بشتابند در مسیر
\\
فرمان کنم چو گفت خمش من خمش کنم
&&
خود شرح این بگوید یک روز آن امیر
\\
\end{longtable}
\end{center}
