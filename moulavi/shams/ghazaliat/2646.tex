\begin{center}
\section*{غزل شماره ۲۶۴۶: چه باشد گر چو عقل و جان نخسبی}
\label{sec:2646}
\addcontentsline{toc}{section}{\nameref{sec:2646}}
\begin{longtable}{l p{0.5cm} r}
چه باشد گر چو عقل و جان نخسبی
&&
برآری کار محتاجان نخسبی
\\
تو نور خاطر این شب روانی
&&
برای خاطر ایشان نخسبی
\\
شبی بر گرد محبوسان گردون
&&
بگردی ای مه تابان نخسبی
\\
جهان کشتی و تو نوح زمانی
&&
نگاهش داری از طوفان نخسبی
\\
شب قدری که دادی وعده آن روز
&&
دراندیشی از آن پیمان نخسبی
\\
مخسب ای جان که خفتن آن ندارد
&&
چه باشد چون تو داری آن نخسبی
\\
تویی شه پیل و پیش آهنگ پیلان
&&
چو کردی یاد هندستان نخسبی
\\
تو نپسندی ز داد و رحمت خویش
&&
که بستان را کنی زندان نخسبی
\\
اگر خسبی نخسبد جز که چشمت
&&
تویی آن نور جاویدان نخسبی
\\
خمش کردم نگویم تا تو گویی
&&
سخن گویان سخن گویان نخسبی
\\
چو روی شمس تبریزی بدیدی
&&
سزد کز عشق آن سلطان نخسبی
\\
\end{longtable}
\end{center}
