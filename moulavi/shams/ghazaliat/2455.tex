\begin{center}
\section*{غزل شماره ۲۴۵۵: برگذری درنگری جز دل خوبان نبری}
\label{sec:2455}
\addcontentsline{toc}{section}{\nameref{sec:2455}}
\begin{longtable}{l p{0.5cm} r}
برگذری درنگری جز دل خوبان نبری
&&
سر مکش ای دل که از او هر چه کنی جان نبری
\\
تا نشوی خاک درش در نگشاید به رضا
&&
تا نکشی خار غمش گل ز گلستان نبری
\\
تا نکنی کوه بسی دست به لعلی نرسد
&&
تا سوی دریا نروی گوهر و مرجان نبری
\\
سر ننهد چرخ تو را تا که تو بی‌سر نشوی
&&
کس نخرد نقد تو را تا سوی میزان نبری
\\
تا نشوی مست خدا غم نشود از تو جدا
&&
تا صفت گرگ دری یوسف کنعان نبری
\\
تا تو ایازی نکنی کی همه محمود شوی
&&
تا تو ز دیوی نرهی ملک سلیمان نبری
\\
نعمت تن خام کند محنت تن رام کند
&&
محنت دین تا نکشی دولت ایمان نبری
\\
خیره میا خیره مرو جانب بازار جهان
&&
ز آنک در این بیع و شری این ندهی آن نبری
\\
خاک که خاکی نهلد سوسن و نسرین نشود
&&
تا نکنی دلق کهن خلعت سلطان نبری
\\
آه گدارو شده‌ای خاطر تو خوش نشود
&&
تا نکنی کافریی مال مسلمان نبری
\\
هیچ نبرده‌ست کسی مهره ز انبان جهان
&&
رنجه مشو ز آنک تو هم مهره ز انبان نبری
\\
مهره ز انبان نبرم گوهر ایمان ببرم
&&
گو تو به جان بخل کنی جان بر جانان نبری
\\
ای کشش عشق خدا می‌ننشیند کرمت
&&
دست نداری ز کهان تا دل از ایشان نبری
\\
هین بکشان هین بکشان دامن ما را به خوشان
&&
ز آنک دلی که تو بری راه پریشان نبری
\\
راست کنی وعده خود دست نداری ز کشش
&&
تا همه را رقص کنان جانب میدان نبری
\\
هیچ مگو ای لب من تا دل من باز شود
&&
ز آنک تو تا سنگ دلی لعل بدخشان نبری
\\
گر چه که صد شرط کنی بی‌همه شرطی بدهی
&&
ز آنک تو بس بی‌طمعی زر به حرمدان نبری
\\
\end{longtable}
\end{center}
