\begin{center}
\section*{غزل شماره ۳۰۲۹: آه که دلم برد غمزه‌های نگاری}
\label{sec:3029}
\addcontentsline{toc}{section}{\nameref{sec:3029}}
\begin{longtable}{l p{0.5cm} r}
آه که دلم برد غمزه‌های نگاری
&&
شیر شگرف آمد و ضعیف شکاری
\\
هیچ دلی چون نبود خالی از اندوه
&&
درد و غم چون تو یار و دلبر باری
\\
از پی این عشق اشک‌هاست روانه
&&
خوب شهی آمد و لطیف نثاری
\\
چشم پیاپی چو ابر آب فشاند
&&
تا ننشیند بر آن نیاز غباری
\\
کان شکر آن لبست باد بقایش
&&
تا که نماند حزین و غوره فشاری
\\
نک شب قدرست و بدر کرد عنایت
&&
بر دل هر شب روی ستاره شماری
\\
بی مه او جان چو چرخ زیر و زبر بود
&&
ماهی بی‌آب را کی دید قراری
\\
خود تو چو عقلی و این جهان همه چون تن
&&
از تن بی‌عقل کی بیاید کاری
\\
خلعت نو پوش بر زمین و زمانه
&&
خلعت گل یافت از جناب تو خاری
\\
گر نبدی خوی دوست روح فشانی
&&
خود نبدی عاشقی و روح سپاری
\\
خرقه بده در قمارخانه عالم
&&
خوب حریفی و سودناک قماری
\\
بهر کنارش همی کنار گشایم
&&
هیچ کس آن بحر را ندید کناری
\\
تن بزنم تا بگوید آن مه خوش رو
&&
آنک ز حلمش بیافت کوه وقاری
\\
\end{longtable}
\end{center}
