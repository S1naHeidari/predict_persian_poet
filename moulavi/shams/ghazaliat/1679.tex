\begin{center}
\section*{غزل شماره ۱۶۷۹: من اگر مستم اگر هشیارم}
\label{sec:1679}
\addcontentsline{toc}{section}{\nameref{sec:1679}}
\begin{longtable}{l p{0.5cm} r}
من اگر مستم اگر هشیارم
&&
بنده چشم خوش آن یارم
\\
بی‌خیال رخ آن جان و جهان
&&
از خود و جان و جهان بیزارم
\\
بنده صورت آنم که از او
&&
روز و شب در گل و در گلزارم
\\
این چنین آینه‌ای می بینم
&&
چشم از این آینه چون بردارم
\\
دم فروبسته‌ام و تن زده‌ام
&&
دم مده تا علالا برنارم
\\
بت من گفت منم جان بتان
&&
گفتم این است بتا اقرارم
\\
گفت اگر در سر تو شور من است
&&
از تو من یک سر مو نگذارم
\\
منم آن شمع که در آتش خود
&&
هر چه پروانه بود بسپارم
\\
گفتمش هر چه بسوزی تو ز من
&&
دود عشق تو بود آثارم
\\
راست کن لاف مرا با دیده
&&
جز چنان راست نیاید کارم
\\
من ز پرگار شدم وین عجب است
&&
کاندر این دایره چون پرگارم
\\
ساقی آمد که حریفانه بده
&&
گفتم اینک به گرو دستارم
\\
غلطم سر بستان لیک دمی
&&
مددم ده قدری هشیارم
\\
آن جهان پنهان را بنما
&&
کاین جهان را به عدم انگارم
\\
\end{longtable}
\end{center}
