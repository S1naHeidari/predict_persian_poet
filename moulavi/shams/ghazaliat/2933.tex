\begin{center}
\section*{غزل شماره ۲۹۳۳: ای آنک امام عشقی تکبیر کن که مستی}
\label{sec:2933}
\addcontentsline{toc}{section}{\nameref{sec:2933}}
\begin{longtable}{l p{0.5cm} r}
ای آنک امام عشقی تکبیر کن که مستی
&&
دو دست را برافشان بیزار شو ز هستی
\\
موقوف وقت بودی تعجیل می‌نمودی
&&
وقت نماز آمد برجه چرا نشستی
\\
بر بوی قبله حق صد قبله می‌تراشی
&&
بر بوی عشق آن بت صد بت همی‌پرستی
\\
بالاترک پر ای جان ای جان بنده فرمان
&&
که مه بود به بالا سایه بود به پستی
\\
همچون گدای هر در بر هر دری مزن سر
&&
حلقه در فلک زن زیرا درازدستی
\\
سغراق آسمانت چون کرد آن چنانت
&&
بیگانه شو ز عالم کز خویش هم برستی
\\
می‌گویمت که چونی هرگز کسی بگوید
&&
با جان بی‌چگونه چونی چگونه استی
\\
امشب خراب و مستی فردا شود ببینی
&&
چه خیک‌ها دریدی چه شیشه‌ها شکستی
\\
هر شیشه که شکستم بر تو توکلستم
&&
که صد هزار گونه اشکسته را تو بستی
\\
ای نقش بند پنهان کاندر درونه ای جان
&&
داری هزار صورت جز ماه و جز مهستی
\\
صد حلق را گشودی گر حلقه‌ای ربودی
&&
صد جان و دل بدادی گر سینه‌ای بخستی
\\
دیوانه گشته‌ام من هر چه از جنون بگویم
&&
زودتر بلی بلی گو گر محرم الستی
\\
\end{longtable}
\end{center}
