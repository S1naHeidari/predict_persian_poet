\begin{center}
\section*{غزل شماره ۷۰۴: آن جا که چو تو نگار باشد}
\label{sec:0704}
\addcontentsline{toc}{section}{\nameref{sec:0704}}
\begin{longtable}{l p{0.5cm} r}
آن جا که چو تو نگار باشد
&&
سالوس و حفاظ عار باشد
\\
سالوس و حیل کنار گیرد
&&
چون رحمت بی‌کنار باشد
\\
بوسی به دغا ربودم از تو
&&
ای دوست دغا سه بار باشد
\\
امروز وفا کن آن سوم را
&&
امروز یکی هزار باشد
\\
من جوی و تو آب و بوسه آب
&&
هم بر لب جویبار باشد
\\
از بوسه آب بر لب جوی
&&
اشکوفه و سبزه زار باشد
\\
از سبزه چه کم شود که سبزه
&&
در دیده خیره خار باشد
\\
موسی ز عصا چرا گریزد
&&
گر بر فرعون مار باشد
\\
بر فرعونان که نیل خون گشت
&&
بر مؤمن خوشگوار باشد
\\
هرگز نرمد خلیل ز آتش
&&
گر بر نمرود نار باشد
\\
یعقوب کجا رمد ز یوسف
&&
گر بر پسرانش بار باشد
\\
آن باد بهار جان باغست
&&
بر شوره اگر غبار باشد
\\
زان باغ درخت برگ یابد
&&
اشکوفه بر او سوار باشد
\\
احمد چو تو راست پس ز بوجهل
&&
عشقا سزدت که عار باشد
\\
این را بر دست و آن بدین مات
&&
کار دنیا قمار باشد
\\
آن کس که ز بخت خود گریزد
&&
بگریخته شرمسار باشد
\\
هین دام منه به صید خرگوش
&&
تا شیر تو را شکار باشد
\\
ای دل ز عبیر عشق کم گوی
&&
خود بو برد آن که یار باشد
\\
\end{longtable}
\end{center}
