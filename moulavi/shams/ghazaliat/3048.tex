\begin{center}
\section*{غزل شماره ۳۰۴۸: تو آسمان منی من زمین به حیرانی}
\label{sec:3048}
\addcontentsline{toc}{section}{\nameref{sec:3048}}
\begin{longtable}{l p{0.5cm} r}
تو آسمان منی من زمین به حیرانی
&&
که دم به دم ز دل من چه چیز رویانی
\\
زمین خشک لبم من ببار آب کرم
&&
زمین ز آب تو باید گل و گلستانی
\\
زمین چه داند کاندر دلش چه کاشته‌ای
&&
ز توست حامله و حمل او تو می‌دانی
\\
ز توست حامله هر ذره‌ای به سر دگر
&&
به درد حامله را مدتی بپیچانی
\\
چه‌هاست در شکم این جهان پیچاپیچ
&&
کز او بزاید اناالحق و بانگ سبحانی
\\
گهی بنالد و ناقه بزاید از شکمش
&&
عصا بیفتد و گیرد طریق ثعبانی
\\
رسول گفت چو اشتر شناس مؤمن را
&&
همیشه مست خدا کش کند شتربانی
\\
گهیش داغ کند گه نهد علف پیشش
&&
گهیش بندد زانو به بند عقلانی
\\
گهی گشاید زانوش بهر رقص جعل
&&
که تا مهار به درد کند پریشانی
\\
چمن نگر که نمی‌گنجد از طرب در پوست
&&
که نقش چند بدو داد باغ روحانی
\\
ببین تو قوت تفهیم نفس کلی را
&&
که خاک کودن از او شد مصور جانی
\\
چو نفس کل همه کلی حجاب و روپوشست
&&
ز آفتاب جلالت که نیستش ثانی
\\
از آفتاب قدیمی که از غروب بری است
&&
که نور روش نه دلوی بود نه میزانی
\\
یکان یکان بنماید هر آنچ کاشت خموش
&&
که حامله‌ست صدف‌ها ز در ربانی
\\
\end{longtable}
\end{center}
