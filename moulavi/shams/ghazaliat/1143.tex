\begin{center}
\section*{غزل شماره ۱۱۴۳: تو شاخ خشک چرایی به روی یار نگر}
\label{sec:1143}
\addcontentsline{toc}{section}{\nameref{sec:1143}}
\begin{longtable}{l p{0.5cm} r}
تو شاخ خشک چرایی به روی یار نگر
&&
تو برگ زرد چرایی به نوبهار نگر
\\
درآ به حلقه رندان که مصلحت اینست
&&
شراب و شاهد و ساقی بی‌شمار نگر
\\
بدانک عشق جهانی است بی‌قرار در او
&&
هزار عاشق بی‌جان و بی‌قرار نگر
\\
چو دررسی تو بدان شه که نام او نبرم
&&
به حق شاهی آن شه که شاهوار نگر
\\
چو دیده سرمه کشی باز رو از این سو کن
&&
بدین جهان پر از دود و پرغبار نگر
\\
هزار دود مرکب که چیست این فلکست
&&
غبار رنگ برآرد که سبزه زار نگر
\\
نگه مکن تو به خورشید چونک درتابد
&&
به گاه شام ورا زرد و شرمسار نگر
\\
چو ماه نیز به دریوزه پر کند زنبیل
&&
ز بعد پانزده روزش تو خوار و زار نگر
\\
بیا به بحر ملاحت به سوی کان وصال
&&
بدان دو غمزه مخمور یار غار نگر
\\
چو روح قدس ببوسید نعل مرکب او
&&
ز نعل نعره برآمد که حال و کار نگر
\\
اگر نه عفو کند حلم شمس تبریزی
&&
تو روح را ز چنین یار شرمسار نگر
\\
\end{longtable}
\end{center}
