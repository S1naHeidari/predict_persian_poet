\begin{center}
\section*{غزل شماره ۱۹۵۹: روی او فتوی دهد کز کعبه بر بتخانه زن}
\label{sec:1959}
\addcontentsline{toc}{section}{\nameref{sec:1959}}
\begin{longtable}{l p{0.5cm} r}
روی او فتوی دهد کز کعبه بر بتخانه زن
&&
زلف او دعوی کند کاینک رسن بازی رسن
\\
عقل گوید گوهرم گوهر شکستن شرط نیست
&&
عشق گوید سنگ ما بستان و بر گوهر بزن
\\
سنگ ما گوهر شکست و حیف هم بر سنگ ماست
&&
حیف هم بر روح باشد گر شدش قربان بدن
\\
این نه بس دل را که دلبر دست در خونش کند
&&
این نه بس بت را که باشد چون خلیلش بت شکن
\\
هر که را جست او به رحمت وارهید از جست و جو
&&
هر که را گفت آن مایی وارهید از ما و من
\\
آن لبی کانگشت خود لیسید روزی زان عسل
&&
وصف آن لب را چه گویم کان نگنجد در دهن
\\
هر که صحرایی بود ایمن بود از زلزله
&&
هر که دریایی بود کی غم خورد از جامه کن
\\
کی سلیمان را زیان شد گر شد او ماهی فروش
&&
اهرمن گر ملک بستد اهرمن بد اهرمن
\\
گر بشد انگشتری انگشت او انگشتری است
&&
پرده بود انگشتری کای چشم بد بر وی مزن
\\
چشم بد خود را خورد خود ماه ما زان فارغ است
&&
شمع کی بدنام شد گر نور او بستد لگن
\\
\end{longtable}
\end{center}
