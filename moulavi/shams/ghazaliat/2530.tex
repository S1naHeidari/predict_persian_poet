\begin{center}
\section*{غزل شماره ۲۵۳۰: دلم همچون قلم آمد در انگشتان دلداری}
\label{sec:2530}
\addcontentsline{toc}{section}{\nameref{sec:2530}}
\begin{longtable}{l p{0.5cm} r}
دلم همچون قلم آمد در انگشتان دلداری
&&
که امشب می‌نویسد زی نویسد باز فردا ری
\\
قلم را هم تراشد او رقاع و نسخ و غیر آن
&&
قلم گوید که تسلیمم تو دانی من کیم باری
\\
گهی رویش سیه دارد گهی در موی خود مالد
&&
گه او را سرنگون دارد گهی سازد بدو کاری
\\
به یک رقعه جهانی را قلم بکشد کند بی‌سر
&&
به یک رقعه قرانی را رهاند از بلا آری
\\
کر و فر قلم باشد به قدر حرمت کاتب
&&
اگر در دست سلطانی اگر در کف سالاری
\\
سرش را می‌شکافد او برای آنچ او داند
&&
که جالینوس به داند صلاح حال بیماری
\\
نیارد آن قلم گفتن به عقل خویش تحسینی
&&
نداند آن قلم کردن به طبع خویش انکاری
\\
اگر او را قلم خوانم و اگر او را علم خوانم
&&
در او هوش است و بی‌هوشی زهی بی‌هوش هشیاری
\\
نگنجد در خرد وصفش که او را جمع ضدین است
&&
چه بی‌ترکیب ترکیبی عجب مجبور مختاری
\\
\end{longtable}
\end{center}
