\begin{center}
\section*{غزل شماره ۱۲۱۵: ای مست ماه روی تو استاره و گردون خوش}
\label{sec:1215}
\addcontentsline{toc}{section}{\nameref{sec:1215}}
\begin{longtable}{l p{0.5cm} r}
ای مست ماه روی تو استاره و گردون خوش
&&
رویت خوش و مویت خوش و آن دیگرت بیرون خوش
\\
هرگز ندیدست آسمان هرگز نبوده در جهان
&&
مانند تو لیلی جان مانند من مجنون خوش
\\
باور کند خود عاقلی در ظلمت آب و گلی
&&
مانند تو موسی دلی مانند من هارون خوش
\\
ای قطب این هفت آسیا هم کان زر هم کیمیا
&&
ای عیسی دوران بیا بر ما بخوان افسون خوش
\\
چون گوهری ناسفته‌ام فارغ ز خام و پخته‌ام
&&
در سایه‌ات خوش خفته‌ام سرمست از آن افیون خوش
\\
از نغمه تو ذره‌ها گر رقص آرد چه عجب
&&
نک طور موسی از وله رقصان در آن هامون خوش
\\
ای دل برای دلخوشی زر و هنر چون می‌کشی
&&
دیدی تو از زر و هنر بی‌خسف یک قارون خوش
\\
باشد به صورت خوش نما راه خوشی بسته شده
&&
چون زهر مار کوهیی بنهفته در معجون خوش
\\
یا همچو گور کافران پرمحنت و زخم گران
&&
پیچیده بیرون گور را در اطلس و اکسون خوش
\\
زان گوش همچون جیم تو زان چشم همچو صاد تو
&&
زان قامت همچون الف زان ابروی چون نون خوش
\\
شاگرد لوح جان شدم زین حرف‌ها خط خوان شدم
&&
کشتی و کشتی بان شدم اندر چنین جیحون خوش
\\
ایوان کجا ماند مرا با منجنیق کبریا
&&
میزان کجا ماند مرا در عشقت ای موزون خوش
\\
ای مایه صد بی‌هشی دی از طریق سرکشی
&&
گفتی مرا چونی خوشی در حیرت بی‌چون خوش
\\
هر ناخوشی را در قود عدل رخت گردن بزد
&&
کان ناخوشی‌ها خورده بد در غیبت تو خون خوش
\\
ای شمس تبریزی تویی کاندر جلالت صدتویی
&&
جان منست آن ماهیی در وی چو تو ذاالنون خوش
\\
\end{longtable}
\end{center}
