\begin{center}
\section*{غزل شماره ۱۵۹۵: سر قدم کردیم و آخر سوی جیحون تاختیم}
\label{sec:1595}
\addcontentsline{toc}{section}{\nameref{sec:1595}}
\begin{longtable}{l p{0.5cm} r}
سر قدم کردیم و آخر سوی جیحون تاختیم
&&
عالمی برهم زدیم و چست و بیرون تاختیم
\\
چون براق عشق عرشی بود زیر ران ما
&&
گنبدی کردیم و سوی چرخ گردون تاختیم
\\
عالم چون را مثال ذره‌ها برهم زدیم
&&
تا به پیش تخت آن سلطان بی‌چون تاختیم
\\
فهم و وهم و عقل انسان جملگی در ره بریخت
&&
چونک از شش حد انسان سخت افزون تاختیم
\\
چونک در سینور مجنونان آن لیلی شدیم
&&
سرکش آمد مرکب و از حد مجنون تاختیم
\\
نفس چون قارون ز سعی ما درون خاک شد
&&
بعد از آن مردانه سوی گنج قارون تاختیم
\\
دشت و هامون روح گیرد گر بیابد ذره‌ای
&&
ز آنچ ما از نور او در دشت و هامون تاختیم
\\
بس صدف‌های چو گوهر زیر سنگی کوفتیم
&&
تا به سوی گنج‌های در مکنون تاختیم
\\
سوی شمع شمس تبریزی به بیشه شیر جان
&&
بوده پروانه نپنداری که اکنون تاختیم
\\
\end{longtable}
\end{center}
