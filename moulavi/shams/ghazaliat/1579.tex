\begin{center}
\section*{غزل شماره ۱۵۷۹: ما آب دریم ما چه دانیم}
\label{sec:1579}
\addcontentsline{toc}{section}{\nameref{sec:1579}}
\begin{longtable}{l p{0.5cm} r}
ما آب دریم ما چه دانیم
&&
چه شور و شریم ما چه دانیم
\\
هر دم ز شراب بی‌نشانی
&&
خود مستتریم ما چه دانیم
\\
تا گوهر حسن تو بدیدیم
&&
رخ همچو زریم ما چه دانیم
\\
تا عشق تو پای ما گرفته‌ست
&&
بی‌پا و سریم ما چه دانیم
\\
خشک و تر ما همه تویی تو
&&
خوش خشک و تریم ما چه دانیم
\\
سرحلقه زلف تو گرفتیم
&&
خوش می شمریم ما چه دانیم
\\
گر زیر و زبر شود دو عالم
&&
زیر و زبریم ما چه دانیم
\\
گر سبزه و باغ خشک گردد
&&
ما از تو چریم ما چه دانیم
\\
گلزار اگر همه بریزد
&&
گل از تو بریم ما چه دانیم
\\
گر چرخ هزار مه نماید
&&
در تو نگریم ما چه دانیم
\\
گر زانک شکر جهان بگیرد
&&
ما باده خوریم ما چه دانیم
\\
شمس تبریز ز آفتابت
&&
همچون قمریم ما چه دانیم
\\
\end{longtable}
\end{center}
