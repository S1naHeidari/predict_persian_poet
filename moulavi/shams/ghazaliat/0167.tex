\begin{center}
\section*{غزل شماره ۱۶۷: کی بپرسد جز تو خسته و رنجور تو را}
\label{sec:0167}
\addcontentsline{toc}{section}{\nameref{sec:0167}}
\begin{longtable}{l p{0.5cm} r}
کی بپرسد جز تو خسته و رنجور تو را
&&
ای مسیح از پی پرسیدن رنجور بیا
\\
دست خود بر سر رنجور بنه که چونی
&&
از گناهش بمیندیش و به کین دست مخا
\\
آنک خورشید بلا بر سر او تیغ زدست
&&
گستران بر سر او سایه احسان و رضا
\\
این مقصر به دو صد رنج سزاوار شدست
&&
لیک زان لطف به جز عفو و کرم نیست سزا
\\
آن دلی را که به صد شیر و شکر پروردی
&&
مچشانش پس از آن هر نفسی زهر جفا
\\
تا تو برداشته‌ای دل ز من و مسکن من
&&
بند بشکست و درآمد سوی من سیل بلا
\\
تو شفایی چو بیایی خوش و رو بنمایی
&&
سپه رنج گریزند و نمایند قفا
\\
به طبیبش چه حواله کنی ای آب حیات
&&
از همان جا که رسد درد همان جاست دوا
\\
همه عالم چو تنند و تو سر و جان همه
&&
کی شود زنده تنی که سر او گشت جدا
\\
ای تو سرچشمه حیوان و حیات همگان
&&
جوی ما خشک شده‌ست آب از این سو بگشا
\\
جز از این چند سخن در دل رنجور بماند
&&
تا نبیند رخ خوب تو نگوید به خدا
\\
\end{longtable}
\end{center}
