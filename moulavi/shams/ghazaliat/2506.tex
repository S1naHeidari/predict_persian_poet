\begin{center}
\section*{غزل شماره ۲۵۰۶: مبارک باشد آن رو را بدیدن بامدادانی}
\label{sec:2506}
\addcontentsline{toc}{section}{\nameref{sec:2506}}
\begin{longtable}{l p{0.5cm} r}
مبارک باشد آن رو را بدیدن بامدادانی
&&
به بوسیدن چنان دستی ز شاهنشاه سلطانی
\\
بدیدن بامدادانی چنان رو را چه خوش باشد
&&
هم از آغاز روز او را بدیدن ماه تابانی
\\
دو خورشید از بگه دیدن یکی خورشید از مشرق
&&
دگر خورشید بر افلاک هستی شاد و خندانی
\\
بدیدن آفتابی را که خورشیدش سجود آرد
&&
ولیک او را کجا بیند که این جسم است و او جانی
\\
زهی صبحی که او آید نشیند بر سر بالین
&&
تو چشم از خواب بگشایی ببینی شاه شادانی
\\
زهی روز و زهی ساعت زهی فر و زهی دولت
&&
چنان دشواریابی را بگه بینی تو آسانی
\\
اگر از ناز بنشیند گدازد آهن از غصه
&&
وگر از لطف پیش آید به هر مفلس رسد کانی
\\
اگر در شب ببینندش شود از روز روشنتر
&&
ور از چاهی ببینندش شود آن چاه ایوانی
\\
که خورشیدش لقب تاش است شمس الدین تبریزی
&&
که او آن است و صد چون آن که صوفی گویدش آنی
\\
\end{longtable}
\end{center}
