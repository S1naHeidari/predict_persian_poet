\begin{center}
\section*{غزل شماره ۱۶۱۰: منم آن عاشق عشقت که جز این کار ندارم}
\label{sec:1610}
\addcontentsline{toc}{section}{\nameref{sec:1610}}
\begin{longtable}{l p{0.5cm} r}
منم آن عاشق عشقت که جز این کار ندارم
&&
که بر آن کس که نه عاشق به جز انکار ندارم
\\
دل غیر تو نجویم سوی غیر تو نپویم
&&
گل هر باغ نبویم سر هر خار ندارم
\\
به تو آوردم ایمان دل من گشت مسلمان
&&
به تو دل گفت که ای جان چو تو دلدار ندارم
\\
چو تویی چشم و زبانم دو نبینم دو نخوانم
&&
جز یک جان که تویی آن به کس اقرار ندارم
\\
چو من از شهد تو نوشم ز چه رو سرکه فروشم
&&
جهت رزق چه کوشم نه که ادرار ندارم
\\
ز شکربوره سلطان نه ز مهمانی شیطان
&&
بخورم سیر بر این خوان سر ناهار ندارم
\\
نخورم غم نخورم غم ز ریاضت نزنم دم
&&
رخ چون زر بنگر گر زر بسیار ندارم
\\
نخورد خسرو دل غم مگر الا غم شیرین
&&
به چه دل غم خورم آخر دل غمخوار ندارم
\\
پی هر خایف و ایمن کنمی شرح ولیکن
&&
ز سخن گفتن باطن دل گفتار ندارم
\\
تو که بی‌داغ جنونی خبری گوی که چونی
&&
که من از چون و چگونه دگر آثار ندارم
\\
چو ز تبریز برآمد مه شمس الحق و دینم
&&
سر این ماه شبستان سپهدار ندارم
\\
\end{longtable}
\end{center}
