\begin{center}
\section*{غزل شماره ۹۶۸: سیبکی نیم سرخ و نیمی زرد}
\label{sec:0968}
\addcontentsline{toc}{section}{\nameref{sec:0968}}
\begin{longtable}{l p{0.5cm} r}
سیبکی نیم سرخ و نیمی زرد
&&
از گل و زعفران حکایت کرد
\\
چون جدا گشت عاشق از معشوق
&&
برد معشوق ناز و عاشق درد
\\
این دو رنگ مخالف از یک هجر
&&
بر رخ هر دو عشق پیدا کرد
\\
رخ معشوق زرد لایق نیست
&&
سرخی و فربهی عاشق سرد
\\
چونک معشوق ناز آغازید
&&
ناز کش عاشقا مگیر نبرد
\\
انا کالشوک سیدی کالورد
&&
فهما اثنان فی الحقیقه فرد
\\
انه الشمس اننی کالظل
&&
منه حر البقا و منی البرد
\\
ان جالوت بارز الطالوت
&&
ان داوود قدروا فی السرد
\\
دل ز تن زاد لیک شاه تنست
&&
همچنانک بزاید از زن مرد
\\
باز در دل یکی دلیست نهان
&&
چون سواری نهان شده در گرد
\\
جنبش گرد از سوار بود
&&
اوست کاین گرد را به رقص آورد
\\
نیست شطرنج تا تو فکر کنی
&&
با توکل بریز مهره چو نرد
\\
شمس تبریز آفتاب دلست
&&
میوه‌های دل آن تفش پرورد
\\
\end{longtable}
\end{center}
