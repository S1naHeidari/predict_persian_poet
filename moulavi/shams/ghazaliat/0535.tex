\begin{center}
\section*{غزل شماره ۵۳۵: سودای تو در جوی جان چون آب حیوان می‌رود}
\label{sec:0535}
\addcontentsline{toc}{section}{\nameref{sec:0535}}
\begin{longtable}{l p{0.5cm} r}
سودای تو در جوی جان چون آب حیوان می‌رود
&&
آب حیات از عشق تو در جوی جویان می‌رود
\\
عالم پر از حمد و ثنا از طوطیان آشنا
&&
مرغ دلم بر می‌پرد چون ذکر مرغان می‌رود
\\
بر ذکر ایشان جان دهم جان را خوش و خندان دهم
&&
جان چون نخندد چون ز تن در لطف جانان می‌رود
\\
هر مرغ جان چون فاخته در عشق طوقی ساخته
&&
چون من قفس پرداخته سوی سلیمان می‌رود
\\
از جان هر سبحانیی هر دم یکی روحانیی
&&
مست و خراب و فانیی تا عرش سبحان می‌رود
\\
جان چیست خم خسروان در وی شراب آسمان
&&
زین رو سخن چون بیخودان هر دم پریشان می‌رود
\\
در خوردنم ذوقی دگر در رفتنم ذوقی دگر
&&
در گفتنم ذوقی دگر باقی بر این سان می‌رود
\\
میدان خوش است ای ماه رو با گیر و دار ما و تو
&&
ای هر که لنگست اسب او لنگان ز میدان می‌رود
\\
مه از پی چوگان تو خود را چو گویی ساخته
&&
خورشید هم جان باخته چون گوی غلطان می‌رود
\\
این دو بسی بشتافته پیش تو ره نایافته
&&
در نور تو دربافته بیرون ایوان می‌رود
\\
چون نور بیرون این بود پس او که دولت بین بود
&&
یا رب چه باتمکین بود یا رب چه رخشان می‌رود
\\
\end{longtable}
\end{center}
