\begin{center}
\section*{غزل شماره ۲۲۸۷: آمد یار و بر کفش جام میی چو مشعله}
\label{sec:2287}
\addcontentsline{toc}{section}{\nameref{sec:2287}}
\begin{longtable}{l p{0.5cm} r}
آمد یار و بر کفش جام میی چو مشعله
&&
گفت بیا حریف شو گفتم آمدم هله
\\
جام میی که تابشش جان ببرد ز مشتری
&&
چرخ زند ز بوی او بر سر چرخ سنبله
\\
کوه از او سبک شده مغز از او گران شده
&&
روح سبوکشش شده عقل شکسته بلبله
\\
پاک نی و پلید نی در دو جهان بدید نی
&&
قفل گشا کلید نی کنده هزار سلسله
\\
تازه کند ملول را مایه دهد فضول را
&&
آنک زند ز بی‌رهه راه هزار قافله
\\
پیش رو بدان شده رهزن زاهدان شده
&&
دایه شاهدان شده مایه بانگ و غلغله
\\
هر کی خورد ز نیک و بد مست بمانده تا ابد
&&
هر که نخورد تا رود جانب غصه بی‌گله
\\
غرقه شو اندر آب حق مست شو از شراب حق
&&
نیست شو و خراب حق ای دل تنگ حوصله
\\
هر کی بدان گمان برد از کف مرگ جان برد
&&
آنک نگویم آن برد اینت عظیم منزله
\\
\end{longtable}
\end{center}
