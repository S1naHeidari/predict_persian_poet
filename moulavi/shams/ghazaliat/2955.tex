\begin{center}
\section*{غزل شماره ۲۹۵۵: چون روی آتشین را یک دم تو می‌نپوشی}
\label{sec:2955}
\addcontentsline{toc}{section}{\nameref{sec:2955}}
\begin{longtable}{l p{0.5cm} r}
چون روی آتشین را یک دم تو می‌نپوشی
&&
ای دوست چند جوشم گویی که چند جوشی
\\
ای جان و عقل مسکین کی یابد از تو تسکین
&&
زین سان که تو نهادی قانون می فروشی
\\
سرنای جان‌ها را در می دمی تو دم دم
&&
نی را چه جرم باشد چون تو همی‌خروشی
\\
روپوش برنتابد گر تاب روی این است
&&
پنهان نگردد این رو گر صد هزار پوشی
\\
بر گرد شید گردی ای جان عشق ساده
&&
یا نیک سرخ چشمی یا خود سیاه گوشی
\\
گر ز آنک عقل داری دیوانه چون نگشتی
&&
ور نه از اصل عشقی با عشق چند کوشی
\\
اجزای خویش دیدم اندر حضور خامش
&&
بس نعره‌ها شنیدم در زیر هر خموشی
\\
گفتم به شمس تبریز کاین خامشان کیانند
&&
گفتا چو وقت آید تو نیز هم نپوشی
\\
\end{longtable}
\end{center}
