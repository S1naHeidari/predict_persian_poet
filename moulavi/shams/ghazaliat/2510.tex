\begin{center}
\section*{غزل شماره ۲۵۱۰: مرا پرسید آن سلطان به نرمی و سخن خایی}
\label{sec:2510}
\addcontentsline{toc}{section}{\nameref{sec:2510}}
\begin{longtable}{l p{0.5cm} r}
مرا پرسید آن سلطان به نرمی و سخن خایی
&&
عجب امسال ای عاشق بدان اقبالگه آیی
\\
برای آنک واگوید نمودم گوش کرانه
&&
که یعنی من گران گوشم سخن را بازفرمایی
\\
مگر کوری بود کان دم نسازد خویشتن را کر
&&
که تا باشد که واگوید سخن آن کان زیبایی
\\
شهم دریافت بازی را بخندید و بگفت این را
&&
بدان کس گو که او باشد چو تو بی‌عقل و هیهایی
\\
یکی حمله دگر چون کر ببردم گوش و سر پیشش
&&
بگفتا شید آوردی تو جز استیزه نفزایی
\\
چون دعوی کری کردم جواب و عذر چون گویم
&&
همه در هام شد بسته بدان فرهنگ و بدرایی
\\
به دربانش نظر کردم که یک نکته درافکن تو
&&
بپرسیدش ز نام من بگفتا گیج و سودایی
\\
نظر کردم دگربارش که اندرکش به گفتارش
&&
که شاگرد در اویی چو او عیارسیمایی
\\
مرا چشمک زد آن دربان که تو او را نمی‌دانی
&&
که حیلت گر به پیش او نبیند غیر رسوایی
\\
مکن حیلت که آن حلوا گهی در حلق تو آید
&&
که جوشی بر سر آتش مثال دیگ حلوایی
\\
\end{longtable}
\end{center}
