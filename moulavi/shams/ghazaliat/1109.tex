\begin{center}
\section*{غزل شماره ۱۱۰۹: ساقیا باده گلرنگ بیار}
\label{sec:1109}
\addcontentsline{toc}{section}{\nameref{sec:1109}}
\begin{longtable}{l p{0.5cm} r}
ساقیا باده گلرنگ بیار
&&
داروی درد دل تنگ بیار
\\
روز بزمست نه روز رزمست
&&
خنجر جنگ ببر چنگ بیار
\\
ای ز تو دردکشان دردکشان
&&
دردیی که کندم دنگ بیار
\\
من ز هر درد نمی‌گردم دنگ
&&
دردی آن سره سرهنگ بیار
\\
روز جامست نه نام و ناموس
&&
نام از پیش ببر ننگ بیار
\\
کیمیایی که کند سنگ عقیق
&&
آزمون کن بر او سنگ بیار
\\
صیقل آینه نه فلکست
&&
ز امتحان آهن پرزنگ بیار
\\
چشمه خضر تو را می‌خواند
&&
که سبو کش دو سه فرسنگ بیار
\\
پس گردن ز چه رو می‌خاری
&&
نک ظفر هست تو آهنگ بیار
\\
حرف رنگست اگر خوش بویست
&&
جان بی‌صورت و بی‌رنگ بیار
\\
کم کنی رنگ بیفزاید روح
&&
بوی روح صنم شنگ بیار
\\
لب ببند از دغل و از حیلت
&&
جان بی‌حیلت و فرهنگ بیار
\\
\end{longtable}
\end{center}
