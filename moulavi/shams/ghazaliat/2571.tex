\begin{center}
\section*{غزل شماره ۲۵۷۱: ای شاه مسلمانان وی جان مسلمانی}
\label{sec:2571}
\addcontentsline{toc}{section}{\nameref{sec:2571}}
\begin{longtable}{l p{0.5cm} r}
ای شاه مسلمانان وی جان مسلمانی
&&
پنهان شده و افکنده در شهر پریشانی
\\
ای آتش در آتش هم می‌کش و هم می‌کش
&&
سلطان سلاطینی بر کرسی سبحانی
\\
شاهنشه هر شاهی صد اختر و صد ماهی
&&
هر حکم که می‌خواهی می‌کن که همه جانی
\\
گفتی که تو را یارم رخت تو نگهدارم
&&
از شیر عجب باشد بس نادره چوپانی
\\
گر نیست و گر هستم گر عاقل و گر مستم
&&
ور هیچ نمی‌دانم دانم که تو می‌دانی
\\
گر در غم و در رنجم در پوست نمی‌گنجم
&&
کز بهر چو تو عیدی قربانم و قربانی
\\
گه چون شب یغمایی هر مدرکه بربایی
&&
روز از تن همچون شب چون صبح برون رانی
\\
گه جامه بگردانی گویی که رسولم من
&&
یا رب که چه گردد جان چون جامه بگردانی
\\
در رزم تویی فارس بر بام تویی حارس
&&
آن چیست عجب جز تو کو را تو نگهبانی
\\
ای عشق تویی جمله بر کیست تو را حمله
&&
ای عشق عدم‌ها را خواهی که برنجانی
\\
ای عشق تویی تنها گر لطفی و گر قهری
&&
سرنای تو می‌نالد هم تازی و سریانی
\\
گر دیده ببندی تو ور هیچ نخندی تو
&&
فر تو همی‌تابد از تابش پیشانی
\\
پنهان نتوان بردن در خانه چراغی را
&&
ای ماه چه می‌آیی در پرده پنهانی
\\
ای چشم نمی‌بینی این لشکر سلطان را
&&
وی گوش نمی‌نوشی این نوبت سلطانی
\\
گفتم که به چه دهی آن گفتا که به بذل جان
&&
گنجی است به یک حبه در غایت ارزانی
\\
لاحول کجا راند دیوی که تو بگماری
&&
باران نکند ساکن گردی که تو ننشانی
\\
چون سرمه جادویی در دیده کشی دل را
&&
تمییز کجا ماند در دیده انسانی
\\
هر نیست بود هستی در دیده از آن سرمه
&&
هر وهم برد دستی از عقل به آسانی
\\
از خاک درت باید در دیده دل سرمه
&&
تا سوی درت آید جوینده ربانی
\\
تا جزو به کل تازد حبه سوی کان یازد
&&
قطره سوی بحر آید از سیل کهستانی
\\
نی سیل بود این جا نی بحر بود آن جا
&&
خامش که نشد ظاهر هرگز سر روحانی
\\
\end{longtable}
\end{center}
