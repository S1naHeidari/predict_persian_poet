\begin{center}
\section*{غزل شماره ۳۱۳۱: الا هات حمرا کالعندم}
\label{sec:3131}
\addcontentsline{toc}{section}{\nameref{sec:3131}}
\begin{longtable}{l p{0.5cm} r}
الا هات حمرا کالعندم
&&
کانی ما زجتها عن دمی
\\
و یبدو سناها علی وجنتی
&&
اذا انحدرت کاسها عن فمی
\\
فطوبی لسکراء من مغنم
&&
و تعسا لصحواء من مغرم
\\
می درغمی خور اگر در غمی
&&
که شادی فزاید می درغمی
\\
بیا نوش کن ای بت نوش لب
&&
شراب محرم اگر محرمی
\\
مگو نام فردا اگر صوفیی
&&
همین دم یکی شو اگر همدمی
\\
برای چنین جام عالم بها
&&
بهل مملکت را اگر ادهمی
\\
درآشام یک جام دریا دلا
&&
اگر ظاهر کند گوهر آدمی
\\
چرا بسته باشی چو در مجلسی
&&
چرا خشک باشی چو در زمزمی
\\
چرا می‌نگیری نخستین قدح
&&
چپ و راست بنما که از کی کمی
\\
ز جام فلک پاک و صافیتری
&&
که برتر از این گنبد اعظمی
\\
بنوش ای ندیمی که هم خرقه‌ای
&&
بجوش ای شرابی که خوش مرهمی
\\
چو موسی عمران توی عمر جان
&&
چو عیسی مریم روان بر یمی
\\
چو یوسف همه فتنه مجلسی
&&
چو اقبال و باده عدوی غمی
\\
ز هر باد چون کاه از جا مرو
&&
که چون کوه در مرتبت محکمی
\\
بحل برج کژدم سوی زهره رو
&&
که کژدم ندارد به جز کژدمی
\\
به تو آمدم زانک نشکیفتم
&&
ز احسان و بخشایش و مردمی
\\
چنین خال زیبا که بر روی توست
&&
پناه غریبی و خال و عمی
\\
فانت الربیع و انت المدام
&&
و مولی الملوک الا فاحکمی
\\
خلایق ز تو واله و درهمند
&&
تو چون زلف جعدت چرا درهمی
\\
مگر شمس تبریز عقلت ببرد
&&
که چون من خرابی و لایعلمی
\\
\end{longtable}
\end{center}
