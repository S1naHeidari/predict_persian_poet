\begin{center}
\section*{غزل شماره ۱۵۴۸: روی تو چو نوبهار دیدم}
\label{sec:1548}
\addcontentsline{toc}{section}{\nameref{sec:1548}}
\begin{longtable}{l p{0.5cm} r}
روی تو چو نوبهار دیدم
&&
گل را ز تو شرمسار دیدم
\\
تا در دل من قرار کردی
&&
دل را ز تو بی‌قرار دیدم
\\
من چشم شدم همه چو نرگس
&&
کان نرگس پرخمار دیدم
\\
در عشق روم که عشق را من
&&
از جمله بلا حصار دیدم
\\
از ملک جهان و عیش عالم
&&
من عشق تو اختیار دیدم
\\
خود ملک تویی و جان عالم
&&
یک بود و منش هزار دیدم
\\
من مردم و از تو زنده گشتم
&&
پس عالم را دو بار دیدم
\\
ای مطرب اگر تو یار مایی
&&
این پرده بزن که یار دیدم
\\
در شهر شما چه یار جویم
&&
چون یاری شهریار دیدم
\\
چون در بر خود خوشش فشردم
&&
آیین شکرفشار دیدم
\\
چون بستم من دهان ز گفتن
&&
بس گفتن بی‌شمار دیدم
\\
چون پای نماند اندر این ره
&&
من رفتن راهوار دیدم
\\
سر درنکشم ز ضر که بی‌سر
&&
سرهای کلاه دار دیدم
\\
بس کن که ملول گشت دلبر
&&
بر خاطر او غبار دیدم
\\
\end{longtable}
\end{center}
