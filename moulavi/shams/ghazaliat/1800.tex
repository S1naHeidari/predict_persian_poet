\begin{center}
\section*{غزل شماره ۱۸۰۰: هذا رشاد الکافرین هذا جزاء الصابرین}
\label{sec:1800}
\addcontentsline{toc}{section}{\nameref{sec:1800}}
\begin{longtable}{l p{0.5cm} r}
هذا رشاد الکافرین هذا جزاء الصابرین
&&
هذا معاد الغابرین نعم الرجا نعم المعین
\\
صد آفتاب از تو خجل او خوشه چین تو مشتعل
&&
نعره زنان در سینه دل استدرکوا عین الیقین
\\
از آسمان در هر غذا از علویان آید ندا
&&
کای روح پاک مقتدا یا رحمة للعالمین
\\
حبس حقایق را دری باغ شقایق را تری
&&
هم از دقایق مخبری پیش از ظهور یوم دین
\\
ای دل ز دیده دام کن دیده نداری وام کن
&&
ای جان نفیر عام کن تا برجهی زین آب و طین
\\
ای جان تو باری لمتری شیر جهاد اکبری
&&
باید که صف‌ها بردری و آیی بر آن قلعه حصین
\\
هان ای حبیب و ای محب بشنو صلا و فاستجب
&&
گر گشت جانان محتجب جان می رود نیکوش بین
\\
گفته‌ست جان ذوفنون چون غرقه شد در بحر خون
&&
یا لیت قومی یعلمون که با کیانم همنشین
\\
سیلم سوی دریا روم روحم سوی بالا روم
&&
لعلم به گوهرها روم یا تاج باشم یا نگین
\\
هر کس که یابد این رشد زان قند بی‌حد او چشد
&&
مانند موسی برکشد از خاره او ماء معین
\\
چون مست گشتم برجهم بر رخش دل زین برنهم
&&
زیرا که مشتاق شهم آن ماه از مه‌ها مهین
\\
گفتن رها کن ای پدر گفتن حجاب است از نظر
&&
گر می خوری زان می بخور ور می گزینی زان گزین
\\
الصمت اولی بالرصد فی النطق تهییج العدد
&&
جاء المدد جاء المدد استنصروا یا مسلمین
\\
مستفعلن مستفعلن یا سیدا یا اقربا
&&
فی نشونا او مشینا من قربه العرق الوتین
\\
\end{longtable}
\end{center}
