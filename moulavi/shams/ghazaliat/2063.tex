\begin{center}
\section*{غزل شماره ۲۰۶۳: ای رخ خندان تو مایه صد گلستان}
\label{sec:2063}
\addcontentsline{toc}{section}{\nameref{sec:2063}}
\begin{longtable}{l p{0.5cm} r}
ای رخ خندان تو مایه صد گلستان
&&
باغ خدایی درآ خار بده گل ستان
\\
جامه تن را بکن جان برهنه ببین
&&
جان برهنه خوش است تا چه کنی جامه دان
\\
هین که نه‌ای بی‌زبان پیش چنین جان‌ها
&&
قصه نی بی‌زبان نعره جان بی‌دهان
\\
آمد امروز یار گفت سلام علیک
&&
چرخ و زمین را مجو از نفسش آن زمان
\\
خسرو خوبان بخواست از صنمان سرخراج
&&
خاست غریو از فلک وز سوی مه کالامان
\\
لعل لب او که دور از لب و دندان تو
&&
خواند فسون‌های عشق خواجه ببین این نشان
\\
آمد غماز عشق گفت در این گوش من
&&
یار میان شماست خوب و لطیف و نهان
\\
دامن دل را کشید یار به یک گوشه‌ای
&&
گوشه بس بوالعجب زان سوی هفت آسمان
\\
گفت ترایم ولیک هر که بگوید ز من
&&
شرح دهد از لبم ده بزنش بر دهان
\\
و آنک بگوید ز تو برد مرا و تو را
&&
و آنک بگوید ز من دور شد از هر دوان
\\
\end{longtable}
\end{center}
