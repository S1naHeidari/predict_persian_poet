\begin{center}
\section*{غزل شماره ۱۳۱۳: جان و سر تو که بگو بی‌نفاق}
\label{sec:1313}
\addcontentsline{toc}{section}{\nameref{sec:1313}}
\begin{longtable}{l p{0.5cm} r}
جان و سر تو که بگو بی‌نفاق
&&
در کرم و حسن چرایی تو طاق
\\
روی چو خورشید تو بخشش کند
&&
روز وصالی که ندارد فراق
\\
دل ز همه برکنم از بهر تو
&&
بهر وفای تو ببندم نطاق
\\
گر تو مرا گویی رو صبر کن
&&
باشد تکلیف بما لایطاق
\\
سخت بود هجر و فراق ای حبیب
&&
خاصه فراقی ز پی اعتناق
\\
چون پدر و مادر عقلست و روح
&&
هر دو تویی چون شوم ای دوست عاق
\\
روم چو در مهر تو آهی کنند
&&
دود رسد جانب شام و عراق
\\
در تتق سینه عشاق تو
&&
ماه رخان قندلبان سیم ساق
\\
رقص کنان در خضر لطف تو
&&
نوش کنان ساغر صدق و وفاق
\\
دست زنان جمله و گویان بلاغ
&&
طاق و طرنبین و طرنبین و طاق
\\
مژده کسی را که زرش دزد برد
&&
مژده کسی را که دهد زن طلاق
\\
خاصه کسی را که جهان را همه
&&
ترک کند فرد شود بی‌شقاق
\\
لاجرمش عشق کشد پیشکش
&&
همچو محمد به سحرگه براق
\\
بربردش زود براق دلش
&&
فوق سماوات رفاع طباق
\\
جان و سر تو که بگو باقیش
&&
که دهنم بسته شد از اشتیاق
\\
هر چه بگفتم کژ و مژ راست کن
&&
چونک مهندس تویی و من مشاق
\\
\end{longtable}
\end{center}
