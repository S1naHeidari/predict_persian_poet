\begin{center}
\section*{غزل شماره ۶۴: تو دیدی هیچ عاشق را که سیری بود از این سودا}
\label{sec:0064}
\addcontentsline{toc}{section}{\nameref{sec:0064}}
\begin{longtable}{l p{0.5cm} r}
تو دیدی هیچ عاشق را که سیری بود از این سودا
&&
تو دیدی هیچ ماهی را که او شد سیر از این دریا
\\
تو دیدی هیچ نقشی را که از نقاش بگریزد
&&
تو دیدی هیچ وامق را که عذرا خواهد از عذرا
\\
بود عاشق فراق اندر چو اسمی خالی از معنی
&&
ولی معنی چو معشوقی فراغت دارد از اسما
\\
تویی دریا منم ماهی چنان دارم که می‌خواهی
&&
بکن رحمت بکن شاهی که از تو مانده‌ام تنها
\\
ایا شاهنشه قاهر چه قحط رحمتست آخر
&&
دمی که تو نه‌ای حاضر گرفت آتش چنین بالا
\\
اگر آتش تو را بیند چنان در گوشه بنشیند
&&
کز آتش هر که گل چیند دهد آتش گل رعنا
\\
عذابست این جهان بی‌تو مبادا یک زمان بی‌تو
&&
به جان تو که جان بی‌تو شکنجه‌ست و بلا بر ما
\\
خیالت همچو سلطانی شد اندر دل خرامانی
&&
چنانک آید سلیمانی درون مسجد اقصی
\\
هزاران مشعله برشد همه مسجد منور شد
&&
بهشت و حوض کوثر شد پر از رضوان پر از حورا
\\
تعالی الله تعالی الله درون چرخ چندین مه
&&
پر از حورست این خرگه نهان از دیده اعمی
\\
زهی دلشاد مرغی کو مقامی یافت اندر عشق
&&
به کوه قاف کی یابد مقام و جای جز عنقا
\\
زهی عنقای ربانی شهنشه شمس تبریزی
&&
که او شمسیست نی شرقی و نی غربی و نی در جا
\\
\end{longtable}
\end{center}
