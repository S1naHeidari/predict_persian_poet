\begin{center}
\section*{غزل شماره ۷۳۴: مطربا این پرده زن کز رهزنان فریاد و داد}
\label{sec:0734}
\addcontentsline{toc}{section}{\nameref{sec:0734}}
\begin{longtable}{l p{0.5cm} r}
مطربا این پرده زن کز رهزنان فریاد و داد
&&
خاصه این رهزن که ما را این چنین بر باد داد
\\
مطربا این ره زدن زان رهزنان آموختی
&&
زانک از شاگرد آید شیوه‌های اوستاد
\\
مطربا رو بر عدم زن زانک هستی ره‌زنست
&&
زانک هستی خایفست و هیچ خایف نیست شاد
\\
می‌زن ای هستی ره هستان که جان انگاشتست
&&
کاندر این هستی نیامد وز عدم هرگز نزاد
\\
ما بیابان عدم گیریم هم در بادیه
&&
در وجود این جمله بند و در عدم چندین گشاد
\\
این عدم دریا و ما ماهی و هستی همچو دام
&&
ذوق دریا کی شناسد هر که در دام اوفتاد
\\
هر که اندر دام شد از چار طبع او چارمیخ
&&
دانک روزی می‌دوید از ابلهی سوی مراد
\\
آتش صبر تو سوزد آتش هستیت را
&&
آتش اندر هست زن و اندر تن هستی نژاد
\\
قدحه و الموریاتش نیست الا سوز صبر
&&
ضبحه و العادیاتش نیست جز جان‌های راد
\\
برد و ماندی هست آخر تا کی ماند کی برد
&&
ور نه این شطرنج عالم چیست با جنگ و جهاد
\\
گه ره شه را بگیرد بیدق کژرو به ظلم
&&
چیست فرزین گشته‌ام گر کژ روم باشد سداد
\\
من پیاده رفته‌ام در راستی تا منتها
&&
تا شدم فرزین و فرزین بندهاام دست داد
\\
رخ بدو گوید که منزل‌هات ما را منزلیست
&&
خط و تین ماست این جمله منازل تا معاد
\\
تن به صد منزل رود دل می‌رود یک تک به حج
&&
ره روی باشد چو جسم و ره روی همچون فؤاد
\\
شاه گوید مر شما را از منست این یاد و بود
&&
گر نباشد سایه من بود جمله گشت باد
\\
اسب را قیمت نماند پیل چون پشه شود
&&
خانه‌ها ویرانه‌ها گردد چو شهر قوم عاد
\\
اندر این شطرنج برد و ماند یک سان شد مرا
&&
تا بدیدم کاین هزاران لعب یک کس می‌نهاد
\\
در نجاتش مات هست و هست در ماتش نجات
&&
زان نظر ماتیم ای شه آن نظر بر مات باد
\\
\end{longtable}
\end{center}
