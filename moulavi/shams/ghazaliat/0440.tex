\begin{center}
\section*{غزل شماره ۴۴۰: امروز شهر ما را صد رونق‌ست و جانست}
\label{sec:0440}
\addcontentsline{toc}{section}{\nameref{sec:0440}}
\begin{longtable}{l p{0.5cm} r}
امروز شهر ما را صد رونق‌ست و جانست
&&
زیرا که شاه خوبان امروز در میانست
\\
حیران چرا نباشد خندان چرا نباشد
&&
شهری که در میانش آن صارم زمانست
\\
آن آفتاب خوبی چون بر زمین بتابد
&&
آن دم زمین خاکی بهتر ز آسمانست
\\
بر چرخ سبزپوشان پر می‌زنند یعنی
&&
سلطان و خسرو ما آن‌ست و صد چنانست
\\
ای جان جان جانان از ما سلام برخوان
&&
رحم آر بر ضعیفان عشق تو بی‌امانست
\\
چون سبز و خوش نباشد عالم چو تو بهاری
&&
چون ایمنی نباشد چون شیر پاسبانست
\\
چون کوفت او در دل ناآمده به منزل
&&
دانست جان ز بویش کان یار مهربانست
\\
آن کو کشید دستت او آفریده‌ستت
&&
وان کو قرین جان شد او صاحب قرانست
\\
او ماه بی‌خسوف‌ست خورشید بی‌کسوفست
&&
او خمر بی‌خمارست او سود بی‌زیانست
\\
آن شهریار اعظم بزمی نهاد خرم
&&
شمع و شراب و شاهد امروز رایگانست
\\
چون مست گشت مردم شد گوهرش برهنه
&&
پهلو شکست کان را زان کس که پهلوانست
\\
دلاله چون صبا شد از خار گل جدا شد
&&
باران نبات‌ها را در باغ امتحانست
\\
بی عز و نازنینی کی کرد ناز و بینی
&&
هر کس که کرد والله خام‌ست و قلتبانست
\\
خامش که تا بگوید بی‌حرف و بی‌زبان او
&&
خود چیست این زبان‌ها گر آن زبان زبانست
\\
\end{longtable}
\end{center}
