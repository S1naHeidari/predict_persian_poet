\begin{center}
\section*{غزل شماره ۲۸۵۵: هله ای دلی که خفته تو به زیر ظل مایی}
\label{sec:2855}
\addcontentsline{toc}{section}{\nameref{sec:2855}}
\begin{longtable}{l p{0.5cm} r}
هله ای دلی که خفته تو به زیر ظل مایی
&&
شب و روز در نمازی به حقیقت و غزالی
\\
مه بدر نور بارد سگ کوی بانگ دارد
&&
ز برای بانگ هر سگ مگذار روشنایی
\\
به نماز نان برسته جز نان دگر چه خواهد
&&
دل همچو بحر باید که گهر کند گدایی
\\
اگر آن میی که خوردی به سحر نبود گیرا
&&
بستان میی که یابی ز تفش ز خود رهایی
\\
به خدا به ذات پاکش که میی است کز حراکش
&&
برهد تن از هلاکش به سعادت سمایی
\\
بستان مکن ستیزه تو بدین حیات ریزه
&&
که حیات کامل آمد ز ورای جان فزایی
\\
بهلم دگر نگویم که دریغ باشد ای جان
&&
بر کور یوسفی را حرکات و خودنمایی
\\
\end{longtable}
\end{center}
