\begin{center}
\section*{غزل شماره ۲۴۹۴: زرگر آفتاب را بسته گاز می‌کنی}
\label{sec:2494}
\addcontentsline{toc}{section}{\nameref{sec:2494}}
\begin{longtable}{l p{0.5cm} r}
زرگر آفتاب را بسته گاز می‌کنی
&&
کرته شام را ز مه نقش و طراز می‌کنی
\\
روز و شب و نتایج این حبشی و روم را
&&
بر مثل اصولشان گرد و دراز می‌کنی
\\
گاه مجاز بنده را حق و حقیقتی دهی
&&
و آنک حقیقتی بود هزل و مجاز می‌کنی
\\
این چه کرامت است ای نقش خیال روی او
&&
با درهای بسته در خانه جواز می‌کنی
\\
خاطر همچو باد را نقش جحود می‌دهی
&&
خاطر بی‌نیاز را پر ز نیاز می‌کنی
\\
در شب ابرگین غم مشعله‌ها درآوری
&&
در دل تنگ پرگره پنجره باز می‌کنی
\\
ما به دمشق عشق تو مست و مقیم بهر تو
&&
تو ز دلال و عز خود عزم عزاز می‌کنی
\\
گاه ز نیم زلتی برهمشان همی‌زنی
&&
گاه خود از کبیرها چشم فراز می‌کنی
\\
گاه گدای راه را همت شاه می‌دهی
&&
گاه قباد و شاه را بنده آز می‌کنی
\\
می‌شکنی به زیر پا نای طرب نوای را
&&
چنگ شکسته بسته را لایق ساز می‌کنی
\\
بربط عشرت مرا گاه سه تا همی‌کنی
&&
پرده بوسلیک را گاه حجاز می‌کنی
\\
جان ز وجود جود تو آمد و مغز نغز شد
&&
باز ز پوست‌هاش چون همچو پیاز می‌کنی
\\
\end{longtable}
\end{center}
