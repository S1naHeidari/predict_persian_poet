\begin{center}
\section*{غزل شماره ۲۹۲۶: گر در آب و گر در آتش می‌روی}
\label{sec:2926}
\addcontentsline{toc}{section}{\nameref{sec:2926}}
\begin{longtable}{l p{0.5cm} r}
گر در آب و گر در آتش می‌روی
&&
آن نمی‌دانم برو خوش می‌روی
\\
در رخت پیداست والله رنگ او
&&
رو که سوی یار مه وش می‌روی
\\
نقش‌ها را پشت و پایی می‌زنی
&&
سوی نقش نامنقش می‌روی
\\
ذوق جان‌ها می‌زند بر جان تو
&&
مست و دست انداز و سرکش می‌روی
\\
در پی تو می‌دود اقبال رو
&&
گر به عرش و گر به مفرش می‌روی
\\
آنک در سر داری از سودای یار
&&
چه عجب گر تو مشوش می‌روی
\\
شه صلاح الدین برآ زین شش جهت
&&
گر چه ظاهر اندر این شش می‌روی
\\
\end{longtable}
\end{center}
