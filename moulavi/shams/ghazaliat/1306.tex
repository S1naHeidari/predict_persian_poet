\begin{center}
\section*{غزل شماره ۱۳۰۶: بیا بیا که تویی شیر شیر شیر مصاف}
\label{sec:1306}
\addcontentsline{toc}{section}{\nameref{sec:1306}}
\begin{longtable}{l p{0.5cm} r}
بیا بیا که تویی شیر شیر شیر مصاف
&&
ز مرغزار برون آ و صف‌ها بشکاف
\\
به مدحت آنچ بگویند نیست هیچ دروغ
&&
ز هر چه از تو بلافند صادقست نه لاف
\\
عجب که کرت دیگر ببیند این چشمم
&&
به سلطنت تو نشسته ملوک بر اطراف
\\
تو بر مقامه خویشی وز آنچ گفتم بیش
&&
ولیک دیده ز هجرت نه روشنست نه صاف
\\
شعاع چهره او خود نهان نمی‌گردد
&&
برو تو غیرت بافنده پرده‌ها می‌باف
\\
تو دلفریب صفت‌های دلفریب آری
&&
ولیک آتش من کی رها کند اوصاف
\\
چو عاشقان به جهان جان‌ها فدا کردند
&&
فدا بکردم جانی و جان جان به مصاف
\\
اگر چه کعبه اقبال جان من باشد
&&
هزار کعبه جان را بگرد تست طواف
\\
دهان ببسته‌ام از راز چون جنین غمم
&&
که کودکان به شکم در غذا خورند از ناف
\\
تو عقل عقلی و من مست پرخطای توام
&&
خطای مست بود پیش عقل عقل معاف
\\
خمار بی‌حد من بحرهای می‌خواهد
&&
که نیست مست تو را رطل‌ها و جره کفاف
\\
بجز به عشق تو جایی دگر نمی‌گنجم
&&
که نیست موضع سیمرغ عشق جز که قاف
\\
نه عاشق دم خویشم ولیک بوی تست
&&
چو دم زنم ز غمت از مت و از آلاف
\\
نه الف گیرد اجزای من به غیر تو دوست
&&
اگر هزار بخوانند سوره ایلاف
\\
به نور دیده سلف بسته‌ام به عشق رخت
&&
که گوش من نگشاید به قصه اسلاف
\\
منم کمانچه نداف شمس تبریزی
&&
فتاده آتش او در دکان این نداف
\\
\end{longtable}
\end{center}
