\begin{center}
\section*{غزل شماره ۲۱۶۷: دل آتش پذیر از توست برق و سنگ و آهن تو}
\label{sec:2167}
\addcontentsline{toc}{section}{\nameref{sec:2167}}
\begin{longtable}{l p{0.5cm} r}
دل آتش پذیر از توست برق و سنگ و آهن تو
&&
مرا سیران کجا باشد مرا تحویل و رفتن تو
\\
بدیدم بی‌تو من خود را تو دیدی بیخودم هم تو
&&
به زیر خاک دررفتم نرفتم من بیا من تو
\\
اگر گویم تو می‌گویی من آن ظلمت ز خود بینم
&&
از آن ظلمت که می‌گریم سری چون ماه برزن تو
\\
گریبانم دریدستم ز خود دامن کشیدستم
&&
که تا گیری گریبانم کشی از مهر دامن تو
\\
گریبانم دریدی تو و دامانم کشیدی تو
&&
کدامم من چه نامم من مرا جان تو مرا تن تو
\\
پشیمانم پشیمانم پشیمان تو پشیمان تو
&&
چو سوسن صد زبانم من زبان و نطق و سوسن تو
\\
دو چشمم خیره در رویت گهی چوگان گهی گویت
&&
تویی حیران تویی چوگان تویی دو چشم روشن تو
\\
به یک اندیشه حنظل را کنی بر من چو صد شکر
&&
به یک اندیشه شکر را کنی چون زهر دشمن تو
\\
تویی شکر تویی حنظل تویی اندیشه مبدل
&&
تویی مور و سلیمان تو تویی خورشید و روزن تو
\\
بدم من کافر احول شدم توحید را اکمل
&&
تویی احول کن کافر تویی ایمان و مؤمن تو
\\
\end{longtable}
\end{center}
