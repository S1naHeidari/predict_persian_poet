\begin{center}
\section*{غزل شماره ۱۶۳۹: ای خوشا روز که پیش چو تو سلطان میرم}
\label{sec:1639}
\addcontentsline{toc}{section}{\nameref{sec:1639}}
\begin{longtable}{l p{0.5cm} r}
ای خوشا روز که پیش چو تو سلطان میرم
&&
پیش کان شکر تو شکرافشان میرم
\\
صد هزاران گل صدبرگ ز خاکم روید
&&
چونک در سایه آن سرو گلستان میرم
\\
ای بسا دست که خایند حریصان حیات
&&
چونک در پای تو من دست فشانان میرم
\\
شربت مرگ چو اندر قدح من ریزی
&&
بر قدح بوسه دهم مست و خرامان میرم
\\
چون به بوی خوش یک سیب تو موسی جان داد
&&
پس عجب نیست کز آسیب تو چون جان میرم
\\
چون خزان از خبر مرگ اگر زرد شوم
&&
چون بهار از لب خندان تو خندان میرم
\\
بارها مردم من وز دم تو زنده شدم
&&
گر بمیرم ز تو صد بار بدان سان میرم
\\
من پراکنده بدم خاک بدم جمع شدم
&&
پیش جمع تو نشاید که پریشان میرم
\\
همچو فرزند که اندر بر مادر میرد
&&
در بر رحمت و بخشایش رحمان میرم
\\
چه حدیث است کجا مرگ بود عاشق را
&&
این محالت که در چشمه حیوان میرم
\\
شمس تبریز کسانی که به تو زنده نیند
&&
سوی تو زنده شوم از سوی ایشان میرم
\\
\end{longtable}
\end{center}
