\begin{center}
\section*{غزل شماره ۳۸۷: خاک آن کس شو که آب زندگانش روشنست}
\label{sec:0387}
\addcontentsline{toc}{section}{\nameref{sec:0387}}
\begin{longtable}{l p{0.5cm} r}
خاک آن کس شو که آب زندگانش روشنست
&&
نیم نانی دررسد تا نیم جانی در تنست
\\
گفتمش آخر پی یک وصل چندین هجر چیست
&&
گفت آری من قصابم گردران با گردنست
\\
دی تماشا رفته بودم جانب صحرای دل
&&
آن نگنجد در نظر چه جای پیدا کردنست
\\
چشم مست یار گویان هر زمان با چشم من
&&
در دو عالم می‌نگنجد آنچ در چشم منست
\\
رو فزون شو از دو عالم تا بریزم بر سرت
&&
آنچ دل را جان جان و دیدگان را دیدنست
\\
ذره ذره عاشقانه پهلوی معشوق خویش
&&
می‌زند پهلو که وقت عقد و کابین کردنست
\\
اندر آن پیوند کردن آب و آتش یک شده‌ست
&&
غنچه آن جا سنبلست و سرو آن جا سوسنست
\\
زیر پاشان گنج‌ها و سوی بالا باغ‌ها
&&
بشنو از بالا نه وقت زیر و بالا گفتنست
\\
من اگر پیدا نگویم بی‌صفت پیداست آن
&&
ذوق آن اندر سرست و طوق آن در گردنست
\\
شمس تبریزی تو خورشیدی چه گویم مدح تو
&&
صد زبان دارم چو تیغ اما به وصفت الکنست
\\
\end{longtable}
\end{center}
