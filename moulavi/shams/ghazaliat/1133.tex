\begin{center}
\section*{غزل شماره ۱۱۳۳: نه در وفات گذارد نه در جفا دلدار}
\label{sec:1133}
\addcontentsline{toc}{section}{\nameref{sec:1133}}
\begin{longtable}{l p{0.5cm} r}
نه در وفات گذارد نه در جفا دلدار
&&
نه منکرت بگذارد نه بر سر اقرار
\\
به هر کجا که نهی دل به قهر برکندت
&&
به هیچ جای منه دل دلا و پا مفشار
\\
به شب قرار نهی روز آن بگرداند
&&
بگیر عبرت از این اختلاف لیل و نهار
\\
ز جهل توبه و سوگند می‌تند غافل
&&
چه حیله دارد مقهور در کف قهار
\\
برادرا سر و کار تو با کی افتادست
&&
کز اوست بی‌سر و پا گشته گنبد دوار
\\
برادرا تو کجا خفته‌ای نمی‌دانی
&&
که بر سر تو نشستست افعی بیدار
\\
چه خواب‌هاست که می‌بینی ای دل مغرور
&&
چه دیگ بهر تو پختست پیر خوان سلار
\\
هزار تاجر بر بوی سود شد به سفر
&&
ببرد دمدمه حکم حق ز جانش قرار
\\
چنانش کرد که در شهرها نمی‌گنجید
&&
ملول شد ز بیابان و رفت سوی بحار
\\
رود که گیرد مرجان ولیک بدهد جان
&&
که در کمین بنشستست بر رهش جرار
\\
دوید در پی آب و نیافت غیر سراب
&&
دوید در پی نور و نیافت الا نار
\\
قضا گرفته دو گوشش کشان کشان که بیا
&&
چنین کشند به سوی جوال گوش حمار
\\
بتر ز گاوی کاین چرخ را نمی‌بینی
&&
که گردن تو ببستست از برای دوار
\\
در این دوار طبیبان همه گرفتارند
&&
کز این دوار بود مست کله بیمار
\\
به بر و بحر و به دشت و به کوه می‌کشدش
&&
که تا کجاش دراند به پنجه شیر شکار
\\
ولیک عاشق حق را چو بردراند شیر
&&
هلا دریدن او را چو دیگران مشمار
\\
دل و جگر چو نیابد درونه تن او
&&
همان کسی که دریدش همو شود معمار
\\
چو در حیات خود او کشته گشت در کف عشق
&&
به امر موتوا من قبل ان تموتوا زار
\\
که بی‌دلست و جگرخون عاشقست یقین
&&
شکار را ندرانید هیچ شیر دو بار
\\
وگر درید به سهوش بدوزدش در حال
&&
در او دمد دم جان و بگیردش به کنار
\\
حرام کرد خدا شحم و لحم عاشق را
&&
که تا طمع نکند در فناش مردم خوار
\\
تو عشق نوش که تریاق خاک فاروقیست
&&
که زهر زهره ندارد که دم زند ز ضرار
\\
سخن رسید به عشق و همی‌جهد دل من
&&
کجا جهد ز چنین زخم بی‌محابا تار
\\
چو قطب می‌نجهد از میان دور فلک
&&
کجا جهد تو بگو نقطه از چنین پرگار
\\
خموش باش که این هم کشاکش قدرست
&&
تو را به شعر و به اطلس مرا سوی اشعار
\\
\end{longtable}
\end{center}
