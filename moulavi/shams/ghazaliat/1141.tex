\begin{center}
\section*{غزل شماره ۱۱۴۱: ز بامداد چه دشمن کشست دیدن یار}
\label{sec:1141}
\addcontentsline{toc}{section}{\nameref{sec:1141}}
\begin{longtable}{l p{0.5cm} r}
ز بامداد چه دشمن کشست دیدن یار
&&
بشارتیست ز عمر عزیز روی نگار
\\
ز خواب برجهی و روی یار را بینی
&&
زهی سعادت و اقبال و دولت بیدار
\\
همو گشاید کار و همو بگوید شکر
&&
چنان بود که گلی رست بی‌قرینه خار
\\
چو دست بر تو نهد یار و گویدت برخیز
&&
زهی قیامت و جنات و تحتها الانهار
\\
بگو به موسی عمران که شد همه دیده
&&
که نعره ارنی خیزد از دم دیدار
\\
برای مغلطه می‌دید و دیدنش می‌جست
&&
زهی مقام تجلی و آفتاب مدار
\\
ز بامداد چو افیون فضل او خوردیم
&&
برون شدیم ز عقل و برآمدیم ز کار
\\
ببین تو حال مرا و مرا ز حال مپرس
&&
چو عقل اندک داری برو مگو بسیار
\\
برو مگوی جنون را ز کوره معقولات
&&
که صد دریغ که دیوانه گشته‌ای یک بار
\\
مرا در این شب دولت ز جفت و طاق مپرس
&&
که باده جفت دماغست و یار جفت کنار
\\
مرا مپرس عزیزا که چند می‌گردی
&&
که هیچ نقطه نپرسد ز گردش پرگار
\\
غبار و گرد مینگیز در ره یاری
&&
که او به حسن ز دریا برآورید غبار
\\
منه تو بر سر زانو سر خود ای صوفی
&&
کز این تو پی نبری گر فروروی بسیار
\\
چو هیچ کوه احد برنیامد از بن و بیخ
&&
چه دست درزده‌ای در کمرگه کهسار
\\
در آن زمان که عسل‌های فقر می‌لیسیم
&&
به چشم ما مگسی می‌شود سپه سالار
\\
چه ایمنست دهم از خراج و نعل بها
&&
چو نعل ماست در آتش ز عشق تیزشرار
\\
\end{longtable}
\end{center}
