\begin{center}
\section*{غزل شماره ۱۱۲۰: مستیم و بیخودیم و جمال تو پرده در}
\label{sec:1120}
\addcontentsline{toc}{section}{\nameref{sec:1120}}
\begin{longtable}{l p{0.5cm} r}
مستیم و بیخودیم و جمال تو پرده در
&&
زین پس مباش ماها در ابر و پرده در
\\
ما جمع عاشقان تو خوش قد و قامتیم
&&
ما را صلای فتنه و شور و هزار شر
\\
خورشید تافتست ز روی تو چاشتگاه
&&
در عشق قرص روی تو رفتیم بام بر
\\
مستیست در سر از می و این تاب آفتاب
&&
در سر بتافتست پس از دست رفت سر
\\
ای مطرب هوای دل عاشقان روح
&&
بنواز لحن جان که تننتن لطیفتر
\\
تا جان‌ها ز خرقه تن‌ها برون شود
&&
تا بر سرین خرقه رود جان باخبر
\\
از جام صاف باده تو خاشاک جسم را
&&
بردار تا نهیم به اقبال بر به بر
\\
تا دیده‌ها گذاره شود از حجاب‌ها
&&
تا وارهد ز خانه و مان و ز بام و در
\\
سیمرغ جان و مفخر تبریز شمس دین
&&
بیند هزار روضه و یابد هزار پر
\\
\end{longtable}
\end{center}
