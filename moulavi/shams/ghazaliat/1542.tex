\begin{center}
\section*{غزل شماره ۱۵۴۲: چنان مستم چنان مستم من این دم}
\label{sec:1542}
\addcontentsline{toc}{section}{\nameref{sec:1542}}
\begin{longtable}{l p{0.5cm} r}
چنان مستم چنان مستم من این دم
&&
که حوا را بنشناسم ز آدم
\\
ز شور من بشوریده‌ست دریا
&&
ز سرمستی من مست است عالم
\\
زهی سر ده که سر ببریده جلاد
&&
که تا دنیا نبیند هیچ ماتم
\\
حلال اندر حلال اندر حلال است
&&
می خنب خدا نبود محرم
\\
از این باده جوان گر خورده بودی
&&
نبودی پشت پیر چرخ را خم
\\
زمین ار خورده بودی فارغستی
&&
از آن که ابر تر بارد بر او نم
\\
دل بی‌عقل شرح این بگفتی
&&
اگر بودی به عالم نیم محرم
\\
ز آب و گل برون بردی شما را
&&
اگر بودی شما را پای محکم
\\
\end{longtable}
\end{center}
