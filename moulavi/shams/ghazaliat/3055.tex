\begin{center}
\section*{غزل شماره ۳۰۵۵: بیا بیا که نیابی چو ما دگر یاری}
\label{sec:3055}
\addcontentsline{toc}{section}{\nameref{sec:3055}}
\begin{longtable}{l p{0.5cm} r}
بیا بیا که نیابی چو ما دگر یاری
&&
چو ما به هر دو جهان خود کجاست دلداری
\\
بیا بیا و به هر سوی روزگار مبر
&&
که نیست نقد تو را پیش غیر بازاری
\\
تو همچو وادی خشکی و ما چو بارانی
&&
تو همچو شهر خرابی و ما چو معماری
\\
به غیر خدمت ما که مشارق شادیست
&&
ندید خلق و نبیند ز شادی آثاری
\\
هزار صورت جنبان به خواب می‌بینی
&&
چو خواب رفت نبینی ز خلق دیاری
\\
ببند چشم خر و برگشای چشم خرد
&&
که نفس همچو خر افتاد و حرص افساری
\\
ز باغ عشق طلب کن عقیده شیرین
&&
که طبع سرکه فروشست و غوره افشاری
\\
بیا به جانب دارالشفای خالق خویش
&&
کز آن طبیب ندارد گریز بیماری
\\
جهان مثال تن بی‌سرست بی‌آن شاه
&&
بپیچ گرد چنان سر مثال دستاری
\\
اگر سیاه نه‌ای آینه مده از دست
&&
که روح آینه توست و جسم زنگاری
\\
کجاست تاجر مسعود مشتری طالع
&&
که گرمدار منش باشم و خریداری
\\
بیا و فکرت من کن که فکرتت دادم
&&
چو لعل می‌خری از کان من بخر باری
\\
به پای جانب آن کس برو که پایت داد
&&
بدو نگر به دو دیده که داد دیداری
\\
دو کف به شادی او زن که کف ز بحر ویست
&&
که نیست شادی او را غمی و تیماری
\\
تو بی‌ز گوش شنو بی‌زبان بگو با او
&&
که نیست گفت زبان بی‌خلاف و آزاری
\\
\end{longtable}
\end{center}
