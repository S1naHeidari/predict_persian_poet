\begin{center}
\section*{غزل شماره ۱۸۸۴: آن ساعد سیمین را در گردن ما افکن}
\label{sec:1884}
\addcontentsline{toc}{section}{\nameref{sec:1884}}
\begin{longtable}{l p{0.5cm} r}
آن ساعد سیمین را در گردن ما افکن
&&
بر سینه ما بنشین ای جان منت مسکن
\\
سرمست شدم ای جان وز دست شدم ای جان
&&
ای دوست خمارم را از لعل لبت بشکن
\\
ای ساقی هر نادر این می ز چه خم داری
&&
من بنده ظلم تو از بیخ و بنم برکن
\\
هم پرده من می در هم خون دلم می خور
&&
آخر نه تویی با من شاباش زهی ای من
\\
از دوست ستم نبود بر مست قلم نبود
&&
جز عفو و کرم نبود بر مست چنین مسکن
\\
از معدن خویش ای جان بخرام در این میدان
&&
رونق نبود زر را تا باشد در معدن
\\
با لعل چو تو کانی غمگین نشود جانی
&&
در گور و کفن ناید تا باشد جان در تن
\\
\end{longtable}
\end{center}
