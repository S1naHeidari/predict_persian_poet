\begin{center}
\section*{غزل شماره ۲۳۵: مرا بدید و نپرسید آن نگار چرا}
\label{sec:0235}
\addcontentsline{toc}{section}{\nameref{sec:0235}}
\begin{longtable}{l p{0.5cm} r}
مرا بدید و نپرسید آن نگار چرا
&&
ترش ترش بگذشت از دریچه یار چرا
\\
سبب چه بود چه کردم که بد نمود ز من
&&
که خاطرش بگرفتست این غبار چرا
\\
ز بامداد چرا قصد خون عاشق کرد
&&
چرا کشید چنین تیغ ذوالفقار چرا
\\
چو دیدم آن گل او را که رنگ ریخته بود
&&
دمید از دل مسکین هزار خار چرا
\\
چو لب به خنده گشاید گشاده گردد دل
&&
در آن لبست همیشه گشاد کار چرا
\\
میان ابروی خود چون گره زند از خشم
&&
گره گره شود از غم دل فکار چرا
\\
زهی تعلق جان با گشاد و خنده او
&&
یکی دمش که نبینم شوم نزار چرا
\\
جهان سیه شود آن دم که رو بگرداند
&&
نه روز ماند و نی عقل برقرار چرا
\\
یکی نفس که دل یار ما ز ما برمید
&&
چرا رمید ز ما لطف کردگار چرا
\\
مگر که لطف خدا اوست ما غلط کردیم
&&
وگر نه خوبی او گشت بی‌کنار چرا
\\
برون صورت اگر لطف محض دادی روی
&&
پیمبران ز چه گشتند پرده دار چرا
\\
\end{longtable}
\end{center}
