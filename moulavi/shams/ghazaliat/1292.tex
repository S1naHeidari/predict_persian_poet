\begin{center}
\section*{غزل شماره ۱۲۹۲: آمد آن خواجه سیماترش}
\label{sec:1292}
\addcontentsline{toc}{section}{\nameref{sec:1292}}
\begin{longtable}{l p{0.5cm} r}
آمد آن خواجه سیماترش
&&
وان شکرش گشته چو سرکا ترش
\\
با همگان روترش است ای عجب
&&
یا که به بیرون خوش و با ما ترش
\\
از کرم خواجه روا نیست این
&&
با همه خوش با من تنها ترش
\\
زین بگذشتیم دریغست و حیف
&&
آن رخ خوش طلعت زیبا ترش
\\
ای ز تو خندان شده هر جا حزین
&&
وی ز تو شیرین شده هر جا ترش
\\
شاد زمانی که نهان زیر لب
&&
یار همی‌خندد و لالا ترش
\\
گر ترشی این دم شرطی بنه
&&
که نبود روی تو فردا ترش
\\
بهر خدا قاعده نو منه
&&
هیچ بود قاعده حلوا ترش
\\
این ترشی در چه و زندان بود
&&
دید کسی باغ و تماشا ترش
\\
یوسف خوبان چو به زندان بماند
&&
هیچ نگشت آن گل رعنا ترش
\\
تا به سخن آمد دیوار و در
&&
کز چه نه‌ای ای شه و مولا ترش
\\
گفت اگر غرقه سرکا شوم
&&
کی هلدم رحمت بالا ترش
\\
می‌دهم عشق و ندیمی کند
&&
غرقه شود در می و صهبا ترش
\\
دست فشان روح رود مست تا
&&
میمنه که نیست بدان جا ترش
\\
بس کن و در شهد و شکر غوطه خور
&&
کت نهلد فضل موفا ترش
\\
\end{longtable}
\end{center}
