\begin{center}
\section*{غزل شماره ۱۲۲۹: زلفی که به جان ارزد هر تار بشوریدش}
\label{sec:1229}
\addcontentsline{toc}{section}{\nameref{sec:1229}}
\begin{longtable}{l p{0.5cm} r}
زلفی که به جان ارزد هر تار بشوریدش
&&
بس مشک نهان دارد زنهار بشوریدش
\\
در شام دو زلف او صد صبح نهان بیشست
&&
هر لحظه و هر ساعت صد بار بشوریدش
\\
آن دولت عالم را وان جنت خرم را
&&
کز وی شکفد در جان گلزار بشوریدش
\\
آن باده همی‌جوشد وز خلق همی‌پوشد
&&
تا روی شود از وی خمار بشوریدش
\\
چشم و دل مریم شد روشن از آن خرما
&&
نخلیست از آن خرما پربار بشوریدش
\\
گم گشت دل مسکین اندر خم زلف او
&&
باشد که بدید آید بسیار بشوریدش
\\
شمس الحق تبریزی در عشق مسیح آمد
&&
هر کس که از او دارد زنار بشوریدش
\\
\end{longtable}
\end{center}
