\begin{center}
\section*{غزل شماره ۹۴۲: اگر مرا تو نخواهی دلم تو را خواهد}
\label{sec:0942}
\addcontentsline{toc}{section}{\nameref{sec:0942}}
\begin{longtable}{l p{0.5cm} r}
اگر مرا تو نخواهی دلم تو را خواهد
&&
تو هم به صلح گرایی اگر خدا خواهد
\\
هزار عاشق داری تو را به جان جویان
&&
که تا سعادت و دولت ز ما که را خواهد
\\
ز عشق عاشق درویش خلق در عجبند
&&
که آنچ رشک شهانست او چرا خواهد
\\
عجب نباشد اگر مرده‌ای بجوید جان
&&
و یا گیاه بپژمرده‌ای صبا خواهد
\\
و یا دو دیده کور از خدا بصر جوید
&&
و یا گرسنه ده ساله‌ای نوا خواهد
\\
همه دعا شده‌ام من ز بس دعا کردن
&&
که هر که بیند رویم ز من دعا خواهد
\\
ولی به چشم تو من رنگ کافران دارم
&&
که چشم خیره کشت بیندم غزا خواهد
\\
اگر مرا نکشد هجر تو ز من بحلست
&&
اسیر کشته ز غازی چه خونبها خواهد
\\
سلام و خدمت کردم بگفتیم چونی
&&
چنان بود مس مسکین که کیمیا خواهد
\\
چنان برآید صورت که بست صورتگر
&&
چنان بود تن خسته کیش دوا خواهد
\\
ز آفتاب مزن گفت و گوی چون سایه
&&
ز سایه ذره گریزد همه ضیا خواهد
\\
زهی سخاوت و ایثار شمس تبریزی
&&
که شمس گنبد خضرا از او عطا خواهد
\\
\end{longtable}
\end{center}
