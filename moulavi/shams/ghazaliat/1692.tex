\begin{center}
\section*{غزل شماره ۱۶۹۲: یا رب چه یار دارم شیرین شکار دارم}
\label{sec:1692}
\addcontentsline{toc}{section}{\nameref{sec:1692}}
\begin{longtable}{l p{0.5cm} r}
یا رب چه یار دارم شیرین شکار دارم
&&
در سینه از نی او صد مرغزار دارم
\\
قاصد به خشم آید چون سوی من گراید
&&
گوید کجا گریزی من با تو کار دارم
\\
من دوش ماه نو را پرسیدم از مه خود
&&
گفتا پیش دوانم پا در غبار دارم
\\
خورشید چون برآمد گفتم چه زردرویی
&&
گفتا ز شرم رویش رنگ نضار دارم
\\
ای آب در سجودی بر روی و سر دوانی
&&
گفتا که از فسونش رفتار مار دارم
\\
ای میرداد آتش پیچان چنین چرایی
&&
گفتا ز برق رویش دل بی‌قرار دارم
\\
ای باد پیک عالم تو دل سبک چرایی
&&
گفتا بسوزد این دل گر اختیار دارم
\\
ای خاک در چه فکری خاموشی و مراقب
&&
گفتا که در درونه باغ و بهار دارم
\\
بگذر از این عناصر ما را خداست ناصر
&&
در سر خمار دارم در کف عقار دارم
\\
گر خواب ما ببستی بازست راه مستی
&&
می دردهد دودستی چون دستیار دارم
\\
خاموش باش تا دل بی‌این زبان بگوید
&&
چون گفت دل نیوشم زین گفت عار دارم
\\
\end{longtable}
\end{center}
