\begin{center}
\section*{غزل شماره ۵۷۱: بیا کامشب به جان بخشی به زلف یار می‌ماند}
\label{sec:0571}
\addcontentsline{toc}{section}{\nameref{sec:0571}}
\begin{longtable}{l p{0.5cm} r}
بیا کامشب به جان بخشی به زلف یار می‌ماند
&&
جمال ماه نورافشان بدان رخسار می‌ماند
\\
به گرد چرخ استاره چو مشتاقان آواره
&&
که از سوز دل ایشان خرد از کار می‌ماند
\\
سقای روح یک باده ز جام غیب درداده
&&
ببین تا کیست افتاده و کی بیدار می‌ماند
\\
به شب نالان و بیداران نیابی جز که بیماران
&&
و من گر هم نمی‌نالم دلم بیمار می‌ماند
\\
در این دریای بی‌مونس دلا می‌نال چون یونس
&&
نهنگ شب در این دریا به مردم خوار می‌ماند
\\
بدان سان می‌خورد ما را ز خاص و عام اندر شب
&&
نه دکان و نه سودا و نه این بازار می‌ماند
\\
چه شد ناصر عبادالله چه شد حافظ بلادالله
&&
ببین جز مبدع جان‌ها اگر دیار می‌ماند
\\
فلک بازار کیوانست در او استاره گردان است
&&
شب ما روز ایشانست که بی‌اغیار می‌ماند
\\
جز این چرخ و زمین در جان عجب چرخیست و بازاری
&&
ولیک از غیرت آن بازار در اسرار می‌ماند
\\
\end{longtable}
\end{center}
