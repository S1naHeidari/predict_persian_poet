\begin{center}
\section*{غزل شماره ۳۰۴۵: من آن نیم که تو دیدی چو بینیم نشناسی}
\label{sec:3045}
\addcontentsline{toc}{section}{\nameref{sec:3045}}
\begin{longtable}{l p{0.5cm} r}
من آن نیم که تو دیدی چو بینیم نشناسی
&&
تو جز خیال نبینی که مست خواب و نعاسی
\\
مرا بپرس که چونی در این کمی و فزونی
&&
چگونه باشد یوسف به دست کور نخاسی
\\
به چشم عشق توان دید روی یوسف جان را
&&
تو چشم عشق نداری تو مرد وهم و قیاسی
\\
بهای نعمت دیده سپاس و شکر خدا دان
&&
مرم چو قلب ز کوره که کان شکر و سپاسی
\\
وگر ز کوره بترسی یقین خیال پرستی
&&
بت خیال تراشی وزان خیال هراسی
\\
بت خیال تو سازی به پیش بت به نمازی
&&
چو گبر اسیر بتانی چو زن حریف نفاسی
\\
خیال فرع تو باشد که فرع فرع تو را شد
&&
تو مه نه‌ای تو غباری تو زر نه‌ای تو نحاسی
\\
به جان جمله مردان اگر چه جمله یکی اند
&&
که زیر چرخه گردون تنا چو گاو خراسی
\\
وگر ز چنبر گردون برون کشی سر و گردن
&&
ز خرگله برهیدی فرشته‌ای و ز ناسی
\\
\end{longtable}
\end{center}
