\begin{center}
\section*{غزل شماره ۳۰۰۳: ای کاشکی تو خویش زمانی بدانیی}
\label{sec:3003}
\addcontentsline{toc}{section}{\nameref{sec:3003}}
\begin{longtable}{l p{0.5cm} r}
ای کاشکی تو خویش زمانی بدانیی
&&
وز روی خوب خویشت بودی نشانیی
\\
در آب و گل تو همچو ستوران نخفتیی
&&
خود را به عیش خانه خوبان کشانیی
\\
بر گرد خویش گشتی کاظهار خود کنی
&&
پنهان بماند زیر تو گنج نهانیی
\\
از روح بی‌خبر بدیی گر تو جسمیی
&&
در جان قرار داشتیی گر تو جانیی
\\
با نیک و بد بساختیی همچو دیگران
&&
با این و آنیی تو اگر این و آنیی
\\
یک ذوق بودیی تو اگر یک اباییی
&&
یک نوع جوشییی چو یکی قازغانیی
\\
زین جوش در دوار اگر صاف گشتیی
&&
چون صاف گشتگان تو بر این آسمانیی
\\
گویی به هر خیال که جان و جهان من
&&
گر گم شدی خیال تو جان و جهانیی
\\
بس کن که بند عقل شدست این زبان تو
&&
ور نی چو عقل کلی جمله زبانیی
\\
بس کن که دانش‌ست که محجوب دانشست
&&
دانستیی که شاهی کی ترجمانیی
\\
\end{longtable}
\end{center}
