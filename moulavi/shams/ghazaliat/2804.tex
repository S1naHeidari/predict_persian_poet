\begin{center}
\section*{غزل شماره ۲۸۰۴: چون تو آن روبند را از روی چون مه برکنی}
\label{sec:2804}
\addcontentsline{toc}{section}{\nameref{sec:2804}}
\begin{longtable}{l p{0.5cm} r}
چون تو آن روبند را از روی چون مه برکنی
&&
چون قضای آسمانی توبه‌ها را بشکنی
\\
منگر اندر شور و بدمستی من ای نیک عهد
&&
بنگر آخر در میی کاندر سرم می‌افکنی
\\
اول از دست فراقت عاشقان را تی کنی
&&
وآنگه اندر پوستشان تا سر همه در زر کنی
\\
مه رخا سیمرغ جانی منزل تو کوه قاف
&&
از تو پرسیدن چه حاجت کز کدامین مسکنی
\\
چون کلام تو شنید از بخت نفس ناطقه
&&
کرد صد اقرار بر خود بهر جهل و الکنی
\\
چون ز غیر شمس تبریزی بریدی ای بدن
&&
در حریر و در زر و در دیبه و در ادکنی
\\
\end{longtable}
\end{center}
