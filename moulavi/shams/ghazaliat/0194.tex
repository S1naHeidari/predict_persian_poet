\begin{center}
\section*{غزل شماره ۱۹۴: خواهم گرفتن اکنون آن مایه صور را}
\label{sec:0194}
\addcontentsline{toc}{section}{\nameref{sec:0194}}
\begin{longtable}{l p{0.5cm} r}
خواهم گرفتن اکنون آن مایه صور را
&&
دامی نهاده‌ام خوش آن قبله نظر را
\\
دیوار گوش دارد آهسته‌تر سخن گو
&&
ای عقل بام بررو ای دل بگیر در را
\\
اعدا که در کمینند در غصه همینند
&&
چون بشنوند چیزی گویند همدگر را
\\
گر ذره‌ها نهانند خصمان و دشمنانند
&&
در قعر چه سخن گو خلوت گزین سحر را
\\
ای جان چه جای دشمن روزی خیال دشمن
&&
در خانه دلم شد از بهر رهگذر را
\\
رمزی شنید زین سر زو پیش دشمنان شد
&&
می‌خواند یک به یک را می‌گفت خشک و تر را
\\
زان روز ما و یاران در راه عهد کردیم
&&
پنهان کنیم سر را پیش افکنیم سر را
\\
ما نیز مردمانیم نی کم ز سنگ کانیم
&&
بی زخم‌های میتین پیدا نکرد زر را
\\
دریای کیسه بسته تلخ و ترش نشسته
&&
یعنی خبر ندارم کی دیده‌ام گهر را
\\
\end{longtable}
\end{center}
