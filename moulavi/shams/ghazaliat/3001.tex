\begin{center}
\section*{غزل شماره ۳۰۰۱: ای نای بس خوش است کز اسرار آگهی}
\label{sec:3001}
\addcontentsline{toc}{section}{\nameref{sec:3001}}
\begin{longtable}{l p{0.5cm} r}
ای نای بس خوش است کز اسرار آگهی
&&
کار او کند که دارد از کار آگهی
\\
ای نای همچو بلبل نالان آن گلی
&&
گردن مخار کز گل بی‌خار آگهی
\\
گفتم به نای همدم یاری مدزد راز
&&
گفتا هلاک توست به یک بار آگهی
\\
گفتم خلاص من به هلاک من اندر است
&&
آتش بنه بسوز بمگذار آگهی
\\
گفتا چگونه رهزن این قافله شوم
&&
دانم که هست قافله سالار آگهی
\\
گفتم چو یار گم شدگان را نمی‌نواخت
&&
از آگهی همی‌شد بیزار آگهی
\\
نه چشم گشته‌ای تو که بی‌آگهی ز خویش
&&
ما را حجاب دیده و دیدار آگهی
\\
زان همدم لبی که تو را سر بریده‌اند
&&
ای ننگ سر در این ره و ای عار آگهی
\\
از خود تهی شدی و ز اسرار پر شدی
&&
زیرا ز خودپرست و ز انکار آگهی
\\
چون می‌چشی ز لعل لب یار ناله چیست
&&
بگذار تا کند گله‌ای زار آگهی
\\
نی نی ز بهر خود تو نمی‌نالی ای کریم
&&
بگری بر آنک دارد ز اغیار آگهی
\\
گردون اگر بنالد گاو است زیر بار
&&
زین نعل بازگونه غلط کار آگهی
\\
\end{longtable}
\end{center}
