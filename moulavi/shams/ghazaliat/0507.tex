\begin{center}
\section*{غزل شماره ۵۰۷: کیست که او بنده رای تو نیست}
\label{sec:0507}
\addcontentsline{toc}{section}{\nameref{sec:0507}}
\begin{longtable}{l p{0.5cm} r}
کیست که او بنده رای تو نیست
&&
کیست که او مست لقای تو نیست
\\
غصه کشی کو که ز خوف تو نیست
&&
یا طربی کان ز رجای تو نیست
\\
بخل کفی کو که ز قبض تو نیست
&&
یا کرمی کان ز عطای تو نیست
\\
لعل لبی کو که ز کان تو نیست
&&
محتشمی کو که گدای تو نیست
\\
متصل اوصاف تو با جان‌ها
&&
یک رگ بی‌بند و گشای تو نیست
\\
هر دو جهان چون دو کف و تو چو جان
&&
کف چه دهد کان ز سخای تو نیست
\\
چشم کی دیدست در این باغ کون
&&
رقص گلی کان ز هوای تو نیست
\\
غافل ناله کند از جور خلق
&&
خلق به جز شبه عصای تو نیست
\\
جنبش این جمله عصاها ز توست
&&
هر یک جز درد و دوای تو نیست
\\
زخم معلم زند آن چوب کیست
&&
کیست که او بند قضای تو نیست
\\
همچو سگان چوب تو را می‌گزند
&&
در سرشان فهم جزای تو نیست
\\
دفع بلای تن و آزار خلق
&&
جز به مناجات و ثنای تو نیست
\\
بشکنی این چوب نه چوبش کمست
&&
دفع دو سه چوب رهای تو نیست
\\
صاحب حوت از غم امت گریخت
&&
جان به کجا برد که جای تو نیست
\\
بس کن وز محنت یونس بترس
&&
با قدر استیزه به پای تو نیست
\\
\end{longtable}
\end{center}
