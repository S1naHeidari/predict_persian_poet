\begin{center}
\section*{غزل شماره ۱۷۵۹: اه چه بی‌رنگ و بی‌نشان که منم}
\label{sec:1759}
\addcontentsline{toc}{section}{\nameref{sec:1759}}
\begin{longtable}{l p{0.5cm} r}
اه چه بی‌رنگ و بی‌نشان که منم
&&
کی ببینم مرا چنان که منم
\\
گفتی اسرار در میان آور
&&
کو میان اندر این میان که منم
\\
کی شود این روان من ساکن
&&
این چنین ساکن روان که منم
\\
بحر من غرقه گشت هم در خویش
&&
بوالعجب بحر بی‌کران که منم
\\
این جهان و آن جهان مرا مطلب
&&
کاین دو گم شد در آن جهان که منم
\\
فارغ از سودم و زیان چو عدم
&&
طرفه بی‌سود و بی‌زیان که منم
\\
گفتم ای جان تو عین مایی گفت
&&
عین چه بود در این عیان که منم
\\
گفتم آنی بگفت‌های خموش
&&
در زبان نامده‌ست آن که منم
\\
گفتم اندر زبان چو درنامد
&&
اینت گویای بی‌زبان که منم
\\
می شدم در فنا چو مه بی‌پا
&&
اینت بی‌پای پادوان که منم
\\
بانگ آمد چه می دوی بنگر
&&
در چنین ظاهر نهان که منم
\\
شمس تبریز را چو دیدم من
&&
نادره بحر و گنج و کان که منم
\\
\end{longtable}
\end{center}
