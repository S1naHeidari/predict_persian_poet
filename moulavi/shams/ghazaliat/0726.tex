\begin{center}
\section*{غزل شماره ۷۲۶: شب رفت حریفکان کجایید}
\label{sec:0726}
\addcontentsline{toc}{section}{\nameref{sec:0726}}
\begin{longtable}{l p{0.5cm} r}
شب رفت حریفکان کجایید
&&
شب تا برود شما بیایید
\\
از لعل لبش شراب نوشید
&&
وز خنده او شکر بخایید
\\
چون روز شود به هوشیاران
&&
زین باده نشانه وانمایید
\\
در جیب شما چو دردمیدند
&&
عیسی زایید اگر بزایید
\\
بی هشت بهشت و هفت دوزخ
&&
همچون مه چهارده برآیید
\\
یک موی ز هفت و هشت گر هست
&&
این خلوت خاص را نشایید
\\
مویی در چشم نیست اندک
&&
زنهار که سرمه‌ای بسایید
\\
چون چشم ز موی پاک گردد
&&
در عشق چو چشم پیشوایید
\\
در عشق خدیو شمس تبریز
&&
انصاف که بی‌شما شمایید
\\
\end{longtable}
\end{center}
