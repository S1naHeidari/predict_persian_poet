\begin{center}
\section*{غزل شماره ۲۵۰۸: مرا آن دلبر پنهان همی‌گوید به پنهانی}
\label{sec:2508}
\addcontentsline{toc}{section}{\nameref{sec:2508}}
\begin{longtable}{l p{0.5cm} r}
مرا آن دلبر پنهان همی‌گوید به پنهانی
&&
به من ده جان به من ده جان چه باشد این گران جانی
\\
یکی لحظه قلندر شو قلندر را مسخر شو
&&
سمندر شو سمندر شو در آتش رو به آسانی
\\
در آتش رو در آتش رو در آتشدان ما خوش رو
&&
که آتش با خلیل ما کند رسم گلستانی
\\
نمی‌دانی که خار ما بود شاهنشه گل‌ها
&&
نمی‌دانی که کفر ما بود جان مسلمانی
\\
سراندازان سراندازان سراندازی سراندازی
&&
مسلمانان مسلمانان مسلمانی مسلمانی
\\
خداوندا تو می‌دانی که صحرا از قفس خوشتر
&&
ولیکن جغد نشکیبد ز گورستان ویرانی
\\
کنون دوران جان آمد که دریا را درآشامد
&&
زهی دوران زهی حلقه زهی دوران سلطانی
\\
خمش چون نیست پوشیده فقیر باده نوشیده
&&
که هست اندر رخش پیدا فر و انوار سبحانی
\\
\end{longtable}
\end{center}
