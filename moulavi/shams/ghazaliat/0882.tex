\begin{center}
\section*{غزل شماره ۸۸۲: جامه سیه کرد کفر نور محمد رسید}
\label{sec:0882}
\addcontentsline{toc}{section}{\nameref{sec:0882}}
\begin{longtable}{l p{0.5cm} r}
جامه سیه کرد کفر نور محمد رسید
&&
طبل بقا کوفتند ملک مخلد رسید
\\
روی زمین سبز شد جیب درید آسمان
&&
بار دگر مه شکافت روح مجرد رسید
\\
گشت جهان پرشکر بست سعادت کمر
&&
خیز که بار دگر آن قمرین خد رسید
\\
دل چو سطرلاب شد آیت هفت آسمان
&&
شرح دل احمدی هفت مجلد رسید
\\
عقل معقل شبی شد بر سلطان عشق
&&
گفت به اقبال تو نفس مقید رسید
\\
پیک دل عاشقان رفت به سر چون قلم
&&
مژده همچون شکر در دل کاغد رسید
\\
چند کند زیر خاک صبر روان‌های پاک
&&
هین ز لحد برجهید نصر مؤید رسید
\\
طبل قیامت زدند صور حشر می‌دمد
&&
وقت شد ای مردگان حشر مجدد رسید
\\
بعثر ما فی القبور حصل ما فی الصدور
&&
آمد آواز صور روح به مقصد رسید
\\
دوش در استارگان غلغله افتاده بود
&&
کز سوی نیک اختران اختر اسعد رسید
\\
رفت عطارد ز دست لوح و قلم درشکست
&&
در پی او زهره جست مست به فرقد رسید
\\
قرص قمر رنگ ریخت سوی اسد می‌گریخت
&&
گفتم خیرست گفت ساقی بیخود رسید
\\
عقل در آن غلغله خواست که پیدا شود
&&
کودک هم کودکست گو چه به ابجد رسید
\\
خیز که دوران ماست شاه جهان آن ماست
&&
چون نظرش جان ماست عمر مؤبد رسید
\\
ساقی بی‌رنگ و لاف ریخت شراب از گزاف
&&
رقص جمل کرد قاف عیش ممدد رسید
\\
باز سلیمان روح گفت صلای صبوح
&&
فتنه بلقیس را صرح ممرد رسید
\\
رغم حسودان دین کوری دیو لعین
&&
کحل دل و دیده در چشم مرمد رسید
\\
از پی نامحرمان قفل زدم بر دهان
&&
خیز بگو مطربا عشرت سرمد رسید
\\
\end{longtable}
\end{center}
