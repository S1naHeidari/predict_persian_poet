\begin{center}
\section*{غزل شماره ۱۷۹۹: در غیب پر این سو مپر ای طایر چالاک من}
\label{sec:1799}
\addcontentsline{toc}{section}{\nameref{sec:1799}}
\begin{longtable}{l p{0.5cm} r}
در غیب پر این سو مپر ای طایر چالاک من
&&
هم سوی پنهان خانه رو ای فکرت و ادراک من
\\
عالم چه دارد جز دهل از عیدگاه عقل کل
&&
گردون چه دارد جز که که از خرمن افلاک من
\\
من زخم کردم بر دلت مرهم منه بر زخم من
&&
من چاک کردم خرقه‌ات بخیه مزن بر چاک من
\\
در من از این خوشتر نگر کآب حیاتم سر به سر
&&
چندین گمان بد مبر ای خایف از اهلاک من
\\
دریا نباشد قطره‌ای با ساحل دریای جان
&&
شادی نیرزد حبه‌ای در همت غمناک من
\\
خرگوش و کبک و آهوان باشد شکار خسروان
&&
شیران نر بین سرنگون بربسته بر فتراک من
\\
دل‌های شیران خون شده صحرا ز خون گلگون شده
&&
مجنون کنان مجنون شده از شاهد لولاک من
\\
گر کاهلی باری بیا درکش یکی جام خدا
&&
کوه احد جنبان شود برپرد از محراک من
\\
جامی که تفش می زند بر آسمان بی‌سند
&&
دانی چه جوشش‌ها بود از جرعه‌اش بر خاک من
\\
آن باده بر مغزت زند چشم و دلت روشن کند
&&
وانگه ببینی گوهری در جسم چون خاشاک من
\\
عالم چو مرغی خفته‌ای بر بیضه پرچوژه‌ای
&&
زان بیضه یابد پرورش بال و پر املاک من
\\
روزی که مرغ از یک لگد از روی بیضه برجهد
&&
هفت آسمان فانی شود در نو بیضه پاک من
\\
خری که او را نیست بن می گوید ای خاک کهن
&&
دامن گشا گوهرستان کی دیده‌ای امساک من
\\
در وهم ناید ذات من اندیشه‌ها شد مات من
&&
جز احولی از احولی کی دم زند ز اشراک من
\\
خامش که اندر خامشی غرقه تری در بی‌هشی
&&
گر چه دهان خوش می شود زین حرف چون مسواک من
\\
\end{longtable}
\end{center}
