\begin{center}
\section*{غزل شماره ۲۲۳۴: ای دیده من جمال خود اندر جمال تو}
\label{sec:2234}
\addcontentsline{toc}{section}{\nameref{sec:2234}}
\begin{longtable}{l p{0.5cm} r}
ای دیده من جمال خود اندر جمال تو
&&
آیینه گشته‌ام همه بهر خیال تو
\\
و این طرفه‌تر که چشم نخسپد ز شوق تو
&&
گرمابه رفته هر سحری از وصال تو
\\
خاتون خاطرم که بزاید به هر دمی
&&
آبستن است لیک ز نور جلال تو
\\
آبستن است نه مهه کی باشدش قرار
&&
او را خبر کجاست ز رنج و ملال تو
\\
ای عشق اگر بجوشد خونم به غیر تو
&&
بادا به بی‌مرادی خونم حلال تو
\\
سر تا قدم ز عشق مرا شد زبان حال
&&
افغان به عرش برده و پرسان ز حال تو
\\
گر از عدم هزار جهان نو شود دگر
&&
بر صفحه جمال تو باشد چو خال تو
\\
از بس که غرقه‌ام چو مگس در حلاوتت
&&
پروا نباشدم به نظر در خصال تو
\\
در پیش شمس خسرو تبریز ای فلک
&&
می‌باش در سجود که این شد کمال تو
\\
\end{longtable}
\end{center}
