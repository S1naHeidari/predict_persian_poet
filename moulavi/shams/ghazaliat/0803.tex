\begin{center}
\section*{غزل شماره ۸۰۳: بر سر کوی تو عقل از سر جان برخیزد}
\label{sec:0803}
\addcontentsline{toc}{section}{\nameref{sec:0803}}
\begin{longtable}{l p{0.5cm} r}
بر سر کوی تو عقل از سر جان برخیزد
&&
خوشتر از جان چه بود از سر آن برخیزد
\\
بر حصار فلک ار خوبی تو حمله برد
&&
از مقیمان فلک بانگ امان برخیزد
\\
بگذر از باغ جهان یک سحر ای رشک بهار
&&
تا ز گلزار و چمن رسم خزان برخیزد
\\
پشت افلاک خمیدست از این بار گران
&&
ای سبک روح ز تو بار گران برخیزد
\\
من چو از تیر توام بال و پری بخش مرا
&&
خوش پرد تیر زمانی که کمان برخیزد
\\
رمه خفتست همی‌گردد گرگ از چپ و راست
&&
سگ ما بانگ برآرد که شبان برخیزد
\\
من گمانم تو عیان پیش تو من محو به هم
&&
چون عیان جلوه کند چهره گمان برخیزد
\\
هین خمش دل پنهانست کجا زیر زبان
&&
آشکارا شود این دل چو زبان برخیزد
\\
این مجابات مجیر است در آن قطعه که گفت
&&
بر سر کوی تو عقل از سر جان برخیزد
\\
\end{longtable}
\end{center}
