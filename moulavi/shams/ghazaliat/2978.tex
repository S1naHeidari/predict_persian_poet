\begin{center}
\section*{غزل شماره ۲۹۷۸: ای دل ز بامداد تو بر حال دیگری}
\label{sec:2978}
\addcontentsline{toc}{section}{\nameref{sec:2978}}
\begin{longtable}{l p{0.5cm} r}
ای دل ز بامداد تو بر حال دیگری
&&
وز شور خویش در من شوریده ننگری
\\
بر چهره نزار تو صفرای دلبری است
&&
تا خود چه دیده‌ای که ز صفراش اصفری
\\
ای دل چه آتشی که به هر باد برجهی
&&
نی نی دلا کز آتش و از باد برتری
\\
ای دل تو هر چه هستی دانم که این زمان
&&
خورشیدوار پرده افلاک می‌دری
\\
جانم فدات یا رب ای دل چه گوهری
&&
نی چرخ قیمت تو شناسد نه مشتری
\\
سی سال در پی تو چو مجنون دویده‌ام
&&
اندر جزیره‌ای که نه خشکی است و نی تری
\\
غافل بدم از آن که تو مجموع هستیی
&&
مشغول بود فکر به ایمان و کافری
\\
ایمان و کفر و شبهه و تعطیل عکس توست
&&
هم جنتی و دوزخ و هم حوض کوثری
\\
ای دل تو کل کونی بیرون ز هر دو کون
&&
ای جمله چیزها تو و از چیزها بری
\\
ای رو و پشت عالم در روی من نگر
&&
تا از رخ مزعفر من زعفران بری
\\
طاقت نماند و این سخنم ماند در دهان
&&
با صد هزار غم که نهانند چون پری
\\
\end{longtable}
\end{center}
