\begin{center}
\section*{غزل شماره ۱۲۳۹: ای خواجه تو عاقلانه می‌باش}
\label{sec:1239}
\addcontentsline{toc}{section}{\nameref{sec:1239}}
\begin{longtable}{l p{0.5cm} r}
ای خواجه تو عاقلانه می‌باش
&&
چون بی‌خبری ز شور اوباش
\\
آن چهره که رشک فخر فقرست
&&
با ناخن زشت خویش مخراش
\\
آن بت به خیال درنگنجد
&&
بت‌ها به خیال خانه متراش
\\
جمله بت و بت پرست چون اوست
&&
غیر کل و جمله چیست جز لاش
\\
نی فهم کنند خلق این را
&&
نی دستوری که دم زنم فاش
\\
این ماش برنج احولانست
&&
ور نی نه برنج هست و نی ماش
\\
پایان‌ها را کجا شناسند
&&
چون پوشیدست رشک روهاش
\\
گر می‌دزدی ز زندگان دزد
&&
ای دزد کفن به شب چو نباش
\\
اما ز قضاست مات من مات
&&
هم حکم قضاست عاش من عاش
\\
خامش که ز شب خبر ندارد
&&
آن کس که به روز خورد خشخاش
\\
\end{longtable}
\end{center}
