\begin{center}
\section*{غزل شماره ۲۵۵۸: الا ای نقش روحانی چرا از ما گریزانی}
\label{sec:2558}
\addcontentsline{toc}{section}{\nameref{sec:2558}}
\begin{longtable}{l p{0.5cm} r}
الا ای نقش روحانی چرا از ما گریزانی
&&
تو خود از خانه آخر ز حال بنده می دانی
\\
به حق اشک گرم من به حق روی زرد من
&&
به پیوندی که با تستم ورای طور انسانی
\\
اگر عالم بود خندان مرا بی‌تو بود زندان
&&
بس است آخر بکن رحمی بر این محروم زندانی
\\
اگر با جمله خویشانم چو تو دوری پریشانم
&&
مبادا ای خدا کس را بدین غایت پریشانی
\\
بر آن پای گریزانت چه بربندم که نگریزی
&&
به جان بی‌وفا مانی چو یار ما گریزانی
\\
ور از نه چرخ برتازی بسوزی هفت دریا را
&&
بدرم چرخ و دریا را به عشق و صبر و پیشانی
\\
وگر چو آفتابی هم روی بر طارم چارم
&&
چو سایه در رکاب تو همی‌آیم به پنهانی
\\
\end{longtable}
\end{center}
