\begin{center}
\section*{غزل شماره ۲۸۳۳: سوی باغ ما سفر کن بنگر بهار باری}
\label{sec:2833}
\addcontentsline{toc}{section}{\nameref{sec:2833}}
\begin{longtable}{l p{0.5cm} r}
سوی باغ ما سفر کن بنگر بهار باری
&&
سوی یار ما گذر کن بنگر نگار باری
\\
نرسی به باز پران پی سایه‌اش همی‌دو
&&
به شکارگاه غیب آ بنگر شکار باری
\\
به نظاره و تماشا به سواحل آ و دریا
&&
بستان ز اوج موجش در شاهوار باری
\\
چو شکار گشت باید به کمند شاه اولی
&&
چو برهنه گشت باید به چنین قمار باری
\\
بکشان تو لنگ لنگان ز بدن به عالم جان
&&
بنگر ترنج و ریحان گل و سبزه زار باری
\\
هله چنگیان بالا ز برای سیم و کالا
&&
به سماع زهره ما بزنید تار باری
\\
به میان این ظریفان به سماع این حریفان
&&
ره بوسه گر نباشد برسد کنار باری
\\
به چنین شراب ارزد ز خمار خسته بودن
&&
پی این قرار برگو دل بی‌قرار باری
\\
ز سبو فغان برآمد که ز تف می‌شکستم
&&
هله ای قدح به پیش آ بستان عقار باری
\\
پی خسروان شیرین هنر است شور کردن
&&
به چنین حیات جان‌ها دل و جان سپار باری
\\
به دکان عشق روزی ز قضا گذار کردم
&&
دل من رمید کلی ز دکان و کار باری
\\
من از آن درج گذشتم که مرا تو چاره سازی
&&
دل و جان به باد دادم تو نگاه دار باری
\\
هله بس کنم که شرحش شه خوش بیان بگوید
&&
هله مطرب معانی غزلی بیار باری
\\
\end{longtable}
\end{center}
