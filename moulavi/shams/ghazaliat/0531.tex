\begin{center}
\section*{غزل شماره ۵۳۱: صوفی چرا هوشیار شد ساقی چرا بی‌کار شد}
\label{sec:0531}
\addcontentsline{toc}{section}{\nameref{sec:0531}}
\begin{longtable}{l p{0.5cm} r}
صوفی چرا هوشیار شد ساقی چرا بی‌کار شد
&&
مستی اگر در خواب شد مستی دگر بیدار شد
\\
خورشید اگر در گور شد عالم ز تو پرنور شد
&&
چشم خوشت مخمور شد چشم دگر خمار شد
\\
گر عیش اول پیر شد صد عیش نو توفیر شد
&&
چون زلف تو زنجیر شد دیوانگی ناچار شد
\\
ای مطرب شیرین نفس عشرت نگر از پیش و پس
&&
کس نشنود افسون کس چون واقف اسرار شد
\\
ما موسییم و تو مها گاهی عصا گه اژدها
&&
ای شاهدان ارزان بها چون غارت بلغار شد
\\
لعلت شکرها کوفته چشمت ز رشک آموخته
&&
جان خانه دل روفته هین نوبت دیدار شد
\\
هر بار عذری می‌نهی وز دست مستی می‌جهی
&&
ای جان چه دفعم می‌دهی این دفع تو بسیار شد
\\
ای کرده دل چون خاره‌ای امشب نداری چاره‌ای
&&
تو ماه و ما استاره‌ای استاره با مه یار شد
\\
ای ماه بیرون از افق ای ما تو را امشب قنق
&&
چون شب جهان را شد تتق پنهان روان را کار شد
\\
گر زحمت از تو برده‌ام پنداشتی من مرده‌ام
&&
تو صافی و من درده‌ام بی‌صاف دردی خوار شد
\\
از وصل همچون روز تو در هجر عالم سوز تو
&&
در عشق مکرآموز تو بس ساده دل عیار شد
\\
نی تب بدم نی درد سر سر می‌زدم دیوار بر
&&
کز طمع آن خوش گلشکر قاصد دلم بیمار شد
\\
\end{longtable}
\end{center}
