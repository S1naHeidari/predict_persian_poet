\begin{center}
\section*{غزل شماره ۱۴۱۹: بگفتم حال دل گویم از آن نوعی که دانستم}
\label{sec:1419}
\addcontentsline{toc}{section}{\nameref{sec:1419}}
\begin{longtable}{l p{0.5cm} r}
بگفتم حال دل گویم از آن نوعی که دانستم
&&
برآمد موج آب چشم و خون دل نتانستم
\\
شکسته بسته می گفتم پریر از شرح دل چیزی
&&
تنک شد جام فکر و من چو شیشه خرد بشکستم
\\
چو تخته تخته بشکستند کشتی‌ها در این طوفان
&&
چه باشد زورق من خود که من بی‌پا و بی‌دستم
\\
شکست از موج این کشتی نه خوبی ماند و نه زشتی
&&
شدم بی‌خویش و خود را من سبک بر تخته‌ای بستم
\\
نه بالایم نه پست اما ولیک این حرف پست آمد
&&
که گه زین موج بر اوجم گهی زان اوج در پستم
\\
چه دانم نیستم هستم ولیک این مایه می دانم
&&
چو هستم نیستم ای جان ولی چون نیستم هستم
\\
چه شک ماند مرا در حشر چون صد ره در این محشر
&&
چو اندیشه بمردم زار و چون اندیشه برجستم
\\
جگر خون شد ز صیادی مرا باری در این وادی
&&
ز صیدم چون نبد شادی شدم من صید و وارستم
\\
بود اندیشه چون بیشه در او صد گرگ و یک میشه
&&
چه اندیشه کنم پیشه که من ز اندیشه ده مستم
\\
به هر چاهی که برکندم ز اول من درافتادم
&&
به هر دامی که بنهادم من اندر دام پیوستم
\\
خسی که مشتریش آمد خیال خام ریش آمد
&&
سبال از کبر می مالد که رو من کار کردستم
\\
چه کردی آخر ای کودن نشاندی گل در این گلخن
&&
نرست از گلشنت برگی ولیک از خار تو خستم
\\
مرا واجب کند که من برون آیم چو گل از تن
&&
که عمرم شد به شصت و من چو سین و شین در این شستم
\\
\end{longtable}
\end{center}
