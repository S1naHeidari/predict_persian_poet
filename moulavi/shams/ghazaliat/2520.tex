\begin{center}
\section*{غزل شماره ۲۵۲۰: گر آبت بر جگر بودی دل تو پس چه کاره ستی}
\label{sec:2520}
\addcontentsline{toc}{section}{\nameref{sec:2520}}
\begin{longtable}{l p{0.5cm} r}
گر آبت بر جگر بودی دل تو پس چه کاره ستی
&&
تنت گر آن چنان بودی که گفتی دل نگاره ستی
\\
وگر بر کار بودی دل درون کارگاه عشق
&&
ملالت بر برون تو نمی‌گویی چه کاره ستی
\\
غنیمت دار رمضان را چو عیدت روی ننموده‌ست
&&
و عیدت گر کنارستی ز غم جان برکناره ستی
\\
چو روشن گشتی از طاعت شدی تاریک از عصیان
&&
دل بیچاره را می‌دان که او محتاج چاره ستی
\\
وگر محتاج این طاعت نماندستی دل مسکین
&&
ورای کفر و ایمان دل همیشه در نظاره ستی
\\
تو گویی جان من لعل است مگر نبود بدین لعلی
&&
ز تابش‌های خورشیدش مبر گو سنگ خاره ستی
\\
به گرد قلعه ظلمت نماندی سنگ یک پاره
&&
اگر خود منجنیق صوم دایم سوی باره ستی
\\
بزن این منجنیق صوم قلعه کفر و ظلمت بر
&&
اگر بودی مسلمانی مؤذن بر مناره ستی
\\
اگر از عید قربان سرافرازان بدانندی
&&
نه هر پاره ز گاو نفس آویز قناره ستی
\\
اگر سوز دل مسکین بدیدییی از این لقمه
&&
ز بهر ساکنی سوزش شکم سوزی هماره ستی
\\
در اول منزلت این عشق با این لوت ضدانند
&&
اگر این عشق باره ستی چرا او لوت باره ستی
\\
همه عالم خر و گاوان به عیش اندرخزیدندی
&&
اگر عاشق بدی آن کس که دایم لوت خواره ستی
\\
اگر دیدی تو ظلمت‌ها ز قوت‌های این لقمه
&&
ز جور نفس تردامن گریبان‌هات پاره ستی
\\
به تدریج ار کنی تو پی خر دجال از روزه
&&
ببینی عیسی مریم که در میدان سواره ستی
\\
اگر امر تصوموا را نگهداری به امر رب
&&
به هر یا رب که می‌گویی تو لبیکت دوباره ستی
\\
\end{longtable}
\end{center}
