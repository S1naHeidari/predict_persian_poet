\begin{center}
\section*{غزل شماره ۲۰۰۳: همه خوردند و بخفتند و تهی گشت وطن}
\label{sec:2003}
\addcontentsline{toc}{section}{\nameref{sec:2003}}
\begin{longtable}{l p{0.5cm} r}
همه خوردند و بخفتند و تهی گشت وطن
&&
وقت آن شد که درآییم خرامان به چمن
\\
دامن سیب کشانیم سوی شفتالو
&&
ببریم از گل تر چند سخن سوی سمن
\\
نوبهاران چون مسیحی است فسون می‌خواند
&&
تا برآیند شهیدان نباتی ز کفن
\\
آن بتان چون جهت شکر دهان بگشادند
&&
جان به بوسه نرسد مست شد از بوی دهن
\\
تاب رخسار گل و لاله خبر می‌دهدم
&&
که چراغی است نهان گشته در این زیر لگن
\\
برگ می‌لرزد و بر شاخ دلم می‌لرزد
&&
لرزه برگ ز باد و دلم از خوب ختن
\\
دست دستان صبا لخلخه را شورانید
&&
تا بیاموخت به طفلان چمن خلق حسن
\\
باد روح قدس افتاد و درختان مریم
&&
دست بازی نگر آن سان که کند شوهر و زن
\\
ابر چون دید که در زیر تتق خوبانند
&&
برفشانید نثار گهر و در عدن
\\
چون گل سرخ گریبان ز طرب بدرانید
&&
وقت آن شد که به یعقوب رسد پیراهن
\\
چون عقیق یمنی لب دلبر خندید
&&
بوی یزدان به محمد رسد از سوی یمن
\\
چند گفتیم پراکنده دل آرام نیافت
&&
جز بر آن زلف پراکنده آن شاه زمن
\\
\end{longtable}
\end{center}
