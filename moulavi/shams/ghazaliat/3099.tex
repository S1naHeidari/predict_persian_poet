\begin{center}
\section*{غزل شماره ۳۰۹۹: بداد پندم استاد عشق از استادی}
\label{sec:3099}
\addcontentsline{toc}{section}{\nameref{sec:3099}}
\begin{longtable}{l p{0.5cm} r}
بداد پندم استاد عشق از استادی
&&
که هین بترس ز هر کس که دل بدو دادی
\\
هر آن کسی که تو از نوش او بنوشیدی
&&
ز بعد نوش کند نیش اوت فصادی
\\
چو چشم مست کسی کرد حلقه در گوشت
&&
ز گوش پنبه برون کن مجوی آزادی
\\
بر این بنه دل خود را چو دخل خنده رسید
&&
که غم نجوید عشرت ز خرمن شادی
\\
مگر زمین مسلم دهد تو را سلطان
&&
چنانک داد به بشر و جنید بغدادی
\\
چو طوق موهبت آمد شکست گردن غم
&&
رسید داد خدا و بمرد بیدادی
\\
به هر کجا که روی ماه بر تو می‌تابد
&&
مهست نورفشان بر خراب و آبادی
\\
غلام ماه شدی شب تو را به از روزست
&&
که پشتدار تو باشد میان هر وادی
\\
خنک تو را و خنک جمله همرهان تو را
&&
که سعد اکبری و نیکبخت افتادی
\\
به وعده‌های خوشش اعتماد کن ای جان
&&
که شاه مثل ندارد به راست میعادی
\\
به گوش تو همه تفسیر این بگوید شاه
&&
چنانک اشتر خود را نوا زند حادی
\\
\end{longtable}
\end{center}
