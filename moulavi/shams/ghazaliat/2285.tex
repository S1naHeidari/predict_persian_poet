\begin{center}
\section*{غزل شماره ۲۲۸۵: یا رجلا حصیده مجبنه و مبخله}
\label{sec:2285}
\addcontentsline{toc}{section}{\nameref{sec:2285}}
\begin{longtable}{l p{0.5cm} r}
یا رجلا حصیده مجبنه و مبخله
&&
لیس یلذک الهوی لیس لفیک حوصله
\\
معتمد الهوی معی مستندی و سیدی
&&
لا کرجاک ضایع یطلبه به غربله
\\
ای گله بیش کرده تو سیر نگشتی از گله
&&
چون بکری است این دکان چاره نباشد از غله
\\
حج پیاده می‌روی تا سر حاجیان شوی
&&
جامه چرا دری اگر شد کف پات آبله
\\
از پی نیم آبله شرم نیایدت که تو
&&
هر قدمی درافکنی غلغله ای به قافله
\\
کشتی نفس آدمی لنگری است و سست رو
&&
زین دریا بنگذرد بی ز کشاکش و خله
\\
گر نبدی چنین چرا جهد و جهاد آمدی
&&
صوم و صلات و شب روی حج و مناسک و چله
\\
صبر سوی نران رود نوحه سوی زنان رود
&&
گردن اسب شاه را ننگ بود ز زنگله
\\
خوش به میان صف درآ تنگ میا و دلگشا
&&
هست ز تنگ آمدن بانگ گلوی بلبله
\\
خاص احد چه غم خورد از بد و نیک عام خس
&&
کوه احد چه برتپد از سر سیل و زلزله
\\
دل مطپان به خیر و شر جانب غیب درنگر
&&
کلکله ملایکه روح میان کلکله
\\
عزت زر بود اگر محنت او شود شرر
&&
هیبت و بیم شیر دان بستن او به سلسله
\\
کم نشود انار اگر بهر شراب بفشری
&&
بهر فضیلتی بود کوفتگی آمله
\\
حامله است تن ز جان درد زه است رنج تن
&&
آمدن جنین بود درد و عذاب حامله
\\
تلخی باده را مبین عشرت مستیان نگر
&&
محنت حامله مبین بنگر امید قابله
\\
هست بلادر این ستم پیش بلا و پس دری
&&
هست سر محاسبه جبر و پیش مقابله
\\
زر به کسی به قرض ده کش بود آسیا و رز
&&
با خلجی و مفلسی هیچ مکن معامله
\\
نه فلک چو آسیا ملک کیست غیر حق
&&
باغ و چراگه زمین پر ز شبان و از گله
\\
قرض بدو ده ای پسر نفس و نفس زر و درم
&&
گنج و گهر ستان از او از پی فرض و نافله
\\
لب بگشاد ناطقی تا که بیان این کند
&&
کان زر او است و نقد او فکرت خلق ناقله
\\
\end{longtable}
\end{center}
