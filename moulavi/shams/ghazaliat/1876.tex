\begin{center}
\section*{غزل شماره ۱۸۷۶: بی جا شو در وحدت در عین فنا جا کن}
\label{sec:1876}
\addcontentsline{toc}{section}{\nameref{sec:1876}}
\begin{longtable}{l p{0.5cm} r}
بی جا شو در وحدت در عین فنا جا کن
&&
هر سر که دوی دارد در گردن ترسا کن
\\
اندر قفس هستی این طوطی قدسی را
&&
زان پیش که برپرد شکرانه شکرخا کن
\\
چون مست ازل گشتی شمشیر ابد بستان
&&
هندوبک هستی را ترکانه تو یغما کن
\\
دردی وجودت را صافی کن و پالوده
&&
وان شیشه معنی را پرصافی صهبا کن
\\
تا مار زمین باشی کی ماهی دین باشی
&&
ما را چو شدی ماهی پس حمله به دریا کن
\\
اندر حیوان بنگر سر سوی زمین دارد
&&
گر آدمیی آخر سر جانب بالا کن
\\
در مدرسه آدم با حق چو شدی محرم
&&
بر صدر ملک بنشین تدریس ز اسما کن
\\
چون سلطنت الا خواهی بر لالا شو
&&
جاروب ز لا بستان فراشی اشیاء کن
\\
گر عزم سفر داری بر مرکب معنی رو
&&
ور زانک کنی مسکن بر طارم خضرا کن
\\
می باش چو مستسقی کو را نبود سیری
&&
هر چند شوی عالی تو جهد به اعلا کن
\\
هر روح که سر دارد او روی به در دارد
&&
داری سر این سودا سر در سر سودا کن
\\
بی سایه نباشد تن سایه نبود روشن
&&
برپر تو سوی روزن پرواز تو تنها کن
\\
بر قاعده مجنون سرفتنه غوغا شو
&&
کاین عشق همی‌گوید کز عقل تبرا کن
\\
هم آتش سوزان شو هم پخته و بریان شو
&&
هم مست شو و هم می بی‌هر دو تو گیرا کن
\\
هم سر شو و محرم شو هم دم زن و همدم شو
&&
هم ما شو و ما را شو هم بندگی ما کن
\\
تا ره نبرد ترسا دزدیده به دیر تو
&&
گه عاشق زناری گه قصد چلیپا کن
\\
دانا شده‌ای لیکن از دانش هستانه
&&
بی دیده هستانه رو دیده تو بینا کن
\\
موسی خضرسیرت شمس الحق تبریزی
&&
از سر تو قدم سازش قصد ید بیضا کن
\\
\end{longtable}
\end{center}
