\begin{center}
\section*{غزل شماره ۲۵۴: باده ده آن یار قدح باره را}
\label{sec:0254}
\addcontentsline{toc}{section}{\nameref{sec:0254}}
\begin{longtable}{l p{0.5cm} r}
باده ده آن یار قدح باره را
&&
یار ترش روی شکرپاره را
\\
منگر آن سوی بدین سو گشا
&&
غمزه غمازه خون خواره را
\\
دست تو می‌مالد بیچاره وار
&&
نه به کفش چاره بیچاره را
\\
خیره و سرگشته و بی‌کار کن
&&
این خرد پیر همه کاره را
\\
ای کرمت شاه هزاران کرم
&&
چشمه فرستی جگر خاره را
\\
طفل دوروزه چو ز تو بو برد
&&
می‌کشد او سوی تو گهواره را
\\
ترک کند دایه و صد شیر را
&&
ای بدل روغن کنجاره را
\\
خوب کلیدی در بربسته را
&&
خوب کمندی دل آواره را
\\
کار تو این باشد ای آفتاب
&&
نور فرستی مه و استاره را
\\
منتظرش باش و چو مه نور گیر
&&
ترک کن این گنگل و نظاره را
\\
رحمت تو مهره دهد مار را
&&
خانه دهد عقرب جراره را
\\
یاد دهد کار فراموش را
&&
باد دهد خاطر سیاره را
\\
هر بت سنگین ز دمش زنده شد
&&
تا چه دمست آن بت سحاره را
\\
خامش کن گفت از این عالم است
&&
ترک کن این عالم غداره را
\\
\end{longtable}
\end{center}
