\begin{center}
\section*{غزل شماره ۳۰۶۴: ز بامداد دلم می‌پرد به سودایی}
\label{sec:3064}
\addcontentsline{toc}{section}{\nameref{sec:3064}}
\begin{longtable}{l p{0.5cm} r}
ز بامداد دلم می‌پرد به سودایی
&&
چو وام دار مرا می‌کند تقاضایی
\\
عجب به خواب چه دیده‌ست دوش این دل من
&&
که هست در سرم امروز شور و صفرایی
\\
ولی دلم چه کند چون موکلان قضا
&&
همی‌رسند پیاپی به دل ز بالایی
\\
پرست خانه دل از موکل عجمی
&&
که نیست یک سر سوزن بهانه را جایی
\\
بهانه نیست وگر هست کو زبان و دلی
&&
گریز نیست وگر هست کو مرا پایی
\\
جهان که آمد و ما همچو سیل از سر کوه
&&
روان و رقص کنانیم تا به دریایی
\\
اگر چه سیل بنالد ز راه ناهموار
&&
قدم قدم بودش در سفر تماشایی
\\
چگونه زار ننالم من از کسی که گرفت
&&
به هر دو دست و دهان او مرا چو سرنایی
\\
هوس نشسته که فردا چنین کنیم و چنان
&&
خبر ندارد کو را نماند فردایی
\\
غلام عشقم کو نقد وقت می‌جوید
&&
نه وعده دارد و نه نسیه‌ای و نی رایی
\\
\end{longtable}
\end{center}
