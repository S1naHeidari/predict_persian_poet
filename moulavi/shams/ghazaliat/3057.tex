\begin{center}
\section*{غزل شماره ۳۰۵۷: اگر ز حلقه این عاشقان کران گیری}
\label{sec:3057}
\addcontentsline{toc}{section}{\nameref{sec:3057}}
\begin{longtable}{l p{0.5cm} r}
اگر ز حلقه این عاشقان کران گیری
&&
دلت بمیرد و خوی فسردگان گیری
\\
گر آفتاب جهانی چو ابر تیره شوی
&&
وگر بهار نوی مذهب خزان گیری
\\
چو کاسه تا تهیی تو بر آب رقص کنی
&&
چو پر شدی به بن حوض و جو مکان گیری
\\
خدای داد دو دستت که دامن من گیر
&&
بداد عقل که تا راه آسمان گیری
\\
که عقل جنس فرشته‌ست سوی او پوید
&&
ببینیش چو به کف آینه نهان گیری
\\
بگیر کیسه پرزر باقرضواالله آی
&&
قراضه قرض دهی صد هزار کان گیری
\\
به غیر خم فلک خم‌های صدرنگ است
&&
به هر خمی که درآیی از او نشان گیری
\\
ز شیر چرخ گریزی به برج گاو روی
&&
خری شوی به صفت راه کهکشان گیری
\\
وگر تو خود سرطانی چو پهلوی شیری
&&
یقین ز پهلوی او خوی پهلوان گیری
\\
چو آفتاب جهان را پر از حیات کنی
&&
چو زین جهان بجهی ملک آن جهان گیری
\\
برآ چو آب ز تنور نوح و عالمگیر
&&
چرا تنور خبازی که جمله نان گیری
\\
خموش باش و همی‌تاز تا لب دریا
&&
چو دم گسسته شوی گر ره دهان گیری
\\
\end{longtable}
\end{center}
