\begin{center}
\section*{غزل شماره ۶۹۸: آن کس که ز تو نشان ندارد}
\label{sec:0698}
\addcontentsline{toc}{section}{\nameref{sec:0698}}
\begin{longtable}{l p{0.5cm} r}
آن کس که ز تو نشان ندارد
&&
گر خورشیدست آن ندارد
\\
ما بر در و بام عشق حیران
&&
آن بام که نردبان ندارد
\\
دل چون چنگست و عشق زخمه
&&
پس دل به چه دل فغان ندارد
\\
امروز فغان عاشقان را
&&
بشنو که تو را زیان ندارد
\\
هر ذره پر از فغان و ناله‌ست
&&
اما چه کند زیان ندارد
\\
رقص است زبان ذره زیرا
&&
جز رقص دگر بیان ندارد
\\
هر سو نگران تست دل‌ها
&&
وان سو که تویی گمان ندارد
\\
این عالم را کرانه‌ای هست
&&
عشق من و تو کران ندارد
\\
مانند خیال تو ندیدم
&&
بوسه دهد و دهان ندارد
\\
ماننده غمزه‌ات ندیدم
&&
تیر اندازد کمان ندارد
\\
دادی کمری که بر میان بند
&&
طفل دل من میان ندارد
\\
گفتی که به سوی ما روان شو
&&
بی لطف تو جان روان ندارد
\\
\end{longtable}
\end{center}
