\begin{center}
\section*{غزل شماره ۲۶۰۵: ای باغ همی‌دانی کز باد کی رقصانی}
\label{sec:2605}
\addcontentsline{toc}{section}{\nameref{sec:2605}}
\begin{longtable}{l p{0.5cm} r}
ای باغ همی‌دانی کز باد کی رقصانی
&&
آبستن میوه ستی سرمست گلستانی
\\
این روح چرا داری گر ز آنک تو این جسمی
&&
وین نقش چرا بندی گر ز آنک همه جانی
\\
جان پیشکشت چه بود خرما به سوی بصره
&&
وز گوهر چون گویم چون غیرت عمانی
\\
عقلا ز قیاس خود زین رو تو زنخ می‌زن
&&
زان رو تو کجا دانی چون مست زنخدانی
\\
دشوار بود با کر طنبور نوازیدن
&&
یا بر سر صفرایی رسم شکرافشانی
\\
می وام کند ایمان صد دیده به دیدارش
&&
تا مست شود ایمان زان باده یزدانی
\\
در پای دل افتم من هر روز همی‌گویم
&&
راز تو شود پنهان گر راز تو نجهانی
\\
کان مهره شش گوشه هم لایق آن نطع است
&&
کی گنجد در طاسی شش گوشه انسانی
\\
شمس الحق تبریزی من باز چرا گردم
&&
هر لحظه به دست تو گر ز آنک نه سلطانی
\\
\end{longtable}
\end{center}
