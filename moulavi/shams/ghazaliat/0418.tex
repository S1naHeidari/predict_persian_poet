\begin{center}
\section*{غزل شماره ۴۱۸: سر مپیچان و مجنبان که کنون نوبت تو است}
\label{sec:0418}
\addcontentsline{toc}{section}{\nameref{sec:0418}}
\begin{longtable}{l p{0.5cm} r}
سر مپیچان و مجنبان که کنون نوبت توست
&&
بستان جام و درآشام که آن شربت توست
\\
عدد ذره در این جو هوا عشاقند
&&
طرب و حالت ایشان مدد حالت توست
\\
همگی پرده و پوشش ز پی باشش تو است
&&
جرس و طبل رحیل از جهت رحلت توست
\\
هر که را همت عالی بود و فکر بلند
&&
دانک آن همت عالی اثر همت توست
\\
فکرتی کان نبود خاسته از طبع و دماغ
&&
نیست در عالم اگر باشد آن فکرت توست
\\
ای دل خسته ز هجران و ز اسباب دگر
&&
هم از او جوی دوا را که ولی نعمت توست
\\
ز آن سوی کآمد محنت هم از آن سو است دوا
&&
هم از او شبهه تو است و هم از او حجت توست
\\
هم خمار از می آید هم از او دفع خمار
&&
هم از او عسرت تو است و هم از او عشرت توست
\\
بس که هر مستمعی را هوس و سوداییست
&&
نه همه خلق خدا را صفت و فطرت توست
\\
\end{longtable}
\end{center}
