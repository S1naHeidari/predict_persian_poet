\begin{center}
\section*{غزل شماره ۲۶۰۰: گفتم که بجست آن مه از خانه چو عیاری}
\label{sec:2600}
\addcontentsline{toc}{section}{\nameref{sec:2600}}
\begin{longtable}{l p{0.5cm} r}
گفتم که بجست آن مه از خانه چو عیاری
&&
تشنیع زنان بودم بر عهد وفاداری
\\
غماز غمت گفتا در خانه بجوی آخر
&&
آن طره که دل دزدد ماننده طراری
\\
در سوخته جان زن از آهن و از سنگش
&&
در پیه دو دیده خود بر آب بزن ناری
\\
بفروز چنین شمعی در خانه همی‌گردان
&&
باشد که نهان باشد او از پس دیواری
\\
اندر پس دیواری در سایه خورشیدش
&&
در نیم شب هجران بگشود مرا کاری
\\
در خانه همی‌گشتم در دست چنین شمعی
&&
تا تیره شد این شمعم از تابش انواری
\\
گفتم که در این زندان چون یافتمت ای جان
&&
در بی‌نمکی چون ره بردم به نمکساری
\\
ای شوخ گریزنده وی شاه ستیزنده
&&
وی از تو جهان زنده چون یافتمت باری
\\
در حال نهانی شد پنهان چو معانی شد
&&
چون گوهر کانی شد غیرت شده ستاری
\\
من دست زنان بر سر چون حلقه شده بر در
&&
وین طعنه زنان بر من هم یافته بازاری
\\
از پرتو مخدومی شمس الحق تبریزی
&&
چون مه که ز خورشیدش شد تیره خجل واری
\\
\end{longtable}
\end{center}
