\begin{center}
\section*{غزل شماره ۱۸۳۳: آمده‌ام به عذر تو ای طرب و قرار جان}
\label{sec:1833}
\addcontentsline{toc}{section}{\nameref{sec:1833}}
\begin{longtable}{l p{0.5cm} r}
آمده‌ام به عذر تو ای طرب و قرار جان
&&
عفو نما و درگذر از گنه و عثار جان
\\
نیست به جز رضای تو قفل گشای عقل و دل
&&
نیست به جز هوای تو قبله و افتخار جان
\\
سوخته شد ز هجر تو گلشن و کشت زار من
&&
زنده کنش به فضل خود ای دم تو بهار جان
\\
بی لب می فروش تو کی شکند خمار دل
&&
بی خم ابروی کژت راست نگشت کار جان
\\
از تو چو مشرقی شود روشن پشت و روی دل
&&
بر چو تو دلبری سزد هر نفسی نثار جان
\\
تافتن شعاع تو در سر روزن دلی
&&
تبصره خرد بود هر دم اعتبار جان
\\
از غم دوری لقا راه حبیب طی شود
&&
در ره و منهج خدا هست خدای یار جان
\\
گلبن روی غیبیان چون برسد بدیده‌ای
&&
از گل سرخ پر شود بی‌چمنی کنار جان
\\
لاف زدم که هست او همدم و یار غار من
&&
یار منی تو بی‌گمان خیز بیا به غار جان
\\
گفت اناالحق و بشد دل سوی دار امتحان
&&
آن دم پای دار شد دولت پایدار جان
\\
باغ که بی‌تو سبز شد دی بدهد سزای او
&&
جان که جز از تو زنده شد نیست وی از شمار جان
\\
دانه نمود دام تو در نظر شکار دل
&&
خانه گرفت عشق تو ناگه در جوار جان
\\
نیم حدیث گفته شد نیم دگر مگو خمش
&&
شهره کند حدیث را بر همه شهریار جان
\\
\end{longtable}
\end{center}
