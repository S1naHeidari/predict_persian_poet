\begin{center}
\section*{غزل شماره ۲۷۶۷: گر یار لطیف و باوفایی}
\label{sec:2767}
\addcontentsline{toc}{section}{\nameref{sec:2767}}
\begin{longtable}{l p{0.5cm} r}
گر یار لطیف و باوفایی
&&
ور از دل و جان از آن مایی
\\
خواهم که در این میان درآیی
&&
ای ماه بگو که کی برآیی
\\
چون صورت جان لطیف کاری
&&
از حلقه چرا تو برکناری
\\
وز یارک خود دریغ داری
&&
ای ماه بگو که کی برآیی
\\
برخیز که ما و تو چو جانیم
&&
وز رازک همدگر بدانیم
\\
آخر نه من و تو یارکانیم
&&
ای ماه بگو که کی برآیی
\\
دریاب که بر در خداییم
&&
آخر بنگر که ما کجاییم
\\
تا رقص کنان ز در درآییم
&&
ای ماه بگو که کی برآیی
\\
ای جان و جهان چرا چنینی
&&
چون یارک خویش را نبینی
\\
در گوشه روی ترش نشینی
&&
ای ماه بگو که کی برآیی
\\
چونی تو و آن دل لطیفت
&&
و آن صورت و قامت ظریفت
\\
خواهم که شوم شبی حریفت
&&
ای ماه بگو که کی برآیی
\\
در جمله عالم الهی
&&
وز دامن ماه تا به ماهی
\\
آن شد که تو گویی و بخواهی
&&
ای ماه بگو که کی برآیی
\\
\end{longtable}
\end{center}
