\begin{center}
\section*{غزل شماره ۲۵۷۲: جانا به غریبستان چندین به چه می‌مانی}
\label{sec:2572}
\addcontentsline{toc}{section}{\nameref{sec:2572}}
\begin{longtable}{l p{0.5cm} r}
جانا به غریبستان چندین به چه می‌مانی
&&
بازآ تو از این غربت تا چند پریشانی
\\
صد نامه فرستادم صد راه نشان دادم
&&
یا راه نمی‌دانی یا نامه نمی‌خوانی
\\
گر نامه نمی‌خوانی خود نامه تو را خواند
&&
ور راه نمی‌دانی در پنجه ره دانی
\\
بازآ که در آن محبس قدر تو نداند کس
&&
با سنگ دلان منشین چون گوهر این کانی
\\
ای از دل و جان رسته دست از دل و جان شسته
&&
از دام جهان جسته بازآ که ز بازانی
\\
هم آبی و هم جویی هم آب همی‌جویی
&&
هم شیر و هم آهویی هم بهتر از ایشانی
\\
چند است ز تو تا جان تو طرفه تری یا جان
&&
آمیخته‌ای با جان یا پرتو جانانی
\\
نور قمری در شب قند و شکری در لب
&&
یا رب چه کسی یا رب اعجوبه ربانی
\\
هر دم ز تو زیب و فر از ما دل و جان و سر
&&
بازار چنین خوشتر خوش بدهی و بستانی
\\
از عشق تو جان بردن وز ما چو شکر مردن
&&
زهر از کف تو خوردن سرچشمه حیوانی
\\
\end{longtable}
\end{center}
