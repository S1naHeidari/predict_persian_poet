\begin{center}
\section*{غزل شماره ۷۷۶: عاشقان بر درت از اشک چو باران کارند}
\label{sec:0776}
\addcontentsline{toc}{section}{\nameref{sec:0776}}
\begin{longtable}{l p{0.5cm} r}
عاشقان بر درت از اشک چو باران کارند
&&
خوش به هر قطره دو صد گوهر جان بردارند
\\
همه از کار از آن روی معطل شده‌اند
&&
چو از آن سر نگری موی به مو در کارند
\\
گر چه بی‌دست و دهانند درختان چمن
&&
لیک سرسبز و فزاینده و دردی خوارند
\\
صد هزارند ولیکن همه یک نور شوند
&&
شمع‌ها یک صفتند ار به عدد بسیارند
\\
نورهاشان به هم اندرشده بی‌حد و قیاس
&&
چون برآید مه تو جمله به تو بسپارند
\\
چشم‌هاشان همه وامانده در بحر محیط
&&
لب فروبسته از آن موج که در سر دارند
\\
ای بسا جان سلیمان نهان همچو پری
&&
که به لشکرگهشان مور نمی‌آزارند
\\
هست اندر پس دل واقف از این جاسوسی
&&
کو بگوید همه اسرار گرش بفشارند
\\
بی کلیدیست که چون حلقه ز در بیرونند
&&
ور نه هر جزو از آن نقده کل انبارند
\\
این بدن تخت شه و چار طبایع پایش
&&
تاجداران فلک تخت به تو نگذارند
\\
شمس تبریز اگر تاج بقا می‌بخشد
&&
دل و جان را تو بشارت ده اگر بیدارند
\\
\end{longtable}
\end{center}
