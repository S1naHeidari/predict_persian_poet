\begin{center}
\section*{غزل شماره ۱۵۹۰: چون ز صورت برتر آمد آفتاب و اخترم}
\label{sec:1590}
\addcontentsline{toc}{section}{\nameref{sec:1590}}
\begin{longtable}{l p{0.5cm} r}
چون ز صورت برتر آمد آفتاب و اخترم
&&
از معانی در معانی تا روم من خوشترم
\\
در معانی گم شدستم همچنین شیرینتر است
&&
سوی صورت بازنایم در دو عالم ننگرم
\\
در معانی می گدازم تا شوم همرنگ او
&&
زانک معنی همچو آب و من در او چون شکرم
\\
دل نگیرد هیچ کس را از حیات جان خویش
&&
من از این معنی ز صورت یاد نارم لاجرم
\\
می خرامم من به باغ از باغ با روحانیان
&&
چون گل سرخ لطیف و تازه چون نیلوفرم
\\
کشتی تن را چو موجم تخته تخته بشکنم
&&
خویشتن را بسکلم چون خویشتن را لنگرم
\\
ور من از سختی دل در کار خود سستی کنم
&&
زود از دریا برآید شعله‌های آذرم
\\
همچو زر خندان خوشم اندر میان آتشش
&&
زانک گر ز آتش برآیم همچو زر من بفسرم
\\
من ز افسونی چو ماری سر نهادم بر خطش
&&
تا چه افتد ای برادر از خط او بر سرم
\\
من ز صورت سیر گشتم آمدم سوی صفات
&&
هر صفت گوید درآ این جا که بحر اخضرم
\\
چون سکندر ملک دارم شمس تبریزی ز لطف
&&
سوی لشکرهای معنی لاجرم سرلشکرم
\\
\end{longtable}
\end{center}
