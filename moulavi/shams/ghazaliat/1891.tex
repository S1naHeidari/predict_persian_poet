\begin{center}
\section*{غزل شماره ۱۸۹۱: ما دست تو را خواجه بخواهیم کشیدن}
\label{sec:1891}
\addcontentsline{toc}{section}{\nameref{sec:1891}}
\begin{longtable}{l p{0.5cm} r}
ما دست تو را خواجه بخواهیم کشیدن
&&
وز نیک و بدت پاک بخواهیم بریدن
\\
هر چند شب غفلت و مستیت دراز است
&&
ما بر همه چون صبح بخواهیم دمیدن
\\
در پرده ناموس و دغل چند گریزی
&&
نزدیک رسیده‌ست تو را پرده دریدن
\\
هر میوه که در باغ جهان بود همه پخت
&&
ای غوره چون سنگ نخواهی تو پزیدن
\\
رحم آر بر این جان که طپان است در این دام
&&
نشنود مگر گوش تو آواز طپیدن
\\
چشمی است تو را در دل و آن چشم به درد است
&&
پس چیست غم تو به جز آن چشم خلیدن
\\
چون می خلد آن چشم بجو دارو و درمان
&&
تا بازرهی از خلش و آب دویدن
\\
داروی دل و دیده نبوده‌ست و نباشد
&&
ای یوسف خوبان به جز از روی تو دیدن
\\
هین مخلص این را تو بفرما به تمامی
&&
که گفت تو و قول تو مزد است شنیدن
\\
\end{longtable}
\end{center}
