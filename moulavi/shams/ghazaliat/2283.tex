\begin{center}
\section*{غزل شماره ۲۲۸۳: ساقی فرخ رخ من جام چو گلنار بده}
\label{sec:2283}
\addcontentsline{toc}{section}{\nameref{sec:2283}}
\begin{longtable}{l p{0.5cm} r}
ساقی فرخ رخ من جام چو گلنار بده
&&
بهر من ار می‌ندهی بهر دل یار بده
\\
ساقی دلدار تویی چاره بیمار تویی
&&
شربت شادی و شفا زود به بیمار بده
\\
باده در آن جام فکن گردن اندیشه بزن
&&
هین دل ما را مشکن ای دل و دلدار بده
\\
باز کن آن میکده را ترک کن این عربده را
&&
عاشق تشنه زده را از خم خمار بده
\\
جان بهار و چمنی رونق سرو و سمنی
&&
هین که بهانه نکنی ای بت عیار بده
\\
پای چو در حیله نهی وز کف مستان بجهی
&&
دشمن ما شاد شود کوری اغیار بده
\\
غم مده و آه مده جز به طرب راه مده
&&
آه ز بیراه بود ره بگشا بار بده
\\
ما همه مخمور لقا تشنه سغراق بقا
&&
بهر گرو پیش سقا خرقه و دستار بده
\\
تشنه دیرینه منم گرم دل و سینه منم
&&
جام و قدح را بشکن بی‌حد و بسیار بده
\\
خود مه و مهتاب تویی ماهی این آب منم
&&
ماه به ماهی نرسد پس ز مه ادرار بده
\\
\end{longtable}
\end{center}
