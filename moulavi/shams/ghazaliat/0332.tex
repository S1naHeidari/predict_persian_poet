\begin{center}
\section*{غزل شماره ۳۳۲: این خانه که پیوسته در او بانگ چغانه‌ست}
\label{sec:0332}
\addcontentsline{toc}{section}{\nameref{sec:0332}}
\begin{longtable}{l p{0.5cm} r}
این خانه که پیوسته در او بانگ چغانه‌ست
&&
از خواجه بپرسید که این خانه چه خانه‌ست
\\
این صورت بت چیست اگر خانه کعبه‌ست
&&
وین نور خدا چیست اگر دیر مغانه‌ست
\\
گنجی‌ست در این خانه که در کون نگنجد
&&
این خانه و این خواجه همه فعل و بهانه‌ست
\\
بر خانه منه دست که این خانه طلسم‌ست
&&
با خواجه مگویید که او مست شبانه‌ست
\\
خاک و خس این خانه همه عنبر و مشک‌ست
&&
بانگ در این خانه همه بیت و ترانه‌ست
\\
فی الجمله هر آن کس که در این خانه رهی یافت
&&
سلطان زمینست و سلیمان زمانه‌ست
\\
ای خواجه یکی سر تو از این بام فروکن
&&
کاندر رخ خوب تو ز اقبال نشانه‌ست
\\
سوگند به جان تو که جز دیدن رویت
&&
گر ملک زمینست فسونست و فسانه‌ست
\\
حیران شده بستان که چه برگ و چه شکوفه‌ست
&&
واله شده مرغان که چه دامست و چه دانه‌ست
\\
این خواجه چرخست که چون زهره و ماه‌ست
&&
وین خانه عشق است که بی‌حد و کرانه‌ست
\\
چون آینه جان نقش تو در دل بگرفته‌ست
&&
دل در سر زلف تو فرورفته چو شانه‌ست
\\
در حضرت یوسف که زنان دست بریدند
&&
ای جان تو به من آی که جان آن میانه‌ست
\\
مستند همه خانه کسی را خبری نیست
&&
از هر کی درآید که فلانست و فلانه‌ست
\\
شومست بر آستانه مشین خانه درآ زود
&&
تاریک کند آنک ورا جاش ستانه‌ست
\\
مستان خدا گر چه هزارند یکی اند
&&
مستان هوا جمله دوگانه‌ست و سه گانه‌ست
\\
در بیشه شیران رو وز زخم میندیش
&&
کاندیشه ترسیدن اشکال زنانه‌ست
\\
کان جا نبود زخم همه رحمت و مهرست
&&
لیکن پس در وهم تو ماننده فانه‌ست
\\
در بیشه مزن آتش و خاموش کن ای دل
&&
درکش تو زبان را که زبان تو زبانه‌ست
\\
\end{longtable}
\end{center}
