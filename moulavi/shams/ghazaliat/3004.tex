\begin{center}
\section*{غزل شماره ۳۰۰۴: بزم و شراب لعل و خرابات و کافری}
\label{sec:3004}
\addcontentsline{toc}{section}{\nameref{sec:3004}}
\begin{longtable}{l p{0.5cm} r}
بزم و شراب لعل و خرابات و کافری
&&
ملک قلندرست و قلندر از او بری
\\
گویی قلندرم من و این دلپذیر نیست
&&
زیرا که آفریده نباشد قلندری
\\
تا کی عطارد از زحل آرد مدبری
&&
مریخ نیز چند زند زخم خنجری
\\
تا چند نعل ریز کند پیک ماه نیز
&&
تا چند زهره بخش کند جام احمری
\\
تا چند آفتاب به تف مطبخی کند
&&
بازار تنگ دارد بر خلق مشتری
\\
تا چند آب ریزد دولاب آسمان
&&
تا چند آب نشف کند برج آذری
\\
تا چند شب پناه حریفان بد شود
&&
تا چند روز پرده درد بر مستری
\\
تا چند دی برآرد از باغ‌ها دمار
&&
تا کی بهار دوزد دیباج اخضری
\\
زین فرقت و غریبی طبعم ملول شد
&&
ای مرغ روح وقت نیامد که برپری
\\
وین پر درشکسته پرخون خویش را
&&
سوی جناب مالک و مخدوم خود بری
\\
اندر زمین چه چفسی نی کوه و آهنی
&&
زیر فلک چه باشی نی ابر و اختری
\\
زان حسن آبدار چو تازه کنی جگر
&&
نی آب خضر جویی نی حوض کوثری
\\
ای آب و روغنی که گرفتار آمدی
&&
با آنچ در دلست نگویی چه درخوری
\\
\end{longtable}
\end{center}
