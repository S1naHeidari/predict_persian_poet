\begin{center}
\section*{غزل شماره ۲۴۳۹: دامن کشانم می‌کشد در بتکده عیاره‌ای}
\label{sec:2439}
\addcontentsline{toc}{section}{\nameref{sec:2439}}
\begin{longtable}{l p{0.5cm} r}
دامن کشانم می‌کشد در بتکده عیاره‌ای
&&
من همچو دامن می‌دوم اندر پی خون خواره‌ای
\\
یک لحظه هستم می‌کند یک لحظه پستم می‌کند
&&
یک لحظه مستم می‌کند خودکامه‌ای خماره‌ای
\\
چون مهره‌ام در دست او چون ماهیم در شست او
&&
بر چاه بابل می‌تنم از غمزه سحاره‌ای
\\
لاهوت و ناسوت من او هاروت و ماروت من او
&&
مرجان و یاقوت من او بر رغم هر بدکاره‌ای
\\
در صورت آب خوشی ماهی چو برج آتشی
&&
در سینه دلبر دلی چون مرمری چون خاره‌ای
\\
اسرار آن گنج جهان با تو بگویم در نهان
&&
تو مهلتم ده تا که من با خویش آیم پاره‌ای
\\
روزی ز عکس روی او بردم سبوی تا جوی او
&&
دیدم ز عکس نور او در آب جو استاره‌ای
\\
گفتم که آنچ از آسمان جستم بدیدم در زمین
&&
ناگاه فضل ایزدی شد چاره بیچاره‌ای
\\
شکر است در اول صفم شمشیر هندی در کفم
&&
در باغ نصرت بشکفم از فر گل رخساره‌ای
\\
آن رفت کز رنج و غمان خم داده بودم چون کمان
&&
بود این تنم چون استخوان در دست هر سگساره‌ای
\\
خورشید دیدم نیم شب زهره درآمد در طرب
&&
در شهر خویش آمد عجب سرگشته‌ای آواره‌ای
\\
اندر خم طغرای کن نو گشت این چرخ کهن
&&
عیسی درآمد در سخن بربسته در گهواره‌ای
\\
در دل نیفتد آتشی در پیش ناید ناخوشی
&&
سر برنیارد سرکشی نفسی نماند اماره‌ای
\\
خوش شد جهان عاشقان آمد قران عاشقان
&&
وارست جان عاشقان از مکر هر مکاره‌ای
\\
جان لطیف بانمک بر عرش گردد چون ملک
&&
نبود دگر زیر فلک مانند هر سیاره‌ای
\\
مانند موران عقل و جان گشتند در طاس جهان
&&
آن رخنه جویان را نهان وا شد در و درساره‌ای
\\
بی‌خار گردد شاخ گل زیرا که ایمن شد ز ذل
&&
زیرا نماندش دشمنی گل چین و گل افشاره‌ای
\\
خاموش خاموش ای زبان همچون زبان سوسنان
&&
مانند نرگس چشم شو در باغ کن نظاره‌ای
\\
\end{longtable}
\end{center}
