\begin{center}
\section*{غزل شماره ۱۱۲: ز روی تست عید آثار ما را}
\label{sec:0112}
\addcontentsline{toc}{section}{\nameref{sec:0112}}
\begin{longtable}{l p{0.5cm} r}
ز روی تست عید آثار ما را
&&
بیا ای عید و عیدی آر ما را
\\
تو جان عید و از روی تو جانا
&&
هزاران عید در اسرار ما را
\\
چو ما در نیستی سر درکشیدیم
&&
نگیرد غصه دستار ما را
\\
چو ما بر خویشتن اغیار گشتیم
&&
نباشد غصه اغیار ما را
\\
شما را اطلس و شعر خیالی
&&
خیال خوب آن دلدار ما را
\\
کتاب مکر و عیاری شما را
&&
عتاب دلبر عیار ما را
\\
شما را عید در سالی دو بارست
&&
دو صد عیدست هر دم کار ما را
\\
شما را سیم و زر بادا فراوان
&&
جمال خالق جبار ما را
\\
شما را اسب تازی باد بی‌حد
&&
براق احمد مختار ما را
\\
اگر عالم همه عیدست و عشرت
&&
برو عالم شما را یار ما را
\\
بیا ای عید اکبر شمس تبریز
&&
به دست این و آن مگذار ما را
\\
چو خاموشانه عشقت قوی شد
&&
سخن کوتاه شد این بار ما را
\\
\end{longtable}
\end{center}
