\begin{center}
\section*{غزل شماره ۸۰۶: یا رب این بوی خوش از روضه جان می‌آید}
\label{sec:0806}
\addcontentsline{toc}{section}{\nameref{sec:0806}}
\begin{longtable}{l p{0.5cm} r}
یا رب این بوی خوش از روضه جان می‌آید
&&
یا نسیمیست کز آن سوی جهان می‌آید
\\
یا رب این آب حیات از چه وطن می‌جوشد
&&
یا رب این نور صفات از چه مکان می‌آید
\\
عجب این غلغله از جوق ملک می‌خیزد
&&
عجب این قهقهه از حور جنان می‌آید
\\
چه سماعست که جان رقص کنان می‌گردد
&&
چه صفیرست که دل بال زنان می‌آید
\\
چه عروسیست چه کابین که فلک چون تتقیست
&&
ماه با این طبق زر به نشان می‌آید
\\
چه شکارست که این تیر قضا پرانست
&&
ور چنین نیست چرا بانگ کمان می‌آید
\\
مژده مژده همه عشاق بکوبید دو دست
&&
کانک از دست بشد دست زنان می‌آید
\\
از حصار فلکی بانگ امان می‌خیزد
&&
وز سوی بحر چنین موج گمان می‌آید
\\
چشم اقبال به اقبال شما مخمورست
&&
این دلیلست که از عین عیان می‌آید
\\
برهیدیت از این عالم قحطی که در او
&&
از برای دو سه نان زخم سنان می‌آید
\\
خوشتر از جان چه بود جان برود باک مدار
&&
غم رفتن چه خوری چون به از آن می‌آید
\\
هر کسی در عجبی و عجب من اینست
&&
کو نگنجد به میان چون به میان می‌آید
\\
بس کنم گر چه که رمزست بیانش نکنم
&&
خود بیان را چه کنیم جان بیان می‌آید
\\
\end{longtable}
\end{center}
