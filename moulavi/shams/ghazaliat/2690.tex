\begin{center}
\section*{غزل شماره ۲۶۹۰: چو عشق آمد که جان با من سپاری}
\label{sec:2690}
\addcontentsline{toc}{section}{\nameref{sec:2690}}
\begin{longtable}{l p{0.5cm} r}
چو عشق آمد که جان با من سپاری
&&
چرا زوتر نگویی کآری آری
\\
جهان سوزید ز آتش‌های خوبان
&&
جمال عشق و روی عشق باری
\\
چو جان بیند جمال عشق گوید
&&
شدم از دست و دست از من نداری
\\
بدیدم عشق را چون برج نوری
&&
درون برج نوری اه چه ناری
\\
چو اشترمرغ جان‌ها گرد آن برج
&&
غذاشان آتشی بس خوشگواری
\\
ز دور استاده جانم در تماشا
&&
به پیش آمد مرا خوش شهسواری
\\
یکی رویی چو ماهی ماه سوزی
&&
یکی مریخ چشمی پرخماری
\\
که جان‌ها پیش روی او خیالی
&&
جهان در پای اسب او غباری
\\
همی‌رست از غبار نعل اسبش
&&
بیابان در بیابان خوش عذاری
\\
همی‌تازید عقلم اندک اندک
&&
همی‌پرید از سر چون طیاری
\\
همین دانم دگر از من مپرسید
&&
که صد من نیست آن جا در شماری
\\
من آن آبم که ریگ عشق خوردش
&&
چه ریگی بلک بحر بی‌کناری
\\
چو لاله کفته‌ای در شهر تبریز
&&
شدم بر دست شمس الدین نگاری
\\
\end{longtable}
\end{center}
