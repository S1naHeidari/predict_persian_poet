\begin{center}
\section*{غزل شماره ۲۶۷۴: هلا ای آب حیوان از نوایی}
\label{sec:2674}
\addcontentsline{toc}{section}{\nameref{sec:2674}}
\begin{longtable}{l p{0.5cm} r}
هلا ای آب حیوان از نوایی
&&
همی‌گردان مرا چون آسیایی
\\
چنین می‌کن که تا بادا چنین باد
&&
پریشان دل به جایی من به جایی
\\
نجنبد شاخ و برگی جز به بادی
&&
نپرد برگ که بی‌کهربایی
\\
چو کاهی جز به بادی می‌نجنبد
&&
کجا جنبد جهانی بی‌هوایی
\\
همه اجزای عالم عاشقانند
&&
و هر جزو جهان مست لقایی
\\
ولیک اسرار خود با تو نگویند
&&
نشاید گفت سر جز با سزایی
\\
چراخواران چراشان هم چراخوار
&&
ز کاسه و خوان شیرین کدخدایی
\\
نه موران با سلیمان راز گفتند
&&
نه با داوود می‌زد که صدایی
\\
اگر این آسمان عاشق نبودی
&&
نبودی سینه او را صفایی
\\
وگر خورشید هم عاشق نبودی
&&
نبودی در جمال او ضیایی
\\
زمین و کوه اگر نه عاشق اندی
&&
نرستی از دل هر دو گیاهی
\\
اگر دریا ز عشق آگه نبودی
&&
قراری داشتی آخر به جایی
\\
تو عاشق باش تا عاشق شناسی
&&
وفا کن تا ببینی باوفایی
\\
نپذرفت آسمان بار امانت
&&
که عاشق بود و ترسید از خطایی
\\
\end{longtable}
\end{center}
