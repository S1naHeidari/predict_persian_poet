\begin{center}
\section*{غزل شماره ۲۰۶۰: یک غزل آغاز کن بر صفت حاضران}
\label{sec:2060}
\addcontentsline{toc}{section}{\nameref{sec:2060}}
\begin{longtable}{l p{0.5cm} r}
یک غزل آغاز کن بر صفت حاضران
&&
ای رخ تو همچو شمع خیز درآ در میان
\\
نور ده آن شمع را روح ده این جمع را
&&
از دوزخ همچو شمع وز قدح همچو جان
\\
سوی قدح دست کن ما همه را مست کن
&&
ز آنک کسی خوش نشد تا نشد از خود نهان
\\
چون شدی از خود نهان زود گریز از جهان
&&
روی تو واپس مکن جانب خود هان و هان
\\
این سخن همچو تیر راست کشش سوی گوش
&&
تا نکشی سوی گوش کی بجهد از کمان
\\
بس کن از اندیشه بس کو گودت هر نفس
&&
کای عجب آن را چه شد اه چه کنم کو فلان
\\
\end{longtable}
\end{center}
