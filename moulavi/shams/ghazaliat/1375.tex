\begin{center}
\section*{غزل شماره ۱۳۷۵: بازآمدم چون عید نو تا قفل زندان بشکنم}
\label{sec:1375}
\addcontentsline{toc}{section}{\nameref{sec:1375}}
\begin{longtable}{l p{0.5cm} r}
بازآمدم چون عید نو تا قفل زندان بشکنم
&&
وین چرخ مردم خوار را چنگال و دندان بشکنم
\\
هفت اختر بی‌آب را کاین خاکیان را می خورند
&&
هم آب بر آتش زنم هم بادهاشان بشکنم
\\
از شاه بی‌آغاز من پران شدم چون باز من
&&
تا جغد طوطی خوار را در دیر ویران بشکنم
\\
ز آغاز عهدی کرده‌ام کاین جان فدای شه کنم
&&
بشکسته بادا پشت جان گر عهد و پیمان بشکنم
\\
امروز همچون آصفم شمشیر و فرمان در کفم
&&
تا گردن گردن کشان در پیش سلطان بشکنم
\\
روزی دو باغ طاغیان گر سبز بینی غم مخور
&&
چون اصل‌های بیخشان از راه پنهان بشکنم
\\
من نشکنم جز جور را یا ظالم بدغور را
&&
گر ذره‌ای دارد نمک گیرم اگر آن بشکنم
\\
هر جا یکی گویی بود چوگان وحدت وی برد
&&
گویی که میدان نسپرد در زخم چوگان بشکنم
\\
گشتم مقیم بزم او چون لطف دیدم عزم او
&&
گشتم حقیر راه او تا ساق شیطان بشکنم
\\
چون در کف سلطان شدم یک حبه بودم کان شدم
&&
گر در ترازویم نهی می دان که میزان بشکنم
\\
چون من خراب و مست را در خانه خود ره دهی
&&
پس تو ندانی این قدر کاین بشکنم آن بشکنم
\\
گر پاسبان گوید که هی بر وی بریزم جام می
&&
دربان اگر دستم کشد من دست دربان بشکنم
\\
چرخ ار نگردد گرد دل از بیخ و اصلش برکنم
&&
گردون اگر دونی کند گردون گردان بشکنم
\\
خوان کرم گسترده‌ای مهمان خویشم برده‌ای
&&
گوشم چرا مالی اگر من گوشهٔ نان بشکنم
\\
نی نی منم سرخوان تو سرخیل مهمانان تو
&&
جامی دو بر مهمان کنم تا شرم مهمان بشکنم
\\
ای که میان جان من تلقین شعرم می کنی
&&
گر تن زنم خامش کنم ترسم که فرمان بشکنم
\\
از شمس تبریزی اگر باده رسد مستم کند
&&
من لاابالی وار خود استون کیوان بشکنم
\\
\end{longtable}
\end{center}
