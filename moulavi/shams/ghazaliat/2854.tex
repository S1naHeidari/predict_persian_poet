\begin{center}
\section*{غزل شماره ۲۸۵۴: برسید لک لک جان که بهار شد کجایی}
\label{sec:2854}
\addcontentsline{toc}{section}{\nameref{sec:2854}}
\begin{longtable}{l p{0.5cm} r}
برسید لک لک جان که بهار شد کجایی
&&
بشکفت جمله عالم گل و برگ جان فزایی
\\
رخ یوسفان ببینی که ز چاه سر برآرد
&&
همه گلرخان ببینی که کنند خودنمایی
\\
ثمرات دل شکسته به درون خاک بسته
&&
بگشاده دیده دیده ز بلای دی رهایی
\\
خضر و سمن چو رندان بشکسته‌اند زندان
&&
گل و لاله شاد و خندان ز سعادت عطایی
\\
همه مریمان کامل همه بکر و گشته حامل
&&
بنموده عارفان دل به جناب کبریایی
\\
چو شکوفه کرد به بستان ز ره دهن چو مستان
&&
تو نصیب خویش بستان ز زمانه گر ز مایی
\\
به مثال گربه هر یک به دهان گرفته کودک
&&
سوی مادران گلشن به نظاره چون نیایی
\\
بنگر به مرغ خوش پر چو خطیب فوق منبر
&&
به ثنا و حمد داور بگرفته خوش نوایی
\\
\end{longtable}
\end{center}
