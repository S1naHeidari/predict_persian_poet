\begin{center}
\section*{غزل شماره ۲۴۸۳: تلخ کنی دهان من قند به دیگران دهی}
\label{sec:2483}
\addcontentsline{toc}{section}{\nameref{sec:2483}}
\begin{longtable}{l p{0.5cm} r}
تلخ کنی دهان من قند به دیگران دهی
&&
نم ندهی به کشت من آب به این و آن دهی
\\
جان منی و یار من دولت پایدار من
&&
باغ من و بهار من باغ مرا خزان دهی
\\
یا جهت ستیز من یا جهت گریز من
&&
وقت نبات ریز من وعده و امتحان دهی
\\
عود که جود می‌کند بهر تو دود می‌کند
&&
شیر سجود می‌کند چون به سگ استخوان دهی
\\
برگذرم ز نه فلک گر گذری به کوی من
&&
پای نهم بر آسمان گر به سرم امان دهی
\\
عقل و خرد فقیر تو پرورشش ز شیر تو
&&
چون نشود ز تیر تو آنک بدو کمان دهی
\\
در دو جهان بننگرد آنک بدو تو بنگری
&&
خسرو خسروان شود گر به گدا تو نان دهی
\\
جمله تن شکر شود هر که بدو شکر دهی
&&
لقمه کند دو کون را آنک تواش دهان دهی
\\
گشتم جمله شهرها نیست شکر مگر تو را
&&
با تو مکیس چون کنم گر تو شکر گران دهی
\\
گه بکشی گران دهی گه همه رایگان دهی
&&
یک نفسی چنین دهی یک نفسی چنان دهی
\\
مفخر مهر و مشتری در تبریز شمس دین
&&
زنده شود دل قمر گر به قمر قران دهی
\\
\end{longtable}
\end{center}
