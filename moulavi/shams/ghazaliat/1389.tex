\begin{center}
\section*{غزل شماره ۱۳۸۹: ای پاک رو چون جام جم وز عشق آن مه متهم}
\label{sec:1389}
\addcontentsline{toc}{section}{\nameref{sec:1389}}
\begin{longtable}{l p{0.5cm} r}
ای پاک رو چون جام جم وز عشق آن مه متهم
&&
این مرگ خود پیدا کند پاکی تو را کم خور تو غم
\\
ای جان من با جان تو جویای در در بحر خون
&&
تا در که را پیدا شود پیدا شود ای جان عم
\\
من چون شوم کوته نظر در عشق آن بحر گهر
&&
کز ساحل دریای جان آید بشارت دم به دم
\\
من ترک فضل و فاضلی کردم به عشق از کاهلی
&&
کز عشق شه کم بیشی است وز عشق شه بیشی است کم
\\
بیخ دل از صفرای او می خورد زد زردی به رخ
&&
چون دیده عشقش بر رخم زد بر رخم آن شه رقم
\\
تلوین این رخسار بین در عشق بی‌تلوین شهی
&&
گاه از غمش چون زعفران گاه از خجالت چون بقم
\\
من فانی مطلق شدم تا ترجمان حق شدم
&&
گر مست و هشیارم ز من کس نشنود خود بیش و کم
\\
بازار مصر اندرشدم تا جانب مهتر شدم
&&
دیدم یکی یوسف رخی گفتم به غفلت ذابکم
\\
گفتا عزیز مصر گر تو عاشقی بخشیدمت
&&
من غایه الاحسان او من جوده او من کرم
\\
من قدر آن نشناختم آن را هوس پنداشتم
&&
یا حسرتی من هجره یا غبنتی یا ذا الندم
\\
ای صد محال از قوتش گشته حقیقت عین حال
&&
ما کان فی الدارین قط و الله مثل ذالقدم
\\
تبریز این تعظیم را تو از الست آورده‌ای
&&
از مفخر من شمس دین از اول جف القلم
\\
\end{longtable}
\end{center}
