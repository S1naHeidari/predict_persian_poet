\begin{center}
\section*{غزل شماره ۲۹۱۰: باز گردد عاقبت این در بلی}
\label{sec:2910}
\addcontentsline{toc}{section}{\nameref{sec:2910}}
\begin{longtable}{l p{0.5cm} r}
باز گردد عاقبت این در بلی
&&
رو نماید یار سیمین بر بلی
\\
ساقی ما یاد این مستان کند
&&
بار دیگر با می و ساغر بلی
\\
نوبهار حسن آید سوی باغ
&&
بشکفد آن شاخه‌های تر بلی
\\
طاق‌های سبز چون بندد چمن
&&
جفت گردد ورد و نیلوفر بلی
\\
دامن پرخاک و خاشاک زمین
&&
پر شود از مشک و از عنبر بلی
\\
آن بر سیمین و این روی چو زر
&&
اندرآمیزند سیم و زر بلی
\\
این سر مخمور اندیشه پرست
&&
مست گردد زان می احمر بلی
\\
این دو چشم اشکبار نوحه گر
&&
روشنی یابد از آن منظر بلی
\\
گوش‌ها که حلقه در گوش وی است
&&
حلقه‌ها یابند از آن زرگر بلی
\\
شاهد جان چون شهادت عرضه کرد
&&
یابد ایمان این دل کافر بلی
\\
چون براق عشق از گردون رسید
&&
وارهد عیسی جان زین خر بلی
\\
جمله خلق جهان در یک کس است
&&
او بود از صد جهان بهتر بلی
\\
من خمش کردم ولیکن در دلم
&&
تا ابد روید نی و شکر بلی
\\
\end{longtable}
\end{center}
