\begin{center}
\section*{غزل شماره ۷۱۴: اول نظر ار چه سرسری بود}
\label{sec:0714}
\addcontentsline{toc}{section}{\nameref{sec:0714}}
\begin{longtable}{l p{0.5cm} r}
اول نظر ار چه سرسری بود
&&
سرمایه و اصل دلبری بود
\\
آن جام شراب ارغوانی
&&
وان آب حیات زندگانی
\\
جمعیت جان‌های خرم
&&
در سایه آن دو زلف درهم
\\
از رنگ تو گشته‌ایم بی‌رنگ
&&
زان سوی جهان هزار فرسنگ
\\
در عشق پدید شد سپاهی
&&
در سایه چتر پادشاهی
\\
همچون مه نو ز غم خمیدن
&&
چون سایه به رو و سر دویدن
\\
آن مه که بسوخت مشتری را
&&
بشکست بتان آزری را
\\
گر هجده هزار عالم ای جان
&&
پر گشت ز قال و قال ای جان
\\
گر داد طریق عشق دادیم
&&
ور زان مه و آفتاب شادیم
\\
آن دم که ز ننگ خویش رستیم
&&
وان می که ز بوش بود مستیم
\\
باغی که حیات گشت وصلش
&&
خوشتر ز بهار و چار فصلش
\\
\end{longtable}
\end{center}
