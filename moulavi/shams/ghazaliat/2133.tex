\begin{center}
\section*{غزل شماره ۲۱۳۳: بیدار شو بیدار شو هین رفت شب بیدار شو}
\label{sec:2133}
\addcontentsline{toc}{section}{\nameref{sec:2133}}
\begin{longtable}{l p{0.5cm} r}
بیدار شو بیدار شو هین رفت شب بیدار شو
&&
بیزار شو بیزار شو وز خویش هم بیزار شو
\\
در مصر ما یک احمقی نک می‌فروشد یوسفی
&&
باور نمی‌داری مرا اینک سوی بازار شو
\\
بی‌چون تو را بی‌چون کند روی تو را گلگون کند
&&
خار از کفت بیرون کند وآنگه سوی گلزار شو
\\
مشنو تو هر مکر و فسون خون را چرا شویی به خون
&&
همچون قدح شو سرنگون و آن گاه دردی خوار شو
\\
در گردش چوگان او چون گوی شو چون گوی شو
&&
وز بهر نقل کرکسش مردار شو مردار شو
\\
آمد ندای آسمان آمد طبیب عاشقان
&&
خواهی که آید پیش تو بیمار شو بیمار شو
\\
این سینه را چون غار دان خلوتگه آن یار دان
&&
گر یار غاری هین بیا در غار شو در غار شو
\\
تو مرد نیک ساده‌ای زر را به دزدان داده‌ای
&&
خواهی بدانی دزد را طرار شو طرار شو
\\
خاموش وصف بحر و در کم گوی در دریای او
&&
خواهی که غواصی کنی دم دار شو دم دار شو
\\
\end{longtable}
\end{center}
