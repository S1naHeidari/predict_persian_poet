\begin{center}
\section*{غزل شماره ۱۹۵۲: هست ما را هر زمانی از نگار راستین}
\label{sec:1952}
\addcontentsline{toc}{section}{\nameref{sec:1952}}
\begin{longtable}{l p{0.5cm} r}
هست ما را هر زمانی از نگار راستین
&&
لقمه‌ای اندر دهان و دیگری در آستین
\\
این حد خوبی نباشد ای خدایا چیست این
&&
هیچ سروی این ندارد خوش قد و بالا است این
\\
این چنین خورشید پیدا چونک پنهان می شود
&&
او چنین پنهان ز عالم از برای ماست این
\\
جمع خواهد آن بت و تنهاروان خود دیگرند
&&
هر کجا خوبی بود او طالب غوغاست این
\\
شمس تبریز ار چه جانی گر چو جان پنهان شوی
&&
بر دلم تهمت نشیند کز کجا برخاست این
\\
\end{longtable}
\end{center}
