\begin{center}
\section*{غزل شماره ۱۲۵۵: اندک اندک راه زد سیم و زرش}
\label{sec:1255}
\addcontentsline{toc}{section}{\nameref{sec:1255}}
\begin{longtable}{l p{0.5cm} r}
اندک اندک راه زد سیم و زرش
&&
مرگ و جسک نو فتاد اندر سرش
\\
عشق گردانید با او پوستین
&&
می‌گریزد خواجه از شور و شرش
\\
اندک اندک روی سرخش زرد شد
&&
اندک اندک خشک شد چشم ترش
\\
وسوسه و اندیشه بر وی در گشاد
&&
راند عشق لاابالی از درش
\\
اندک اندک شاخ و برگش خشک گشت
&&
چون بریده شد رگ بیخ آورش
\\
اندک اندک دیو شد لاحول گو
&&
سست شد در عاشقی بال و پرش
\\
اندک اندک گشت صوفی خرقه دوز
&&
رفت وجد و حالت خرقه درش
\\
عشق داد و دل بر این عالم نهاد
&&
در برش زین پس نیاید دلبرش
\\
زان همی‌جنباند سر او سست سست
&&
کآمد اندر پا و افتاد اکثرش
\\
بهر او پر می‌کنم من ساغری
&&
گر بنوشد برجهاند ساغرش
\\
دست‌ها زان سان برآرد کآسمان
&&
بشنود آواز الله اکبرش
\\
میر ما سیرست از این گفت و ملول
&&
درکشان اندر حدیث دیگرش
\\
کشته عشقم نترسم از امیر
&&
هر کی شد کشته چه خوف از خنجرش
\\
بترین مرگ‌ها بی‌عشقی است
&&
بر چه می‌لرزد صدف بر گوهرش
\\
برگ‌ها لرزان ز بیم خشکی اند
&&
تا نگردد خشک شاخ اخضرش
\\
در تک دریا گریزد هر صدف
&&
تا بنربایند گوهر از برش
\\
چون ربودند از صدف دانه گهر
&&
بعد از آن چه آب خوش چه آذرش
\\
آن صدف بی‌چشم و بی‌گوش است شاد
&&
در به باطن درگشاده منظرش
\\
گر بماند عاشقی از کاروان
&&
بر سر ره خضر آید رهبرش
\\
خواجه می‌گرید که ماند از قافله
&&
لیک می‌خندد خر اندر آخرش
\\
عشق را بگذاشت و دم خر گرفت
&&
لاجرم سرگین خر شد عنبرش
\\
ملک را بگذاشت و بر سرگین نشست
&&
لاجرم شد خرمگس سرلشکرش
\\
خرمگس آن وسوسه‌ست و آن خیال
&&
که همی خارش دهد همچون گرش
\\
گر ندارد شرم و واناید از این
&&
وانمایم شاخ‌های دیگرش
\\
تو مکن شاخش چو مرد اندر خری
&&
گاو خیزد با سه شاخ از محشرش
\\
\end{longtable}
\end{center}
