\begin{center}
\section*{غزل شماره ۱۸۳۰: تا تو حریف من شدی ای مه دلستان من}
\label{sec:1830}
\addcontentsline{toc}{section}{\nameref{sec:1830}}
\begin{longtable}{l p{0.5cm} r}
تا تو حریف من شدی ای مه دلستان من
&&
همچو چراغ می جهد نور دل از دهان من
\\
ذره به ذره چون گهر از تف آفتاب تو
&&
دل شده‌ست سر به سر آب و گل گران من
\\
پیشتر آ دمی بنه آن بر و سینه بر برم
&&
گر چه که در یگانگی جان تو است جان من
\\
در عجبی فتم که این سایه کیست بر سرم
&&
فضل توام ندا زند کان من است آن من
\\
از تو جهان پربلا همچو بهشت شد مرا
&&
تا چه شود ز لطف تو صورت آن جهان من
\\
تاج من است دست تو چون بنهیش بر سرم
&&
طره توست چون کمر بسته بر این میان من
\\
عشق برید کیسه‌ام گفتم هی چه می کنی
&&
گفت تو را نه بس بود نعمت بی‌کران من
\\
برگ نداشتم دلم می لرزید برگ وش
&&
گفت مترس کآمدی در حرم امان من
\\
در برت آن چنان کشم کز بر و برگ وارهی
&&
تا همه شب نظر کنی پیش طرب کنان من
\\
بر تو زنم یگانه‌ای مست ابد کنم تو را
&&
تا که یقین شود تو را عشرت جاودان من
\\
سینه چو بوستان کند دمدمه بهار من
&&
روی چو گلستان کند خمر چو ارغوان من
\\
\end{longtable}
\end{center}
