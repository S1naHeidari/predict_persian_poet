\begin{center}
\section*{غزل شماره ۸۶۳: آتش پریر گفت نهانی به گوش دود}
\label{sec:0863}
\addcontentsline{toc}{section}{\nameref{sec:0863}}
\begin{longtable}{l p{0.5cm} r}
آتش پریر گفت نهانی به گوش دود
&&
کز من نمی‌شکیبد و با من خوش است عود
\\
قدر من او شناسد و شکر من او کند
&&
کاندر فنای خویش بدیدست عود سود
\\
سر تا به پای عود گره بود بند بند
&&
اندر گشایش عدم آن عقدها گشود
\\
ای یار شعله خوار من اهلا و مرحبا
&&
ای فانی و شهید من و مفخر شهود
\\
بنگر که آسمان و زمین رهن هستی اند
&&
اندر عدم گریز از این کور و زان کبود
\\
هر جان که می‌گریزد از فقر و نیستی
&&
نحسی بود گریزان از دولت و سعود
\\
بی محو کس ز لوح عدم مستفید نیست
&&
صلحی فکن میان من و محو ای ودود
\\
آن خاک تیره تا نشد از خویشتن فنا
&&
نی در فزایش آمد و نی رست از رکود
\\
تا نطفه نطفه بود و نشد محو از منی
&&
نی قد سرو یافت نه زیبایی خدود
\\
در معده چون بسوزد آن نان و نان خورش
&&
آن گاه عقل و جان شود و حسرت حسود
\\
سنگ سیاه تا نشد از خویشتن فنا
&&
نی زر و نقره گشت و نی ره یافت در نقود
\\
خواریست و بندگیست پس آنگه شهنشهیست
&&
اندر نماز قامه بود آنگهی قعود
\\
عمری بیازمودی هستی خویش را
&&
یک بار نیستی را هم باید آزمود
\\
طاق و طرنب فقر و فنا هم گزاف نیست
&&
هر جا که دود آمد بی‌آتشی نبود
\\
گر نیست عشق را سر ما و هوای ما
&&
چون از گزافه او دل و دستار ما ربود
\\
عشق آمدست و گوش کشانمان همی‌کشد
&&
هر صبح سوی مکتب یوفون بالعهود
\\
از چشممؤمنآب ندم می‌کند روان
&&
تا سینه را بشوید از کینه و جحود
\\
تو خفته‌ای و آب خضر بر تو می‌زند
&&
کز خواب برجه و بستان ساغر خلود
\\
باقیش عشق گوید با تو نهان ز من
&&
ز اصحاب کهف باش هم ایقاظ و رقود
\\
\end{longtable}
\end{center}
