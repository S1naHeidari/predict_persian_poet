\begin{center}
\section*{غزل شماره ۲۱۳۲: مستی ببینی رازدان می‌دانک باشد مست او}
\label{sec:2132}
\addcontentsline{toc}{section}{\nameref{sec:2132}}
\begin{longtable}{l p{0.5cm} r}
مستی ببینی رازدان می‌دانک باشد مست او
&&
هستی ببینی زنده دل می‌دانک باشد هست او
\\
گر سر ببینی پرطرب پر گشته از وی روز و شب
&&
می‌دانک آن سر را یقین خاریده باشد دست او
\\
عالم چو ضد یک دگر در قصد خون و شور و شر
&&
لیکن نیارد دم زدن از هیبت پابست او
\\
هر دم یکی را می‌دهد تا چون درختی برجهد
&&
حیران شود دیو و پری در خیز و در برج است او
\\
سبلت قوی مالیده‌ای از شیر نقشی دیده‌ای
&&
ای فربه از بایست خود باری ببین بایست او
\\
زو قالبت پیوسته شد پیوسته گردد حالتت
&&
ای رغبت پیوندها از رحمت پیوست او
\\
ای خوش بیابان که در او عشق است تازان سو به سو
&&
جز حق نباشد فوق او جز فقر نبود پست او
\\
شست سخن کم باف چون صیدت نمی‌گردد زبون
&&
تا او بگیرد صیدها ای صید مست شست او
\\
\end{longtable}
\end{center}
