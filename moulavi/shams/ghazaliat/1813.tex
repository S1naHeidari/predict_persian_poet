\begin{center}
\section*{غزل شماره ۱۸۱۳: کو خر من کو خر من پار بمرد آن خر من}
\label{sec:1813}
\addcontentsline{toc}{section}{\nameref{sec:1813}}
\begin{longtable}{l p{0.5cm} r}
کو خر من کو خر من پار بمرد آن خر من
&&
شکر خدا را که خرم برد صداع از سر من
\\
گاو اگر نیز رود تا برود غم نخورم
&&
نیست ز گاو و شکمش بوی خوش عنبر من
\\
گاو و خری گر برود باد ابد در دو جهان
&&
دلبر من دلبر من دلبر من دلبر من
\\
حلقه به گوش است خرم گوش خر و حلقه زر
&&
حیف نگر حیف نگر وازر من وازر من
\\
سر کشد و ره نرود ناز کند جو نخورد
&&
جز تل سرگین نبود خدمت او بر در من
\\
گاو بر این چرخ بر این گاو دگر زیر زمین
&&
زین دو اگر من بجهم بخت بود چنبر من
\\
رفتم بازار خران این سو و آن سو نگران
&&
از خر و از بنده خر سیر شد این منظر من
\\
گفت کسی چون خر تو مرد خری هست بخر
&&
گفتم خاموش که خر بود به ره لنگر من
\\
\end{longtable}
\end{center}
