\begin{center}
\section*{غزل شماره ۱۹۹۷: هر که را گشت سر از غایت برگردیدن}
\label{sec:1997}
\addcontentsline{toc}{section}{\nameref{sec:1997}}
\begin{longtable}{l p{0.5cm} r}
هر که را گشت سر از غایت برگردیدن
&&
ساکنان را همه سرگشته تواند دیدن
\\
هر کی از ضعف خود اندر رخ مردان نگرد
&&
بر دو چشم کژ او فرض بود خندیدن
\\
هر کی صفرا شودش غالب از شیرینی
&&
تلخ گردد دهنش گاه شکر خاییدن
\\
عقل میدانی او خود خر لنگ افتاده است
&&
در براق احدی دید کسی لنگیدن
\\
ای کسی کز حدثان در حدثی افتادی
&&
چون چنینی تو روا نیست تو را جنبیدن
\\
باید اول ز حدث سوی قدم پیوستن
&&
وانگهان بر قدمش نیمچه‌ای ببریدن
\\
خانه شاه بزن نقب اگر نقب زنی
&&
گوهری دزد از آن خانه گه دزدیدن
\\
من علامات گهر گفتم لیکن چه کنم
&&
کورموشی چو ندارد نظر بگزیدن
\\
شمس تبریز سخن‌های تو می بخشد چشم
&&
لیک کو گوش که داند سخنت بشنیدن
\\
\end{longtable}
\end{center}
