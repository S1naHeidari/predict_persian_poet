\begin{center}
\section*{غزل شماره ۱۰۳۳: ای دیده مرا بر در واپس بکشیده سر}
\label{sec:1033}
\addcontentsline{toc}{section}{\nameref{sec:1033}}
\begin{longtable}{l p{0.5cm} r}
ای دیده مرا بر در واپس بکشیده سر
&&
باز از طرفی پنهان بنموده رخ عبهر
\\
یک لحظه سلف دیده کاین جایم تا دانی
&&
بر حیرت من گاهی خندیده تو چون شکر
\\
در بسته به روی من یعنی که برو واپس
&&
بر بام شده در پی یعنی نمطی دیگر
\\
سر را تو چنان کرده رو رو که رقیب آمد
&&
من سجده کنان گشته یعنی که از این بگذر
\\
من در تو نظر کرده تو چشم بدزدیده
&&
زان ناز و کرشم تو صد فتنه و شور و شر
\\
تو دست گزان بر من کاین جمله ز دست تو
&&
من بوسه زنان گشته بر خاک به عذر اندر
\\
کی باشد کان بوسه بر لعل لبت یابم
&&
وان گاه تو بخراشی رخساره چون زعفر
\\
ای کافر زلف تو شاه حشم زنگی
&&
فریاد که ایمان شد اندر سر تو کافر
\\
چون طره بیفشانی مشک افتد در پایت
&&
چون جعد براندازی خطیت دهد عنبر
\\
احسنت زهی نقشی کز عطسه او جان شد
&&
ای کشته به پیش تو صد مانی و صد آزر
\\
ناگه ز جمال تو یک برق برون جسته
&&
تا محو شد این خانه هم بام فنا هم در
\\
در عین فنا گفتم ای شاه همه شاهان
&&
بگداخت‌همی نقشی بفسرده بدین آذر
\\
گفتا که خطاب تو هم باقی این برفست
&&
تا برف بود باقی غیبست گل احمر
\\
گفتم که الا ای مه از تابش روی تو
&&
خورشید کند سجده چون بنده گک کمتر
\\
آخر بنگر در من گفتا که نمی‌ترسی
&&
از آتش رخسارم وانگه تو نه سامندر
\\
گفتم بتکی باشم دو چشم بپوشیده
&&
اندر حجب غیرت پوشیده من این مغفر
\\
گفتا که تو را این عشق در صبر دهد رنگی
&&
شایسته آن گردی هم ناظر و هم منظر
\\
گفتم چه نشان باشد در بنده از این وعده
&&
گفتا که درخش جان در آتش دل چون زر
\\
وان گاه نکو بنگر در صحن عیار جان
&&
در حال درخشانی وز تابش او برخور
\\
گفتم که همی‌ترسم وز ترس همی‌میرم
&&
کز دیدن جان خود از من رود آن جوهر
\\
آن جوهر بی‌چونی کز حسن خیال تو
&&
در چشم نشستستم ای طرفه سیمین بر
\\
گفتا که مترس آخر نی منت همی‌گویم
&&
کز باغ جمال ما هم بر بخوری هم بر
\\
آن نقش خداوندی شمس الحق تبریزی
&&
پرنور از او عالم تبریز از او انور
\\
او بود خلاصه کن او را تو سجودی کن
&&
تا تو شنوی از خود کالله هو الاکبر
\\
\end{longtable}
\end{center}
