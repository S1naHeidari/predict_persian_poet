\begin{center}
\section*{غزل شماره ۲۴۷۷: سرکه هفت ساله را از لب او حلاوتی}
\label{sec:2477}
\addcontentsline{toc}{section}{\nameref{sec:2477}}
\begin{longtable}{l p{0.5cm} r}
سرکه هفت ساله را از لب او حلاوتی
&&
خاربنان خشک را از گل او طراوتی
\\
جان و دل فسرده را از نظرش گشایشی
&&
سنگ سیاه مرده را از گذرش سعادتی
\\
از گذری که او کند گردد سرد دوزخی
&&
وز نظری که افکند زنده شود ولایتی
\\
مرده ز گور برجهد آید و مستمع شود
&&
گر بت من ز مرده‌ای یاد کند حکایتی
\\
آنک ز چشم شوخ او هر نفسی است فتنه‌ای
&&
آنک ز لطف قامتش هر طرفی قیامتی
\\
آه که در فراق او هر قدمی است آتشی
&&
آه که از هوای او می‌رسدم ملامتی
\\
\end{longtable}
\end{center}
