\begin{center}
\section*{غزل شماره ۲۵۹۵: امشب پریان را من تا روز به دلداری}
\label{sec:2595}
\addcontentsline{toc}{section}{\nameref{sec:2595}}
\begin{longtable}{l p{0.5cm} r}
امشب پریان را من تا روز به دلداری
&&
در خوردن و شب گردی خواهم که کنم یاری
\\
من شیوه پریان را آموخته‌ام شب‌ها
&&
وقت حشرانگیزی در چالش و میخواری
\\
جنی پنهان باشد در ستر و امان باشد
&&
پوشیده‌تر از پریان ماییم به ستاری
\\
بر صورت ما واقف پریان و ز جان غافل
&&
در مکر خدا مانده آن قوم ز اغیاری
\\
خود را تو نمی‌دانی جویای پری ز آنی
&&
مفروش چنین ارزان خود را به سبکباری
\\
و آن جنی ما بهتر زیبارخ و خوش گوهر
&&
از دیو و پری برده صد گوی به عیاری
\\
شب از مه او حیران مه عاشق آن سیران
&&
نی بی‌مزه و رنگین پالوده بازاری
\\
از سیخ کباب او وز جام شراب او
&&
وز چنگ و رباب او وز شیوه خماری
\\
دیوانه شده شب‌ها آلوده شده لب‌ها
&&
در جمله مذهب‌ها او راست سزاواری
\\
خواب از شب او مرده شلوار گرو کرده
&&
کس نیست در این پرده تو پشت کی می‌خاری
\\
بردی ز حد ای مکثر بربند دهان آخر
&&
نی عاشق عشقی تو تو عاشق گفتاری
\\
\end{longtable}
\end{center}
