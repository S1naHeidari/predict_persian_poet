\begin{center}
\section*{غزل شماره ۱۲۰۲: به آفتاب شهم گفت هین مکن این ناز}
\label{sec:1202}
\addcontentsline{toc}{section}{\nameref{sec:1202}}
\begin{longtable}{l p{0.5cm} r}
به آفتاب شهم گفت هین مکن این ناز
&&
که گر تو روی بپوشی کنیم ما رو باز
\\
دمی که شعشعه این جمال درتابد
&&
صد آفتاب شود آن زمان سیاه و مجاز
\\
کسی شود به تو غره که روی دوست ندید
&&
کسی که دید مرا کی کند تو را اعزاز
\\
ز گازران مگریز و به زیر ابر مرو
&&
که ابر را و تو را من درآورم به نیاز
\\
اگر چه جان و جهانی خوش به توست جهان
&&
نگون شوی چو رخم دلبری کند آغاز
\\
مرا هزار جهانست پر ز نور و نعیم
&&
چه ناز می‌رسدت با من ای کمین خباز
\\
عباد را برهانم ز نان و از نانبا
&&
حیات من بدهدشان حیات و عمر دراز
\\
ز آفتاب گذشتیم خیز ای ناهید
&&
بیار باده و نقل و نبات و نی بنواز
\\
زمانه با تو نسازد تو سازوارش کن
&&
به چنگ ما ده سغراق و چنگ را ده ساز
\\
نبات و جامد و حیوان همه ز تو مستند
&&
دمی بدین دو سه مخمور بی‌نوا پرداز
\\
حیات با تو خوشست و ممات با تو خوشست
&&
گهیم همچو شکر بفسران گهی بگداز
\\
چو ماه همره من شد سفر مرا حضرست
&&
به زیر سایه او می‌روم نشیب و فراز
\\
ز آسمان شنوم من که عاقبت محمود
&&
خموش باش که محمود گشت کار ایاز
\\
\end{longtable}
\end{center}
