\begin{center}
\section*{غزل شماره ۵۴۵: بی تو به سر می نشود با دگری می‌نشود}
\label{sec:0545}
\addcontentsline{toc}{section}{\nameref{sec:0545}}
\begin{longtable}{l p{0.5cm} r}
بی تو به سر می نشود با دگری می‌نشود
&&
هر چه کنم عشق بیان بی‌جگری می‌نشود
\\
اشک دوان هر سحری از دلم آرد خبری
&&
هیچ کسی را ز دلم خود خبری می‌نشود
\\
یک سر مو از غم تو نیست که اندر تن من
&&
آب حیاتی ندهد یا گهری می‌نشود
\\
ای غم تو راحت جان چیستت این جمله فغان
&&
تا بزنم بانگ و فغان خود حشری می‌نشود
\\
میل تو سوی حشرست پیشه تو شور و شرست
&&
بی ره و رای تو شها رهگذری می‌نشود
\\
چیست حشر از خود خود رفتن جان‌ها به سفر
&&
مرغ چو در بیضه خود بال و پری می‌نشود
\\
بیست چو خورشید اگر تابد اندر شب من
&&
تا تو قدم درننهی خود سحری می‌نشود
\\
دانه دل کاشته‌ای زیر چنین آب و گلی
&&
تا به بهارت نرسد او شجری می‌نشود
\\
در غزلم جبر و قدر هست از این دو بگذر
&&
زانک از این بحث به جز شور و شری می‌نشود
\\
\end{longtable}
\end{center}
