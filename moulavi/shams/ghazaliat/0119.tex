\begin{center}
\section*{غزل شماره ۱۱۹: برخیز و صبوح را بیارا}
\label{sec:0119}
\addcontentsline{toc}{section}{\nameref{sec:0119}}
\begin{longtable}{l p{0.5cm} r}
برخیز و صبوح را بیارا
&&
پرلخلخه کن کنار ما را
\\
پیش آر شراب رنگ آمیز
&&
ای ساقی خوب خوب سیما
\\
از من پرسید کو چه ساقیست
&&
قندست و هزار رطل حلوا
\\
آن ساغر پرعقار برریز
&&
بر وسوسه محال پیما
\\
آن می که چو صعوه زو بنوشد
&&
آهنگ کند به صید عنقا
\\
زان پیش که دررسد گرانی
&&
برجه سبک و میان ما آ
\\
می‌گرد و چو ماه نور می‌ده
&&
حمرا می ده بدان حمیرا
\\
ما را همه مست و کف زنان کن
&&
وان گاه نظاره کن تماشا
\\
در گردش و شیوه‌های مستان
&&
در عربده‌های در علالا
\\
در گردن این فکنده آن دست
&&
کان شاه من و حبیب و مولا
\\
او نیز ببرده روی چون گل
&&
می‌بوسد یار را کف پا
\\
این کیسه گشاده از سخاوت
&&
که خرج کنید بی‌محابا
\\
دستار و قبا فکنده آن نیز
&&
کاین را به گرو نهید فردا
\\
صد مادر و صد پدر ندارد
&&
آن مهر که می‌بجوشد آن جا
\\
این می آمد اصول خویشی
&&
کز سکر چنین شدند اعدا
\\
آن عربده در شراب دنیاست
&&
در بزم خدا نباشد آن‌ها
\\
نی شورش و نی قیست و نی جنگ
&&
ساقیست و شراب مجلس آرا
\\
خاموش که ز سکر نفس کافر
&&
می‌گوید لا اله الا
\\
\end{longtable}
\end{center}
