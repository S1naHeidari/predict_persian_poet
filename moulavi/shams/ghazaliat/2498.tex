\begin{center}
\section*{غزل شماره ۲۴۹۸: مرا سودای آن دلبر ز دانایی و قرایی}
\label{sec:2498}
\addcontentsline{toc}{section}{\nameref{sec:2498}}
\begin{longtable}{l p{0.5cm} r}
مرا سودای آن دلبر ز دانایی و قرایی
&&
برون آورد تا گشتم چنین شیدا و سودایی
\\
سر سجاده و مسند گرفتم من به جهد و جد
&&
شعار زهد پوشیدم پی خیرات افزایی
\\
درآمد عشق در مسجد بگفت ای خواجه مرشد
&&
بدران بند هستی را چه دربند مصلایی
\\
به پیش زخم تیغ من ملرزان دل بنه گردن
&&
اگر خواهی سفر کردن ز دانایی به بینایی
\\
بده تو داد اوباشی اگر رندی و قلاشی
&&
پس پرده چه می‌باشی اگر خوبی و زیبایی
\\
فراری نیست خوبان را ز عرضه کردن سیما
&&
بتان را صبر کی باشد ز غنج و چهره آرایی
\\
گهی از روی خود داده خرد را عشق و بی‌صبری
&&
گهی از چشم خود کرده سقیمان را مسیحایی
\\
گهی از زلف خود داده به مؤمن نقش حبل الله
&&
ز پیچ جعد خود داده به ترسایان چلیپایی
\\
تو حسن خود اگر دیدی که افزونتر ز خورشیدی
&&
چه پژمردی چه پوسیدی در این زندان غبرایی
\\
چرا تازه نمی‌باشی ز الطاف ربیع دل
&&
چرا چون گل نمی‌خندی چرا عنبر نمی‌سایی
\\
چرا در خم این دنیا چو باده بر نمی‌جوشی
&&
که تا جوشت برون آرد از این سرپوش مینایی
\\
ز برق چهره خوبت چه محروم است یعقوبت
&&
الا ای یوسف خوبان به قعر چه چه می‌پایی
\\
ببین حسن خود ای نادان ز تاب جان او تا دان
&&
که مؤمن آینه مؤمن بود در وقت تنهایی
\\
ببیند خاک سر خود درون چهره بستان
&&
که من در دل چه‌ها دارم ز زیبایی و رعنایی
\\
ببیند سنگ سر خود درون لعل و پیروزه
&&
که گنجی دارم اندر دل کند آهنگ بالایی
\\
ببیند آهن تیره دل خود را در آیینه
&&
که من هم قابل نورم کنم آخر مصفایی
\\
عدم‌ها مر عدم‌ها را چو می‌بیند به دل گشته
&&
به هستی پیش می‌آید که تا دزدد پذیرایی
\\
به هر سرگین کجا گشتی مگس را گر خبر بودی
&&
که آید از سرشت او به سعی و فضل عنقایی
\\
چو ابن الوقت شد صوفی نگردد کاهل فردا
&&
سبک کاهل شود آن کس که باشد گول و فردایی
\\
میان دلبران بنشین اگر نه غری و عنین
&&
میان عاشقان خو کن مباش ای دوست هرجایی
\\
ایا ماهی یقین گشتت ز دریای پس پشتت
&&
بگردان روی و واپس رو چو تو از اهل دریایی
\\
ندای ارجعی بشنو به آب زندگی بگرو
&&
درآ در آب و خوش می‌رو به آب و گل چه می‌پایی
\\
به جان و دل شدی جایی که نی جان ماند و نی دل
&&
به پای خود شدی جایی که آن جا دست می‌خایی
\\
ز خورشید ازل زر شو به زر غیر کمتر رو
&&
که عشق زر کند زردت اگر چه سیم سیمایی
\\
تو را دنیا همی‌گوید چرا لالای من گشتی
&&
تو سلطان زاده‌ای آخر منم لایق به لالایی
\\
تو را دریا همی‌گوید منت مرکب شوم خوشتر
&&
که تو مرکب شوی ما را به حمالی و سقایی
\\
خمش کن من چو تو بودم خمش کردم بیاسودم
&&
اگر تو بشنوی از من خمش باشی بیاسایی
\\
\end{longtable}
\end{center}
