\begin{center}
\section*{غزل شماره ۱۶۱۵: من اگر دست زنانم نه من از دست زنانم}
\label{sec:1615}
\addcontentsline{toc}{section}{\nameref{sec:1615}}
\begin{longtable}{l p{0.5cm} r}
من اگر دست زنانم نه من از دست زنانم
&&
نه از اینم نه از آنم من از آن شهر کلانم
\\
نه پی زمر و قمارم نه پی خمر و عقارم
&&
نه خمیرم نه خمارم نه چنینم نه چنانم
\\
من اگر مست و خرابم نه چو تو مست شرابم
&&
نه ز خاکم نه ز آبم نه از این اهل زمانم
\\
خرد پوره آدم چه خبر دارد از این دم
&&
که من از جمله عالم به دو صد پرده نهانم
\\
مشنو این سخن از من و نه زین خاطر روشن
&&
که از این ظاهر و باطن نه پذیرم نه ستانم
\\
رخ تو گر چه که خوب است قفس جان تو چوب است
&&
برم از من که بسوزی که زبانه‌ست زبانم
\\
نه ز بویم نه ز رنگم نه ز نامم نه ز ننگم
&&
حذر از تیر خدنگم که خدایی است کمانم
\\
نه می خام ستانم نه ز کس وام ستانم
&&
نه دم و دام ستانم هله ای بخت جوانم
\\
چو گلستان جنانم طربستان جهانم
&&
به روان همه مردان که روان است روانم
\\
شکرستان خیالت بر من گلشکر آرد
&&
به گلستان حقایق گل صدبرگ فشانم
\\
چو درآیم به گلستان گل افشان وصالت
&&
ز سر پا بنشانم که ز داغت به نشانم
\\
عجب ای عشق چه جفتی چه غریبی چه شگفتی
&&
چو دهانم بگرفتی به درون رفت بیانم
\\
چو به تبریز رسد جان سوی شمس الحق و دینم
&&
همه اسرار سخن را به نهایت برسانم
\\
\end{longtable}
\end{center}
