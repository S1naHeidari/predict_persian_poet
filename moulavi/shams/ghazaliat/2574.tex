\begin{center}
\section*{غزل شماره ۲۵۷۴: از آتش ناپیدا دارم دل بریانی}
\label{sec:2574}
\addcontentsline{toc}{section}{\nameref{sec:2574}}
\begin{longtable}{l p{0.5cm} r}
از آتش ناپیدا دارم دل بریانی
&&
فریاد مسلمانان از دست مسلمانی
\\
شهد و شکرش گویم کان گهرش گویم
&&
شمع و سحرش خوانم یا نادره سلطانی
\\
زین فتنه و غوغایی آتش زده هر جایی
&&
وز آتش و دود ما برخاسته ایوانی
\\
با این همه سلطانی آن خصم مسلمانی
&&
بربود به قهر از من در راه حرمدانی
\\
بگشاد حرمدانم بربود دل و جانم
&&
آن کس که به پیش او جانی به یکی نانی
\\
من دوش ز بوی او رفتم سر کوی او
&&
ناگاه پدید آمد باغی و گلستانی
\\
آن جا دل و دلداری هم عالم اسراری
&&
هم واقف و بیداری هم شهره و پنهانی
\\
در خدمت خاک او عیشی و تماشایی
&&
در آتش عشق او هر چشمه حیوانی
\\
\end{longtable}
\end{center}
