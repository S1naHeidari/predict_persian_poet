\begin{center}
\section*{غزل شماره ۱۴۸۴: بشکن قدح باده که امروز چنانیم}
\label{sec:1484}
\addcontentsline{toc}{section}{\nameref{sec:1484}}
\begin{longtable}{l p{0.5cm} r}
بشکن قدح باده که امروز چنانیم
&&
کز توبه شکستن سر توبه شکنانیم
\\
گر باده فنا گشت فنا باده ما بس
&&
ما نیک بدانیم گر این رنگ ندانیم
\\
باده ز فنا دارد آن چیز که دارد
&&
گر باده بمانیم از آن چیز نمانیم
\\
از چیزی خود بگذر ای چیز به ناچیز
&&
کاین چیز نه پرده‌ست نه ما پرده درانیم
\\
با غمزه سرمست تو میریم و اسیریم
&&
با عشق جوان بخت تو پیریم و جوانیم
\\
گفتی چه دهی پند و زین پند چه سود است
&&
کان نقش که نقاش ازل کرد همانیم
\\
این پند من از نقش ازل هیچ جدا نیست
&&
زین نقش بدان نقش ازل فرق ندانیم
\\
گفتی که جدا مانده‌ای از بر معشوق
&&
ما در بر معشوق ز انده در امانیم
\\
معشوق درختی است که ما از بر اوییم
&&
از ما بر او دور شود هیچ نمانیم
\\
چون هیچ نمانیم ز غم هیچ نپیچیم
&&
چون هیچ نمانیم هم اینیم و هم آنیم
\\
شادی شود آن غم که خوریمش چو شکر خوش
&&
ای غم بر ما آی که اکسیر غمانیم
\\
چون برگ خورد پیله شود برگ بریشم
&&
ما پیله عشقیم که بی‌برگ جهانیم
\\
ماییم در آن وقت که ما هیچ نمانیم
&&
آن وقت که پا نیست شود پای دوانیم
\\
بستیم دهان خود و باقی غزل را
&&
آن وقت بگوییم که ما بسته دهانیم
\\
\end{longtable}
\end{center}
