\begin{center}
\section*{غزل شماره ۹۰۵: شدم ز عشق به جایی که عشق نیز نداند}
\label{sec:0905}
\addcontentsline{toc}{section}{\nameref{sec:0905}}
\begin{longtable}{l p{0.5cm} r}
شدم ز عشق به جایی که عشق نیز نداند
&&
رسید کار به جایی که عقل خیره بماند
\\
هزار ظلم رسیده ز عقل گشت رهیده
&&
چو عقل بسته شد این جا بگو کیش برهاند
\\
دلا مگر که تو مستی که دل به عقل ببستی
&&
که او نشست نیابد تو را کجا بنشاند
\\
متاع عقل نشانست و عشق روح فشانست
&&
که عشق وقت نظاره نثار جان بفشاند
\\
هزار جان و دل و عقل گر به هم تو ببندی
&&
چو عشق با تو نباشد به روزنش نرساند
\\
به روی بت نرسی تو مگر به دام دو زلفش
&&
ولیک کوشش می‌کن که کوششت بپزاند
\\
چو باز چشم تو را بست دست اوست گشایش
&&
ولی به هر سر کویی تو را چو کبک دواند
\\
هر آنک بالش دارد ز آستان عنایت
&&
غلام خفتن اویم که هیچ خفته نماند
\\
میانه گیرد آهو میانه دل شیری
&&
هزار آهوی دیگر ز شیر او برهاند
\\
چو در درونه صیاد مرغ یافت قبولی
&&
هزار مرغ گرفته ز دام او بپراند
\\
هر آن دلی که به تبریز و شمس دین شده باشد
&&
چو شاه ماه به میدان چرخ اسب دواند
\\
\end{longtable}
\end{center}
