\begin{center}
\section*{غزل شماره ۲۷۲۳: دیدی که چه کرد یار ما دیدی}
\label{sec:2723}
\addcontentsline{toc}{section}{\nameref{sec:2723}}
\begin{longtable}{l p{0.5cm} r}
دیدی که چه کرد یار ما دیدی
&&
منصوبه یار باوفا دیدی
\\
زین نوع که مات کرد دل‌ها را
&&
آن چشمه زندگی کجا دیدی
\\
در صورت مات برد می‌بخشد
&&
مقلوب گری چو او که را دیدی
\\
ای بسته بند عشق حقستت
&&
کز عشق هزار دلگشا دیدی
\\
بستان باغی اگر گلی دادی
&&
برخور ز وفا اگر جفا دیدی
\\
از بستانش سر خر است این تن
&&
زان بحر گهر تو کهربا دیدی
\\
از فرعونی چو احولی دادت
&&
آن بود عصا و اژدها دیدی
\\
امروز چو موسیت مداوا کرد
&&
صد برگ فشان از آن عصا دیدی
\\
صیاد جهان فشاند شه دانه
&&
آن را تو ز سادگی عطا دیدی
\\
چون مرغ سلیم سوی او رفتی
&&
دام و دغل و فن و دغا دیدی
\\
بازت بخرید لطف نجینا
&&
تا لطف و عنایت خدا دیدی
\\
در طالع مه چو مشتری گشتی
&&
ز الله عطای اشتری دیدی
\\
چندان کرث که در عدد ناید
&&
این بستگی و گشاد را دیدی
\\
تا آخر کار آن ولی نعمت
&&
چشمت بگشاد توتیا دیدی
\\
از چشمه سلسبیل می خوردی
&&
عشرت گه خاص اولیا دیدی
\\
چون دعوت اشربوا پری دادت
&&
جولانگه عرصه هوا دیدی
\\
وآنگه ز هوا به سوی هو رفتی
&&
بر قاف پریدن هما دیدی
\\
پرواز همای کبریایی را
&&
از کیف و چگونگی جدا دیدی
\\
باقیش مجیب هر دعا گوید
&&
کز وی تو اجابت دعا دیدی
\\
\end{longtable}
\end{center}
