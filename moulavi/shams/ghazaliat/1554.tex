\begin{center}
\section*{غزل شماره ۱۵۵۴: چون ذره به رقص اندرآییم}
\label{sec:1554}
\addcontentsline{toc}{section}{\nameref{sec:1554}}
\begin{longtable}{l p{0.5cm} r}
چون ذره به رقص اندرآییم
&&
خورشید تو را مسخر آییم
\\
در هر سحری ز مشرق عشق
&&
همچون خورشید ما برآییم
\\
در خشک و تر جهان بتابیم
&&
نی خشک شویم و نی تر آییم
\\
بس ناله مس‌ها شنیدیم
&&
کای نور بتاب تا زر آییم
\\
از بهر نیاز و درد ایشان
&&
ما بر سر چرخ و اختر آییم
\\
از سیمبری که هست دلبر
&&
از بهر قلاده عنبر آییم
\\
زان خرقه خویش ضرب کردیم
&&
تا زین به قبای ششتر آییم
\\
ما صرف کشان راه فقریم
&&
سرمست نبیذ احمر آییم
\\
گر زهر جهان نهند بر ما
&&
از باطن خویش شکر آییم
\\
آن روز که پردلان گریزند
&&
در عین وغا چو سنجر آییم
\\
از خون عدو نبیذ سازیم
&&
وانگه بکشیم و خنجر آییم
\\
ما حلقه عاشقان مستیم
&&
هر روز چو حلقه بر در آییم
\\
طغرای امان ما نوشت او
&&
کی از اجلی به غرغر آییم
\\
اندر ملکوت و لامکان ما
&&
بر کره چرخ اخضر آییم
\\
از عالم جسم خفیه گردیم
&&
در عالم عشق اظهر آییم
\\
در جسم شده‌ست روح طاهر
&&
بی‌جسم شویم و اطهر آییم
\\
شمس تبریز جان جان است
&&
در برج ابد برابر آییم
\\
\end{longtable}
\end{center}
