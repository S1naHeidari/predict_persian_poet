\begin{center}
\section*{غزل شماره ۲۰۷۴: مکن مکن که روا نیست بی‌گنه کشتن}
\label{sec:2074}
\addcontentsline{toc}{section}{\nameref{sec:2074}}
\begin{longtable}{l p{0.5cm} r}
مکن مکن که روا نیست بی‌گنه کشتن
&&
مرو مرو که چراغی و دیده روشن
\\
چو برگشادی از لطف خویشتن سر خم
&&
دماغ ما ز خمار تو است آبستن
\\
مبند آن سر خم را چو کیسه مدخل
&&
که خانه گردد تاری به بستن روزن
\\
چو آدمی به غم آماج تیر را ماند
&&
ندارد او جز مستی و بیخودی جوشن
\\
دو دست عشق مثال دو دست داوود است
&&
که همچو موم همی‌گردد از کفش آهن
\\
حدیث عشق هم از عشقباز باید جست
&&
که او چو آینه هم ناطق است و هم الکن
\\
دلا دو دست برآور سبک به گردن عشق
&&
اگر چه دارد او خون خلق در گردن
\\
ز خونبها بنترسد که گنج‌ها دارد
&&
که مرده زنده شود زان و وارهد ز کفن
\\
گرفت خواب گریبان تو بپر سوی غیب
&&
بگه ز غیب بیایی کشان کشان دامن
\\
که تا تمام غزل را بگویمت فردا
&&
که گل پگاه بچینند مردم از گلشن
\\
\end{longtable}
\end{center}
