\begin{center}
\section*{غزل شماره ۶۵۹: دلم امروز خوی یار دارد}
\label{sec:0659}
\addcontentsline{toc}{section}{\nameref{sec:0659}}
\begin{longtable}{l p{0.5cm} r}
دلم امروز خوی یار دارد
&&
هوای روی چون گلنار دارد
\\
که طاووس آن طرف پر می‌فشاند
&&
که بلبل آن طرف تکرار دارد
\\
صدای نای آن جا نکته گوید
&&
نوای چنگ بس اسرار دارد
\\
بگه برخیز فردا سوی او رو
&&
که او عاشق چو من بسیار دارد
\\
چو بگشاید رخان تو دل نگهدار
&&
که بس آتش در آن رخسار دارد
\\
ولیکن عقل کو آن لحظه دل را
&&
که دل‌ها را لبش خمار دارد
\\
ز ما کاری مجو چون داده‌ای می
&&
که می مر مرد را بی‌کار دارد
\\
دلم افتان و خیزان دوش آمد
&&
که می مستی او اظهار دارد
\\
دویدم پیش و گفتم باده خوردی
&&
نمی‌ترسی که عقل انکار دارد
\\
چو بو کردم دهانش را بدیدم
&&
که بوی آن پری دیدار دارد
\\
خداوندی شمس الدین تبریز
&&
که بوی خالق جبار دارد
\\
ز بو تا بوی فرقی بس عظیمست
&&
و او بی‌حد و بی‌مقدار دارد
\\
\end{longtable}
\end{center}
