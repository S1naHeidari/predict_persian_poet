\begin{center}
\section*{غزل شماره ۲۷۴۸: می‌آید سنجق بهاری}
\label{sec:2748}
\addcontentsline{toc}{section}{\nameref{sec:2748}}
\begin{longtable}{l p{0.5cm} r}
می‌آید سنجق بهاری
&&
لشکرکش شور و بی‌قراری
\\
گلزار نقاب می‌گشاید
&&
بلبل بگرفت باز زاری
\\
بر کف بنهاده لاله جامی
&&
کای نرگس مست بر چه کاری
\\
امروز بنفشه در رکوع است
&&
می‌جوید از خدای یاری
\\
سرها ز مغاره کرده بیرون
&&
آن لاله رخان کوهساری
\\
یا رب که که را همی‌فریبند
&&
خوش می‌نگرند در شکاری
\\
منگر به سمن به چشم خردی
&&
منگر به چمن به چشم خواری
\\
زیرا به مسافران عزت
&&
گر خوار نظر کنی نیاری
\\
بشنو ز زبان سبز هر برگ
&&
کز عیب بروید آنچ کاری
\\
گشته‌ست زبان گاو ناطق
&&
در حمد و ثنا و شکر آری
\\
عذرت نبود ز یأس از آن کو
&&
بخشد به کلوخ خوش عذاری
\\
بابرگ شد آن کلوخ جان یافت
&&
در شکر نمود جان سپاری
\\
صد میوه چو شیشه‌های شربت
&&
هر یک مزه‌ای به خوشگواری
\\
بعضی چو شکر اگر شکوری
&&
بعضی ترشند اگر خماری
\\
خاموش نشین و مستمع باش
&&
نی واعظ خلق شو نه قاری
\\
\end{longtable}
\end{center}
