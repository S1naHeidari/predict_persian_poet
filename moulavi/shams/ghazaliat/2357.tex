\begin{center}
\section*{غزل شماره ۲۳۵۷: ای گشته دلت چو سنگ خاره}
\label{sec:2357}
\addcontentsline{toc}{section}{\nameref{sec:2357}}
\begin{longtable}{l p{0.5cm} r}
ای گشته دلت چو سنگ خاره
&&
با خاره و سنگ چیست چاره
\\
با خاره چه چاره شیشه‌ها را
&&
جز آنک شوند پاره پاره
\\
زان می‌خندی چو صبح صادق
&&
تا پیش تو جان دهد ستاره
\\
تا عشق کنار خویش بگشاد
&&
اندیشه گریخت بر کناره
\\
چون صبر بدید آن هزیمت
&&
او نیز بجست یک سواره
\\
شد صبر و خرد بماند سودا
&&
می‌گرید و می‌کند حراره
\\
خلقی ز جدایی عصیرت
&&
بر راه فتاده چون عصاره
\\
هر چند شده‌ست خون جگرشان
&&
چستند در این ره و چه کاره
\\
بیگانه شدیم بهر این کار
&&
با عقل و دل هزارکاره
\\
العشق حقیقه الاماره
&&
و الشعر طباله الاماره
\\
احذر فامیرنا مغیر
&&
کل سحر لدیه غاره
\\
اترک هذا وصف فراقا
&&
تنشق لهوله العباره
\\
بگریخت امام ای مؤذن
&&
خاموش فرورو از مناره
\\
\end{longtable}
\end{center}
