\begin{center}
\section*{غزل شماره ۱۲۴۷: عارفان را شمع و شاهد نیست از بیرون خویش}
\label{sec:1247}
\addcontentsline{toc}{section}{\nameref{sec:1247}}
\begin{longtable}{l p{0.5cm} r}
عارفان را شمع و شاهد نیست از بیرون خویش
&&
خون انگوری نخورده باده شان هم خون خویش
\\
هر کسی اندر جهان مجنون لیلی شدند
&&
عارفان لیلی خویش و دم به دم مجنون خویش
\\
ساعتی میزان آنی ساعتی موزون این
&&
بعد از این میزان خود شو تا شوی موزون خویش
\\
گر تو فرعون منی از مصر تن بیرون کنی
&&
در درون حالی ببینی موسی و هارون خویش
\\
لنگری از گنج مادون بسته‌ای بر پای جان
&&
تا فروتر می‌روی هر روز با قارون خویش
\\
یونسی دیدم نشسته بر لب دریای عشق
&&
گفتمش چونی جوابم داد بر قانون خویش
\\
گفت بودم اندر این دریا غذای ماهیی
&&
پس چو حرف نون خمیدم تا شدم ذاالنون خویش
\\
زین سپس ما را مگو چونی و از چون درگذر
&&
چون ز چونی دم زند آن کس که شد بی‌چون خویش
\\
باده غمگینان خورند و ما ز می خوش دلتریم
&&
رو به محبوسان غم ده ساقیا افیون خویش
\\
خون ما بر غم حرام و خون غم بر ما حلال
&&
هر غمی کو گرد ما گردید شد در خون خویش
\\
باده گلگونه‌ست بر رخسار بیماران غم
&&
ما خوش از رنگ خودیم و چهره گلگون خویش
\\
من نیم موقوف نفخ صور همچون مردگان
&&
هر زمانم عشق جانی می‌دهد ز افسون خویش
\\
در بهشت استبرق سبزست و خلخال و حریر
&&
عشق نقدم می‌دهد از اطلس و اکسون خویش
\\
دی منجم گفت دیدم طالعی داری تو سعد
&&
گفتمش آری ولیک از ماه روزافزون خویش
\\
مه کی باشد با مه ما کز جمال و طالعش
&&
نحس اکبر سعد اکبر گشت بر گردون خویش
\\
\end{longtable}
\end{center}
