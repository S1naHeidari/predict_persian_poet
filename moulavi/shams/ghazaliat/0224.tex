\begin{center}
\section*{غزل شماره ۲۲۴: چه خیره می‌نگری در رخ من ای برنا}
\label{sec:0224}
\addcontentsline{toc}{section}{\nameref{sec:0224}}
\begin{longtable}{l p{0.5cm} r}
چه خیره می‌نگری در رخ من ای برنا
&&
مگر که در رخمست آیتی از آن سودا
\\
مگر که بر رخ من داغ عشق می‌بینی
&&
میان داغ نبشته که نحن نزلنا
\\
هزار مشک همی‌خواهم و هزار شکم
&&
که آب خضر لذیذست و من در استسقا
\\
وفا چه می‌طلبی از کسی که بی‌دل شد
&&
چو دل برفت برفت از پیش وفا و جفا
\\
به حق این دل ویران و حسن معمورت
&&
خوش است گنج خیالت در این خرابه ما
\\
غریو و ناله جان‌ها ز سوی بی‌سویی
&&
مرا ز خواب جهانید دوش وقت دعا
\\
ز ناله گویم یا از جمال ناله کنان
&&
ز ناله گوش پرست از جمالش آن عینا
\\
قرار نیست زمانی تو را برادر من
&&
ببین که می‌کشدت هر طرف تقاضاها
\\
مثال گویی اندر میان صد چوگان
&&
دوانه تا سر میدان و گه ز سر تا پا
\\
کجاست نیت شاه و کجاست نیت گوی
&&
کجاست قامت یار و کجاست بانگ صلا
\\
ز جوش شوق تو من همچو بحر غریدم
&&
بگو تو ای شه دانا و گوهر دریا گویا
\\
\end{longtable}
\end{center}
