\begin{center}
\section*{غزل شماره ۲۰۱۹: ای ببرده دل تو قصد جان مکن}
\label{sec:2019}
\addcontentsline{toc}{section}{\nameref{sec:2019}}
\begin{longtable}{l p{0.5cm} r}
ای ببرده دل تو قصد جان مکن
&&
و آنچ من کردم تو جانا آن مکن
\\
بنگر اندر درد من گر صاف نیست
&&
درد خود مفرستم و درمان مکن
\\
داد ایمان داد زلف کافرت
&&
یک سر مویی ز کفر ایمان مکن
\\
عادت خوبان جفا باشد جفا
&&
هم بر آن عادت بر او احسان مکن
\\
گر چه دل بر مرگ خود بنهاده‌ایم
&&
در جفا آهسته‌تر چندان مکن
\\
عیش ما را مرگ باشد پرده دار
&&
پرده پوش و مرگ را خندان مکن
\\
ای زلیخا فتنه عشق از تو است
&&
یوسفی را هرزه در زندان مکن
\\
چون سر رندان نداری وقت عیش
&&
وعده‌ها اندر سر رندان مکن
\\
نور چشم عاشقان آخر تویی
&&
عیش‌ها بر کوری ایشان مکن
\\
نقدکی را از یکی مفلس مبر
&&
از حریصی نقد او در کان مکن
\\
شب روان را همچو استاره مسوز
&&
راه خود را پر ز رهبانان مکن
\\
شمس تبریزی یکی رویی نمای
&&
تا ابد تو روی با جانان مکن
\\
\end{longtable}
\end{center}
