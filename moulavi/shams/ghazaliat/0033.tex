\begin{center}
\section*{غزل شماره ۳۳: می ده گزافه ساقیا تا کم شود خوف و رجا}
\label{sec:0033}
\addcontentsline{toc}{section}{\nameref{sec:0033}}
\begin{longtable}{l p{0.5cm} r}
می ده گزافه ساقیا تا کم شود خوف و رجا
&&
گردن بزن اندیشه را ما از کجا او از کجا
\\
پیش آر نوشانوش را از بیخ برکن هوش را
&&
آن عیش بی‌روپوش را از بند هستی برگشا
\\
در مجلس ما سرخوش آ برقع ز چهره برگشا
&&
زان سان که اول آمدی ای یفعل الله ما یشا
\\
دیوانگان جسته بین از بند هستی رسته بین
&&
در بی‌دلی دل بسته بین کاین دل بود دام بلا
\\
زوتر بیا هین دیر شد دل زین ولایت سیر شد
&&
مستش کن و بازش رهان زین گفتن زوتر بیا
\\
بگشا ز دستم این رسن بربند پای بوالحسن
&&
پر ده قدح را تا که من سر را بنشناسم ز پا
\\
بی ذوق آن جانی که او در ماجرا و گفت و گو
&&
هر لحظه گرمی می‌کند با بوالعلی و بوالعلا
\\
نانم مده آبم مده آسایش و خوابم مده
&&
ای تشنگی عشق تو صد همچو ما را خونبها
\\
امروز مهمان توام مست و پریشان توام
&&
پر شد همه شهر این خبر کامروز عیش است الصلا
\\
هر کو به جز حق مشتری جوید نباشد جز خری
&&
در سبزه این گولخن همچون خران جوید چرا
\\
می‌دان که سبزه گولخن گنده کند ریش و دهن
&&
زیرا ز خضرای دمن فرمود دوری مصطفی
\\
دورم ز خضرای دمن دورم ز حورای چمن
&&
دورم ز کبر و ما و من مست شراب کبریا
\\
از دل خیال دلبری برکرد ناگاهان سری
&&
ماننده ماه از افق ماننده گل از گیا
\\
جمله خیالات جهان پیش خیال او دوان
&&
مانند آهن پاره‌ها در جذبه آهن ربا
\\
بد لعل‌ها پیشش حجر شیران به پیشش گورخر
&&
شمشیرها پیشش سپر خورشید پیشش ذره‌ها
\\
عالم چو کوه طور شد هر ذره‌اش پرنور شد
&&
مانند موسی روح هم افتاد بی‌هوش از لقا
\\
هر هستییی در وصل خود در وصل اصل اصل خود
&&
خنبک زنان بر نیستی دستک زنان اندر نما
\\
سرسبز و خوش هر تره‌ای نعره زنان هر ذره‌ای
&&
کالصبر مفتاح الفرج و الشکر مفتاح الرضا
\\
گل کرد بلبل را ندا کای صد چو من پیشت فدا
&&
حارس بدی سلطان شدی تا کی زنی طال بقا
\\
ذرات محتاجان شده اندر دعا نالان شده
&&
برقی بر ایشان برزده مانده ز حیرت از دعا
\\
السلم منهاج الطلب الحلم معراج الطرب
&&
و النار صراف الذهب و النور صراف الولا
\\
العشق مصباح العشا و الهجر طباخ الحشا
&&
و الوصل تریاق الغشا یا من علی قلبی مشا
\\
الشمس من افراسنا و البدر من حراسنا
&&
و العشق من جلاسنا من یدر ما فی راسنا
\\
یا سایلی عن حبه اکرم به انعم به
&&
کل المنی فی جنبه عند التجلی کالهبا
\\
یا سایلی عن قصتی العشق قسمی حصتی
&&
و السکر افنی غصتی یا حبذا لی حبذا
\\
الفتح من تفاحکم و الحشر من اصباحکم
&&
القلب من ارواحکم فی الدور تمثال الرحا
\\
اریاحکم تجلی البصر یعقوبکم یلقی النظر
&&
یا یوسفینا فی البشر جودوا بما الله اشتری
\\
الشمس خرت و القمر نسکا مع الاحدی عشر
&&
قدامکم فی یقظه قدام یوسف فی الکری
\\
اصل العطایا دخلنا ذخر البرایا نخلنا
&&
یا من لحب او نوی یشکوا مخالیب النوی
\\
\end{longtable}
\end{center}
