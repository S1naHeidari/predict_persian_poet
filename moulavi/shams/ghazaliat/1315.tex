\begin{center}
\section*{غزل شماره ۱۳۱۵: روان شد اشک یاقوتی ز راه دیدگان اینک}
\label{sec:1315}
\addcontentsline{toc}{section}{\nameref{sec:1315}}
\begin{longtable}{l p{0.5cm} r}
روان شد اشک یاقوتی ز راه دیدگان اینک
&&
ز عشق بی‌نشان آمد نشان بی‌نشان اینک
\\
ببین در رنگ معشوقان نگر در رنگ مشتاقان
&&
که آمد این دو رنگ خوش از آن بی‌رنگ جان اینک
\\
فلک مر خاک را هر دم هزاران رنگ می‌بخشد
&&
که نی رنگ زمین دارد نه رنگ آسمان اینک
\\
چو اصل رنگ بی‌رنگست و اصل نقش بی‌نقشست
&&
چو اصل حرف بی‌حرفست چو اصل نقد کان اینک
\\
تویی عاشق تویی معشوق تویی جویان این هر دو
&&
ولی تو توی بر تویی ز رشک این و آن اینک
\\
تو مشک آب حیوانی ولی رشکت دهان بندد
&&
دهان خاموش و جان نالان ز عشق بی‌امان اینک
\\
سحرگه ناله مرغان رسولی از خموشانست
&&
جهان خامش نالان نشانش در دهان اینک
\\
ز ذوقش گر ببالیدی چرا از هجر نالیدی
&&
تو منکر می‌شوی لیکن هزاران ترجمان اینک
\\
اگر نه صید یاری تو بگو چون بی‌قراری تو
&&
چو دیدی آسیا گردان بدان آب روان اینک
\\
اشارت می‌کند جانم که خامش که مرنجانم
&&
خموشم بنده فرمانم رها کردم بیان اینک
\\
\end{longtable}
\end{center}
