\begin{center}
\section*{غزل شماره ۲۸۶۱: چند روز است که شطرنج عجب می‌بازی}
\label{sec:2861}
\addcontentsline{toc}{section}{\nameref{sec:2861}}
\begin{longtable}{l p{0.5cm} r}
چند روز است که شطرنج عجب می‌بازی
&&
دانه بوالعجب و دام عجب می‌سازی
\\
کی برد جان ز تو گر ز آنک تو دل سخت کنی
&&
کی برد سر ز تو گر ز آنک بدین پردازی
\\
صفت حکم تو در خون شهیدان رقصد
&&
مرگ موش است ولیکن بر گربه بازی
\\
بدگمان باشد عاشق تو از این‌ها دوری
&&
همه لطفی و ز سر لطف دگر آغازی
\\
همچو نایم ز لبت می‌چشم و می‌نالم
&&
کم زنم تا نکند کس طمع انبازی
\\
نای اگر ناله کند لیک از او بوی لبت
&&
برسد سوی دماغ و بکند غمازی
\\
تو که می ناله کنی گر نه پی طراری است
&&
از گزافه تو چنین خوش دم و خوش آوازی
\\
نه هر آواز گواه است خبر می‌آرد
&&
این خبر فهم کن ار همنفس آن رازی
\\
ای دل از خویش و از اندیشه تهی شو زیرا
&&
نی تهی گشت از آن یافت ز وی دمسازی
\\
\end{longtable}
\end{center}
