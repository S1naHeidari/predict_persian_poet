\begin{center}
\section*{غزل شماره ۸۰۴: صنما گر ز خط و خال تو فرمان آرند}
\label{sec:0804}
\addcontentsline{toc}{section}{\nameref{sec:0804}}
\begin{longtable}{l p{0.5cm} r}
صنما گر ز خط و خال تو فرمان آرند
&&
این دل خسته مجروح مرا جان آرند
\\
عاشقان نقش خیال تو چو بینند به خواب
&&
ای بسا سیل که از دیده گریان آرند
\\
خنک آن روز خوشا وقت که در مجلس ما
&&
ساقیان دست تو گیرند و به مهمان آرند
\\
صوفیان طاق دو ابروی تو را سجده برند
&&
عارفان آنچ نداری بر تو آن آرند
\\
چشم شوخ تو چو آغاز کند بوالعجبی
&&
آدم کافر و ابلیس مسلمان آرند
\\
بت پرستان رخ خورشید تو را گر بینند
&&
بر قد و قامت زیبای تو ایمان آرند
\\
شمه‌ای گر ز تو در عالم علوی برسد
&&
قدسیان رقص بر این گنبد گردان آرند
\\
گر بدین عاشق دلسوخته مسکینی
&&
شکری زان لب چون لعل بدخشان آرند
\\
جان و دل هر دو فدای شکرستان تو باد
&&
آب حیوان چو از آن چاه زنخدان آرند
\\
شمس تبریز اگر بلبل باغ ارمی
&&
باش تا قوت تو از روضه رضوان آرند
\\
\end{longtable}
\end{center}
