\begin{center}
\section*{غزل شماره ۸۷۲: این عشق جمله عاقل و بیدار می‌کشد}
\label{sec:0872}
\addcontentsline{toc}{section}{\nameref{sec:0872}}
\begin{longtable}{l p{0.5cm} r}
این عشق جمله عاقل و بیدار می‌کشد
&&
بی تیغ می‌برد سر و بی‌دار می‌کشد
\\
مهمان او شدیم که مهمان همی‌خورد
&&
یار کسی شدیم که او یار می‌کشد
\\
چون یوسفی بدید چو گرگان همی‌درد
&&
چونمؤمنی بدید چو کفار می‌کشد
\\
ما دل نهاده‌ایم که دلداریی کند
&&
یا گر کشد به رحم و به هنجار می‌کشد
\\
نی نی که کشته را دم او جان همی‌دهد
&&
گر چه به غمزه عاشق بسیار می‌کشد
\\
هل تا کشد تو را نه که آب حیات اوست
&&
تلخی مکن که دوست عسل وار می‌کشد
\\
همت بلند دار که آن عشق همتی
&&
شاهان برگزیده و احرار می‌کشد
\\
ما چون شبیم ظل زمین و وی آفتاب
&&
شب را به تیغ صبح گهردار می‌کشد
\\
زنگی شب ببرد چو طرار عقل ما
&&
شحنه صبوح آمد و طرار می‌کشد
\\
شب شرق تا به غرب گرفته سپاه زنگ
&&
رومی روزشان به یکی بار می‌کشد
\\
حاصل مرا چو بلبل مستی ز گلشنیست
&&
چون بلبلم جدایی گلزار می‌کشد
\\
\end{longtable}
\end{center}
