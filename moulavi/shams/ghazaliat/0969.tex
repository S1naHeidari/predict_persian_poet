\begin{center}
\section*{غزل شماره ۹۶۹: سیبکی نیم سرخ و نیمی زرد}
\label{sec:0969}
\addcontentsline{toc}{section}{\nameref{sec:0969}}
\begin{longtable}{l p{0.5cm} r}
سیبکی نیم سرخ و نیمی زرد
&&
زعفران لاله را حکایت کرد
\\
چون جدا گشت عاشق از معشوق
&&
نیمه‌ای خنده بود و نیمی درد
\\
سست پایی بمانده بر جایی
&&
پاک می‌کرد از رخ مه گرد
\\
دست می‌کوفت نیز می‌لافید
&&
کاین چنین صنعتی کسی ناورد
\\
صعوه پرشکسته‌ای دیدی
&&
بیضه چرخ زیر پر پرورد
\\
باز شد خنده خانه این جا
&&
رو بجو یار خنده‌ای ای مرد
\\
ناز تا کی کنند این زشتان
&&
بازگونه همی‌رود این نرد
\\
جفت و طاق از چه روی می‌بازند
&&
چون ندانند جفت را از فرد
\\
بهل این تا بیار خویش رویم
&&
آنک رویش هزار لاله و ورد
\\
\end{longtable}
\end{center}
