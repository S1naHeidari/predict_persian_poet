\begin{center}
\section*{غزل شماره ۲۵۴۶: سحرگه گفتم آن مه را که ای من جسم و تو جانی}
\label{sec:2546}
\addcontentsline{toc}{section}{\nameref{sec:2546}}
\begin{longtable}{l p{0.5cm} r}
سحرگه گفتم آن مه را که ای من جسم و تو جانی
&&
بدین حالم که می‌بینی وزان نالم که می‌دانی
\\
ورای کفر و ایمانی و مرکب تند می‌رانی
&&
چه بس بی‌باک سلطانی همین می‌کن که تو آنی
\\
یکی بازآ به ما بگذر به بیشه جان‌ها بنگر
&&
درختان بین ز خون تر به شکل شاخ مرجانی
\\
شنودی تو که یک خامی ز مردان می‌برد نامی
&&
نمی‌ترسد که خودکامی نهد داغش به پیشانی
\\
مشو تو منکر پاکان بترس از زخم بی‌باکان
&&
که صبر جان غمناکان تو را فانی کند فانی
\\
تو باخویشی به بی‌خویشان مپیچ ای خصم درویشان
&&
مزن تو پنجه با ایشان به دستانی که نتوانی
\\
که شمس الدین تبریزی به جان بخشی و خون ریزی
&&
ز آتش برکند تیزی به قدرت‌های ربانی
\\
\end{longtable}
\end{center}
