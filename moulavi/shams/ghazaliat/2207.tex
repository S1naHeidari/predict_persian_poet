\begin{center}
\section*{غزل شماره ۲۲۰۷: در خلاصه عشق آخر شیوه اسلام کو}
\label{sec:2207}
\addcontentsline{toc}{section}{\nameref{sec:2207}}
\begin{longtable}{l p{0.5cm} r}
در خلاصه عشق آخر شیوه اسلام کو
&&
در کشوف مشکلاتش صاحب اعلام کو
\\
آهوی عرشی که او خود عاشق نافه خود است
&&
التفات او به دانه طوف او بر دام کو
\\
گر چه هر روزی به هجران همچو سالی می‌بود
&&
چونک از هجران گذشتی لیل یا ایام کو
\\
جانور را زادنش از ماده و نر وز رحم
&&
در ولادت‌های روحانی بگو ارحام کو
\\
ساقیا هشیار نتوان عشق را دریافتن
&&
بوی جامت بی‌قرارم کرد آخر جام کو
\\
هست احرامت در این حج جامه هستیت را
&&
از سر سرت بکندن شرط این احرام کو
\\
چونک هستی را فکندی روح اندر روح بین
&&
جوق جوق و جمله فرد آن جایگه اجرام کو
\\
وین همه جان‌های تشنه بحر را چون یافتند
&&
محو گشتند اندر آن جا جز یکی علام کو
\\
دور و نزدیک و ضیاع و شهر و اقلیم و سواد
&&
زین سوی بحر است از آن سو شهر یا اقلام کو
\\
آنچ این تن می‌نویسد بی‌قلم نبود یقین
&&
آنک جان بر خود نویسد حاجت اقلام کو
\\
هوش و عقل آدمیزادی ز سردی وی است
&&
چونک آن می گرم کردش عقل یا احلام کو
\\
اندر آن بی‌هوشی آری هوش دیگر لون هست
&&
هوش بیداری کجا و رأیت احلام کو
\\
مرغ تا اندر قفس باشد به حکم دیگری است
&&
چون قفس بشکست و شد بر وی از آن احکام کو
\\
با حضور عقل آثام است بر نفس از گنه
&&
با حضور عقل عقل این نفس را آثام کو
\\
در مساس تن به تن محتاج حمام است مرد
&&
در مساس روح‌ها خود حاجت حمام کو
\\
گر شوی تو رام خود رامت شود جمله جهان
&&
گر تو رستم زاده‌ای این رخشت آخر رام کو
\\
گر تو ترک پخته گویی خام مسکر باشدت
&&
پس تو را در جام سر آثار و بوی خام کو
\\
چون بخوردی بی‌قدم بخرام در دریای غیب
&&
تو اگر مستی بیا مستانه‌ای بخرام کو
\\
فرض لازم شد عبادت عشق را آخر بگو
&&
فرض و ندب و واجب و تعلیم و استلزام کو
\\
عشقبازی‌های جان و آنگهی اکراه و زور
&&
عشق بربسته کجا و ای ولی اکرام کو
\\
رنج بر رخسار عاشق راحت اندر جان او
&&
رنج خود آوازه ای آن جا به جز انعام کو
\\
خدمتی از خوف خود انعام را باشد ولیک
&&
خدمتی از عشق را امثال کالانعام کو
\\
یک قدم راه است گر توفیق باشد دستگیر
&&
پس حدیث راه دور و رفتن اعوام کو
\\
لیک سایه آن صنم باید که بر تو اوفتد
&&
آن صنم کش مثل اندر جمله اصنام کو
\\
آن خداوند به حق شمس الحق و دین کفو او
&&
در همه آبا و در اجداد و در اعمام کو
\\
درخور در یتیمش کی شود آن هفت بحر
&&
گر نظیرش هست در ارواح یا اجسام کو
\\
در رکاب اسپ عشقش از قبیل روحیان
&&
جز قباد و سنجر و کاووس یا بهرام کو
\\
دیده را از خاک تبریز ارمغان آراد باد
&&
ز آنک جز آن خاک این خاکیش را آرام کو
\\
\end{longtable}
\end{center}
