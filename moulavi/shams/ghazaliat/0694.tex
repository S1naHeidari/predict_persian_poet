\begin{center}
\section*{غزل شماره ۶۹۴: آن کز دهن تو رنگ دارد}
\label{sec:0694}
\addcontentsline{toc}{section}{\nameref{sec:0694}}
\begin{longtable}{l p{0.5cm} r}
آن کز دهن تو رنگ دارد
&&
انصاف که رزق تنگ دارد
\\
وان کس که جدل ببست با تو
&&
با عمر عزیز جنگ دارد
\\
ماهی که بیافت آب حیوان
&&
بر خشک چرا درنگ دارد
\\
در آینه عکس قیصر روم
&&
گر نیست بدانک زنگ دارد
\\
در قدس دلت چو خوک دیدی
&&
ملک قدست فرنگ دارد
\\
ما را باری نگار خوش قول
&&
اندر بر خود چو چنگ دارد
\\
زان زخمه او همیشه این چنگ
&&
پس تن تن و بس ترنگ دارد
\\
هر ذره که پای کوفت با ما
&&
از مشرق چرخ ننگ دارد
\\
هر جان که در این روش بلنگد
&&
جان تو که عذر لنگ دارد
\\
زیرا کاین بحر بس کریمست
&&
آن نیست که او نهنگ دارد
\\
سگ طبع کسی که با چنین شیر
&&
او سرکشی پلنگ دارد
\\
سنگین جانی که با چنین لعل
&&
سودای کلوخ و سنگ دارد
\\
خامش کن و جاه گفت کم جوی
&&
کاین جاه مزاج بنگ دارد
\\
\end{longtable}
\end{center}
