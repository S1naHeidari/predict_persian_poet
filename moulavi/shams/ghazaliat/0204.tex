\begin{center}
\section*{غزل شماره ۲۰۴: سر بر گریبان درست صوفی اسرار را}
\label{sec:0204}
\addcontentsline{toc}{section}{\nameref{sec:0204}}
\begin{longtable}{l p{0.5cm} r}
سر بر گریبان درست صوفی اسرار را
&&
تا چه برآرد ز غیب عاقبت کار را
\\
می که به خم حقست راز دلش مطلق‌ست
&&
لیک بر او هم دق‌ست عاشق بیدار را
\\
آب چو خاکی بده باد در آتش شده
&&
عشق به هم برزده خیمه این چار را
\\
عشق که چادرکشان در پی آن سرخوشان
&&
بر فلک بی‌نشان نور دهد نار را
\\
حلقه این در مزن لاف قلندر مزن
&&
مرغ نه‌ای پر مزن قیر مگو قار را
\\
حرف مرا گوش کن باده جان نوش کن
&&
بیخود و بی‌هوش کن خاطر هشیار را
\\
پیش ز نفی وجود خانه خمار بود
&&
قبله خود ساز زود آن در و دیوار را
\\
مست شود نیک مست از می جام الست
&&
پر کن از می پرست خانه خمار را
\\
داد خداوند دین شمس حق‌ست این ببین
&&
ای شده تبریز چین آن رخ گلنار را
\\
\end{longtable}
\end{center}
