\begin{center}
\section*{غزل شماره ۲۵۶۰: الا ای جان قدس آخر به سوی من نمی‌آیی}
\label{sec:2560}
\addcontentsline{toc}{section}{\nameref{sec:2560}}
\begin{longtable}{l p{0.5cm} r}
الا ای جان قدس آخر به سوی من نمی‌آیی
&&
هماره جان به تن آید تو سوی تن نمی‌آیی
\\
بدم دامن کشان تا تو ز من دامن کشیدستی
&&
ز اشک خون همی‌ریزم در این دامن نمی‌آیی
\\
زهی بی‌آبی جانم چو نیسانت نمی‌بارد
&&
زهی خرمن که سوی این سیه خرمن نمی‌آیی
\\
چو دورم زان نظر کردن نظاره عالمی گشتم
&&
نظاره من بیا گر تو نظر کردن نمی‌آیی
\\
الا ای دل پری خوانی نگویی آن پری را تو
&&
چرا خوابم ببردی گر به سحر و فن نمی‌آیی
\\
الا ای طوق وصل او که در گردن همی‌زیبی
&&
چو قمری ناله می‌دارم که در گردن نمی‌آیی
\\
دل تو همچو سنگ و من چو آهن ثابت اندر عشق
&&
ایا آهن ربا آخر سوی آهن نمی‌آیی
\\
ز ما و من برست آن کس که تو رویی بدو آری
&&
چرا تو سوی این هجران صد چون من نمی‌آیی
\\
فزایش از کجا باشد بهارا چون نمی‌باری
&&
سکونت از کجا آخر سوی مسکن نمی‌آیی
\\
الا ای نور غایب بین در این دیده نمی‌تابی
&&
الا ای ناطقه کلی بدین الکن نمی‌آیی
\\
چو ارزن خرد گشتستم ز بهر مرغ مژده آور
&&
الا ای مرغ مژده آور بدین ارزن نمی‌آیی
\\
همه جان‌ها شده لرزان در این مکمن گه هجران
&&
برای امن این جان‌ها در این مکمن نمی‌آیی
\\
زبان چون سوسن تازه به مدحت ای خوش آوازه
&&
الا گلزار ربانی بدین سوسن نمی‌آیی
\\
الا ای باده شادان به عشق اندر چو استادان
&&
درونت خنب سرمستی چرا از دن نمی‌آیی
\\
معاش خانه جانم اگر نه از قرص خورشید است
&&
چرا ای خانه بی‌خورشید تو روشن نمی‌آیی
\\
اگر نه طالب اویی به خانه خانه خورشید
&&
چرا چون شکل شب دزدان به هر روزن نمی‌آیی
\\
چو صحرای جمال او برای جان بود مؤمن
&&
چرا در خوف می‌باشی چرامؤمننمی‌آیی
\\
تو بشکن جوز این تن را بکوب این مغز را درهم
&&
چرا اندر چراغ عشق چون روغن نمی‌آیی
\\
تو آب و روغنی کردی به نورت ره کجا باشد
&&
مبر تو آب بی‌روغن که بی‌دشمن نمی‌آیی
\\
چه نقد پاک می‌دانی تو خود را وین نمی‌بینی
&&
که اندر دست خود ماندی و در مخزن نمی‌آیی
\\
ز عشق شمس تبریزی چو موسی گفته‌ام ارنی
&&
ز سوی طور تبریزی چرا چون لن نمی‌آیی
\\
\end{longtable}
\end{center}
