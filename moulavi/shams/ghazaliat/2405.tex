\begin{center}
\section*{غزل شماره ۲۴۰۵: ای همه منزل شده از تو ره بی‌رهه}
\label{sec:2405}
\addcontentsline{toc}{section}{\nameref{sec:2405}}
\begin{longtable}{l p{0.5cm} r}
ای همه منزل شده از تو ره بی‌رهه
&&
بی‌قدمی رقص بین بی‌دهنی قهقهه
\\
از سر پستان عشق چونک دمی شیر یافت
&&
قامت سروی گرفت کودکک یک مهه
\\
روی ببینید روی بهر خدا عاشقان
&&
گر چه زنخ زد بسی کوردلی ابلهه
\\
والله کو یوسف است بشنو از من از آنک
&&
بودم با یوسفی هم نمک و هم چهه
\\
چونک نماید جمال گوش سوی غیب دار
&&
عرش پر از نعره‌هاست فرش پر از وه وهه
\\
عاشق باشد کمان خاص بتی همچو تیر
&&
هیچ نپرد کمان گر بشود ده زهه
\\
آنک ز تبریز دید یک نظر شمس دین
&&
طعنه زند بر چله سخره کند بر دهه
\\
\end{longtable}
\end{center}
