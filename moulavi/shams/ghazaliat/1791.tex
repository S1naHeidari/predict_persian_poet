\begin{center}
\section*{غزل شماره ۱۷۹۱: بویی همی‌آید مرا مانا که باشد یار من}
\label{sec:1791}
\addcontentsline{toc}{section}{\nameref{sec:1791}}
\begin{longtable}{l p{0.5cm} r}
بویی همی‌آید مرا مانا که باشد یار من
&&
بر یاد من پیمود می آن باوفا خمار من
\\
کی یاد من رفت از دلش ای در دل و جان منزلش
&&
هر لحظه معجونی کند بهر دل بیمار من
\\
خاصه کنون از جوش او زان جوش بی‌روپوش او
&&
رحمت چو جیحون می رود در قلزم اسرار من
\\
پرده‌ست بر احوال من این گفتی و این قال من
&&
ای ننگ گلزار ضمیر از فکرت چون خار من
\\
کو نعره‌ای یا بانگی اندرخور سودای من
&&
کو آفتابی یا مهی ماننده انوار من
\\
این را رها کن قیصری آمد ز روم اندر حبش
&&
تا زنگ را برهم زند در بردن زنگار من
\\
نظاره کن کز بام او هر لحظه‌ای پیغام او
&&
از روزن دل می رسد در جان آتشخوار من
\\
لاف وصالش چون زنم شرح جمالش چون کنم
&&
کان طوطیان سر می کشند از دام این گفتار من
\\
اندرخور گفتار من منگر به سوی یار من
&&
سینای موسی را نگر در سینه افکار من
\\
امشب در این گفتارها رمزی از آن اسرارها
&&
در پیش بیداران نهد آن دولت بیدار من
\\
آن پیل بی‌خواب ای عجب چون دید هندستان به شب
&&
لیلی درآمد در طلب در جان مجنون وار من
\\
امشب ز سیلاب دلم ویران شود آب و گلم
&&
کآمد به میرابی دل سرچشمه انهار من
\\
بر گوش من زد غره‌ای زان مست شد هر ذره‌ای
&&
بانگ پریدن می رسد زان جعفر طیار من
\\
یا رب به غیر این زبان جان را زبانی ده روان
&&
در قطع و وصل وحدتت تا بسکلد زنار من
\\
صبر از دل من برده‌ای مست و خرابم کرده‌ای
&&
کو علم من کو حلم من کو عقل زیرکسار من
\\
این را بپوشان ای پسر تا نشنود آن سیمبر
&&
ای هر چه غیر داد او گر جان بود اغیار من
\\
ای دلبر بی‌جفت من ای نامده در گفت من
&&
این گفت را زیبی ببخش از زیور ای ستار من
\\
ای طوطی هم خوان ما جز قند بی‌چونی مخا
&&
نی عین گو و نی عرض نی نقش و نی آثار من
\\
از کفر و از ایمان رهد جان و دلم آن سو رود
&&
دوزخ بود گر غیر آن باشد فن و کردار من
\\
ای طبله‌ام پرشکرت من طبل دیگر چون زنم
&&
ای هر شکن از زلف تو صد نافه و عطار من
\\
مهمانیم کن ای پسر این پرده می زن تا سحر
&&
این است لوت و پوت من باغ و رز و دینار من
\\
خفته دلم بیدار شد مست شبم هشیار شد
&&
برقی بزد بر جان من زان ابر بامدرار من
\\
در اولین و آخرین عشقی بننمود این چنین
&&
ابصار عبرت دیده را ای عبره الابصار من
\\
بس سنگ و بس گوهر شدم بس مؤمن و کافر شدم
&&
گه پا شدم گه سر شدم در عودت و تکرار من
\\
روزی برون آیم ز خود فارغ شوم از نیک و بد
&&
گویم صفات آن صمد با نطق درانبار من
\\
جانم نشد زین‌ها خنک یا ذا السماء و الحبک
&&
ای گلرخ و گلزار من ای روضه و ازهار من
\\
امشب چه باشد قرن‌ها ننشاند آن نار و لظی
&&
من آب گشتم از حیا ساکن نشد این نار من
\\
هر دم جوانتر می شوم وز خود نهانتر می شوم
&&
همواره آنتر می شوم از دولت هموار من
\\
چون جزو جانم کل شوم خار گلم هم گل شوم
&&
گشتم سمعنا قل شوم در دوره دوار من
\\
ای کف زنم مختل مشو وی مطربم کاهل مشو
&&
روزی بخواهد عذر تو آن شاه باایثار من
\\
روزی شوی سرمست او روزی ببوسی دست او
&&
روزی پریشانی کنی در عشق چون دستار من
\\
کرده‌ست امشب یاد او جان مرا فرهاد او
&&
فریاد از این قانون نو کاسکست چنگش تار من
\\
مجنون کی باشد پیش او لیلی بود دل ریش او
&&
ناموس لیلییان برد لیلی خوش هنجار من
\\
دست پدر گیر ای پسر با او وفا کن تا سحر
&&
کامشب منم اندر شرر زان ابر آتشبار من
\\
زان می حرام آمد که جان بی‌صبر گردد در زمان
&&
نحس زحل ندهد رهش در دید مه دیدار من
\\
جان گر همی‌لرزد از او صد لرزه را می ارزد او
&&
کو دیده‌های موج جو در قلزم زخار من
\\
من تا قیامت گویمش ای تاجدار پنج و شش
&&
حیرت همی‌حیران شود در مبعث و انشار من
\\
خواهی بگو خواهی مگو صبری ندارم من از او
&&
ای روی او امسال من ای زلف جعدش پار من
\\
خلقان ز مرگ اندر حذر پیشش مرا مردن شکر
&&
ای عمر بی‌او مرگ من وی فخر بی‌او عار من
\\
آه از مه مختل شده وز اختر کاهل شده
&&
از عقده من فارغ شده بی‌دانش فوار من
\\
بر قطب گردم ای صنم از اختران خلوت کنم
&&
کو صبح مصبوحان من کو حلقه احرار من
\\
پهلو بنه ای ذوالبیان با پهلوان کاهلان
&&
بیزار گشتم زین زبان وز قطعه و اشعار من
\\
جز شمس تبریزی مگو جز نصر و پیروزی مگو
&&
جز عشق و دلسوزی مگو جز این مدان اقرار من
\\
\end{longtable}
\end{center}
