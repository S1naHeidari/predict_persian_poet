\begin{center}
\section*{غزل شماره ۲۴۴۷: یک ساعت ار دو قبلکی از عقل و جان برخاستی}
\label{sec:2447}
\addcontentsline{toc}{section}{\nameref{sec:2447}}
\begin{longtable}{l p{0.5cm} r}
یک ساعت ار دو قبلکی از عقل و جان برخاستی
&&
این عقل ما آدم بدی این نفس ما حواستی
\\
ور آدم از ایوان دل درنامدی در آب و گل
&&
تدریس با تقدیس او بالاتر از اسماستی
\\
ور لانسلم گوی ظن اسلمت گفتی چون خلیل
&&
نفس چو سایه سرنگون خورشید سربالاستی
\\
ور هستی تن لا شدی این نفس سربالا شدی
&&
بعد از تمامی لا شدن در وحدت الاستی
\\
گر ضعف و سستی نیستی در دیده خفاش تن
&&
بر جای یک خورشید صد خورشید جان افزاستی
\\
گر نیک و بد نزد خدا یک سان بدی در ابتلا
&&
با جبرئیل ماه رو ابلیس هم سیماستی
\\
ور رازدارستی بشر پیدا نکردی خیر و شر
&&
هر چه که ناپیداستش بر وی همه پیداستی
\\
این حس چون جاسوس ما شد بسته و محبوس ما
&&
چون می‌نبیند اصل را ای کاشکی اعماستی
\\
بنشسته حس نفس خس نزدیک کاسه چون مگس
&&
گر کاسه نگزیدی مگس در حین مگس عنقاستی
\\
استاره‌ها چون کاس‌ها مانند زرین طاس‌ها
&&
آراستش بر طامعان ای کاشکی ناراستی
\\
خاموش باش اندیشه کن کز لامکان آید سخن
&&
با گفت کی پردازیی گر چشم تو آن جاستی
\\
از شمس تبریزی ببین هر ذره را نور یقین
&&
گر ذوق در گفتن بدی هر ذره‌ای گویاستی
\\
\end{longtable}
\end{center}
