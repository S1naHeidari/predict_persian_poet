\begin{center}
\section*{غزل شماره ۲۴۸۹: گر ز تو بوسه ای خرد صد مه و مهر و مشتری}
\label{sec:2489}
\addcontentsline{toc}{section}{\nameref{sec:2489}}
\begin{longtable}{l p{0.5cm} r}
گر ز تو بوسه ای خرد صد مه و مهر و مشتری
&&
تا نفروشی ای صنم کز مه و مهر خوشتری
\\
ور دو هزار جان و دل بر در تو وطن کند
&&
در مگشای ای صنم کز دل و جان تو برتری
\\
آینه کیست تا تو را در دل خویش جا دهد
&&
ای صنما به جان تو کینه در بننگری
\\
دست مده تو چرخ را تا که به پیش اسب او
&&
غاشیه تو را کشد بر سر خود به چاکری
\\
دولت سنگ پاره‌ای گر چه بیافت چاره‌ای
&&
در تن خویش بنگرد بیند وصف گوهری
\\
ای دل بازشکل من جانب دست عشق او
&&
با پر عشق او بپر چند به پر خود پری
\\
در پی شاه شمس دین تا تبریز می‌دوان
&&
لشکر عشق با وی است رو که تو هم ز لشکری
\\
\end{longtable}
\end{center}
