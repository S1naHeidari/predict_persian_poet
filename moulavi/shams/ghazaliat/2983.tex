\begin{center}
\section*{غزل شماره ۲۹۸۳: ای ساقیی که آن می احمر گرفته‌ای}
\label{sec:2983}
\addcontentsline{toc}{section}{\nameref{sec:2983}}
\begin{longtable}{l p{0.5cm} r}
ای ساقیی که آن می احمر گرفته‌ای
&&
وی مطربی که آن غزل تر گرفته‌ای
\\
ای زهره‌ای که آتش در آسمان زدی
&&
مریخ را بگو که چه خنجر گرفته‌ای
\\
از جان و از جهان دل عاشق ربوده‌ای
&&
الحق شکار نازک و لاغر گرفته‌ای
\\
ای هجر تو ز روز قیامت درازتر
&&
این چه قیامتی است که از سر گرفته‌ای
\\
ای آسمان چو دور ندیمانش دیده‌ای
&&
در دور خویش شکل مدور گرفته‌ای
\\
پیلان شیردل چو کفت را مسخرند
&&
این چند پشه را چه مسخر گرفته‌ای
\\
هان ای فقیر روز فقیری گله مکن
&&
زیرا که صد چو ملکت سنجر گرفته‌ای
\\
ای روی خویش دیده تو در روی خوب یار
&&
آیینه‌ای عظیم منور گرفته‌ای
\\
ای دل طپان چرایی چون برگ هر دمی
&&
چون دامن بهار معنبر گرفته‌ای
\\
ای چشم گریه چیست به هر ساعتی تو را
&&
چون کحل از مسیح پیمبر گرفته‌ای
\\
هجده هزار عالم اگر ملک تو شود
&&
بی روی دوست چیز محقر گرفته‌ای
\\
داری تکی که بگذری از خنگ آسمان
&&
کاهل چرا شدی صفت خر گرفته‌ای
\\
خامش کن و زبان دگر گو و رسم نو
&&
این رسم کهنه را چه مکرر گرفته‌ای
\\
\end{longtable}
\end{center}
