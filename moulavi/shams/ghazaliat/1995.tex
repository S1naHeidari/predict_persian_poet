\begin{center}
\section*{غزل شماره ۱۹۹۵: اینک آن انجم روشن که فلک چاکرشان}
\label{sec:1995}
\addcontentsline{toc}{section}{\nameref{sec:1995}}
\begin{longtable}{l p{0.5cm} r}
اینک آن انجم روشن که فلک چاکرشان
&&
اینک آن پردگیانی که خرد چادرشان
\\
همچو اندیشه به هر سینه بود مسکنشان
&&
همچو خورشید به هر خانه فتد لشکرشان
\\
نظر اولشان زنده کند عالم را
&&
در نظر هیچ نگنجد نظر دیگرشان
\\
ای بسا شب که من از آتششان همچو سپند
&&
بوده‌ام نعره زنان رقص کنان بر درشان
\\
گر تو بو می نبری بوی کن اجزای مرا
&&
بو گرفته‌ست دل و جان من از عنبرشان
\\
ور تو بس خشک دماغی به تو بو می نرسد
&&
سر بنه تا برسد بر تو دماغ ترشان
\\
خود چه باشد تر و خشک حیوانی و نبات
&&
مه نبات و حیوان و مه زمین مادرشان
\\
همه عالم به یکی قطره دریا غرقند
&&
چه قدر خورد تواند مگس از شکرشان
\\
\end{longtable}
\end{center}
