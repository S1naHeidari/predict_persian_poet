\begin{center}
\section*{غزل شماره ۱۱۸۲: تو چشم شیخ را دیدن میاموز}
\label{sec:1182}
\addcontentsline{toc}{section}{\nameref{sec:1182}}
\begin{longtable}{l p{0.5cm} r}
تو چشم شیخ را دیدن میاموز
&&
فلک را راست گردیدن میاموز
\\
تو کل را جمع این اجزا مپندار
&&
تو گل را لطف و خندیدن میاموز
\\
تو بگشا چشم تا مهتاب بینی
&&
تو مه را نور بخشیدن میاموز
\\
تو عقل خویش را از می نگهدار
&&
تو می را عقل دزدیدن میاموز
\\
تو باز عقل را صیادی آموز
&&
چنین بیهوده پریدن میاموز
\\
یتیمان فراقش را بخندان
&&
یتیمان را تو نالیدن میاموز
\\
دل مظلوم را ایمن کن از ترس
&&
دل او را تو لرزیدن میاموز
\\
تو ظالم را مده رخصت به تأویل
&&
ستیزا را ستیزیدن میاموز
\\
زبان را پردگی می‌دار چون دل
&&
زبان را پرده بدریدن میاموز
\\
تو در معنی گشا این چشم سر را
&&
چو گوشش حرف برچیدن میاموز
\\
\end{longtable}
\end{center}
