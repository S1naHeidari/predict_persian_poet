\begin{center}
\section*{غزل شماره ۲۷۰۶: چه دلشادم به دلدار خدایی}
\label{sec:2706}
\addcontentsline{toc}{section}{\nameref{sec:2706}}
\begin{longtable}{l p{0.5cm} r}
چه دلشادم به دلدار خدایی
&&
خدایا تو نگهدار از جدایی
\\
بیا ای خواجه بنگر یار ما را
&&
چو از اصحاب و از یاران مایی
\\
بدان شرطی که با ما کژ نبازی
&&
وگر بازی تو با ما برنیایی
\\
دغایانی که با جسم چو پیلند
&&
سوار اسب فرهنگ و کیانی
\\
پیاده گشته و رخ زرد ماندند
&&
ز فرزین بند شاهان بقایی
\\
چه بودی گر بدانستی مهی را
&&
شکسته اختری در بی‌وفایی
\\
وگر مه را نداند ماه ماه است
&&
چگونه مه نه ارضی نی سمایی
\\
که ارضی و سمایی را غروب است
&&
فتد بی‌اختیارش اختفایی
\\
ظهور و اختفای ماه جانی
&&
به دست او است در قدرت نمایی
\\
بسوز ای تن که جان را چون سپندی
&&
به دفع چشم بد چون کیمیایی
\\
که چشم بد به جز بر جسم ناید
&&
به معنی کی رسد چشم هوایی
\\
کناری گیرمش در جامه تن
&&
که جان را زو است هر دم جان فزایی
\\
خیالت هر دمی این جاست با ما
&&
الا ای شمس تبریزی کجایی
\\
\end{longtable}
\end{center}
