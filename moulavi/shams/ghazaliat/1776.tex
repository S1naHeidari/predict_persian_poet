\begin{center}
\section*{غزل شماره ۱۷۷۶: منم آن بنده مخلص که از آن روز که زادم}
\label{sec:1776}
\addcontentsline{toc}{section}{\nameref{sec:1776}}
\begin{longtable}{l p{0.5cm} r}
منم آن بنده مخلص که از آن روز که زادم
&&
دل و جان را ز تو دیدم دل و جان را به تو دادم
\\
کتب العشق بانی بهوی العاشق اعلم
&&
فالیه نتراجع و الیه نتحاکم
\\
چو شراب تو بنوشم چو شراب تو بجوشم
&&
چو قبای تو بپوشم ملکم شاه قبادم
\\
قمر الحسن اتانی و الی الوصل دعانی
&&
و رعانی و سقانی هو فی الفضل مقدم
\\
ز میانم چو گزیدی کمر مهر تو بستم
&&
چو بدیدم کرم تو به کرم دست گشادم
\\
نصر العشق اجیبوا و الی الوصل انیبوا
&&
طلع البدر فطیبوا قدم الحب و انعم
\\
چه کنم نام و نشان را چو ز تو گم نشود کس
&&
چه کنم سیم و درم را چو در این گنج فتادم
\\
لمع العشق توالی و علی الصبر تعالی
&&
طمس البدر هلالا خضع القلب و اسلم
\\
چو تویی شادی و عیدم چه نکوبخت و سعیدم
&&
دل خود بر تو نهادم به خدا نیک نهادم
\\
خدعونی نهبونی اخذونی غلبونی
&&
وعدونی کذبونی فالی من اتظلم
\\
نه بدرم نه بدوزم نه بسازم نه بسوزم
&&
نه اسیر شب و روزم نه گرفتار کسادم
\\
ملک الشرق تشرق و علی الروح تعلق
&&
غسق النفس تفرق ربض الکفر تهدم
\\
چه کساد آید آن را که خریدار تو باشی
&&
چو فزودی تو بهایم که کند طمع مزادم
\\
نفس العشق عتادی و عمیدی و عمادی
&&
فمن العشق تدثر و من العشق تختم
\\
روش زاهد و عابد همگی ترک مراد است
&&
بنما ترک چه گویم چو تویی جمله مرادم
\\
لک یا عشق وجودی و رکوعی و سجودی
&&
لک بخلی لک جودی و لک الدهر منظم
\\
چو مرا دیو ربودی طربم یاد تو بودی
&&
تو چنانم بربودی که بشد یاد ز یادم
\\
الف الدهر بعادی جرح البعد فؤادی
&&
فقد النوم وسادی و سعاداتی نوم
\\
به صفت کشتی نوحم که به باد تو روانم
&&
چو مرا باد تو دادی مده ای دوست به بادم
\\
فاری الشمل تفرق و اری الستر تمزق
&&
و اری السقف تخرق و اری الموج تلاطم
\\
من اگر کشتی نوحم چه عجب چون همه روحم
&&
من اگر فتح و فتوحم چه عجب شاه نژادم
\\
و اری البدر تکور و اری النجم تکدر
&&
و اری البحر تسجر و اری الهلک تفاقم
\\
چو به بحر تو درآیم به مزاج آب حیاتم
&&
چو فتم جانب ساحل حجرم سنگ و جمادم
\\
فقد اهدانی ربی و اتی الجد بحبی
&&
نهض الحب لطبی و تدارک و ترحم
\\
به خدا باز سپیدم که به شاه است امیدم
&&
سوی مردار چه گردم نه چو زاغم نه چو خادم
\\
نزل العشق بداری معه کاس عقاری
&&
هو معراج سواری و علی السطح کسلم
\\
چو بسازیم چو عیدم چو بسوزیم چو عودم
&&
ز تو گریم ز تو خندم ز تو غمگین ز تو شادم
\\
بک احیی و اموت بک امسک و افوت
&&
بک فی الدهر سکوت بک قلبی یتکلم
\\
چو ز تبریز بتابد مه شمس الحق والدین
&&
بفروزد ز مه او فلک جهد و جهادم
\\
\end{longtable}
\end{center}
