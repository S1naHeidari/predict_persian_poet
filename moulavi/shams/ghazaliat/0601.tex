\begin{center}
\section*{غزل شماره ۶۰۱: آن عشق که از پاکی از روح حشم دارد}
\label{sec:0601}
\addcontentsline{toc}{section}{\nameref{sec:0601}}
\begin{longtable}{l p{0.5cm} r}
آن عشق که از پاکی از روح حشم دارد
&&
بشنو که چه می‌گوید بنگر که چه دم دارد
\\
گر جسم تنک دارد جان تو سبک دارد
&&
هر چند که صد لشکر در کتم عدم دارد
\\
گر مانده‌ای در گل روی آر به صاحب دل
&&
کو ملک ابد بخشد کو تاج قدم دارد
\\
ای دل که جهان دیدی بسیار بگردیدی
&&
بنمای که را دیدی کز عشق رقم دارد
\\
ای مرکب خود کشته وی گرد جهان گشته
&&
بازآی به خورشیدی کز سینه کرم دارد
\\
آن سینه و چون سینه صیقل ده آیینه
&&
آن سینه که اندر خود صد باغ ارم دارد
\\
این عشق همی‌گوید کان کس که مرا جوید
&&
شرطیست که همچون زر در کوره قدم دارد
\\
من سیمتنی خواهم من همچو منی خواهم
&&
بیزارم از آن زشتی کو سیم و درم دارد
\\
القاب صلاح الدین بر لوح چو پیدا شد
&&
انصاف بسی منت بر لوح و قلم دارد
\\
\end{longtable}
\end{center}
