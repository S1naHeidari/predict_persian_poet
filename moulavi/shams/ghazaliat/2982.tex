\begin{center}
\section*{غزل شماره ۲۹۸۲: ای ساقیی که آن می احمر گرفته‌ای}
\label{sec:2982}
\addcontentsline{toc}{section}{\nameref{sec:2982}}
\begin{longtable}{l p{0.5cm} r}
ای ساقیی که آن می احمر گرفته‌ای
&&
وی مطربی که آن غزل تر گرفته‌ای
\\
ای دلبری که ساقی و مطرب فنا شدند
&&
تا تو نقاب از رخ عبهر گرفته‌ای
\\
ای میر مجلسی که تو را عشق نام گشت
&&
این چه قیامت است که از سر گرفته‌ای
\\
ای خم خسروان که تو داروی هر غمی
&&
رنجور نیستی تو چرا سر گرفته‌ای
\\
جانی است بس لطیف و جهانی است بس ظریف
&&
وین هر دو پرده را ز میان برگرفته‌ای
\\
از جان و از جهان دل عاشق ربوده‌ای
&&
الحق شکار نازک و لاغر گرفته‌ای
\\
ای آنک تو شکار چنین دام گشته‌ای
&&
ملک هزار خسرو و سنجر گرفته‌ای
\\
در عین کفر جوهر ایمان ربوده‌ای
&&
در دوزخی و جنت و کوثر گرفته‌ای
\\
ای عارفی که از سر معروف واقفی
&&
وی ساده‌ای که رنگ قلندر گرفته‌ای
\\
در بحر قلزمی و تو را بحر تا به کعب
&&
در آتشی و خوی سمندر گرفته‌ای
\\
ای گل که جامه‌ها بدریدی ز عاشقی
&&
تا خانه‌ای میانه شکر گرفته‌ای
\\
ای باد از تکبر پرهیز کن ز مشک
&&
چون بوی آن دو زلف معنبر گرفته‌ای
\\
ای غمزه‌هات مست چو ساقی تویی بده
&&
یک دم خمش مباد چو ساغر گرفته‌ای
\\
بهر نثار مفخر تبریز شمس دین
&&
ای روی زرد سکه زرگر گرفته‌ای
\\
\end{longtable}
\end{center}
