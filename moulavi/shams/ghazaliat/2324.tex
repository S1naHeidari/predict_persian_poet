\begin{center}
\section*{غزل شماره ۲۳۲۴: روزی تو مرا بینی میخانه درافتاده}
\label{sec:2324}
\addcontentsline{toc}{section}{\nameref{sec:2324}}
\begin{longtable}{l p{0.5cm} r}
روزی تو مرا بینی میخانه درافتاده
&&
دستار گرو کرده بیزار ز سجاده
\\
من مست و حریفم مست زلف خوش او در دست
&&
احسنت زهی شاهد شاباش زهی باده
\\
لب نیز شده مستک گم کرده ره بوسه
&&
من مستک و لب مستک و آن بوسه قواده
\\
این دلبر پرفتنه با جمله دستان‌ها
&&
خوش خفته و جمله شب این عشرت آماده
\\
این صورت‌ها جمله از پرتو او باشد
&&
و آن روح قدس پاک است از صورت‌ها ساده
\\
شمس الحق تبریزی شرحی است مر این‌ها را
&&
آن خسرو روحانی شاهنشه شه زاده
\\
\end{longtable}
\end{center}
