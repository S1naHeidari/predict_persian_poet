\begin{center}
\section*{غزل شماره ۹۷۲: عاشقانی که باخبر میرند}
\label{sec:0972}
\addcontentsline{toc}{section}{\nameref{sec:0972}}
\begin{longtable}{l p{0.5cm} r}
عاشقانی که باخبر میرند
&&
پیش معشوق چون شکر میرند
\\
از الست آب زندگی خوردند
&&
لاجرم شیوه دگر میرند
\\
چونک در عاشقی حشر کردند
&&
نی چو این مردم حشر میرند
\\
از فرشته گذشته‌اند به لطف
&&
دور از ایشان که چون بشر میرند
\\
تو گمان می‌بری که شیران نیز
&&
چون سگان از برون در میرند
\\
بدود شاه جان به استقبال
&&
چونک عشاق در سفر میرند
\\
همه روشن شوند چون خورشید
&&
چونک در پای آن قمر میرند
\\
عاشقانی که جان یک دگرند
&&
همه در عشق همدگر میرند
\\
همه را آب عشق بر جگر است
&&
همه آیند و در جگر میرند
\\
همه هستند همچو در یتیم
&&
نه بر مادر و پدر میرند
\\
عاشقان جانب فلک پرند
&&
منکران در تک سقر میرند
\\
عاشقان چشم غیب بگشایند
&&
باقیان جمله کور و کر میرند
\\
و آنک شب‌ها نخفته‌اند ز بیم
&&
جمله بی‌خوف و بی‌خطر میرند
\\
و آنک این جا علف پرست بدند
&&
گاو بودند و همچو خر میرند
\\
و آنک امروز آن نظر جستند
&&
شاد و خندان در آن نظر میرند
\\
شاهشان بر کنار لطف نهد
&&
نی چنین خوار و محتضر میرند
\\
و انک اخلاق مصطفی جویند
&&
چون ابوبکر و چون عمر میرند
\\
دور از ایشان فنا و مرگ ولیک
&&
این به تقدیر گفتم ار میرند
\\
\end{longtable}
\end{center}
