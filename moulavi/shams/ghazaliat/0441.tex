\begin{center}
\section*{غزل شماره ۴۴۱: بنمای رخ که باغ و گلستانم آرزوست}
\label{sec:0441}
\addcontentsline{toc}{section}{\nameref{sec:0441}}
\begin{longtable}{l p{0.5cm} r}
بنمای رخ که باغ و گلستانم آرزوست
&&
بگشای لب که قند فراوانم آرزوست
\\
ای آفتاب حسن برون آ دمی ز ابر
&&
کآن چهره مشعشع تابانم آرزوست
\\
بشنیدم از هوای تو آواز طبل باز
&&
باز آمدم که ساعد سلطانم آرزوست
\\
گفتی ز ناز بیش مرنجان مرا برو
&&
آن گفتنت که بیش مرنجانم آرزوست
\\
وآن دفع گفتنت که برو شه به خانه نیست
&&
وآن ناز و باز و تندی دربانم آرزوست
\\
در دست هر که هست ز خوبی قراضه‌هاست
&&
آن معدن ملاحت و آن کانم آرزوست
\\
این نان و آب چرخ چو سیل است بی وفا
&&
من ماهیم نهنگم عمانم آرزوست
\\
یعقوب وار وا اسفاها همی‌زنم
&&
دیدار خوب یوسف کنعانم آرزوست
\\
والله که شهر بی تو مرا حبس می‌شود
&&
آوارگی و کوه و بیابانم آرزوست
\\
زین همرهان سست عناصر دلم گرفت
&&
شیر خدا و رستم دستانم آرزوست
\\
جانم ملول گشت ز فرعون و ظلم او
&&
آن نور روی موسی عمرانم آرزوست
\\
زین خلق پرشکایت گریان شدم ملول
&&
آن های هوی و نعره مستانم آرزوست
\\
گویاترم ز بلبل اما ز رشک عام
&&
مهر است بر دهانم و افغانم آرزوست
\\
دی شیخ با چراغ همی‌گشت گرد شهر
&&
کز دیو و دد ملولم و انسانم آرزوست
\\
گفتند یافت می‌نشود جسته‌ایم ما
&&
گفت آنک یافت می‌نشود آنم آرزوست
\\
هر چند مفلسم نپذیرم عقیق خرد
&&
کان عقیق نادر ارزانم آرزوست
\\
پنهان ز دیده‌ها و همه دیده‌ها از اوست
&&
آن آشکار صنعت پنهانم آرزوست
\\
خود کار من گذشت ز هر آرزو و آز
&&
از کان و از مکان پی ارکانم آرزوست
\\
گوشم شنید قصه ایمان و مست شد
&&
کو قسم چشم؟ صورت ایمانم آرزوست
\\
یک دست جام باده و یک دست جعد یار
&&
رقصی چنین میانه میدانم آرزوست
\\
می‌گوید آن رباب که مردم ز انتظار
&&
دست و کنار و زخمه عثمانم آرزوست
\\
من هم رباب عشقم و عشقم ربابی است
&&
وآن لطف‌های زخمه رحمانم آرزوست
\\
باقی این غزل را ای مطرب ظریف
&&
زین سان همی‌شمار که زین سانم آرزوست
\\
بنمای شمس مفخر تبریز رو ز شرق
&&
من هدهدم حضور سلیمانم آرزوست
\\
\end{longtable}
\end{center}
