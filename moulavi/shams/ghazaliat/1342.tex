\begin{center}
\section*{غزل شماره ۱۳۴۲: چه کارستان که داری اندر این دل}
\label{sec:1342}
\addcontentsline{toc}{section}{\nameref{sec:1342}}
\begin{longtable}{l p{0.5cm} r}
چه کارستان که داری اندر این دل
&&
چه بت‌ها می‌نگاری اندر این دل
\\
بهار آمد زمان کشت آمد
&&
کی داند تا چه کاری اندر این دل
\\
حجاب عزت ار بستی ز بیرون
&&
به غایت آشکاری اندر این دل
\\
در آب و گل فروشد پای طالب
&&
سرش را می‌بخاری اندر این دل
\\
دل از افلاک اگر افزون نبودی
&&
نکردی مه سواری اندر این دل
\\
اگر دل نیستی شهر معظم
&&
نکردی شهریاری اندر این دل
\\
عجایب بیشه‌ای آمد دل ای جان
&&
که تو میر شکاری اندر این دل
\\
ز بحر دل هزاران موج خیزد
&&
چو جوهرها بیاری اندر این دل
\\
خمش کردم که در فکرت نگنجد
&&
چو وصف دل شماری اندر این دل
\\
\end{longtable}
\end{center}
