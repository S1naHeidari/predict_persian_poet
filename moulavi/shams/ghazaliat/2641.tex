\begin{center}
\section*{غزل شماره ۲۶۴۱: گیرم که نبینی رخ آن دختر چینی}
\label{sec:2641}
\addcontentsline{toc}{section}{\nameref{sec:2641}}
\begin{longtable}{l p{0.5cm} r}
گیرم که نبینی رخ آن دختر چینی
&&
از جنبش او جنبش این پرده نبینی
\\
از تابش آن مه که در افلاک نهان است
&&
صد ماه بدیدی تو در اجزای زمینی
\\
ای برگ پریشان شده در باد مخالف
&&
گر باد نبینی تو نبینی که چنینی
\\
گر باد ز اندیشه نجنبد تو نجنبی
&&
و آن باد اگر هیچ نشیند تو نشینی
\\
عرش و فلک و روح در این گردش احوال
&&
اشتر به قطارند و تو آن بازپسینی
\\
می‌جنب تو بر خویش و همی‌خور تو از این خون
&&
کاندر شکم چرخ یکی طفل جنینی
\\
در چرخ دلت ناگه یک درد درآید
&&
سر برزنی از چرخ بدانی که نه اینی
\\
ماه نهمت چهره شمس الحق تبریز
&&
ای آنک امان دو جهان را تو امینی
\\
تا ماه نهم صبر کن ای دل تو در این خون
&&
آن مه تویی ای شاه که شمس الحق و دینی
\\
\end{longtable}
\end{center}
