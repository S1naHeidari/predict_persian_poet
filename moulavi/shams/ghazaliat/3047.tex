\begin{center}
\section*{غزل شماره ۳۰۴۷: به جان تو ای طایی که سوی ما بازآیی}
\label{sec:3047}
\addcontentsline{toc}{section}{\nameref{sec:3047}}
\begin{longtable}{l p{0.5cm} r}
به جان تو ای طایی که سوی ما بازآیی
&&
تو هر چه می‌فرمایی همه شکر می‌خایی
\\
برآ به بام ای خوش خو به بام ما آور رو
&&
دو سه قدم نه این سو رضای این مستان جو
\\
اگر ملولی بستان قنینه‌ای از مستان
&&
که راحت جانست آن بدار دست از دستان
\\
ایا بت جان افزا نه وعده کردی ما را
&&
که من بیایم فردا زهی فریب و سودا
\\
ایا بت ناموسی لب مرا گر بوسی
&&
رها کنی سالوسی جلا کنی طاووسی
\\
سری ز روزن درکن وثاق پرشکر کن
&&
جهان پر از گوهر کن بیا ز ما باور کن
\\
نهال نیکی بنشان درخت گل را بفشان
&&
بیا به نزد خویشان دغل مکن با ایشان
\\
دو دیده را خوابی ده زمانه را تابی ده
&&
به تشنگان آبی ده به غوره دوشابی ده
\\
بگیر چنگ و تنتن دل از جدایی برکن
&&
بیار باده روشن خمار ما را بشکن
\\
از این ملولی بگذر به سوی روزن منگر
&&
شراب با یاران خور میان یاران خوشتر
\\
ز بیخودی آشفتم به دلبر خود گفتم
&&
که با غمت من جفتم به هر سوی که افتم
\\
به ضرب دستش بنگر به چشم مستش بنگر
&&
به زلف شستش بنگر به هر چه هستش بنگر
\\
چو دامن او گیرم عظیم باتوفیرم
&&
چو انگبین و شیرم به پیش لطفش میرم
\\
مزن نگارا بربط به پیش مشتی خربط
&&
مران تو کشتی بی‌شط بگیر راه اوسط
\\
بکار تخم زیبا که سبز گردد فردا
&&
که هر چه کاری این جا تو را بروید ده تا
\\
اگر تو تخمی کشتی چرا پشیمان گشتی
&&
اگر به کوه و دشتی برو که زرین طشتی
\\
ملول گشتی‌ای کش بخسب و رو اندرکش
&&
ز عالم پرآتش گریز پنهان خوش خوش
\\
ببند از این سو دیده برو ره دزدیده
&&
به غیب آرامیده به پر جان پریده
\\
نشسته خسبد عاشق که هست صبرش لایق
&&
بود خفیف و سابق برای عذرا وامق
\\
مگو دگر کوته کن سکوت را همره کن
&&
نظر به شاهنشه کن نظاره آن مه کن
\\
\end{longtable}
\end{center}
