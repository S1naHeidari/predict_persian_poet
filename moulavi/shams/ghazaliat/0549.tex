\begin{center}
\section*{غزل شماره ۵۴۹: آب زنید راه را هین که نگار می‌رسد}
\label{sec:0549}
\addcontentsline{toc}{section}{\nameref{sec:0549}}
\begin{longtable}{l p{0.5cm} r}
آب زنید راه را هین که نگار می‌رسد
&&
مژده دهید باغ را بوی بهار می‌رسد
\\
راه دهید یار را آن مه ده چهار را
&&
کز رخ نوربخش او نور نثار می‌رسد
\\
چاک شدست آسمان غلغله ایست در جهان
&&
عنبر و مشک می‌دمد سنجق یار می‌رسد
\\
رونق باغ می‌رسد چشم و چراغ می‌رسد
&&
غم به کناره می‌رود مه به کنار می‌رسد
\\
تیر روانه می‌رود سوی نشانه می‌رود
&&
ما چه نشسته‌ایم پس شه ز شکار می‌رسد
\\
باغ سلام می‌کند سرو قیام می‌کند
&&
سبزه پیاده می‌رود غنچه سوار می‌رسد
\\
خلوتیان آسمان تا چه شراب می‌خورند
&&
روح خراب و مست شد عقل خمار می‌رسد
\\
چون برسی به کوی ما خامشی است خوی ما
&&
زان که ز گفت و گوی ما گرد و غبار می‌رسد
\\
\end{longtable}
\end{center}
