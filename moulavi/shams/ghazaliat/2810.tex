\begin{center}
\section*{غزل شماره ۲۸۱۰: ای بداده دیده‌های خلق را حیرانیی}
\label{sec:2810}
\addcontentsline{toc}{section}{\nameref{sec:2810}}
\begin{longtable}{l p{0.5cm} r}
ای بداده دیده‌های خلق را حیرانیی
&&
وی ز لشکرهای عشقت هر طرف ویرانیی
\\
ای مبارک چاشتگاهی کآفتاب روی تو
&&
عالم دل را کند اندر صفا نورانیی
\\
دم به دم خط می‌دهد جان‌ها که ما بنده توایم
&&
ای سراسر بندگی عشق تو سلطانیی
\\
تا چه می‌بینند جان‌ها هر دمی در روی تو
&&
وز چه باشد هر زمانیشان چنین رقصانیی
\\
از چه هر شب پاسبان بام عشق تو شوند
&&
وز چه هر روزی بودشان بر درت دربانیی
\\
این چه جام است این که گردان کرده‌ای بر جان‌ها
&&
آب حیوان است این یا آتشی روحانیی
\\
این چه سر گفتی تو با دل‌ها که خصم جان شدند
&&
این چه دادی درد را تا می‌کند درمانیی
\\
روستایی را چه آموزید نور عشق تو
&&
تا ز لوح غیب دادش هر دمی خط خوانیی
\\
شمس تبریزی فروکن سر از این قصر بلند
&&
تا بقایی دیده آید در جهان فانیی
\\
\end{longtable}
\end{center}
