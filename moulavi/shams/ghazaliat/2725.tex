\begin{center}
\section*{غزل شماره ۲۷۲۵: مندیش از آن بت مسیحایی}
\label{sec:2725}
\addcontentsline{toc}{section}{\nameref{sec:2725}}
\begin{longtable}{l p{0.5cm} r}
مندیش از آن بت مسیحایی
&&
تا دل نشود سقیم و سودایی
\\
لاحول کن و ره سلامت گیر
&&
مندیش از آن جمال و زیبایی
\\
فرصت ز کجا که تا کنی لاحول
&&
چون نیست از او دمی شکیبایی
\\
ماهی ز کجا شکیبد از دریا
&&
یا طوطی روح از شکرخایی
\\
چون دین نشود مشوش و ایمان
&&
زان زلف مشوش چلیپایی
\\
اخگر شده دل در آتش رویش
&&
بگرفته عقول بادپیمایی
\\
دل با دو جهان چراست بیگانه
&&
کز جا برمد صفات بی‌جایی
\\
ای تن تو و تره زار این عالم
&&
چون خو کردی که ژاژ می‌خایی
\\
ای عقل برو مشاطگی می‌کن
&&
می‌ناز بدین که عالم آرایی
\\
بگرفته معلمی در این مکتب
&&
با حفصی اگر چه کارافزایی
\\
ای بر لب بحر همچو بوتیمار
&&
دستور نه تا لبی بیالایی
\\
این‌ها همه رفت ساقیا برخیز
&&
با تشنه دلان نمای سقایی
\\
مشرق چه کند چراغ افروزی
&&
سلطان چه کند شهی و مولایی
\\
مصقول شود چو چهره گردون
&&
چون دود سیاه را تو بزدایی
\\
درده تو شراب جان فزایی را
&&
کز وی آموخت باده صهبایی
\\
یکتا عیشی است و عشرتی کز وی
&&
جان عارف گرفت یکتایی
\\
از دست تو هر که را دهد این دست
&&
بی عقبه لا شده است الایی
\\
ای شاد دمی که آن صراحی را
&&
از دور به مست خویش بنمایی
\\
چون گوهر می‌بتافت بر خاکم
&&
خاک تن من نمود مینایی
\\
دریای صفات عشق می‌جوشد
&&
رمزی دو بگویم ار بفرمایی
\\
ور نی بهلم ستیر و بربسته
&&
من دانم و یار من به تنهایی
\\
زین بگذشتم بیار حمرا را
&&
صفراشکن هزار صفرایی
\\
تا روز رهد ز غصه روزی
&&
وین هندوی شب رهد ز لالایی
\\
در حال مگر درت فروبسته‌ست
&&
کاندر پیکار قال می‌آیی
\\
\end{longtable}
\end{center}
