\begin{center}
\section*{غزل شماره ۱۵۵۲: ما آفت جان عاشقانیم}
\label{sec:1552}
\addcontentsline{toc}{section}{\nameref{sec:1552}}
\begin{longtable}{l p{0.5cm} r}
ما آفت جان عاشقانیم
&&
نی خانه نشین و خانه بانیم
\\
اندر دل تو اگر خیال است
&&
می پنداری که ما ندانیم
\\
اسرار خیال‌ها نه ماییم
&&
هر سودا را نه ما پزانیم
\\
دل‌ها بر ما کبوترانند
&&
هر لحظه به جانبی پرانیم
\\
تن گفت به جان از این نشان کو
&&
جان گفت که سر به سر نشانیم
\\
آخر تو به گفت خویش بنگر
&&
کاندر دهن تو می نشانیم
\\
هر دم بغل تو را گرفته
&&
در راحت و رنج می کشانیم
\\
تا آتش و آب و بادطبعی
&&
ما باده خاکیت چشانیم
\\
وان گاه دهان تو بشوییم
&&
آن جا برسی که ما نهانیم
\\
چون رخت تو در نهان کشیدیم
&&
آنگه بینی که ما چه سانیم
\\
چون نقش تو از زمین ببردیم
&&
دانی که عجایب زمانیم
\\
هر سو نگری زمان نبینی
&&
پس لاف زنی که لامکانیم
\\
همرنگ دلت شود تن تو
&&
در رقص آیی که جمله جانیم
\\
لب بر لب ما نهی تو بی‌لب
&&
اقرار کنی که همزبانیم
\\
ای شمس الدین و شاه تبریز
&&
از بندگیت شهنشهانیم
\\
\end{longtable}
\end{center}
