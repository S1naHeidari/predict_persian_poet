\begin{center}
\section*{غزل شماره ۵۳۸: گر آتش دل برزند بر مؤمن و کافر زند}
\label{sec:0538}
\addcontentsline{toc}{section}{\nameref{sec:0538}}
\begin{longtable}{l p{0.5cm} r}
گر آتش دل برزند بر مؤمن و کافر زند
&&
صورت همه پران شود گر مرغ معنی پر زند
\\
عالم همه ویران شود جان غرقه طوفان شود
&&
آن گوهری کو آب شد آب بر گوهر زند
\\
پیدا شود سر نهان ویران شود نقش جهان
&&
موجی برآید ناگهان بر گنبد اخضر زند
\\
گاهی قلم کاغذ شود کاغذ گهی بیخود شود
&&
جان خصم نیک و بد شود هر لحظه‌ای خنجر زند
\\
هر جان که اللهی شود در لامکان پیدا شود
&&
ماری بود ماهی شود از خاک بر کوثر زند
\\
از جا سوی بی‌جا شود در لامکان پیدا شود
&&
هر سو که افتد بعد از این بر مشک و بر عنبر زند
\\
در فقر درویشی کند بر اختران پیشی کند
&&
خاک درش خاقان بود حلقه درش سنجر زند
\\
از آفتاب مشتعل هر دم ندا آید به دل
&&
تو شمع این سر را بهل تا باز شمعت سر زند
\\
تو خدمت جانان کنی سر را چرا پنهان کنی
&&
زر هر دمی خوشتر شود از زخم کان زرگر زند
\\
دل بیخود از باده ازل می‌گفت خوش خوش این غزل
&&
گر می فروگیرد دمش این دم از این خوشتر زند
\\
\end{longtable}
\end{center}
