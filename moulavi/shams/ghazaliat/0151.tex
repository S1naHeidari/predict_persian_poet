\begin{center}
\section*{غزل شماره ۱۵۱: سر برون کن از دریچه جان ببین عشاق را}
\label{sec:0151}
\addcontentsline{toc}{section}{\nameref{sec:0151}}
\begin{longtable}{l p{0.5cm} r}
سر برون کن از دریچه جان ببین عشاق را
&&
از صبوحی‌های شاه آگاه کن فساق را
\\
از عنایت‌های آن شاه حیات انگیز ما
&&
جان نو ده مر جهاد و طاعت و انفاق ما
\\
چون عنایت‌های ابراهیم باشد دستگیر
&&
سر بریدن کی زیان دارد دلا اسحاق را
\\
طاق و ایوانی بدیدم شاه ما در وی چو ماه
&&
نقش‌ها می‌رست و می‌شد در نهان آن طاق را
\\
غلبه جان‌ها در آن جا پشت پا بر پشت پا
&&
رنگ رخ‌ها بی‌زبان می‌گفت آن اذواق را
\\
سرد گشتی باز ذوق مستی و نقل و سماع
&&
چون بدیدندی به ناگه ماه خوب اخلاق را
\\
چون بدید آن شاه ما بر در نشسته بندگان
&&
وان در از شکلی که نومیدی دهد مشتاق را
\\
شاه ما دستی بزد بشکست آن در را چنانک
&&
چشم کس دیگر نبیند بند یا اغلاق را
\\
پاره‌های آن در بشکسته سبز و تازه شد
&&
کنچ دست شه برآمد نیست مر احراق را
\\
جامه جانی که از آب دهانش شسته شد
&&
تا چه خواهد کرد دست و منت دقاق را
\\
آن که در حبسش از او پیغام پنهانی رسید
&&
مست آن باشد نخواهد وعده اطلاق را
\\
بوی جانش چون رسد اندر عقیم سرمدی
&&
زود از لذت شود شایسته مر اعلاق را
\\
شاه جانست آن خداوند دل و سر شمس دین
&&
کش مکان تبریز شد آن چشمه رواق را
\\
ای خداوندا برای جانت در هجرم مکوب
&&
همچو گربه می‌نگر آن گوشت بر معلاق را
\\
ور نه از تشنیع و زاری‌ها جهانی پر کنم
&&
از فراق خدمت آن شاه من آفاق را
\\
پرده صبرم فراق پای دارت خرق کرد
&&
خرق عادت بود اندر لطف این مخراق را
\\
\end{longtable}
\end{center}
