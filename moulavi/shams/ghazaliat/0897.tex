\begin{center}
\section*{غزل شماره ۸۹۷: شرح دهم من که شب از چه سیه دل بود}
\label{sec:0897}
\addcontentsline{toc}{section}{\nameref{sec:0897}}
\begin{longtable}{l p{0.5cm} r}
شرح دهم من که شب از چه سیه دل بود
&&
هر کی خورد خون خلق زشت و سیه دل شود
\\
چون جگر عاشقان می‌خورد این شب به ظلم
&&
دود سیاهی ظلم بر دل شب می‌دمد
\\
عاقله شب تویی بازرهانش ز ظلم
&&
نیم شبی بر فلک راه بزن بر رصد
\\
تا برهد شب ز ظلم ما برهیم از ظلام
&&
ای که جهان فراخ بی‌تو چو گور و لحد
\\
شب همه روشن شود دوزخ گلشن شود
&&
چونک بتابد ز تو پرتو نور احد
\\
سینه کبودی چرخ پرتو سینه منست
&&
جرعه خون دلم تا به شفق می‌رسد
\\
فارغ و دلخوش بدم سرخوش و سرکش بدم
&&
بولهب غم ببست گردن من در مسد
\\
تیر غم تو روان ما هدف آسمان
&&
جان پی غم هم دوان زانک غمش می‌کشد
\\
جانم اگر صافیست دردی لطف توست
&&
لطف تو پاینده باد بر سر جان تا ابد
\\
قافله عصمتت گشت خفیر ار نه خود
&&
راه زن از ریگ ره بود فزون در عدد
\\
سر به خس اندرکشید مرغ غم از بیم آنک
&&
بر سر غم می‌زند شادی تو صد لگد
\\
چشم چپم می‌پرد بازو من می‌جهد
&&
شاید اگر جان من دیگ هوس‌ها پزد
\\
جان مثل گلبنان حامله غنچه‌هاست
&&
جانب غنچه صبی باد صبا می‌وزد
\\
زود دهانم ببند چون دهن غنچه‌ها
&&
زانک چنین لقمه‌ای خورد و زبان می‌گزد
\\
\end{longtable}
\end{center}
