\begin{center}
\section*{غزل شماره ۱۶۷۸: من اگر نالم اگر عذر آرم}
\label{sec:1678}
\addcontentsline{toc}{section}{\nameref{sec:1678}}
\begin{longtable}{l p{0.5cm} r}
من اگر نالم اگر عذر آرم
&&
پنبه در گوش کند دلدارم
\\
هر جفایی که کند می رسدش
&&
هر جفایی که کند بردارم
\\
گر مرا او به عدم انگارد
&&
ستمش را به کرم انگارم
\\
داروی درد دلم درد وی است
&&
دل به دردش ز چه رو نسپارم
\\
عزت و حرمتم آنگه باشد
&&
که کند عشق عزیزش خوارم
\\
باده آنگه شود انگور تنم
&&
که بکوبد به لگد عصارم
\\
جان دهم زیر لگد چون انگور
&&
تا طرب ساز شود اسرارم
\\
گر چه انگور همه خون گرید
&&
که از این جور و جفا بیزارم
\\
پنبه در گوش کند کوبنده
&&
که من از جهل نمی‌افشارم
\\
تو گر انکار کنی معذوری
&&
لیک من بوالحکم این کارم
\\
چون ز سعی و قدمم سر کردی
&&
آنگهی شکر کنی بسیارم
\\
\end{longtable}
\end{center}
