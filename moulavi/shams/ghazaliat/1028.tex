\begin{center}
\section*{غزل شماره ۱۰۲۸: ذاتت عسلست ای جان گفتت عسلی دیگر}
\label{sec:1028}
\addcontentsline{toc}{section}{\nameref{sec:1028}}
\begin{longtable}{l p{0.5cm} r}
ذاتت عسلست ای جان گفتت عسلی دیگر
&&
ای عشق تو را در جان هر دم عملی دیگر
\\
از روی تو در هر جان باغ و چمنی خندان
&&
وز جعد تو در هر دل از مشک تلی دیگر
\\
مه را ز غمت باشد گه دق و گه استسقا
&&
مه زین خللی رسته از صد خللی دیگر
\\
با لطف بهارت دل چون برگ چرا لرزد
&&
ترسد که خزان آید آرد دغلی دیگر
\\
هر سرمه و هر دارو کز خاک درت نبود
&&
در دیده دل آرد درد و سبلی دیگر
\\
ابلیس ز لطف تو اومید نمی‌برد
&&
هر دم ز تو می‌تابد در وی املی دیگر
\\
فرعون ز فرعونی آمنت به جان گفته
&&
بر خرقه جان دیده ز ایمان تکلی دیگر
\\
خورشید وصال تو روزی به جمل آید
&&
در چرخ دلم یابد برج حملی دیگر
\\
اجزای زمین را بین بر روی زمین رقصان
&&
این جوق چو بنشیند آید بدلی دیگر
\\
بر روی زمین جان را چون رو شرف و نوری
&&
در زیر زمین تن را چون تخم اجلی دیگر
\\
تا چند غزل‌ها را در صورت و حرف آری
&&
بی‌صورت و حرف از جان بشنو غزلی دیگر
\\
\end{longtable}
\end{center}
