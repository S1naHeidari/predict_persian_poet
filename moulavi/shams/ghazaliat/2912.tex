\begin{center}
\section*{غزل شماره ۲۹۱۲: با من ای عشق امتحان‌ها می‌کنی}
\label{sec:2912}
\addcontentsline{toc}{section}{\nameref{sec:2912}}
\begin{longtable}{l p{0.5cm} r}
با من ای عشق امتحان‌ها می‌کنی
&&
واقفی بر عجزم اما می‌کنی
\\
ترجمان سر دشمن می‌شوی
&&
ظن کژ را در دلش جا می‌کنی
\\
هم تو اندر بیشه آتش می‌زنی
&&
هم شکایت را تو پیدا می‌کنی
\\
تا گمان آید که بر تو ظلم رفت
&&
چون ضعیفان شور و شکوی می‌کنی
\\
آفتابی ظلم بر تو کی کند
&&
هر چه می‌خواهی ز بالامی کنی
\\
می‌کنی ما را حسود همدگر
&&
جنگ ما را خوش تماشا می‌کنی
\\
عارفان را نقد شربت می‌دهی
&&
زاهدان را مست فردا می‌کنی
\\
مرغ مرگ اندیش را غم می‌دهی
&&
بلبلان را مست و گویا می‌کنی
\\
زاغ را مشتاق سرگین می‌کنی
&&
طوطی خود را شکرخا می‌کنی
\\
آن یکی را می‌کشی در کان و کوه
&&
وین دگر را رو به دریا می‌کنی
\\
از ره محنت به دولت می‌کشی
&&
یا جزای زلت ما می‌کنی
\\
اندر این دریا همه سود است و داد
&&
جمله احسان و مواسا می‌کنی
\\
این سر نکته است پایانش تو گوی
&&
گر چه ما را بی‌سر و پا می‌کنی
\\
\end{longtable}
\end{center}
