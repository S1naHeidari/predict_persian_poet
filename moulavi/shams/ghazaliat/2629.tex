\begin{center}
\section*{غزل شماره ۲۶۲۹: ای ماه اگر باز بر این شکل بتابی}
\label{sec:2629}
\addcontentsline{toc}{section}{\nameref{sec:2629}}
\begin{longtable}{l p{0.5cm} r}
ای ماه اگر باز بر این شکل بتابی
&&
ما را و جهان را تو در این خانه نیابی
\\
چون کوه احد آب شد از شرم عقیقت
&&
چه نادره گر آب شود مردم آبی
\\
از عقل دو صدپر دو سه پر بیش نمانده‌ست
&&
و آن نیز بدان ماند که در زیر نقابی
\\
ای عشق دو عالم ز رخت مست و خرابند
&&
باری تو نگویی ز کی مست و خرابی
\\
تا باده نجوشید در آن خنب ز اول
&&
در جوش نیارد همه را او به شرابی
\\
تا اول با خود نخروشید ربابی
&&
در ناله نیارد همه را او به ربابی
\\
ای گرد جهان گشته و جز نقش ندیده
&&
بر روی زن آبی و یقین دان که بخوابی
\\
در خرمن ما آی اگر طالب کشتی
&&
سوی دل ما آی اگر مرد کبابی
\\
ور ز آنک نیایی بکشیمت به سوی خویش
&&
کز حلقه مایی نه غریبی نه غرابی
\\
مکتب نرود کودک لیکن ببرندش
&&
پنداشته‌ای خواجه که بیرون حسابی
\\
بستان قدح عشرت وز بند برون جه
&&
تا باخبری بند سؤالی و جوابی
\\
آخر بشنو هر نفسی نعره مستان
&&
کای گیج خرف گشته ببین در چه عذابی
\\
دست تو بگیرم دو سه روزی تو همی‌جوش
&&
تا بار دگر روی ز اقبال نتابی
\\
آن جا که شدی مست همان جای بخسبی
&&
و آن سوی که ساقی است همان سوی شتابی
\\
تا چند در آتش روی ای دل نه حدیدی
&&
وی دیده گرینده بس است این نه سحابی
\\
ای ساقی مه روی چه مست است دو چشمت
&&
انگشتک می زن که تو بر راه صوابی
\\
بگشای دهان ز آنچ نگفتم تو بیان کن
&&
بگشا در دل‌ها که تو سلطان خطابی
\\
\end{longtable}
\end{center}
