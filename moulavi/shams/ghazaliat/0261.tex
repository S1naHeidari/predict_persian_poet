\begin{center}
\section*{غزل شماره ۲۶۱: هان ای طبیب عاشقان سوداییی دیدی چو ما}
\label{sec:0261}
\addcontentsline{toc}{section}{\nameref{sec:0261}}
\begin{longtable}{l p{0.5cm} r}
هان ای طبیب عاشقان سوداییی دیدی چو ما
&&
یا صاحبی اننی مستهلک لو لاکما
\\
ای یوسف صد انجمن یعقوب دیدستی چو من
&&
اصفر خدی من جوی و ابیض عینی من بکا
\\
از چشم یعقوب صفی اشکی دوان بین یوسفی
&&
تجری دموعی بالولا من مقلتی عین الولا
\\
صد مصر و صد شکرستان درجست اندر یوسفان
&&
الصید جل او صغر فالکل فی جوف الفرا
\\
اسباب عشرت راست شد هر چه دلم می‌خواست شد
&&
فالوقت سیف قاطع لا تفتکر فیما مضی
\\
جان باز اندر عشق او چون سبط موسی را مگو
&&
اذهب و ربک قاتلا انا قعودها هنا
\\
هرگز نبینی در جهان مظلومتر زین عاشقان
&&
قولوا لاصحاب الحجی رفقا بارباب الهوی
\\
گر درد و فریادی بود در عاقبت دادی بود
&&
من فضل رب محسن عدل علی العرش استوی
\\
گر واقفی بر شرب ما وز ساقی شیرین لقا
&&
الزمه و اعلم ان ذا من غیره لا یرتجی
\\
کردیم جمله حیله‌ها ای حیله آموز نهی
&&
ماذا تری فیما تری یا من یری ما لا یری
\\
خاموش و باقی را بجو از ناطق اکرام خو
&&
فالفهم من ایحائه من کل مکروه شفا
\\
\end{longtable}
\end{center}
