\begin{center}
\section*{غزل شماره ۲۴۴۱: آخر مراعاتی بکن مر بی‌دلان را ساعتی}
\label{sec:2441}
\addcontentsline{toc}{section}{\nameref{sec:2441}}
\begin{longtable}{l p{0.5cm} r}
آخر مراعاتی بکن مر بی‌دلان را ساعتی
&&
ای ماه رو تشریف ده مر آسمان را ساعتی
\\
ای آن که هستت در سخن مستی می‌های کهن
&&
دلداریی تلقین بکن مر ترجمان را ساعتی
\\
تن چون کمانم دل چو زه ای جان کمان بر چرخ نه
&&
سوی فراز چرخ نه آن نردبان را ساعتی
\\
پیر از غمت هر جا فتی زان پیش کید آفتی
&&
بنما که بینم دولتی بس جاودان را ساعتی
\\
ای از کفت دریا نمی‌محروم کردی محرمی
&&
در خواب کن جانا دمی مر پاسبان را ساعتی
\\
عشقت می بی‌چون دهد در می همه افیون نهد
&&
مستت نشانی چون دهد آن بی‌نشان را ساعتی
\\
از رخ جهان پرنور کن چشم فلک مخمور کن
&&
از جان عالم دور کن این اندهان را ساعتی
\\
ای صد درج خوشتر ز جان وصف تو ناید در زبان
&&
الا که صوفی گوید آن پیش آر آن را ساعتی
\\
استغفرالله ای خرد صوفی بدو کی ره برد
&&
هر مرغ زان سو کی پرد درکش زبان را ساعتی
\\
ای کرده مه دراعه شق از عشقت ای خورشید حق
&&
از بهر لعلش ای شفق بگذار کان را ساعتی
\\
جز عشق او در دل مکن تدبیر بی‌حاصل مکن
&&
اندر مکان منزل مکن لا کن مکان را ساعتی
\\
ای امن‌ها در خوف تو ای ساکنی در طوف تو
&&
جان داده طمع سوف تو امن و امان را ساعتی
\\
بنگر در این فریاد کن آخر وفا هم یاد کن
&&
برتاب شاها داد کن این سو عنان را ساعتی
\\
یک دم بدین سو رای کن جان را تو شکرخای کن
&&
در دیده ما جای کن نور عیان را ساعتی
\\
تیرم چو قصد جه کنم پرم بده تا به کنم
&&
ابرو نما تا زه کنم من آن کمان را ساعتی
\\
ای زاغ هجران تهی چون زاغ از من کی رهی
&&
کی گوید آن نور شهی خواهم فلان را ساعتی
\\
ای نفس شیر شیررگ چون یافتی زان عشق تک
&&
انداز تو در پیش سگ این لوت و خوان را ساعتی
\\
ای از می جان بی‌خبر تا چند لافی از هنر
&&
افکن تو در قعر سقر آن دام نان را ساعتی
\\
کو شهریار این زمن مخدوم شمس الدین من
&&
تبریز خدمت کن به تن آن شه نشان را ساعتی
\\
\end{longtable}
\end{center}
