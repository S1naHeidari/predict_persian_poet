\begin{center}
\section*{غزل شماره ۲۸۷۴: سحری کرد ندایی عجب آن رشک پری}
\label{sec:2874}
\addcontentsline{toc}{section}{\nameref{sec:2874}}
\begin{longtable}{l p{0.5cm} r}
سحری کرد ندایی عجب آن رشک پری
&&
که گریزید ز خود در چمن بی‌خبری
\\
رو به دل کردم و گفتم که زهی مژده خوش
&&
که دهد خاک دژم را صفت جانوری
\\
همه ارواح مقدس چو تو را منتظرند
&&
تو چرا جان نشوی و سوی جانان نپری
\\
در مقامی که چنان ماه تو را جلوه کند
&&
کفر باشد که از این سو و از آن سو نگری
\\
گر تو چون پشه به هر باد پراکنده شوی
&&
پس نشاید که تو خود را ز همایان شمری
\\
بمترسان دل خود را تو به تهدید خسان
&&
که نشاید که خسان را به یکی خس بخری
\\
حیله می‌کرد دلم تا ز غمش سر ببرد
&&
گفتم ای ابله اگر سر ببری سر نبری
\\
شمس تبریز خیالت سوی من کژ نگریست
&&
رفتم از دست و بگفتم که چه شیرین نظری
\\
\end{longtable}
\end{center}
