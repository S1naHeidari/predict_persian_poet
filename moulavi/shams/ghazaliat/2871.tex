\begin{center}
\section*{غزل شماره ۲۸۷۱: به دغل کی بگزیند دل یارم یاری}
\label{sec:2871}
\addcontentsline{toc}{section}{\nameref{sec:2871}}
\begin{longtable}{l p{0.5cm} r}
به دغل کی بگزیند دل یارم یاری
&&
کی فریبد شه طرار مرا طراری
\\
کی میان من و آن یار بگنجد مویی
&&
کی در آن گلشن و گلزار بخسپد ماری
\\
عنکبوتی بتند پرده اغیار شود
&&
همچو صدیق و محمد من و او در غاری
\\
گل صدبرگ ز رشک رخ او جامه درید
&&
حال گل چونک چنین است چه باشد خاری
\\
هم بگویم دو سه بیتی که ندانی سر و پاش
&&
لیک بهر دل من ریش بجنبان کآری
\\
بس طبیب است که هشیار کند مجنون را
&&
وین طبیبم نهلد در دو جهان هشیاری
\\
آفتاب رخ او را حشم تیغ زنیم
&&
که نخواهیم به جز دیدن او ادراری
\\
ما چو خورشیدپرستیم بر این بام رویم
&&
تا نپوشد رخ خورشید ز ما دیواری
\\
کیست خورشید بگو شمس حق تبریزی
&&
که نگنجد صفتش در صحف گفتاری
\\
\end{longtable}
\end{center}
