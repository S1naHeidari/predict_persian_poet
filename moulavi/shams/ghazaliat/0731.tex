\begin{center}
\section*{غزل شماره ۷۳۱: پیش از آن کاندر جهان باغ و می و انگور بود}
\label{sec:0731}
\addcontentsline{toc}{section}{\nameref{sec:0731}}
\begin{longtable}{l p{0.5cm} r}
پیش از آن کاندر جهان باغ و می و انگور بود
&&
از شراب لایزالی جان ما مخمور بود
\\
ما به بغداد جهان جان اناالحق می‌زدیم
&&
پیش از آن کاین دار و گیر و نکته منصور بود
\\
پیش از آن کاین نفس کل در آب و گل معمار شد
&&
در خرابات حقایق عیش ما معمور بود
\\
جان ما همچون جهان بد جام جان چون آفتاب
&&
از شراب جان جهان تا گردن اندر نور بود
\\
ساقیا این معجبان آب و گل را مست کن
&&
تا بداند هر یکی کو از چه دولت دور بود
\\
جان فدای ساقیی کز راه جان در می‌رسد
&&
تا براندازد نقاب از هر چه آن مستور بود
\\
ما دهان‌ها باز مانده پیش آن ساقی کز او
&&
خمرهای بی‌خمار و شهد بی‌زنبور بود
\\
یا دهان ما بگیر ای ساقی ور نی فاش شد
&&
آنچ در هفتم زمین چون گنج‌ها گنجور بود
\\
شهر تبریز ار خبر داری بگو آن عهد را
&&
آن زمان کی شمس دین بی‌شمس دین مشهور بود
\\
\end{longtable}
\end{center}
