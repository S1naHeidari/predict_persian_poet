\begin{center}
\section*{غزل شماره ۵۳۷: کاری نداریم ای پدر جز خدمت ساقی خود}
\label{sec:0537}
\addcontentsline{toc}{section}{\nameref{sec:0537}}
\begin{longtable}{l p{0.5cm} r}
کاری نداریم ای پدر جز خدمت ساقی خود
&&
ای ساقی افزون ده قدح تا وارهیم از نیک و بد
\\
هر آدمی را در جهان آورد حق در پیشه‌ای
&&
در پیشه‌ای بی‌پیشگی کردست ما را نام زد
\\
هر روز همچون ذره‌ها رقصان به پیش آن ضیا
&&
هر شب مثال اختران طواف یار ماه خد
\\
کاری ز ما گر خواهدی زین باده ما را ندهدی
&&
اندر سری کاین می‌رود او کی فروشد یا خرد
\\
سرمست کاری کی کند مست آن کند که می‌کند
&&
باده خدایی طی کند هر دو جهان را تا صمد
\\
مستی باده این جهان چون شب بخسپی بگذرد
&&
مستی سغراق احد با تو درآید در لحد
\\
آمد شرابی رایگان زان رحمت ای همسایگان
&&
وان ساقیان چون دایگان شیرین و مشفق بر ولد
\\
ای دل از این سرمست شو هر جا روی سرمست رو
&&
تو دیگران را مست کن تا او تو را دیگر دهد
\\
هر جا که بینی شاهدی چون آینه پیشش نشین
&&
هر جا که بینی ناخوشی آیینه درکش در نمد
\\
می‌گرد گرد شهر خوش با شاهدان در کش مکش
&&
می‌خوان تو لااقسم نهان تا حبذا هذا البلد
\\
چون خیره شد زین می سرم خامش کنم خشک آورم
&&
لطف و کرم را نشمرم کان درنیاید در عدد
\\
\end{longtable}
\end{center}
