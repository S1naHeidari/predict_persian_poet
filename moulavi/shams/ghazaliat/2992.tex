\begin{center}
\section*{غزل شماره ۲۹۹۲: ای بس فراز و شیب که کردم طلب گری}
\label{sec:2992}
\addcontentsline{toc}{section}{\nameref{sec:2992}}
\begin{longtable}{l p{0.5cm} r}
ای بس فراز و شیب که کردم طلب گری
&&
گه لوح دل بخواندم و گه نقش کافری
\\
گه در زمین خدمت چون خاک ره شدم
&&
بر چرخ روح گاه دویدم باختری
\\
گم گشته از خود و دل و دلبر هزار بار
&&
گه سر دل بجسته و گه سر دلبری
\\
بر کوه طور طالب ارنی کلیم وار
&&
وز خلق دررمیده به عالم چو سامری
\\
در وادیی رسیدم کان جا نبرد بوی
&&
نی معجز و کرامت و نی مکر و ساحری
\\
وادی ز بوی دوست مرا رهبری شده
&&
کان بو نه مشک دارد نی زلف عنبری
\\
آن جا نتان دویدن ای دوست بر قدم
&&
پر نیز می‌بسوزد گر ز آنک می‌پری
\\
کز گرم و سرد و خشک و تر است این نهاد حس
&&
وین چار مرغ هست از این باغ عنصری
\\
آن جا بپر دوست که روید ز بوی دوست
&&
پری و گر نه زرد درافتی به شش دری
\\
ای کامل کمال کز این سو تو کاملی
&&
زان سو که سوی نیست حذر کن که قاصری
\\
آن مرغ خاکیی که به خشکی کمال داشت
&&
در بحر عاجز آمد و رسوا شد از تری
\\
با آنک بر و بحر یکی جنس و یک فنند
&&
هر یک به حس درآید چونشان درآوری
\\
صد بر و بحر و چرخ و فلک در فضای غیب
&&
در پا فتاده باشد چون نقش سرسری
\\
زین بر و بحر آن رسد آن سو که او ز عشق
&&
گردد هزار بار از این هر دو او بری
\\
حقا به ذات پاک خداوند هر کی هست
&&
از تیغ غیب سر نبرد گر برد سری
\\
در آتش خلیل کجا آید آن خسی
&&
کو خشک شد ز عشق دلارام آزری
\\
جان خلیل عشق به شادی و خرمی
&&
در آتش آ چو زر که ز هر غش طاهری
\\
گر محو می‌نمایی در دودمان حس
&&
در عشق آتشین دلارام ظاهری
\\
این عشق همچو آتش بر جمله قاهر است
&&
تو بس عجایبی که بر آتش تو قادری
\\
هر چند کوشد آتش تا تو سیه شوی
&&
بر رغم او لطیف و شریفی و احمری
\\
دانم که پرتو نظری داری از شهی
&&
چشم و چراغ غیب به شاهی و سروری
\\
بر خار خشک گر نظری افکند ز لطف
&&
پیدا شود ز خار دو صد گونه عبهری
\\
نی خود اگر به محو و عدم غمزه‌ای کند
&&
ظاهر شود ز نیست دل و دیده پروری
\\
در لطف و در نوازش آن شه نگاه کن
&&
ای تیغ هجر چند زنی زخم خنجری
\\
نی نی خود از نوازش او تند شد فراق
&&
کز یک نهاله آمد این لطف و قاهری
\\
گر خوگری به لطف نباشد دل مرا
&&
او کی فراق داند در دور دایری
\\
حنجر غذا خورد ز غذا رست حنجرش
&&
پس او غذا دهد به غذا رسم حنجری
\\
این جمله من بگفتم و القاب شمس دین
&&
از رشک کرده در غم تبریز ساتری
\\
آن است اصل و قصد و غرض زین همه حدیث
&&
لیکن مزاد نیست که من رام یشتری
\\
\end{longtable}
\end{center}
