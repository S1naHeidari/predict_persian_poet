\begin{center}
\section*{غزل شماره ۳۸۴: عاشقان را گر چه در باطن جهانی دیگرست}
\label{sec:0384}
\addcontentsline{toc}{section}{\nameref{sec:0384}}
\begin{longtable}{l p{0.5cm} r}
عاشقان را گر چه در باطن جهانی دیگرست
&&
عشق آن دلدار ما را ذوق و جانی دیگرست
\\
سینه‌های روشنان بس غیب‌ها دانند لیک
&&
سینه عشاق او را غیب دانی دیگرست
\\
بس زبان حکمت اندر شوق سرش گوش شد
&&
زانک مر اسرار او را ترجمانی دیگرست
\\
یک زمین نقره بین از لطف او در عین جان
&&
تا بدانی کان مهم را آسمانی دیگرست
\\
عقل و عشق و معرفت شد نردبان بام حق
&&
لیک حق را در حقیقت نردبانی دیگرست
\\
شب روان از شاه عقل و پاسبان آن سو شوند
&&
لیک آن جان را از آن سو پاسبانی دیگرست
\\
دلبران راه معنی با دلی عاجز بدند
&&
وحیشان آمد که دل را دلستانی دیگرست
\\
ای زبان‌ها برگشاده بر دل بربوده‌ای
&&
لب فروبندید کو را همزبانی دیگرست
\\
شمس تبریزی چو جمع و شمع‌ها پروانه‌اش
&&
زانک اندر عین دل او را عیانی دیگرست
\\
\end{longtable}
\end{center}
