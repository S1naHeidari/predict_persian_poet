\begin{center}
\section*{غزل شماره ۱۸۱۷: قصد جفاها نکنی ور بکنی با دل من}
\label{sec:1817}
\addcontentsline{toc}{section}{\nameref{sec:1817}}
\begin{longtable}{l p{0.5cm} r}
قصد جفاها نکنی ور بکنی با دل من
&&
وا دل من وا دل من وا دل من وا دل من
\\
قصد کنی بر تن من شاد شود دشمن من
&&
وانگه از این خسته شود یا دل تو یا دل من
\\
واله و شیدا دل من بی‌سر و بی‌پا دل من
&&
وقت سحرها دل من رفته به هر جا دل من
\\
بیخود و مجنون دل من خانه پرخون دل من
&&
ساکن و گردان دل من فوق ثریا دل من
\\
سوخته و لاغر تو در طلب گوهر تو
&&
آمده و خیمه زده بر لب دریا دل من
\\
گه چو کباب این دل من پر شده بویش به جهان
&&
گه چو رباب این دل من کرده علالا دل من
\\
زار و معاف است کنون غرق مصاف است کنون
&&
بر که قاف است کنون در پی عنقا دل من
\\
طفل دلم می نخورد شیر از این دایه شب
&&
سینه سیه یافت مگر دایه شب را دل من
\\
صخره موسی گر از او چشمه روان گشت چو جو
&&
جوی روان حکمت حق صخره و خارا دل من
\\
عیسی مریم به فلک رفت و فروماند خرش
&&
من به زمین ماندم و شد جانب بالا دل من
\\
بس کن کاین گفت زبان هست حجاب دل و جان
&&
کاش نبودی ز زبان واقف و دانا دل من
\\
\end{longtable}
\end{center}
