\begin{center}
\section*{غزل شماره ۸۹۲: آمد شهر صیام سنجق سلطان رسید}
\label{sec:0892}
\addcontentsline{toc}{section}{\nameref{sec:0892}}
\begin{longtable}{l p{0.5cm} r}
آمد شهر صیام سنجق سلطان رسید
&&
دست بدار از طعام مایده جان رسید
\\
جان ز قطیعت برست دست طبیعت ببست
&&
قلب ضلالت شکست لشکر ایمان رسید
\\
لشکر والعادیات دست به یغما نهاد
&&
ز آتش والموریات نفس به افغان رسید
\\
البقره راست بود موسی عمران نمود
&&
مرده از او زنده شد چونک به قربان رسید
\\
روزه چو قربان ماست زندگی جان ماست
&&
تن همه قربان کنیم جان چو به مهمان رسید
\\
صبر چو ابریست خوش حکمت بارد از او
&&
زانک چنین ماه صبر بود که قرآن رسید
\\
نفس چو محتاج شد روح به معراج شد
&&
چون در زندان شکست جان بر جانان رسید
\\
پرده ظلمت درید دل به فلک برپرید
&&
چون ز ملک بود دل باز بدیشان رسید
\\
زود از این چاه تن دست بزن در رسن
&&
بر سر چاه آب گو یوسف کنعان رسید
\\
عیسی چو از خر برست گشت دعایش قبول
&&
دست بشو کز فلک مایده و خوان رسید
\\
دست و دهان را بشو نه بخور و نه بگو
&&
آن سخن و لقمه جو کان به خموشان رسید
\\
\end{longtable}
\end{center}
