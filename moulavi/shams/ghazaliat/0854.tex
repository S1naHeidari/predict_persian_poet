\begin{center}
\section*{غزل شماره ۸۵۴: گفتم مکن چنین‌ها ای جان چنین نباشد}
\label{sec:0854}
\addcontentsline{toc}{section}{\nameref{sec:0854}}
\begin{longtable}{l p{0.5cm} r}
گفتم مکن چنین‌ها ای جان چنین نباشد
&&
غم قصد جان ما کرد گفتا خود این نباشد
\\
غم خود چه زهره دارد تا دست و پا برآرد
&&
چون خرده‌اش بسوزم گر خرده بین نباشد
\\
غم ترسد و هراسد ما را نکو شناسد
&&
صد دود از او برآرم گر آتشین نباشد
\\
غم خصم خویش داند هم حد خویش داند
&&
در خدمت مطیعان جز چون زمین نباشد
\\
چون تو از آن مایی در زهر اگر درآیی
&&
کی زهر زهره دارد تا انگبین نباشد
\\
در عین دود و آتش باشد خلیل را خوش
&&
آن را خدای داند هر کس امین نباشد
\\
هر کس که او امین شد با غیب همنشین شد
&&
هر جنس جنس خود را چون همنشین نباشد
\\
ای دست تو منور چون موسی پیمبر
&&
خواهم که دست موسی در آستین نباشد
\\
زیرا گل سعادت بی‌روی تو نروید
&&
ایاک نعبد ای جان بی‌نستعین نباشد
\\
\end{longtable}
\end{center}
