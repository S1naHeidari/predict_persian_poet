\begin{center}
\section*{غزل شماره ۴۴۷: ای گل تو را اگر چه رخسار نازکست}
\label{sec:0447}
\addcontentsline{toc}{section}{\nameref{sec:0447}}
\begin{longtable}{l p{0.5cm} r}
ای گل تو را اگر چه که رخسار نازکست
&&
رخ بر رخش مدار که آن یار نازکست
\\
در دل مدار نیز که رخ بر رخش نهی
&&
کو سر دل بداند و دلدار نازکست
\\
چون آرزو ز حد شد دزدیده سجده کن
&&
بسیار هم مکوش که بسیار نازکست
\\
گر بیخودی ز خویش همه وقت وقت تو است
&&
گر نی به وقت آی که اسرار نازکست
\\
دل را ز غم بروب که خانه خیال او است
&&
زیرا خیال آن بت عیار نازک است
\\
روزی بتافت سایه گل بر خیال دوست
&&
بر دوست کار کرد که این کار نازکست
\\
اندر خیال مفخر تبریز شمس دین
&&
منگر تو خوار کان شه خون خوار نازکست
\\
\end{longtable}
\end{center}
