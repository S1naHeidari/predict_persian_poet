\begin{center}
\section*{غزل شماره ۹۰۱: اگر دمی بنوازد مرا نگار چه باشد}
\label{sec:0901}
\addcontentsline{toc}{section}{\nameref{sec:0901}}
\begin{longtable}{l p{0.5cm} r}
اگر دمی بنوازد مرا نگار چه باشد
&&
گر این درخت بخندد از آن بهار چه باشد
\\
وگر به پیش من آید خیال یار که چونی
&&
حیات نو بپذیرد تن نزار چه باشد
\\
شکار خسته اویم به تیر غمزه جادو
&&
گرم به مهر بخواند که ای شکار چه باشد
\\
چو کاسه بر سر آبم ز بی‌قراری عشقش
&&
اگر رسم به لب دوست کوزه وار چه باشد
\\
کنار خاک ز اشکم چو لعل و گوهر پر شد
&&
اگر به وصل گشاید دمی کنار چه باشد
\\
بگفت چیست شکایت هزار بار گشادم
&&
ز بهر ماهی جان را هزار بار چه باشد
\\
من از قطار حریفان مهار عقل گسستم
&&
به پیش اشتر مستش یکی مهار چه باشد
\\
اگر مهار گسستم وگرچه بار فکندم
&&
یکی شتر کم گیری از این قطار چه باشد
\\
دلم به خشم نظر می‌کند که کوته کن هین
&&
اگر بجست یکی نکته از هزار چه باشد
\\
چو احمدست و ابوبکر یار غار دل و عشق
&&
دو نام بود و یکی جان دو یار غار چه باشد
\\
انار شیرین گر خود هزار باشد وگر یک
&&
چو شد یکی به فشردن دگر شمار چه باشد
\\
خمار و خمر یکستی ولی الف نگذارد
&&
الف چو شد ز میانه ببین خمار چه باشد
\\
چو شمس مفخر تبریز ماه نو بنماید
&&
در آن نمایش موزون ز کار و بار چه باشد
\\
\end{longtable}
\end{center}
