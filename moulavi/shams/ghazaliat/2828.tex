\begin{center}
\section*{غزل شماره ۲۸۲۸: خبری است نورسیده تو مگر خبر نداری}
\label{sec:2828}
\addcontentsline{toc}{section}{\nameref{sec:2828}}
\begin{longtable}{l p{0.5cm} r}
خبری است نورسیده تو مگر خبر نداری
&&
جگر حسود خون شد تو مگر جگر نداری
\\
قمری است رونموده پر نور برگشوده
&&
دل و چشم وام بستان ز کسی اگر نداری
\\
عجب از کمان پنهان شب و روز تیر پران
&&
بسپار جان به تیرش چه کنی سپر نداری
\\
مس هستیت چو موسی نه ز کیمیاش زر شد
&&
چه غم است اگر چو قارون به جوال زر نداری
\\
به درون توست مصری که تویی شکرستانش
&&
چه غم است اگر ز بیرون مدد شکر نداری
\\
شده ای غلام صورت به مثال بت پرستان
&&
تو چو یوسفی ولیکن به درون نظر نداری
\\
به خدا جمال خود را چو در آینه ببینی
&&
بت خویش هم تو باشی به کسی گذر نداری
\\
خردانه ظالمی تو که ورا چو ماه گویی
&&
ز چه روش ماه گویی تو مگر بصر نداری
\\
سر توست چون چراغی بگرفته شش فتیله
&&
همه شش ز چیست روشن اگر آن شرر نداری
\\
تن توست همچو اشتر که برد به کعبه دل
&&
ز خری به حج نرفتی نه از آنک خر نداری
\\
تو به کعبه گر نرفتی بکشاندت سعادت
&&
مگریز ای فضولی که ز حق عبر نداری
\\
\end{longtable}
\end{center}
