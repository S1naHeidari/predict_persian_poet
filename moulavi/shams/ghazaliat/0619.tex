\begin{center}
\section*{غزل شماره ۶۱۹: آن صبح سعادت‌ها چون نورفشان آید}
\label{sec:0619}
\addcontentsline{toc}{section}{\nameref{sec:0619}}
\begin{longtable}{l p{0.5cm} r}
آن صبح سعادت‌ها چون نورفشان آید
&&
آن گاه خروس جان در بانگ و فغان آید
\\
خور نور درخشاند پس نور برافشاند
&&
تن گرد چو بنشاند جانان بر جان آید
\\
مسکین دل آواره آن گمشده یک باره
&&
چون بشنود این چاره خوش رقص کنان آید
\\
جان به قدم رفته در کتم عدم رفته
&&
با قد به خم رفته در حین به میان آید
\\
دل مریم آبستن یک شیوه کند با من
&&
عیسی دوروزه تن درگفت زبان آید
\\
دل نور جهان باشد جان در لمعان باشد
&&
این رقص کنان باشد آن دست زنان آید
\\
شمس الحق تبریزی هر جا که کنی مقدم
&&
آن جا و مکان در دم بی‌جان و مکان باشد
\\
\end{longtable}
\end{center}
