\begin{center}
\section*{غزل شماره ۹۳۱: مها به دل نظری کن که دل تو را دارد}
\label{sec:0931}
\addcontentsline{toc}{section}{\nameref{sec:0931}}
\begin{longtable}{l p{0.5cm} r}
مها به دل نظری کن که دل تو را دارد
&&
به روز و شب به مراعاتت اقتضا دارد
\\
ز شادی و ز فرح در جهان نمی‌گنجد
&&
دلی که چون تو دلارام خوش لقا دارد
\\
ز آفتاب تو آن را که پشت گرم شود
&&
چرا دلیر نباشد حذر چرا دارد
\\
ز بهر شادی توست ار دلم غمی دارد
&&
ز دست و کیسه توست ار کفم سخا دارد
\\
خیال خوب تو چون وحشیان ز من برمد
&&
که صورتیست تن بنده دست و پا دارد
\\
مرا و صد چو مرا آن خیال بی‌صورت
&&
ز نقش سیر کند عاشق فنا دارد
\\
برهنه خلعت خورشید پوشد و گوید
&&
خنک کسی که ز زربفت او قبا دارد
\\
تنی که تابش خورشید جان بر او آید
&&
گمان مبر که سر سایه هما دارد
\\
بدانک موسی فرعون کش در این شهرست
&&
عصاش را تو نبینی ولی عصا دارد
\\
همی‌رسد به عنان‌های آسمان دستش
&&
که اصبع دل او خاتم وفا دارد
\\
غمش جفا نکند ور کند حلالش باد
&&
به هر چه آب کند تشنه صد رضا دارد
\\
فزون از آن نبود کش کشد به استسقا
&&
در آن زمان دل و جان عاشق سقا دارد
\\
اگر صبا شکند یک دو شاخ اندر باغ
&&
نه هر چه دارد آن باغ از صبا دارد
\\
شراب عشق چو خوردی شنو صلای کباب
&&
ز مقبلی که دلش داغ انبیا دارد
\\
زمین ببسته دهان تاسه مه که می‌داند
&&
که هر زمین به درون در نهان چه‌ها دارد
\\
بهار که بنماید زمین نیشکرت
&&
از آن زمین به درون ماش و لوبیا دارد
\\
چرا چو دال دعا در دعا نمی‌خمد
&&
کسی که از کرمش قبله دعا دارد
\\
چو پشت کرد به خورشید او نمازی نیست
&&
از آنک سایه خود پیش و مقتدا دارد
\\
خموش کن خبر من صمت نجا بشنو
&&
اگر رقیب سخن جوی ما روا دارد
\\
\end{longtable}
\end{center}
