\begin{center}
\section*{غزل شماره ۲۴۷۵: باز ترش شدی مگر یار دگر گزیده‌ای}
\label{sec:2475}
\addcontentsline{toc}{section}{\nameref{sec:2475}}
\begin{longtable}{l p{0.5cm} r}
باز ترش شدی مگر یار دگر گزیده‌ای
&&
دست جفا گشاده‌ای پای وفا کشیده‌ای
\\
دوش ز درد دل مها تا به سحر نخفته‌ام
&&
ز آنک تو مکر دشمنان در حق من شنیده‌ای
\\
ای دم آتشین من خیز تویی گواه دل
&&
ای شب دوش من بیا راست بگو چه دیده‌ای
\\
آینه‌ای خریده‌ای می‌نگری به روی خود
&&
در پس پرده رفته‌ای پرده من دریده‌ای
\\
عقل کجا که من کنون چاره کار خود کنم
&&
عقل برفت یاوه شد تا تو به من رسیده‌ای
\\
لعبت صورت مرا دوخته‌ای به جادوی
&&
سوزن‌های بوالعجب در دل من خلیده‌ای
\\
بر در و بام دل نگر جمله نشان پای توست
&&
بر در و بام مردمان دوش چرا دویده‌ای
\\
هر کی حدیث می‌کند بر لب او نظر کنم
&&
از هوس دهان تو تا لب کی گزیده‌ای
\\
تهمت دزد برنهم هر کی دهد نشان تو
&&
کاین ز کجا گرفته‌ای وین ز کجا خریده‌ای
\\
\end{longtable}
\end{center}
