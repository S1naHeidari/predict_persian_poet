\begin{center}
\section*{غزل شماره ۳۰۱۲: ای دل چون آهنت بوده چو آیینه‌ای}
\label{sec:3012}
\addcontentsline{toc}{section}{\nameref{sec:3012}}
\begin{longtable}{l p{0.5cm} r}
ای دل چون آهنت بوده چو آیینه‌ای
&&
آینه با جان من مونس دیرینه‌ای
\\
در دل آیینه من در دل من آینه
&&
تن کی بود محدثی دی و پریرینه‌ای
\\
خواجه چرایی چنین کز تو رمد عشق دین
&&
زانک همی‌بیندت احمد پارینه‌ای
\\
مرغ گزینی یقین دانه شیرین بچین
&&
کآمد از سوی چین مرغ تو را چینه‌ای
\\
شیر خدایی خدا شیر نرت نام داد
&&
از چه سبب گشته‌ای همدم بوزینه‌ای
\\
صورت تن را مبین زانک نه درخورد توست
&&
پوشد سلطان گهی خرقه پشمینه‌ای
\\
هین دل خود را تمام در کف دلبر سپار
&&
تا که نپوسد دلت در حسد و کینه‌ای
\\
سینه پاکی که او گشت خوش و عشق خو
&&
سینه سینا بود فرش چنین سینه‌ای
\\
تشنه آن شربتی خسته آن ضربتی
&&
تا تو در این غربتی نیست طمأنینه‌ای
\\
هست خرد چون شکر هست صور همچو نی
&&
هست معانی چو می حرف چو قنینه‌ای
\\
خوب چو نبود عروس خوش نشود زو نفوس
&&
از حفه و از رفه ز اطلس و زرینه‌ای
\\
چون نروی زین جهان خوی خرابات جان
&&
در عوض می بگیر بی‌مزه ترخینه‌ای
\\
خانه تن را بساز باغچه و گلشنی
&&
گوشه دل را بساز مسجد آدینه‌ای
\\
هر نفسی شاهدی در نظر واحدی
&&
آوردش بر طبق نادره لوزینه‌ای
\\
خامش با مرغ خاک قصه دریا مگو
&&
بکر چه عرضه کنی بر شه عنینه‌ای
\\
\end{longtable}
\end{center}
