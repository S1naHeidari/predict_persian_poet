\begin{center}
\section*{غزل شماره ۲۹۱۹: گر سران را بی‌سری درواستی}
\label{sec:2919}
\addcontentsline{toc}{section}{\nameref{sec:2919}}
\begin{longtable}{l p{0.5cm} r}
گر سران را بی‌سری درواستی
&&
سرنگونان را سری درواستی
\\
از برای شرح آتش‌های غم
&&
یا زبانی یا دلی برجاستی
\\
یا شعاعی زان رخ مهتاب او
&&
در شب تاریک غم با ماستی
\\
یا کسی دیگر برای همدمی
&&
هم از آن رو بی‌سر و بی‌پاستی
\\
گر اثر بودی از آن مه بر زمین
&&
ناله‌ها از آسمان برخاستی
\\
ور نه دست غیر تستی بر دهان
&&
راست و چپ بی‌این دهان غوغاستی
\\
گر از آن در پرتوی بر دل زدی
&&
یا به دریا یا خود او دریاستی
\\
ور نه غیرت خاک زد در چشم دل
&&
چشمه چشمه سوی دریاهاستی
\\
نیست پروای دو عالم عشق را
&&
ور نه ز الا هر دو عالم لاستی
\\
عشق را خود خاک باشی آرزو است
&&
ور نه عاشق بر سر جوزاستی
\\
تا چو برف این هر دو عالم در گداز
&&
ز آتش عشق جحیم آساستی
\\
اژدهای عشق خوردی جمله را
&&
گر عصا در پنجه موساستی
\\
لقمه‌ای کردی دو عالم را چنانک
&&
پیش جوع کلب نان یکتاستی
\\
پیش شمس الدین تبریز آمدی
&&
تا تجلی‌هاش مستوفاستی
\\
\end{longtable}
\end{center}
