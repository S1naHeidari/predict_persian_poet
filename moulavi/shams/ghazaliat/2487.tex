\begin{center}
\section*{غزل شماره ۲۴۸۷: هست به خطه عدم شور و غبار و غارتی}
\label{sec:2487}
\addcontentsline{toc}{section}{\nameref{sec:2487}}
\begin{longtable}{l p{0.5cm} r}
هست به خطه عدم شور و غبار و غارتی
&&
آتش عشق درزده تا نبود عمارتی
\\
ز آنک عمارت ار بود سایه کند وجود را
&&
سایه ز آفتاب او کی نگرد شرارتی
\\
روح که سایگی بود سرد و ملول و بی‌طرب
&&
منتظرک نشسته او تا که رسد بشارتی
\\
جان که در آفتاب شد هر گنهی که او کند
&&
برق زد از گناه او هر طرفی کفارتی
\\
شعله آفتاب را بر که و بر زمین است رنگ
&&
نیست بدید در هوا از لطف و طهارتی
\\
جان به مثال ذره‌ها رقص کنان در آفتاب
&&
نورپذیریش نگر لعل وش و مهارتی
\\
جان چو سنگ می‌دهد جان چو لعل می‌خرد
&&
رقص کنان ترانه زن گشته که خوش تجارتی
\\
قرص فلک درآید و روی به گوش جان‌ها
&&
سر ازل بگویدش بی‌سخن و عبارتی
\\
آنک به هر دمی نهان شعله زند به روح بر
&&
آن دل و زهره کو کز آن دم بزند اشارتی
\\
محرم حق شمس دین ای تبریز را تو شه
&&
کشته عشق خویش را شاه ازل زیارتی
\\
\end{longtable}
\end{center}
