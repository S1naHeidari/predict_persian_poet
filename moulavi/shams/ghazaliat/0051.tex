\begin{center}
\section*{غزل شماره ۵۱: گر تو ملولی ای پدر جانب یار من بیا}
\label{sec:0051}
\addcontentsline{toc}{section}{\nameref{sec:0051}}
\begin{longtable}{l p{0.5cm} r}
گر تو ملولی ای پدر جانب یار من بیا
&&
تا که بهار جان‌ها تازه کند دل تو را
\\
بوی سلام یار من لخلخه بهار من
&&
باغ و گل و ثمار من آرد سوی جان صبا
\\
مستی و طرفه مستیی هستی و طرفه هستیی
&&
ملک و درازدستیی نعره زنان که الصلا
\\
پای بکوب و دست زن دست در آن دو شست زن
&&
پیش دو نرگس خوشش کشته نگر دل مرا
\\
زنده به عشق سرکشم بینی جان چرا کشم
&&
پهلوی یار خود خوشم یاوه چرا روم چرا
\\
جان چو سوی وطن رود آب به جوی من رود
&&
تا سوی گولخن رود طبع خسیس ژاژخا
\\
دیدن خسرو زمن شعشعه عقار من
&&
سخت خوش است این وطن می‌نروم از این سرا
\\
جان طرب پرست ما عقل خراب مست ما
&&
ساغر جان به دست ما سخت خوش است ای خدا
\\
هوش برفت گو برو جایزه گو بشو گرو
&&
روز شدشت گو بشو بی‌شب و روز تو بیا
\\
مست رود نگار من در بر و در کنار من
&&
هیچ مگو که یار من باکرمست و باوفا
\\
آمد جان جان من کوری دشمنان من
&&
رونق گلستان من زینت روضه رضا
\\
\end{longtable}
\end{center}
