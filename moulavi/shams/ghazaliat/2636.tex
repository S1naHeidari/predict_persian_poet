\begin{center}
\section*{غزل شماره ۲۶۳۶: ای مونس ما خواجه ابوبکر ربابی}
\label{sec:2636}
\addcontentsline{toc}{section}{\nameref{sec:2636}}
\begin{longtable}{l p{0.5cm} r}
ای مونس ما خواجه ابوبکر ربابی
&&
گر دلشده‌ای چند پی نان و کبابی
\\
آتش خور در عشق به مانند شترمرغ
&&
اندر عقب طعمه چه شاگرد عقابی
\\
لقمه دهدت تا کند او لقمه خویشت
&&
این چرخ فریبنده و این برق سحابی
\\
هین لقمه مخور لقمه مشو آتش او را
&&
بی‌لقمه او در دل و جان رزق بیابی
\\
آن وقت که از ناف همی‌خورد تنت خون
&&
نی حلق و گلو بود و نه خرمای رطابی
\\
آن ماهی چه خورده‌ست که او لقمه ما شد
&&
در چشم نیاید خورش مردم آبی
\\
از نعمت پنهان خورد این نعمت پیدا
&&
زان راه شود فربه و زان ماه خضابی
\\
گر ز آنک خرابت کند این عشق برونی
&&
چون سنبله شد دانه در این روز خرابی
\\
آن سنبله از خاک برآورد سر و گفت
&&
من مردم و زنده شدم از داد ثوابی
\\
خواهی که قیامت نگری نقد به باغ آی
&&
نظاره سرسبزی اموات ترابی
\\
ماییم که پوسیده و ریزیده خاکیم
&&
امروز چو سرویم سرافراز و خطابی
\\
بی‌حرف سخن گوی که تا خصم نگوید
&&
کاین گفت کسان است و سخن‌های کتابی
\\
\end{longtable}
\end{center}
