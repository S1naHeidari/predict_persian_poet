\begin{center}
\section*{غزل شماره ۲۹۴۴: ای مبدعی که سگ را بر شیر می‌فزایی}
\label{sec:2944}
\addcontentsline{toc}{section}{\nameref{sec:2944}}
\begin{longtable}{l p{0.5cm} r}
ای مبدعی که سگ را بر شیر می‌فزایی
&&
سنگ سیه بگیری آموزیش سقایی
\\
بس شاه و بس فریدون کز تیغشان چکد خون
&&
زان روی همچو لاله لولی است و لالکایی
\\
ناموسیان سرکش جبارتر ز آتش
&&
در کوی عشق گردان امروز در گدایی
\\
قهر است کار آتش گریه‌ست پیشه شمع
&&
از ما وفا و خدمت وز یار بی‌وفایی
\\
آتش که او نخندد خاکستر است و دودی
&&
شمعی که او نگرید چوبی بود عصایی
\\
آن خر بود که آید در بوستان دنیا
&&
خاونده را نجوید افتد به ژاژخایی
\\
خاوند بوستان را اول بجوی ای خر
&&
تا از خری رهی تو زان لطف و کبریایی
\\
آمد غریبی از ره مهمان مهتری شد
&&
مهمانیی بکردش باکار و باکیایی
\\
بریانه‌های فاخر سنبوسه‌های نادر
&&
شمع و شراب و شاهد بس خلعت عطایی
\\
ماهیش کرد مهمان هر روز به ز روزی
&&
چون حسن دلبر ما در دلبری فزایی
\\
هر شب غریب گفتی نیکو است این ولیکن
&&
مهمانیت نمایم چون شهر ما بیایی
\\
آن مهتر از تحیر گفت ای عجب چه باشد
&&
بهتر از این تنعم وین خلعت بهایی
\\
زین گفت حاج کوله شد در دلش گلوله
&&
زیرا ندیده بود او مهمانیی سمایی
\\
این میوه‌های دنیا گل پاره‌هاست رنگین
&&
چه بود نعیم دنیا جز نان و نان ربایی
\\
می‌گفت ای خدایا ما را به شهر او بر
&&
تا حاصل آید آن جا دل را گره گشایی
\\
بگذشت چند سالی در انتظار این دم
&&
بی انتظار ندهد هرگز دوا دوایی
\\
می‌گفت ای مسبب برساز یک بهانه
&&
زیرا سبب تو سازی در دام ابتلایی
\\
بسیار شد دعایش آمد ز حق اجابت
&&
تا مرد ای خدا گو دید از خدا خدایی
\\
شه جست یک رسولی تا آن طرف فرستد
&&
تا آن طرف رساند پیغام کدخدایی
\\
این میرداد رشوت پنهان و آشکارا
&&
تا میر را فرستد شاه از کرم نمایی
\\
شه هم قبول کردش گفتا تو بر بدان جا
&&
پیغام ما ازیرا طوطی خوش نوایی
\\
پس ساز کرد ره را همراه شد سپه را
&&
در پیش کرد مه را از بهر روشنایی
\\
منزل به منزل آن سو می‌شد چو سیل در جو
&&
سجده کنان و جویان اسرار اولیایی
\\
چون موسی پیمبر از بهر خضر انور
&&
کرده سفر به صد پر چون هدهد هوایی
\\
چون پر جبرئیلی کو پیک عرش آمد
&&
تا زان سفر دهد او احکام را روایی
\\
مه کو منور آمد دایم مسافر آمد
&&
ای ماه رو سفر کن چون شمع این سرایی
\\
هر حالتت چو برجی در وی دری و درجی
&&
غم آتشی و برقی شادی تو ضیایی
\\
کوته کنم بیان را رفت آن رسول آن جا
&&
چون برگ که کشیدش دلبر به کهربایی
\\
ما چون قطار پویان دست کشنده پنهان
&&
دستی نهان که نبود کس را از او رهایی
\\
این را به چپ کشاند و آن را به راست آرد
&&
این را به وصل آرد و آن را سوی جدایی
\\
وصلش نماید آن سو تا مست و گرم گردد
&&
و آن سوی هجر باشد مکری است این دغایی
\\
دررفت آن معلا در شهر همچو دریا
&&
از کو به کو همی‌شد کای مقصدم کجایی
\\
جوینده چون شتابد مطلوب را بیابد
&&
ما آگهیم که تو در جست و جوی مایی
\\
شد ناگهان به کویی سرمست شد ز بویی
&&
عقلش پرید از سر پا را نماند پایی
\\
پیغام کیقبادش جمله بشد ز یادش
&&
کو دانش رسولی تا محفل اندرآیی
\\
چل روز بر سر کو سرمست ماند از آن بو
&&
حیران شده رعیت با میرهای‌هایی
\\
نی حکم و نی امارت نی غسل و نی طهارت
&&
نی گفت و نی اشارت نی میل اغتذایی
\\
زو هر کی جست کاری می‌گفت خیره آری
&&
آری و نی یکی دان در وقت خیره رایی
\\
کو خیمه و طویله کو کار و حال و حیله
&&
کو دمنه و کلیله کو کد کدخدایی
\\
سیلاب عشق آمد نی دام ماند نی دد
&&
چون سیل شد به بحری بی‌بدو و منتهایی
\\
گفت ای رفیق جفتی کردی هر آنچ گفتی
&&
بردی مرا از اسفل تا مصعد علایی
\\
این درس که شنودم هرگز نخوانده بودم
&&
درسی است نی وسیطی نی نیز منتقایی
\\
دعویت به ز معنی معنیت به ز دعوی
&&
جان روی در تو دارد که قبله دعایی
\\
این جمله بد بدایت کو باقی حکایت
&&
واپرس از او که دادت در گوش اشنوایی
\\
یا رب ظلمت نفسی بردر حجاب حسی
&&
گر مس نمود مسی آخر تو کیمیایی
\\
صدر الرجال حقا فی مصدر البلا
&&
والله ما علونا الا باعتنا
\\
یا سادتی و قومی یوفون بالعهود
&&
ما خاب من تحلی بالصدق و الوفا
\\
\end{longtable}
\end{center}
