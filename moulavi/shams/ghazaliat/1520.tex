\begin{center}
\section*{غزل شماره ۱۵۲۰: مرا پرسی که چونی بین که چونم}
\label{sec:1520}
\addcontentsline{toc}{section}{\nameref{sec:1520}}
\begin{longtable}{l p{0.5cm} r}
مرا پرسی که چونی بین که چونم
&&
خرابم بیخودم مست جنونم
\\
مرا از کاف و نون آورد در دام
&&
از آن هیبت دوتا چون کاف و نونم
\\
پری زاده مرا دیوانه کرده‌ست
&&
مسلمانان که می داند فسونم
\\
پری را چهره‌ای چون ارغوان است
&&
بنالم کارغوان را ارغنونم
\\
مگر من خانه ماهم چو گردون
&&
که چون گردون ز عشقش بی‌سکونم
\\
غلط گفتم مزاج عشق دارم
&&
ز دوران و سکونت‌ها برونم
\\
درون خرقه صدرنگ قالب
&&
خیال بادشکل آبگونم
\\
چه جای باد و آب است ای برادر
&&
که همچون عقل کلی ذوفنونم
\\
ولیک آنگه که جزو آید به کلش
&&
بخیزد تل مشک از موج خونم
\\
چه داند جزو راه کل خود را
&&
مگر هم کل فرستد رهنمونم
\\
بکش ای عشق کلی جزو خود را
&&
که این جا در کشاکش‌ها زبونم
\\
ز هجرت می کشم بار جهانی
&&
که گویی من جهانی را ستونم
\\
به صورت کمترم از نیم ذره
&&
ز روی عشق از عالم فزونم
\\
یکی قطره که هم قطره‌ست و دریا
&&
من این اشکال‌ها را آزمونم
\\
نمی‌گویم من این این گفت عشق است
&&
در این نکته من از لایعلمونم
\\
که این قصه هزاران سالگان است
&&
چه دانم من که من طفل از کنونم
\\
ولی طفلم طفیل آن قدیم است
&&
که می دارد قرانش در قرونم
\\
سخن مقلوب می گویم که کرده‌ست
&&
جهان بازگونه بازگونم
\\
سخن آنگه شنو از من که بجهد
&&
از این گرداب‌ها جان حرونم
\\
حدیث آب و گل جمله شجون است
&&
چه یک رنگی کنم چون در شجونم
\\
غلط گفتم که یک رنگم چو خورشید
&&
ولی در ابر این دنیای دونم
\\
خمش کن خاک آدم را مشوران
&&
که این جا چون پری من در کمونم
\\
\end{longtable}
\end{center}
