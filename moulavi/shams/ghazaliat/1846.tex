\begin{center}
\section*{غزل شماره ۱۸۴۶: حرام است ای مسلمانان از این خانه برون رفتن}
\label{sec:1846}
\addcontentsline{toc}{section}{\nameref{sec:1846}}
\begin{longtable}{l p{0.5cm} r}
حرام است ای مسلمانان از این خانه برون رفتن
&&
می چون ارغوان هشتن ز بانگ ارغنون رفتن
\\
برون زرق است یا استم هزاران بار دیدستم
&&
از این پس ابلهی باشد برای آزمون رفتن
\\
مرو زین خانه ای مجنون که خون گریی ز هجران خون
&&
چو دستی را فروبری عجایب نیست خون رفتن
\\
ز شمع آموز ای خواجه میان گریه خندیدن
&&
ز چشم آموز ای زیرک به هنگام سکون رفتن
\\
اگر باشد تو را روزی ز استادان بیاموزی
&&
چو مرغ جان معصومان به چرخ نیلگون رفتن
\\
بیا ای جان که وقتت خوش چو استن بار ما می کش
&&
که تا صبرت بیاموزد به سقف بی‌ستون رفتن
\\
فسون عیسی مریم نکرد از درد عاشق کم
&&
وظیفه درد دل نبود به دارو و فسون رفتن
\\
چو طاسی سرنگون گردد رود آنچ در او باشد
&&
ولی سودا نمی‌تاند ز کاسه سر نگون رفتن
\\
اگر پاکی و ناپاکی مرو زین خانه‌ای زاکی
&&
گناهی نیست در عالم تو را ای بنده چون رفتن
\\
تویی شیر اندر این درگه عدو راه تو روبه
&&
بود بر شیر بدنامی از این چالش زبون رفتن
\\
چو نازی می کشی باری بیا ناز چنین شه کش
&&
که بس بداختری باشد به زیر چرخ دون رفتن
\\
ز دانش‌ها بشویم دل ز خود خود را کنم غافل
&&
که سوی دلبر مقبل نشاید ذوفنون رفتن
\\
شناسد جان مجنونان که این جان است قشر جان
&&
بباید بهر این دانش ز دانش در جنون رفتن
\\
کسی کو دم زند بی‌دم مباح او راست غواصی
&&
کسی کو کم زند در کم رسد او را فزون رفتن
\\
رها کن تا بگوید او خموشی گیر و توبه جو
&&
که آن دلدار خو دارد به سوی تایبون رفتن
\\
\end{longtable}
\end{center}
