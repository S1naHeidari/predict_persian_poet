\begin{center}
\section*{غزل شماره ۳۴۲: مرا چون تا قیامت یار اینست}
\label{sec:0342}
\addcontentsline{toc}{section}{\nameref{sec:0342}}
\begin{longtable}{l p{0.5cm} r}
مرا چون تا قیامت یار اینست
&&
خراب و مست باشم کار اینست
\\
ز کار و کسب ماندم کسبم اینست
&&
رخا زر زن تو را دینار اینست
\\
نه عقلی ماند و نی تمیز و نی دل
&&
چه چاره فعل آن دیدار اینست
\\
گل صدبرگ دید آن روی خوبش
&&
به بلبل گفت گل گلزار اینست
\\
چو خوبان سایه‌های طیر غیبند
&&
به سوی غیب آ طیار این‌ست
\\
مکرر بنگر آن سو چشم می‌مال
&&
که جان را مدرسه و تکرار اینست
\\
چو لب بگشاد جان‌ها جمله گفتند
&&
شفای جان هر بیمار اینست
\\
چو یک ساغر ز دست عشق خوردند
&&
یقینشان شد که خود خمار اینست
\\
گرو کردی به می دستار و جبه
&&
سزای جبه و دستار اینست
\\
خبر آمد که یوسف شد به بازار
&&
هلا کو یوسف ار بازار اینست
\\
فسونی خواند و پنهان کرد خود را
&&
کمینه لعب آن طرار اینست
\\
ز ملک و مال عالم چاره دارم
&&
مرا دین و دل و ناچار اینست
\\
میان گر پیش غیر عشق بندم
&&
مسیحی باشم و زنار اینست
\\
به گرد حوض گشتم درفتادم
&&
جزای آن چنان کردار اینست
\\
دلا چون درفتادی در چنین حوض
&&
تو را غسل قیامت وار اینست
\\
رخ شه جسته‌ای شهمات اینست
&&
چو دزدی کردی ای دل دار اینست
\\
مشین با خود نشین با هر که خواهی
&&
ز نفس خود ببر اغیار اینست
\\
خمش کن خواجه لاغ پار کم گو
&&
دلم پاره‌ست و لاغ پار اینست
\\
خمش باش و در این حیرت فرورو
&&
بهل اسرار را کاسرار اینست
\\
\end{longtable}
\end{center}
