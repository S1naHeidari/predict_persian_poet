\begin{center}
\section*{غزل شماره ۱۳۱۷: آن میر دروغین بین با اسپک و با زینک}
\label{sec:1317}
\addcontentsline{toc}{section}{\nameref{sec:1317}}
\begin{longtable}{l p{0.5cm} r}
آن میر دروغین بین با اسپک و با زینک
&&
شنگینک و منگینک سربسته به زرینک
\\
چون منکر مرگست او گوید که اجل کو کو
&&
مرگ آیدش از شش سو گوید که منم اینک
\\
گوید اجلش کای خر کو آن همه کر و فر
&&
وان سبلت و آن بینی وان کبرک و آن کینک
\\
کو شاهد و کو شادی مفرش به کیان دادی
&&
خشتست تو را بالین خاکست نهالینک
\\
ترک خور و خفتن گو رو دین حقیقی جو
&&
تا میر ابد باشی بی‌رسمک و آیینک
\\
بی‌جان مکن این جان را سرگین مکن این نان را
&&
ای آنک فکندی تو در در تک سرگینک
\\
ما بسته سرگین دان از بهر دریم ای جان
&&
بشکسته شو و در جو ای سرکش خودبینک
\\
چون مرد خدابینی مردی کن و خدمت کن
&&
چون رنج و بلا بینی در رخ مفکن چینک
\\
این هجو منست ای تن وان میر منم هم من
&&
تا چند سخن گفتن از سینک و از شینک
\\
شمس الحق تبریزی خود آب حیاتی تو
&&
وان آب کجا یابد جز دیده نمگینک
\\
\end{longtable}
\end{center}
