\begin{center}
\section*{غزل شماره ۲۳۷۸: صد خمار است و طرب در نظر آن دیده}
\label{sec:2378}
\addcontentsline{toc}{section}{\nameref{sec:2378}}
\begin{longtable}{l p{0.5cm} r}
صد خمار است و طرب در نظر آن دیده
&&
که در آن روی نظر کرده بود دزدیده
\\
صد نشاط است و هوس در سر آن سرمستی
&&
که رخ خود به کف پاش بود مالیده
\\
عشوه و مکر زمانه نپذیرد گوشی
&&
که سلام از لب آن یار بود بشنیده
\\
پیچ زلفش چو ندیدی تو برو معذوری
&&
ای تو در نیک و بد دور زمان پیچیده
\\
نی تراشی است که اندر نی صورت بدمد
&&
هیچ دیدی تو نیی بی‌نفسی نالیده
\\
گر بداند که حریف لب کی خواهد شد
&&
کی برنجد ز بریدن قلم بالیده
\\
گر بپرسند چه فرق است میان تو و غیر
&&
فرق این بس که تویی فرق مرا خاریده
\\
جرعه‌ای کن فیکون بر سر آن خاک بریخت
&&
لب عشاق جهان خاک تو را لیسیده
\\
شمس تبریز تو را عشق شناسد نه خرد
&&
بر دم باد بهاری نرسد پوسیده
\\
\end{longtable}
\end{center}
