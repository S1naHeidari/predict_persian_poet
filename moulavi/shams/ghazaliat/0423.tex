\begin{center}
\section*{غزل شماره ۴۲۳: مگر این دم سر آن زلف پریشان شده است}
\label{sec:0423}
\addcontentsline{toc}{section}{\nameref{sec:0423}}
\begin{longtable}{l p{0.5cm} r}
مگر این دم سر آن زلف پریشان شده است
&&
که چنین مشک تتاری عبرافشان شده است
\\
مگر از چهره او باد صبا پرده ربود
&&
که هزاران قمر غیب درخشان شده است
\\
هست جانی که ز بوی خوش او شادان نیست
&&
گر چه جان بو نبرد کو ز چه شادان شده است
\\
ای بسا شاد گلی کز دم حق خندان است
&&
لیک هر جان بنداند ز چه خندان شده است
\\
آفتاب رخش امروز زهی خوش که بتافت
&&
که هزاران دل از او لعل بدخشان شده است
\\
عاشق آخر ز چه رو تا به ابد دل ننهد
&&
بر کسی کز لطفش تن همگی جان شده است
\\
مگرش دل سحری دید بدان سان که وی است
&&
که از آن دیدنش امروز بدین سان شده است
\\
تا بدیده است دل آن حسن پری زاد مرا
&&
شیشه بر دست گرفته است و پری خوان شده است
\\
بر درخت تن اگر باد خوشش می‌نوزد
&&
پس دو صد برگ دو صد شاخ چه لرزان شده است
\\
بهر هر کشته او جان ابد گر نبود
&&
جان سپردن بر عاشق ز چه آسان شده است
\\
از حیات و خبرش باخبران بی‌خبرند
&&
که حیات و خبرش پرده ایشان شده است
\\
گر نه در نای دلی مطرب عشقش بدمید
&&
هر سر موی چو سرنای چه نالان شده است
\\
شمس تبریز ز بام ار نه کلوخ اندازد
&&
سوی دل پس ز چه جان‌هاش چو دربان شده است
\\
\end{longtable}
\end{center}
