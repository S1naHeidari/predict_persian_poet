\begin{center}
\section*{غزل شماره ۳: ای دل چه اندیشیده‌ای در عذر آن تقصیرها}
\label{sec:0003}
\addcontentsline{toc}{section}{\nameref{sec:0003}}
\begin{longtable}{l p{0.5cm} r}
ای دل چه اندیشیده‌ای در عذر آن تقصیرها
&&
زان سوی او چندان وفا زین سوی تو چندین جفا
\\
زان سوی او چندان کرم زین سو خلاف و بیش و کم
&&
زان سوی او چندان نعم زین سوی تو چندین خطا
\\
زین سوی تو چندین حسد چندین خیال و ظن بد
&&
زان سوی او چندان کشش چندان چشش چندان عطا
\\
چندین چشش از بهر چه تا جان تلخت خوش شود
&&
چندین کشش از بهر چه تا دررسی در اولیا
\\
از بد پشیمان می‌شوی الله گویان می‌شوی
&&
آن دم تو را او می‌کشد تا وارهاند مر تو را
\\
از جرم ترسان می‌شوی وز چاره پرسان می‌شوی
&&
آن لحظه ترساننده را با خود نمی‌بینی چرا
\\
گر چشم تو بربست او چون مهره‌ای در دست او
&&
گاهی بغلطاند چنین گاهی ببازد در هوا
\\
گاهی نهد در طبع تو سودای سیم و زر و زن
&&
گاهی نهد در جان تو نور خیال مصطفی
\\
این سو کشان سوی خوشان وان سو کشان با ناخوشان
&&
یا بگذرد یا بشکند کشتی در این گرداب‌ها
\\
چندان دعا کن در نهان چندان بنال اندر شبان
&&
کز گنبد هفت آسمان در گوش تو آید صدا
\\
بانک شعیب و ناله‌اش وان اشک همچون ژاله‌اش
&&
چون شد ز حد از آسمان آمد سحرگاهش ندا
\\
گر مجرمی بخشیدمت وز جرم آمرزیدمت
&&
فردوس خواهی دادمت خامش رها کن این دعا
\\
گفتا نه این خواهم نه آن دیدار حق خواهم عیان
&&
گر هفت بحر آتش شود من درروم بهر لقا
\\
گر رانده آن منظرم بستست از او چشم ترم
&&
من در جحیم اولیترم جنت نشاید مر مرا
\\
جنت مرا بی‌روی او هم دوزخست و هم عدو
&&
من سوختم زین رنگ و بو کو فر انوار بقا
\\
گفتند باری کم گری تا کم نگردد مبصری
&&
که چشم نابینا شود چون بگذرد از حد بکا
\\
گفت ار دو چشمم عاقبت خواهند دیدن آن صفت
&&
هر جزو من چشمی شود کی غم خورم من از عمی
\\
ور عاقبت این چشم من محروم خواهد ماندن
&&
تا کور گردد آن بصر کو نیست لایق دوست را
\\
اندر جهان هر آدمی باشد فدای یار خود
&&
یار یکی انبان خون یار یکی شمس ضیا
\\
چون هر کسی درخورد خود یاری گزید از نیک و بد
&&
ما را دریغ آید که خود فانی کنیم از بهر لا
\\
روزی یکی همراه شد با بایزید اندر رهی
&&
پس بایزیدش گفت چه پیشه گزیدی ای دغا
\\
گفتا که من خربنده‌ام پس بایزیدش گفت رو
&&
یا رب خرش را مرگ ده تا او شود بنده خدا
\\
\end{longtable}
\end{center}
