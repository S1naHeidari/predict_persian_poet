\begin{center}
\section*{غزل شماره ۱۴۷۴: حکیمیم طبیبیم ز بغداد رسیدیم}
\label{sec:1474}
\addcontentsline{toc}{section}{\nameref{sec:1474}}
\begin{longtable}{l p{0.5cm} r}
حکیمیم طبیبیم ز بغداد رسیدیم
&&
بسی علتیان را ز غم بازخریدیم
\\
سبل‌های کهن را غم بی‌سر و بن را
&&
ز رگ هاش و پی‌هاش به چنگاله کشیدیم
\\
طبیبان فصیحیم که شاگرد مسیحیم
&&
بسی مرده گرفتیم در او روح دمیدیم
\\
بپرسید از آن‌ها که دیدند نشان‌ها
&&
که تا شکر بگویند که ما از چه رهیدیم
\\
رسیدند طبیبان ز ره دور غریبان
&&
غریبانه نمودند دواها که ندیدیم
\\
سر غصه بکوبیم غم از خانه بروبیم
&&
همه شاهد و خوبیم همه چون مه عیدیم
\\
طبیبان الهیم ز کس مزد نخواهیم
&&
که ما پاک روانیم نه طماع و پلیدیم
\\
مپندار که این نیز هلیله‌ست و بلیله‌ست
&&
که این شهره عقاقیر ز فردوس کشیدیم
\\
حکیمان خبیریم که قاروره نگیریم
&&
که ما در تن رنجور چو اندیشه دویدیم
\\
دهان باز مکن هیچ که اغلب همه جغدند
&&
دگر لاف مپران که ما بازپریدیم
\\
\end{longtable}
\end{center}
