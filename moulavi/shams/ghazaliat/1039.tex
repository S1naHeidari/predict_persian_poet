\begin{center}
\section*{غزل شماره ۱۰۳۹: بگرد فتنه می‌گردی دگربار}
\label{sec:1039}
\addcontentsline{toc}{section}{\nameref{sec:1039}}
\begin{longtable}{l p{0.5cm} r}
بگرد فتنه می‌گردی دگربار
&&
لب بامست و مستی هوش می‌دار
\\
کجا گردم دگر کو جای دیگر
&&
که ما فی الدار غیر الله دیار
\\
نگردد نقش جز بر کلک نقاش
&&
بگرد نقطه گردد پای پرگار
\\
چو تو باشی دل و جان کم نیاید
&&
چو سر باشد بیاید نیز دستار
\\
گرفتارست دل در قبضه حق
&&
گرفته صعوه را بازی به منقار
\\
ز منقارش فلک سوراخ سوراخ
&&
ز چنگالش گران جانان سبکبار
\\
رها کن این سخن‌ها را ندا کن
&&
به مخموران که آمد شاه خمار
\\
غم و اندیشه را گردن بریدند
&&
که آمد دور وصل و لطف و ایثار
\\
هلا ای ساربان اشتر بخوابان
&&
از این خوشتر کجا باشد علف زار
\\
چو مهمانان بدین دولت رسیدند
&&
بیا ای خازن و بگشای انبار
\\
شب مشتاق را روزی نیاید
&&
چنین پنداشتی دیگر مپندار
\\
خمش کن تا خموش ما بگوید
&&
ویست اصل سخن سلطان گفتار
\\
\end{longtable}
\end{center}
