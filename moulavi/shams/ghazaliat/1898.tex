\begin{center}
\section*{غزل شماره ۱۸۹۸: ندا آمد به جان از چرخ پروین}
\label{sec:1898}
\addcontentsline{toc}{section}{\nameref{sec:1898}}
\begin{longtable}{l p{0.5cm} r}
ندا آمد به جان از چرخ پروین
&&
که بالا رو چو دردی پست منشین
\\
کسی اندر سفر چندین نماند
&&
جدا از شهر و از یاران پیشین
\\
ندای ارجعی آخر شنیدی
&&
از آن سلطان و شاهنشاه شیرین
\\
در این ویرانه جغدانند ساکن
&&
چه مسکن ساختی ای باز مسکین
\\
چه آساید به هر پهلو که گردد
&&
کسی کز خار سازد او نهالین
\\
چه پیوندی کند صراف و قلاب
&&
چه نسبت زاغ را با باز و شاهین
\\
چه آرایی به گچ ویرانه‌ای را
&&
که بالا نقش دارد زیر سجین
\\
چرا جان را نیارایی به حکمت
&&
که ارزد هر دمش صد چین و ماچین
\\
نه آن حکمت که مایه گفت و گوی است
&&
از آن حکمت که گردد جان خدابین
\\
تو گوهر شو که خواهند و نخواهند
&&
نشانندت همه بر تاج زرین
\\
رها کن پس روی چون پای کژمژ
&&
الف می باش فرد و راست بنشین
\\
چو معنی اسب آمد حرف چون زین
&&
بگو تا کی کشی بی‌اسب این زین
\\
کلوخ انداز کن در عشق مردان
&&
تو هم مردی ولی مرد کلوخین
\\
عروسی کلوخی با کلوخی
&&
کلوخ آرد نثار و سنگ کابین
\\
به گورستان به زیر خشت بنگر
&&
که نشناسی تو سارانشان ز پایین
\\
خدایا دررسان جان را به جان‌ها
&&
بدان راهی که رفتند آل یاسین
\\
دعای ما و ایشان را درآمیز
&&
چنان کز ما دعای و از تو آمین
\\
عنایت آن چنان فرما که باشد
&&
ز ما احسان اندک وز تو تحسین
\\
ز شهوانی به عقلانی رسانمان
&&
بر اوج فوق بر زین لوح زیرین
\\
\end{longtable}
\end{center}
