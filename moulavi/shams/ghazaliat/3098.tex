\begin{center}
\section*{غزل شماره ۳۰۹۸: ترش ترش بنشستی بهانه دربستی}
\label{sec:3098}
\addcontentsline{toc}{section}{\nameref{sec:3098}}
\begin{longtable}{l p{0.5cm} r}
ترش ترش بنشستی بهانه دربستی
&&
که ندهم آبت زیرا که کوزه بشکستی
\\
هزار کوزه زرین به جای آن بدهم
&&
مگیر سخت مرا ز آنچ رفت در مستی
\\
تو را که آب حیاتی چه کم شود کوزه
&&
چه حاجت آید جان و جهان چو تو هستی
\\
بیا که روز عزیزست مجلسی برساز
&&
ولی چو دوش مکن کز میان برون جستی
\\
پریر رفتم سرمست تو به خانه عشق
&&
به خنده گفت بیا کز زحیر وارستی
\\
هزار جان بفزودی اگر دلی بردی
&&
هزار مرهم دادی اگر تنی خستی
\\
چرا نگیرم پایت که تاج سرهایی
&&
چرا نبوسم دستت که صاحب دستی
\\
دلا میی بستان کز خمارها برهی
&&
چنین بتی بپرست ای صنم چو بپرستی
\\
برو دلا به سعادت به سوی عالم دل
&&
به شکر آنک به اقبال و بخت پیوستی
\\
خموش باش اگر چه که جمله سیمبران
&&
به آب زر بنویسند هر چه گفتستی
\\
ضیای حق و امام الهدی حسام الدین
&&
مجیر خلق به بالای روح از این پستی
\\
\end{longtable}
\end{center}
