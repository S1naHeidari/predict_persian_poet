\begin{center}
\section*{غزل شماره ۲۲۴۸: من آن نیم که بگویم حدیث نعمت او}
\label{sec:2248}
\addcontentsline{toc}{section}{\nameref{sec:2248}}
\begin{longtable}{l p{0.5cm} r}
من آن نیم که بگویم حدیث نعمت او
&&
که مست و بیخودم از چاشنی محنت او
\\
اگر چو چنگ بزارم از او شکایت نیست
&&
که همچو چنگم من بر کنار رحمت او
\\
ز من نباشد اگر پرده‌ای بگردانم
&&
که هر رگم متعلق بود به ضربت او
\\
اگر چه قند ندارم چو نی نوا دارم
&&
از آنک بر لب فضلش چشم ز شربت او
\\
کنون که نوبت خشم است لطف از این دست است
&&
چگونه باشد چون دررسم به نوبت او
\\
اگر بدزدم من ز آفتاب ننگی نیست
&&
چه ننگ باشد مر لعل را ز زینت او
\\
وگر چو لعل ندزدم ز آفتاب کمال
&&
گذر ز طینت خود چون کنم به طینت او
\\
نه لولیان سیاه دو چشم دزد ویند
&&
همی‌کشند نهان نور از بصیرت او
\\
ز آدمی چو بدزدی به کم قناعت کن
&&
که شح نفس قرین است با جبلت او
\\
از او مدزد به جز گوهر زمانه بها
&&
اگر تو واقفی از لطف و از سریرت او
\\
که نیست قهر خدا را به جز ز دزد خسیس
&&
که سوی کاله فانی بود عزیمت او
\\
دریغ شرح نگشت و ز شرح می‌ترسم
&&
که تیغ شرع برهنه‌ست در شریعت او
\\
گمان برد که مگر جرم او طمع بوده‌ست
&&
نه بلک خس طمعی بود آن جریمت او
\\
\end{longtable}
\end{center}
