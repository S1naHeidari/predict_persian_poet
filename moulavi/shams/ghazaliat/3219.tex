\begin{center}
\section*{غزل شماره ۳۲۱۹: اسفا لقلبی یوما هجرالحبیب داری}
\label{sec:3219}
\addcontentsline{toc}{section}{\nameref{sec:3219}}
\begin{longtable}{l p{0.5cm} r}
اسفا لقلبی یوما هجرالحبیب داری
&&
و تحرقت ضلوعی و جوانحی بناری
\\
و سعادة لیوم نظرالسعود فینا
&&
نزل السهیل سهلا و اقام فی جواری
\\
فدخلت لج بحر بطرا بما اتانی
&&
فغرقت فیه لکن نظرالحبیب جاری
\\
فتحت عیون قلبی فرأیت الف بحر
&&
و مراکبا علیها بهوی الهوا سواری
\\
تبریز حض فضلا و ترابه کمالا
&&
بشعاع نور صدر هو افضل الکبار
\\
تبریز اشفعی لی بشفاعة الی من
&&
زعقات وجد قلبی لحقتة بالتواری
\\
و لاجل سؤ حالی بتواضعی لدیه
&&
و تعرضی هوانی بهواه والصغار
\\
و تقول لا تقطع کبدا رهین شوق
&&
برجاک ما یرجی و یذوب بالبواری
\\
و تتوب من ذنوبی و تجاسری علیه
&&
و لیه عود قلبی و نهایة الفرار
\\
لمعات شمس دین هو سیدی حقیقا
&&
هی اصل اصل روحی و وراء هاعواری
\\
جمع الاله شملا قطعته شقوة لی
&&
فهو الکبیر یعفو لجنایة العصاری
\\
\end{longtable}
\end{center}
