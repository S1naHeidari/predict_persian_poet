\begin{center}
\section*{غزل شماره ۳۱۴۴: بحر ما را کنار بایستی}
\label{sec:3144}
\addcontentsline{toc}{section}{\nameref{sec:3144}}
\begin{longtable}{l p{0.5cm} r}
بحر ما را کنار بایستی
&&
وین سفر را قرار بایستی
\\
شیر بیشه میان زنجیرست
&&
شیر در مرغزار بایستی
\\
ماهیان می‌طپند اندر ریگ
&&
راه در جویبار بایستی
\\
بلبل مست سخت مخمورست
&&
گلشن و سبزه زار بایستی
\\
دیده‌ها از غبار خسته شدست
&&
دیده اعتبار بایستی
\\
همه گل خواره‌اند این طفلان
&&
مشفقی دایه وار بایستی
\\
ره به آب حیات می‌نبرند
&&
خضر را آبخوار بایستی
\\
دل پشیمان شدست ز آنچ گذشت
&&
دل امسال پار بایستی
\\
اندر این شهر قحط خورشیدست
&&
سایه شهریار بایستی
\\
شهر سرگین پرست پر گشته‌ست
&&
مشک نافه تتار بایستی
\\
مشک از پشک کس نمی‌داند
&&
مشک را انتشار بایستی
\\
دولت کودکانه می‌جویند
&&
دولت بی‌عثار بایستی
\\
مرگ تا در پیست روز شبست
&&
شب ما را نهار بایستی
\\
چون بمیری بمیرد این هنرت
&&
زین هنرهات عار بایستی
\\
چنگ در ما زدست این کمپیر
&&
چنگ او تار تار بایستی
\\
طالب کار و بار بسیارند
&&
طالب کردگار بایستی
\\
دم معدود اندکی ماندست
&&
نفسی بی‌شمار بایستی
\\
نفس ایزدی ز سوی یمن
&&
بر خلایق نثار بایستی
\\
مرگ دیگی برای ما پخته‌ست
&&
آن خورش را گوار بایستی
\\
یاد مردن چو دافع مرگست
&&
هر دمی یادگار بایستی
\\
هر دمی صد جنازه می‌گذرد
&&
دیده‌ها سوگوار بایستی
\\
ملک‌ها ماند و مالکان مردند
&&
ملکتی پایدار بایستی
\\
عقل بسته شد و هوا مختار
&&
عقل را اختیار بایستی
\\
هوش‌ها چون مگس در آن دوغست
&&
هوش را هوشیار بایستی
\\
زین چنین دوغ زشت گندیده
&&
این مگس را حذار بایستی
\\
معده پردوغ و گوش پر ز دروغ
&&
همت الفرار بایستی
\\
گوش‌ها بسته است لب بربند
&&
از خرد گوشوار بایستی
\\
از کنایات شمس تبریزی
&&
شرح معنی گذار بایستی
\\
\end{longtable}
\end{center}
