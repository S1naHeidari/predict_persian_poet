\begin{center}
\section*{غزل شماره ۱۶۸۰: من اگر پرغم اگر شادانم}
\label{sec:1680}
\addcontentsline{toc}{section}{\nameref{sec:1680}}
\begin{longtable}{l p{0.5cm} r}
من اگر پرغم اگر شادانم
&&
عاشق دولت آن سلطانم
\\
تا که خاک قدمش تاج من است
&&
اگرم تاج دهی نستانم
\\
تا لب قند خوشش پندم داد
&&
قند روید بن هر دندانم
\\
گلم ار چند که خارم در پاست
&&
یوسفم گر چه در این زندانم
\\
هر کی یعقوب من است او را من
&&
مونس زاویه احزانم
\\
در وصال شب او همچو نیم
&&
قند می نوشم و در افغانم
\\
پای من گر چه در این گل مانده‌ست
&&
نه که من سرو چنین بستانم
\\
ز جهان گر پنهانم چه عجب
&&
که نهان باشد جان من جانم
\\
گر چه پرخارم سر تا به قدم
&&
کوری خار چو گل خندانم
\\
بوده‌ام مؤمن توحید کنون
&&
مؤمنان را پس از این ایمانم
\\
سایه شخصم و اندازه او
&&
قامتش چند بود چندانم
\\
هر کی او سایه ندارد چو فلک
&&
او بداند که ز خورشیدانم
\\
قیمتم نبود هر چند زرم
&&
که به بازار نیم در کانم
\\
من درون دل این سنگ دلان
&&
چون زر و خاک به کان یک سانم
\\
چونک از کان جهان بازرهم
&&
زان سوی کون و مکان من دانم
\\
\end{longtable}
\end{center}
