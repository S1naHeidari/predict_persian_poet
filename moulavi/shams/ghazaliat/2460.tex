\begin{center}
\section*{غزل شماره ۲۴۶۰: تو نه چنانی که منم من نه چنانم که تویی}
\label{sec:2460}
\addcontentsline{toc}{section}{\nameref{sec:2460}}
\begin{longtable}{l p{0.5cm} r}
تو نه چنانی که منم من نه چنانم که تویی
&&
تو نه بر آنی که منم من نه بر آنم که تویی
\\
من همه در حکم توام تو همه در خون منی
&&
گر مه و خورشید شوم من کم از آنم که تویی
\\
با همه ای رشک پری چون سوی من برگذری
&&
باش چنین تیز مران تا که بدانم که تویی
\\
دوش گذشتی ز درم بوی نبردم ز تو من
&&
کرد خبر گوش مرا جان و روانم که تویی
\\
چون همه جان روید و دل همچو گیاه خاک درت
&&
جان و دلی را چه محل ای دل و جانم که تویی
\\
ای نظرت ناظر ما ای چو خرد حاضر ما
&&
لیک مرا زهره کجا تا به جهانم که تویی
\\
چون تو مرا گوش کشان بردی از آن جا که منم
&&
بر سر آن منظره‌ها هم بنشانم که تویی
\\
مستم و تو مست ز من سهو و خطا جست ز من
&&
من نرسم لیک بدان هم تو رسانم که تویی
\\
زین همه خاموش کنم صبر و صبر نوش کنم
&&
عذر گناهی که کنون گفت زبانم که تویی
\\
\end{longtable}
\end{center}
