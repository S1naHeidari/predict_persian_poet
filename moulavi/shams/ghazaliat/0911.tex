\begin{center}
\section*{غزل شماره ۹۱۱: به روز مرگ چو تابوت من روان باشد}
\label{sec:0911}
\addcontentsline{toc}{section}{\nameref{sec:0911}}
\begin{longtable}{l p{0.5cm} r}
به روز مرگ چو تابوت من روان باشد
&&
گمان مبر که مرا درد این جهان باشد
\\
برای من مگری و مگو دریغ دریغ
&&
به دوغ دیو درافتی دریغ آن باشد
\\
جنازه‌ام چو ببینی مگو فراق فراق
&&
مرا وصال و ملاقات آن زمان باشد
\\
مرا به گور سپاری مگو وداع وداع
&&
که گور پرده جمعیت جنان باشد
\\
فروشدن چو بدیدی برآمدن بنگر
&&
غروب شمس و قمر را چرا زبان باشد
\\
تو را غروب نماید ولی شروق بود
&&
لحد چو حبس نماید خلاص جان باشد
\\
کدام دانه فرورفت در زمین که نرست
&&
چرا به دانه انسانت این گمان باشد
\\
کدام دلو فرورفت و پر برون نامد
&&
ز چاه یوسف جان را چرا فغان باشد
\\
دهان چو بستی از این سوی آن طرف بگشا
&&
که های هوی تو در جو لامکان باشد
\\
\end{longtable}
\end{center}
