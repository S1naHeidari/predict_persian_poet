\begin{center}
\section*{غزل شماره ۱۸۶۲: چون چنگ شدم جانا آن چنگ تو دروا کن}
\label{sec:1862}
\addcontentsline{toc}{section}{\nameref{sec:1862}}
\begin{longtable}{l p{0.5cm} r}
چون چنگ شدم جانا آن چنگ تو دروا کن
&&
صد جان به عوض بستان وان شیوه تو با ما کن
\\
عیسی چو تویی ما را همکاسه مریم کن
&&
طنبور دل ما را هم ناله سرنا کن
\\
دستی بنه ای چنگی بر نبض چنین پیری
&&
وان خون دل زر را در ساغر صهبا کن
\\
جمعیت رندان را بر شاهد نقدی زن
&&
ور زهد سخن گوید تو وعده به فردا کن
\\
دیوانه و مستی را خواهی که بشورانی
&&
زنجیر خودم بنما وز دور تماشا کن
\\
دیدم ز تو من نقشی بر کالبدی بسته
&&
جان گفت علی الله گو دل گفت علالا کن
\\
زان روز من مسکین بی‌عقل شدم بی‌دین
&&
زان زلف خوش مشکین ما را تو چلیپا کن
\\
زنار ببند ای دل در دیر بکن منزل
&&
زان راهب پرحاصل یک بوسه تقاضا کن
\\
در چهره مخدومی شمس الحق تبریزی
&&
گر رغبت ما بینی این قصه غرا کن
\\
\end{longtable}
\end{center}
