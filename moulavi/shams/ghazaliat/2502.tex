\begin{center}
\section*{غزل شماره ۲۵۰۲: امیر دل همی‌گوید تو را گر تو دلی داری}
\label{sec:2502}
\addcontentsline{toc}{section}{\nameref{sec:2502}}
\begin{longtable}{l p{0.5cm} r}
امیر دل همی‌گوید تو را گر تو دلی داری
&&
که عاشق باش تا گیری ز نان و جامه بیزاری
\\
تو را گر قحط نان باشد کند عشق تو خبازی
&&
وگر گم گشت دستارت کند عشق تو دستاری
\\
ببین بی‌نان و بی‌جامه خوش و طیار و خودکامه
&&
ملایک را و جان‌ها را بر این ایوان زنگاری
\\
چو زین لوت و از این فرنی شود آزاد و مستغنی
&&
پی ملکی دگر افتد تو را اندیشه و زاری
\\
وگر دربند نان مانی بیاید یار روحانی
&&
تو را گوید که یاری کن نیاری کردنش یاری
\\
عصای عشق از خارا کند چشمه روان ما را
&&
تو زین جوع البقر یارا مکن زین بیش بقاری
\\
فروریزد سخن در دل مرا هر یک کند لابه
&&
که اول من برون آیم خمش مانم ز بسیاری
\\
الا یا صاحب الدار رایت الحسن فی جاری
&&
فاوقد بیننا نارا یطفی نوره ناری
\\
چو من تازی همی‌گویم به گوشم پارسی گوید
&&
مگر بدخدمتی کردم که رو این سو نمی‌آری
\\
نکردی جرم ای مه رو ولی انعام عام او
&&
به هر باغی گلی سازد که تا نبود کسی عاری
\\
غلامان دارد او رومی غلامان دارد او زنگی
&&
به نوبت روی بنماید به هندو و به ترکاری
\\
غلام رومیش شادی غلام زنگیش انده
&&
دمی این را دمی آن را دهد فرمان و سالاری
\\
همه روی زمین نبود حریف آفتاب و مه
&&
به شب پشت زمین روشن شود روی زمین تاری
\\
شب این روز آن باشد فراق آن وصال این
&&
قدح در دور می‌گردد ز صحت‌ها و بیماری
\\
گرت نبود شبی نوبت مبر گندم از این طاحون
&&
که بسیار آسیا بینی که نبود جوی او جاری
\\
چو من قشر سخن گفتم بگو ای نغز مغزش را
&&
که تا دریا بیاموزد درافشانی و درباری
\\
\end{longtable}
\end{center}
