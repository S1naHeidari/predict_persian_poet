\begin{center}
\section*{غزل شماره ۸۶۴: بلبل نگر که جانب گلزار می‌رود}
\label{sec:0864}
\addcontentsline{toc}{section}{\nameref{sec:0864}}
\begin{longtable}{l p{0.5cm} r}
بلبل نگر که جانب گلزار می‌رود
&&
گلگونه بین که بر رخ گلنار می‌رود
\\
میوه تمام گشته و بیرون شده ز خویش
&&
منصوروار خوش به سر دار می‌رود
\\
اشکوفه برگ ساخته نهر نثار شاه
&&
کاندر بهار شاه به ایثار می‌رود
\\
آن لاله‌ای چو راهب دل سوخته بدرد
&&
در خون دیده غرق به کهسار می‌رود
\\
نه ماه خار کرد فغان در وفای گل
&&
گل آن وفا چو دید سوی خار می‌رود
\\
ماندست چشم نرگس حیران به گرد باغ
&&
کاین جا حدیث دیده و دیدار می‌رود
\\
آب حیات گشته روان در بن درخت
&&
چون آتشی که در دل احرار می‌رود
\\
هر گلرخی که بود ز سرما اسیر خاک
&&
بر عشق گرمدار به بازار می‌رود
\\
اندر بهار وحی خدا درس عام گفت
&&
بنوشت باغ و مرغ به تکرار می‌رود
\\
این طالبان علم که تحصیل کرده‌اند
&&
هر یک گرفته خلعت و ادرار می‌رود
\\
گویی بهار گفت که الله مشتریست
&&
گل جندره زده به خریدار می‌رود
\\
گل از درون دل دم رحمان فزون شنید
&&
زودتر ز جمله بی‌دل و دستار می‌رود
\\
دل در بهار بیند هر شاخ جفت یار
&&
یاد آورد ز وصل و سوی یار می‌رود
\\
ای دل تو مفلسی و خریدار گوهری
&&
آن جا حدیث زر به خروار می‌رود
\\
نی نی حدیث زر به خروار کی کنند
&&
کان جا حدیث جان به انبار می‌رود
\\
این نفس مطمئنه خموشی غذای اوست
&&
وین نفس ناطقه سوی گفتار می‌رود
\\
\end{longtable}
\end{center}
