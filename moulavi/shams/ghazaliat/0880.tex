\begin{center}
\section*{غزل شماره ۸۸۰: صد مصر مملکت ز تعدی خراب شد}
\label{sec:0880}
\addcontentsline{toc}{section}{\nameref{sec:0880}}
\begin{longtable}{l p{0.5cm} r}
صد مصر مملکت ز تعدی خراب شد
&&
صد بحر سلطنت ز تطاول سراب شد
\\
صد برج حرص و بخل به خندق دراوفتاد
&&
صد بخت نیم خواب به کلی به خواب شد
\\
آن شاهراه غیب بر آن قوم بسته بود
&&
وان ماه زنگ ظلم به زیر حجاب شد
\\
وان چشم کو چو برق همی‌سوخت خلق را
&&
در نوحه اوفتاد و به گریه سحاب شد
\\
وان دل که صد هزار دل از وی کباب بود
&&
در آتش خدای کنون او کباب شد
\\
ای شاد آن کسی که از این عبرتی گرفت
&&
او را از این سیاست شه فتح باب شد
\\
چون روز گشت و دید که او شب چه کرده بود
&&
سودش نداشت سخره صد اضطراب شد
\\
چون بخت روسپید شب اندر دعا گذار
&&
زیرا دعای نوح به شب مستجاب شد
\\
\end{longtable}
\end{center}
