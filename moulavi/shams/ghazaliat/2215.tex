\begin{center}
\section*{غزل شماره ۲۲۱۵: گر رود دیده و عقل و خرد و جان تو مرو}
\label{sec:2215}
\addcontentsline{toc}{section}{\nameref{sec:2215}}
\begin{longtable}{l p{0.5cm} r}
گر رود دیده و عقل و خرد و جان تو مرو
&&
که مرا دیدن تو بهتر از ایشان تو مرو
\\
آفتاب و فلک اندر کنف سایه توست
&&
گر رود این فلک و اختر تابان تو مرو
\\
ای که درد سخنت صافتر از طبع لطیف
&&
گر رود صفوت این طبع سخندان تو مرو
\\
اهل ایمان همه در خوف دم خاتمتند
&&
خوفم از رفتن توست ای شه ایمان تو مرو
\\
تو مرو گر بروی جان مرا با خود بر
&&
ور مرا می‌نبری با خود از این خوان تو مرو
\\
با تو هر جزو جهان باغچه و بستان است
&&
در خزان گر برود رونق بستان تو مرو
\\
هجر خویشم منما هجر تو بس سنگ دل است
&&
ای شده لعل ز تو سنگ بدخشان تو مرو
\\
کی بود ذره که گوید تو مرو ای خورشید
&&
کی بود بنده که گوید به تو سلطان تو مرو
\\
لیک تو آب حیاتی همه خلقان ماهی
&&
از کمال کرم و رحمت و احسان تو مرو
\\
هست طومار دل من به درازی ابد
&&
برنوشته ز سرش تا سوی پایان تو مرو
\\
گر نترسم ز ملال تو بخوانم صد بیت
&&
که ز صد بهتر وز هجده هزاران تو مرو
\\
\end{longtable}
\end{center}
