\begin{center}
\section*{غزل شماره ۹۴۱: ز بعد خاک شدن یا زیان بود یا سود}
\label{sec:0941}
\addcontentsline{toc}{section}{\nameref{sec:0941}}
\begin{longtable}{l p{0.5cm} r}
ز بعد خاک شدن یا زیان بود یا سود
&&
به نقد خاک شوم بنگرم چه خواهد بود
\\
به نقد خاک شدن کار عاشقان باشد
&&
که راه بند شکستن خدایشان بنمود
\\
به امر موتوا من قبل ان تموتوا ما
&&
کنیم همچو محمد غزای نفس جهود
\\
جهود و مشرک و ترسا نتیجه نفس است
&&
ز پشک باشد دود خبیث نی از عود
\\
شود دمی همه خاک و شود دمی همه آب
&&
شود دمی همه آتش شود دمی همه دود
\\
شود دمی همه یار و شود دمی همه غار
&&
شود دمی همه تار و شود دمی همه پود
\\
به پیش خلق نشسته هزار نقش شود
&&
ولیک در نظر تو نه کم شود نه فزود
\\
به پیش چشم محمد بهشت و دوزخ عین
&&
به پیش چشم دگر کس مستر و مغمود
\\
مذللست قطوف بهشت بر احمد
&&
که کرد دست دراز و از آن بخواست ربود
\\
که تا دهد به صحابه ولیک آن بگداخت
&&
شد آب در کفش ایرا نبود وقت نمود
\\
\end{longtable}
\end{center}
