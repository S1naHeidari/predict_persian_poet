\begin{center}
\section*{غزل شماره ۹۸۲: یوسف آخرزمان خرامان شد}
\label{sec:0982}
\addcontentsline{toc}{section}{\nameref{sec:0982}}
\begin{longtable}{l p{0.5cm} r}
یوسف آخرزمان خرامان شد
&&
شکر و شهد مصر ارزان شد
\\
لعل عرشی تو چو رو بنمود
&&
تن کی باشد که سنگ‌ها جان شد
\\
تخته بند فراق تخت نشست
&&
تاج بر سر که چیست خاقان شد
\\
عشق مهمان بس شگرف آمد
&&
خانه‌ها خرد بود ویران شد
\\
پر و بال از جلال حق رویید
&&
قفس و مرغ و بیضه پران شد
\\
بادلان خیره گشته کاین دل کو
&&
بی دلان بی‌خبر که دل آن شد
\\
پای می‌کوب و عیش از سر گیر
&&
به سر من مگو که پایان شد
\\
زر چو درباخت خواجه صراف
&&
صرفه او برد زانک در کان شد
\\
شمس تبریز نردبانی ساخت
&&
بام گردون برآ که آسان شد
\\
\end{longtable}
\end{center}
