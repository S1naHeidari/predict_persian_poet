\begin{center}
\section*{غزل شماره ۳۰۳۲: از پگه ای یار زان عقار سمایی}
\label{sec:3032}
\addcontentsline{toc}{section}{\nameref{sec:3032}}
\begin{longtable}{l p{0.5cm} r}
از پگه ای یار زان عقار سمایی
&&
ده به کف ما که نور دیده مایی
\\
زانک وظیفه‌ست هر سحر ز کف تو
&&
دور بگردان که آفتاب لقایی
\\
هم به منش ده مها مده به دگر کس
&&
عهد و وفا کن که شهریار وفایی
\\
در تتق گردها لطیف هلالی
&&
وز جهت دردها لطیف دوایی
\\
دور بگردان که دور عشق تو آمد
&&
خلق کجااند و تو غریب کجایی
\\
بر عدد ذره جان فدای تو کردی
&&
چرخ فلک گر بدی مه تو بهایی
\\
با همه شاهی چو تشنگان خماریم
&&
ساقی ما شو بکن به لطف سقایی
\\
بهر تو آدم گرفت دبه و زنبیل
&&
بهر تو حوا نمود نیز حوایی
\\
آدم و حوا نبود بهر قدومت
&&
خالق می‌کرد گونه گونه خدایی
\\
در قدح تو چهار جوی بهشتست
&&
نه از شش و پنجست این سرورفزایی
\\
جمله اجزای ما شکفته کن این دم
&&
تا به فلک بررود غریو گوایی
\\
غبغب غنچه در این چمن بنخندد
&&
تا تو به خنده دهان او نگشایی
\\
طلعت خورشید تو اگر ننماید
&&
یمن نیاید ز سایه‌های همایی
\\
خانه بی‌جام نیست خوب و منور
&&
راه رهاوی بزن کز اوست رهایی
\\
مشک که ارزد هزار بحر فروریز
&&
کوه وقاری و بحر جود و سخایی
\\
هر شب آید ز غیب چون گله بانی
&&
جان رهد از تن چو اشتران چرایی
\\
در عدمستان کشد نهان شتران را
&&
خوش بچراند ز سبزه‌های عطایی
\\
بند کند چشمشان که راه نبینند
&&
راه الهیست نیست راه هوایی
\\
چون بنهد رخ پیاده در قدم شاه
&&
جست دواسبه ز نیستی و گدایی
\\
کژ نرود زان سپس به راه چو فرزین
&&
خواب ببیند چو پیل هند رجایی
\\
مات شو و لعب گفت و گوی رها کن
&&
کان شه شطرنج راست راه نمایی
\\
\end{longtable}
\end{center}
