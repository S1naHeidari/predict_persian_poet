\begin{center}
\section*{غزل شماره ۶۲۴: هر ذره که بر بالا می‌نوشد و پا کوبد}
\label{sec:0624}
\addcontentsline{toc}{section}{\nameref{sec:0624}}
\begin{longtable}{l p{0.5cm} r}
هر ذره که بر بالا می‌نوشد و پا کوبد
&&
خورشید ازل بیند وز عشق خدا کوبد
\\
آن را که بخنداند خوش دست برافشاند
&&
وان را که بترساند دندان به دعا کوبد
\\
مستست از آن باده با قامت خم داده
&&
این چرخ بر این بالا ناقوس صلا کوبد
\\
این عشق که مست آمد در باغ الست آمد
&&
کانگور وجودم را در جهد و عنا کوبد
\\
گر عشق نی مستستی یا باده پرستستی
&&
در باغ چرا آید انگور چرا کوبد
\\
تو پای همی‌کوبی و انگور نمی‌بینی
&&
کاین صوفی جان تو در معصره‌ها کوبد
\\
گویی همه رنج و غم بر من نهد آن همدم
&&
چون باغ تو را باشد انگور که را کوبد
\\
همخرقه ایوبی زان پای همی‌کوبی
&&
هر کو شنود ارکض او پای وفا کوبد
\\
از زمزمه یوسف یعقوب به رقص آمد
&&
وان یوسف شیرین لب پا کوبد پا کوبد
\\
ای طایفه پا کوبید چون حاضر آن جویید
&&
باشد که سعادت پا در پای شما کوبد
\\
این عشق چو بارانست ما برگ و گیا ای جان
&&
باشد که دمی باران بر برگ و گیا کوبد
\\
پا کوفت خلیل الله در آتش نمرودی
&&
تا حلق ذبیح الله بر تیغ بلا کوبد
\\
پا کوفته روح الله در بحر چو مرغابی
&&
با طایر معراجی تا فوق هوا کوبد
\\
خاموش کن و بی‌لب خوش طال بقا می‌زن
&&
می‌ترس که چشم بد بر طال بقا کوبد
\\
\end{longtable}
\end{center}
