\begin{center}
\section*{غزل شماره ۲۸۳۵: ز گزاف ریز باده که تو شاه ساقیانی}
\label{sec:2835}
\addcontentsline{toc}{section}{\nameref{sec:2835}}
\begin{longtable}{l p{0.5cm} r}
ز گزاف ریز باده که تو شاه ساقیانی
&&
تو نه‌ای ز جنس خلقان تو ز خلق آسمانی
\\
دو هزار خنب باده نرسد به جرعه تو
&&
ز کجا شراب خاکی ز کجا شراب جانی
\\
می و نقل این جهانی چو جهان وفا ندارد
&&
می و ساغر خدایی چو خداست جاودانی
\\
دل و جان و صد دل و جان به فدای آن ملاحت
&&
جز صورتی که داری تو به خاکیان چه مانی
\\
بزن آتشی که داری به جهان بی‌قراری
&&
بشکاف ز آتش خود دل قبه دخانی
\\
پر و بال بخش جان را که بسی شکسته پر شد
&&
پر و بال جان شکستی پی حکمتی که دانی
\\
سخنم به هوشیاری نمکی ندارد ای جان
&&
قدحی دو موهبت کن چو ز من سخن ستانی
\\
که هر آنچ مست گوید همه باده گفته باشد
&&
نکند به کشتی جان جز باده بادبانی
\\
مددی که نیم مستم بده آن قدح به دستم
&&
که به دولت تو رستم ز ملولی و گرانی
\\
هله ای بلای توبه بدران قبای توبه
&&
بر تو چه جای توبه که قضای ناگهانی
\\
تو خراب هر دکانی تو بلای خان و مانی
&&
زه کوه قاف گیری چو شتر همی‌کشانی
\\
عجب آن دگر بگویم که به گفت می‌نیاید
&&
تو بگو که از تو خوشتر که شه شکربیانی
\\
\end{longtable}
\end{center}
