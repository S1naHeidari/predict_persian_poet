\begin{center}
\section*{غزل شماره ۲۲۷۶: ای جبرئیل از عشق تو اندر سما پا کوفته}
\label{sec:2276}
\addcontentsline{toc}{section}{\nameref{sec:2276}}
\begin{longtable}{l p{0.5cm} r}
ای جبرئیل از عشق تو اندر سما پا کوفته
&&
ای انجم و چرخ و فلک اندر هوا پا کوفته
\\
تا گاو و ماهی زیر این هفتم زمین خرم شده
&&
هر برج تا گاو و سمک اندر علا پا کوفته
\\
انگور دل پرخون شده رفته به سوی میکده
&&
تا آتشی در می‌زده در خنب‌ها پا کوفته
\\
دل دیده آب روی خود در خاک کوی عشق او
&&
چون آن عنایت دید دل اندر عنا پا کوفته
\\
جان همچو ایوب نبی در ذوق آن لطف و کرم
&&
با قالب پرکرم خود اندر بلا پا کوفته
\\
خلقی که خواهند آمدن از نسل آدم بعد از این
&&
جان‌های ایشان بهر تو هم در فنا پا کوفته
\\
اندر خرابات فنا شاهنشهان محتشم
&&
هم بی‌کله سرور شده هم بی‌قبا پا کوفته
\\
قومی بدیده چیزکی عاشق شده لیک از حسد
&&
از کبر و ناموس و حیا هم در خلاء پا کوفته
\\
اصحاب کبر و نفس کی باشند لایق شاه را
&&
کز عزت این شاه ما صد کبریا پا کوفته
\\
قومی ببینی رقص کن در عشق نان و شوربا
&&
قومی دگر در عشقشان نان و ابا پا کوفته
\\
خوش گوهری کو گوهری هشت از هوای بحر او
&&
تا بحر شد در سر خود در اصطفا پا کوفته
\\
کو او و کو بیچاره‌ای کو هست در تقلید خود
&&
در خون خود چرخی زده و اندر رجا پا کوفته
\\
با این همه او به بود از غافل منکر که او
&&
گه می‌کند اقرارکی گه او ز لا پا کوفته
\\
قومی به عشق آن فتی بگذشت از هست و فنا
&&
قومی به عشق خود که من هستم فنا پا کوفته
\\
خفاش در تاریکیی در عشق ظلمت‌ها به رقص
&&
مرغان خورشیدی سحر تا والضحی پا کوفته
\\
تو شمس تبریزی بگو ای باد صبح تیزرو
&&
با من بگو احوال او با من درآ پا کوفته
\\
\end{longtable}
\end{center}
