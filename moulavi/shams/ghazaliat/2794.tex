\begin{center}
\section*{غزل شماره ۲۷۹۴: ای ملامت گر تو عاشق را سبک پنداشتی}
\label{sec:2794}
\addcontentsline{toc}{section}{\nameref{sec:2794}}
\begin{longtable}{l p{0.5cm} r}
ای ملامت گر تو عاشق را سبک پنداشتی
&&
تا به پیش عاشقان بند و فسون برداشتی
\\
گه مثال و رمز گویی گه صریح و آشکار
&&
تخم را اندر زمین ریگ ما چون کاشتی
\\
ای زمین ریگ شرمت نیست از انبار تخم
&&
فارغی چون تخم‌ها را تو عدم انگاشتی
\\
ای زمین تخم گیر آخر تویی هم اصل تخم
&&
کز نتیجه خویش شاخ سنبلی افراشتی
\\
چونک هر جزوی به غیر اصل خود پیوند نیست
&&
تو چرا طیره شدی و پند و جنگ آغاشتی
\\
ریش خندی می‌کند بر پند تاب عاشقی
&&
کی شود سرد آتشی از پند و جنگ و آشتی
\\
ماهتاب ار چه جهان گیرد تو در تبریز باش
&&
در شعاع شمس دین زیرا که مرغ چاشتی
\\
\end{longtable}
\end{center}
