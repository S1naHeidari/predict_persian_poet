\begin{center}
\section*{غزل شماره ۹۲۷: به باغ بلبل از این پس حدیث ما گوید}
\label{sec:0927}
\addcontentsline{toc}{section}{\nameref{sec:0927}}
\begin{longtable}{l p{0.5cm} r}
به باغ بلبل از این پس حدیث ما گوید
&&
حدیث خوبی آن یار دلربا گوید
\\
چو باد در سر بید افتد و شود رقصان
&&
خدای داند کو با هوا چه‌ها گوید
\\
چنار فهم کند اندکی ز سوز چمن
&&
دو دست پهن برآرد خوش و دعا گوید
\\
بپرسم از گل کان حسن از که دزدیدی
&&
ز شرم سست بخندد ولی کجا گوید
\\
اگر چه مست بود گل خراب نیست چو من
&&
که راز نرگس مخمور با شما گوید
\\
چو رازها طلبی در میان مستان رو
&&
که راز را سر سرمست بی‌حیا گوید
\\
که باده دختر کرمست و خاندان کرم
&&
دهان کیسه گشادست و از سخا گوید
\\
خصوص باده عرشی ز ذوالجلال کریم
&&
سخاوت و کرم آن مگر خدا گوید
\\
ز شیردانه عارف بجوشد آن شیره
&&
ز قعر خم تن او تو را صلا گوید
\\
چو سینه شیر دهد شیره هم تواند داد
&&
ز سینه چشمه جاریش ماجرا گوید
\\
چو مستتر شود آن روح خرقه باز شود
&&
کلاه و سر بنهد ترک این قبا گوید
\\
چو خون عقل خورد باده لاابالی وار
&&
دهان گشاید و اسرار کبریا گوید
\\
خموش باش که کس باورت نخواهد کرد
&&
که مس بد نخورد آنچ کیمیا گوید
\\
خبر ببر سوی تبریز مفخر آفاق
&&
مگر که مدح تو را شمس دین ما گوید
\\
\end{longtable}
\end{center}
