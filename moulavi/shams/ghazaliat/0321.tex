\begin{center}
\section*{غزل شماره ۳۲۱: آن خواجه را از نیم شب بیماریی پیدا شده‌ست}
\label{sec:0321}
\addcontentsline{toc}{section}{\nameref{sec:0321}}
\begin{longtable}{l p{0.5cm} r}
آن خواجه را از نیم شب بیماریی پیدا شده‌ست
&&
تا روز بر دیوار ما بی‌خویشتن سر می‌زده‌ست
\\
چرخ و زمین گریان شده وز ناله‌اش نالان شده
&&
دم‌های او سوزان شده گویی که در آتشکده‌ست
\\
بیماریی دارد عجب نی درد سر نی رنج تب
&&
چاره ندارد در زمین کز آسمانش آمده‌ست
\\
چون دید جالینوس را نبضش گرفت و گفت او
&&
دستم بهل دل را ببین رنجم برون قاعده‌ست
\\
صفراش نی سوداش نی قولنج و استسقاش نی
&&
زین واقعه در شهر ما هر گوشه‌ای صد عربده‌ست
\\
نی خواب او را نی خورش از عشق دارد پرورش
&&
کاین عشق اکنون خواجه را هم دایه و هم والده‌ست
\\
گفتم خدایا رحمتی کرام گیرد ساعتی
&&
نی خون کس را ریخته‌ست نی مال کس را بستده‌ست
\\
آمد جواب از آسمان کو را رها کن در همان
&&
کاندر بلای عاشقان دارو و درمان بیهدست
\\
این خواجه را چاره مجو بندش منه پندش مگو
&&
کان جا که افتادست او نی مفسقه نی معبده‌ست
\\
تو عشق را چون دیده‌ای از عاشقان نشنیده‌ای
&&
خاموش کن افسون مخوان نی جادوی نی شعبده‌ست
\\
ای شمس تبریزی بیا ای معدن نور و ضیا
&&
کاین روح باکار و کیا بی‌تابش تو جامدست
\\
\end{longtable}
\end{center}
