\begin{center}
\section*{غزل شماره ۲۶۶۹: برفتیم ای عقیق لامکانی}
\label{sec:2669}
\addcontentsline{toc}{section}{\nameref{sec:2669}}
\begin{longtable}{l p{0.5cm} r}
برفتیم ای عقیق لامکانی
&&
ز شهر تو تو باید که بمانی
\\
سفر کردیم چون استارگان ما
&&
ز تو هم سوی تو که آسمانی
\\
یکی صورت رود دیگر بیاید
&&
به مهمانخانه‌ات زیرا که جانی
\\
که مهمانان مثال چار فصلند
&&
تو اصل فصل‌هایی که جهانی
\\
خیال خوب تو در سینه بردیم
&&
شفق از آفتاب آمد نشانی
\\
به پیشت ماند دل با ما نیامد
&&
دل از تو کی رود چون دلستانی
\\
سر دل‌ها به زیر سایه‌ات باد
&&
که دل‌ها را در این مرعا شبانی
\\
فروریزید دندان‌های گرگان
&&
از آنگه که نمودی مهربانی
\\
بهل تا بحر گوید قصه خویش
&&
که تا باری ببینی قصه خوانی
\\
\end{longtable}
\end{center}
