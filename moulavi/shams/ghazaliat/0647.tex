\begin{center}
\section*{غزل شماره ۶۴۷: تدبیر کند بنده و تقدیر نداند}
\label{sec:0647}
\addcontentsline{toc}{section}{\nameref{sec:0647}}
\begin{longtable}{l p{0.5cm} r}
تدبیر کند بنده و تقدیر نداند
&&
تدبیر به تقدیر خداوند چه ماند
\\
بنده چو بیندیشد پیداست چه بیند
&&
حیلت بکند لیک خدایی بنداند
\\
گامی دو چنان آید کو راست نهادست
&&
وان گاه که داند که کجاهاش کشاند
\\
استیزه مکن مملکت عشق طلب کن
&&
کاین مملکتت از ملک الموت رهاند
\\
شه را تو شکاری شو کم گیر شکاری
&&
کاشکار تو را باز اجل بازستاند
\\
خامش کن و بگزین تو یکی جای قراری
&&
کان جا که گزینی ملک آن جات نشاند
\\
\end{longtable}
\end{center}
