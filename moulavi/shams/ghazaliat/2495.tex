\begin{center}
\section*{غزل شماره ۲۴۹۵: آنک بخورد دم به دم سنگ جفای صدمنی}
\label{sec:2495}
\addcontentsline{toc}{section}{\nameref{sec:2495}}
\begin{longtable}{l p{0.5cm} r}
آنک بخورد دم به دم سنگ جفای صدمنی
&&
غم نخورد از آنک تو روی بر او ترش کنی
\\
می چو در او عمل کند رقص کند بغل زند
&&
ز آنک نهاد در بغل خاص عقیق معدنی
\\
مرد قمارخانه‌ام عالم بی‌کرانه‌ام
&&
چشم بیار در رخم بنگر پیش روشنی
\\
ننگرد او به رنگ تو غم نخورد ز جنگ تو
&&
خواجه مگر ندیده‌ای ملک و مقام ایمنی
\\
هیچ عسل ترش شود سرکه اگر ترش رود
&&
از پی آب کی هلد روغن طبع روغنی
\\
من که در آن نظاره‌ام مست و سماع باره‌ام
&&
لیک سماع هر کسی پاک نباشد از منی
\\
هست سماع ما نظر هست سماع او بطر
&&
لیک نداند ای پسر ترک زبان ارمنی
\\
در تک گور مؤمنان رقص کنان و کف زنان
&&
مست به بزم لامکان خورده شراب مؤمنی
\\
پیش تو است این دم او می‌نبری ز یار بو
&&
می‌نگری تو سو به سو پله چشم می‌زنی
\\
\end{longtable}
\end{center}
