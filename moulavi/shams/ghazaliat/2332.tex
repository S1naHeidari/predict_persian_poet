\begin{center}
\section*{غزل شماره ۲۳۳۲: ای آنک تو را ما ز همه کون گزیده}
\label{sec:2332}
\addcontentsline{toc}{section}{\nameref{sec:2332}}
\begin{longtable}{l p{0.5cm} r}
ای آنک تو را ما ز همه کون گزیده
&&
بگذاشته ما را تو و در خود نگریده
\\
تو شرم نداری که تو را آینه ماییم
&&
تو آینه ناقص کژشکل خریده
\\
ای بی‌خبر از خویش که از عکس دل تو
&&
بر عارض جان‌ها گل و گلزار دمیده
\\
صد روح غلام تو تو هر دم چو کنیزک
&&
آراسته خود را و به بازار دویده
\\
بر چرخ ز شادی جمال تو عروسی است
&&
ای همچو کمان جان تو در غصه خمیده
\\
صد خرمن نعمت جهت پیشکش تو
&&
وز بهر یکی دانه در این دام پریده
\\
ای آنک شنیدی سخن عشق ببین عشق
&&
کو حالت بشنیده و کو حالت دیده
\\
در عشق همان کس که تو را دوش بیاراست
&&
امشب تو به خلوتگه عشق آی جریده
\\
چون صبر بود از شه شمس الحق تبریز
&&
ای آب حیات ابد از شاه چشیده
\\
\end{longtable}
\end{center}
