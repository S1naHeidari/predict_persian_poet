\begin{center}
\section*{غزل شماره ۱۲۹: مشکن دل مرد مشتری را}
\label{sec:0129}
\addcontentsline{toc}{section}{\nameref{sec:0129}}
\begin{longtable}{l p{0.5cm} r}
مشکن دل مرد مشتری را
&&
بگذار ره ستمگری را
\\
رحم آر مها که در شریعت
&&
قربان نکنند لاغری را
\\
مخمور توام به دست من ده
&&
آن جام شراب گوهری را
\\
پندی بده و به صلح آور
&&
آن چشم خمار عبهری را
\\
فرمای به هندوان جادو
&&
کز حد نبرند ساحری را
\\
در شش دره‌ای فتاد عاشق
&&
بشکن در حبس شش دری را
\\
یک لحظه معزمانه پیش آ
&&
جمع آور حلقه پری را
\\
سر می نهد این خمار از بن
&&
هر لحظه شراب آن سری را
\\
صد جا چو قلم میان ببسته
&&
تنگ شکر معسکری را
\\
ای عشق برادرانه پیش آ
&&
بگذار سلام سرسری را
\\
ای ساقی روح از در حق
&&
مگذار حق برادری را
\\
ای نوح زمانه هین روان کن
&&
این کشتی طبع لنگری را
\\
ای نایب مصطفی بگردان
&&
آن ساغر زفت کوثری را
\\
پیغام ز نفخ صور داری
&&
بگشای لب پیمبری را
\\
ای سرخ صباغت علمدار
&&
بگشا پر و بال جعفری را
\\
پرلاله کن و پر از گل سرخ
&&
این صحن رخ مزعفری را
\\
اسپید نمی‌کنم دگر من
&&
درریز رحیق احمری را
\\
\end{longtable}
\end{center}
