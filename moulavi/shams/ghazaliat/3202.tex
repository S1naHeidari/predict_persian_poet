\begin{center}
\section*{غزل شماره ۳۲۰۲: هذا سیدی، هذا سندی}
\label{sec:3202}
\addcontentsline{toc}{section}{\nameref{sec:3202}}
\begin{longtable}{l p{0.5cm} r}
هذا سیدی، هذا سندی
&&
هذا سکنی، هذا مددی
\\
هذا کنفی، هذا عمدی
&&
هذا ازلی، هذا ابدی
\\
یا من وجهه، ضعف القمر
&&
یا من قده صعف‌الشجر
\\
یا من زارنی، وقت‌السحر
&&
یا من عشقه نور نظری
\\
گر تو بدوی، ور تو بپری
&&
زین دلبر جان، خود جان نبری
\\
ور جان ببری از دست غمش
&&
از مرده خری، والله بتری
\\
ایلا کلیمو ایلا شاهمو
&&
خراذی دیذیس ذوزمس آنیمو
\\
پوذپسه بنی، پوپونی لالی
&&
میذن چاکوس کالی تو یالی
\\
از لیلی خود مجنون شده‌ام
&&
وز صد مجنون افزون شده‌ام
\\
وز خون جگر پرخون شده‌ام
&&
باری بنگر تا چون شده‌ام
\\
گر زانک مرا زین جان بکشی
&&
من غرقه شوم، در عین خوشی
\\
دریا شود این دو چشم سرم
&&
گر گوش مرا زان سو بکشی
\\
یا منبسطا فی تربیتی
&&
یا مبتشرا فی تهنیتی
\\
ان کنت تری ان تقتلنی
&&
یا قاتلنا انت دیتی
\\
گر خویش تو بر مستی بزنی
&&
هستی تو بر هستی بزنی
\\
در حلقه درآ بهر دل ما
&&
شکلی بکنی دستی بزنی
\\
صدگونه خوشی دیدم ز کسی
&&
گفتم که: « لبت »، گفتا: « نچشی »
\\
بر گورم اگر آیی بنگر
&&
پرعشق بود چشمم ز کشی
\\
آن باغ بود بی‌صورت بر
&&
وآن گنج بود بی‌صورت زر
\\
شب عیش بود بی‌نقل و سمر
&&
لاتسألنی زان چیز دگر
\\
\end{longtable}
\end{center}
