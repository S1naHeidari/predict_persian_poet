\begin{center}
\section*{غزل شماره ۲۵۲: نذر کند یار که امشب تو را}
\label{sec:0252}
\addcontentsline{toc}{section}{\nameref{sec:0252}}
\begin{longtable}{l p{0.5cm} r}
نذر کند یار که امشب تو را
&&
خواب نباشد ز طمع برتر آ
\\
حفظ دماغ آن مدمغ بود
&&
چونک سهر باید یار مرا
\\
هست دماغ تو چو زیت چراغ
&&
هست چراغ تن ما بی‌وفا
\\
گر دبه پر زیت بود سود نیست
&&
صبح شود گشت چراغت فنا
\\
دعوت خورشید به از زیت تو
&&
چند چراغ ارزد آن یک صلا
\\
چشم خوشش را ابدا خواب نیست
&&
مست کند چشم همه خلق را
\\
جمله بخسپند و تبسم کند
&&
چشم خوشش بر خلل چشم‌ها
\\
پس لمن الملک برآید به چرخ
&&
کو ملکان خوش زرین قبا
\\
کو امرا کو وزرا کو مهان
&&
بهر بلادالله حافظ کجا
\\
اهل علم چون شد و اهل قلم
&&
دیو نیابی تو به دیوان سرا
\\
خانه و تنشان شده تاریک و تنگ
&&
چونک ببردیم یکی دم ضیا
\\
گرد که بادش برود چون شود
&&
افتد بر خاک سیه بی‌نوا
\\
چون بجهند از حجب خواب خویش
&&
بازبمالند سبال جفا
\\
اه چه فراموش گرند این گروه
&&
دانششان هیچ ندارد بقا
\\
زود فراموش شود سوز شمع
&&
بر دل پروانه ز جهل و عما
\\
بازبیاید به پر نیم سوز
&&
بازبسوزد چو دل ناسزا
\\
نذر تو کن حکم تو کن حاکمی
&&
بر شب و بر روز و سحر ای خدا
\\
\end{longtable}
\end{center}
