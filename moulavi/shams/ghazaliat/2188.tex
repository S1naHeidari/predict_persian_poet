\begin{center}
\section*{غزل شماره ۲۱۸۸: در این رقص و در این های و در این هو}
\label{sec:2188}
\addcontentsline{toc}{section}{\nameref{sec:2188}}
\begin{longtable}{l p{0.5cm} r}
در این رقص و در این های و در این هو
&&
میان ماست گردان میر مه رو
\\
اگر چه روی می‌دزدد ز مردم
&&
کجا پنهان شود آن روی نیکو
\\
چو چشمت بست آن جادوی استاد
&&
درآ در آب جو و آب می‌جو
\\
تو گویی کو و کو او نیز سر را
&&
به هر سو می‌کند یعنی که کو کو
\\
ز کوی عشق می‌آید ندایی
&&
رها کن کو و کو دررو در این کو
\\
برو دامان خاقان گیر محکم
&&
چو او باشد چه اندیشی ز باجو
\\
برو پهلوی قصرش خانه‌ای گیر
&&
که تا ایمن شوی از درد پهلو
\\
گریزان درد و دارو در پی تو
&&
زهی لطف و زهی احسان و دارو
\\
سیه کاری و تلخی را رها کن
&&
بر ما زو بیا غلطان چو مازو
\\
از او یابد طرب هم مست و هم می
&&
از او گیرد نمک هم رو و هم خو
\\
از او اندیش و گفتن را رها کن
&&
لطیف اندیش باشد مرد کم گو
\\
\end{longtable}
\end{center}
