\begin{center}
\section*{غزل شماره ۱۷۴۳: اگر چه ما نه خروس و نه ماکیان داریم}
\label{sec:1743}
\addcontentsline{toc}{section}{\nameref{sec:1743}}
\begin{longtable}{l p{0.5cm} r}
اگر چه ما نه خروس و نه ماکیان داریم
&&
ز بیضه سر کن و بنگر که ما کیان داریم
\\
به آفتاب حقایق به هر سحر گوییم
&&
تو جمله جانی و ما از تو نیم جان داریم
\\
گر از صفات تو نتوان نشان نمود ولی
&&
ز بی‌نشانی اوصاف او نشان داریم
\\
دل چو شبنم ما را به بحر بازرسان
&&
که دم به دم ز غریبی دو صد زیان داریم
\\
چو یوسف از کف گرگان دریده پیرهنم
&&
ولی ز همت یعقوب پاسبان داریم
\\
به دام تو که همه دام‌ها زبون ویند
&&
که هر قدم ز قدم دام امتحان داریم
\\
ولیک بندگشا هر دم آن کند با ما
&&
که مادر و پدر و عم مگر که آن داریم
\\
بنوش کردن زهر این چه جرات است مگر
&&
ز کان فضل تو تریاق بی‌کران داریم
\\
به خرج کردن این نقد عمر مبتشریم
&&
ز عمربخش مگر عمر جاودان داریم
\\
نگیرد آینه زنگار هیچ اگر گیرد
&&
ز عین زنگ بدان روی دیدمان داریم
\\
یقین بنشکند آن نردبان وگر شکند
&&
ز عین رخنه اشکست نردبان داریم
\\
رهین روز چرایی چو شب کند روزی
&&
مکان بهل که مکانی ز لامکان داریم
\\
بهار حله دریدی ز رشک و زرد شدی
&&
اگر بدیش خبر کاین چنین خزان داریم
\\
دهان پر است و خموشم که تا بگویی تو
&&
کز آن لب شکرینت شکرفشان داریم
\\
\end{longtable}
\end{center}
