\begin{center}
\section*{غزل شماره ۱۷۳۱: اگر زمین و فلک را پر از سلام کنیم}
\label{sec:1731}
\addcontentsline{toc}{section}{\nameref{sec:1731}}
\begin{longtable}{l p{0.5cm} r}
اگر زمین و فلک را پر از سلام کنیم
&&
وگر سگان تو را فرش سیم خام کنیم
\\
وگر همای تو را هر سحر که می آید
&&
ز جان و دیده و دل حلقه‌های دام کنیم
\\
وگر هزار دل پاک را به هر سر راه
&&
به دست نامه پرخون به تو پیام کنیم
\\
وگر چو نقره و زر پاک و خالص از پی تو
&&
میان آتش تو منزل و مقام کنیم
\\
به ذات پاک منزه که بعد این همه کار
&&
به هر طرف نگرانیم تا کدام کنیم
\\
قرار عاقبت کار هم بر این افتاد
&&
که خویش را همه حیران و خیره نام کنیم
\\
و آنگهی که رسد باده‌های حیرانان
&&
ز شیشه خانه دل صد هزار جام کنیم
\\
چو سیمبر به صفا تنگمان به بر گیرد
&&
فلک که کره تند است ماش رام کنیم
\\
چو مغز روح از آن باده‌ها به جوش آید
&&
چهار حد جهان را به تک دو گام کنیم
\\
ز شمس تبریز انگشتری چو بستانیم
&&
هزار خسرو تمغاج را غلام کنیم
\\
\end{longtable}
\end{center}
