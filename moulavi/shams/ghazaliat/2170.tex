\begin{center}
\section*{غزل شماره ۲۱۷۰: هر شش جهتم ای جان منقوش جمال تو}
\label{sec:2170}
\addcontentsline{toc}{section}{\nameref{sec:2170}}
\begin{longtable}{l p{0.5cm} r}
هر شش جهتم ای جان منقوش جمال تو
&&
در آینه درتابی چون یافت صقال تو
\\
آیینه تو را بیند اندازه عرض خود
&&
در آینه کی گنجد اشکال کمال تو
\\
خورشید ز خورشیدت پرسید کیت بینم
&&
گفتا که شوم طالع در وقت زوال تو
\\
رهوار نتانی شد این سوی که چون ناقه
&&
بسته‌ست تو را زانو ای عقل عقال تو
\\
عقلی که نمی‌گنجد در هفت فلک فرش
&&
ای عشق چرا رفت او در دام و جوال تو
\\
این عقل یکی دانه از خرمن عشق آمد
&&
شد بسته آن دانه جمله پر و بال تو
\\
در بحر حیات حق خوردی تو یکی غوطه
&&
جان ابدی دیدی جان گشت وبال تو
\\
ملکش به چه کار آید با ملکت عشق تو
&&
جاهش به چه کار آید با جاه و جلال تو
\\
صد حلقه زرین بین در گوش جهان اکنون
&&
از لطف جواب تو وز ذوق سؤال تو
\\
خامان که زر پخته از دست تو نامدشان
&&
شادند به جای زر با سنگ و سفال تو
\\
صد چرخ طواف آرد بر گرد زمین تو
&&
صد بدر سجود آرد در پیش هلال تو
\\
با تو سگ نفس ما روباهی و مکر آرد
&&
که شیر سجود آرد در پیش شغال تو
\\
بی‌پای چو روز و شب اندر سفریم ای جان
&&
چون می‌رسد از گردون هر لحظه تعال تو
\\
تاریکی ما چه بود در حضرت نور تو
&&
فعل بد ما چه بود با حسن فعال تو
\\
روزیم چو سایه ما بر گرد درخت تو
&&
شب تا به سحر نالان ایمن ز ملال تو
\\
از شوق عتاب تو آن آدم بگزیده
&&
از صدر جنان آمد در صف نعال تو
\\
دریای دل از مدحت می‌غرد و می‌جوشد
&&
لیکن لب خود بستم از شوق مقال تو
\\
\end{longtable}
\end{center}
