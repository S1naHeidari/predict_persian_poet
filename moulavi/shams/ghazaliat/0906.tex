\begin{center}
\section*{غزل شماره ۹۰۶: گرفت خشم ز بستان سرخری و برون شد}
\label{sec:0906}
\addcontentsline{toc}{section}{\nameref{sec:0906}}
\begin{longtable}{l p{0.5cm} r}
گرفت خشم ز بستان سرخری و برون شد
&&
چو زشت بود به صورت به خوی زشت فزون شد
\\
چون دل سیاه بد و قلب کوره دید و سیه شد
&&
چو قازغان تهی بد به کنج خانه نگون شد
\\
چو ژیوه بود به جنبش نبود زنده اصلی
&&
نمود جنبش عاریه بازرفت و سکون شد
\\
نیافت صیقل احمد ز کفر بولهب ار چه
&&
ز سرکشی و ز مکرش دلش قنینه خون شد
\\
فروکشم به نمد در چو آینه رخ فکرت
&&
چو آینه بنمایم کی رام شد کی حرون شد
\\
منم که هجو نگویم به جز خواطر خود را
&&
که خاطرم نفسی عقل گشت و گاه جنون شد
\\
مرا درونه تو شهری جدا شمر به سر خود
&&
به آب و گل نشد آن شهر من به کن فیکون شد
\\
سخن ندارم با نیک و بد من از بیرون
&&
که آن چه کرد و کجا رفت و این ز وسوسه چون شد
\\
خموش کن که هجا را به خود کشد دل نادان
&&
همیشه بود نظرهای کژنگر نه کنون شد
\\
\end{longtable}
\end{center}
