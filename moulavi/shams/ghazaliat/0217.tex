\begin{center}
\section*{غزل شماره ۲۱۷: چه نیکبخت کسی که خدای خواند تو را}
\label{sec:0217}
\addcontentsline{toc}{section}{\nameref{sec:0217}}
\begin{longtable}{l p{0.5cm} r}
چه نیکبخت کسی که خدای خواند تو را
&&
درآ درآ به سعادت درت گشاد خدا
\\
که برگشاید درها مفتح الابواب
&&
که نزل و منزل بخشید نحن نزلنا
\\
که دانه را بشکافد ندا کند به درخت
&&
که سر برآر به بالا و می فشان خرما
\\
که دردمید در آن نی که بود زیر زمین
&&
که گشت مادر شیرین و خسرو حلوا
\\
کی کرد در کف کان خاک را زر و نقره
&&
کی کرد در صدفی آب را جواهرها
\\
ز جان و تن برهیدی به جذبه جانان
&&
ز قاب و قوس گذشتی به جذب او ادنی
\\
هم آفتاب شده مطربت که خیز سجود
&&
به سوی قامت سروی ز دست لاله صلا
\\
چنین بلند چرا می‌پرد همای ضمیر
&&
شنید بانگ صفیری ز ربی الاعلی
\\
گل شکفته بگویم که از چه می‌خندد
&&
که مستجاب شد او را از آن بهار دعا
\\
چو بوی یوسف معنی گل از گریبان یافت
&&
دهان گشاد به خنده که‌های یا بشرا
\\
به دی بگوید گلشن که هر چه خواهی کن
&&
به فر عدل شهنشه نترسم از یغما
\\
چو آسمان و زمین در کفش کم از سیبی‌ست
&&
تو برگ من بربایی کجا بری و کجا
\\
چو اوست معنی عالم به اتفاق همه
&&
بجز به خدمت معنی کجا روند اسما
\\
شد اسم مظهر معنی کاردت ان اعرف
&&
وز اسم یافت فراغت بصیرت عرفا
\\
کلیم را بشناسد به معرفت‌هارون
&&
اگر عصاش نباشد وگر ید بیضا
\\
چگونه چرخ نگردد بگرد بام و درش
&&
که آفتاب و مه از نور او کنند سخا
\\
چو نور گفت خداوند خویشتن را نام
&&
غلام چشم شو ایرا ز نور کرد چرا
\\
از این همه بگذشتم نگاه دار تو دست
&&
که می‌خرامد از آن پرده مست یوسف ما
\\
چه جای دست بود عقل و هوش شد از دست
&&
که ساقی‌ست دلارام و باده اش گیرا
\\
خموش باش که تا شرح این همو گوید
&&
که آب و تاب همان به که آید از بالا
\\
\end{longtable}
\end{center}
