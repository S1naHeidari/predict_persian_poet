\begin{center}
\section*{غزل شماره ۲۸۴۸: سحر است خیز ساقی بکن آنچ خوی داری}
\label{sec:2848}
\addcontentsline{toc}{section}{\nameref{sec:2848}}
\begin{longtable}{l p{0.5cm} r}
سحر است خیز ساقی بکن آنچ خوی داری
&&
سر خنب برگشای و برسان شراب ناری
\\
چه شود اگر ز عیسی دو سه مرده زنده گردد
&&
خوش و شیرگیر گردد ز کفت دو سه خماری
\\
قدح چو آفتابت چو به دور اندرآید
&&
برهد جهان تیره ز شب و ز شب شماری
\\
ز شراب چون عقیقت شکفد گل حقیقت
&&
که حیات مرغ زاری و بهار مرغزاری
\\
بدهیم جان شیرین به شراب خسروانی
&&
چو سر خمار ما را به کف کرم بخاری
\\
که ز فکرت دقیقه خللی است در شقیقه
&&
تو روان کن آب درمان بگشا ره مجاری
\\
همه آتشی تو مطلق بر ما شد این محقق
&&
که هزار دیگ سر را به تفی به جوش آری
\\
همه مطربان خروشان همه از تو گشته جوشان
&&
همه رخت خود فروشان خوششان همی‌فشاری
\\
\end{longtable}
\end{center}
