\begin{center}
\section*{غزل شماره ۲۲۴۴: سیر نیم سیر نی از لب خندان تو}
\label{sec:2244}
\addcontentsline{toc}{section}{\nameref{sec:2244}}
\begin{longtable}{l p{0.5cm} r}
سیر نیم سیر نی از لب خندان تو
&&
ای که هزار آفرین بر لب و دندان تو
\\
هیچ کسی سیر شد ای پسر از جان خویش
&&
جان منی چون یکی است جان من و جان تو
\\
تشنه و مستسقیم مرگ و حیاتم ز آب
&&
دور بگردان که من بنده دوران تو
\\
پیش کشی می‌کنی پیش خودم کش تمام
&&
تا که برآرد سرم سر ز گریبان تو
\\
گر چه دو دستم بخست دست من آن تو است
&&
دست چه کار آیدم بی‌دم و دستان تو
\\
عشق تو گفت ای کیا در حرم ما بیا
&&
تا نکند هیچ دزد قصد حرمدان تو
\\
گفتم ای ذوالقدم حلقه این در شدم
&&
تا که نرنجد ز من خاطر دربان تو
\\
گفت که هم بر دری واقف و هم در بری
&&
خارج و داخل توی هر دو وطن آن تو
\\
خامش و دیگر مخوان بس بود این نزل و خوان
&&
تا به ابد روم و ترک برخورد از خوان تو
\\
\end{longtable}
\end{center}
