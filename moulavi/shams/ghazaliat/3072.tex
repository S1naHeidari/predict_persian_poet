\begin{center}
\section*{غزل شماره ۳۰۷۲: به من نگر که بجز من به هر کی درنگری}
\label{sec:3072}
\addcontentsline{toc}{section}{\nameref{sec:3072}}
\begin{longtable}{l p{0.5cm} r}
به من نگر که به جز من به هر کی درنگری
&&
یقین شود که ز عشق خدای بی‌خبری
\\
بدان رخی بنگر که کو نمک ز حق دارد
&&
بود که ناگه از آن رخ تو دولتی ببری
\\
تو را چو عقل پدر بوده‌ست و تن مادر
&&
جمال روی پدر درنگر اگر پسری
\\
بدانک پیر سراسر صفات حق باشد
&&
وگر چه پیر نماید به صورت بشری
\\
به پیش تو چو کفست و به وصف خود دریا
&&
به چشم خلق مقیمست و هر دم او سفری
\\
هنوز مشکل مانده‌ست حال پیر تو را
&&
هزار آیت کبری در او چه بی‌هنری
\\
رسید صورت روحانیی به مریم دل
&&
ز بارگاه منزه ز خشکی و ز تری
\\
از آن نفس که در او سر روح پنهان شد
&&
بکرد حامله دل را رسول رهگذری
\\
ایا دلی که تو حامل شدی از آن خسرو
&&
به وقت جنبش آن حمل تا در او نگری
\\
چو حمل صورت گیرد ز شمس تبریزی
&&
چو دل شوی تو و چون دل به سوی غیب پری
\\
\end{longtable}
\end{center}
