\begin{center}
\section*{غزل شماره ۲۹۳۲: چه باشد ای برادر یک شب اگر نخسپی}
\label{sec:2932}
\addcontentsline{toc}{section}{\nameref{sec:2932}}
\begin{longtable}{l p{0.5cm} r}
چه باشد ای برادر یک شب اگر نخسپی
&&
چون شمع زنده باشی همچون شرر نخسپی
\\
درهای آسمان را شب سخت می‌گشاید
&&
نیک اختریت باشد گر چون قمر نخسپی
\\
گر مرد آسمانی مشتاق آن جهانی
&&
زیر فلک نمانی جز بر زبر نخسپی
\\
چون لشکر حبش شب بر روم حمله آرد
&&
باید که همچو قیصر در کر و فر نخسپی
\\
عیسی روزگاری سیاح باش در شب
&&
در آب و در گل ای جان تا همچو خر نخسپی
\\
شب رو که راه‌ها را در شب توان بریدن
&&
گر شهر یار خواهی اندر سفر نخسپی
\\
در سایه خدایی خسپند نیکبختان
&&
زنهار ای برادر جای دگر نخسپی
\\
چون از پدر جدا شد یوسف نه مبتلا شد
&&
تو یوسفی هلا تا جز با پدر نخسپی
\\
زیرا برادرانت دارند قصد جانت
&&
هان تا میان ایشان جز با حذر نخسپی
\\
تبریز شمس دین را جز ره روی نیابد
&&
گر تو ز ره روانی بر ره گذر نخسپی
\\
\end{longtable}
\end{center}
