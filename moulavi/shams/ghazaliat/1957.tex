\begin{center}
\section*{غزل شماره ۱۹۵۷: هست عاقل هر زمانی در غم پیدا شدن}
\label{sec:1957}
\addcontentsline{toc}{section}{\nameref{sec:1957}}
\begin{longtable}{l p{0.5cm} r}
هست عاقل هر زمانی در غم پیدا شدن
&&
هست عاشق هر زمانی بیخود و شیدا شدن
\\
عاقلان از غرقه گشتن بر گریز و بر حذر
&&
عاشقان را کار و پیشه غرقه دریا شدن
\\
عاقلان را راحت از راحت رسانیدن بود
&&
عاشقان را ننگ باشد بند راحت‌ها شدن
\\
عاشق اندر حلقه باشد از همه تن‌ها چنانک
&&
زیت را و آب را در یک محل تنها شدن
\\
و آنک باشد در نصیحت دادن عشاق عشق
&&
نیست او را حاصلی جز سخره سودا شدن
\\
عشق بوی مشک دارد زان سبب رسوا بود
&&
مشک را کی چاره باشد از چنین رسوا شدن
\\
عشق باشد چون درخت و عاشقان سایه درخت
&&
سایه گر چه دور افتد بایدش آن جا شدن
\\
بر مقام عقل باید پیر گشتن طفل را
&&
در مقام عشق بینی پیر را برنا شدن
\\
شمس تبریزی به عشقت هر کی او پستی گزید
&&
همچو عشق تو بود در رفعت و بالا شدن
\\
\end{longtable}
\end{center}
