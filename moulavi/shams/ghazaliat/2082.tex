\begin{center}
\section*{غزل شماره ۲۰۸۲: برای چشم تو صد چشم بد توان دیدن}
\label{sec:2082}
\addcontentsline{toc}{section}{\nameref{sec:2082}}
\begin{longtable}{l p{0.5cm} r}
برای چشم تو صد چشم بد توان دیدن
&&
چه چشم داری ای چشم ما به تو روشن
\\
پی رضای تو آدم گریست سیصد سال
&&
که تا ز خنده وصلش گشاده گشت دهن
\\
به قدر گریه بود خنده تو یقین می‌دان
&&
جزای گریه ابر است خنده‌های چمن
\\
اگر نه از نسب آدمی برو مگری
&&
که نیست از سیهی زنگ را بکا و حزن
\\
چو خود سپید ندیده‌ست روسیه شاد است
&&
چو پور قیصر رومی تو راه زنگ بزن
\\
بسی خدنگ خورد اسپ تازی غازی
&&
که تازی است نه پالانی است و نی کودن
\\
خصوص مرکب تازی که تو بر او باشی
&&
نشسته‌ای شه هیجا و پهلوان زمن
\\
چو خارپشت شود پشت و پهلوش از تیر
&&
که هست در صف هیجاش کر و فر وطن
\\
چو شاه دست به پشت و سرش فرومالد
&&
که ای گزیده سرآخر تویی مخصص من
\\
شوند آن همه تیرش چو چوب‌های نبات
&&
همه حلاوت و لذت همه عطا و منن
\\
خبر ندارد پالانیی از این لذت
&&
سپر سلامت و محروم و بی‌بها و ثمن
\\
ز گفت توبه کنم توبه سود نیست مرا
&&
به پیش پنجه‌ات ای ارسلان توبه شکن
\\
\end{longtable}
\end{center}
