\begin{center}
\section*{غزل شماره ۹۳۴: میان باغ گل سرخ‌های و هو دارد}
\label{sec:0934}
\addcontentsline{toc}{section}{\nameref{sec:0934}}
\begin{longtable}{l p{0.5cm} r}
میان باغ گل سرخ‌های و هو دارد
&&
که بو کنید دهان مرا چه بو دارد
\\
به باغ خود همه مستند لیک نی چون گل
&&
که هر یکی به قدح خورد و او سبو دارد
\\
چو سال سال نشاطست و روز روز طرب
&&
خنک مرا و کسی را که عیش خو دارد
\\
چرا مقیم نباشد چو ما به مجلس گل
&&
کسی که ساقی باقی ماه رو دارد
\\
هزار جان مقدس فدای آن جانی
&&
که او به مجلس ما امر اشربوا دارد
\\
سؤال کردم گل را که بر کی می‌خندی
&&
جواب داد بر آن زشت کو دو شو دارد
\\
هزار بار خزان کرد نوبهار تو را
&&
چه عشق دارد با ما چه جست و جو دارد
\\
پیاله‌ای به من آورد گل که باده خوری
&&
خورم چرا نخورم بنده هم گلو دارد
\\
چه حاجتیست گلو باده خدایی را
&&
که ذره ذره همه نقل و می از او دارد
\\
عجب که خار چه بدمست و تیز و روترشست
&&
ز رشک آنک گل و لاله صد عدو دارد
\\
به طور موسی بنگر که از شراب گزاف
&&
دهان ندارد و اشکم چهارسو دارد
\\
به مستیان درختان نگر به فصل بهار
&&
شکوفه کرده که در شرب می غلو دارد
\\
\end{longtable}
\end{center}
