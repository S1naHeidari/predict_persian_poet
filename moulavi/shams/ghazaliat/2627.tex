\begin{center}
\section*{غزل شماره ۲۶۲۷: عاشق شو و عاشق شو بگذار زحیری}
\label{sec:2627}
\addcontentsline{toc}{section}{\nameref{sec:2627}}
\begin{longtable}{l p{0.5cm} r}
عاشق شو و عاشق شو بگذار زحیری
&&
سلطان بچه‌ای آخر تا چند اسیری
\\
سلطان بچه را میر و وزیری همه عار است
&&
زنهار به جز عشق دگر چیز نگیری
\\
آن میر اجل نیست اسیر اجل است او
&&
جز وزر نیامد همه سودای وزیری
\\
گر صورت گرمابه نه‌ای روح طلب کن
&&
تا عاشق نقشی ز کجا روح پذیری
\\
در خاک میامیز که تو گوهر پاکی
&&
در سرکه میامیز که تو شکر و شیری
\\
هر چند از این سوی تو را خلق ندانند
&&
آن سوی که سو نیست چه بی‌مثل و نظیری
\\
این عالم مرگ است و در این عالم فانی
&&
گر ز آنک نه میری نه بس است این که نمیری
\\
در نقش بنی آدم تو شیر خدایی
&&
پیداست در این حمله و چالیش و دلیری
\\
تا فضل و مقامات و کرامات تو دیدم
&&
بیزارم از این فضل و مقامات حریری
\\
بی‌گاه شد این عمر ولیکن چو تو هستی
&&
در نور خدایی چه به گاهی و چه دیری
\\
اندازه معشوق بود عزت عاشق
&&
ای عاشق بیچاره ببین تا ز چه تیری
\\
زیبایی پروانه به اندازه شمع است
&&
آخر نه که پروانه این شمع منیری
\\
شمس الحق تبریز از آنت نتوان دید
&&
که اصل بصر باشی یا عین بصیری
\\
\end{longtable}
\end{center}
