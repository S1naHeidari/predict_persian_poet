\begin{center}
\section*{غزل شماره ۱۸۲۵: من طربم طرب منم زهره زند نوای من}
\label{sec:1825}
\addcontentsline{toc}{section}{\nameref{sec:1825}}
\begin{longtable}{l p{0.5cm} r}
من طربم طرب منم زهره زند نوای من
&&
عشق میان عاشقان شیوه کند برای من
\\
عشق چو مست و خوش شود بیخود و کش مکش شود
&&
فاش کند چو بی‌دلان بر همگان هوای من
\\
ناز مرا به جان کشد بر رخ من نشان کشد
&&
چرخ فلک حسد برد ز آنچ کند به جای من
\\
من سر خود گرفته‌ام من ز وجود رفته‌ام
&&
ذره به ذره می زند دبدبه فنای من
\\
آه که روز دیر شد آهوی لطف شیر شد
&&
دلبر و یار سیر شد از سخن و دعای من
\\
یار برفت و ماند دل شب همه شب در آب و گل
&&
تلخ و خمار می طپم تا به صبوح وای من
\\
تا که صبوح دم زند شمس فلک علم زند
&&
باز چو سرو تر شود پشت خم دوتای من
\\
باز شود دکان گل ناز کنند جزو و کل
&&
نای عراق با دهل شرح دهد ثنای من
\\
ساقی جان خوبرو باده دهد سبو سبو
&&
تا سر و پای گم کند زاهد مرتضای من
\\
بهر خدای ساقیا آن قدح شگرف را
&&
بر کف پیر من بنه از جهت رضای من
\\
گفت که باده دادمش در دل و جهان نهادمش
&&
بال و پری گشادمش از صفت صفای من
\\
پیر کنون ز دست شد سخت خراب و مست شد
&&
نیست در آن صفت که او گوید نکته‌های من
\\
ساقی آدمی کشم گر بکشد مرا خوشم
&&
راح بود عطای او روح بود سخای من
\\
باده تویی سبو منم آب تویی و جو منم
&&
مست میان کو منم ساقی من سقای من
\\
از کف خویش جسته‌ام در تک خم نشسته‌ام
&&
تا همگی خدا بود حاکم و کدخدای من
\\
شمس حقی که نور او از تبریز تیغ زد
&&
غرقه نور او شد این شعشعه ضیای من
\\
\end{longtable}
\end{center}
