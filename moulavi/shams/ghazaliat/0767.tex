\begin{center}
\section*{غزل شماره ۷۶۷: صنما جفا رها کن کرم این روا ندارد}
\label{sec:0767}
\addcontentsline{toc}{section}{\nameref{sec:0767}}
\begin{longtable}{l p{0.5cm} r}
صنما جفا رها کن کرم این روا ندارد
&&
بنگر به سوی دردی که ز کس دوا ندارد
\\
ز فلک فتاد طشتم به محیط غرقه گشتم
&&
به درون بحر جز تو دلم آشنا ندارد
\\
ز صبا همی‌رسیدم خبری که می‌پزیدم
&&
ز غمت کنون دل من خبر از صبا ندارد
\\
به رخان چون زر من به بر چو سیم خامت
&&
به زر او ربوده شد که چو تو دلربا ندارد
\\
هله ساقیا سبکتر ز درون ببند آن در
&&
تو بگو به هر کی آید که سر شما ندارد
\\
همه عمر این چنین دم نبدست شاد و خرم
&&
به حق وفای یاری که دلش وفا ندارد
\\
به از این چه شادمانی که تو جانی و جهانی
&&
چه غمست عاشقان را که جهان بقا ندارد
\\
برویم مست امشب به وثاق آن شکرلب
&&
چه ز جامه کن گریزد چو کسی قبا ندارد
\\
به چه روز وصل دلبر همه خاک می‌شود زر
&&
اگر آن جمال و منظر فر کیمیا ندارد
\\
به چه چشم‌های کودن شود از نگار روشن
&&
اگر آن غبار کویش سر توتیا ندارد
\\
هله من خموش کردم برسان دعا و خدمت
&&
چه کند کسی که در کف به جز از دعا ندارد
\\
\end{longtable}
\end{center}
