\begin{center}
\section*{غزل شماره ۱۰۷۴: گرم در گفتار آمد آن صنم این الفرار}
\label{sec:1074}
\addcontentsline{toc}{section}{\nameref{sec:1074}}
\begin{longtable}{l p{0.5cm} r}
گرم در گفتار آمد آن صنم این الفرار
&&
بانگ خیزاخیز آمد در عدم این الفرار
\\
صد هزاران شعله بر در صد هزاران مشعله
&&
کیست بر در کیست بر در هم منم این الفرار
\\
از درون نی آن منم گویان که بر در کیست آن
&&
هم منم بر در که حلقه می‌زنم این الفرار
\\
هر که پندارد دو نیمم پس دو نیمش کرد قهر
&&
ور یکی ام پس هم آب و روغنم این الفرار
\\
چون یکی باشم که زلفم صد هزاران ظلمتست
&&
چون دو باشم چونک ماه روشنم این الفرار
\\
گرد خانه چند جویی تو مرا چون کاله دزد
&&
بنگر این دزدی که شد بر روزنم این الفرار
\\
زین قفس سر را ز هر سوراخ بیرون می‌کنم
&&
سوی وصلت پر خود را می‌کنم این الفرار
\\
در درون این قفس تن در سر سودا گداخت
&&
وز قفس بیرون به هر دم گردنم این الفرار
\\
بی‌می از شمس الحق تبریز مست گفتنم
&&
طوطیم یا بلبلم یا سوسنم این الفرار
\\
\end{longtable}
\end{center}
