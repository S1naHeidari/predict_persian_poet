\begin{center}
\section*{غزل شماره ۱۱۳۱: گفت لبم چون شکر ارزد گنج گهر}
\label{sec:1131}
\addcontentsline{toc}{section}{\nameref{sec:1131}}
\begin{longtable}{l p{0.5cm} r}
گفت لبم چون شکر ارزد گنج گهر
&&
آه ندارم گهر گفت نداری بخر
\\
از گهرم دام کن ور نبود وام کن
&&
خانه غلط کرده‌ای عاشق بی‌سیم و زر
\\
آمده‌ای در قمار کیسه پرزر بیار
&&
ور نه برو از کنار غصه و زحمت ببر
\\
راه زنانیم ما جامه کنانیم ما
&&
گر تو ز مایی درآ کاسه بزن کوزه خور
\\
دام همه ما دریم مال همه ما خوریم
&&
از همه ما خوشتریم کوری هر کور و کر
\\
جامه خران دیگرند جامه دران دیگرند
&&
جامه دران برکنند سبلت هر جامه خر
\\
سبلت فرعون تن موسی جان برکند
&&
تا همه تن جان شود هر سر مو جانور
\\
در ره عشاق او روی معصفر شناس
&&
گوهر عشق اشک دان اطلس خون جگر
\\
قیمت روی چو زر چیست بگو لعل یار
&&
قیمت اشک چو در چیست بگو آن نظر
\\
بنده آن ساقیم تا به ابد باقیم
&&
عالم ما برقرار عالمیان برگذر
\\
هر کی بزاد او بمرد جان به موکل سپرد
&&
عاشق از کس نزاد عشق ندارد پدر
\\
گر تو از این رو نه‌ای همچو قفا پس نشین
&&
ور تو قفا نیستی پیش درآ چون سپر
\\
چون سپر بی‌خبر پیش درآ و ببین
&&
از نظر زخم دوست باخبران بی‌خبر
\\
\end{longtable}
\end{center}
