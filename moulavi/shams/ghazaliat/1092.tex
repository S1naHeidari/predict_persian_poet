\begin{center}
\section*{غزل شماره ۱۰۹۲: اختران را شب وصلست و نثارست و نثار}
\label{sec:1092}
\addcontentsline{toc}{section}{\nameref{sec:1092}}
\begin{longtable}{l p{0.5cm} r}
اختران را شب وصلست و نثارست و نثار
&&
چون سوی چرخ عروسیست ز ماه ده و چار
\\
زهره در خویش نگنجد ز نواهای لطیف
&&
همچو بلبل که شود مست ز گل فصل بهار
\\
جدی را بین به کرشمه به اسد می‌نگرد
&&
حوت را بین که ز دریا چه برآورد غبار
\\
مشتری اسب دوانید سوی پیر زحل
&&
که جوانی تو ز سر گیر و بر او مژده بیار
\\
کف مریخ که پرخون بود از قبضه تیغ
&&
گشت جان بخش چو خورشید مشرف آثار
\\
دلو گردون چو از آن آب حیات آمد پر
&&
شود آن سنبله خشک از او گوهربار
\\
جوز پرمغز ز میزان و شکستن نرمد
&&
حمل از مادر خود کی بگریزد به نفار
\\
تیر غمزه چو رسید از سوی مه بر دل قوس
&&
شب روی پیشه گرفت از هوسش عقرب وار
\\
اندر این عید برو گاو فلک قربان کن
&&
گر نه‌ای چون سرطان در وحلی کژرفتار
\\
این فلک هست سطرلاب و حقیقت عشقست
&&
هر چه گوییم از این گوش سوی معنی دار
\\
شمس تبریز در آن صبح که تو درتابی
&&
روز روشن شود از روی چو ماهت شب تار
\\
\end{longtable}
\end{center}
