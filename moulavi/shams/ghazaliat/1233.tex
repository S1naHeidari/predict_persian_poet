\begin{center}
\section*{غزل شماره ۱۲۳۳: درون ظلمتی می‌جو صفاتش}
\label{sec:1233}
\addcontentsline{toc}{section}{\nameref{sec:1233}}
\begin{longtable}{l p{0.5cm} r}
درون ظلمتی می‌جو صفاتش
&&
که باشد نور و ظلمت محو ذاتش
\\
در آن ظلمت رسی در آب حیوان
&&
نه در هر ظلمتست آب حیاتش
\\
بسی دل‌ها رسد آن جا چو برقی
&&
ولی مشکل بود آن جا ثباتش
\\
خنک آن بیدق فرخ رخی را
&&
که هر دم می‌رساند شه به ماتش
\\
بسی دل‌ها چو شکر شد شکسته
&&
نگشته صاف و نابسته نباتش
\\
بپوشیده ز خود تشریف فقرش
&&
هم از یاقوت خود داده زکاتش
\\
اگر رویش به قبله می‌نبینی
&&
درون کعبه شد جای صلاتش
\\
شب قدرست او دریاب او را
&&
امان یابی چو برخوانی براتش
\\
ز هجران خداوند شمس تبریز
&&
شده نالان حیاتش از مماتش
\\
\end{longtable}
\end{center}
