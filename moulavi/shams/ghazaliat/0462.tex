\begin{center}
\section*{غزل شماره ۴۶۲: یوسف کنعانیم روی چو ماهم گواست}
\label{sec:0462}
\addcontentsline{toc}{section}{\nameref{sec:0462}}
\begin{longtable}{l p{0.5cm} r}
یوسف کنعانیم روی چو ماهم گواست
&&
هیچ کس از آفتاب خط و گواهان نخواست
\\
سرو بلندم تو را راست نشانی دهم
&&
راستتر از سروقد نیست نشانی راست
\\
هست گواه قمر چستی و خوبی و فر
&&
شعشعه اختران خط و گواه سماست
\\
ای گل و گلزارها کیست گواه شما
&&
بوی که در مغزهاست رنگ که در چشم‌هاست
\\
عقل اگر قاضیست کو خط و منشور او
&&
دیدن پایان کار صبر و وقار و وفاست
\\
عشق اگر محرم است چیست نشان حرم
&&
آنک به جز روی دوست در نظر او فناست
\\
عالم دون روسپیست چیست نشانی آن
&&
آنک حریفیش پیش و آن دگرش در قفاست
\\
چونک به راهش کند آن به برش درکشد
&&
بوسه او نه از وفاست خلعت او نه از عطاست
\\
چیست نشانی آنک هست جهانی دگر
&&
نو شدن حال‌ها رفتن این کهنه‌هاست
\\
روز نو و شام نو باغ نو و دام نو
&&
هر نفس اندیشه نو نوخوشی و نوغناست
\\
نو ز کجا می‌رسد کهنه کجا می‌رود
&&
گر نه ورای نظر عالم بی‌منتهاست
\\
عالم چون آب جوست بسته نماید ولیک
&&
می‌رود و می‌رسد نو نو این از کجاست
\\
خامش و دیگر مگو آنک سخن بایدش
&&
اصل سخن گو بجو اصل سخن شاه ماست
\\
شاه شهی بخش جان مفخر تبریزیان
&&
آنک در اسرار عشق همنفس مصطفاست
\\
\end{longtable}
\end{center}
