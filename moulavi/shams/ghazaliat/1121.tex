\begin{center}
\section*{غزل شماره ۱۱۲۱: آمد بهار خرم و آمد رسول یار}
\label{sec:1121}
\addcontentsline{toc}{section}{\nameref{sec:1121}}
\begin{longtable}{l p{0.5cm} r}
آمد بهار خرم و آمد رسول یار
&&
مستیم و عاشقیم و خماریم و بی‌قرار
\\
ای چشم و ای چراغ روان شو به سوی باغ
&&
مگذار شاهدان چمن را در انتظار
\\
اندر چمن ز غیب غریبان رسیده‌اند
&&
رو رو که قاعدست که القادم یزار
\\
گل از پی قدوم تو در گلشن آمدست
&&
خار از پی لقای تو گشتست خوش عذار
\\
ای سرو گوش دار که سوسن به شرح تو
&&
سر تا به سر زبان شد بر طرف جویبار
\\
غنچه گره گره شد و لطفت گره گشاست
&&
از تو شکفته گردد و بر تو کند نثار
\\
گویی قیامتست که برکرد سر ز خاک
&&
پوسیدگان بهمن و دی مردگان پار
\\
تخمی که مرده بود کنون یافت زندگی
&&
رازی که خاک داشت کنون گشت آشکار
\\
شاخی که میوه داشت همی‌نازد از نشاط
&&
بیخی که آن نداشت خجل گشت و شرمسار
\\
آخر چنین شوند درختان روح نیز
&&
پیدا شود درخت نکوشاخ بختیار
\\
لشکر کشیده شاه بهار و بساخت برگ
&&
اسپر گرفته یاسمن و سبزه ذوالفقار
\\
گویند سر بریم فلان را جو گندنا
&&
آن را ببین معاینه در صنع کردگار
\\
آری چو دررسد مدد نصرت خدا
&&
نمرود را برآید از پشه‌ای دمار
\\
\end{longtable}
\end{center}
