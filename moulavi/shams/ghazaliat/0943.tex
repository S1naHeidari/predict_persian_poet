\begin{center}
\section*{غزل شماره ۹۴۳: نماز شام چو خورشید در غروب آید}
\label{sec:0943}
\addcontentsline{toc}{section}{\nameref{sec:0943}}
\begin{longtable}{l p{0.5cm} r}
نماز شام چو خورشید در غروب آید
&&
ببندد این ره حس راه غیب بگشاید
\\
به پیش درکند ارواح را فرشته خواب
&&
به شیوه گله بانی که گله را پاید
\\
به لامکان به سوی مرغزار روحانی
&&
چه شهرها و چه روضاتشان که بنماید
\\
هزار صورت و شخص عجب ببیند روح
&&
چو خواب نقش جهان را از او فروساید
\\
هماره گویی جان خود مقیم آن جا بود
&&
نه یاد این کند و نی ملالش افزاید
\\
ز بار و رخت که این جا بر آن همی‌لرزید
&&
دلش چنان برهد که غمیش نگزاید
\\
\end{longtable}
\end{center}
