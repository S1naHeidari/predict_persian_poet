\begin{center}
\section*{غزل شماره ۲۰۶۱: بوسه بده خویش را ای صنم سیمتن}
\label{sec:2061}
\addcontentsline{toc}{section}{\nameref{sec:2061}}
\begin{longtable}{l p{0.5cm} r}
بوسه بده خویش را ای صنم سیمتن
&&
ای به خطا تو مجوی خویشتن اندر ختن
\\
گر به بر اندرکشی سیمبری چون تو کو
&&
بوسه جان بایدت بر دهن خویش زن
\\
بهر جمال تو است جندره حوریان
&&
عکس رخ خوب توست خوبی هر مرد و زن
\\
پرده خوبی تو شقه زلف تو است
&&
ور نه برون تافتی نور تو ای خوش ذقن
\\
آمد نقاش تن سوی بتان ضمیر
&&
دست و دلش درشکست باز بماندش دهن
\\
این قفس پرنگار پرده مرغ دل است
&&
دل تو بنشناختی از قفس دل شکن
\\
پرده برانداخت دل از گل آدم چنانک
&&
سجده درآمد ملک گشت به دل مفتتن
\\
واسطه برخاستی گر نفسی ترک عشق
&&
پیش نشستی به لطف کای چلپی کیمسن
\\
چشم شدی غیب بین گر نظر شمس دین
&&
مفخر تبریزیان بر تو شدی غمزه زن
\\
\end{longtable}
\end{center}
