\begin{center}
\section*{غزل شماره ۱۰۰۷: گفت کسی خواجه سنایی بمرد}
\label{sec:1007}
\addcontentsline{toc}{section}{\nameref{sec:1007}}
\begin{longtable}{l p{0.5cm} r}
گفت کسی خواجه سنایی بمرد
&&
مرگ چنین خواجه نه کاریست خرد
\\
قالب خاکی به زمین بازداد
&&
روح طبیعی به فلک واسپرد
\\
ماه وجودش ز غباری برست
&&
آب حیاتش به درآمد ز درد
\\
پرتو خورشید جدا شد ز تن
&&
هر چه ز خورشید جدا شد فسرد
\\
صافی انگور به میخانه رفت
&&
چونک اجل خوشه تن را فشرد
\\
شد همگی جان مثل آفتاب
&&
جان شده را مرده نباید شمرد
\\
مغز تو نغزست مگر پوست مرد
&&
مغز نمیرد مگرش دوست برد
\\
پوست بهل دست در آن مغز زن
&&
یا بشنو قصه آن ترک و کرد
\\
کرد پی دزدی انبان ترک
&&
خرقه بپوشید و سر و مو سترد
\\
\end{longtable}
\end{center}
