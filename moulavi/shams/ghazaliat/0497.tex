\begin{center}
\section*{غزل شماره ۴۹۷: صوفیان آمدند از چپ و راست}
\label{sec:0497}
\addcontentsline{toc}{section}{\nameref{sec:0497}}
\begin{longtable}{l p{0.5cm} r}
صوفیان آمدند از چپ و راست
&&
در به در کو به کو که باده کجاست
\\
در صوفی دل‌ست و کویش جان
&&
باده صوفیان ز خم خداست
\\
سر خم را گشاد ساقی و گفت
&&
الصلا هر کسی که عاشق ماست
\\
این چنین باده و چنین مستی
&&
در همه مذهبی حلال و رواست
\\
توبه بشکن که در چنین مجلس
&&
از خطا توبه صد هزار خطاست
\\
چون شکستی تو زاهدان را نیز
&&
الصلا زن که روز روز صلاست
\\
مردمت گر ز چشم خویش انداخت
&&
مردم چشم عاشقانت جاست
\\
گر برفت آب روی کمتر غم
&&
جای عاشق برون آب و هواست
\\
آشنایان اگر ز ما گشتند
&&
غرقه را آشنا در آن دریاست
\\
\end{longtable}
\end{center}
