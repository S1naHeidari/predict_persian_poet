\begin{center}
\section*{غزل شماره ۲۲۴۰: ننشیند آتشم چو ز حق خاست آرزو}
\label{sec:2240}
\addcontentsline{toc}{section}{\nameref{sec:2240}}
\begin{longtable}{l p{0.5cm} r}
ننشیند آتشم چو ز حق خاست آرزو
&&
زین سو نظر مکن که از آن جاست آرزو
\\
تردامنم مبین که از آن بحر تر شدم
&&
گر گوهری ببین که چه دریاست آرزو
\\
شست حق است آرزو و روح ماهی است
&&
صیاد جان فداست چه زیباست آرزو
\\
چون این جهان نبود خدا بود در کمال
&&
ز آوردن من و تو چه می‌خواست آرزو
\\
گر آرزو کژ است در او راستی بسی است
&&
نی کز کژی و راست مبراست آرزو
\\
آن کان دولتی که نهان شد به نام بد
&&
آن چیست کژ نشین و بگو راست آرزو
\\
موری است نقب کرده میان سرای عشق
&&
هر چند بی‌پر است و به پرواست آرزو
\\
مورش مگو ز جهل سلیمان وقت او است
&&
زیرا که تخت و ملک بیاراست آرزو
\\
بگشای شمس مفخر تبریز این گره
&&
چیزی است کو نه ماست و نه جز ماست آرزو
\\
\end{longtable}
\end{center}
