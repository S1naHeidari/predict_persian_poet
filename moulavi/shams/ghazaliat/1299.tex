\begin{center}
\section*{غزل شماره ۱۲۹۹: گویند شاه عشق ندارد وفا دروغ}
\label{sec:1299}
\addcontentsline{toc}{section}{\nameref{sec:1299}}
\begin{longtable}{l p{0.5cm} r}
گویند شاه عشق ندارد وفا دروغ
&&
گویند صبح نبود شام تو را دروغ
\\
گویند بهر عشق تو خود را چه می‌کشی
&&
بعد از فنای جسم نباشد بقا دروغ
\\
گویند اشک چشم تو در عشق بیهده‌ست
&&
چون چشم بسته گشت نباشد لقا دروغ
\\
گویند چون ز دور زمانه برون شدیم
&&
زان سو روان نباشد این جان ما دروغ
\\
گویند آن کسان که نرستند از خیال
&&
جمله خیال بد قصص انبیا دروغ
\\
گویند آن کسان که نرفتند راه راست
&&
ره نیست بنده را به جناب خدا دروغ
\\
گویند رازدان دل اسرار و راز غیب
&&
بی‌واسطه نگوید مر بنده را دروغ
\\
گویند بنده را نگشایند راز دل
&&
وز لطف بنده را نبرد بر سما دروغ
\\
گویند آن کسی که بود در سرشت خاک
&&
با اهل آسمان نشود آشنا دروغ
\\
گویند جان پاک از این آشیان خاک
&&
با پر عشق برنپرد بر هوا دروغ
\\
گویند ذره ذره بد و نیک خلق را
&&
آن آفتاب حق نرساند جزا دروغ
\\
خاموش کن ز گفت وگر گویدت کسی
&&
جز حرف و صوت نیست سخن را ادا دروغ
\\
\end{longtable}
\end{center}
