\begin{center}
\section*{غزل شماره ۱۰۷۵: آینه چینی تو را با زنگی اعشی چه کار}
\label{sec:1075}
\addcontentsline{toc}{section}{\nameref{sec:1075}}
\begin{longtable}{l p{0.5cm} r}
آینه چینی تو را با زنگی اعشی چه کار
&&
کر مادرزاد را با ناله سرنا چه کار
\\
هر مخنث از کجا و ناز معشوق از کجا
&&
طفلک نوزاد را با باده حمرا چه کار
\\
دست زهره در حنی او کی سلحشوری کند
&&
مرغ خاکی را به موج و غره دریا چه کار
\\
بر سر چرخی که عیسی از بلندی بو نبرد
&&
مر خرش را ای مسلمانان بر آن بالا چه کار
\\
قوم رندانیم در کنج خرابات فنا
&&
خواجه ما را با جهاز و مخزن و کالا چه کار
\\
صد هزاران ساله از دیوانگی بگذشته‌ایم
&&
چون تو افلاطون عقلی رو تو را با ما چه کار
\\
با چنین عقل و دل آیی سوی قطاعان راه
&&
تاجر ترسنده را اندر چنین غوغا چه کار
\\
زخم شمشیرست این جا زخم زوبین هر طرف
&&
جمع خاتونان نازک ساق رعنا را چه کار
\\
رستمان امروز اندر خون خود غلطان شدند
&&
زالکان پیر را با قامت دوتا چه کار
\\
عاشقان را منبلان دان زخم خوار و زخم دوست
&&
عاشقان عافیت را با چنین سودا چه کار
\\
عاشقان بوالعجب تا کشته‌تر خود زنده تر
&&
در جهان عشق باقی مرگ را حاشا چه کار
\\
وانگهی این مست عشق اندر هوای شمس دین
&&
رفته تبریز و شنیده رو تو را آن جا چه کار
\\
از ورای هر دو عالم بانگ آید روح را
&&
پس تو را با شمس دین باقی اعلا چه کار
\\
\end{longtable}
\end{center}
