\begin{center}
\section*{غزل شماره ۲۱۵۵: کی ز جهان برون شود جزو جهان هله بگو}
\label{sec:2155}
\addcontentsline{toc}{section}{\nameref{sec:2155}}
\begin{longtable}{l p{0.5cm} r}
کی ز جهان برون شود جزو جهان هله بگو
&&
کی برهد ز آب نم چون بجهد یکی ز دو
\\
هیچ نمیرد آتشی ز آتش دیگر ای پسر
&&
ای دل من ز عشق خون خون مرا به خون مشو
\\
چند گریختم نشد سایه من ز من جدا
&&
سایه بود موکلم گر چه شوم چو تار مو
\\
نیست جز آفتاب را قوت دفع سایه‌ها
&&
بیش کند کمش کند این تو ز آفتاب جو
\\
ور دو هزار سال تو در پی سایه می‌دوی
&&
آخر کار بنگری تو سپسی و پیش او
\\
جرم تو گشت خدمتت رنج تو گشت نعمتت
&&
شمع تو گشت ظلمتت بند تو گشت جست و جو
\\
شرح بدادمی ولی پشت دل تو بشکند
&&
شیشه دل چو بشکنی سود نداردت رفو
\\
سایه و نور بایدت هر دو به هم ز من شنو
&&
سر بنه و دراز شو پیش درخت اتقوا
\\
چون ز درخت لطف او بال و پری برویدت
&&
تن زن چون کبوتران بازمکن بقوبقو
\\
چغز در آب می‌رود مار نمی‌رسد بدو
&&
بانگ زند خبر کند مار بداندش که کو
\\
گر چه که چغز حیله گر بانگ زند چو مار هم
&&
آن دم سست چغزیش بازدهد ز بانگ بو
\\
چغز اگر خمش بدی مار شدی شکار او
&&
چونک به کنج وارود گنج شود جو و تسو
\\
گنج چو شد تسوی زر کم نشود به خاک در
&&
گنج شود تسوی جان چون برسد به گنج هو
\\
ختم کنم بر این سخن یا بفشارمش دگر
&&
حکم تو راست من کیم ای ملک لطیف خو
\\
\end{longtable}
\end{center}
