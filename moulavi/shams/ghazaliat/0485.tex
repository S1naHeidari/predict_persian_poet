\begin{center}
\section*{غزل شماره ۴۸۵: سه روز شد که نگارین من دگرگونست}
\label{sec:0485}
\addcontentsline{toc}{section}{\nameref{sec:0485}}
\begin{longtable}{l p{0.5cm} r}
سه روز شد که نگارین من دگرگونست
&&
شکر ترش نبود آن شکر ترش چونست
\\
به چشمه‌ای که در او آب زندگانی بود
&&
سبو ببردم و دیدم که چشمه پرخونست
\\
به روضه‌ای که در او صد هزار گل می‌رست
&&
به جای میوه و گل خار و سنگ و هامونست
\\
فسون بخوانم و بر روی آن پری بدمم
&&
از آنک کار پری خوان همیشه افسونست
\\
پری من به فسون‌ها زبون شیشه نشد
&&
که کار او ز فسون و فسانه بیرونست
\\
میان ابروی او خشم‌های دیرینه‌ست
&&
گره در ابروی لیلی هلاک مجنونست
\\
بیا بیا که مرا بی‌تو زندگانی نیست
&&
ببین ببین که مرا بی‌تو چشم جیحونست
\\
به حق روی چو ماهت که چشم روشن کن
&&
اگر چه جرم من از جمله خلق افزونست
\\
به گرد خویش برآید دلم که جرمم چیست
&&
از آنک هر سببی با نتیجه مقرونست
\\
ندا همی‌رسدم از نقیب حکم ازل
&&
که گرد خویش مجو کاین سبب نه زان کونست
\\
خدای بخشد و گیرد بیارد و ببرد
&&
که کار او نه به میزان عقل موزونست
\\
بیا بیا که هم اکنون به لطف کن فیکون
&&
بهشت در بگشاید که غیر ممنونست
\\
ز عین خار ببینی شکوفه‌های عجیب
&&
ز عین سنگ ببینی که گنج قارونست
\\
که لطف تا ابدست و از آن هزار کلید
&&
نهان میانه کاف و سفینه نونست
\\
\end{longtable}
\end{center}
