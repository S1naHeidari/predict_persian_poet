\begin{center}
\section*{غزل شماره ۲۴۲۷: گر باغ از او واقف بدی از شاخ تر خون آمدی}
\label{sec:2427}
\addcontentsline{toc}{section}{\nameref{sec:2427}}
\begin{longtable}{l p{0.5cm} r}
گر باغ از او واقف بدی از شاخ تر خون آمدی
&&
ور عقل از او آگه بدی از چشم جیحون آمدی
\\
گر سر برون کردی مهش روزی ز قرص آفتاب
&&
ذره به ذره در هوا لیلی و مجنون آمدی
\\
ور گنج‌های لعل او یک گوشه بر پستی زدی
&&
هر گوشه ویرانه‌ای صد گنج قارون آمدی
\\
نقشی که بر دل می‌زند بر دیده گر پیدا شدی
&&
هر دست و رو ناشسته‌ای چون شیخ ذاالنون آمدی
\\
ور سحر آن کس نیستی کو چشم بندی می‌کند
&&
چون چشم و دل این جسم و تن بر سقف گردون آمدی
\\
ای خواجه نظاره گر تا چند باشد این نظر
&&
ارزان بدی گر زین نظر معشوق بیرون آمدی
\\
مهمان نو آمد ولی این لوت عالم را بس است
&&
دو کون اگر مهمان شدی این لوت افزون آمدی
\\
\end{longtable}
\end{center}
