\begin{center}
\section*{غزل شماره ۱۵۰۰: چنان مست است از آن دم جان آدم}
\label{sec:1500}
\addcontentsline{toc}{section}{\nameref{sec:1500}}
\begin{longtable}{l p{0.5cm} r}
چنان مست است از آن دم جان آدم
&&
که نشناسد از آن دم جان آدم
\\
ز شور اوست چندین جوش دریا
&&
ز سرمستی او مست است عالم
\\
زهی سرده که گردن زد اجل را
&&
که تا دنیا نبیند هیچ ماتم
\\
شراب حق حلال اندر حلال است
&&
می خنب خدا نبود محرم
\\
از این باده جوان گر خورده بودی
&&
نبودی پشت پیر چرخ را خم
\\
زمین ار خورده بودی فارغستی
&&
از آنک ابر تر بارد بر او نم
\\
دل محرم بیان این بگفتی
&&
اگر بودی به عالم نیم محرم
\\
ز آب و گل برون بردی شما را
&&
اگر بودی شما را پای محکم
\\
رسید این عشق تا پای شما را
&&
کند محکم ز هر سستی مسلم
\\
بگو باقی تو شمس الدین تبریز
&&
که بر تو ختم شد والله اعلم
\\
\end{longtable}
\end{center}
