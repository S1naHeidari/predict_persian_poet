\begin{center}
\section*{غزل شماره ۱۶۳۱: در فروبند که ما عاشق این میکده‌ایم}
\label{sec:1631}
\addcontentsline{toc}{section}{\nameref{sec:1631}}
\begin{longtable}{l p{0.5cm} r}
در فروبند که ما عاشق این میکده‌ایم
&&
درده آن باده جان را که سبک دل شده‌ایم
\\
برجه ای ساقی چالاک میان را بربند
&&
به خدا کز سفر دور و دراز آمده‌ایم
\\
برگشا مشک طرب را که ز رشک کف تو
&&
از کف زهره به صد لابه قدح نستده‌ایم
\\
در فروبند و ز رحمت در پنهان بگشا
&&
چاره رطل گران کن که همه می زده‌ایم
\\
زان سبو غسل قیامت بده از وسوسه‌ام
&&
به حق آنک ز آغاز حریفان بده‌ایم
\\
ما همه خفته تو بر ما لگدی چند زدی
&&
برجهیدیم خمارانه در این عربده‌ایم
\\
گر علی الریق تو را باده دهی قاعده نیست
&&
هین بده ما ملک الموت چنین قاعده‌ایم
\\
فلسفی زین بخورد فلسفه‌اش غرق شود
&&
که گمان داشت که ما زان علل فاسده‌ایم
\\
آن نهنگیم که دریا بر ما یک قدح است
&&
ما نه مردان ثرید و عدس و مایده‌ایم
\\
هله خاموش کن و فایده و فضل بهل
&&
که ز فضله فایده فایده‌ایم
\\
\end{longtable}
\end{center}
