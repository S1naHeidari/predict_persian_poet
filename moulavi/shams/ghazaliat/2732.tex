\begin{center}
\section*{غزل شماره ۲۷۳۲: ای ساقی باده معانی}
\label{sec:2732}
\addcontentsline{toc}{section}{\nameref{sec:2732}}
\begin{longtable}{l p{0.5cm} r}
ای ساقی باده معانی
&&
درده تو شراب ارغوانی
\\
زان باده پیر تلخ پاسخ
&&
بفزای حلاوت جوانی
\\
در بزم سرای شاه جانان
&&
نظاره شاهدان جانی
\\
جان‌ها بینی چو روز روشن
&&
از لذت عشرت شبانی
\\
بینی که جهان به حیرت آید
&&
در حلقه خلق آن جهانی
\\
مه را ز فلک فروفرستد
&&
در مجلسشان به ارمغانی
\\
و آن زهره نوای خوش برآورد
&&
کو مطرب کیست آسمانی
\\
این‌ها به همند و ما به خلوت
&&
با دلبر خوب پرمعانی
\\
رخ بر رخ ما نهاد آن شه
&&
و آن باقی را تو خود بدانی
\\
آن شاه کیست شمس تبریز
&&
آن خسرو ملک بی‌نشانی
\\
\end{longtable}
\end{center}
