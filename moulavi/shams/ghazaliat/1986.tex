\begin{center}
\section*{غزل شماره ۱۹۸۶: صنما به چشم شوخت که به چشم اشارتی کن}
\label{sec:1986}
\addcontentsline{toc}{section}{\nameref{sec:1986}}
\begin{longtable}{l p{0.5cm} r}
صنما به چشم شوخت که به چشم اشارتی کن
&&
نفسی خراب خود را به نظر عمارتی کن
\\
دل و جان شهید عشقت به درون گور قالب
&&
سوی گور این شهیدان بگذر زیارتی کن
\\
تو چو یوسفی رسیده همه مصر کف بریده
&&
بنما جمال و بستان دل و جان تجارتی کن
\\
و اگر قدم فشردی به جفا و نذر کردی
&&
بشکن تو نذر خود را چه شود کفارتی کن
\\
تو مگو کز این نثارم ز شما چه سود دارم
&&
تو ز سود بی‌نیازی بده و خسارتی کن
\\
رخ همچو زعفران را چو گل و چو لاله گردان
&&
سه چهار قطره خون را دل بابشارتی کن
\\
چو غلام توست دولت نکشد ز امر تو سر
&&
به میان ما و دولت ملکا سفارتی کن
\\
چو به پیش کوه حلمت گنهان چو کاه آمد
&&
به گناه چون که ما نظر حقارتی کن
\\
تن ما دو قطره خون بد که نظیف و آدمی شد
&&
صفت پلید را هم صفت طهارتی کن
\\
ز جهان روح جان‌ها چو اسیر آب و گل شد
&&
تو ز دار حرب گلشان برهان و غارتی کن
\\
چو ز حرف توبه کردم تو برای طالبان را
&&
جز حرف پرمعانی علم و امارتی کن
\\
ز برای گرم کردن بود این دم چو آتش
&&
جز دم تو تابشی را سبب حرارتی کن
\\
تو که شاه شمس دینی تبریز نازنین را
&&
به ظهور نیر خود وطن بصارتی کن
\\
\end{longtable}
\end{center}
