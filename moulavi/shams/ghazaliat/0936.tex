\begin{center}
\section*{غزل شماره ۹۳۶: مرا عقیق تو باید شکر چه سود کند}
\label{sec:0936}
\addcontentsline{toc}{section}{\nameref{sec:0936}}
\begin{longtable}{l p{0.5cm} r}
مرا عقیق تو باید شکر چه سود کند
&&
مرا جمال تو باید قمر چه سود کند
\\
چو مست چشم تو نبود شراب را چه طرب
&&
چو همرهم تو نباشی سفر چه سود کند
\\
مرا زکات تو باید خزینه را چه کنم
&&
مرا میان تو باید کمر چه سود کند
\\
چو یوسفم تو نباشی مرا به مصر چه کار
&&
چو رفت سایه سلطان حشر چه سود کند
\\
چو آفتاب تو نبود ز آفتاب چه نور
&&
چو منظرم تو نباشی نظر چه سود کند
\\
لقای تو چو نباشد بقای عمر چه سود
&&
پناه تو چو نباشد سپر چه سود کند
\\
شبم چو روز قیامت دراز گشت ولی
&&
دلم سحور تو خواهد سحر چه سود کند
\\
شبی که ماه نباشد ستارگان چه زنند
&&
چو مرغ را نبود سر دو پر چه سود کند
\\
چو زور و زهره نباشد سلاح و اسب چه سود
&&
چو دل دلی ننماید جگر چه سود کند
\\
چو روح من تو نباشی ز روح ریح چه سود
&&
بصیرتم چو نبخشی بصر چه سود کند
\\
مرا به جز نظر تو نبود و نیست هنر
&&
عنایتت چو نباشد هنر چه سود کند
\\
جهان مثال درختست برگ و میوه ز توست
&&
چو برگ و میوه نباشد شجر چه سود کند
\\
گذر کن از بشریت فرشته باش دلا
&&
فرشتگی چو نباشد بشر چه سود کند
\\
خبر چو محرم او نیست بی‌خبر شو و مست
&&
چو مخبرش تو نباشی خبر چه سود کند
\\
ز شمس مفخر تبریز آنک نور نیافت
&&
وجود تیره او را دگر چه سود کند
\\
\end{longtable}
\end{center}
