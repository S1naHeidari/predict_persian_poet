\begin{center}
\section*{غزل شماره ۶۳۹: آن سرخ قبایی که چو مه پار برآمد}
\label{sec:0639}
\addcontentsline{toc}{section}{\nameref{sec:0639}}
\begin{longtable}{l p{0.5cm} r}
آن سرخ قبایی که چو مه پار برآمد
&&
امسال در این خرقه زنگار برآمد
\\
آن ترک که آن سال به یغماش بدیدی
&&
آنست که امسال عرب وار برآمد
\\
آن یار همانست اگر جامه دگر شد
&&
آن جامه به در کرد و دگربار برآمد
\\
آن باده همانست اگر شیشه بدل شد
&&
بنگر که چه خوش بر سر خمار برآمد
\\
ای قوم گمان برده که آن مشعله‌ها مرد
&&
آن مشعله زین روزن اسرار برآمد
\\
این نیست تناسخ سخن وحدت محضست
&&
کز جوشش آن قلزم زخار برآمد
\\
یک قطره از آن بحر جدا شد که جدا نیست
&&
کآدم ز تک صلصل فخار برآمد
\\
رومی پنهان گشت چو دوران حبش دید
&&
امروز در این لشکر جرار برآمد
\\
گر شمس فروشد به غروب او نه فنا شد
&&
از برج دگر آن مه انوار برآمد
\\
گفتار رها کن بنگر آینه عین
&&
کان شبهه و اشکال ز گفتار برآمد
\\
شمس الحق تبریز رسیدست مگویید
&&
کز چرخ صفا آن مه اسرار برآمد
\\
\end{longtable}
\end{center}
