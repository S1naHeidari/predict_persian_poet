\begin{center}
\section*{غزل شماره ۲۹۹۴: ای نای خوش نوای که دلدار و دلخوشی}
\label{sec:2994}
\addcontentsline{toc}{section}{\nameref{sec:2994}}
\begin{longtable}{l p{0.5cm} r}
ای نای خوش نوای که دلدار و دلخوشی
&&
دم می‌دهی تو گرم و دم سرد می‌کشی
\\
خالی است اندرون تو از بند لاجرم
&&
خالی کننده دل و جان مشوشی
\\
نقشی کنی به صورت معشوق هر کسی
&&
هر چند امیی تو به معنی منقشی
\\
ای صورت حقایق کل در چه پرده‌ای
&&
سر برزن از میانه نی چون شکروشی
\\
نه چشم گشته‌ای تو و ده گوش گشته جان
&&
دردم به شش جهت که تو دمساز هر ششی
\\
ای نای سربریده بگو سر بی‌زبان
&&
خوش می‌چشان ز حلق از آن دم که می‌چشی
\\
آتش فتاد در نی و عالم گرفت دود
&&
زیرا ندای عشق ز نی هست آتشی
\\
بنواز سر لیلی و مجنون ز عشق خویش
&&
دل را چه لذتی تو و جان را چه مفرشی
\\
بویی است در دم تو ز تبریز لاجرم
&&
بس دل که می‌ربایی از حسن و از کشی
\\
\end{longtable}
\end{center}
