\begin{center}
\section*{غزل شماره ۲۴۸۱: با همگان فضولکی چون که به ما ملولکی}
\label{sec:2481}
\addcontentsline{toc}{section}{\nameref{sec:2481}}
\begin{longtable}{l p{0.5cm} r}
با همگان فضولکی چون که به ما ملولکی
&&
رو که بدین عاشقی سخت عظیم گولکی
\\
ای تو فضول در هوا ای تو ملول در خدا
&&
چون تو از آن قان نه‌ای رو که یکی مغولکی
\\
مستک خویش گشته‌ای گه ترشک گهی خوشک
&&
نازک و کبرکت که چه در هنرک نغولکی
\\
گر تو کتاب خانه‌ای طالب باغ جان نه‌ای
&&
گر چه اصیلکی ولی خواجه تو بی‌اصولکی
\\
رو تو به کیمیای جان مس وجود خرج کن
&&
تا نشوی از او چو زر در غم نیم پولکی
\\
گفتم با ضمیر خود چند خیال جسمیان
&&
یا تو ز هر فسرده‌ای سوی دلم رسولکی
\\
نور خدایگان جان در تبریز شمس دین
&&
کرد طریق سالکان ایمن اگر تو غولکی
\\
\end{longtable}
\end{center}
