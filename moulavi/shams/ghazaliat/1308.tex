\begin{center}
\section*{غزل شماره ۱۳۰۸: گر خمار آرد صداعی بر سر سودای عشق}
\label{sec:1308}
\addcontentsline{toc}{section}{\nameref{sec:1308}}
\begin{longtable}{l p{0.5cm} r}
گر خمار آرد صداعی بر سر سودای عشق
&&
دررسد در حین مدد از ساقی صهبای عشق
\\
ور بدرد طبل شادی لشکر عشاق را
&&
مژده انافتحنا دردمد سرنای عشق
\\
زهر اندر کام عاشق شهد گردد در زمان
&&
زان شکرهایی که روید هر دم از نی‌های عشق
\\
یک زمان ابری بیاید تا بپوشد ماه را
&&
ابر را در حین بسوزد برق جان افزای عشق
\\
در میان ریگ سوزان در طریق بادیه
&&
بانگ‌های رعد بینی می‌زند سقای عشق
\\
ساقیا از بهر جانت ساغری بر خلق ریز
&&
یا صلا درده به سوی قامت و بالای عشق
\\
شمس تبریز ار بتاند از قباب رشک حق
&&
قبه‌های موج خیزد آن دم از دریای عشق
\\
\end{longtable}
\end{center}
