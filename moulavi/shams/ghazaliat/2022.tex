\begin{center}
\section*{غزل شماره ۲۰۲۲: ای زیان و ای زیان و ای زیان}
\label{sec:2022}
\addcontentsline{toc}{section}{\nameref{sec:2022}}
\begin{longtable}{l p{0.5cm} r}
ای زیان و ای زیان و ای زیان
&&
هوشیاری در میان مستیان
\\
گر بیاید هوشیاری راه نیست
&&
ور بیاید مست گیر اندرکشان
\\
گر خماری باده خواهی اندرآ
&&
نان پرستی رو که این جا نیست نان
\\
آنک او نان را بت خود کرده است
&&
کی درآید در میان این بتان
\\
ور درآید چادر اندر رو کشند
&&
تا نبیند رویشان آن قلتبان
\\
سیمبر خواهیم و زیبا همچو خویش
&&
سیم نستانیم پیدا و نهان
\\
آنک او خوبی به سیم و زر فروخت
&&
روسپی باشد نه حوران جنان
\\
تا نگردی پاک دل چون جبرئیل
&&
گر چه گنجی درنگنجی در جهان
\\
چشم خود را شسته عارف بیست سال
&&
مشک مشک آورده از اشک روان
\\
معتمد شو تا درآیی در حرم
&&
اولا بربند از گفتن دهان
\\
شمس تبریزی گشاید راه شرق
&&
چون شوی بسته دهان و رازدان
\\
\end{longtable}
\end{center}
