\begin{center}
\section*{غزل شماره ۲۳۶۶: ای سراندازان همه در عشق تو پا کوفته}
\label{sec:2366}
\addcontentsline{toc}{section}{\nameref{sec:2366}}
\begin{longtable}{l p{0.5cm} r}
ای سراندازان همه در عشق تو پا کوفته
&&
گوهر جان همچو موسی روی دریا کوفته
\\
زیر این هفت آسیا هستی ما را خوش بکوب
&&
روشنایی کی فزاید سرمه ناکوفته
\\
عاشقان با عاقلان اندرنیامیزد از آنک
&&
درنیامیزد کسی ناکوفته با کوفته
\\
عاقلان از مور مرده درکشند از احتیاط
&&
عاشقان از لاابالی اژدها را کوفته
\\
مردم چشم از خیالت چون شود پی کوب عشق
&&
فرق‌ها پیدا شود از کوفته تا کوفته
\\
از شکار تو به بیشه جان شیران خون شده
&&
در هوای قاف قربت پر عنقا کوفته
\\
عشق چون خورشید دامن گستریده بر زمین
&&
عاشقان چون اخترانش راه بالا کوفته
\\
لا چو لالایان زده بر عاشقانش دست رد
&&
غیرت الا شده بر مغز لالا کوفته
\\
حاجیان راه جان خسته نگردند از نشاط
&&
اشترانشان زیر بار از راه اعضا کوفته
\\
ساربان این غزل گو تا ز بعد خستگی
&&
اشتران را مست بینی راه بطحا کوفته
\\
\end{longtable}
\end{center}
