\begin{center}
\section*{غزل شماره ۱۰۶۹: در سماع عاشقان زد فر و تابش بر اثیر}
\label{sec:1069}
\addcontentsline{toc}{section}{\nameref{sec:1069}}
\begin{longtable}{l p{0.5cm} r}
در سماع عاشقان زد فر و تابش بر اثیر
&&
گر سماع منکران اندرنگیرد گو مگیر
\\
قسمت حقست قومی در میان آفتاب
&&
پای کوبانند و قومی در میان زمهریر
\\
قسمت حقست قومی در میان آب شور
&&
تلخ و غمگینند و قومی در میان شهد و شیر
\\
نوبت الفقر فخری تا قیامت می‌زنند
&&
تو که داری می‌خور و می‌ده شب و روز ای فقیر
\\
فقر را در نور یزدان جو مجو اندر پلاس
&&
هر برهنه مرد بودی مرد بودی نیز سیر
\\
بانگ مرغان می‌رسد بر می‌فشانی پر و بال
&&
لیک اگر خواهی بپری پای را برکش ز قیر
\\
عقل تو دربند جان و طبع تو دربند نان
&&
مغزها اندر خمار و دست‌ها اندر خمیر
\\
عارفا گر کاهلی آمد قران کاهلان
&&
جاء نصرالله آمد ابشروا جاء البشیر
\\
گرمی خود را دگر جا خرج کردی ای جوان
&&
هر کی آن جا گرم باشد این طرف باشد زحیر
\\
گرمیی با سردیی و سردیی با گرمیی
&&
چونک آن جا گرم بودی سردی این جا ناگزیر
\\
لیک نومیدی رها کن گرمی حق بی‌حدست
&&
پیش این خورشید گرمی ذره‌ای باشد سعیر
\\
همچو مغناطیس می‌کش طالبان را بی‌زبان
&&
بس بود بسیار گفتی ای نذیر بی‌نظیر
\\
\end{longtable}
\end{center}
