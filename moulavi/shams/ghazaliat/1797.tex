\begin{center}
\section*{غزل شماره ۱۷۹۷: ای دل شکایت‌ها مکن تا نشنود دلدار من}
\label{sec:1797}
\addcontentsline{toc}{section}{\nameref{sec:1797}}
\begin{longtable}{l p{0.5cm} r}
ای دل شکایت‌ها مکن تا نشنود دلدار من
&&
ای دل نمی‌ترسی مگر از یار بی‌زنهار من
\\
ای دل مرو در خون من در اشک چون جیحون من
&&
نشنیده‌ای شب تا سحر آن ناله‌های زار من
\\
یادت نمی‌آید که او می کرد روزی گفت گو
&&
می گفت بس دیگر مکن اندیشه گلزار من
\\
اندازه خود را بدان نامی مبر زین گلستان
&&
این بس نباشد خود تو را کآگه شوی از خار من
\\
گفتم امانم ده به جان خواهم که باشی این زمان
&&
تو سرده و من سرگران ای ساقی خمار من
\\
خندید و می گفت ای پسر آری ولیک از حد مبر
&&
وانگه چنین می کرد سر کای مست و ای هشیار من
\\
چون لطف دیدم رای او افتادم اندر پای او
&&
گفتم نباشم در جهان گر تو نباشی یار من
\\
گفتا مباش اندر جهان تا روی من بینی عیان
&&
خواهی چنین گم شو چنان در نفی خود دان کار من
\\
گفتم منم در دام تو چون گم شوم بی‌جام تو
&&
بفروش یک جامم به جان وانگه ببین بازار من
\\
\end{longtable}
\end{center}
