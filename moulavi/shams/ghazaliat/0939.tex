\begin{center}
\section*{غزل شماره ۹۳۹: به پیش تو چه زند جان و جان کدام بود}
\label{sec:0939}
\addcontentsline{toc}{section}{\nameref{sec:0939}}
\begin{longtable}{l p{0.5cm} r}
به پیش تو چه زند جان و جان کدام بود
&&
که جان تویی و دگر جمله نقش و نام بود
\\
اگر چه ماه به ده دست روی خود شوید
&&
چه زهره دارد کان چهره را غلام بود
\\
اگر چه عاشقی و عشق بهترین کار است
&&
بدانک بی‌رخ معشوق ما حرام بود
\\
به جان عشق که تا هر دو جان نیامیزد
&&
جداییست و ملاقات بی‌نظام بود
\\
شراب لطف خداوند را کرانی نیست
&&
وگر کرانه نماید قصور جام بود
\\
به قدر روزنه افتد به خانه نور قمر
&&
اگر به مشرق و مغرب ضیاش عام بود
\\
تو جام هستی خود را برو قوامی ده
&&
که آن شراب قدیمست و باقوام بود
\\
هزار جان طلبید و یکی ببردم پیش
&&
بگفت باقی گفتم بهل که وام بود
\\
رفیق گشته دو چشمش میان خوف و رجا
&&
برای پختن هر عاشقی که خام بود
\\
هزار خانه به تاراج برد و خوش قنقیست
&&
سلامتی همه تاراج آن سلام بود
\\
درون خانه بود نقش‌ها نه آن نقاش
&&
به سوی بام نگر کان قمر به بام بود
\\
رسید مژده به شامست شمس تبریزی
&&
چه صبح‌ها که نماید اگر به شام بود
\\
\end{longtable}
\end{center}
