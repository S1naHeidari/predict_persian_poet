\begin{center}
\section*{غزل شماره ۱۸۲۷: دوش چه خورده‌ای دلا راست بگو نهان مکن}
\label{sec:1827}
\addcontentsline{toc}{section}{\nameref{sec:1827}}
\begin{longtable}{l p{0.5cm} r}
دوش چه خورده‌ای دلا راست بگو نهان مکن
&&
چون خمشان بی‌گنه روی بر آسمان مکن
\\
باده خاص خورده‌ای نقل خلاص خورده‌ای
&&
بوی شراب می زند خربزه در دهان مکن
\\
روز الست جان تو خورد میی ز خوان تو
&&
خواجه لامکان تویی بندگی مکان مکن
\\
دوش شراب ریختی وز بر ما گریختی
&&
بار دگر گرفتمت بار دگر چنان مکن
\\
من همگی تراستم مست می وفاستم
&&
با تو چو تیر راستم تیر مرا کمان مکن
\\
ای دل پاره پاره‌ام دیدن او است چاره‌ام
&&
او است پناه و پشت من تکیه بر این جهان مکن
\\
ای همه خلق نای تو پر شده از نوای تو
&&
گر نه سماع باره‌ای دست به نای جان مکن
\\
نفخ نفخت کرده‌ای در همه دردمیده‌ای
&&
چون دم توست جان نی بی‌نی ما فغان مکن
\\
کار دلم به جان رسد کارد به استخوان رسد
&&
ناله کنم بگویدم دم مزن و بیان مکن
\\
ناله مکن که تا که من ناله کنم برای تو
&&
گرگ تویی شبان منم خویش چو من شبان مکن
\\
هر بن بامداد تو جانب ما کشی سبو
&&
کای تو بدیده روی من روی به این و آن مکن
\\
شیر چشید موسی از مادر خویش ناشتا
&&
گفت که مادرت منم میل به دایگان مکن
\\
باده بنوش مات شو جمله تن حیات شو
&&
باده چون عقیق بین یاد عقیق کان مکن
\\
باده عام از برون باده عارف از درون
&&
بوی دهان بیان کند تو به زبان بیان مکن
\\
از تبریز شمس دین می رسدم چو ماه نو
&&
چشم سوی چراغ کن سوی چراغدان مکن
\\
\end{longtable}
\end{center}
