\begin{center}
\section*{غزل شماره ۵۱۵: خانه دل باز کبوتر گرفت}
\label{sec:0515}
\addcontentsline{toc}{section}{\nameref{sec:0515}}
\begin{longtable}{l p{0.5cm} r}
خانه دل باز کبوتر گرفت
&&
مشغله و بقر بقو درگرفت
\\
غلغل مستان چو به گردون رسید
&&
کرکس زرین فلک پر گرفت
\\
بوطربون گشت مه و مشتری
&&
زهره مطرب طرب از سر گرفت
\\
خالق ارواح ز آب و ز گل
&&
آینه‌ای کرد و برابر گرفت
\\
ز آینه صد نقش شد و هر یکی
&&
آنچ مر او راست میسر گرفت
\\
هر که دلی داشت به پایش فتاد
&&
هر که سر او سر منبر گرفت
\\
خرمن ارواح نهایت نداشت
&&
مورچه‌ای چیز محقر گرفت
\\
گر ز تو پر گشت جهان همچو برف
&&
نیست شوی چون تف خود درگرفت
\\
نیست شو ای برف و همه خاک شو
&&
بنگر کاین خاک چه زیور گرفت
\\
خاک به تدریج بدان جا رسید
&&
کز فر او هر دو جهان فر گرفت
\\
بس که زبان این دم معزول شد
&&
بس که جهان جان سخنور گرفت
\\
\end{longtable}
\end{center}
