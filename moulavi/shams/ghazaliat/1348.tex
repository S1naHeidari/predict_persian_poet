\begin{center}
\section*{غزل شماره ۱۳۴۸: امروز روز شادی و امسال سال گل}
\label{sec:1348}
\addcontentsline{toc}{section}{\nameref{sec:1348}}
\begin{longtable}{l p{0.5cm} r}
امروز روز شادی و امسال سال گل
&&
نیکوست حال ما که نکو باد حال گل
\\
گل را مدد رسید ز گلزار روی دوست
&&
تا چشم ما نبیند دیگر زوال گل
\\
مستست چشم نرگس و خندان دهان باغ
&&
از کر و فر و رونق و لطف و کمال گل
\\
سوسن زبان گشاده و گفته به گوش سرو
&&
اسرار عشق بلبل و حسن خصال گل
\\
جامه دران رسید گل از بهر داد ما
&&
زان می‌دریم جامه به بوی وصال گل
\\
گل آن جهانیست نگنجد در این جهان
&&
در عالم خیال چه گنجد خیال گل
\\
گل کیست قاصدیست ز بستان عقل و جان
&&
گل چیست رقعه ایست ز جاه و جمال گل
\\
گیریم دامن گل و همراه گل شویم
&&
رقصان همی‌رویم به اصل و نهال گل
\\
اصل و نهال گل عرق لطف مصطفاست
&&
زان صدر بدر گردد آن جا هلال گل
\\
زنده کنند و باز پر و بال نو دهند
&&
هر چند برکنید شما پر و بال گل
\\
مانند چار مرغ خلیل از پی فنا
&&
در دعوت بهار ببین امتثال گل
\\
خاموش باش و لب مگشا خواجه غنچه وار
&&
می‌خند زیر لب تو به زیر ظلال گل
\\
\end{longtable}
\end{center}
