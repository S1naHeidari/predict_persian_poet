\begin{center}
\section*{غزل شماره ۲۷۷۸: ای خدایی که مفرح بخش رنجوران تویی}
\label{sec:2778}
\addcontentsline{toc}{section}{\nameref{sec:2778}}
\begin{longtable}{l p{0.5cm} r}
ای خدایی که مفرح بخش رنجوران تویی
&&
در میان لطف و رحمت همچو جان پنهان تویی
\\
خسته کردی بندگان را تا تو را زاری کنند
&&
چون خریدار نفیر و لابه و افغان تویی
\\
جمله درمان خواه و آن درمانشان خواهان توست
&&
آنک درد و دارو از وی خاست بی‌شک آن تویی
\\
دردهایی کآدمی را بر در خلقان برد
&&
آن حجاب از اول است و آخر و پایان تویی
\\
هر کجا کاری فروبندد تو باشی چشم بند
&&
هر کجا روشن شود آن شعله تابان تویی
\\
ناله بخشی خستگان را تا بدان ساکن شوند
&&
چون حقیقت بنگرم در درد ما نالان تویی
\\
هم تویی آن کس که می‌گوید تویی والله تویی
&&
گوی و چوگان و نظاره گر در این میدان تویی
\\
و آنک منکر می‌شود این را و علت می‌نهد
&&
در میان وسوسه او نفس علت خوان تویی
\\
و آنک می‌گوید تویی زین گفت ترسان می‌شود
&&
در میان جان او در پرده ترسان تویی
\\
کنج زندان را به یک اندیشه بستان می‌کنی
&&
رنج هر زندان ز توست و ذوق هر بستان تویی
\\
در یکی کار آن یکی راغب و آن دیگر نفور
&&
تو مخالف کرده‌ای شان فتنه ایشان تویی
\\
آن یکی محبوب این و باز او مکروه آن
&&
چشم بندی چشم و دل را قبله و سامان تویی
\\
صد هزاران نقش را تو بنده نقشی کنی
&&
گویی سلطان است آن دام است خود سلطان تویی
\\
بندگی و خواجگی و سلطنت خط‌های توست
&&
خط کژ و خط راست این دبیرستان تویی
\\
صورت ما خانه‌ها و روح ما مهمان در آن
&&
نقش و جان‌ها سایه تو جان آن مهمان تویی
\\
دست در طاعت زنیم و چشم در ایمان نهیم
&&
بر امید آنک بنمایی که خود ایمان تویی
\\
دست احسان بر سر ما نه ز احسانی که ما
&&
چشم روشن در تو آویزیم کان احسان تویی
\\
غفلت و بیداری ما در توی بر کار و بس
&&
غفلت ما بی‌فضولی بر چو خود یقضان تویی
\\
توبه با تو خود فضول است و شکستن خود بتر
&&
نقش پیمان گر شکست ارواح آن پیمان تویی
\\
روح‌ها می‌پروری همچون زر و مس و عقیق
&&
چون مخالف شد جواهر ای عجب چون کان تویی
\\
روز درپیچد صفت در ما و تابد تا به شب
&&
شب صفات از ما به تو آید صفاتستان تویی
\\
روز تا شب ما چنین بر همدگر رحمت کنیم
&&
شب همه رحمت رود سوی تو چون رحمان تویی
\\
کو سلاطین جهان گر شاه ایوان بوده‌اند
&&
پس بدانستیم بی‌شک کاندر این ایوان تویی
\\
\end{longtable}
\end{center}
