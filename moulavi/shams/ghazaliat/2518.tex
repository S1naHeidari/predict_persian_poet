\begin{center}
\section*{غزل شماره ۲۵۱۸: اگر امروز دلدارم کند چون دوش بدمستی}
\label{sec:2518}
\addcontentsline{toc}{section}{\nameref{sec:2518}}
\begin{longtable}{l p{0.5cm} r}
اگر امروز دلدارم کند چون دوش بدمستی
&&
درافتد در جهان غوغا درافتد شور در هستی
\\
الا ای عقل شوریده بد و نیک جهان دیده
&&
که امروز است دست خون اگر چه دوش از او رستی
\\
درآمد ترک در خرگه چه جای ترک قرص مه
&&
کی دیده است ای مسلمانان مه گردون در این پستی
\\
چو گرد راه هین برجه هلا پا دار و گردن نه
&&
که مردن پیش دلبر به تو را زین عمر سردستی
\\
برو بی‌سر به میخانه بخور بی‌رطل و پیمانه
&&
کز این خم جهان چون می بجوشیدی برون جستی
\\
غلام و خاک آن مستم که شد هم جام و هم دستم
&&
غلامش چون شوی ای دل که تو خود عین آنستی
\\
چه غم داری در این وادی چو روی یوسفان دیدی
&&
اگر چه چون زنان حیران ز خنجر دست خود خستی
\\
منال ای دست از این خنجر چو در کف آمدت گوهر
&&
هزاران درد زه ارزد ز عشق یوسف آبستی
\\
خمش کن ای دل دریا از این جوش و کف اندازی
&&
زهی طرفه که دریایی چو ماهی چون در این شستی
\\
چه باشد شست روباهان به پیش پنجه شیران
&&
بدران شست اگر خواهی برو در بحر پیوستی
\\
نمی‌دانی که سلطانی تو عزرائیل شیرانی
&&
تو آن شیر پریشانی که صندوق خود اشکستی
\\
عجب نبود که صندوقی شکسته گردد از شیری
&&
عجب از چون تو شیر آید که در صندوق بنشستی
\\
خمش کردم درآ ساقی بگردان جام راواقی
&&
زهی دوران و دور ما که بهر ما میان بستی
\\
\end{longtable}
\end{center}
