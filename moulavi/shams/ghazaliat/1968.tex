\begin{center}
\section*{غزل شماره ۱۹۶۸: ای دل من در هوایت همچو آب و ماهیان}
\label{sec:1968}
\addcontentsline{toc}{section}{\nameref{sec:1968}}
\begin{longtable}{l p{0.5cm} r}
ای دل من در هوایت همچو آب و ماهیان
&&
ماهی جانم بمیرد گر بگردی یک زمان
\\
ماهیان را صبر نبود یک زمان بیرون آب
&&
عاشقان را صبر نبود در فراق دلستان
\\
جان ماهی آب باشد صبر بی‌جان چون بود
&&
چونک بی‌جان صبر نبود چون بود بی‌جان جان
\\
هر دو عالم بی‌جمالت مر مرا زندان بود
&&
آب حیوان در فراقت گر خورم دارد زیان
\\
این نگارستان عالم پرنشان و نقش توست
&&
لیک جای تو نگیرد کو نشان کو بی‌نشان
\\
قطره خون دلم را چون جهانی کرده‌ای
&&
تا ز حیرانی ندانم قطره‌ای را از جهان
\\
بر دهان من به دست خویش بنهادی قدح
&&
تا ز سرمستی ندانم من قدح را از دهان
\\
من کی باشم از زمین تا آسمان مستان پرند
&&
کز شراب تو ندانند از زمین تا آسمان
\\
صد شبان چون من سپرده گوسفند خود به گرگ
&&
گوسفندان را چه کردی با کی گویم کو شبان
\\
در بیان آرم نیایی ور نهان دارم بتر
&&
درنگنجی از بزرگی در جهان و در نهان
\\
گر نهان را می شناسم از جهان در عاشقی
&&
مؤمن عشقم مخوان و کافرم خوان ای فلان
\\
\end{longtable}
\end{center}
