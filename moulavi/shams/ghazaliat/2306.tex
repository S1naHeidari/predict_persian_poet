\begin{center}
\section*{غزل شماره ۲۳۰۶: ناموس مکن پیش آ ای عاشق بیچاره}
\label{sec:2306}
\addcontentsline{toc}{section}{\nameref{sec:2306}}
\begin{longtable}{l p{0.5cm} r}
ناموس مکن پیش آ ای عاشق بیچاره
&&
تا مرد نظر باشی نی مردم نظاره
\\
ای عاشق الاهو ز استاره بگیر این خو
&&
خورشید چو درتابد فانی شود استاره
\\
آن‌ها که قوی دستند دست تو چرا بستند
&&
زیرا تو کنون طفلی وین عالم گهواره
\\
چون در سخن‌ها سفت و الارض مهادا گفت
&&
ای میخ زمین گشته وز شهر دل آواره
\\
ای بنده شیر تن هستی تو اسیر تن
&&
دندان خرد بنما نعمت خور همواره
\\
تا طفل بود سلطان دایه کندش زندان
&&
تا شیر خورد ز ایشان نبود شه میخواره
\\
از سنگ سبو ترسد اما چو شود چشمه
&&
هر لحظه سبو آید تازان به سوی خاره
\\
گوید که اگر زین پس او بشکندم شادم
&&
جان داد مرا آبش یک باره و صد باره
\\
گر در ره او مردم هم زنده بدو گردم
&&
خود پاره دهم او را تا او کندم پاره
\\
\end{longtable}
\end{center}
