\begin{center}
\section*{غزل شماره ۳۱۰۳: پدید گشت یکی آهوی در این وادی}
\label{sec:3103}
\addcontentsline{toc}{section}{\nameref{sec:3103}}
\begin{longtable}{l p{0.5cm} r}
پدید گشت یکی آهوی در این وادی
&&
به چشم آتش افکند در همه نادی
\\
همه سوار و پیاده طلب درافتادند
&&
بجهد و جد نه چون تو که سست افتادی
\\
چو یک دو حمله دویدند ناپدید شد او
&&
که هیچ بوی نبردی کسی به استادی
\\
لگام‌ها بکشیدند تا که واگردند
&&
نمود باز بدیشان فزودشان شادی
\\
چو باز حمله بکردند باز تک برداشت
&&
که باد در پی او گم کند همی‌بادی
\\
بر این صفت چو ز حد رفت هر کسی ز هوس
&&
ز هم شدند جدا و بکرد وحادی
\\
یکی به تک دم خرگوش برگرفت غلط
&&
یکی پی بز کوهی و راه بغدادی
\\
گروه گمشده با همدگر دو قسم شدند
&&
یکی به طمع در آهو یکی به آزادی
\\
جماعتی که بدیشانست میل آن آهو
&&
چو گم شدندی بنمودی آهو آبادی
\\
از این جماعت قومی که خاصتر بودند
&&
به چشم مست بیاموختشان هم اورادی
\\
چو خو و طبع ورا خوبتر بدانستند
&&
ز طبع او نشدندی به هیچ رو عادی
\\
جمال خویش چو بنمودشان ز رحمت خود
&&
که اندک اندک گستاخ کردشان هادی
\\
به هر دو روز یکی شکل دیگر آوردی
&&
به شکل‌های عجایب مثال شیادی
\\
ازانک زهره بدرد دل ضعیفان را
&&
چه تاب دارد خود جان آدمیزادی
\\
که آسمان و زمین بردرد اگر بیند
&&
یکی صفت ز صفت‌های مبدی بادی
\\
که باشد آنک بگفتم خیال شمس الدین
&&
که او مراست خدیو و مجیر بیدادی
\\
ز عشق او نتوانم که توبه آرم من
&&
وگر شود به نصیحت هزار عبادی
\\
که اوست اصل بصیرت پناه عالم کشف
&&
کز او بیابد بنیاد دید بنیادی
\\
ایا جمال تو را او جمال داد و نمک
&&
ایا کمال تو از رشک او بیفزادی
\\
حرام باشد یاد کسی به هر دو جهان
&&
از آن گهی که تو اندر ضمیر و دل یادی
\\
اگر چه طینت تبریز بس شهان زادی
&&
ولیک چون وی شاهی بگو که کی زادی
\\
کفیل قافیه عمر سایه‌اش بادا
&&
ففی الحقیقه منه الدلیل و الحادی
\\
\end{longtable}
\end{center}
