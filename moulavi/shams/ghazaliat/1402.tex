\begin{center}
\section*{غزل شماره ۱۴۰۲: دوش چه خورده‌ای بگو ای بت همچو شکرم}
\label{sec:1402}
\addcontentsline{toc}{section}{\nameref{sec:1402}}
\begin{longtable}{l p{0.5cm} r}
دوش چه خورده‌ای بگو ای بت همچو شکرم
&&
تا همه عمر بعد از این من شب و روز از آن خورم
\\
ای که ابیت گفته‌ای هر شب عند ربکم
&&
شرح بده از آن ابا بیشتر ای پیمبرم
\\
گر تو ز من نهان کنی شعشعه جمال تو
&&
نوبت ملک می زند ای قمر مصورم
\\
لذت نامه‌های تو ذوق پیام‌های تو
&&
می نرود سوی لبم سخت شده‌ست در برم
\\
لابه کنم که هی بیا درده بانگ الصلا
&&
او کتف این چنین کند که به درونه خوشترم
\\
گشت فضای هر سری میل دل و میسرش
&&
شکر که عشق شد همه میل دل و میسرم
\\
گفتم عشق را شبی راست بگو تو کیستی
&&
گفت حیات باقیم عمر خوش مکررم
\\
گفتمش ای برون ز جا خانه تو کجاست گفت
&&
همره آتش دلم پهلوی دیده ترم
\\
رنگرزم ز من بود هر رخ زعفرانیی
&&
چست الاقم و ولی عاشق اسب لاغرم
\\
غازه لاله‌ها منم قیمت کاله‌ها منم
&&
لذت ناله‌ها منم کاشف هر مسترم
\\
او به کمینه شیوه‌ای صد چو مرا ز ره برد
&&
خواجه مرا تو ره نما من به چه از رهش برم
\\
چرخ نداش می کند کز پی توست گردشم
&&
ماه نداش می کند کز رخ تو منورم
\\
عقل ز جای می جهد روح خراج می دهد
&&
سر به سجود می رود کز پی تو مدورم
\\
من که فضول این دهم وز فن خویش فربهم
&&
ز آتش آفتاب او آب شده‌ست اکثرم
\\
بس کن ای فسانه گو سیر شدم ز گفت و گو
&&
تا به سخن درآید آنک مست شده‌ست از او سرم
\\
\end{longtable}
\end{center}
