\begin{center}
\section*{غزل شماره ۲۶۸۵: دگرباره شه ساقی رسیدی}
\label{sec:2685}
\addcontentsline{toc}{section}{\nameref{sec:2685}}
\begin{longtable}{l p{0.5cm} r}
دگرباره شه ساقی رسیدی
&&
مرا در حلقه مستان کشیدی
\\
دگرباره شکستی تو بها را
&&
به جامی پرده‌ها را بردریدی
\\
دگربار ای خیال فتنه انگیز
&&
چو می بر مغز مستان بردویدی
\\
بیا ای آهو از نافت پدید است
&&
که از نسرین و نیلوفر چریدی
\\
همه صحرا گل است و ارغوان است
&&
بدان یک دم که در صحرا دمیدی
\\
مکن ای آسمان ناموس کم کن
&&
که از سودای ماه من خمیدی
\\
بگو ای جان وگر نی من بگویم
&&
که از شرم جمالش ناپدیدی
\\
بگویم ای بهشت این دم به گوشت
&&
که بی‌او بسته‌ای و بی‌کلیدی
\\
چو خاتونان مصری ای شفق تو
&&
چو دیدی یوسفم را کف بریدی
\\
بدیدم دوش کبریتی به دستت
&&
یقین کردم که دیکی می‌پزیدی
\\
تو هم ای دل در آن مطبخ که او بود
&&
پس دیوار چیزی می‌شنیدی
\\
نه عیدی که دو بار آید به سالی
&&
به رغم عید هر روزی تو عیدی
\\
خداوندا به قدرت بی‌نظیری
&&
که حسنی لانظیری برتنیدی
\\
چنین نوری دهی اشکمبه‌ای را
&&
چنینی را گزافه کی گزیدی
\\
بگو ای گل که این لطف از کی داری
&&
نه خار خشک بودی می‌خلیدی
\\
تو هم ای چشم جنس خاک بودی
&&
بگفتی من چه بینم هم بدیدی
\\
تو هم ای پای برجا مانده بودی
&&
دوانیدت دواننده دویدی
\\
دم عیسی و علمش را عدوی
&&
عجب ای خر بدین دعوت رسیدی
\\
چو مال این علم ماند مرد ریگت
&&
نه تو مانی نه علمی که گزیدی
\\
جهان پیر را گفتم جوان شو
&&
ببین بخت جوان تا کی قدیدی
\\
بیا امید بین که نیک نبود
&&
در این امید بی‌حد ناامیدی
\\
بدو پیوندم از گفتن ببرم
&&
نبرم زان شهی که تو بریدی
\\
\end{longtable}
\end{center}
