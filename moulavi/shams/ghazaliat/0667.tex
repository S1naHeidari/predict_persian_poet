\begin{center}
\section*{غزل شماره ۶۶۷: سماع صوفیان می درنگیرد}
\label{sec:0667}
\addcontentsline{toc}{section}{\nameref{sec:0667}}
\begin{longtable}{l p{0.5cm} r}
سماع صوفیان می درنگیرد
&&
که آتش هیزمی را تر نگیرد
\\
یقین می‌دانک جسمانیست آفت
&&
مکوپ این دست تا پا برنگیرد
\\
بیابد خلوت عشرت مسیحا
&&
اگر مجلس ز گاو و خر نگیرد
\\
چرا در بزم خلوت بی‌گرانان
&&
دل ما عیش را از سر نگیرد
\\
نه اصل این بنا باشد کلوخی
&&
کلوخی لطف آن دلبر نگیرد
\\
که چشم حقد یوسف را نداند
&&
که بانگ چنگ گوش کر نگیرد
\\
ز هر آهو نه صحرا مشک یابد
&&
ز هر گاوی جهان عنبر نگیرد
\\
ز هر نی ناله مشتاق ناید
&&
و هر مرغی ز نی شکر نگیرد
\\
چه داند لطف زهره زهره رفته
&&
که او را گوشه چادر نگیرد
\\
می جان را به جز جانی ننوشد
&&
که جسمانی می انور نگیرد
\\
نه هر ابری حریف ماه گردد
&&
که اختر را به جز اختر نگیرد
\\
اگر دلدار گیرد در جهان کس
&&
از این دلدار ما خوشتر نگیرد
\\
خداوند شمس دین آن نور تبریز
&&
که هر کس را چو من چاکر نگیرد
\\
\end{longtable}
\end{center}
