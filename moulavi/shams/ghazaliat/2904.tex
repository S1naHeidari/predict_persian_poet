\begin{center}
\section*{غزل شماره ۲۹۰۴: عاقبت از عاشقان بگریختی}
\label{sec:2904}
\addcontentsline{toc}{section}{\nameref{sec:2904}}
\begin{longtable}{l p{0.5cm} r}
عاقبت از عاشقان بگریختی
&&
وز مصاف ای پهلوان بگریختی
\\
سوی شیران حمله بردی همچو شیر
&&
همچو روبه از میان بگریختی
\\
قصد بام آسمان می‌داشتی
&&
از میان نردبان بگریختی
\\
تو چگونه دارویی هر درد را
&&
کز صداع این و آن بگریختی
\\
پس روی انبیا چون می‌کنی
&&
چون ز تهدید خسان بگریختی
\\
مرده رنگی و نداری زندگی
&&
مرده باشی چون ز جان بگریختی
\\
دستمزد شادمانی صبر توست
&&
رو که وقت امتحان بگریختی
\\
صبر می‌کن در حصار غم کنون
&&
چون ز بانگ پاسبان بگریختی
\\
کی ببینی چشم تیرانداز را
&&
چون ز تیر خرکمان بگریختی
\\
زخم تیغ و تیر چون خواهی کشید
&&
چون تو از زخم زبان بگریختی
\\
رو خمش کن بی‌نشانی خامشی است
&&
پس چرا سوی نشان بگریختی
\\
\end{longtable}
\end{center}
