\begin{center}
\section*{غزل شماره ۲۴۱۰: مقام خلوت و یار و سماع و تو خفته}
\label{sec:2410}
\addcontentsline{toc}{section}{\nameref{sec:2410}}
\begin{longtable}{l p{0.5cm} r}
مقام خلوت و یار و سماع و تو خفته
&&
که شرم بادت از آن زلف‌های آشفته
\\
از این سپس منم و شب روی و حلقه یار
&&
شب دراز و تب و رازهای ناگفته
\\
برون پرده درند آن بتان و سوزانند
&&
که لطف‌های بتان در شب است بنهفته
\\
به خواب کن همه را طاق شو از این جفتان
&&
به سوی طاق و رواقش مرو به شب جفته
\\
بدانک خلوت شب بر مثال دریایی است
&&
به قعر بحر بود درهای ناسفته
\\
رخ چو کعبه نما شاه شمس تبریزی
&&
که باشدت عوض حج‌های پذرفته
\\
\end{longtable}
\end{center}
