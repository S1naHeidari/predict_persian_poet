\begin{center}
\section*{غزل شماره ۲۵۷۷: همرنگ جماعت شو تا لذت جان بینی}
\label{sec:2577}
\addcontentsline{toc}{section}{\nameref{sec:2577}}
\begin{longtable}{l p{0.5cm} r}
همرنگ جماعت شو تا لذت جان بینی
&&
در کوی خرابات آ تا دردکشان بینی
\\
درکش قدح سودا هل تا بشوی رسوا
&&
بربند دو چشم سر تا چشم نهان بینی
\\
بگشای دو دست خود گر میل کنارستت
&&
بشکن بت خاکی را تا روی بتان بینی
\\
از بهر عجوزی را تا چند کشی کابین
&&
وز بهر سه نان تا کی شمشیر و سنان بینی
\\
نک ساقی بی‌جوری در مجلس او دوری
&&
در دور درآ بنشین تا کی دوران بینی
\\
این جاست ربا نیکو جانی ده و صد بستان
&&
گرگی و سگی کم کن تا مهر شبان بینی
\\
شب یار همی‌گردد خشخاش مخور امشب
&&
بربند دهان از خور تا طعم دهان بینی
\\
گویی که فلانی را ببرید ز من دشمن
&&
رو ترک فلانی گو تا بیست فلان بینی
\\
اندیشه مکن الا از خالق اندیشه
&&
اندیشه جانان به کاندیشه نان بینی
\\
با وسعت ارض الله بر حبس چه چفسیدی
&&
ز اندیشه گره کم زن تا شرح جنان بینی
\\
خامش کن از این گفتن تا گفت بری باری
&&
از جان و جهان بگذر تا جان و جهان بینی
\\
\end{longtable}
\end{center}
