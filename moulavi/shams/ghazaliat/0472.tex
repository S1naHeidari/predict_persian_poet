\begin{center}
\section*{غزل شماره ۴۷۲: کالبد ما ز خواب کاهل و مشغول خاست}
\label{sec:0472}
\addcontentsline{toc}{section}{\nameref{sec:0472}}
\begin{longtable}{l p{0.5cm} r}
کالبد ما ز خواب کاهل و مشغول خاست
&&
آنک به رقص آورد کاهل ما را کجاست
\\
آنک به رقص آورد پرده دل بردرد
&&
این همه بویش کند دیدن او خود جداست
\\
جنبش خلقان ز عشق جنبش عشق از ازل
&&
رقص هوا از فلک رقص درخت از هواست
\\
دل چو شد از عشق گرم رفت ز دل ترس و شرم
&&
شد نفسش آتشین عشق یکی اژدهاست
\\
ساقی جان در قدح دوش اگر درد ریخت
&&
دردی ساقی ما جمله صفا در صفاست
\\
باده عشق ای غلام نیست حلال و حرام
&&
پر کن و پیش آر جام بنگر نوبت که راست
\\
ای دل پاک تمام بر تو هزاران سلام
&&
جمله خوبان غلام جمله خوبی تو راست
\\
سجده کنم پیش یار گوید دل هوش دار
&&
دادن جان در سجود جان همه سجده‌هاست
\\
\end{longtable}
\end{center}
