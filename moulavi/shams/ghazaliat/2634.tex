\begin{center}
\section*{غزل شماره ۲۶۳۴: امروز در این شهر نفیر است و فغانی}
\label{sec:2634}
\addcontentsline{toc}{section}{\nameref{sec:2634}}
\begin{longtable}{l p{0.5cm} r}
امروز در این شهر نفیر است و فغانی
&&
از جادوی چشم یکی شعبده خوانی
\\
در شهر به هر گوشه یکی حلقه به گوشی است
&&
از عشق چنین حلقه ربا چرب زبانی
\\
بی‌زخم نیابی تو در این شهر یکی دل
&&
از تیر نظرهای چنین سخته کمانی
\\
ای شهر چه شهری تو که هر روز تو عید است
&&
ای شهر مکان تو شد از لطف زمانی
\\
چه جای مکان است و چه سودای زمان است
&&
ای هر دو شده از دم تو نادره لانی
\\
شهری است که او تختگه عشق خدایی است
&&
بغداد نهان است وز او دل همدانی
\\
امروز در این مصر از این یوسف خوبی
&&
بی‌زجر و سیاست شده هر گرگ شبانی
\\
صد پیر دو صدساله از این یوسف خوش دم
&&
مانند زلیخا شده در عشق جوانی
\\
او حاکم دل‌ها و روان‌هاست در این شهر
&&
ماننده تقدیر خدا حکم روانی
\\
صد نور یقین سجده کن روی چو ماهش
&&
کی سوی مهش راه بزد ابر گمانی
\\
صد چون من و تو محو چنان بی‌من و مایی
&&
چون ظلمت شب محو رخ ماه جهانی
\\
جز حضرت او نیست فقیرانه حضوری
&&
جز سایه خورشید رخش نیست امانی
\\
از حیله او یک دو سخن دارم بشنو
&&
چون زهره ندارم که بگویم که فلانی
\\
گر نام نگوییم و نشان نیز نگوییم
&&
زین باده شکافیده شود شیشه جانی
\\
هین دست ملرزان و فروکش قدح عشق
&&
پازهر چو داری نکند زهر زیانی
\\
هر چیز که خواهی تو ز عطار بیابی
&&
دکان محیط است و جز این نیست دکانی
\\
\end{longtable}
\end{center}
