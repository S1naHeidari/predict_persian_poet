\begin{center}
\section*{غزل شماره ۷: بنشسته‌ام من بر درت تا بوک برجوشد وفا}
\label{sec:0007}
\addcontentsline{toc}{section}{\nameref{sec:0007}}
\begin{longtable}{l p{0.5cm} r}
بنشسته‌ام من بر درت تا بوک برجوشد وفا
&&
باشد که بگشایی دری گویی که برخیز اندرآ
\\
غرقست جانم بر درت در بوی مشک و عنبرت
&&
ای صد هزاران مرحمت بر روی خوبت دایما
\\
ماییم مست و سرگران فارغ ز کار دیگران
&&
عالم اگر برهم رود عشق تو را بادا بقا
\\
عشق تو کف برهم زند صد عالم دیگر کند
&&
صد قرن نو پیدا شود بیرون ز افلاک و خلا
\\
ای عشق خندان همچو گل وی خوش نظر چون عقل کل
&&
خورشید را درکش به جل ای شهسوار هل اتی
\\
امروز ما مهمان تو مست رخ خندان تو
&&
چون نام رویت می‌برم دل می‌رود والله ز جا
\\
کو بام غیر بام تو کو نام غیر نام تو
&&
کو جام غیر جام تو ای ساقی شیرین ادا
\\
گر زنده جانی یابمی من دامنش برتابمی
&&
ای کاشکی درخوابمی در خواب بنمودی لقا
\\
ای بر درت خیل و حشم بیرون خرام ای محتشم
&&
زیرا که سرمست و خوشم زان چشم مست دلربا
\\
افغان و خون دیده بین صد پیرهن بدریده بین
&&
خون جگر پیچیده بین بر گردن و روی و قفا
\\
آن کس که بیند روی تو مجنون نگردد کو بگو
&&
سنگ و کلوخی باشد او او را چرا خواهم بلا
\\
رنج و بلایی زین بتر کز تو بود جان بی‌خبر
&&
ای شاه و سلطان بشر لا تبل نفسا بالعمی
\\
جان‌ها چو سیلابی روان تا ساحل دریای جان
&&
از آشنایان منقطع با بحر گشته آشنا
\\
سیلی روان اندر وله سیلی دگر گم کرده ره
&&
الحمدلله گوید آن وین آه و لا حول و لا
\\
ای آفتابی آمده بر مفلسان ساقی شده
&&
بر بندگان خود را زده باری کرم باری عطا
\\
گل دیده ناگه مر تو را بدریده جان و جامه را
&&
وان چنگ زار از چنگ تو افکنده سر پیش از حیا
\\
مقبلترین و نیک پی در برج زهره کیست نی
&&
زیرا نهد لب بر لبت تا از تو آموزد نوا
\\
نی‌ها و خاصه نیشکر بر طمع این بسته کمر
&&
رقصان شده در نیستان یعنی تعز من تشا
\\
بد بی‌تو چنگ و نی حزین برد آن کنار و بوسه این
&&
دف گفت می‌زن بر رخم تا روی من یابد بها
\\
این جان پاره پاره را خوش پاره پاره مست کن
&&
تا آن چه دوشش فوت شد آن را کند این دم قضا
\\
حیفست ای شاه مهین هشیار کردن این چنین
&&
والله نگویم بعد از این هشیار شرحت ای خدا
\\
یا باده ده حجت مجو یا خود تو برخیز و برو
&&
یا بنده را با لطف تو شد صوفیانه ماجرا
\\
\end{longtable}
\end{center}
