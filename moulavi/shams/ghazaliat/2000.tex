\begin{center}
\section*{غزل شماره ۲۰۰۰: ای ز هجران تو مردن طرب و راحت من}
\label{sec:2000}
\addcontentsline{toc}{section}{\nameref{sec:2000}}
\begin{longtable}{l p{0.5cm} r}
ای ز هجران تو مردن طرب و راحت من
&&
مرگ بر من شده بی‌تو مثل شهد و لبن
\\
می تپد ماهی بی‌آب بر آن ریگ خشن
&&
تا جدا گردد آن جان نزارش ز بدن
\\
آب تلخی شده بر جانوران آب حیات
&&
شکر خشک بر ایشان بتر از گور و کفن
\\
نیست بازی کشش جزو به اصل کل خویش
&&
چند پیغامبر بگریست پی حب وطن
\\
کودکی کو نشناسد وطن و مولد خویش
&&
دایه خواهد چه ستنبول مر او را چه یمن
\\
شد چراگاه ستاره سوی مرعای فلک
&&
حیوان خاک پرستد مثل سرو و سمن
\\
من از این ناله اگر چه که دهان می بندم
&&
نتوان در شکم آب فروبست دهن
\\
نفس چغز ز آب است نه از باد هوا
&&
بحریان را هله این باشد معهوده و فن
\\
عارفانی که نهانند در آن قلزم نور
&&
دمشان جمله ز نوری است ظلامات شکن
\\
قلم و لوح چو این جا برسیدیم شکست
&&
شکند کوه چو آگه شود از رب منن
\\
\end{longtable}
\end{center}
