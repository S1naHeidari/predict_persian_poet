\begin{center}
\section*{غزل شماره ۷۲۳: ساقی برخیز کان مه آمد}
\label{sec:0723}
\addcontentsline{toc}{section}{\nameref{sec:0723}}
\begin{longtable}{l p{0.5cm} r}
ساقی برخیز کان مه آمد
&&
بشتاب که سخت بی‌گه آمد
\\
ترکانه بتاز وقت تنگست
&&
کان ترک ختا به خرگه آمد
\\
در وهم نبود این سعادت
&&
اقبال نگر که ناگه آمد
\\
عاشق چو پیاله پر ز خون بود
&&
چون ساغر می به قهقه آمد
\\
با چون تو مه آنک وقت دریافت
&&
تعجیل نکرد ابله آمد
\\
از خرمن عشق هر کی بگریخت
&&
کاهست به خرمن که آمد
\\
بی گه شد و هر کی اوست مقبل
&&
بگریخت ز خود به درگه آمد
\\
اندر تبریزهای و هوییست
&&
آن را که ز هجر با ره آمد
\\
\end{longtable}
\end{center}
