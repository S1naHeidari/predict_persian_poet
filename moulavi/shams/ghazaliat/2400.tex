\begin{center}
\section*{غزل شماره ۲۴۰۰: گل را نگر ز لطف سوی خار آمده}
\label{sec:2400}
\addcontentsline{toc}{section}{\nameref{sec:2400}}
\begin{longtable}{l p{0.5cm} r}
گل را نگر ز لطف سوی خار آمده
&&
دل ناز و باز کرده و دلدار آمده
\\
مه را نگر برآمده مهمان شب شده
&&
دامن کشان ز عالم انوار آمده
\\
خورشید را نگر که شهنشاه اختر است
&&
از بهر عذر گازر غمخوار آمده
\\
منگر به نقطه خوار تو آن را نگر که دوست
&&
اندر طواف نقطه چو پرگار آمده
\\
آن دلبری که دل ز همه دلبران ربود
&&
اندر وثاق این دل بیمار آمده
\\
این عشق همچو روح در این خاکدان غریب
&&
مانند مصطفاست به کفار آمده
\\
همچون بهار سوی درختان خشک ما
&&
آن نوبهار حسن به ایثار آمده
\\
پنهان بود بهار ولی در اثر نگر
&&
زو باغ زنده گشته و در کار آمده
\\
جان را اگر نبینی در دلبران نگر
&&
با قد سرو و روی چو گلنار آمده
\\
گر عشق را نبینی در عاشقان نگر
&&
منصوروار شاد سوی دار آمده
\\
در عین مرگ چشمه آب حیات دید
&&
آن چشمه ای که مایه دیدار آمده
\\
آمد بهار عشق به بستان جان درآ
&&
بنگر به شاخ و برگ به اقرار آمده
\\
اقرار می‌کنند که حشر و قیامت است
&&
آن مردگان باغ دگربار آمده
\\
ای دل ز خود چو باخبری رو خموش کن
&&
چون بی‌خبر مباش به اخبار آمده
\\
\end{longtable}
\end{center}
