\begin{center}
\section*{غزل شماره ۲۰۸۴: بیا بیا که ز هجرت نه عقل ماند نه دین}
\label{sec:2084}
\addcontentsline{toc}{section}{\nameref{sec:2084}}
\begin{longtable}{l p{0.5cm} r}
بیا بیا که ز هجرت نه عقل ماند نه دین
&&
قرار و صبر برفته‌ست زین دل مسکین
\\
ز روی زرد و دل درد و سوز سینه مپرس
&&
که آن به شرح نگنجد بیا به چشم ببین
\\
چو نان پخته ز تاب تو سرخ رو بودم
&&
چو نان ریزه کنونم ز خاک ره برچین
\\
چو آینه ز جمالت خیال چین بودم
&&
کنون تو چهره من زرد بین و چین بر چین
\\
مثال آبم در جوی کژروان چپ و راست
&&
فراق از چپ و از راستم گشاده کمین
\\
به روز و شب چو زمین رو بر آسمان دارم
&&
ز روی تو که نگنجد در آسمان و زمین
\\
سحر ز درد نوشتیم نامه پیش صبا
&&
که از برای خدا ره سوی سفر بگزین
\\
اگر سر تو به گل دربود مشوی بیا
&&
وگر به خار رسد پا به کندنش منشین
\\
بیا بیا و خلاصم ده از بیا و برو
&&
بیا چنانک رهد جانم از چنان و چنین
\\
پیام کردم کای تو پیمبر عشاق
&&
بگو برای خدا زود ای رسول امین
\\
که غرق آبم و آتش ز موج دیده و دل
&&
مرا چه چاره نوشت او که چاره تو همین
\\
نشست نقش دعایم به عالم گردون
&&
کجاست گوش نمازی که بشنود آمین
\\
هزار آینه و صد هزار صورت را
&&
دهم به عشق صلاح جهان صلاح الدین
\\
\end{longtable}
\end{center}
