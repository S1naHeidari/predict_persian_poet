\begin{center}
\section*{غزل شماره ۱۰۱۳: اگر حریف منی پس بگو که دوش چه بود}
\label{sec:1013}
\addcontentsline{toc}{section}{\nameref{sec:1013}}
\begin{longtable}{l p{0.5cm} r}
اگر حریف منی پس بگو که دوش چه بود
&&
میان این دل و آن یار می فروش چه بود
\\
فدیت سیدنا انه یری و یجود
&&
الی البقاء یبلغ من الفناء یذود
\\
اگر به چشم بدیدی جمال ماهم دوش
&&
مرا بگو که در آن حلقه‌های گوش چه بود
\\
معاد کل شرود طغی و منه نی
&&
مثال ظلک ان طال هو الیک یعود
\\
وگر تو با من هم خرقه‌ای و همرازی
&&
بگو که صورت آن شیخ خرقه پوش چه بود
\\
بامر حافظ الله المکان یعی
&&
بمس عاطفه الله الزمان ولود
\\
اگر فقیری و ناگفته راز می‌شنوی
&&
بگو اشارت آن ناطق خموش چه بود
\\
ایا فؤاد فذب فی لظی محبته
&&
ایا حیاه فدومی فقد اتاک خلود
\\
وگر نخفتی و از حال دوش آگاهی
&&
بگو که نیم شب آن نعره و خروش چه بود
\\
ترید جبر جبیر الفؤاد فانکسرن
&&
ترید نحله تاج فلا تنی به سجود
\\
از آنچ جامه و تن پاره پاره می‌کردیم
&&
بیار پارگکی تا که رنگ و بوش چه بود
\\
برغم انفک لا تنکسر کما الحیوان
&&
به نصف وجهک لا تسجدن شبیه یهود
\\
وگر چو یونس رستی ز حبس ماهی و بحر
&&
بگو که معنی آن بحر و موج و جوش چه بود
\\
یقول لیت حبیبی یحبنی کرما
&&
الیس حبک تأثیر حب ود ودود
\\
وگر شناخته‌ای کاصل انس و جان ز کجاست
&&
یکیست اصل پس این وحشت وحوش چه بود
\\
ایا نضاره عیشی بما تهیجنی
&&
متی تقر عیونی و صاحبی مفقود
\\
وگر بدیدی جانی که پشت و رویش نیست
&&
گه تصور عشاق پشت و روش چه بود
\\
لئن سکرت بما قد سقیتنی یا دهر
&&
اکون مثلک لدا لربه لکنود
\\
وگر ز عشق تو سردفتر غرض ماییم
&&
هزار دفتر و پیغام و گفت و گوش چه بود
\\
\end{longtable}
\end{center}
