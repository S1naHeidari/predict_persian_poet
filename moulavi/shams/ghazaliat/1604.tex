\begin{center}
\section*{غزل شماره ۱۶۰۴: بده آن باده دوشین که من از نوش تو مستم}
\label{sec:1604}
\addcontentsline{toc}{section}{\nameref{sec:1604}}
\begin{longtable}{l p{0.5cm} r}
بده آن باده دوشین که من از نوش تو مستم
&&
بده ای حاتم عالم قدح زفت به دستم
\\
ز من ای ساقی مردان نفسی روی مگردان
&&
دل من مشکن اگر نه قدح و شیشه شکستم
\\
قدحی بود به دستم بفکندم بشکستم
&&
کف صد پای برهنه من از آن شیشه بخستم
\\
تو بدان شیشه پرستی که ز شیشه است شرابت
&&
می من نیست ز شیره ز چه رو شیشه پرستم
\\
بکش ای دل می جانی و بخسب ایمن و فارغ
&&
که سر غصه بریدم ز غم و غصه برستم
\\
دل من رفت به بالا تن من رفت به پستی
&&
من بیچاره کجایم نه به بالا نه به پستم
\\
چه خوش آویخته سیبم که ز سنگت نشکیبم
&&
ز بلی چون بشکیبم من اگر مست الستم
\\
تو ز من پرس که این عشق چه گنج است و چه دارد
&&
تو مرا نیز از او پرس که گوید چه کسستم
\\
به لب جوی چه گردی بجه از جوی چو مردی
&&
بجه از جوی و مرا جو که من از جوی بجستم
\\
فلئن قمت اقمنا و لئن رحت رحلنا
&&
چو بخوردی تو بخوردم چو نشستی تو نشستم
\\
منم آن مست دهلزن که شدم مست به میدان
&&
دهل خویش چو پرچم به سر نیزه ببستم
\\
چه خوش و بیخود شاهی هله خاموش چو ماهی
&&
چو ز هستی برهیدم چه کشی باز به هستم
\\
\end{longtable}
\end{center}
