\begin{center}
\section*{غزل شماره ۲۵۲۶: نکو بنگر به روی من نه آنم من که هر باری}
\label{sec:2526}
\addcontentsline{toc}{section}{\nameref{sec:2526}}
\begin{longtable}{l p{0.5cm} r}
نکو بنگر به روی من نه آنم من که هر باری
&&
ببین دریای شیرینی ببین موج گهر باری
\\
کی بگریزد ز دست حق کی پرهیزد ز شست حق
&&
قیامت کو که تا بیند به نقد این شور و شر باری
\\
یکی دستش چو قبض آمد یکی دستش چو بسط آمد
&&
نداری زین دو بیرون شو گه باش و سفر باری
\\
چو عیسی گر شکر خندی شکرخنده ببین از وی
&&
چو موسی گر کمر بندی بر آن کوه کمر باری
\\
شدی دربان هر دونی به زیر بام گردونی
&&
به کوی یار ما دررو که بینی بام و در باری
\\
به شاخ گل همی‌گفتم چه می‌رقصی در این گلخن
&&
درآ در باغ جان بنگر شکوفه و شاخ تر باری
\\
عطارد را همی‌گفتم به فضل و فن شدی غره
&&
قلم بشکن بیا بشنو پیام نیشکر باری
\\
به گوش زهره می‌گفتم که گوشت گرم شد از می
&&
سر اندر بزم سلطان کن ببین سودای سر باری
\\
چو سوسن صد زبان داری زبان درکش از این زاری
&&
ز غنچه بسته لب بشنو ز خاموشان خبر باری
\\
\end{longtable}
\end{center}
