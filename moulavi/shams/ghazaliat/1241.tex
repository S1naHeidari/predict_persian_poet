\begin{center}
\section*{غزل شماره ۱۲۴۱: ما نعره به شب زنیم و خاموش}
\label{sec:1241}
\addcontentsline{toc}{section}{\nameref{sec:1241}}
\begin{longtable}{l p{0.5cm} r}
ما نعره به شب زنیم و خاموش
&&
تا درنرود درون هر گوش
\\
تا بو نبرد دماغ هر خام
&&
بر دیگ وفا نهیم سرپوش
\\
بخلی نبود ولی نشاید
&&
این شهره گلاب و خانه موش
\\
شب آمد و جوش خلق بنشست
&&
برخیز کز آن ماست سرجوش
\\
امشب ز تو قدر یافت و عزت
&&
بر دوش ز کبر می‌زند دوش
\\
یک چند سماع گوش کردیم
&&
بردار سماع جان بی‌هوش
\\
ای تن دهنت پر از شکر شد
&&
پیشت گله نیست هیچ مخروش
\\
ای چنبر دف رسن گسستی
&&
با چرخه و دلو و چاه کم کوش
\\
چون گشت شکار شیر جانی
&&
بیزار شد از شکار خرگوش
\\
خرگوش که صورتند بی‌جان
&&
گرمابه پر از نگار منقوش
\\
با نفس حدیث روح کم گوی
&&
وز ناقه مرده شیر کم دوش
\\
از شر بگریز یار شب باش
&&
کاندر سر شب نهند شب پوش
\\
تا صبح وصال دررسیدن
&&
درکش شب تیره را در آغوش
\\
از یاد لقای یار بی‌خواب
&&
از خواب شدستمان فراموش
\\
شب چتر سیاه دان و با وی
&&
نعره دهلست و بانک چاووش
\\
این فتنه به هر دمی فزونست
&&
امشب بترست عشق از دوش
\\
شب چیست نقاب روی مقصود
&&
کای رحمت و آفرین بر آن روش
\\
هین طبلک شب روان فروکوب
&&
زیرا که سوار شد سیاووش
\\
\end{longtable}
\end{center}
