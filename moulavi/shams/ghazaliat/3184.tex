\begin{center}
\section*{غزل شماره ۳۱۸۴: کسی کو را بود خلق خدایی}
\label{sec:3184}
\addcontentsline{toc}{section}{\nameref{sec:3184}}
\begin{longtable}{l p{0.5cm} r}
کسی کو را بود خلق خدایی
&&
ازو یابند جانهای بقایی
\\
به روزی پنج نوبت بر در او
&&
همی کوبند کوس کبریایی
\\
اگر افتد بدین سو بانگ آن کوس
&&
بیابند جملگان از خود رهایی
\\
زمین خود کی تواند بند کردن
&&
هر آنکس را که روحش شد سمایی؟!
\\
عنایت چون ز یزدان برتو باشد
&&
چه غم گر تو به طاعت کمتر آیی؟!
\\
در آن منزل چه طاعت پای دارد؟!
&&
که جان بخشت کند از دلربایی
\\
به جای راستی و صدق گیرند
&&
خیانتها که کردی یا دغایی
\\
اگر تو از دل و جان دوستداری
&&
کسی کو گوهرش نبود بهایی
\\
خداوند خداوندان اسرار
&&
همایان را همی بخشد همایی
\\
ترا گردید رویش رزق باشد
&&
به صد لابه بهشت اندر نیایی
\\
قرار جان شمس‌الدین تبریز
&&
که جانم را مباد از وی جدایی
\\
جدایی تن مرا خود بند کردست
&&
هم از وی چشم می‌دارم رهایی
\\
که دست جان او چندان درازست
&&
که عقل کل کند یاوه کیایی
\\
هزاران شکر ایزد را که جانم
&&
به عشق چشم او دارد روایی
\\
فحمدا ثم حمدا ثم حمدا
&&
بما اروانی خلاق السماء
\\
من‌النور الممدد کل نور
&&
من‌الکنز المکنز فی الخفاء
\\
وآتاهم من‌الاسرار فضلا
&&
و نجاهم بها کل البلاء
\\
و احیاهم بروح عاشقی
&&
طلیق من هجومات الوباء
\\
طلب منی بشیرالوصل یوما
&&
قباء الروح انزعت قبایی
\\
لقیت من فضایلهم مرادا
&&
و اوصافا تجلت بالبهاء
\\
وجاد الصدر شمس‌الدین یوما
&&
حیوتیا دوامیا جزایی
\\
رایت البخت یسجدنی اذاما
&&
تکرم سیدی بالالبهاء
\\
وآتانی علامته بعشق
&&
دوام سرمدی فی بقایی
\\
علمت بابتداء حال عشقی
&&
تمامة دولة فی الانتهاء
\\
فلا اخلالة ظلا علینا
&&
فذاک جمیع طمعی وارنجایی
\\
فحاشا بل عنایته بحور
&&
غریق منه بغیی وابتغائی
\\
معانی روحنا ماء زلال
&&
و بالا لفاظ ما زج بالدماء
\\
\end{longtable}
\end{center}
