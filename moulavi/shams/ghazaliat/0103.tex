\begin{center}
\section*{غزل شماره ۱۰۳: دل و جان را در این حضرت بپالا}
\label{sec:0103}
\addcontentsline{toc}{section}{\nameref{sec:0103}}
\begin{longtable}{l p{0.5cm} r}
دل و جان را در این حضرت بپالا
&&
چو صافی شد رود صافی به بالا
\\
اگر خواهی که ز آب صاف نوشی
&&
لب خود را به هر دردی میالا
\\
از این سیلاب درد او پاک ماند
&&
که جانبازست و چست و بی‌مبالا
\\
نپرد عقل جزوی زین عقیله
&&
چو نبود عقل کل بر جزو لالا
\\
نلرزد دست وقت زر شمردن
&&
چو بازرگان بداند قدر کالا
\\
چه گرگینست وگر خارست این حرص
&&
کسی خود را بر این گرگین ممالا
\\
چو شد ناسور بر گرگین چنین گر
&&
طلی سازش به ذکر حق تعالا
\\
اگر خواهی که این در باز گردد
&&
سوی این در روان و بی‌ملال آ
\\
رها کن صدر و ناموس و تکبر
&&
میان جان بجو صدر معلا
\\
کلاه رفعت و تاج سلیمان
&&
به هر کل کی رسد حاشا و کلا
\\
خمش کردم سخن کوتاه خوشتر
&&
که این ساعت نمی‌گنجد علالا
\\
جواب آن غزل که گفت شاعر
&&
بقایی شاء لیس هم ارتحالا
\\
\end{longtable}
\end{center}
