\begin{center}
\section*{غزل شماره ۳۰۱۵: جان و جهان می‌روی جان و جهان می‌بری}
\label{sec:3015}
\addcontentsline{toc}{section}{\nameref{sec:3015}}
\begin{longtable}{l p{0.5cm} r}
جان و جهان می‌روی جان و جهان می‌بری
&&
کان شکر می‌کشی با شکران می‌خوری
\\
ای رخ تو چون قمر تک مرو آهسته تر
&&
تا نخلد شاخ گل سینه نیلوفری
\\
چهره چون آفتاب می‌بری از ما شتاب
&&
بوی کن آخر کباب زین جگر آذری
\\
یک نظری گر وفاست هم صدقات شماست
&&
گر برسانی رواست شکر چنین توانگری
\\
تا جگر خون ما تا دل مجنون ما
&&
تا غم افزون ما کسب کند بهتری
\\
شکر که ما سوختیم سوختن آموختیم
&&
وز جگر افروختیم شیوه سامندری
\\
فاسد سودای تو مست تماشای تو
&&
بوسد بر پای تو از طرب بی‌سری
\\
عشق من ای خوبرو رونق خوبان به تو
&&
گاه شوی بت شکن گاه کنی آزری
\\
مستی از آن دید و داد شادی از آن بخت شاد
&&
چشم بدت دور باد تا که کنی لمتری
\\
جانب دل رو به جان تا که ببینی عیان
&&
حلقه جوق ملک صورت نقش پری
\\
از ملک و از پری چون قدری بگذری
&&
محو شود در صفات صورت و صورتگری
\\
\end{longtable}
\end{center}
