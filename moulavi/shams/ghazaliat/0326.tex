\begin{center}
\section*{غزل شماره ۳۲۶: حالت ده و حیرت ده ای مبدع بی‌حالت}
\label{sec:0326}
\addcontentsline{toc}{section}{\nameref{sec:0326}}
\begin{longtable}{l p{0.5cm} r}
حالت ده و حیرت ده ای مبدع بی‌حالت
&&
لیلی کن و مجنون کن ای صانع بی‌آلت
\\
صد حاجت گوناگون در لیلی و در مجنون
&&
فریادکنان پیشت کای معطی بی‌حاجت
\\
انگشتری حاجت مهریست سلیمانی
&&
رهنست به پیش تو از دست مده صحبت
\\
بگذشت مه توبه آمد به جهان ماهی
&&
کو بشکند و سوزد صد توبه به یک ساعت
\\
ای گیج سری کان سر گیجیده نگردد ز او
&&
وی گول دلی کان دل یاوه نکند نیت
\\
ما لنگ شدیم این جا بربند در خانه
&&
چرنده و پرنده لنگند در این حضرت
\\
ای عشق تویی کلی هم تاجی و هم غلی
&&
هم دعوت پیغامبر هم ده دلی امت
\\
از نیست برآوردی ما را جگری تشنه
&&
بردوخته‌ای ما را بر چشمه این دولت
\\
خارم ز تو گل گشته و اجزا همه کل گشته
&&
هم اول ما رحمت هم آخر ما رحمت
\\
در خار ببین گل را بیرون همه کس بیند
&&
در جزو ببین کل را این باشد اهلیت
\\
در غوره ببین می را در نیست ببین شیء را
&&
ای یوسف در چه بین شاهنشهی و ملکت
\\
خاری که ندارد گل در صدر چمن ناید
&&
خاکی ز کجا یابد بی‌روح سر و سبلت
\\
کف می‌زن و زین می‌دان تو منشاء هر بانگی
&&
کاین بانگ دو کف نبود بی‌فرقت و بی‌وصلت
\\
خامش که بهار آمد گل آمد و خار آمد
&&
از غیب برون جسته خوبان جهت دعوت
\\
\end{longtable}
\end{center}
