\begin{center}
\section*{غزل شماره ۲۰۱: شب رفت و هم تمام نشد ماجرای ما}
\label{sec:0201}
\addcontentsline{toc}{section}{\nameref{sec:0201}}
\begin{longtable}{l p{0.5cm} r}
شب رفت و هم تمام نشد ماجرای ما
&&
ناچار گفتنی‌ست تمامی ماجرا
\\
والله ز دور آدم تا روز رستخیز
&&
کوته نگشت و هم نشود این درازنا
\\
اما چنین نماید کاینک تمام شد
&&
چون ترک گوید اشپو مرد رونده را
\\
اشپوی ترک چیست که نزدیک منزلی
&&
تا گرمی و جلادت و قوت دهد تو را
\\
چون راه رفتنی‌ست توقف هلاکت‌ست
&&
چونت قنق کند که بیا خرگه اندرآ
\\
صاحب مروتی‌ست که جانش دریغ نیست
&&
لیکن گرت بگیرد ماندی در ابتلا
\\
بر ترک ظن بد مبر و متهم مکن
&&
مستیز همچو هندو بشتاب همرها
\\
کان جا در آتش است سه نعل از برای تو
&&
وان جا به گوش تست دل خویش و اقربا
\\
نگذارد اشتیاق کریمان که آب خوش
&&
اندر گلوی تو رود ای یار باوفا
\\
گر در عسل نشینی تلخت کنند زود
&&
ور با وفا تو جفت شوی گردد آن جفا
\\
خاموش باش و راه رو و این یقین بدان
&&
سرگشته دارد آب غریبی چو آسیا
\\
\end{longtable}
\end{center}
