\begin{center}
\section*{غزل شماره ۸۸۴: پرده دل می‌زند زهره هم از بامداد}
\label{sec:0884}
\addcontentsline{toc}{section}{\nameref{sec:0884}}
\begin{longtable}{l p{0.5cm} r}
پرده دل می‌زند زهره هم از بامداد
&&
مژده که آن بوطرب داد طرب‌ها بداد
\\
بحر کرم کرد جوش پنبه برون کن ز گوش
&&
آنچ کفش داد دوش ما و تو را نوش باد
\\
عشق همایون پیست خطبه به نام ویست
&&
از سر ما کم مباد سایه این کیقباد
\\
روی خوشش چون شرار خوی خوشش نوبهار
&&
وان دگرش زینهار او هو رب العباد
\\
ز اول روز این خمار کرد مرا بی‌قرار
&&
می‌کشدم ابروار عشق تو چون تندباد
\\
دست دل از رنج رست گر چه دلارام مست
&&
بست سر زلف بست خواجه ببین این گشاد
\\
می‌کشدم موکشان من ترش و سرگران
&&
رو که مراد جهان می‌کشدم بی‌مراد
\\
عقل بر آن عقل ساز ناز همی‌کرد ناز
&&
شکر کز آن گشت باز تا به مقام اوفتاد
\\
پای به گل بوده‌ام زانک دودل بوده‌ام
&&
شکر که دودل نماند یک دله شد دل نهاد
\\
لاف دل از آسمان لاف تن از ریسمان
&&
بگسلم این ریسمان بازروم در معاد
\\
دلبر روز الست چیز دگر گفت پست
&&
هیچ کسی هست کو آرد آن را به یاد
\\
گفت به تو تاختم بهر خودت ساختم
&&
ساخته خویش را من ندهم در مزاد
\\
گفتم تو کیستی گفت مراد همه
&&
گفتم من کیستم گفت مراد مراد
\\
مفتعلن فاعلات رفته بدم از صفات
&&
محو شده پیش ذات دل به سخن چون فتاد
\\
داد دل و عقل و جان مفخر تبریزیان
&&
از مدد این سه داد یافت زمانه سداد
\\
\end{longtable}
\end{center}
