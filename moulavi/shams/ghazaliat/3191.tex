\begin{center}
\section*{غزل شماره ۳۱۹۱: کالی تیشی آینوسؤای افندی چلبی}
\label{sec:3191}
\addcontentsline{toc}{section}{\nameref{sec:3191}}
\begin{longtable}{l p{0.5cm} r}
کالی تیشی آینوسؤای افندی چلبی
&&
نیمشب بر بام مایی، تا کرا می‌طلبی
\\
گه سیه‌پوش و عصایی، که منم کالویروس
&&
گه عمامه و نیزه در کف که غریبم عربی
\\
چون عرب گردی، بگویی «فاعلاتن فلاعات
&&
ابصرالدنیا جمیعا فی قمیصی تختبی
\\
علت اولی نمودی خویش را با فلسفی
&&
چه زیان دارد ترا؟! تو یاربی و یاربی
\\
گر چنینی، گر چنانی، جان مایی جان جان
&&
هر زبان خواهی بفرما، خسروا، شیرین لبی
\\
ارتمی اغاپسودی کایکا پراترا
&&
نور حقی یا تو حقی، یا فرشته یا نبی
\\
با نه اینی و نه آنی، صورت عشقی و بس
&&
با کدامین لشکری و در کدامین موکبی؟
\\
چون غم دل می‌خورم، یا رحم بر دل می‌برم
&&
کای دل مسکین، چرا اندر چنین تاب و تبی؟!
\\
دل همی گوید « برو من از کجا، تو از کجا!
&&
من دلم تو قالبی رو، رو، همی کن قالبی
\\
پوستها را رنگها و مغزها را ذوقها
&&
پوستها با مغزها خود کی کند هم مذهبی؟! »
\\
کالی میراسس نزیتن بوستن کالاستن
&&
شب شما را روز گشت و نیست شبها را شبی
\\
من خمش کردم، فسونم، بی‌زبان تعلیم ده
&&
ای ز تو لرزان و ترسان مشرقی و مغربی
\\
شمس تبریزی، برآ چون آفتاب از شرق جان
&&
تا گشایند از میان زنار کفر و معجبی
\\
\end{longtable}
\end{center}
