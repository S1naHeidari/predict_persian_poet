\begin{center}
\section*{غزل شماره ۱۷: آمد ندا از آسمان جان را که بازآ الصلا}
\label{sec:0017}
\addcontentsline{toc}{section}{\nameref{sec:0017}}
\begin{longtable}{l p{0.5cm} r}
آمد ندا از آسمان جان را که بازآ الصلا
&&
جان گفت ای نادی خوش اهلا و سهلا مرحبا
\\
سمعا و طاعه ای ندا هر دم دو صد جانت فدا
&&
یک بار دیگر بانگ زن تا برپرم بر هل اتی
\\
ای نادره مهمان ما بردی قرار از جان ما
&&
آخر کجا می‌خوانیم گفتا برون از جان و جا
\\
از پای این زندانیان بیرون کنم بند گران
&&
بر چرخ بنهم نردبان تا جان برآید بر علا
\\
تو جان جان افزاستی آخر ز شهر ماستی
&&
دل بر غریبی می‌نهی این کی بود شرط وفا
\\
آوارگی نوشت شده خانه فراموشت شده
&&
آن گنده پیر کابلی صد سحر کردت از دغا
\\
این قافله بر قافله پویان سوی آن مرحله
&&
چون برنمی‌گردد سرت چون دل نمی‌جوشد تو را
\\
بانگ شتربان و جرس می‌نشنود از پیش و پس
&&
ای بس رفیق و همنفس آن جا نشسته گوش ما
\\
خلقی نشسته گوش ما مست و خوش و بی‌هوش ما
&&
نعره زنان در گوش ما که سوی شاه آ ای گدا
\\
\end{longtable}
\end{center}
