\begin{center}
\section*{غزل شماره ۴۶۱: شاه گشادست رو دیده شه بین که راست}
\label{sec:0461}
\addcontentsline{toc}{section}{\nameref{sec:0461}}
\begin{longtable}{l p{0.5cm} r}
شاه گشادست رو دیده شه بین که راست
&&
باده گلگون شه بر گل و نسرین که راست
\\
شاه در این دم به بزم پای طرب درنهاد
&&
بر سر زانوی شه تکیه و بالین که راست
\\
پیش رخ آفتاب چرخ پیاپی کی زد
&&
در تتق ابر تن ماه به تعیین که راست
\\
ساغرها می‌شمرد وی بشده از شمار
&&
گر بنشد از شمار ساغر پیشین که راست
\\
از اثر روی شه هر نفسی شاهدی
&&
سر کشد از لامکان گوید کابین که راست
\\
ای بس مرغان آب بر لب دریای عشق
&&
سینه صیاد کو دیده شاهین که راست
\\
هین که براقان عشق در چمنش می‌چرند
&&
تنگ درآمد وصال لایقشان زین که راست
\\
سیمبر خوب عشق رفت به خرگاه دل
&&
چهره زر لایق آن بر سیمین که راست
\\
خسرو جان شمس دین مفخر تبریزیان
&&
در دو جهان همچو او شاه خوش آیین که راست
\\
\end{longtable}
\end{center}
