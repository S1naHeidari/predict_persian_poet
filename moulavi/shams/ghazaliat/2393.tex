\begin{center}
\section*{غزل شماره ۲۳۹۳: برجه ز خواب و بنگر صبحی دگر دمیده}
\label{sec:2393}
\addcontentsline{toc}{section}{\nameref{sec:2393}}
\begin{longtable}{l p{0.5cm} r}
برجه ز خواب و بنگر صبحی دگر دمیده
&&
جویان و پای کوبان از آسمان رسیده
\\
ای جان چرا نشستی وقت می است و مستی
&&
آخر در این کشاکش کس نیست پاکشیده
\\
بهر رضای مستی برجه بکوب دستی
&&
دستی قدح پرستی پرراوق گزیده
\\
ما را مبین چو مستان هر چه خورم می است آن
&&
افیون شود مرا نان مخموری دو دیده
\\
نگذاشت آن قیامت تا من کنم ریاضت
&&
آن دیده‌اش ندیده گوشیش ناشنیده
\\
او آب زندگانی می‌داد رایگانی
&&
از قطره قطره او فردوس بردمیده
\\
از دوست هر چه گفتم بیرون پوست گفتم
&&
زان سر چه دارد آن جان گفتار دم بریده
\\
با این همه دهانم گر رشک او نبستی
&&
صد جای آسمان را تو دیدیی دریده
\\
یخدان چه داند ای جان خورشید و تابشش را
&&
کی داند آفرین را این جان آفریده
\\
با این که می‌نداند چون جرعه‌ای ستاند
&&
مستی خراب گردد از خویش وارهیده
\\
تبریز تو چه دانی اسرار شمس دین را
&&
بیرون نجسته‌ای تو زین چرخه خمیده
\\
\end{longtable}
\end{center}
