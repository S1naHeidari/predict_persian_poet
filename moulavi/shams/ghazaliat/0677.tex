\begin{center}
\section*{غزل شماره ۶۷۷: عجب آن دلبر زیبا کجا شد}
\label{sec:0677}
\addcontentsline{toc}{section}{\nameref{sec:0677}}
\begin{longtable}{l p{0.5cm} r}
عجب آن دلبر زیبا کجا شد
&&
عجب آن سرو خوش بالا کجا شد
\\
میان ما چو شمعی نور می‌داد
&&
کجا شد ای عجب بی‌ما کجا شد
\\
دلم چون برگ می‌لرزد همه روز
&&
که دلبر نیم شب تنها کجا شد
\\
برو بر ره بپرس از رهگذریان
&&
که آن همراه جان افزا کجا شد
\\
برو در باغ پرس از باغبانان
&&
که آن شاخ گل رعنا کجا شد
\\
برو بر بام پرس از پاسبانان
&&
که آن سلطان بی‌همتا کجا شد
\\
چو دیوانه همی‌گردم به صحرا
&&
که آن آهو در این صحرا کجا شد
\\
دو چشم من چو جیحون شد ز گریه
&&
که آن گوهر در این دریا کجا شد
\\
ز ماه و زهره می‌پرسم همه شب
&&
که آن مه رو بر این بالا کجا شد
\\
چو آن ماست چون با دیگرانست
&&
چو این جا نیست او آن جا کجا شد
\\
دل و جانش چو با الله پیوست
&&
اگر زین آب و گل شد لاکجا شد
\\
بگو روشن که شمس الدین تبریز
&&
چو گفت الشمس لا یخفی کجا شد
\\
\end{longtable}
\end{center}
