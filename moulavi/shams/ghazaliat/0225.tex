\begin{center}
\section*{غزل شماره ۲۲۵: بپخته است خدا بهر صوفیان حلوا}
\label{sec:0225}
\addcontentsline{toc}{section}{\nameref{sec:0225}}
\begin{longtable}{l p{0.5cm} r}
بپخته است خدا بهر صوفیان حلوا
&&
که حلقه حلقه نشستند و در میان حلوا
\\
هزار کاسه سر رفت سوی خوان فلک
&&
چو درفتاد از آن دیگ در دهان حلوا
\\
به شرق و غرب فتادست غلغلی شیرین
&&
چنین بود چو دهد شاه خسروان حلوا
\\
پیاپی از سوی مطبخ رسول می‌آید
&&
که پخته‌اند ملایک بر آسمان حلوا
\\
به آبریز برد چونک خورد حلوا تن
&&
به سوی عرش برد چونک خورد جان حلوا
\\
به گرد دیگ دل ای جان چو کفچه گرد به سر
&&
که تا چو کفچه دهان پر کنی از آن حلوا
\\
دلی که از پی حلوا چو دیک سوخت سیاه
&&
کرم بود که ببخشد به تای نان حلوا
\\
خموش باش که گر حق نگویدش که بده
&&
چه جای نان ندهد هم به صد سنان حلوا
\\
\end{longtable}
\end{center}
