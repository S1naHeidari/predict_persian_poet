\begin{center}
\section*{غزل شماره ۲۰۷۵: توی که بدرقه باشی گهی گهی رهزن}
\label{sec:2075}
\addcontentsline{toc}{section}{\nameref{sec:2075}}
\begin{longtable}{l p{0.5cm} r}
توی که بدرقه باشی گهی گهی رهزن
&&
توی که خرمن مایی و آفت خرمن
\\
هزار جامه بدوزی ز عشق و پاره کنی
&&
و آنگهان بنویسی تو جرم آن بر من
\\
تو قلزمی و دو عالم ز توست یک قطره
&&
قراضه‌ای است دو عالم تویی دو صد معدن
\\
تو راست حکم که گویی به کور چشم گشا
&&
سخن تو بخشی و گویی که گفت آن الکن
\\
بساختی ز هوس صد هزار مغناطیس
&&
که نیست لایق آن سنگ خاص هر آهن
\\
مرا چو مست کشانی به سنگ و آهن خویش
&&
مرا چه کار که من جان روشنم یا تن
\\
تو باده‌ای تو خماری تو دشمنی و تو دوست
&&
هزار جان مقدس فدای این دشمن
\\
تو شمس دین به حقی و مفخر تبریز
&&
بهار جان که بدادی سزای صد بهمن
\\
\end{longtable}
\end{center}
