\begin{center}
\section*{غزل شماره ۱۲۶۶: روحیست بی‌نشان و ما غرقه در نشانش}
\label{sec:1266}
\addcontentsline{toc}{section}{\nameref{sec:1266}}
\begin{longtable}{l p{0.5cm} r}
روحیست بی‌نشان و ما غرقه در نشانش
&&
روحیست بی‌مکان و سر تا قدم مکانش
\\
خواهی که تا بیابی یک لحظه‌ای مجویش
&&
خواهی که تا بدانی یک لحظه‌ای مدانش
\\
چون در نهانش جویی دوری ز آشکارش
&&
چون آشکار جویی محجوبی از نهانش
\\
چون ز آشکار و پنهان بیرون شدی به برهان
&&
پاها دراز کن خوش می‌خسب در امانش
\\
چون تو ز ره بمانی جانی روانه گردد
&&
وانگه چه رحمت آید از جان و از روانش
\\
ای حبس کرده جان را تا کی کشی عنان را
&&
درتاز درجهانش اما نه در جهانش
\\
بی‌حرص کوب پایی از کوری حسد را
&&
زیرا حسد نگوید از حرص ترجمانش
\\
آخر ز بهر دو نان تا کی دوی چو دونان
&&
و آخر ز بهر سه نان تا کی خوری سنانش
\\
\end{longtable}
\end{center}
