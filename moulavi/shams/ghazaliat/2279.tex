\begin{center}
\section*{غزل شماره ۲۲۷۹: این کیست این این کیست این در حلقه ناگاه آمده}
\label{sec:2279}
\addcontentsline{toc}{section}{\nameref{sec:2279}}
\begin{longtable}{l p{0.5cm} r}
این کیست این این کیست این در حلقه ناگاه آمده
&&
این نور اللهی است این از پیش الله آمده
\\
این لطف و رحمت را نگر وین بخت و دولت را نگر
&&
در چاره بداختران با روی چون ماه آمده
\\
لیلی زیبا را نگر خوش طالب مجنون شده
&&
و آن کهربای روح بین در جذب هر کاه آمده
\\
از لذت بوهای او وز حسن و از خوهای او
&&
وز قل تعالوهای او جان‌ها به درگاه آمده
\\
صد نقش سازد بر عدم از چاکر و صاحب علم
&&
در دل خیالات خوشش زیبا و دلخواه آمده
\\
تخییل‌ها را آن صمد روزی حقیقت‌ها کند
&&
تا دررسد در زندگی اشکال گمراه آمده
\\
از چاه شور این جهان در دلو قرآن رو برآ
&&
ای یوسف آخر بهر توست این دلو در چاه آمده
\\
کی باشد ای گفت زبان من از تو مستغنی شده
&&
با آفتاب معرفت در سایه شاه آمده
\\
یا رب مرا پیش از اجل فارغ کن از علم و عمل
&&
خاصه ز علم منطقی در جمله افواه آمده
\\
\end{longtable}
\end{center}
