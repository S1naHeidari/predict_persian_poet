\begin{center}
\section*{غزل شماره ۱۷۳۵: به گوش من برسانید هجر تلخ پیام}
\label{sec:1735}
\addcontentsline{toc}{section}{\nameref{sec:1735}}
\begin{longtable}{l p{0.5cm} r}
به گوش من برسانید هجر تلخ پیام
&&
که خواب شیرین بر عاشقان شده‌ست حرام
\\
بکرد بر خور و بر خواب چارتکبیری
&&
هر آن کسی که بر او کرد عشق نیم سلام
\\
به من نگر که بدیدم هزار آزادی
&&
چو عشق را دل و جانم کنیزک است و غلام
\\
عظیم نور قدیم است عشق پیش خواص
&&
اگر چه صورت و شهوت بود به پیش عوام
\\
دلم چو زخم نیابد رود که توبه کند
&&
مخند بر من و بر خود کدام توبه کدام
\\
زهی گناه که کفر است توبه کردن از او
&&
نه پس طریق گریز و نه پیش جای مقام
\\
به چار مذهب خونش حلال و ریختنی
&&
از آنک عشق نریزد به غیر خون کرام
\\
بکش مرا که چو کشتی به عشق زنده شدم
&&
خموش کردم و مردم تمام گشت کلام
\\
\end{longtable}
\end{center}
