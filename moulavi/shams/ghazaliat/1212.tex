\begin{center}
\section*{غزل شماره ۱۲۱۲: دست بنه بر دلم از غم دلبر مپرس}
\label{sec:1212}
\addcontentsline{toc}{section}{\nameref{sec:1212}}
\begin{longtable}{l p{0.5cm} r}
دست بنه بر دلم از غم دلبر مپرس
&&
چشم من اندرنگر از می و ساغر مپرس
\\
جوشش خون را ببین از جگر مؤمنان
&&
وز ستم و ظلم آن طره کافر مپرس
\\
سکه شاهی ببین در رخ همچون زرم
&&
نقش تمامی بخوان پس تو ز زرگر مپرس
\\
عشق چو لشکر کشید عالم جان را گرفت
&&
حال من از عشق پرس از من مضطر مپرس
\\
هست دل عاشقان همچو دل مرغ از او
&&
جز سخن عاشقی نکته دیگر مپرس
\\
خاصیت مرغ چیست آنک ز روزن پرد
&&
گر تو چو مرغی بیا برپر و از در مپرس
\\
چون پدر و مادر عاشق هم عشق اوست
&&
بیش مگو از پدر بیش ز مادر مپرس
\\
هست دل عاشقان همچو تنوری به تاب
&&
چون به تنور آمدی جز که ز آذر مپرس
\\
مرغ دل تو اگر عاشق این آتشست
&&
سوخته پر خوشتری هیچ تو از پر مپرس
\\
گر تو و دلدار سر هر دو یکی کرده ایت
&&
پای دگر کژ منه خواجه از این سر مپرس
\\
دیده و گوش بشر دان که همه پرگلست
&&
از بصر پروحل گوهر منظر مپرس
\\
چونک بشستی بصر از مدد خون دل
&&
مجلس شاهی تو راست جز می احمر مپرس
\\
رو تو به تبریز زود از پی این شکر را
&&
با لطف شمس حق از می و شکر مپرس
\\
\end{longtable}
\end{center}
