\begin{center}
\section*{غزل شماره ۲۴۴۴: چون درشوی در باغ دل مانند گل خوش بو شوی}
\label{sec:2444}
\addcontentsline{toc}{section}{\nameref{sec:2444}}
\begin{longtable}{l p{0.5cm} r}
چون درشوی در باغ دل مانند گل خوش بو شوی
&&
چون برپری سوی فلک همچون ملک مه رو شوی
\\
گر همچو روغن سوزدت خود روشنی کردی همه
&&
سرخیل عشرت‌ها شوی گر چه ز غم چون مو شوی
\\
هم ملک و هم سلطان شوی هم خلد و هم رضوان شوی
&&
هم کفر و هم ایمان شوی هم شیر و هم آهو شوی
\\
از جای در بی‌جا روی وز خویشتن تنها روی
&&
بی‌مرکب و بی‌پا روی چون آب اندر جو شوی
\\
چون جان و دل یکتا شوی پیدای ناپیدا شوی
&&
هم تلخ و هم حلوا شوی با طبع می همخو شوی
\\
از طبع خشکی و تری همچون مسیحا برپری
&&
گرداب‌ها را بردری راهی کنی یک سو شوی
\\
شیرین کنی هر شور را حاضر کنی هر دور را
&&
پرده نباشی نور را گر چون فلک نه تو شوی
\\
شه باش دولت ساخته مه باش رفعت یافته
&&
تا چند همچون فاخته جوینده و کوکو شوی
\\
خالی کنی سر از هوس گردی تو زنده بی‌نفس
&&
یاهو نگویی زان سپس چون غرقه یاهو شوی
\\
هر خانه را روزن شوی هر باغ را گلشن شوی
&&
با من نباشی من شوی چون تو ز خود بی‌تو شوی
\\
سر در زمین چندین مکش سر را برآور شاد کش
&&
تا تازه و خندان و خوش چون شاخ شفتالو شوی
\\
دیگر نخواهی روشنی از خویشتن گردی غنی
&&
چون شاه مسکین پروری چون ماه ظلمت جو شوی
\\
تو جان نخواهی جان دهی هر درد را درمان دهی
&&
مرهم نجویی زخم را خود زخم را دارو شوی
\\
\end{longtable}
\end{center}
