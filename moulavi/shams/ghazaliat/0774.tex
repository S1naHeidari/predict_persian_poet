\begin{center}
\section*{غزل شماره ۷۷۴: به میان دل خیال مه دلگشا درآمد}
\label{sec:0774}
\addcontentsline{toc}{section}{\nameref{sec:0774}}
\begin{longtable}{l p{0.5cm} r}
به میان دل خیال مه دلگشا درآمد
&&
چو نه راه بود و نی در عجب از کجا درآمد
\\
بت و بت پرست و مؤمن همه در سجود رفتند
&&
چو بدان جمال و خوبی بت خوش لقا درآمد
\\
دل آهنم چو آتش چه خواست در منارش
&&
نه که آینه شود خوش چو در او صفا درآمد
\\
به چه نوع شکر گویم که شکرستان شکرم
&&
ز در جفا برون شد ز در وفا درآمد
\\
همه جورها وفا شد همه تیرگی صفا شد
&&
صفت بشر فنا شد صفت خدا درآمد
\\
همه نقش‌ها برون شد همه بحر آبگون شد
&&
همه کبریا برون شد همه کبریا درآمد
\\
همه خانه‌ها که آمد در آن به سوی دریا
&&
چو فزود موج دریا همه خانه‌ها درآمد
\\
همه خانه‌ها یکی شد دو مبین به آب بنگر
&&
که جدا نیند اگر چه که جدا جدا درآمد
\\
همه کوزه‌ها بیارید همه خنب‌ها بشویید
&&
که رسید آب حیوان و چنین سقا درآمد
\\
\end{longtable}
\end{center}
