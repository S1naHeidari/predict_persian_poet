\begin{center}
\section*{غزل شماره ۲۴۳۵: من دوش دیدم سر دل اندر جمال دلبری}
\label{sec:2435}
\addcontentsline{toc}{section}{\nameref{sec:2435}}
\begin{longtable}{l p{0.5cm} r}
من دوش دیدم سر دل اندر جمال دلبری
&&
سنگین دلی لعلین لبی ایمان فزایی کافری
\\
از جان و دل گوید کسی پیش چنان جانانه‌ای
&&
از سیم و زر گوید کسی پیش چنان سیمین بری
\\
لقمه شدی جمله جهان گر عشق را بودی دهان
&&
دربان شدی جان شهان گر عشق را بودی دری
\\
من می‌شنیدم نام دل ای جان و دل از تو خجل
&&
ای مانده اندر آب و گل از عشق دلدل چون خری
\\
ای جان بیا گوهر بچین ای دل بیا خوبی ببین
&&
المستغاث ای مسلمین زین آفتی شور و شری
\\
تن خود کی باشد تا بود فرش سواران غمش
&&
سر کیست تا او سر نهد پیش چنان شه سروری
\\
نک نوبهار آمد کز او سرسبز گردد عالمی
&&
چون یار من شیرین دمی چون لعل او حلواگری
\\
هر دم به من گوید رخش داری چو من زیبارخی
&&
هر دم بدو گوید دلم داری چو بنده چاکری
\\
آمد بهار ای دوستان خیزید سوی بوستان
&&
اما بهار من تویی من ننگرم در دیگری
\\
اشکوفه‌ها و میوه‌ها دارند غنج و شیوه‌ها
&&
ما در گلستان رخت روییده چون نیلوفری
\\
بلبل چو مطرب دف زنی برگ درختان کف زنی
&&
هر غنچه گوید چون منی باشد خوشی کشی تری
\\
آمد بهار مهربان سرسبز و خوش دامن کشان
&&
تا باغ یابد زینتی تا مرغ یابد شهپری
\\
تا خلق از او حیران شود تا یار من پنهان شود
&&
تا جان ما را جان شود کوری هر کور و کری
\\
آن جا که باشد شاه او بنده شود هر شاه خو
&&
آن جا که باشد ناز او هر دل شود سامندری
\\
مست و خرامان می‌رود در دل خیال یار من
&&
ماهی شریفی بی‌حدی شاهی کریمی بافری
\\
\end{longtable}
\end{center}
