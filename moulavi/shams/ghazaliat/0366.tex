\begin{center}
\section*{غزل شماره ۳۶۶: دود دل ما نشان سوداست}
\label{sec:0366}
\addcontentsline{toc}{section}{\nameref{sec:0366}}
\begin{longtable}{l p{0.5cm} r}
دود دل ما نشان سوداست
&&
وان دود که از دلست پیداست
\\
هر موج که می‌زند دل از خون
&&
آن دل نبود مگر که دریاست
\\
بیگانه شدند آشنایان
&&
دل نیز به دشمنی چه برخاست
\\
هر سوی که عشق رخت بنهاد
&&
هر جا که ملامت‌ست آن جاست
\\
ما نگریزیم از این ملامت
&&
زیرا که قدیم خانه ماست
\\
در عشق حسد برند شاهان
&&
زان روی که عشق شمع دل‌هاست
\\
پا بر سر چرخ هفتمین نه
&&
کاین عشق به حجره‌های بالاست
\\
هشیار مباش زان که هشیار
&&
در مجلس عشق سخت رسواست
\\
میری مطلب که میر مجلس
&&
گر چشم ببسته‌ست بیناست
\\
این عشق هنوز زیر چادر
&&
این گرد سیاه بین که برخاست
\\
هر چند که زیر هفت پرده‌ست
&&
پیداست که سخت خوب و زیباست
\\
شب خیز کنید ای حریفان
&&
شمعست و شراب و یار تنهاست
\\
\end{longtable}
\end{center}
