\begin{center}
\section*{غزل شماره ۱۶۸۵: ای مطرب این غزل گو کی یار توبه کردم}
\label{sec:1685}
\addcontentsline{toc}{section}{\nameref{sec:1685}}
\begin{longtable}{l p{0.5cm} r}
ای مطرب این غزل گو کی یار توبه کردم
&&
از هر گلی بریدم وز خار توبه کردم
\\
گه مست کار بودم گه در خمار بودم
&&
زان کار دست شستم زین کار توبه کردم
\\
در جرم توبه کردن بودیم تا به گردن
&&
از توبه‌های کرده این بار توبه کردم
\\
ای می فروش این ده ساغر به دست من ده
&&
من ننگ را شکستم وز عار توبه کردم
\\
مانند مست صرعم بیرون ز چار طبعم
&&
از گرم و سرد و خشکی هر چار توبه کردم
\\
ای مطرب الله الله می بی‌رهم تو بر ره
&&
بردار چنگ می زن بر تار توبه کردم
\\
ز اندیشه‌های چاره دل بود پاره پاره
&&
بیچارگی است چاره ناچار توبه کردم
\\
بنمای روی مه را خوش کن شب سیه را
&&
کز ذوق آن گنه را بسیار توبه کردم
\\
گفتم که وقت توبه‌ست شوریده‌ای مرا گفت
&&
من تایب قدیمم من پار توبه کردم
\\
بهر صلاح دین را محروسه یقین را
&&
منکر به عشق گوید ز انکار توبه کردم
\\
\end{longtable}
\end{center}
