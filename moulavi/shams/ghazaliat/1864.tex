\begin{center}
\section*{غزل شماره ۱۸۶۴: در پرده دل بنگر صد دختر آبستان}
\label{sec:1864}
\addcontentsline{toc}{section}{\nameref{sec:1864}}
\begin{longtable}{l p{0.5cm} r}
در پرده دل بنگر صد دختر آبستان
&&
زان گنجگه دل‌ها زان سجده گه مستان
\\
بشنو چه به اسرارم می آید از آن طارم
&&
یک دم که از این سو آ یک دم که قدح بستان
\\
در عربده افتاده از عشق چنین خوبان
&&
هم لشکر ترکستان هم لشکر هندستان
\\
از عقل بپرسیدم کاین شهره بتان چونند
&&
گفتا پنهان صورت پیدا به فن و دستان
\\
در شرق خداوندی شمس الحق تبریزی
&&
آیند و روند این‌ها در هر چمن و بستان
\\
\end{longtable}
\end{center}
