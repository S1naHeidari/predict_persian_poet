\begin{center}
\section*{غزل شماره ۳۰۶۱: اگر تو یار نداری چرا طلب نکنی}
\label{sec:3061}
\addcontentsline{toc}{section}{\nameref{sec:3061}}
\begin{longtable}{l p{0.5cm} r}
اگر تو یار نداری چرا طلب نکنی
&&
وگر به یار رسیدی چرا طرب نکنی
\\
وگر رفیق نسازد چرا تو او نشوی
&&
وگر رباب ننالد چراش ادب نکنی
\\
وگر حجاب شود مر تو را ابوجهلی
&&
چرا غزای ابوجهل و بولهب نکنی
\\
به کاهلی بنشینی که این عجب کاریست
&&
عجب تویی که هوای چنان عجب نکنی
\\
تو آفتاب جهانی چرا سیاه دلی
&&
که تا دگر هوس عقده ذنب نکنی
\\
مثال زر تو به کوره از آن گرفتاری
&&
که تا دگر طمع کیسه ذهب نکنی
\\
چو وحدتست عزبخانه یکی گویان
&&
تو روح را ز جز حق چرا عزب نکنی
\\
تو هیچ مجنون دیدی که با دو لیلی ساخت
&&
چرا هوای یکی روی و یک غبب نکنی
\\
شب وجود تو را در کمین چنان ماهیست
&&
چرا دعا و مناجات نیم شب نکنی
\\
اگر چه مست قدیمی و نوشراب نه‌ای
&&
شراب حق نگذارد که تو شغب نکنی
\\
شرابم آتش عشقست و خاصه از کف حق
&&
حرام باد حیاتت که جان حطب نکنی
\\
اگر چه موج سخن می‌زند ولیک آن به
&&
که شرح آن به دل و جان کنی به لب نکنی
\\
\end{longtable}
\end{center}
