\begin{center}
\section*{غزل شماره ۳۸: رستم از این نفس و هوا زنده بلا مرده بلا}
\label{sec:0038}
\addcontentsline{toc}{section}{\nameref{sec:0038}}
\begin{longtable}{l p{0.5cm} r}
رستم از این نفس و هوا زنده بلا مرده بلا
&&
زنده و مرده وطنم نیست به جز فضل خدا
\\
رستم از این بیت و غزل ای شه و سلطان ازل
&&
مفتعلن مفتعلن مفتعلن کشت مرا
\\
قافیه و مغلطه را گو همه سیلاب ببر
&&
پوست بود پوست بود درخور مغز شعرا
\\
ای خمشی مغز منی پرده آن نغز منی
&&
کمتر فضل خمشی کش نبود خوف و رجا
\\
بر ده ویران نبود عشر زمین کوچ و قلان
&&
مست و خرابم مطلب در سخنم نقد و خطا
\\
تا که خرابم نکند کی دهد آن گنج به من
&&
تا که به سیلم ندهد کی کشدم بحر عطا
\\
مرد سخن را چه خبر از خمشی همچو شکر
&&
خشک چه داند چه بود ترلللا ترلللا
\\
آینه‌ام آینه‌ام مرد مقالات نه‌ام
&&
دیده شود حال من ار چشم شود گوش شما
\\
دست فشانم چو شجر چرخ زنان همچو قمر
&&
چرخ من از رنگ زمین پاکتر از چرخ سما
\\
عارف گوینده بگو تا که دعای تو کنم
&&
چونک خوش و مست شوم هر سحری وقت دعا
\\
دلق من و خرقه من از تو دریغی نبود
&&
و آنک ز سلطان رسدم نیم مرا نیم تو را
\\
از کف سلطان رسدم ساغر و سغراق قدم
&&
چشمه خورشید بود جرعه او را چو گدا
\\
من خمشم خسته گلو عارف گوینده بگو
&&
زانک تو داود دمی من چو کهم رفته ز جا
\\
\end{longtable}
\end{center}
