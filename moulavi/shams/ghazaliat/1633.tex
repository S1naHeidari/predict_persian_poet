\begin{center}
\section*{غزل شماره ۱۶۳۳: در فروبند که ما عاشق این انجمنیم}
\label{sec:1633}
\addcontentsline{toc}{section}{\nameref{sec:1633}}
\begin{longtable}{l p{0.5cm} r}
در فروبند که ما عاشق این انجمنیم
&&
تا که با یار شکرلب نفسی دم بزنیم
\\
نقل و باده چه کم آید چو در این بزم دریم
&&
سرو و سوسن چه کم آید چو میان چمنیم
\\
باده تو به کف و باد تو اندر سر ماست
&&
فارغ از باد و بروت حسن و بوالحسنیم
\\
چو تویی مشعله ما ز تو شمع فلکیم
&&
چو تویی ساقی بگزیده گزین زمنیم
\\
رسن دام تو ما را چو رهانید ز چاه
&&
ما از آن روز رسن باز و حریف رسنیم
\\
عقل عقل و دل دل جان دو صد جان چو تویی
&&
واجب آید که به اقبال تو بر تن نتنیم
\\
چونک بر بام فلک از پی ما خیمه زدند
&&
ما از این خرگله خرگاه چرا برنکنیم
\\
همچو سیمرغ دعاییم که بر چرخ پریم
&&
همچو سرهنگ قضاییم که لشکر شکنیم
\\
ما چو سیلیم و تو دریا ز تو دور افتادیم
&&
به سر و روی دوان گشته به سوی وطنیم
\\
روکشان نعره زنانیم در این راه چو سیل
&&
نه چو گردابه گندیده به خود مرتهنیم
\\
هین از آن رطل گران ده سبکم بیش مگو
&&
ور بگویی تو همین گو که غریق مننیم
\\
شمس تبریز که سرمایه لعل است و عقیق
&&
ما از او لعل بدخشان و عقیق یمنیم
\\
\end{longtable}
\end{center}
