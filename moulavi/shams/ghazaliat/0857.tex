\begin{center}
\section*{غزل شماره ۸۵۷: گفتی که در چه کاری با تو چه کار ماند}
\label{sec:0857}
\addcontentsline{toc}{section}{\nameref{sec:0857}}
\begin{longtable}{l p{0.5cm} r}
گفتی که در چه کاری با تو چه کار ماند
&&
کاری که بی‌تو گیرم والله که زار ماند
\\
گر خمر خلد نوشم با جام‌های زرین
&&
جمله صداع گردد جمله خمار ماند
\\
در کارگاه عشقت بی‌تو هر آنچ بافم
&&
والله نه پود ماند والله نه تار ماند
\\
تو جوی بی‌کرانی پیشت جهان چو پولی
&&
حاشا که با چنین جو بر پل گذار ماند
\\
عالم چهار فصلست فصلی خلاف فصلی
&&
با جنگ چار دشمن هرگز قرار ماند
\\
پیش آ بهار خوبی تو اصل فصل‌هایی
&&
تا فصل‌ها بسوزد جمله بهار ماند
\\
\end{longtable}
\end{center}
