\begin{center}
\section*{غزل شماره ۱۳۰۴: باده نمی‌بایدم فارغم از درد و صاف}
\label{sec:1304}
\addcontentsline{toc}{section}{\nameref{sec:1304}}
\begin{longtable}{l p{0.5cm} r}
باده نمی‌بایدم فارغم از درد و صاف
&&
تشنه خون خودم آمد وقت مصاف
\\
برکش شمشیر تیز خون حسودان بریز
&&
تا سر بی‌تن کند گرد تن خود طواف
\\
کوه کن از کله‌ها بحر کن از خون ما
&&
تا بخورد خاک و ریگ جرعه خون از گزاف
\\
ای ز دل من خبیر رو دهنم را مگیر
&&
ور نه شکافد دلم خون بجهد از شکاف
\\
گوش به غوغا مکن هیچ محابا مکن
&&
سلطنت و قهرمان نیست چنین دست باف
\\
در دل آتش روم لقمه آتش شوم
&&
جان چو کبریت را بر چه بریدند ناف
\\
آتش فرزند ماست تشنه و دربند ماست
&&
هر دو یکی می‌شویم تا نبود اختلاف
\\
چک چک و دودش چراست زانک دورنگی به جاست
&&
چونک شود هیزم او چک چک نبود ز لاف
\\
ور بجهد نیم سوز فحم بود او هنوز
&&
تشنه دل و رو سیه طالب وصل و زفاف
\\
آتش گوید برو تو سیهی من سپید
&&
هیزم گوید که تو سوخته‌ای من معاف
\\
این طرفش روی نی وان طرفش روی نی
&&
کرده میان دو یار در سیهی اعتکاف
\\
همچو مسلمان غریب نی سوی خلقش رهی
&&
نی سوی شاهنشهی بر طرفی چون سجاف
\\
بلک چو عنقا که او از همه مرغان فزود
&&
بر فلکش ره نبود ماند بر آن کوه قاف
\\
با تو چه گویم که تو در غم نان مانده‌ای
&&
پشت خمی همچو لام تنگ دلی همچو کاف
\\
هین بزن ای فتنه جو بر سر سنگ آن سبو
&&
تا نکشم آب جو تا نکنم اغتراف
\\
ترک سقایی کنم غرقه دریا شوم
&&
دور ز جنگ و خلاف بی‌خبر از اعتراف
\\
همچو روان‌های پاک خامش در زیر خاک
&&
قالبشان چون عروس خاک بر او چون لحاف
\\
\end{longtable}
\end{center}
