\begin{center}
\section*{غزل شماره ۵۳۶: آمد بهار عاشقان تا خاکدان بستان شود}
\label{sec:0536}
\addcontentsline{toc}{section}{\nameref{sec:0536}}
\begin{longtable}{l p{0.5cm} r}
آمد بهار عاشقان تا خاکدان بستان شود
&&
آمد ندای آسمان تا مرغ جان پران شود
\\
هم بحر پرگوهر شود هم شوره چون گوهر شود
&&
هم سنگ لعل کان شود هم جسم جمله جان شود
\\
گر چشم و جان عاشقان چون ابر طوفان بار شد
&&
اما دل اندر ابر تن چون برق‌ها رخشان شود
\\
دانی چرا چون ابر شد در عشق چشم عاشقان
&&
زیرا که آن مه بیشتر در ابرها پنهان شود
\\
ای شاد و خندان ساعتی کان ابرها گرینده شد
&&
یا رب خجسته حالتی کان برق‌ها خندان شود
\\
زان صد هزاران قطره‌ها یک قطره ناید بر زمین
&&
ور زانک آید بر زمین جمله جهان ویران شود
\\
جمله جهان ویران شود وز عشق هر ویرانه‌ای
&&
با نوح هم کشتی شود پس محرم طوفان شود
\\
طوفان اگر ساکن بدی گردان نبودی آسمان
&&
زان موج بیرون از جهت این شش جهت جنبان شود
\\
ای مانده زیر شش جهت هم غم بخور هم غم مخور
&&
کان دانه‌ها زیر زمین یک روز نخلستان شود
\\
از خاک روزی سر کند آن بیخ شاخ تر کند
&&
شاخی دو سه گر خشک شد باقیش آبستان شود
\\
وان خشک چون آتش شود آتش چو جان هم خوش شود
&&
آن این نباشد این شود این آن نباشد آن شود
\\
چیزی دهانم را ببست یعنی کنار بام و مست
&&
هر چه تو زان حیران شوی آن چیز از او حیران شود
\\
\end{longtable}
\end{center}
