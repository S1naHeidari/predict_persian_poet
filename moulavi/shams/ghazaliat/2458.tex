\begin{center}
\section*{غزل شماره ۲۴۵۸: سنگ مزن بر طرف کارگه شیشه گری}
\label{sec:2458}
\addcontentsline{toc}{section}{\nameref{sec:2458}}
\begin{longtable}{l p{0.5cm} r}
سنگ مزن بر طرف کارگه شیشه گری
&&
زخم مزن بر جگر خسته خسته جگری
\\
بر دل من زن همه را ز آنک دریغ است و غبین
&&
زخم تو و سنگ تو بر سینه و جان دگری
\\
بازرهان جمله اسیران جفا را جز من
&&
تا به جفا هم نکنی در جز بنده نظری
\\
هم به وفا با تو خوشم هم به جفا با تو خوشم
&&
نی به وفا نی به جفا بی‌تو مبادم سفری
\\
چونک خیالت نبود آمده در چشم کسی
&&
چشم بز کشته بود تیره و خیره نگری
\\
پیش ز زندان جهان با تو بدم من همگی
&&
کاش بر این دامگهم هیچ نبودی گذری
\\
چند بگفتم که خوشم هیچ سفر می‌نروم
&&
این سفر صعب نگر ره ز علی تا به ثری
\\
لطف تو بفریفت مرا گفت برو هیچ مرم
&&
بدرقه باشد کرمم بر تو نباشد خطری
\\
چون به غریبی بروی فرجه کنی پخته شوی
&&
بازبیایی به وطن باخبری پرهنری
\\
گفتم ای جان خبر بی‌تو خبر را چه کنم
&&
بهر خبر خود که رود از تو مگر بی‌خبری
\\
چون ز کفت باده کشم بی‌خبر و مست و خوشم
&&
بی‌خطر و خوف کسی بی‌شر و شور بشری
\\
گفت به گوشم سخنان چون سخن راه زنان
&&
برد مرا شاه ز سر کرد مرا خیره سری
\\
قصه دراز است بلی آه ز مکر و دغلی
&&
گر ننماید کرمش این شب ما را سحری
\\
\end{longtable}
\end{center}
