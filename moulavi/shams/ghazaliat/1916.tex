\begin{center}
\section*{غزل شماره ۱۹۱۶: برآ بر بام و اکنون ماه نو بین}
\label{sec:1916}
\addcontentsline{toc}{section}{\nameref{sec:1916}}
\begin{longtable}{l p{0.5cm} r}
برآ بر بام و اکنون ماه نو بین
&&
درآ در باغ و اکنون سیب می چین
\\
از آن سیبی که بشکافد در روم
&&
رود بوی خوشش تا چین و ماچین
\\
برآ بر خرمن سیب و بکش پا
&&
ز سیب لعل کن فرش و نهالین
\\
اگر سیبش لقب گویم وگر می
&&
وگر نرگس وگر گلزار و نسرین
\\
یکی چیز است در وی چیست کان نیست
&&
خدا پاینده دارش یا رب آمین
\\
بیا اکنون اگر افسانه خواهی
&&
درآ در پیش من چون شمع بنشین
\\
همی‌ترسم که بگریزی ز گوشه
&&
برآ بالا برون انداز نعلین
\\
به پهلویم نشین برچفس بر من
&&
رها کن ناز و آن خوهای پیشین
\\
بیامیز اندکی ای کان رحمت
&&
که تا گردد رخ زرد تو رنگین
\\
روا باشد وگر خود من نگویم
&&
همیشه عشوه و وعده دروغین
\\
از این پاکی تو لیکن عاشقان را
&&
پراکنده سخن‌ها هست آیین
\\
زهی اوصاف شمس الدین تبریز
&&
زهی کر و فر و امکان و تمکین
\\
\end{longtable}
\end{center}
