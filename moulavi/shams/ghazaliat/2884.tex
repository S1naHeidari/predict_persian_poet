\begin{center}
\section*{غزل شماره ۲۸۸۴: به شکرخنده بتا نرخ شکر می‌شکنی}
\label{sec:2884}
\addcontentsline{toc}{section}{\nameref{sec:2884}}
\begin{longtable}{l p{0.5cm} r}
به شکرخنده بتا نرخ شکر می‌شکنی
&&
چه زند پیش عقیق تو عقیق یمنی
\\
گلرخا سوی گلستان دو سه هفته بمرو
&&
تا ز شرم تو نریزد گل سرخ چمنی
\\
گل چه باشد که اگر جانب گردون نگری
&&
سرنگون زهره و مه را ز فلک درفکنی
\\
حق تو را از جهت فتنه و شور آورده‌ست
&&
فتنه و شور و قیامت نکنی پس چه کنی
\\
روی چون آتش از آن داد که دل‌ها سوزی
&&
شکن زلف بدان داد که دل‌ها شکنی
\\
دل ما بتکده‌ها نقش تو در وی شمنی
&&
هر بتی رو به شمن کرده که تو آن منی
\\
برمکن تو دل خود از من ازیرا به جفا
&&
گر که قاف شود دل تو ز بیخش بکنی
\\
در تک چاه زنخدان تو نادر آبی است
&&
که به هر چه که درافتم بنماید رسنی
\\
در غمت بوالحسنان مذهب و دین گم کردند
&&
زان سبب که حسن اندر حسن اندر حسنی
\\
زیرکان را رخ تو مست از آن می‌دارد
&&
تا در این بزم ندانند که تو در چه فنی
\\
کافری ای دل اگر در جز او دل بندی
&&
کافری ای تن اگر بر جز این عشق تنی
\\
بی وی ار بر فلکی تو به خدا در گوری
&&
هر چه پوشی به جز از خلعت او در کفنی
\\
شمس تبریز که در روح وطن ساخته‌ای
&&
جان جان‌هاست وطن چونک تو جان را وطنی
\\
\end{longtable}
\end{center}
