\begin{center}
\section*{غزل شماره ۴۶۵: کار ندارم جز این کارگه و کارم اوست}
\label{sec:0465}
\addcontentsline{toc}{section}{\nameref{sec:0465}}
\begin{longtable}{l p{0.5cm} r}
کار ندارم جز این کارگه و کارم اوست
&&
لاف زنم لاف لاف چونک خریدارم اوست
\\
طوطی گویا شدم چون شکرستانم اوست
&&
بلبل بویا شدم چون گل و گلزارم اوست
\\
پر به ملک برزنم چون پر و بالم از اوست
&&
سر به فلک برزنم چون سر و دستارم اوست
\\
جان و دلم ساکنست زانک دل و جانم اوست
&&
قافله‌ام ایمنست قافله سالارم اوست
\\
بر مثل گلستان رنگرزم خم اوست
&&
بر مثل آفتاب تیغ گهردارم اوست
\\
خانه جسمم چرا سجده گه خلق شد
&&
زانک به روز و به شب بر در و دیوارم اوست
\\
دست به دست جز او می‌نسپارد دلم
&&
زانک طبیب غم این دل بیمارم اوست
\\
بر رخ هر کس که نیست داغ غلامی او
&&
گر پدر من بود دشمن و اغیارم اوست
\\
ای که تو مفلس شدی سنگ به دل برزدی
&&
صله ز من خواه زانک مخزن و انبارم اوست
\\
شاه مرا خوانده است چون نروم پیش شاه
&&
منکر او چون شوم چون همه اقرارم اوست
\\
گفت خمش چند چند لاف تو و گفت تو
&&
من چه کنم ای عزیز گفتن بسیارم اوست
\\
\end{longtable}
\end{center}
