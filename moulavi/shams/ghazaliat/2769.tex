\begin{center}
\section*{غزل شماره ۲۷۶۹: برخیز و بزن یکی نوایی}
\label{sec:2769}
\addcontentsline{toc}{section}{\nameref{sec:2769}}
\begin{longtable}{l p{0.5cm} r}
برخیز و بزن یکی نوایی
&&
بر یاد وصال دلربایی
\\
هین وقت صبوح شد فتوحی
&&
هین وقت دعاست الصلایی
\\
بگشا سر خنب خسروانی
&&
تا خلق زنند دست و پایی
\\
صد گون گره است بر دل و نیست
&&
جز باده جان گره گشایی
\\
از جای ببر به یک قنینه
&&
آن را که قرار نیست جایی
\\
جز دشت عدم قرارگه نیست
&&
چون نیست وجود را وفایی
\\
بر سفره خاک تره‌ای نیست
&&
هر سوی ز چیست ژاژخایی
\\
عالم مردار و عامه چون سگ
&&
کی دید ز دست سگ سخایی
\\
ساقی درده صلا که چون تو
&&
جان‌ها بندید جان فزایی
\\
ما چون مس و آهنیم ثابت
&&
در حیرت چون تو کیمیایی
\\
در مغز فکن تو هوی هویی
&&
وز خلق برآر های هایی
\\
تا روح ز مستی و خرابی
&&
نشناسد هجو از ثنایی
\\
زین باده چو مست شد فلاطون
&&
نشناسد درد از دوایی
\\
دردی ده و عقل را چنان کن
&&
کو درد نداند از صفایی
\\
بر ناطق منطقی فروریز
&&
از جام صبوحیان عطایی
\\
تا دم نزند دگر نجوید
&&
زنبیل و فطیر هر گدایی
\\
خامش که تو را مسلم آمد
&&
برساختن از عدم بقایی
\\
\end{longtable}
\end{center}
