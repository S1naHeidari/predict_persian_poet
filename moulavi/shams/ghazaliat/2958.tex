\begin{center}
\section*{غزل شماره ۲۹۵۸: مطرب چو زخمه‌ها را بر تار می‌کشانی}
\label{sec:2958}
\addcontentsline{toc}{section}{\nameref{sec:2958}}
\begin{longtable}{l p{0.5cm} r}
مطرب چو زخمه‌ها را بر تار می‌کشانی
&&
این کاهلان ره را در کار می‌کشانی
\\
ای عشق چون درآیی در عالم جدایی
&&
این بازماندگان را تا یار می‌کشانی
\\
کوری رهزنان را ایمن کنی جهان را
&&
دزدان شهر دل را بر دار می‌کشانی
\\
مکار را ببینی کورش کنی به مکری
&&
چون یار را ببینی در غار می‌کشانی
\\
بر تازیان چابک بندی تو زین زرین
&&
پالانیان بد را در بار می‌کشانی
\\
سوداییان ما را هر لحظه می‌نوازی
&&
بازاریان ما را بس زار می‌کشانی
\\
عشاق خارکش را گلزار می‌نمایی
&&
خودکام گل طرب را در خار می‌کشانی
\\
آن کو در آتش آید راهش دهی به آبی
&&
و آن کو دود به آبی در نار می‌کشانی
\\
موسی خاک رو را ره می‌دهی به عزت
&&
فرعون بوش جو را در عار می‌کشانی
\\
این نعل بازگونه بی‌چون و بی‌چگونه
&&
موسی عصاطلب را در مار می‌کشانی
\\
\end{longtable}
\end{center}
