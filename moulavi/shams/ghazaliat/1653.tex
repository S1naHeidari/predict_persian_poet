\begin{center}
\section*{غزل شماره ۱۶۵۳: من از این خانه پرنور به در می نروم}
\label{sec:1653}
\addcontentsline{toc}{section}{\nameref{sec:1653}}
\begin{longtable}{l p{0.5cm} r}
من از این خانه پرنور به در می نروم
&&
من از این شهر مبارک به سفر می نروم
\\
منم و این صنم و عاشقی و باقی عمر
&&
من از او گر بکشی جای دگر می نروم
\\
گر جهان بحر شود موج زند سرتاسر
&&
من به جز جانب آن گنج گهر می نروم
\\
شهر ما تختگه و مجلس آن سلطان است
&&
من ز سلطان سلاطین به حشر می نروم
\\
شهر ما از شه ما کان عقیق و گهر است
&&
من ز گنجینه گوهر به حجر می نروم
\\
شهر ما از شه ما جنت و فردوس خوش است
&&
من ز فردوس و ز جنت به سقر می نروم
\\
شهر پر شد که فلان بن فلان می برود
&&
شهر اراجیف چرا پر شد اگر می نروم
\\
این خبر رفت به هر سوی و به هر گوش رسید
&&
من از این بی‌خبری سوی خبر می نروم
\\
یار ما جان و خداوند قضا و قدر است
&&
من از این جان قدر جز به قدر می نروم
\\
تو مسافر شده‌ای تا که مگر سود کنی
&&
من از این سود حقیقت به مگر می نروم
\\
مغز را یافته‌ام پوست نخواهم خایید
&&
ایمنی یافته‌ام سوی خطر می نروم
\\
تو جگرگوشه مایی برو الله معک
&&
من چو دل یافته‌ام سوی جگر می نروم
\\
تو کمربسته چو موری پی حرص روزی
&&
من فکنده کله و سوی کمر می نروم
\\
نشنوم پند کسی پندم مده جان پدر
&&
من پدر یافته‌ام سوی پدر می نروم
\\
شمس تبریز مرا طالع زهره داده‌ست
&&
تا چو زهره همه شب جز به بطر می نروم
\\
\end{longtable}
\end{center}
