\begin{center}
\section*{غزل شماره ۲۳۴۵: چو بی‌گاه است و باران خانه خانه}
\label{sec:2345}
\addcontentsline{toc}{section}{\nameref{sec:2345}}
\begin{longtable}{l p{0.5cm} r}
چو بی‌گاه است و باران خانه خانه
&&
صلای جمله یاران خانه خانه
\\
چو جغدان چند این محروم بودن
&&
به گرداگرد ویران خانه خانه
\\
ایا اصحاب روشن دل شتابید
&&
به کوری جمله کوران خانه خانه
\\
ایا ای عاقل هشیار پرغم
&&
دل ما را مشوران خانه خانه
\\
به نقش دیو چند این عشقبازی
&&
لقبشان کرده حوران خانه خانه
\\
بدیدی دانه و خرمن ندیدی
&&
بدین حالند موران خانه خانه
\\
مکن چون و چرا بگذار یارا
&&
چرا را با ستوران خانه خانه
\\
در آن خانه سماع ختنه سور است
&&
ولیکن با طهوران خانه خانه
\\
بنا کرده‌ست شمس الدین تبریز
&&
برای جمع عوران خانه خانه
\\
\end{longtable}
\end{center}
