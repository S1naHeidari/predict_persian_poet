\begin{center}
\section*{غزل شماره ۱۶۰۷: ز فلک قوت بگیرم دهن از لوت ببندم}
\label{sec:1607}
\addcontentsline{toc}{section}{\nameref{sec:1607}}
\begin{longtable}{l p{0.5cm} r}
ز فلک قوت بگیرم دهن از لوت ببندم
&&
شکم ار زار بگرید من عیار بخندم
\\
مثل بلبل مستم قفس خویش شکستم
&&
سوی بالا بپریدم که من از چرخ بلندم
\\
نه چنان مست و خرابم که خورد آتش و آبم
&&
همگی غرق جنونم همگی سلسله مندم
\\
کله ار رفت بر او گو نه کلم سلسله مویم
&&
خر اگر مرد بر او گو که بر این پشت سمندم
\\
همه پرباد از آنم که منم نای و تو نایی
&&
چو تویی خویش من ای جان پی این خویش پسندم
\\
ز پی قند و نبات تو بسی طبله شکستم
&&
ز پی آب حیات تو بسی جوی بکندم
\\
چو تویی روح جهان را جهت چشم بدان را
&&
اگرم پاک بسوزی سزد ایرا که سپندم
\\
اگر از سوز چو عودم وگر از ساز چو عیدم
&&
نه از آن عید بخندم نه از این عود برندم
\\
سر سودای تو دارم سر اندیشه نخارم
&&
خبرم نیست که چونم نظرم نیست که چندم
\\
ترشی نیست در آن خد ترش او کرد به قاصد
&&
که اگر روترشم من نه همان شهدم و قندم
\\
چو دلم مست تو باشد همه جان‌هاست غلامم
&&
وگر از دست تو آید نکند زهر گزندم
\\
طرف سدره جان را تو فروکش به کفم نه
&&
سوی آن قلعه عالی تو برانداز کمندم
\\
نه بر این دخل بچفسم نه از این چرخ بترسم
&&
چو فزون خرج کنم من نه فزون دخل دهندم
\\
\end{longtable}
\end{center}
