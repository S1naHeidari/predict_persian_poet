\begin{center}
\section*{غزل شماره ۱۲۱۹: یار نخواهم که بود بدخو و غمخوار و ترش}
\label{sec:1219}
\addcontentsline{toc}{section}{\nameref{sec:1219}}
\begin{longtable}{l p{0.5cm} r}
یار نخواهم که بود بدخو و غمخوار و ترش
&&
چون لحد و گور مغان تنگ و دل افشار و ترش
\\
یار چو آیینه بود دوست چو لوزینه بود
&&
ساعت یاری نبود خایف و فرار و ترش
\\
هر کی بود عاشق خود پنج نشان دارد بد
&&
سخت دل و سست قدم کاهل و بی‌کار و ترش
\\
ور چشمش بیش بود هم ترشی بیش کند
&&
دان مثل بیشی او سرکه بسیار ترش
\\
بس کن شرح ترشان این قدری بهر نشان
&&
کی طلبد در دل و جان طبع شکربار ترش
\\
\end{longtable}
\end{center}
