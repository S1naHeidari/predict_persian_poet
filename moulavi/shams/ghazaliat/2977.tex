\begin{center}
\section*{غزل شماره ۲۹۷۷: هر روز بامداد درآید یکی پری}
\label{sec:2977}
\addcontentsline{toc}{section}{\nameref{sec:2977}}
\begin{longtable}{l p{0.5cm} r}
هر روز بامداد درآید یکی پری
&&
بیرون کشد مرا که ز من جان کجا بری
\\
گر عاشقی نیابی مانند من بتی
&&
ور تاجری کجاست چو من گرم مشتری
\\
ور عارفی حقیقت معروف جان منم
&&
ور کاهلی چنان شوی از من که برپری
\\
ور حس فاسدی دهمت نور مصطفی
&&
ور مس کاسدی کنمت زر جعفری
\\
محتاج روی مایی گر پشت عالمی
&&
محتاج آفتابی گر صبح انوری
\\
از بر و بحر بگذر و بر کوه قاف رو
&&
بر خشک و بر تری منشین زین دو برتری
\\
ای دل اگر دلی دل از آن یار درمدزد
&&
وی سر اگر سری مکن این سجده سرسری
\\
چون اسب می‌گریزی و من بر توام سوار
&&
مگریز از او که بر تو بود کان بود خری
\\
صد حیله گر تراشی و صد شهر اگر روی
&&
قربان عید خنجر الله اکبری
\\
خاموش اگر چه بحر دهد در بی‌دریغ
&&
لیکن مباح نیست که من رام یشتری
\\
\end{longtable}
\end{center}
