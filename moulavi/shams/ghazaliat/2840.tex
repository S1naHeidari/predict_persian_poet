\begin{center}
\section*{غزل شماره ۲۸۴۰: منگر به هر گدایی که تو خاص از آن مایی}
\label{sec:2840}
\addcontentsline{toc}{section}{\nameref{sec:2840}}
\begin{longtable}{l p{0.5cm} r}
منگر به هر گدایی که تو خاص از آن مایی
&&
مفروش خویش ارزان که تو بس گران بهایی
\\
به عصا شکاف دریا که تو موسی زمانی
&&
بدران قبای مه را که ز نور مصطفایی
\\
بشکن سبوی خوبان که تو یوسف جمالی
&&
چو مسیح دم روان کن که تو نیز از آن هوایی
\\
به صف اندرآی تنها که سفندیار وقتی
&&
در خیبر است برکن که علی مرتضایی
\\
بستان ز دیو خاتم که تویی به جان سلیمان
&&
بشکن سپاه اختر که تو آفتاب رایی
\\
چو خلیل رو در آتش که تو خالصی و دلخوش
&&
چو خضر خور آب حیوان که تو جوهر بقایی
\\
بسکل ز بی‌اصولان مشنو فریب غولان
&&
که تو از شریف اصلی که تو از بلند جایی
\\
تو به روح بی‌زوالی ز درونه باجمالی
&&
تو از آن ذوالجلالی تو ز پرتو خدایی
\\
تو هنوز ناپدیدی ز جمال خود چه دیدی
&&
سحری چو آفتابی ز درون خود برآیی
\\
تو چنین نهان دریغی که مهی به زیر میغی
&&
بدران تو میغ تن را که مهی و خوش لقایی
\\
چو تو لعل کان ندارد چو تو جان جهان ندارد
&&
که جهان کاهش است این و تو جان جان فزایی
\\
تو چو تیغ ذوالفقاری تن تو غلاف چوبین
&&
اگر این غلاف بشکست تو شکسته دل چرایی
\\
تو چو باز پای بسته تن تو چو کنده بر پا
&&
تو به چنگ خویش باید که گره ز پا گشایی
\\
چه خوش است زر خالص چو به آتش اندرآید
&&
چو کند درون آتش هنر و گهرنمایی
\\
مگریز ای برادر تو ز شعله‌های آذر
&&
ز برای امتحان را چه شود اگر درآیی
\\
به خدا تو را نسوزد رخ تو چو زر فروزد
&&
که خلیل زاده‌ای تو ز قدیم آشنایی
\\
تو ز خاک سر برآور که درخت سربلندی
&&
تو بپر به قاف قربت که شریفتر همایی
\\
ز غلاف خود برون آ که تو تیغ آبداری
&&
ز کمین کان برون آ که تو نقد بس روایی
\\
شکری شکرفشان کن که تو قند نوشقندی
&&
بنواز نای دولت که عظیم خوش نوایی
\\
\end{longtable}
\end{center}
