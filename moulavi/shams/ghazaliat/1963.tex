\begin{center}
\section*{غزل شماره ۱۹۶۳: پرده بردار ای حیات جان و جان افزای من}
\label{sec:1963}
\addcontentsline{toc}{section}{\nameref{sec:1963}}
\begin{longtable}{l p{0.5cm} r}
پرده بردار ای حیات جان و جان افزای من
&&
غمگسار و همنشین و مونس شب‌های من
\\
ای شنیده وقت و بی‌وقت از وجودم ناله‌ها
&&
ای فکنده آتشی در جمله اجزای من
\\
در صدای کوه افتد بانگ من چون بشنوی
&&
جفت گردد بانگ که با نعره و هیهای من
\\
ای ز هر نقشی تو پاک و ای ز جان‌ها پاکتر
&&
صورتت نی لیک مغناطیس صورت‌های من
\\
چون ز بی‌ذوقی دل من طالب کاری بود
&&
بسته باشم گر چه باشد دلگشا صحرای من
\\
بی تو باشد جیش و عیش و باغ و راغ و نقل و عقل
&&
هر یکی رنج دماغ و کنده‌ای بر پای من
\\
تا ز خود افزون گریزم در خودم محبوستر
&&
تا گشایم بند از پا بسته بینم پای من
\\
ناگهان در ناامیدی یا شبی یا بامداد
&&
گوییم اینک برآ بر طارم بالای من
\\
آن زمان از شکر و حلوا چنان گردم که من
&&
گم کنم کاین خود منم یا شکر و حلوای من
\\
امشب از شب‌های تنهایی است رحمی کن بیا
&&
تا بخوانم بر تو امشب دفتر سودای من
\\
همچو نای انبان در این شب من از آن خالی شدم
&&
تا خوش و صافی برآید ناله‌ها و وای من
\\
زین سپس انبان بادم نیستم انبان نان
&&
زانک از این ناله است روشن این دل بینای من
\\
درد و رنجوری ما را داروی غیر تو نیست
&&
ای تو جالینوس جان و بوعلی سینای من
\\
\end{longtable}
\end{center}
