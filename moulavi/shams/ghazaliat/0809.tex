\begin{center}
\section*{غزل شماره ۸۰۹: طرفه گرمابه بانی کو ز خلوت برآید}
\label{sec:0809}
\addcontentsline{toc}{section}{\nameref{sec:0809}}
\begin{longtable}{l p{0.5cm} r}
طرفه گرمابه بانی کو ز خلوت برآید
&&
نقش گرمابه یک یک در سجود اندرآید
\\
نقش‌های فسرده بی‌خبروار مرده
&&
ز انعکاسات چشمش چشمشان عبهر آید
\\
گوش‌هاشان ز گوشش اهل افسانه گردد
&&
چشم‌هاشان ز چشمش قابل منظر آید
\\
نقش گرمابه بینی هر یکی مست و رقصان
&&
چون معاشر که گه گه در می احمر آید
\\
پر شده بانگ و نعره صحن گرمابه ز ایشان
&&
کز هیاهوی و غلغل غره محشر آید
\\
نقش‌ها یک دگر را جانب خویش خوانند
&&
نقش از آن گوشه خندان سوی این دیگر آید
\\
لیک گرمابه بان را صورتی درنیابد
&&
گر چه صورت ز جستن در کر و در فر آید
\\
جمله گشته پریشان او پس و پیش ایشان
&&
ناشناسا شه جان بر سر لشکر آید
\\
گلشن هر ضمیری از رخش پرگل آید
&&
دامن هر فقیری از کفش پرزر آید
\\
دار زنبیل پیشش تا کند پر ز خویشش
&&
تا که زنبیل فقرت حسرت سنجر آید
\\
برهد از بیش وز کم قاضی و مدعی هم
&&
چونک آن ماه یک دم مست در محضر آید
\\
باده خمخانه گردد مرده مستانه گردد
&&
چوب حنانه گردد چونک بر منبر آید
\\
کم کند از لقاشان بفسرد نقش‌هاشان
&&
گم شود چشم‌هاشان گوش‌هاشان کر آید
\\
باز چون رو نماید چشم‌ها برگشاید
&&
باغ پرمرغ گردد بوستان اخضر آید
\\
رو به گلزار و بستان دوستان بین و دستان
&&
در پی این عبارت جان بدان معبر آید
\\
آنچ شد آشکارا کی توان گفت یارا
&&
کلک آن کی نویسد گر چه در محبر آید
\\
\end{longtable}
\end{center}
