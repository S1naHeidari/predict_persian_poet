\begin{center}
\section*{غزل شماره ۱۷۳۹: زهی حلاوت پنهان در این خلای شکم}
\label{sec:1739}
\addcontentsline{toc}{section}{\nameref{sec:1739}}
\begin{longtable}{l p{0.5cm} r}
زهی حلاوت پنهان در این خلای شکم
&&
مثال چنگ بود آدمی نه بیش و نه کم
\\
چنانک گر شکم چنگ پر شود مثلا
&&
نه ناله آید از آن چنگ پر نه زیر و نه بم
\\
اگر ز روزه بسوزد دماغ و اشکم تو
&&
ز سوز ناله برآید ز سینه‌ات هر دم
\\
هزار پرده بسوزی به هر دمی زان سوز
&&
هزار پایه برآری به همت و به قدم
\\
شکم تهی شو و می نال همچو نی به نیاز
&&
شکم تهی شو و اسرار گو به سان قلم
\\
چو پر شود شکمت در زمان حشر آرد
&&
به جای عقل تو شیطان به جای کعبه صنم
\\
چو روزه داری اخلاق خوب جمع شوند
&&
به پیش تو چو غلامان و چاکران و حشم
\\
به روزه باش که آن خاتم سلیمان است
&&
مده به دیو تو خاتم مزن تو ملک به هم
\\
وگر ز کف تو شد ملک و لشکرت بگریخت
&&
فرازآید لشکرت بر فراز علم
\\
رسید مایده از آسمان به اهل صیام
&&
به اهتمام دعاهای عیسی مریم
\\
به روزه خوان کرم را تو منتظر می باش
&&
از آنک خوان کرم به ز شوربای کلم
\\
\end{longtable}
\end{center}
