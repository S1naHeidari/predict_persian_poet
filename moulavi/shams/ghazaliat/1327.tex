\begin{center}
\section*{غزل شماره ۱۳۲۷: تتار اگر چه جهان را خراب کرد به جنگ}
\label{sec:1327}
\addcontentsline{toc}{section}{\nameref{sec:1327}}
\begin{longtable}{l p{0.5cm} r}
تتار اگر چه جهان را خراب کرد به جنگ
&&
خراب گنج تو دارد چرا شود دلتنگ
\\
جهان شکست و تو یار شکستگان باشی
&&
کجاست مست تو را از چنین خرابی ننگ
\\
فلک ز مستی امر تو روز و شب در چرخ
&&
زمین ز شادی گنج تو خیره مانده و دنگ
\\
وظیفه تو رسید و نیافت راه ز در
&&
زهی کرم که ز روزن بکردیش آونگ
\\
شنیده‌ایم که شاهان به جنگ بستانند
&&
ندیده‌ایم که شاهان عطا دهند به جنگ
\\
ز سنگ چشمه روان کرده‌ای و می‌گویی
&&
بیا عطا بستان ای دل فسرده چو سنگ
\\
کنار و بوسه رومی رخانت می‌باید
&&
ز روی آینه دل به عشق بزدا زنگ
\\
تعلقیست عجب زنگ را بدین رومی
&&
تعلقیست نهانی میان موش و پلنگ
\\
دهان ببند که تا دل دهانه بگشاید
&&
فروخورد دو جهان را به یک زمان چو نهنگ
\\
چو ما رویم ره دل هزار فرسنگست
&&
چو خطوتین دل آمد کجا بود فرسنگ
\\
اگر نه مفخر تبریز شمس دین جویاست
&&
چرا شود غم عشقش موکل و سرهنگ
\\
\end{longtable}
\end{center}
