\begin{center}
\section*{بخش ۶۲ - قصهٔ اهل ضروان و حسد ایشان بر درویشان کی پدر ما از سلیمی اغلب دخل باغ را به مسکینان می‌داد چون انگور بودی عشر دادی و چون مویز و دوشاب شدی عشر دادی و چون حلوا و پالوده کردی عشر دادی و از قصیل عشر دادی و چون در خرمن می‌کوفتی از کفهٔ آمیخته عشر دادی و چون گندم از کاه جدا شدی عشر دادی و چون آرد کردی عشر دادی و چون خمیر کردی عشر دادی و چون نان کردی عشر دادی لاجرم حق تعالی در آن باغ و کشت برکتی نهاده بود کی همه اصحاب باغها محتاج او بدندی هم به میوه و هم به سیم و او محتاج هیچ کس نی ازیشان فرزندانشان خرج عشر می‌دیدند منکر و آن برکت را نمی‌دیدند هم‌چون آن زن بدبخت که کدو را ندید و خر را دید}
\label{sec:sh062}
\addcontentsline{toc}{section}{\nameref{sec:sh062}}
\begin{longtable}{l p{0.5cm} r}
بود مردی صالحی ربانیی
&&
عقل کامل داشت و پایان دانیی
\\
در ده ضروان به نزدیک یمن
&&
شهره اندر صدقه و خلق حسن
\\
کعبهٔ درویش بودی کوی او
&&
آمدندی مستمندان سوی او
\\
هم ز خوشه عشر دادی بی‌ریا
&&
هم ز گندم چون شدی از که جدا
\\
آرد گشتی عشر دادی هم از آن
&&
نان شدی عشر دگر دادی ز نان
\\
عشر هر دخلی فرو نگذاشتی
&&
چارباره دادی زانچ کاشتی
\\
بس وصیتها بگفتی هر زمان
&&
جمع فرزندان خود را آن جوان
\\
الله الله قسم مسکین بعد من
&&
وا مگیریدش ز حرص خویشتن
\\
تا بماند بر شما کشت و ثمار
&&
در پناه طاعت حق پایدار
\\
دخلها و میوه‌ها جمله ز غیب
&&
حق فرستادست بی‌تخمین و ریب
\\
در محل دخل اگر خرجی کنی
&&
درگه سودست سودی بر زنی
\\
ترک اغلب دخل را در کشت‌زار
&&
باز کارد که ویست اصل ثمار
\\
بیشتر کارد خورد زان اندکی
&&
که ندارد در بروییدن شکی
\\
زان بیفشاند به کشتن ترک دست
&&
که آن غله‌ش هم زان زمین حاصل شدست
\\
کفشگر هم آنچ افزاید ز نان
&&
می‌خرد چرم و ادیم و سختیان
\\
که اصول دخلم اینها بوده‌اند
&&
هم ازینها می‌گشاید رزق بند
\\
دخل از آنجا آمدستش لاجرم
&&
هم در آنجا می‌کند داد و کرم
\\
این زمین و سختیان پرده‌ست و بس
&&
اصل روزی از خدا دان هر نفس
\\
چون بکاری در زمین اصل کار
&&
تا بروید هر یکی را صد هزار
\\
گیرم اکنون تخم را گر کاشتی
&&
در زمینی که سبب پنداشتی
\\
چون دو سه سال آن نروید چون کنی
&&
جز که در لابه و دعا کف در زنی
\\
دست بر سر می‌زنی پیش اله
&&
دست و سر بر دادن رزقش گواه
\\
تا بدانی اصل اصل رزق اوست
&&
تا همو را جوید آنک رزق‌جوست
\\
رزق از وی جو مجو از زید و عمرو
&&
مستی از وی جو مجو از بنگ و خمر
\\
توانگری زو خو نه از گنج و مال
&&
نصرت از وی خواه نه از عم و خال
\\
عاقبت زینها بخواهی ماندن
&&
هین کرا خواهی در آن دم خواندن
\\
این دم او را خوان و باقی را بمان
&&
تا تو باشی وارث ملک جهان
\\
چون یفر المرء آید من اخیه
&&
یهرب المولود یوما من ابیه
\\
زان شود هر دوست آن ساعت عدو
&&
که بت تو بود و از ره مانع او
\\
روی از نقاش رو می‌تافتی
&&
چون ز نقشی انس دل می‌یافتی
\\
این دم ار یارانت با تو ضد شوند
&&
وز تو برگردند و در خصمی روند
\\
هین بگو نک روز من پیروز شد
&&
آنچ فردا خواست شد امروز شد
\\
ضد من گشتند اهل این سرا
&&
تا قیامت عین شد پیشین مرا
\\
پیش از آنک روزگار خود برم
&&
عمر با ایشان به پایان آورم
\\
کالهٔ معیوب بخریده بدم
&&
شکر کز عیبش بگه واقف شدم
\\
پیش از آن کز دست سرمایه شدی
&&
عاقبت معیوب بیرون آمدی
\\
مال رفته عمر رفته ای نسیب
&&
ماه و جان داده پی کالهٔ معیب
\\
رخت دادم زر قلبی بستدم
&&
شاد شادان سوی خانه می‌شدم
\\
شکر کین زر قلب پیدا شد کنون
&&
پیش از آنک عمر بگذشتی فزون
\\
قلب ماندی تا ابد در گردنم
&&
حیف بودی عمر ضایع کردنم
\\
چون بگه‌تر قلبی او رو نمود
&&
پای خود زو وا کشم من زود زود
\\
یار تو چون دشمنی پیدا کند
&&
گر حقد و رشک او بیرون زند
\\
تو از آن اعراض او افغان مکن
&&
خویشتن را ابله و نادان مکن
\\
بلک شکر حق کن و نان بخش کن
&&
که نگشتی در جوال او کهن
\\
از جوالش زود بیرون آمدی
&&
تا بجویی یار صدق سرمدی
\\
نازنین یاری که بعد از مرگ تو
&&
رشتهٔ یاری او گردد سه تو
\\
آن مگر سلطان بود شاه رفیع
&&
یا بود مقبول سلطان و شفیع
\\
رستی از قلاب و سالوس و دغل
&&
غر او دیدی عیان پیش از اجل
\\
این جفای خلق با تو در جهان
&&
گر بدانی گنج زر آمد نهان
\\
خلق را با تو چنین بدخو کنند
&&
تا ترا ناچار رو آن سو کنند
\\
این یقین دان که در آخر جمله‌شان
&&
خصم گردند و عدو و سرکشان
\\
تو بمانی با فغان اندر لحد
&&
لا تذرنی فرد خواهان از احد
\\
ای جفاات به ز عهد وافیان
&&
هم ز داد تست شهد وافیان
\\
بشنو از عقل خود ای انباردار
&&
گندم خود را به ارض الله سپار
\\
تا شود آمن ز دزد و از شپش
&&
دیو را با دیوچه زوتر بکش
\\
کو همی ترساندت هم دم ز فقر
&&
هم‌چو کبکش صید کن ای نره صقر
\\
باز سلطان عزیزی کامیار
&&
ننگ باشد که کند کبکش شکار
\\
بس وصیت کرد و تخم وعظ کاشت
&&
چون زمین‌شان شوره بد سودی نداشت
\\
گرچه ناصح را بود صد داعیه
&&
پند را اذنی بباید واعیه
\\
تو به صد تلطیف پندش می‌دهی
&&
او ز پندت می‌کند پهلو تهی
\\
یک کس نامستمع ز استیز و رد
&&
صد کس گوینده را عاجز کند
\\
ز انبیا ناصح‌تر و خوش لهجه‌تر
&&
کی بود کی گرفت دمشان در حجر
\\
زانچ کوه و سنگ درکار آمدند
&&
می‌نشد بدبخت را بگشاده بند
\\
آنچنان دلها که بدشان ما و من
&&
نعتشان شدت بل اشد قسوة
\\
\end{longtable}
\end{center}
