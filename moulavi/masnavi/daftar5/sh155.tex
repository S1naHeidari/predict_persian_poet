\begin{center}
\section*{بخش ۱۵۵ - تمثیل تن آدمی به مهمان‌خانه و اندیشه‌های مختلف به مهمانان مختلف عارف در رضا بدان اندیشه‌های غم و شادی چون شخص مهمان‌دوست غریب‌نواز خلیل‌وار کی در خلیل باکرام ضیف پیوسته باز بود بر کافر و ممن و امین و خاین و با همه مهمانان روی تازه داشتی}
\label{sec:sh155}
\addcontentsline{toc}{section}{\nameref{sec:sh155}}
\begin{longtable}{l p{0.5cm} r}
هست مهمان‌خانه این تن ای جوان
&&
هر صباحی ضیف نو آید دوان
\\
هین مگو کین مانند اندر گردنم
&&
که هم اکنون باز پرد در عدم
\\
هرچه آید از جهان غیب‌وش
&&
در دلت ضیفست او را دار خوش
\\
\end{longtable}
\end{center}
