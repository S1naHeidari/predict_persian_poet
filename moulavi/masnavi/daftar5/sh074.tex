\begin{center}
\section*{بخش ۷۴ - قصهٔ ایاز و حجره داشتن او جهت چارق و پوستین و گمان آمدن خواجه تاشانس را کی او را در آن حجره دفینه است به سبب  محکمی در و گرانی قفل}
\label{sec:sh074}
\addcontentsline{toc}{section}{\nameref{sec:sh074}}
\begin{longtable}{l p{0.5cm} r}
آن ایاز از زیرکی انگیخته
&&
پوستین و چارقش آویخته
\\
می‌رود هر روز در حجرهٔ خلا
&&
چارقت اینست منگر درعلا
\\
شاه را گفتند او را حجره‌ایست
&&
اندر آنجا زر و سیم و خمره‌ایست
\\
راه می‌ندهد کسی را اندرو
&&
بسته می‌دارد همیشه آن در او
\\
شاه فرمود ای عجب آن بنده را
&&
چیست خود پنهان و پوشیده ز ما
\\
پس اشارت کرد میری را که رو
&&
نیم‌شب بگشای و اندر حجره شو
\\
هر چه یابی مر ترا یغماش کن
&&
سر او را بر ندیمان فاش کن
\\
با چنین اکرام و لطف بی‌عدد
&&
از لئیمی سیم و زر پنهان کند
\\
می‌نماید او وفا و عشق و جوش
&&
وانگه او گندم‌نمای جوفروش
\\
هر که اندر عشق یابد زندگی
&&
کفر باشد پیش او جز بندگی
\\
نیم‌شب آن میر با سی معتمد
&&
در گشاد حجرهٔ او رای زد
\\
مشعله بر کرده چندین پهلوان
&&
جانب حجره روانه شادمان
\\
که امر سلطانست بر حجره زنیم
&&
هر یکی همیان زر در کش کنیم
\\
آن یکی می‌گفت هی چه جای زر
&&
از عقیق و لعل گوی و از گهر
\\
خاص خاص مخزن سلطان ویست
&&
بلک اکنون شاه را خود جان ویست
\\
چه محل دارد به پیش این عشیق
&&
لعل و یاقوت و زمرد یا عقیق
\\
شاه را بر وی نبودی بد گمان
&&
تسخری می‌کرد بهر امتحان
\\
پاک می‌دانستش از هر غش و غل
&&
باز از وهمش همی‌لرزید دل
\\
که مبادا کین بود خسته شود
&&
من نخواهم که برو خجلت رود
\\
این نکردست او و گر کرد او رواست
&&
هر چه خواهد گو بکن محبوب ماست
\\
هر چه محبوبم کند من کرده‌ام
&&
او منم من او چه گر در پرده‌ام
\\
باز گفتی دور از آن خو و خصال
&&
این چنین تخلیط ژاژست و خیال
\\
از ایاز این خود محالست و بعید
&&
کو یکی دریاست قعرش ناپدید
\\
هفت دریا اندرو یک قطره‌ای
&&
جملهٔ هستی ز موجش چکره‌ای
\\
جمله پاکیها از آن دریا برند
&&
قطره‌هااش یک به یک میناگرند
\\
شاه شاهانست و بلک شاه‌ساز
&&
وز برای چشم بد نامش ایاز
\\
چشمهای نیک هم بر وی به دست
&&
از ره غیرت که حسنش بی‌حدست
\\
یک دهان خواهم به پهنای فلک
&&
تا بگویم وصف آن رشک ملک
\\
ور دهان یابم چنین و صد چنین
&&
تنگ آید در فغان این حنین
\\
این قدر گر هم نگویم ای سند
&&
شیشهٔ دل از ضعیفی بشکند
\\
شیشهٔ دل را چو نازک دیده‌ام
&&
بهر تسکین بس قبا بدریده‌ام
\\
من سر هر ماه سه روز ای صنم
&&
بی‌گمان باید که دیوانه شوم
\\
هین که امروز اول سه روزه است
&&
روز پیروزست نه پیروزه است
\\
هر دلی که اندر غم شه می‌بود
&&
دم به دم او را سر مه می‌بود
\\
قصهٔ محمود و اوصاف ایاز
&&
چون شدم دیوانه رفت اکنون ز ساز
\\
\end{longtable}
\end{center}
