\begin{center}
\section*{بخش ۸۴ - فرمودن شاه ایاز را کی اختیار کن از عفو و مکافات کی از عدل و لطف هر چه کنی اینجا صوابست و در هر یکی مصلحتهاست کی در عدل هزار لطف هست درج و لکم فی القصاص حیوة آنکس کی کراهت می‌دارد قصاص را درین یک حیات قاتل نظر می‌کند و در صد هزار حیات کی معصوم و محقون خواهند شدن در حصن بیم سیاست نمی‌نگرد}
\label{sec:sh084}
\addcontentsline{toc}{section}{\nameref{sec:sh084}}
\begin{longtable}{l p{0.5cm} r}
کن میان مجرمان حکم ای ایاز
&&
ای ایاز پاک با صد احتراز
\\
گر دو صد بارت بجوشم در عمل
&&
در کف جوشت نیابم یک دغل
\\
ز امتحان شرمنده خلقی بی‌شمار
&&
امتحانها از تو جمله شرمسار
\\
بحر بی‌قعرست تنها علم نیست
&&
کوه و صد کوهست این خود حلم نیست
\\
گفت من دانم عطای تست این
&&
ورنه من آن چارقم و آن پوستین
\\
بهر آن پیغامبر این را شرح ساخت
&&
هر که خود بشناخت یزدان را شناخت
\\
چارقت نطفه‌ست و خونت پوستین
&&
باقی ای خواجه عطای اوست این
\\
بهر آن دادست تا جویی دگر
&&
تو مگو که نیستش جز این قدر
\\
زان نماید چند سیب آن باغبان
&&
تا بدانی نخل و دخل بوستان
\\
کف گندم زان دهد خریار را
&&
تا بداند گندم انبار را
\\
نکته‌ای زان شرح گوید اوستاد
&&
تا شناسی علم او را مستزاد
\\
ور بگویی خود همینش بود و بس
&&
دورت اندازد چنانک از ریش خس
\\
ای ایاز اکنون بیا و داده ده
&&
داد نادر در جهان بنیاد نه
\\
مجرمانت مستحق کشتن‌اند
&&
وز طمع بر عفو و حلمت می‌تنند
\\
تا که رحمت غالب آید یا غضب
&&
آب کوثر غالب آید یا لهب
\\
از پی مردم‌ربایی هر دو هست
&&
شاخ حلم و خشم از عهد الست
\\
بهر این لفظ الست مستبین
&&
نفی و اثباتست در لفظی قرین
\\
زانک استفهام اثباتیست این
&&
لیک در وی لفظ لیس شد قرین
\\
ترک کن تا ماند این تقریر خام
&&
کاسهٔ خاصان منه بر خوان عام
\\
قهر و لطفی چون صبا و چون وبا
&&
آن یکی آهن‌ربا وین که‌ربا
\\
می‌کشد حق راستان را تا رشد
&&
قسم باطل باطلان را می‌کشد
\\
معده حلوایی بود حلوا کشد
&&
معده صفرایی بود سرکا کشد
\\
فرش سوزان سردی از جالس برد
&&
فرش افسرده حرارت را خورد
\\
دوست بینی از تو رحمت می‌جهد
&&
خصم بینی از تو سطوت می‌جهد
\\
ای ایاز این کار را زوتر گزار
&&
زانک نوعی انتقامست انتظار
\\
\end{longtable}
\end{center}
