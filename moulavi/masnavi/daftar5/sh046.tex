\begin{center}
\section*{بخش ۴۶ - مثال عالم هست نیست‌نما و عالم نیست هست‌نما}
\label{sec:sh046}
\addcontentsline{toc}{section}{\nameref{sec:sh046}}
\begin{longtable}{l p{0.5cm} r}
نیست را بنمود هست و محتشم
&&
هست را بنمود بر شکل عدم
\\
بحر را پوشید و کف کرد آشکار
&&
باد را پوشید و بنمودت غبار
\\
چون منارهٔ خاک پیچان در هوا
&&
خاک از خود چون برآید بر علا
\\
خاک را بینی به بالا ای علیل
&&
باد را نی جز به تعریف دلیل
\\
کف همی‌بینی روانه هر طرف
&&
کف بی‌دریا ندارد منصرف
\\
کف به حس بینی و دریا از دلیل
&&
فکر پنهان آشکارا قال و قیل
\\
نفی را اثبات می‌پنداشتیم
&&
دیدهٔ معدوم‌بینی داشتیم
\\
دیده‌ای که اندر نعاسی شد پدید
&&
کی تواند جز خیال و نیست دید
\\
لاجرم سرگشته گشتیم از ضلال
&&
چون حقیقت شد نهان پیدا خیال
\\
این عدم را چون نشاند اندر نظر
&&
چون نهان کرد آن حقیقت از بصر
\\
آفرین ای اوستاد سحرباف
&&
که نمودی معرضان را درد صاف
\\
ساحران مهتاب پیمایند زود
&&
پیش بازرگان و زر گیرند سود
\\
سیم بربایند زین گون پیچ پیچ
&&
سیم از کف رفته و کرباس هیچ
\\
این جهان جادوست ما آن تاجریم
&&
که ازو مهتاب پیموده خریم
\\
گز کند کرباس پانصد گز شتاب
&&
ساحرانه او ز نور ماهتاب
\\
چون ستد او سیم عمرت ای رهی
&&
سیم شد کرباس نی کیسه تهی
\\
قل اعوذت خواند باید کای احد
&&
هین ز نفاثات افغان وز عقد
\\
می‌دمند اندر گره آن ساحرات
&&
الغیاث المستغاث از برد و مات
\\
لیک بر خوان از زبان فعل نیز
&&
که زبان قول سستست ای عزیز
\\
در زمانه مر ترا سه همره‌اند
&&
آن یکی وافی و این دو غدرمند
\\
آن یکی یاران و دیگر رخت و مال
&&
وآن سوم وافیست و آن حسن الفعال
\\
مال ناید با تو بیرون از قصور
&&
یار آید لیک آید تا به گور
\\
چون ترا روز اجل آید به پیش
&&
یار گوید از زبان حال خویش
\\
تا بدینجا بیش همره نیستم
&&
بر سر گورت زمانی بیستم
\\
فعل تو وافیست زو کن ملتحد
&&
که در آید با تو در قعر لحد
\\
\end{longtable}
\end{center}
