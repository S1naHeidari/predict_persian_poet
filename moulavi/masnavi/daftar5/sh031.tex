\begin{center}
\section*{بخش ۳۱ - در بیان آنک عقل و روح در آب و گل محبوس‌اند هم‌چون هاروت و ماروت در چاه بابل}
\label{sec:sh031}
\addcontentsline{toc}{section}{\nameref{sec:sh031}}
\begin{longtable}{l p{0.5cm} r}
هم‌چو هاروت و چو ماروت آن دو پاک
&&
بسته‌اند اینجا به چاه سهمناک
\\
عالم سفلی و شهوانی درند
&&
اندرین چه گشته‌اند از جرم‌بند
\\
سحر و ضد سحر را بی‌اختیار
&&
زین دو آموزند نیکان و شرار
\\
لیک اول پند بدهندش که هین
&&
سحر را از ما میاموز و مچین
\\
ما بیاموزیم این سحر ای فلان
&&
از برای ابتلا و امتحان
\\
که امتحان را شرط باشد اختیار
&&
اختیاری نبودت بی‌اقتدار
\\
میلها هم‌چون سگان خفته‌اند
&&
اندریشان خیر و شر بنهفته‌اند
\\
چونک قدرت نیست خفتند این رده
&&
هم‌چو هیزم‌پاره‌ها و تن‌زده
\\
تا که مرداری در آید در میان
&&
نفخ صور حرص کوبد بر سگان
\\
چون در آن کوچه خری مردار شد
&&
صد سگ خفته بدان بیدار شد
\\
حرصهای رفته اندر کتم غیب
&&
تاختن آورد سر بر زد ز جیب
\\
موبه موی هر سگی دندان شده
&&
وز برای حیله دم جنبان شده
\\
نیم زیرش حیله بالا آن غضب
&&
چون ضعیف آتش که یابد او حطب
\\
شعله شعله می‌رسد از لامکان
&&
می‌رود دود لهب تا آسمان
\\
صد چنین سگ اندرین تن خفته‌اند
&&
چون شکاری نیستشان بنهفته‌اند
\\
یا چو بازانند و دیده دوخته
&&
در حجاب از عشق صیدی سوخته
\\
تا کله بردارد و بیند شکار
&&
آنگهان سازد طواف کوهسار
\\
شهوت رنجور ساکن می‌بود
&&
خاطر او سوی صحت می‌رود
\\
چون ببیند نان و سیب و خربزه
&&
در مصاف آید مزه و خوف بزه
\\
گر بود صبار دیدن سود اوست
&&
آن تهیج طبع سستش را نکوست
\\
ور نباشد صبر پس نادیده به
&&
تیر دور اولی ز مرد بی‌زره
\\
\end{longtable}
\end{center}
