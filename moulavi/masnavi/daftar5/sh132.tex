\begin{center}
\section*{بخش ۱۳۲ - حکایت هم در بیان تقریر اختیار خلق و بیان آنک تقدیر و قضا سلب کنندهٔ اختیار نیست}
\label{sec:sh132}
\addcontentsline{toc}{section}{\nameref{sec:sh132}}
\begin{longtable}{l p{0.5cm} r}
گفت دزدی شحنه را کای پادشاه
&&
آنچ کردم بود آن حکم اله
\\
گفت شحنه آنچ من هم می‌کنم
&&
حکم حقست ای دو چشم روشنم
\\
از دکانی گر کسی تربی برد
&&
کین ز حکم ایزدست ای با خرد
\\
بر سرش کوبی دو سه مشت ای کره
&&
حکم حقست این که اینجا باز نه
\\
در یکی تره چو این عذر ای فضول
&&
می‌نیاید پیش بقالی قبول
\\
چون بدین عذر اعتمادی می‌کنی
&&
بر حوالی اژدهایی می‌تنی
\\
از چنین عذر ای سلیم نانبیل
&&
خون و مال و زن همه کردی سبیل
\\
هر کسی پس سبلت تو بر کند
&&
عذر آرد خویش را مضطر کند
\\
حکم حق گر عذر می‌شاید ترا
&&
پس بیاموز و بده فتوی مرا
\\
که مرا صد آرزو و شهوتست
&&
دست من بسته ز بیم و هیبتست
\\
پس کرم کن عذر را تعلیم ده
&&
برگشا از دست و پای من گره
\\
اختیاری کرده‌ای تو پیشه‌ای
&&
که اختیاری دارم و اندیشه‌ای
\\
ورنه چون بگزیده‌ای آن پیشه را
&&
از میان پیشه‌ها ای کدخدا
\\
چونک آید نوبت نفس و هوا
&&
بیست مرده اختیار آید ترا
\\
چون برد یک حبه از تو یار سود
&&
اختیار جنگ در جانت گشود
\\
چون بیاید نوبت شکر نعم
&&
اختیارت نیست وز سنگی تو کم
\\
دوزخت را عذر این باشد یقین
&&
که اندرین سوزش مرا معذور بین
\\
کس بدین حجت چو معذورت نداشت
&&
وز کف جلاد این دورت نداشت
\\
پس بدین داور جهان منظوم شد
&&
حال آن عالم همت معلوم شد
\\
\end{longtable}
\end{center}
