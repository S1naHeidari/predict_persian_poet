\begin{center}
\section*{بخش ۶۷ - فرستادن اسرافیل را علیه‌السلام به خاک کی حفنه‌ای بر گیر از  خاک بهر ترکیب جسم آدم علیه‌السلام}
\label{sec:sh067}
\addcontentsline{toc}{section}{\nameref{sec:sh067}}
\begin{longtable}{l p{0.5cm} r}
گفت اسرافیل را یزدان ما
&&
که برو زان خاک پر کن کف بیا
\\
آمد اسرافیل هم سوی زمین
&&
باز آغازید خاکستان حنین
\\
کای فرشتهٔ صور و ای بحر حیات
&&
که ز دمهای تو جان یابد موات
\\
در دمی از صور یک بانگ عظیم
&&
پر شود محشر خلایق از رمیم
\\
در دمی در صور گویی الصلا
&&
برجهید ای کشتگان کربلا
\\
ای هلاکت دیدگان از تیغ مرگ
&&
برزنید از خاک سر چون شاخ و برگ
\\
رحمت تو وآن دم گیرای تو
&&
پر شود این عالم از احیای تو
\\
تو فرشتهٔ رحمتی رحمت نما
&&
حامل عرشی و قبلهٔ دادها
\\
عرش معدن گاه داد و معدلت
&&
چار جو در زیر او پر مغفرت
\\
جوی شیر و جوی شهد جاودان
&&
جوی خمر و دجلهٔ آب روان
\\
پس ز عرش اندر بهشتستان رود
&&
در جهان هم چیزکی ظاهر شود
\\
گرچه آلوده‌ست اینجا آن چهار
&&
از چه از زهر فنا و ناگوار
\\
جرعه‌ای بر خاک تیره ریختند
&&
زان چهار و فتنه‌ای انگیختند
\\
تا بجویند اصل آن را این خسان
&&
خود برین قانع شدند این ناکسان
\\
شیر داد و پرورش اطفال را
&&
چشمه کرده سینهٔ هر زال را
\\
خمر دفع غصه و اندیشه را
&&
چشمه کرده از عنب در اجترا
\\
انگبین داروی تن رنجور را
&&
چشمه کرده باطن زنبور را
\\
آب دادی عام اصل و فرع را
&&
از برای طهر و بهر کرع را
\\
تا ازینها پی بری سوی اصول
&&
تو برین قانع شدی ای بوالفضول
\\
بشنو اکنون ماجرای خاک را
&&
که چه می‌گوید فسون محراک را
\\
پیش اسرافیل‌گشته او عبوس
&&
می‌کند صد گونه شکل و چاپلوس
\\
که بحق ذات پاک ذوالجلال
&&
که مدار این قهر را بر من حلال
\\
من ازین تقلیب بویی می‌برم
&&
بدگمانی می‌دود اندر سرم
\\
تو فرشتهٔ رحمتی رحمت نما
&&
زانک مرغی را نیازارد هما
\\
ای شفا و رحمت اصحاب درد
&&
تو همان کن کان دو نیکوکار کرد
\\
زود اسرافیل باز آمد به شاه
&&
گفت عذر و ماجرا نزد اله
\\
کز برون فرمان بدادی که بگیر
&&
عکس آن الهام دادی در ضمیر
\\
امر کردی در گرفتن سوی گوش
&&
نهی کردی از قساوت سوی هوش
\\
سبق رحمت گشت غالب بر غضب
&&
ای بدیع افعال و نیکوکار رب
\\
\end{longtable}
\end{center}
