\begin{center}
\section*{بخش ۸۹ - در بیان آنک دعای عارف واصل و درخواست او از حق هم‌چو درخواست حقست از خویشتن کی کنت له سمعا و بصرا و لسانا و یدا و قوله و ما رمیت اذ رمیت و لکن الله رمی و آیات و اخبار و آثار درین بسیارست و شرح سبب ساختن حق تا مجرم را گوش گرفته بتوبهٔ نصوح آورد}
\label{sec:sh089}
\addcontentsline{toc}{section}{\nameref{sec:sh089}}
\begin{longtable}{l p{0.5cm} r}
آن دعا از هفت گردون در گذشت
&&
کار آن مسکین به آخر خوب گشت
\\
که آن دعای شیخ نه چون هر دعاست
&&
فانی است و گفت او گفت خداست
\\
چون خدا از خود سؤال و کد کند
&&
پس دعای خویش را چون رد کند
\\
یک سبب انگیخت صنع ذوالجلال
&&
که رهانیدش ز نفرین و وبال
\\
اندر آن حمام پر می‌کرد طشت
&&
گوهری از دختر شه یاوه گشت
\\
گوهری از حلقه‌های گوش او
&&
یاوه گشت و هر زنی در جست و جو
\\
پس در حمام را بستند سخت
&&
تا بجویند اولش در پیچ رخت
\\
رختها جستند و آن پیدا نشد
&&
دزد گوهر نیز هم رسوا نشد
\\
پس به جد جستن گرفتند از گزاف
&&
در دهان و گوش و اندر هر شکاف
\\
در شکاف تحت و فوق و هر طرف
&&
جست و جو کردند دری خوش صدف
\\
بانگ آمد که همه عریان شوید
&&
هر که هستید ار عجوز و گر نوید
\\
یک به یک را حاجبه جستن گرفت
&&
تا پدید آید گهردانهٔ شگفت
\\
آن نصوح از ترس شد در خلوتی
&&
روی زرد و لب کبود از خشیتی
\\
پیش چشم خویش او می‌دید مرگ
&&
رفت و می‌لرزید او مانند برگ
\\
گفت یارب بارها برگشته‌ام
&&
توبه‌ها و عهدها بشکسته‌ام
\\
کرده‌ام آنها که از من می‌سزید
&&
تا چنین سیل سیاهی در رسید
\\
نوبت جستن اگر در من رسد
&&
وه که جان من چه سختیها کشد
\\
در جگر افتاده‌استم صد شرر
&&
در مناجاتم ببین بوی جگر
\\
این چنین اندوه کافر را مباد
&&
دامن رحمت گرفتم داد داد
\\
کاشکی مادر نزادی مر مرا
&&
یا مرا شیری بخوردی در چرا
\\
ای خدا آن کن که از تو می‌سزد
&&
که ز هر سوراخ مارم می‌گزد
\\
جان سنگین دارم و دل آهنین
&&
ورنه خون گشتی درین رنج و حنین
\\
وقت تنگ آمد مرا و یک نفس
&&
پادشاهی کن مرا فریاد رس
\\
گر مرا این بار ستاری کنی
&&
توبه کردم من ز هر ناکردنی
\\
توبه‌ام بپذیر این بار دگر
&&
تا ببندم بهر توبه صد کمر
\\
من اگر این بار تقصیری کنم
&&
پس دگر مشنو دعا و گفتنم
\\
این همی زارید و صد قطره روان
&&
که در افتادم به جلاد و عوان
\\
تا نمیرد هیچ افرنگی چنین
&&
هیچ ملحد را مبادا این حنین
\\
نوحه‌ها کرد او بر جان خویش
&&
روی عزرائیل دیده پیش پیش
\\
ای خدا و ای خدا چندان بگفت
&&
که آن در و دیوار با او گشت جفت
\\
در میان یارب و یارب بد او
&&
بانگ آمد از میان جست و جو
\\
\end{longtable}
\end{center}
