\begin{center}
\section*{بخش ۵۱ - قصهٔ آن شخص کی دعوی پیغامبری می‌کرد گفتندش چه خورده‌ای کی گیج شده‌ای و یاوه می‌گویی گفت اگر چیزی یافتمی کی خوردمی نه گیج شدمی و نه یاوه گفتمی کی هر سخن نیک کی با غیر اهلش گویند یاوه گفته باشند اگر چه در آن یاوه گفتن مامورند}
\label{sec:sh051}
\addcontentsline{toc}{section}{\nameref{sec:sh051}}
\begin{longtable}{l p{0.5cm} r}
آن یکی می‌گفت من پیغامبرم
&&
از همه پیغامبران فاضلترم
\\
گردنش بستند و بردندش به شاه
&&
کین همی گوید رسولم از اله
\\
خلق بر وی جمع چون مور و ملخ
&&
که چه مکرست و چه تزویر و چه فخ
\\
گر رسول آنست که آید از عدم
&&
ما همه پیغامبریم و محتشم
\\
ما از آنجا آمدیم اینجا غریب
&&
تو چرا مخصوص باشی ای ادیب
\\
نه شما چون طفل خفته آمدیت
&&
بی‌خبر از راه وز منزل بدیت
\\
از منازل خفته بگذشتید و مست
&&
بی‌خبر از راه و از بالا و پست
\\
ما به بیداری روان گشتیم و خوش
&&
از ورای پنج و شش تا پنج و شش
\\
دیده منزلها ز اصل و از اساس
&&
چون قلاووز آن خبیر و ره‌شناس
\\
شاه را گفتند اشکنجه‌ش بکن
&&
تا نگوید جنس او هیچ این سخن
\\
شاه دیدش بس نزار و بس ضعیف
&&
که به یک سیلی بمیرد آن نحیف
\\
کی توان او را فشردن یا زدن
&&
که چو شیشه گشته است او را بدن
\\
لیک با او گویم از راه خوشی
&&
که چرا داری تو لاف سر کشی
\\
که درشتی ناید اینجا هیچ کار
&&
هم به نرمی سر کند از غار مار
\\
مردمان را دور کرد از گرد وی
&&
شه لطیفی بود و نرمی ورد وی
\\
پس نشاندش باز پرسیدش ز جا
&&
که کجا داری معاش و ملتجی
\\
گفت ای شه هستم از دار السلام
&&
آمده از ره درین دار الملام
\\
نه مرا خانه‌ست و نه یک همنشین
&&
خانه کی کردست ماهی در زمین
\\
باز شه از روی لاغش گفت باز
&&
که چه خوردی و چه داری چاشت‌ساز
\\
اشتهی داری چه خوردی بامداد
&&
که چنین سرمستی و پر لاف و باد
\\
گفت اگر نانم بدی خشک و طری
&&
کی کنیمی دعوی پیغامبری
\\
دعوی پیغامبری با این گروه
&&
هم‌چنان باشد که دل جستن ز کوه
\\
کس ز کوه و سنگ عقل و دل نجست
&&
فهم و ضبط نکتهٔ مشکل نجست
\\
هر چه گویی باز گوید که همان
&&
می‌کند افسوس چون مستهزیان
\\
از کجا این قوم و پیغام از کجا
&&
از جمادی جان کرا باشد رجا
\\
گر تو پیغام زنی آری و زر
&&
پیش تو بنهند جمله سیم و سر
\\
که فلان جا شاهدی می‌خواندت
&&
عاشق آمد بر تو او می‌داندت
\\
ور تو پیغام خدا آری چو شهد
&&
که بیا سوی خدا ای نیک‌عهد
\\
از جهان مرگ سوی برگ رو
&&
چون بقا ممکن بود فانی مشو
\\
قصد خون تو کنند و قصد سر
&&
نه از برای حمیت دین و هنر
\\
\end{longtable}
\end{center}
