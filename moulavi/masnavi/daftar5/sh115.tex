\begin{center}
\section*{بخش ۱۱۵ - در معنی لولاک لما خلقت الافلاک}
\label{sec:sh115}
\addcontentsline{toc}{section}{\nameref{sec:sh115}}
\begin{longtable}{l p{0.5cm} r}
شد چنین شیخی گدای کو به کو
&&
عشق آمد لاابالی اتقوا
\\
عشق جوشد بحر را مانند دیگ
&&
عشق ساید کوه را مانند ریگ
\\
عشق‌بشکافد فلک را صد شکاف
&&
عشق لرزاند زمین را از گزاف
\\
با محمد بود عشق پاک جفت
&&
بهر عشق او را خدا لولاک گفت
\\
منتهی در عشق چون او بود فرد
&&
پس مر او را ز انبیا تخصیص کرد
\\
گر نبودی بهر عشق پاک را
&&
کی وجودی دادمی افلاک را
\\
من بدان افراشتم چرخ سنی
&&
تا علو عشق را فهمی کنی
\\
منفعتهای دیگر آید ز چرخ
&&
آن چو بیضه تابع آید این چو فرخ
\\
خاک را من خوار کردم یک سری
&&
تا ز خواری عاشقان بویی بری
\\
خاک را دادیم سبزی و نوی
&&
تا ز تبدیل فقیر آگه شوی
\\
با تو گویند این جبال راسیات
&&
وصف حال عاشقان اندر ثبات
\\
گرچه آن معنیست و این نقش ای پسر
&&
تا به فهم تو کند نزدیک‌تر
\\
غصه را با خار تشبیهی کنند
&&
آن نباشد لیک تنبیهی کنند
\\
آن دل قاسی که سنگش خواندند
&&
نامناسب بد مثالی راندند
\\
در تصور در نیاید عین آن
&&
عیب بر تصویر نه نفیش مدان
\\
\end{longtable}
\end{center}
