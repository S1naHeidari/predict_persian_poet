\begin{center}
\section*{بخش ۱۰۳ - مثل آوردن اشتر در بیان آنک در مخبر دولتی فر و اثر آن  چون نبینی جای متهم داشتن باشد کی او مقلدست در آن}
\label{sec:sh103}
\addcontentsline{toc}{section}{\nameref{sec:sh103}}
\begin{longtable}{l p{0.5cm} r}
آن یکی پرسید اشتر را که هی
&&
از کجا می‌آیی ای اقبال پی
\\
گفت از حمام گرم کوی تو
&&
گفت خود پیداست در زانوی تو
\\
مار موسی دید فرعون عنود
&&
مهلتی می‌خواست نرمی می‌نمود
\\
زیرکان گفتند بایستی که این
&&
تندتر گشتی چو هست او رب دین
\\
معجزه‌گر اژدها گر مار بد
&&
نخوت و خشم خدایی‌اش چه شد
\\
رب اعلی گر ویست اندر جلوس
&&
بهر یک کرمی چیست این چاپلوس
\\
نفس تو تا مست نقلست و نبید
&&
دانک روحت خوشهٔ غیبی ندید
\\
که علاماتست زان دیدار نور
&&
التجافی منک عن دار الغرور
\\
مرغ چون بر آب شوری می‌تند
&&
آب شیرین را ندیدست او مدد
\\
بلک تقلیدست آن ایمان او
&&
روی ایمان را ندیده جان او
\\
پس خطر باشد مقلد را عظیم
&&
از ره و ره‌زن ز شیطان رجیم
\\
چون ببیند نور حق آمن شود
&&
ز اضطرابات شک او ساکن شود
\\
تا کف دریا نیاید سوی خاک
&&
که اصل او آمد بود در اصطکاک
\\
خاکی است آن کف غریبست اندر آب
&&
در غریبی چاره نبود ز اضطراب
\\
چونک چشمش باز شد و آن نقش خواند
&&
دیو را بر وی دگر دستی نماند
\\
گرچه با روباه خر اسرار گفت
&&
سرسری گفت و مقلدوار گفت
\\
آب را بستود و او تایق نبود
&&
رخ درید و جامه او عاشق نبود
\\
از منافق عذر رد آمد نه خوب
&&
زانک در لب بود آن نه در قلوب
\\
بوی سیبش هست جزو سیب نیست
&&
بو درو جز از پی آسیب نیست
\\
حملهٔ زن در میان کارزار
&&
نشکند صف بلک گردد کارزار
\\
گرچه می‌بینی چو شیر اندر صفش
&&
تیغ بگرفته همی‌لرزد کفش
\\
وای آنک عقل او ماده بود
&&
نفس زشتش نر و آماده بود
\\
لاجرم مغلوب باشد عقل او
&&
جز سوی خسران نباشد نقل او
\\
ای خنک آن کس که عقلش نر بود
&&
نفس زشتش ماده و مضطر بود
\\
عقل جزوی‌اش نر و غالب بود
&&
نفس انثی را خرد سالب بود
\\
حملهٔ ماده به صورت هم جریست
&&
آفت او هم‌چو آن خر از خریست
\\
وصف حیوانی بود بر زن فزون
&&
زانک سوی رنگ و بو دارد رکون
\\
رنگ و بوی سبزه‌زار آن خر شنید
&&
جمله حجتها ز طبع او رمید
\\
تشنه محتاج مطر شد وابر نه
&&
نفس را جوع البقر بد صبر نه
\\
اسپر آهن بود صبر ای پدر
&&
حق نبشته بر سپر جاء الظفر
\\
صد دلیل آرد مقلد در بیان
&&
از قیاسی گوید آن را نه از عیان
\\
مشک‌آلودست الا مشک نیست
&&
بوی مشکستش ولی جز پشک نیست
\\
تا که پشکی مشک گردد ای مرید
&&
سالها باید در آن روضه چرید
\\
که نباید خورد و جو هم‌چون خران
&&
آهوانه در ختن چر ارغوان
\\
جز قرنفل یا سمن یا گل مچر
&&
رو به صحرای ختن با آن نفر
\\
معده را خو کن بدان ریحان و گل
&&
تا بیابی حکمت و قوت رسل
\\
خوی معده زین که و جو باز کن
&&
خوردن ریحان و گل آغاز کن
\\
معدهٔ تن سوی کهدان می‌کشد
&&
معدهٔ دل سوی ریحان می‌کشد
\\
هر که کاه و جو خورد قربان شود
&&
هر که نور حق خورد قرآن شود
\\
نیم تو مشکست و نیمی پشک هین
&&
هین میفزا پشک افزا مشک چین
\\
آن مقلد صد دلیل و صد بیان
&&
در زبان آرد ندارد هیچ جان
\\
چونک گوینده ندارد جان و فر
&&
گفت او را کی بود برگ و ثمر
\\
می‌کند گستاخ مردم را به راه
&&
او بجان لرزان‌ترست از برگ کاه
\\
پس حدیثش گرچه بس با فر بود
&&
در حدیثش لرزه هم مضمر بود
\\
\end{longtable}
\end{center}
