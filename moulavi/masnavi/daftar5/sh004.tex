\begin{center}
\section*{بخش ۴ - در حجره گشادن مصطفی علیه‌السلام بر مهمان و خود را پنهان  کردن تا او خیال گشاینده را نبیند و خجل شود و گستاخ  بیرون رود}
\label{sec:sh004}
\addcontentsline{toc}{section}{\nameref{sec:sh004}}
\begin{longtable}{l p{0.5cm} r}
مصطفی صبح آمد و در را گشاد
&&
صبح آن گمراه را او راه داد
\\
در گشاد و گشت پنهان مصطفی
&&
تا نگردد شرمسار آن مبتلا
\\
تا برون آید رود گستاخ او
&&
تا نبیند درگشا را پشت و رو
\\
یا نهان شد در پس چیزی و یا
&&
از ویش پوشید دامان خدا
\\
صبغة الله گاه پوشیده کند
&&
پردهٔ بی‌چون بر آن ناظر تند
\\
تا نبیند خصم را پهلوی خویش
&&
قدرت یزدان از آن بیشست بیش
\\
مصطفی می‌دید احوال شبش
&&
لیک مانع بود فرمان ربش
\\
تا که پیش از خبط بگشاید رهی
&&
تا نیفتد زان فضیحت در چهی
\\
لیک حکمت بود و امر آسمان
&&
تا ببیند خویشتن را او چنان
\\
بس عداوتها که آن یاری بود
&&
بس خرابیها که معماری بود
\\
جامه خواب پر حدث را یک فضول
&&
قاصدا آورد در پیش رسول
\\
که چنین کردست مهمانت ببین
&&
خنده‌ای زد رحمةللعالمین
\\
که بیار آن مطهره اینجا به پیش
&&
تا بشویم جمله را با دست خویش
\\
هر کسی می‌جست کز بهر خدا
&&
جان ما و جسم ما قربان ترا
\\
ما بشوییم این حدث را تو بهل
&&
کار دستست این نمط نه کار دل
\\
ای لعمرک مر ترا حق عمر خواند
&&
پس خلیفه کرد و بر کرسی نشاند
\\
ما برای خدمت تو می‌زییم
&&
چون تو خدمت می‌کنی پس ما چه‌ایم
\\
گفت آن دانم و لیک این ساعتیست
&&
که درین شستن بخویشم حکمتیست
\\
منتظر بودند کین قول نبیست
&&
تا پدید آید که این اسرار چیست
\\
او به جد می‌شست آن احداث را
&&
خاص ز امر حق نه تقلید و ریا
\\
که دلش می‌گفت کین را تو بشو
&&
که درین جا هست حکمت تو بتو
\\
\end{longtable}
\end{center}
