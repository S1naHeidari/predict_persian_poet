\begin{center}
\section*{بخش ۷۱ - در بیان وخامت چرب و شیرین دنیا و مانع شدن او از طعام الله چنانک فرمود الجوع طعام الله یحیی به ابدان الصدیقین ای فی الجوع طعام الله و قوله ابیت عند ربی یطعمنی و یسقینی و قوله  یرزقون فرحین}
\label{sec:sh071}
\addcontentsline{toc}{section}{\nameref{sec:sh071}}
\begin{longtable}{l p{0.5cm} r}
وا رهی زین روزی ریزهٔ کثیف
&&
در فتی در لوت و در قوت شریف
\\
گر هزاران رطل لوتش می‌خوری
&&
می‌روی پاک و سبک هم‌چون پری
\\
که نه حبس باد و قولنجت کند
&&
چارمیخ معده آهنجت کند
\\
گر خوری کم گرسنه مانی چو زاغ
&&
ور خوری پر گیرد آروغت دماغ
\\
کم خوری خوی بد و خشکی و دق
&&
پر خوری شد تخمه را تن مستحق
\\
از طعام الله و قوت خوش‌گوار
&&
بر چنان دریا چو کشتی شو سوار
\\
باش در روزه شکیبا و مصر
&&
دم به دم قوت خدا را منتظر
\\
که آن خدای خوب‌کار بردبار
&&
هدیه‌ها را می‌دهد در انتظار
\\
انتظار نان ندارد مرد سیر
&&
که سبک آید وظیفه یا که دیر
\\
بی‌نوا هر دم همی گوید که کو
&&
در مجاعت منتظر در جست و جو
\\
چون نباشی منتظر ناید به تو
&&
آن نوالهٔ دولت هفتاد تو
\\
ای پدر الانتظار الانتظار
&&
از برای خوان بالا مردوار
\\
هر گرسنه عاقبت قوتی بیافت
&&
آفتاب دولتی بر وی بتافت
\\
ضیف با همت چو ز آشی کم خورد
&&
صاحب خوان آش بهتر آورد
\\
جز که صاحب خوان درویشی لئیم
&&
ظن بد کم بر به رزاق کریم
\\
سر برآور هم‌چو کوهی ای سند
&&
تا نخستین نور خور بر تو زند
\\
که آن سر کوه بلند مستقر
&&
هست خورشید سحر را منتظر
\\
\end{longtable}
\end{center}
