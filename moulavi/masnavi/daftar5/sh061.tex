\begin{center}
\section*{بخش ۶۱ - صاحب‌دلی دید سگ حامله در شکم آن سگ‌بچگان بانگ می‌کردند در تعجب ماند کی حکمت بانگ سگ پاسبانیست بانگ در اندرون شکم مادر پاسبانی نیست و نیز بانگ جهت یاری خواستن و شیر خواستن باشد و غیره و آنجا هیچ این فایده‌ها نیست چون به خویش آمد با حضرت مناجات کرد و ما یعلم تاویله الا الله جواب آمد کی آن صورت حال قومیست از حجاب بیرون نیامده و چشم دل باز ناشده دعوی بصیرت کنند و مقالات گویند از آن نی ایشان را قوتی و یاریی رسد و نه مستمعان را هدایتی  و رشدی}
\label{sec:sh061}
\addcontentsline{toc}{section}{\nameref{sec:sh061}}
\begin{longtable}{l p{0.5cm} r}
آن یکی می‌دید خواب اندر چله
&&
در رهی ماده سگی بد حامله
\\
ناگهان آواز سگ‌بچگان شنید
&&
سگ‌بچه اندر شکم بد ناپدید
\\
بس عجب آمد ورا آن بانگها
&&
سگ‌بچه اندر شکم چون زد ندا
\\
سگ‌بچه اندر شکم ناله کنان
&&
هیچ‌کس دیدست این اندر جهان
\\
چون بجست از واقعه آمد به خویش
&&
حیرت او دم به دم می‌گشت بیش
\\
در چله کس نی که گردد عقده حل
&&
جز که درگاه خدا عز و جل
\\
گفت یا رب زین شکال و گفت و گو
&&
در چله وا مانده‌ام از ذکر تو
\\
پر من بگشای تا پران شوم
&&
در حدیقهٔ ذکر و سیبستان شوم
\\
آمدش آواز هاتف در زمان
&&
که آن مثالی دان ز لاف جاهلان
\\
کز حجاب و پرده بیرون نامده
&&
چشم بسته بیهده گویان شده
\\
بانگ سگ اندر شکم باشد زیان
&&
نه شکارانگیز و نه شب پاسبان
\\
گرگ نادیده که منع او بود
&&
دزد نادیده که دفع او شود
\\
از حریصی وز هوای سروری
&&
در نظر کند و بلافیدن جری
\\
از هوای مشتری و گرم‌دار
&&
بی بصیرت پا نهاده در فشار
\\
ماه نادیده نشانها می‌دهد
&&
روستایی را بدان کژ می‌نهد
\\
از برای مشتری در وصف ماه
&&
صد نشان نادیده گوید بهر جاه
\\
مشتری کو سود دارد خود یکیست
&&
لیک ایشان را درو ریب و شکیست
\\
از هوای مشتری بی‌شکوه
&&
مشتری را باد دادند این گروه
\\
مشتری ماست الله اشتری
&&
از غم هر مشتری هین برتر آ
\\
مشتریی جو که جویان توست
&&
عالم آغاز و پایان توست
\\
هین مکش هر مشتری را تو به دست
&&
عشق‌بازی با دو معشوقه بدست
\\
زو نیابی سود و مایه گر خرد
&&
نبودش خود قیمت عقل و خرد
\\
نیست او را خود بهای نیم نعل
&&
تو برو عرضه کنی یاقوت و لعل
\\
حرص کورت کرد و محرومت کند
&&
دیو هم‌چون خویش مرجومت کند
\\
هم‌چنانک اصحاب فیل و قوم لوط
&&
کردشان مرجوم چون خود آن سخوط
\\
مشتری را صابران در یافتند
&&
چون سوی هر مشتری نشتافتند
\\
آنک گردانید رو زان مشتری
&&
بخت و اقبال و بقا شد زو بری
\\
ماند حسرت بر حریصان تا ابد
&&
هم‌چو حال اهل ضروان در حسد
\\
\end{longtable}
\end{center}
