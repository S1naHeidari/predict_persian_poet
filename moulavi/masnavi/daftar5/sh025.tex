\begin{center}
\section*{بخش ۲۵ - تفسیر و ان یکاد الذین کفروا لیزلقونک بابصارهم الایه}
\label{sec:sh025}
\addcontentsline{toc}{section}{\nameref{sec:sh025}}
\begin{longtable}{l p{0.5cm} r}
یا رسول‌الله در آن نادی کسان
&&
می‌زنند از چشم بد بر کرکسان
\\
از نظرشان کلهٔ شیر عرین
&&
وا شکافد تا کند آن شیر انین
\\
بر شتر چشم افکند هم‌چون حمام
&&
وانگهان بفرستد اندر پی غلام
\\
که برو از پیه این اشتر بخر
&&
بیند اشتر را سقط او راه بر
\\
سر بریده از مرض آن اشتری
&&
کو بتگ با اسب می‌کردی مری
\\
کز حسد وز چشم بد بی‌هیچ شک
&&
سیر و گردش را بگرداند فلک
\\
آب پنهانست و دولاب آشکار
&&
لیک در گردش بود آب اصل کار
\\
چشم نیکو شد دوای چشم بد
&&
چشم بد را لا کند زیر لگد
\\
سبق رحمت‌راست و او از رحمتست
&&
چشم بد محصول قهر و لعنتست
\\
رحمتش بر نقمتش غالب شود
&&
چیره زین شد هر نبی بر ضد خود
\\
کو نتیجهٔ رحمتست و ضد او
&&
از نتیجهٔ قهر بود آن زشت‌رو
\\
حرص بط یکتاست این پنجاه تاست
&&
حرص شهوت مار و منصب اژدهاست
\\
حرص بط از شهوت حلقست و فرج
&&
در ریاست بیست چندانست درج
\\
از الوهیت زند در جاه لاف
&&
طامع شرکت کجا باشد معاف
\\
زلت آدم ز اشکم بود و باه
&&
وآن ابلیس از تکبر بود و جاه
\\
لاجرم او زود استغفار کرد
&&
وآن لعین از توبه استکبار کرد
\\
حرص حلق و فرج هم خود بدرگیست
&&
لیک منصب نیست آن اشکستگیست
\\
بیخ و شاخ این ریاست را اگر
&&
باز گویم دفتری باید دگر
\\
اسپ سرکش را عرب شیطانش خواند
&&
نی ستوری را که در مرعی بماند
\\
شیطنت گردن کشی بد در لغت
&&
مستحق لعنت آمد این صفت
\\
صد خورنده گنجد اندر گرد خوان
&&
دو ریاست‌جو نگنجد در جهان
\\
آن نخواهد کین بود بر پشت خاک
&&
تا ملک بکشد پدر را ز اشتراک
\\
آن شنیدستی که الملک عقیم
&&
قطع خویشی کرد ملکت‌جو ز بیم
\\
که عقیمست و ورا فرزند نیست
&&
هم‌چو آتش با کسش پیوند نیست
\\
هر چه یابد او بسوزد بر درد
&&
چون نیابد هیچ خود را می‌خورد
\\
هیچ شو وا ره تو از دندان او
&&
رحم کم جو از دل سندان او
\\
چونک گشتی هیچ از سندان مترس
&&
هر صباح از فقر مطلق گیر درس
\\
هست الوهیت ردای ذوالجلال
&&
هر که در پوشد برو گردد وبال
\\
تاج از آن اوست آن ما کمر
&&
وای او کز حد خود دارد گذر
\\
فتنهٔ تست این پر طاووسیت
&&
که اشتراکت باید و قدوسیت
\\
\end{longtable}
\end{center}
