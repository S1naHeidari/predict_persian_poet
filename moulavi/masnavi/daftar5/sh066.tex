\begin{center}
\section*{بخش ۶۶ - قصهٔ قوم یونس علیه‌السلام بیان و برهان آنست کی تضرع و زاری دافع بلای آسمانیست و حق تعالی فاعل مختارست پس تضرع و تعظیم پیش او مفید باشد و فلاسفه گویند فاعل به طبع است و بعلت نه مختار پر تضرع طبع را نگرداند}
\label{sec:sh066}
\addcontentsline{toc}{section}{\nameref{sec:sh066}}
\begin{longtable}{l p{0.5cm} r}
قوم یونس را چو پیدا شد بلا
&&
ابر پر آتش جدا شد از سما
\\
برق می‌انداخت می‌سوزید سنگ
&&
ابر می‌غرید رخ می‌ریخت رنگ
\\
جملگان بر بامها بودند شب
&&
که پدید آمد ز بالا آن کرب
\\
جملگان از بامها زیر آمدند
&&
سر برهنه جانب صحرا شدند
\\
مادران بچگان برون انداختند
&&
تا همه ناله و نفیر افراختند
\\
از نماز شام تا وقت سحر
&&
خاک می‌کردند بر سر آن نفر
\\
جملگی آوازها بگرفته شد
&&
رحم آمد بر سر آن قوم لد
\\
بعد نومیدی و آه ناشکفت
&&
اندک‌اندک ابر وا گشتن گرفت
\\
قصهٔ یونس درازست و عریض
&&
وقت خاکست و حدیث مستفیض
\\
چون تضرع را بر حق قدرهاست
&&
وآن بها که آنجاست زاری را کجاست
\\
هین امید اکنون میان را چست بند
&&
خیز ای گرینده و دایم بخند
\\
که برابر می‌نهد شاه مجید
&&
اشک را در فضل با خون شهید
\\
\end{longtable}
\end{center}
