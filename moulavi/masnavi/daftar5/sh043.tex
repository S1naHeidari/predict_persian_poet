\begin{center}
\section*{بخش ۴۳ - بیان آنک کشتن خلیل علیه‌السلام خروس را اشارت به قمع و قهر کدام  صفت بود از صفات مذمومات مهلکان در باطن مرید}
\label{sec:sh043}
\addcontentsline{toc}{section}{\nameref{sec:sh043}}
\begin{longtable}{l p{0.5cm} r}
شهوتی است او و بس شهوت‌پرست
&&
زان شراب زهرناک ژاژ مست
\\
گرنه بهر نسل بود ای وصی
&&
آدم از ننگش بکردی خود خصی
\\
گفت ابلیس لعین دادار را
&&
دام زفتی خواهم این اشکار را
\\
زر و سیم و گلهٔ اسپش نمود
&&
که بدین تانی خلایق را ربود
\\
گفت شاباش و ترش آویخت لنج
&&
شد ترنجیده ترش هم‌چون ترنج
\\
پس زر و گوهر ز معدنهای خوش
&&
کرد آن پس‌مانده را حق پیش‌کش
\\
گیر این دام دگر را ای لعین
&&
گفت زین افزون ده ای نعم‌المعین
\\
چرب و شیرین و شرابات ثمین
&&
دادش و بس جامهٔ ابریشمین
\\
گفت یا رب بیش ازین خواهم مدد
&&
تا ببندمشان به حبل من مسد
\\
تا که مستانت که نر و پر دلند
&&
مردوار آن بندها را بسکلند
\\
تا بدین دام و رسنهای هوا
&&
مرد تو گردد ز نامردان جدا
\\
دام دیگر خواهم ای سلطان تخت
&&
دام مردانداز و حیلت‌ساز سخت
\\
خمر و چنگ آورد پیش او نهاد
&&
نیم‌خنده زد بدان شد نیم‌شاد
\\
سوی اضلال ازل پیغام کرد
&&
که بر آر از قعر بحر فتنه گرد
\\
نی یکی از بندگانت موسی است
&&
پرده‌ها در بحر او از گرد بست
\\
آب از هر سو عنان را واکشید
&&
از تگ دریا غباری برجهید
\\
چونک خوبی زنان فا او نمود
&&
که ز عقل و صبر مردان می‌فزود
\\
پس زد انگشتک به رقص اندر فتاد
&&
که بده زوتر رسیدم در مراد
\\
چون بدید آن چشمهای پرخمار
&&
که کند عقل و خرد را بی‌قرار
\\
وآن صفای عارض آن دلبران
&&
که بسوزد چون سپند این دل بر آن
\\
رو و خال و ابرو و لب چون عقیق
&&
گوییا حق تافت از پردهٔ رقیق
\\
دید او آن غنج و برجست سبک
&&
چون تجلی حق از پردهٔ تنک
\\
\end{longtable}
\end{center}
