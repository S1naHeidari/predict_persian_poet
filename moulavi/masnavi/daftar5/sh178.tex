\begin{center}
\section*{بخش ۱۷۸ - مجرم دانستن ایاز خود را درین شفاعت‌گری و عذر این جرم خواستن و در آن عذرگویی خود را مجرم دانستن و این شکستگی از شناخت و عظمت شاه خیزد کی انا اعلمکم بالله و اخشیکم لله و قال الله تعالی انما یخشی الله من عباده العلما}
\label{sec:sh178}
\addcontentsline{toc}{section}{\nameref{sec:sh178}}
\begin{longtable}{l p{0.5cm} r}
من کی آرم رحم خلم آلود را
&&
ره نمایم حلم علم‌اندود را
\\
صد هزاران صفع را ارزانیم
&&
گر زبون صفعها گردانیم
\\
من چه گویم پیشت اعلامت کنم
&&
یا که وا یادت دهم شرط کرم
\\
آنچ معلوم تو نبود چیست آن
&&
وآنچ یادت نیست کو اندر جهان
\\
ای تو پاک از جهل و علمت پاک از آن
&&
که فراموشی کند بر وی نهان
\\
هیچ کس را تو کسی انگاشتی
&&
هم‌چو خورشیدش به نور افراشتی
\\
چون کسم کردی اگر لابه کنم
&&
مستمع شو لابه‌ام را از کرم
\\
زانک از نقشم چو بیرون برده‌ای
&&
آن شفاعت هم تو خود را کرده‌ای
\\
چون ز رخت من تهی گشت این وطن
&&
تر و خشک خانه نبود آن من
\\
هم دعا از من روان کردی چو آب
&&
هم نباتش بخش و دارش مستجاب
\\
هم تو بودی اول آرندهٔ دعا
&&
هم تو باش آخر اجابت را رجا
\\
تا زنم من لاف کان شاه جهان
&&
بهر بنده عفو کرد از مجرمان
\\
درد بودم سر به سر من خودپسند
&&
کرد شاهم داروی هر دردمند
\\
دوزخی بودم پر از شور و شری
&&
کرد دست فضل اویم کوثری
\\
هر که را سوزید دوزخ در قود
&&
من برویانم دگر بار از جسد
\\
کار کوثر چیست که هر سوخته
&&
گردد از وی نابت و اندوخته
\\
قطره قطره او منادی کرم
&&
کانچ دوزخ سوخت من باز آورم
\\
هست دوزخ هم‌چو سرمای خزان
&&
هست کوثر چون بهار ای گلستان
\\
هست دوزخ هم‌چو مرگ و خاک گور
&&
هست کوثر بر مثال نفخ صور
\\
ای ز دوزخ سوخته اجسامتان
&&
سوی کوثر می‌کشد اکرامتان
\\
چون خلقت الخلق کی یربح علی
&&
لطف تو فرمود ای قیوم حی
\\
لالان اربح علیهم جود تست
&&
که شود زو جمله ناقصها درست
\\
عفو کن زین بندگان تن‌پرست
&&
عفو از دریای عفو اولیترست
\\
عفو خلقان هم‌چو جو و هم‌چو سیل
&&
هم بدان دریای خود تازند خیل
\\
عفوها هر شب ازین دل‌پاره‌ها
&&
چون کبوتر سوی تو آید شها
\\
بازشان وقت سحر پران کنی
&&
تا به شب محبوس این ابدان کنی
\\
پر زنان بار دگر در وقت شام
&&
می‌پرند از عشق آن ایوان و بام
\\
تا که از تن تار وصلت بسکلند
&&
پیش تو آیند کز تو مقبلند
\\
پر زنان آمن ز رجع سرنگون
&&
در هوا که انا الیه راجعون
\\
بانگ می‌آید تعالوا زان کرم
&&
بعد از آن رجعت نماند از حرص و غم
\\
بس غریبیها کشیدیت از جهان
&&
قدر من دانسته باشید ای مهان
\\
زیر سایهٔ این درختم مست ناز
&&
هین بیندازید پاها را دراز
\\
پایهای پر عنا از راه دین
&&
بر کنار و دست حوران خالدین
\\
حوریان گشته مغمز مهربان
&&
کز سفر باز آمدند این صوفیان
\\
صوفیان صافیان چون نور خور
&&
مدتی افتاده بر خاک و قذر
\\
بی‌اثر پاک از قذر باز آمدند
&&
هم‌چو نور خور سوی قرص بلند
\\
این گروه مجرمان هم ای مجید
&&
جمله سرهاشان به دیواری رسید
\\
بر خطا و جرم خود واقف شدند
&&
گرچه مات کعبتین شه بدند
\\
رو به تو کردند اکنون اه‌کنان
&&
ای که لطفت مجرمان را ره‌کنان
\\
راه ده آلودگان را العجل
&&
در فرات عفو و عین مغتسل
\\
تا که غسل آرند زان جرم دراز
&&
در صف پاکان روند اندر نماز
\\
اندر آن صفها ز اندازه برون
&&
غرقگان نور نحن الصافون
\\
چون سخن در وصف این حالت رسید
&&
هم قلم بشکست و هم کاغذ درید
\\
بحر را پیمود هیچ اسکره‌ای
&&
شیر را برداشت هرگز بره‌ای
\\
گر حجابستت برون رو ز احتجاب
&&
تا ببینی پادشاهی عجاب
\\
گرچه بشکستند حامت قوم مست
&&
آنک مست از تو بود عذریش هست
\\
مستی ایشان به اقبال و به مال
&&
نه ز بادهٔ تست ای شیرین فعال
\\
ای شهنشه مست تخصیص توند
&&
عفو کن از مست خود ای عفومند
\\
لذت تخصیص تو وقت خطاب
&&
آن کند که ناید از صد خم شراب
\\
چونک مستم کرده‌ای حدم مزن
&&
شرع مستان را نبیند حد زدن
\\
چون شوم هشیار آنگاهم بزن
&&
که نخواهم گشت خود هشیار من
\\
هرکه از جام تو خورد ای ذوالمنن
&&
تا ابد رست از هش و از حد زدن
\\
خالدین فی فناء سکرهم
&&
من تفانی فی هواکم لم یقم
\\
فضل تو گوید دل ما را که رو
&&
ای شده در دوغ عشق ما گرو
\\
چون مگس در دوغ ما افتاده‌ای
&&
تو نه‌ای مست ای مگس تو باده‌ای
\\
کرگسان مست از تو گردند ای مگس
&&
چونک بر بحر عسل رانی فرس
\\
کوهها چون ذره‌ها سرمست تو
&&
نقطه و پرگار و خط در دست تو
\\
فتنه که لرزند ازو لرزان تست
&&
هر گران‌قیمت گهر ارزان تست
\\
گر خدا دادی مرا پانصد دهان
&&
گفتمی شرح تو ای جان و جهان
\\
یک دهان دارم من آن هم منکسر
&&
در خجالت از تو ای دانای سر
\\
منکسرتر خود نباشم از عدم
&&
کز دهانش آمدستند این امم
\\
صد هزار آثار غیبی منتظر
&&
کز عدم بیرون جهد با لطف و بر
\\
از تقاضای تو می‌گردد سرم
&&
ای ببرده من به پیش آن کرم
\\
رغبت ما از تقاضای توست
&&
جذبهٔ حقست هر جا ره‌روست
\\
خاک بی‌بادی به بالا بر جهد
&&
کشتی بی‌بحر پا در ره نهد
\\
پیش آب زندگانی کس نمرد
&&
پیش آبت آب حیوانست درد
\\
آب حیوان قبلهٔ جان دوستان
&&
ز آب باشد سبز و خندان بوستان
\\
مرگ آشامان ز عشقش زنده‌اند
&&
دل ز جان و آب جان بر کنده‌اند
\\
آب عشق تو چو ما را دست داد
&&
آب حیوان شد به پیش ما کساد
\\
ز آب حیوان هست هر جان را نوی
&&
لیک آب آب حیوانی توی
\\
هر دمی مرگی و حشری دادیم
&&
تا بدیدم دست برد آن کرم
\\
هم‌چو خفتن گشت این مردن مرا
&&
ز اعتماد بعث کردن ای خدا
\\
هفت دریا هر دم ار گردد سراب
&&
گوش گیری آوریش ای آب آب
\\
عقل لرزان از اجل وان عشق شوخ
&&
سنگ کی ترسد ز باران چون کلوخ
\\
از صحاف مثنوی این پنجمست
&&
بر بروج چرخ جان چون انجمست
\\
ره نیابد از ستاره هر حواس
&&
جز که کشتیبان استاره‌شناس
\\
جز نظاره نیست قسم دیگران
&&
از سعودش غافلند و از قران
\\
آشنایی گیر شبها تا به روز
&&
با چنین استارهای دیوسوز
\\
هر یکی در دفع دیو بدگمان
&&
هست نفط‌انداز قلعهٔ آسمان
\\
اختر ار با دیو هم‌چون عقربست
&&
مشتری را او ولی الاقربست
\\
قوس اگر از تیر دوزد دیو را
&&
دلو پر آبست زرع و میو را
\\
حوت اگرچه کشتی غی بشکند
&&
دوست را چون ثور کشتی می‌کند
\\
شمس اگر شب را بدرد چون اسد
&&
لعل را زو خلعت اطلس رسد
\\
هر وجودی کز عدم بنمود سر
&&
بر یکی زهرست و بر دیگر شکر
\\
دوست شو وز خوی ناخوش شو بری
&&
تا ز خمرهٔ زهر هم شکر خوری
\\
زان نشد فاروق را زهری گزند
&&
که بد آن تریاق فاروقیش قند
\\
\end{longtable}
\end{center}
