\begin{center}
\section*{بخش ۶۵ - فرستادن میکائیل را علیه‌السلام به قبض حفنه‌ای خاک از زمین جهت ترکیب ترتیب جسم مبارک ابوالبشر خلیفة  الحق مسجود الملک و معلمهم آدم علیه‌السلام}
\label{sec:sh065}
\addcontentsline{toc}{section}{\nameref{sec:sh065}}
\begin{longtable}{l p{0.5cm} r}
گفت میکائیل را تو رو به زیر
&&
مشت خاکی در ربا از وی چو شیر
\\
چونک میکائیل شد تا خاکدان
&&
دست کرد او تا که برباید از آن
\\
خاک لرزید و درآمد در گریز
&&
گشت او لابه‌کنان و اشک‌ریز
\\
سینه سوزان لابه کرد و اجتهاد
&&
با سرشک پر ز خون سوگند داد
\\
که به یزدان لطیف بی‌ندید
&&
که بکردت حامل عرش مجید
\\
کیل ارزاق جهان را مشرفی
&&
تشنگان فضل را تو مغرفی
\\
زانک میکائیل از کیل اشتقاق
&&
دارد و کیال شد در ارتزاق
\\
که امانم ده مرا آزاد کن
&&
بین که خون‌آلود می‌گویم سخن
\\
معدن رحم اله آمد ملک
&&
گفت چون ریزم بر آن ریش این نمک
\\
هم‌چنانک معدن قهرست دیو
&&
که برآورد از نبی آدم غریو
\\
سبق رحمت بر غضب هست ای فتا
&&
لطف غالب بود در وصف خدا
\\
بندگان دارند لابد خوی او
&&
مشکهاشان پر ز آب جوی او
\\
آن رسول حق قلاوز سلوک
&&
گفت الناس علی دین الملوک
\\
رفت میکائیل سوی رب دین
&&
خالی از مقصود دست و آستین
\\
گفت ای دانای سر و شاه فرد
&&
خاک از زاری و گریه بسته کرد
\\
آب دیده پیش تو با قدر بود
&&
من نتانستم که آرم ناشنود
\\
آه و زاری پیش تو بس قدر داشت
&&
من نتانستم حقوق آن گذاشت
\\
پیش تو بس قدر دارد چشم تر
&&
من چگونه گشتمی استیزه‌گر
\\
دعوت زاریست روزی پنج بار
&&
بنده را که در نماز آ و بزار
\\
نعرهٔ مؤذن که حیا عل فلاح
&&
وآن فلاح این زاری است و اقتراح
\\
آن که خواهی کز غمش خسته کنی
&&
راه زاری بر دلش بسته کنی
\\
تا فرو آید بلا بی‌دافعی
&&
چون نباشد از تضرع شافعی
\\
وانک خواهی کز بلااش وا خری
&&
جان او را در تضرع آوری
\\
گفته‌ای اندر نبی که آن امتان
&&
که بریشان آمد آن قهر گران
\\
چون تضرع می‌نکردند آن نفس
&&
تا بلا زیشان بگشتی باز پس
\\
لیک دلهاشان چون قاسی گشته بود
&&
آن گنههاشان عبادت می‌نمود
\\
تا نداند خویش را مجرم عنید
&&
آب از چشمش کجا داند دوید
\\
\end{longtable}
\end{center}
