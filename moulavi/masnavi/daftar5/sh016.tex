\begin{center}
\section*{بخش ۱۶ - تمثیل لوح محفوظ و ادراک عقل هر کسی از آن لوح آنک امر و قسمت و مقدور هر روزهٔ ویست هم چون ادراک جبرئیل علیه‌السلام  هر روزی از لوح اعظم  عقل مثال جبرئیلست و نظر او به تفکر به سوی غیبی که معهود اوست در  تفکر و اندیشهٔ کیفیت معاش و بیرون شو کارهای هر روزینه مانند نظر  جبرئیلست در لوح و فهم کردن او از لوح}
\label{sec:sh016}
\addcontentsline{toc}{section}{\nameref{sec:sh016}}
\begin{longtable}{l p{0.5cm} r}
چون ملک از لوح محفوظ آن خرد
&&
هر صباحی درس هر روزه برد
\\
بر عدم تحریرها بین بی‌بنان
&&
و از سوادش حیرت سوداییان
\\
هر کسی شد بر خیالی ریش گاو
&&
گشته در سودای گنجی کنج‌کاو
\\
از خیالی گشته شخصی پرشکوه
&&
روی آورده به معدنهای کوه
\\
وز خیالی آن دگر با جهد مر
&&
رو نهاده سوی دریا بهر در
\\
وآن دگر بهر ترهب در کنشت
&&
وآن یکی اندر حریصی سوی کشت
\\
از خیال آن ره‌زن رسته شده
&&
وز خیال این مرهم خسته شده
\\
در پری‌خوانی یکی دل کرده گم
&&
بر نجوم آن دیگری بنهاده سم
\\
این روشها مختلف بیند برون
&&
زان خیالات ملون ز اندرون
\\
این در آن حیران شده کان بر چیست
&&
هر چشنده آن دگر را نافیست
\\
آن خیالات ار نبد نامؤتلف
&&
چون ز بیرون شد روشها مختلف
\\
قبلهٔ جان را چو پنهان کرده‌اند
&&
هر کسی رو جانبی آورده‌اند
\\
\end{longtable}
\end{center}
