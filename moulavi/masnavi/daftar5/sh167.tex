\begin{center}
\section*{بخش ۱۶۷ - حجت منکران آخرت و بیان ضعف آن حجت زیرا حجت ایشان به دین باز می‌گردد کی غیر این نمی‌بینیم}
\label{sec:sh167}
\addcontentsline{toc}{section}{\nameref{sec:sh167}}
\begin{longtable}{l p{0.5cm} r}
حجتش اینست گوید هر دمی
&&
گر بدی چیزی دگر هم دیدمی
\\
گر نبیند کودکی احوال عقل
&&
عاقلی هرگز کند از عقل نقل
\\
ور نبیند عاقلی احوال عشق
&&
کم نگردد ماه نیکوفال عشق
\\
حسن یوسف دیدهٔ اخوان ندید
&&
از دل یعقوب کی شد ناپدید
\\
مر عصا را چشم موسی چوب دید
&&
چشم غیبی افعی و آشوب دید
\\
چشم سر با چشم سر در جنگ بود
&&
غالب آمد چشم سر حجت نمود
\\
چشم موسی دست خود را دست دید
&&
پیش چشم غیب نوری بد پدید
\\
این سخن پایان ندارد در کمال
&&
پیش هر محروم باشد چون خیال
\\
چون حقیقت پیش او فرج و گلوست
&&
کم بیان کن پیش او اسرار دوست
\\
پیش ما فرج و گلو باشد خیال
&&
لاجرم هر دم نماید جان جمال
\\
هر که را فرج و گلو آیین و خوست
&&
آن لکم دین ولی دین بهر اوست
\\
با چنان انکار کوته کن سخن
&&
احمدا کم گوی با گبر کهن
\\
\end{longtable}
\end{center}
