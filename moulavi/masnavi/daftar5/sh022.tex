\begin{center}
\section*{بخش ۲۲ - تفاوت عقول در اصل فطرت خلاف معتزله کی ایشان گویند در اصل عقول جز وی برابرند این افزونی و  تفاوت از تعلم است و ریاضت و تجربه}
\label{sec:sh022}
\addcontentsline{toc}{section}{\nameref{sec:sh022}}
\begin{longtable}{l p{0.5cm} r}
این تفاوت عقلها را نیک دان
&&
در مراتب از زمین تا آسمان
\\
هست عقلی هم‌چو قرص آفتاب
&&
هست عقلی کمتر از زهره و شهاب
\\
هست عقلی چون چراغی سرخوشی
&&
هست عقلی چون ستارهٔ آتشی
\\
زانک ابر از پیش آن چون وا جهد
&&
نور یزدان‌بین خردها بر دهد
\\
عقل جزوی عقل را بدنام کرد
&&
کام دنیا مرد را بی‌کام کرد
\\
آن ز صیدی حسن صیادی بدید
&&
وین ز صیادی غم صیدی کشید
\\
آن ز خدمت ناز مخدومی بیافت
&&
وآن ز مخدومی ز راه عز بتافت
\\
آن ز فرعونی اسیر آب شد
&&
وز اسیری سبط صد سهراب شد
\\
لعب معکوسست و فرزین‌بند سخت
&&
حیله کم کن کار اقبالست و بخت
\\
بر حیال و حیله کم تن تار را
&&
که غنی ره کم دهد مکار را
\\
مکر کن در راه نیکو خدمتی
&&
تا نبوت یابی اندر امتی
\\
مکر کن تا وا رهی از مکر خود
&&
مکر کن تا فرد گردی از جسد
\\
مکر کن تا کمترین بنده شوی
&&
در کمی رفتی خداونده شوی
\\
روبهی و خدمت ای گرگ کهن
&&
هیچ بر قصد خداوندی مکن
\\
لیک چون پروانه در آتش بتاز
&&
کیسه‌ای زان بر مدوز و پاک باز
\\
زور را بگذار و زاری را بگیر
&&
رحم سوی زاری آید ای فقیر
\\
زاری مضطر تشنه معنویست
&&
زاری سرد دروغ آن غویست
\\
گریهٔ اخوان یوسف حیلتست
&&
که درونشان پر ز رشک و علتست
\\
\end{longtable}
\end{center}
