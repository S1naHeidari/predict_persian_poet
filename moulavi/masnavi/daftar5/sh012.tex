\begin{center}
\section*{بخش ۱۲ - عرضه کردن مصطفی علیه‌السلام شهادت را بر مهمان خویش}
\label{sec:sh012}
\addcontentsline{toc}{section}{\nameref{sec:sh012}}
\begin{longtable}{l p{0.5cm} r}
این سخن پایان ندارد مصطفی
&&
عرضه کرد ایمان و پذرفت آن فتی
\\
آن شهادت را که فرخ بوده است
&&
بندهای بسته را بگشوده است
\\
گشت مؤمن گفت او را مصطفی
&&
که امشبان هم باش تو مهمان ما
\\
گفت والله تا ابد ضیف توم
&&
هر کجا باشم بهر جا که روم
\\
زنده کرده و معتق و دربان تو
&&
این جهان و آن جهان بر خوان تو
\\
هر که بگزیند جزین بگزیده خوان
&&
عاقبت درد گلویش ز استخوان
\\
هر که سوی خوان غیر تو رود
&&
دیو با او دان که هم‌کاسه بود
\\
هر که از همسایگی تو رود
&&
دیو بی‌شکی که همسایه‌ش شود
\\
ور رود بی‌تو سفر او دوردست
&&
دیو بد همراه و هم‌سفرهٔ ویست
\\
ور نشیند بر سر اسپ شریف
&&
حاسد ماهست دیو او را ردیف
\\
ور بچه گیرد ازو شهناز او
&&
دیو در نسلش بود انباز او
\\
در نبی شارکهم گفتست حق
&&
هم در اموال و در اولاد ای شفق
\\
گفت پیغامبر ز غیب این را جلی
&&
در مقالات نوادر با علی
\\
یا رسول‌الله رسالت را تمام
&&
تو نمودی هم‌چو شمس بی‌غمام
\\
این که تو کردی دو صد مادر نکرد
&&
عیسی از افسونش با عازر نکرد
\\
از تو جانم از اجل نک جان ببرد
&&
عازر ار شد زنده زان دم باز مرد
\\
گشت مهمان رسول آن شب عرب
&&
شیر یک بز نیمه خورد و بست لب
\\
کرد الحاحش بخور شیر و رقاق
&&
گفت گشتم سیر والله بی‌نفاق
\\
این تکلف نیست نی ناموس و فن
&&
سیرتر گشتم از آنک دوش من
\\
در عجب ماندند جمله اهل بیت
&&
پر شد این قندیل زین یک قطره زیت
\\
آنچ قوت مرغ بابیلی بود
&&
سیری معدهٔ چنین پیلی شود
\\
فجفجه افتاد اندر مرد و زن
&&
قدر پشه می‌خورد آن پیل‌تن
\\
حرص و وهم کافری سرزیر شد
&&
اژدها از قوت موری سیر شد
\\
آن گدا چشمی کفر از وی برفت
&&
لوت ایمانیش لمتر کرد و زفت
\\
آنک از جوع البقر او می‌طپید
&&
هم‌چو مریم میوهٔ جنت بدید
\\
میوهٔ جنت سوی چشمش شتافت
&&
معدهٔ چون دوزخش آرام یافت
\\
ذات ایمان نعمت و لوتیست هول
&&
ای قناعت کرده از ایمان به قول
\\
\end{longtable}
\end{center}
