\begin{center}
\section*{بخش ۱۳ - بیان آنک نور که غذای جانست غذای جسم اولیا می‌شود تا او هم یار می‌شود روح را کی اسلم شیطانی علی یدی}
\label{sec:sh013}
\addcontentsline{toc}{section}{\nameref{sec:sh013}}
\begin{longtable}{l p{0.5cm} r}
گرچه آن مطعوم جانست و نظر
&&
جسم را هم زان نصیبست ای پسر
\\
گر نگشتی دیو جسم آن را اکول
&&
اسلم الشیطان نفرمودی رسول
\\
دیو زان لوتی که مرده حی شود
&&
تا نیاشامد مسلمان کی شود
\\
دیو بر دنیاست عاشق کور و کر
&&
عشق را عشقی دگر برد مگر
\\
از نهان‌خانهٔ یقین چون می‌چشد
&&
اندک‌اندک رخت عشق آنجا کشد
\\
یا حریص االبطن عرج هکذا
&&
انما المنهاج تبدیل الغذا
\\
یا مریض القلب عرج للعلاج
&&
جملة التدبیر تبدیل المزاج
\\
ایها المحبوس فی رهن الطعام
&&
سوف تنجو ان تحملت الفطام
\\
ان فی‌الجوع طعام وافر
&&
افتقدها وارتج یا نافر
\\
اغتذ بالنور کن مثل البصر
&&
وافق الاملاک یا خیر البشر
\\
چون ملک تسبیح حق را کن غذا
&&
تا رهی هم‌چون ملایک از اذا
\\
جبرئیل ار سوی جیفه کم تند
&&
او به قوت کی ز کرکس کم زند
\\
حبذا خوانی نهاده در جهان
&&
لیک از چشم خسیسان بس نهان
\\
گر جهان باغی از نعمت شود
&&
قسم موش و مار هم خاکی بود
\\
\end{longtable}
\end{center}
