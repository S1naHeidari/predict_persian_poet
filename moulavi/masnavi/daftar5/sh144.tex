\begin{center}
\section*{بخش ۱۴۴ - حکایت آن زن کی گفت شوهر را کی گوشت را گربه خورد شوهر گربه را به ترازو بر کشید گربه نیم من برآمد گفت ای زن گوشت نیم من بود و افزون اگر این گوشتست گربه کو و اگر این گربه است گوشت کو}
\label{sec:sh144}
\addcontentsline{toc}{section}{\nameref{sec:sh144}}
\begin{longtable}{l p{0.5cm} r}
بود مردی کدخدا او را زنی
&&
سخت طناز و پلید و ره‌زنی
\\
هرچه آوردی تلف کردیش زن
&&
مرد مضطر بود اندر تن زدن
\\
بهر مهمان گوشت آورد آن معیل
&&
سوی خانه با دو صد جهد طویل
\\
زن بخوردش با کباب و با شراب
&&
مرد آمد گفت دفع ناصواب
\\
مرد گفتش گوشت کو مهمان رسید
&&
پیش مهمان لوت می‌باید کشید
\\
گفت زن این گربه خورد آن گوشت را
&&
گوشت دیگر خر اگر باشد هلا
\\
گفت ای ایبک ترازو را بیار
&&
گربه را من بر کشم اندر عیار
\\
بر کشیدش بود گربه نیم من
&&
پس بگفت آن مرد کای محتال زن
\\
گوشت نیم من بود و افزون یک ستیر
&&
هست گربه نیم‌من هم ای ستیر
\\
این اگر گربه‌ست پس آن گوشت کو
&&
ور بود این گوشت گربه کو بجو
\\
بایزید ار این بود آن روح چیست
&&
ور وی آن روحست این تصویر کیست
\\
حیرت اندر حیرتست ای یار من
&&
این نه کار تست و نه هم کار من
\\
هر دو او باشد ولیک از ریع زرع
&&
دانه باشد اصل و آن که پره فرع
\\
حکمت این اضداد را با هم ببست
&&
ای قصاب این گردران با گردنست
\\
روح بی‌قالب نداند کار کرد
&&
قالبت بی‌جان فسرده بود و سرد
\\
قالبت پیدا و آن جانت نهان
&&
راست شد زین هر دو اسباب جهان
\\
خاک را بر سر زنی سر نشکند
&&
آب را بر سر زنی در نشکند
\\
گر تو می‌خواهی که سر را بشکنی
&&
آب را و خاک را بر هم زنی
\\
چون شکستی سر رود آبش به اصل
&&
خاک سوی خاک آید روز فصل
\\
حکمتی که بود حق را ز ازدواج
&&
گشت حاصل از نیاز و از لجاج
\\
باشد آنگه ازدواجات دگر
&&
لا سمع اذن و لا عین بصر
\\
گر شنیدی اذن کی ماندی اذن
&&
یا کجا کردی دگر ضبط سخن
\\
گر بدیدی برف و یخ خورشید را
&&
از یخی برداشتی اومید را
\\
آب گشتی بی‌عروق و بی‌گره
&&
ز آب داود هوا کردی زره
\\
پس شدی درمان جان هر درخت
&&
هر درختی از قدومش نیک‌بخت
\\
آن یخی بفسرده در خود مانده
&&
لا مساسی با درختان خوانده
\\
لیس یالف لیس یؤلف جسمه
&&
لیس الا شح نفس قسمه
\\
نیست ضایع زو شود تازه جگر
&&
لیک نبود پیک و سلطان خضر
\\
ای ایاز استارهٔ تو بس بلند
&&
نیست هر برجی عبورش را پسند
\\
هر وفا را کی پسندد همتت
&&
هر صفا را کی گزیند صفوتت
\\
\end{longtable}
\end{center}
