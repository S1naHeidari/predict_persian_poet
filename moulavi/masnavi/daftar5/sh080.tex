\begin{center}
\section*{بخش ۸۰ - معشوقی از عاشق پرسید کی خود را دوست‌تر داری یا مرا  گفت من از خود مرده‌ام و به تو زنده‌ام از خود و از صفات خود نیست  شده‌ام و به تو هست شده‌ام علم خود را فراموش کرده‌ام و از علم تو  عالم شده‌ام قدرت خود را از یاد داده‌ام و از قدرت تو قادر شده‌ام اگر خود را دوست دارم ترا دوست داشته باشم و اگر ترا دوست دارم  خود را دوست داشته باشم  هر که را آینهٔ یقین باشد  گرچه خود بین خدای بین باشد  اخرج به صفاتی الی خلقی من رآک رآنی و من قصدک قصدنی و علی هذا}
\label{sec:sh080}
\addcontentsline{toc}{section}{\nameref{sec:sh080}}
\begin{longtable}{l p{0.5cm} r}
گفت معشوقی به عاشق ز امتحان
&&
در صبوحی کای فلان ابن الفلان
\\
مر مرا تو دوست‌تر داری عجب
&&
یا که خود را راست گو یا ذا الکرب
\\
گفت من در تو چنان فانی شدم
&&
که پرم از تتو ز ساران تا قدم
\\
بر من از هستی من جز نام نیست
&&
در وجودم جز تو ای خوش‌کام نیست
\\
زان سبب فانی شدم من این چنین
&&
هم‌چو سرکه در تو بحر انگبین
\\
هم‌چو سنگی کو شود کل لعل ناب
&&
پر شود او از صفات آفتاب
\\
وصف آن سنگی نماند اندرو
&&
پر شود از وصف خور او پشت و رو
\\
بعد از آن گر دوست دارد خویش را
&&
دوستی خور بود آن ای فتا
\\
ور که خود را دوست دارد ای بجان
&&
دوستی خویش باشد بی‌گمان
\\
خواه خود را دوست دارد لعل ناب
&&
خواه تا او دوست دارد آفتاب
\\
اندرین دو دوستی خود فرق نیست
&&
هر دو جانب جز ضیای شرق نیست
\\
تا نشد او لعل خود را دشمنست
&&
زانک یک من نیست آنجا دو منست
\\
زانک ظلمانیست سنگ و روزکور
&&
هست ظلمانی حقیقت ضد نور
\\
خویشتن را دوست دارد کافرست
&&
زانک او مناع شمس اکبرست
\\
پس نشاید که بگوید سنگ انا
&&
او همه تاریکیست و در فنا
\\
گفت فرعونی انا الحق گشت پست
&&
گفت منصوری اناالحق و برست
\\
آن انا را لعنة الله در عقب
&&
وین انا را رحمةالله ای محب
\\
زانک او سنگ سیه بد این عقیق
&&
آن عدوی نور بود و این عشیق
\\
این انا هو بود در سر ای فضول
&&
ز اتحاد نور نه از رای حلول
\\
جهد کن تا سنگیت کمتر شود
&&
تا به لعلی سنگ تو انور شود
\\
صبر کن اندر جهاد و در عنا
&&
دم به دم می‌بین بقا اندر فنا
\\
وصف سنگی هر زمان کم می‌شود
&&
وصف لعلی در تو محکم می‌شود
\\
وصف هستی می‌رود از پیکرت
&&
وصف مستی می‌فزاید در سرت
\\
سمع شو یکبارگی تو گوش‌وار
&&
تا ز حلقهٔ لعل یابی گوشوار
\\
هم‌چو چه کن خاک می‌کن گر کسی
&&
زین تن خاکی که در آبی رسی
\\
گر رسد جذبهٔ خدا آب معین
&&
چاه ناکنده بجوشد از زمین
\\
کار می‌کن تو بگوش آن مباش
&&
اندک اندک خاک چه را می‌تراش
\\
هر که رنجی دید گنجی شد پدید
&&
هر که جدی کرد در جدی رسید
\\
گفت پیغمبر رکوعست و سجود
&&
بر در حق کوفتن حلقهٔ وجود
\\
حلقهٔ آن در هر آنکو می‌زند
&&
بهر او دولت سری بیرون کند
\\
\end{longtable}
\end{center}
