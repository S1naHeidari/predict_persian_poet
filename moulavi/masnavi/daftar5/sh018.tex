\begin{center}
\section*{بخش ۱۸ - تفسیر یا حسرة علی العباد}
\label{sec:sh018}
\addcontentsline{toc}{section}{\nameref{sec:sh018}}
\begin{longtable}{l p{0.5cm} r}
او همی گوید که از اشکال تو
&&
غره گشتم دیر دیدم حال تو
\\
شمع مرده باده رفته دلربا
&&
غوطه خورد از ننگ کژبینی ما
\\
ظلت الارباح خسرا مغرما
&&
نشتکی شکوی الی الله العمی
\\
حبذا ارواح اخوان ثقات
&&
مسلمات مؤمنات قانتات
\\
هر کسی رویی به سویی برده‌اند
&&
وان عزیزان رو به بی‌سو کرده‌اند
\\
هر کبوتر می‌پرد در مذهبی
&&
وین کبوتر جانب بی‌جانبی
\\
ما نه مرغان هوا نه خانگی
&&
دانهٔ ما دانهٔ بی‌دانگی
\\
زان فراخ آمد چنین روزی ما
&&
که دریدن شد قبادوزی ما
\\
\end{longtable}
\end{center}
