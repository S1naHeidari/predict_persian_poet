\begin{center}
\section*{بخش ۱۲۴ - حکایت مریدی کی شیخ از حرص و ضمیر او واقف شد او را نصیحت کرد به زبان و در ضمن نصیحت قوت توکل بخشیدش به امر حق}
\label{sec:sh124}
\addcontentsline{toc}{section}{\nameref{sec:sh124}}
\begin{longtable}{l p{0.5cm} r}
شیخ می‌شد با مریدی بی‌درنگ
&&
سوی شهری نان بدانجا بود تنگ
\\
ترس جوع و قحط در فکر مرید
&&
هر دمی می‌گشت از غفلت پدید
\\
شیخ آگه بود و واقف از ضمیر
&&
گفت او را چند باشی در زحیر
\\
از برای غصهٔ نان سوختی
&&
دیدهٔ صبر و توکل دوختی
\\
تو نه‌ای زان نازنینان عزیز
&&
که ترا دارند بی‌جوز و مویز
\\
جوع رزق جان خاصان خداست
&&
کی زبون هم‌چو تو گیج گداست
\\
باش فارغ تو از آنها نیستی
&&
که درین مطبخ تو بی‌نان بیستی
\\
کاسه بر کاسه‌ست و نان بر نان مدام
&&
از برای این شکم‌خواران عام
\\
چون بمیرد می‌رود نان پیش پیش
&&
کای ز بیم بی‌نوایی کشته خویش
\\
تو برفتی ماند نان برخیز گیر
&&
ای بکشته خویش را اندر زحیر
\\
هین توکل کن ملرزان پا و دست
&&
رزق تو بر تو ز تو عاشق‌ترست
\\
عاشقست و می‌زند او مول‌مول
&&
که ز بی‌صبریت داند ای فضول
\\
گر ترا صبری بدی رزق آمدی
&&
خویشتن چون عاشقان بر تو زدی
\\
این تب لرزه ز خوف جوع چیست
&&
در توکل سیر می‌تانند زیست
\\
\end{longtable}
\end{center}
