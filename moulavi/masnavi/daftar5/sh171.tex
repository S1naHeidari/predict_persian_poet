\begin{center}
\section*{بخش ۱۷۱ - عزم کردن شاه چون واقف شد بر آن خیانت کی بپوشاند و عفو کند و او را به او دهد و دانست کی آن فتنه جزای او بود و قصد او بود و ظلم او بر صاحب موصل کی و من اساء فعلیها و ان ربک لبالمرصاد و ترسیدن کی اگر انتقام کشد آن انتقام هم بر سر او آید چنانک این ظلم و طمع بر سرش آمد}
\label{sec:sh171}
\addcontentsline{toc}{section}{\nameref{sec:sh171}}
\begin{longtable}{l p{0.5cm} r}
شاه با خود آمد استغفار کرد
&&
یاد جرم و زلت و اصرار کرد
\\
گفت با خود آنچ کردم با کسان
&&
شد جزای آن به جان من رسان
\\
قصد جفت دیگران کردم ز جاه
&&
بر من آمد آن و افتادم به چاه
\\
من در خانهٔ کسی دیگر زدم
&&
او در خانهٔ مرا زد لاجرم
\\
هر که با اهل کسان شد فسق‌جو
&&
اهل خود را دان که قوادست او
\\
زانک مثل آن جزای آن شود
&&
چون جزای سیئه مثلش بود
\\
چون سبب کردی کشیدی سوی خویش
&&
مثل آن را پس تو دیوثی و بیش
\\
غصب کردم از شه موصل کنیز
&&
غصب کردند از من او را زود نیز
\\
او کامین من بد و لالای من
&&
خاینش کرد آن خیانتهای من
\\
نیست وقت کین‌گزاری و انتقام
&&
من به دست خویش کردم کار خام
\\
گر کشم کینه بر آن میر و حرم
&&
آن تعدی هم بیاید بر سرم
\\
هم‌چنانک این یک بیامد در جزا
&&
آزمودم باز نزمایم ورا
\\
درد صاحب موصلم گردن شکست
&&
من نیارم این دگر را نیز خست
\\
داد حق‌مان از مکافات آگهی
&&
گفت ان عدتم به عدنا به
\\
چون فزونی کردن اینجا سود نیست
&&
غیر صبر و مرحمت محمود نیست
\\
ربنا انا ظلمنا سهو رفت
&&
رحمتی کن ای رحیمیهات رفت
\\
عفو کردم تو هم از من عفو کن
&&
از گناه نو ز زلات کهن
\\
گفت اکنون ای کنیزک وا مگو
&&
این سخن را که شنیدم من ز تو
\\
با امیرت جفت خواهم کرد من
&&
الله الله زین حکایت دم مزن
\\
تا نگردد او ز رویم شرمسار
&&
کو یکی بد کرد و نیکی صد هزار
\\
بارها من امتحانش کرده‌ام
&&
خوب‌تر از تو بدو بسپرده‌ام
\\
در امانت یافتم او را تمام
&&
این قضایی بود هم از کرده‌هام
\\
پس به خود خواند آن امیر خویش را
&&
کشت در خود خشم قهراندیش را
\\
کرد با او یک بهانهٔ دل‌پذیر
&&
که شدستم زین کنیزک من نفیر
\\
زان سبب کز غیرت و رشک کنیز
&&
مادر فرزند دارد صد ازیز
\\
مادر فرزند را بس حقهاست
&&
او نه درخورد چنین جور و جفاست
\\
رشک و غیرت می‌برد خون می‌خورد
&&
زین کنیزک سخت تلخی می‌برد
\\
چون کسی را داد خواهم این کنیز
&&
پس ترا اولیترست این ای عزیز
\\
که تو جانبازی نمودی بهر او
&&
خوش نباشد دادن آن جز به تو
\\
عقد کردش با امیر او را سپرد
&&
کرد خشم و حرص را او خرد و مرد
\\
\end{longtable}
\end{center}
