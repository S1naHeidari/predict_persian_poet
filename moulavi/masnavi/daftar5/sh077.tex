\begin{center}
\section*{بخش ۷۷ - خلق الجان من مارج من نار و قوله تعالی فی حق ابلیس انه  کان من الجن ففسق}
\label{sec:sh077}
\addcontentsline{toc}{section}{\nameref{sec:sh077}}
\begin{longtable}{l p{0.5cm} r}
شعله می‌زد آتش جان سفیه
&&
که آتشی بود الولد سر ابیه
\\
نه غلط گفتم که بد قهر خدا
&&
علتی را پیش آوردن چرا
\\
کار بی‌علت مبرا از علل
&&
مستمر و مستقرست از ازل
\\
در کمال صنع پاک مستحث
&&
علت حادث چه گنجد یا حدث
\\
سر آب چه بود آب ما صنع اوست
&&
صنع مغزست و آب صورت چو پوست
\\
عشق دان ای فندق تن دوستت
&&
جانت جوید مغز و کوبد پوستت
\\
دوزخی که پوست باشد دوستش
&&
داد بدلنا جلودا پوستش
\\
معنی و مغزت بر آتش حاکمست
&&
لیک آتش را قشورت هیزمست
\\
کوزهٔ چوبین که در وی آب جوست
&&
قدرت آتش همه بر ظرف اوست
\\
معنی انسان بر آتش مالکست
&&
مالک دوزخ درو کی هالکست
\\
پس میفزا تو بدن معنی فزا
&&
تا چو مالک باشی آتش را کیا
\\
پوستها بر پوست می‌افزوده‌ای
&&
لاجرم چون پوست اندر دوده‌ای
\\
زانک آتش را علف جز پوست نیست
&&
قهر حق آن کبر را پوستین کنیست
\\
این تکبر از نتیجهٔ پوستست
&&
جاه و مال آن کبر را زان دوستست
\\
این تکبر چیست غفلت از لباب
&&
منجمد چون غفلت یخ ز آفتاب
\\
چون خبر شد ز آفتابش یخ نماند
&&
نرم گشت و گرم گشت و تیز راند
\\
شد ز دید لب جملهٔ تن طمع
&&
خوار و عاشق شد که ذل من طمع
\\
چون نبیند مغز قانع شد به پوست
&&
بند عز من قنع زندان اوست
\\
عزت اینجا گبریست و ذل دین
&&
سنگ تا فانی نشد کی شد نگین
\\
در مقام سنگی آنگاهی انا
&&
وقت مسکین گشتن تست وفنا
\\
کبر زان جوید همیشه جاه و مال
&&
که ز سرگینست گلحن را کمال
\\
کین دو دایه پوست را افزون کنند
&&
شحم و لحم و کبر و نخوت آکنند
\\
دیده را بر لب لب نفراشتند
&&
پوست را زان روی لب پنداشتند
\\
پیش‌وا ابلیس بود این راه را
&&
کو شکار آمد شبیکهٔ جاه را
\\
مال چون مارست و آن جاه اژدها
&&
سایهٔ مردان زمرد این دو را
\\
زان زمرد مار را دیده جهد
&&
کور گردد مار و ره‌رو وا رهد
\\
چون برین ره خار بنهاد آن رئیس
&&
هر که خست او گفته لعنت بر بلیس
\\
یعنی این غم بر من از غدر ویست
&&
غدر را آن مقتدا سابق‌پیست
\\
بعد ازو خود قرن بر قرن آمدند
&&
جملگان بر سنت او پا زدند
\\
هر که بنهد سنت بد ای فتا
&&
تا در افتد بعد او خلق از عمی
\\
جمع گردد بر وی آن جمله بزه
&&
کو سری بودست و ایشان دم‌غزه
\\
لیک آدم چارق و آن پوستین
&&
پیش می‌آورد که هستم ز طین
\\
چون ایاز آن چارقش مورود بود
&&
لاجرم او عاقبت محمود بود
\\
هست مطلق کارساز نیستیست
&&
کارگاه هست‌کن جز نیست چیست
\\
بر نوشته هیچ بنویسد کسی
&&
یا نهاله کارد اندر مغرسی
\\
کاغذی جوید که آن بنوشته نیست
&&
تخم کارد موضعی که کشته نیست
\\
تو برادر موضع ناکشته باش
&&
کاغذ اسپید نابنوشته باش
\\
تا مشرف گردی از نون والقلم
&&
تا بکارد در تو تخم آن ذوالکرم
\\
خود ازین پالوه نالیسیده گیر
&&
مطبخی که دیده‌ای نادیده گیر
\\
زانک ازین پالوده مستیها بود
&&
پوستین و چارق از یادت رود
\\
چون در آید نزع و مرگ آهی کنی
&&
ذکر دلق و چارق آنگاهی کنی
\\
تا نمانی غرق موج زشتیی
&&
که نباشد از پناهی پشتیی
\\
یاد ناری از سفینهٔ راستین
&&
ننگری رد چارق و در پوستین
\\
چونک درمانی به غرقاب فنا
&&
پس ظلمنا ورد سازی بر ولا
\\
دیو گوید بنگرید این خام را
&&
سر برید این مرغ بی‌هنگام را
\\
دور این خصلت ز فرهنگ ایاز
&&
که پدید آید نمازش بی‌نماز
\\
او خروس آسمان بوده ز پیش
&&
نعره‌های او همه در وقت خویش
\\
\end{longtable}
\end{center}
