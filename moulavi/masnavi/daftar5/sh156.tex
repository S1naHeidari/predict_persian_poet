\begin{center}
\section*{بخش ۱۵۶ - حکایت آن مهمان کی زن خداوند خانه گفت کی باران فرو گرفت و مهمان در گردن ما ماند}
\label{sec:sh156}
\addcontentsline{toc}{section}{\nameref{sec:sh156}}
\begin{longtable}{l p{0.5cm} r}
آن یکی را بیگهان آمد قنق
&&
ساخت او را هم‌چو طوق اندر عنق
\\
خوان کشید او را کرامتها نمود
&&
آن شب اندر کوی ایشان سور بود
\\
مرد زن را گفت پنهانی سخن
&&
که امشب ای خاتون دو جامه خواب کن
\\
پستر ما را بگستر سوی در
&&
بهر مهمان گستر آن سوی دگر
\\
گفت زن خدمت کنم شادی کنم
&&
سمع و طاعه ای دو چشم روشنم
\\
هر دو پستر گسترید و رفت زن
&&
سوی ختنه‌سور کرد آنجا وطن
\\
ماند مهمان عزیز و شوهرش
&&
نقل بنهادند از خشک و ترش
\\
در سمر گفتند هر دو منتجب
&&
سرگذشت نیک و بد تا نیم شب
\\
بعد از آن مهمان ز خواب و از سمر
&&
شد در آن پستر که بد آن سوی در
\\
شوهر از خجلت بدو چیزی نگفت
&&
که ترا این سوست ای جان جای خفت
\\
که برای خواب تو ای بوالکرم
&&
پستر آن سوی دگر افکنده‌ام
\\
آن قراری که به زن او داده بود
&&
گشت مبدل و آن طرف مهمان غنود
\\
آن شب آنجا سخت باران در گرفت
&&
کز غلیظی ابرشان آمد شگفت
\\
زن بیامد بر گمان آنک شو
&&
سوی در خفتست و آن سو آن عمو
\\
رفت عریان در لحاف آن دم عروس
&&
داد مهمان را به رغبت چند بوس
\\
گفت می‌ترسیدم ای مرد کلان
&&
خود همان آمد همان آمد همان
\\
مرد مهمان را گل و باران نشاند
&&
بر تو چون صابون سلطانی بماند
\\
اندرین باران و گل او کی رود
&&
بر سر و جان تو او تاوان شود
\\
زود مهمان جست و گفت این زن بهل
&&
موزه دارم غم ندارم من ز گل
\\
من روان گشتم شما را خیر باد
&&
در سفر یک دم مبادا روح شاد
\\
تا که زوتر جانب معدن رود
&&
کین خوشی اندر سفر ره‌زن شود
\\
زن پشیمان شد از آن گفتار سرد
&&
چون رمید و رفت آن مهمان فرد
\\
زن بسی گفتش که آخر ای امیر
&&
گر مزاحی کردم از طیبت مگیر
\\
سجده و زاری زن سودی نداشت
&&
رفت و ایشان را در آن حسرت گذاشت
\\
جامه ازرق کرد زان پس مرد و زن
&&
صورتش دیدند شمعی بی‌لگن
\\
می‌شد و صحرا ز نور شمع مرد
&&
چون بهشت از ظلمت شب گشته فرد
\\
کرد مهمان خانه خانهٔ خویش را
&&
از غم و از خجلت این ماجرا
\\
در درون هر دو از راه نهان
&&
هر زمان گفتی خیال میهمان
\\
که منم یار خضر صد گنج و جود
&&
می‌فشاندم لیک روزیتان نبود
\\
\end{longtable}
\end{center}
