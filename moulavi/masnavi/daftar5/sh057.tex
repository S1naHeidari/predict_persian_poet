\begin{center}
\section*{بخش ۵۷ - یکی پرسید از عالمی عارفی کی اگر در نماز کسی بگرید به آواز و آه کند و نوحه کند نمازش باطل شود جواب گفت کی نام آن آب دیده است تا آن گرینده چه دیده است اگر شوق خدا دیده است و می‌گرید یا پشیمانی گناهی نمازش تباه نشود بلک کمال گیرد کی لا صلوة الا بحضور القلب و اگر او رنجوری تن یا فراق فرزند دیده است نمازش تباه شود کی اصل نماز ترک تن است و ترک فرزند ابراهیم‌وار کی فرزند را قربان می‌کرد از بهر تکمیل نماز و تن را به آتش نمرود می‌سپرد و امر آمد مصطفی را علیه‌السلام  بدین خصال کی فاتبع ملة ابراهیم لقد کانت لکم اسوة حسنة فی‌ابراهیم}
\label{sec:sh057}
\addcontentsline{toc}{section}{\nameref{sec:sh057}}
\begin{longtable}{l p{0.5cm} r}
آن یکی پرسید از مفتی به راز
&&
گر کسی گرید به نوحه در نماز
\\
آن نماز او عجب باطل شود
&&
یا نمازش جایز و کامل بود
\\
گفت آب دیده نامش بهر چیست
&&
بنگری تا که چه دید او و گریست
\\
آب دیده تا چه دید او از نهان
&&
تا بدان شد او ز چشمهٔ خود روان
\\
آن جهان گر دیده است آن پر نیاز
&&
رونقی یابد ز نوحه آن نماز
\\
ور ز رنج تن بد آن گریه و ز سوک
&&
ریسمان بسکست و هم بشکست دوک
\\
\end{longtable}
\end{center}
