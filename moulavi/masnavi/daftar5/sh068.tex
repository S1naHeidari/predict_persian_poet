\begin{center}
\section*{بخش ۶۸ - فرستادن عزرائیل ملک العزم و الحزم را علیه‌السلام ببر گرفتن  حفنه‌ای خاک تا شود جسم آدم چالاک عیله‌السلام و الصلوة}
\label{sec:sh068}
\addcontentsline{toc}{section}{\nameref{sec:sh068}}
\begin{longtable}{l p{0.5cm} r}
گفت یزدان زو عزرائیل را
&&
که ببین آن خاک پر تخییل را
\\
آن ضعیف زال ظالم را بیاب
&&
مشت خاکی هین بیاور با شتاب
\\
رفت عزرائیل سرهنگ قضا
&&
سوی کرهٔ خاک بهر اقتضا
\\
خاک بر قانون نفیر آغاز کرد
&&
داد سوگندش بسی سوگند خورد
\\
کای غلام خاص و ای حمال عرش
&&
ای مطاع الامر اندر عرش و فرش
\\
رو به حق رحمت رحمن فرد
&&
رو به حق آنک با تو لطف کرد
\\
حق شاهی که جز او معبود نیست
&&
پیش او زاری کس مردود نیست
\\
گفت نتوانم بدین افسون که من
&&
رو بتابم ز آمر سر و علن
\\
گفت آخر امر فرمود او به حلم
&&
هر دو امرند آن بگیر از راه علم
\\
گفت آن تاویل باشد یا قیاس
&&
در صریح امر کم جو التباس
\\
فکر خود را گر کنی تاویل به
&&
که کنی تاویل این نامشتبه
\\
دل همی‌سوزد مرا بر لابه‌ات
&&
سینه‌ام پر خون شد از شورابه‌ات
\\
نیستم بی‌رحم بل زان هر سه پاک
&&
رحم بیشستم ز درد دردناک
\\
گر طبانجه می‌زنم من بر یتیم
&&
ور دهد حلوا به دستش آن حلیم
\\
این طبانجه خوشتر از حلوای او
&&
ور شود غره به حلوا وای او
\\
بر نفیر تو جگر می‌سوزدم
&&
لیک حق لطفی همی‌آموزدم
\\
لطف مخفی در میان قهرها
&&
در حدث پنهان عقیق بی‌بها
\\
قهر حق بهتر ز صد حلم منست
&&
منع کردن جان ز حق جان کندنست
\\
بترین قهرش به از حلم دو کون
&&
نعم رب‌العالمین و نعم عون
\\
لطفهای مضمر اندر قهر او
&&
جان سپردن جان فزاید بهر او
\\
هین رها کن بدگمانی و ضلال
&&
سر قدم کن چونک فرمودت تعال
\\
آن تعال او تعالیها دهد
&&
مستی و جفت و نهالیها دهد
\\
باری آن امر سنی را هیچ هیچ
&&
من نیارم کرد وهن و پیچ پیچ
\\
این همه بشنید آن خاک نژند
&&
زان گمان بد بدش در گوش بند
\\
باز از نوعی دگر آن خاک پست
&&
لابه و سجده همی‌کرد او چو مست
\\
گفت نه برخیز نبود زین زیان
&&
من سر و جان می‌نهم رهن و ضمان
\\
لابه مندیش و مکن لابه دگر
&&
جز بدان شاه رحیم دادگر
\\
بنده فرمانم نیارم ترک کرد
&&
امر او کز بحر انگیزید گرد
\\
جز از آن خلاق گوش و چشم و سر
&&
نشنوم از جان خود هم خیر و شر
\\
گوش من از گفت غیر او کرست
&&
او مرا از جان شیرین جان‌ترست
\\
جان ازو آمد نیامد او ز جان
&&
صدهزاران جان دهم او رایگان
\\
جان کی باشد کش گزینم بر کریم
&&
کیک چه بود که بسوزم زو گلیم
\\
من ندانم خیر الا خیر او
&&
صم و بکم و عمی من از غیر او
\\
گوش من کرست از زاری‌کنان
&&
که منم در کف او هم‌چون سنان
\\
\end{longtable}
\end{center}
