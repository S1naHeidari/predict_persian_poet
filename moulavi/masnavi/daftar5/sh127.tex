\begin{center}
\section*{بخش ۱۲۷ - حکایت آن راهب که روز با چراغ می‌گشت در میان بازار از  سر حالتی کی او را بود}
\label{sec:sh127}
\addcontentsline{toc}{section}{\nameref{sec:sh127}}
\begin{longtable}{l p{0.5cm} r}
آن یکی با شمع برمی‌گشت روز
&&
گرد بازاری دلش پر عشق و سوز
\\
بوالفضولی گفت او را کای فلان
&&
هین چه می‌جویی به سوی هر دکان
\\
هین چه می‌گردی تو جویان با چراغ
&&
در میان روز روشن چیست لاغ
\\
گفت می‌جویم به هر سو آدمی
&&
که بود حی از حیات آن دمی
\\
هست مردی گفت این بازار پر
&&
مردمانند آخر ای دانای حر
\\
گفت خواهم مرد بر جادهٔ دو ره
&&
در ره خشم و به هنگام شره
\\
وقت خشم و وقت شهوت مرد کو
&&
طالب مردی دوانم کو به کو
\\
کو درین دو حال مردی در جهان
&&
تا فدای او کنم امروز جان
\\
گفت نادر چیز می‌جویی ولیک
&&
غافل از حکم و قضایی بین تو نیک
\\
ناظر فرعی ز اصلی بی‌خبر
&&
فرع ماییم اصل احکام قدر
\\
چرخ گردان را قضا گمره کند
&&
صدعطارد را قضا ابله کند
\\
تنگ گرداند جهان چاره را
&&
آب گرداند حدید و خاره را
\\
ای قراری داده ره را گام گام
&&
خام خامی خام خامی خام خام
\\
چون بدیدی گردش سنگ آسیا
&&
آب جو را هم ببین آخر بیا
\\
خاک را دیدی برآمد در هوا
&&
در میان خاک بنگر باد را
\\
دیگهای فکر می‌بینی به جوش
&&
اندر آتش هم نظر می‌کن به هوش
\\
گفت حق ایوب را در مکرمت
&&
من بهر موییت صبری دادمت
\\
هین به صبر خود مکن چندین نظر
&&
صبر دیدی صبر دادن را نگر
\\
چند بینی گردش دولاب را
&&
سر برون کن هم ببین تیز آب را
\\
تو همی‌گویی که می‌بینم ولیک
&&
دید آن را بس علامتهاست نیک
\\
گردش کف را چو دیدی مختصر
&&
حیرتت باید به دریا در نگر
\\
آنک کف را دید سر گویان بود
&&
وانک دریا دید او حیران بود
\\
آنک کف را دید نیتها کند
&&
وانک دریا دید دل دریا کند
\\
آنک کفها دید باشد در شمار
&&
و آنک دریا دید شد بی‌اختیار
\\
آنک او کف دید در گردش بود
&&
وانک دریا دید او بی‌غش بود
\\
\end{longtable}
\end{center}
