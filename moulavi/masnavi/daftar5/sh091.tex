\begin{center}
\section*{بخش ۹۱ - یافته شدن گوهر و حلالی خواستن حاجبکان و کنیزکان شاه‌زاده از نصوح}
\label{sec:sh091}
\addcontentsline{toc}{section}{\nameref{sec:sh091}}
\begin{longtable}{l p{0.5cm} r}
بعد از آن خوفی هلاک جان بده
&&
مژده‌ها آمد که اینک گم شده
\\
بانگ آمد ناگهان که رفت بیم
&&
یافت شد گم گشته آن در یتیم
\\
یافت شد واندر فرح در بافتیم
&&
مژدگانی ده که گوهر یافتیم
\\
از غریو و نعره و دستک زدن
&&
پر شده حمام قد زال الحزن
\\
آن نصوح رفته باز آمد به خویش
&&
دید چشمش تابش صد روز بیش
\\
می حلالی خواست از وی هر کسی
&&
بوسه می‌دادند بر دستش بسی
\\
بد گمان بردیم و کن ما را حلال
&&
گوشت تو خوردیم اندر قیل و قال
\\
زانک ظن جمله بر وی بیش بود
&&
زانک در قربت ز جمله پیش بود
\\
خاص دلاکش بد و محرم نصوح
&&
بلک هم‌چون دو تنی یک گشته روح
\\
گوهر ار بردست او بردست و بس
&&
زو ملازم‌تر به خاتون نیست کس
\\
اول او را خواست جستن در نبرد
&&
بهر حرمت داشتش تاخیر کرد
\\
تا بود کان را بیندازد به جا
&&
اندرین مهلت رهاند خویش را
\\
این حلالیها ازو می‌خواستند
&&
وز برای عذر برمی‌خاستند
\\
گفت بد فضل خدای دادگر
&&
ورنه زآنچم گفته شد هستم بتر
\\
چه حلالی خواست می‌باید ز من
&&
که منم مجرم‌تر اهل زمن
\\
آنچ گفتندم ز بد از صد یکیست
&&
بر من این کشفست ار کس را شکیست
\\
کس چه می‌داند ز من جز اندکی
&&
از هزاران جرم و بد فعلم یکی
\\
من همی دانم و آن ستار من
&&
جرمها و زشتی کردار من
\\
اول ابلیسی مرا استاد بود
&&
بعد از آن ابلیس پیشم باد بود
\\
حق بدید آن جمله را نادیده کرد
&&
تا نگردم در فضیحت روی‌زرد
\\
باز رحمت پوستین دوزیم کرد
&&
توبهٔ شیرین چو جان روزیم کرد
\\
هر چه کردم جمله ناکرده گرفت
&&
طاعت ناکرده آورده گرفت
\\
هم‌چو سرو و سوسنم آزاد کرد
&&
هم‌چو بخت و دولتم دلشاد کرد
\\
نام من در نامهٔ پاکان نوشت
&&
دوزخی بودم ببخشیدم بهشت
\\
آه کردم چون رسن شد آه من
&&
گشت آویزان رسن در چاه من
\\
آن رسن بگرفتم و بیرون شدم
&&
شاد و زفت و فربه و گلگون شدم
\\
در بن چاهی همی‌بودم زبون
&&
در همه عالم نمی‌گنجم کنون
\\
آفرینها بر تو بادا ای خدا
&&
ناگهان کردی مرا از غم جدا
\\
گر سر هر موی من یابد زبان
&&
شکرهای تو نیاید در بیان
\\
می‌زنم نعره درین روضه و عیون
&&
خلق را یا لیت قومی یعلمون
\\
\end{longtable}
\end{center}
