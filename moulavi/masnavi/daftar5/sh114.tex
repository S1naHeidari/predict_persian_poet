\begin{center}
\section*{بخش ۱۱۴ - آمدن شیخ بعد از چندین سال از بیابان به شهر غزنین و  زنبیل گردانیدن به اشارت غیبی و تفرقه کردن آنچ جمع آید بر فقرا  هر که را جان عز لبیکست  نامه بر نامه پیک بر پیکست  چنانک روزن خانه باز باشد آفتاب و ماهتاب و باران و نامه و غیره  منقطع نباشد}
\label{sec:sh114}
\addcontentsline{toc}{section}{\nameref{sec:sh114}}
\begin{longtable}{l p{0.5cm} r}
رو به شهر آورد آن فرمان‌پذیر
&&
شهر غزنین گشت از رویش منیر
\\
از فرح خلقی به استقبال رفت
&&
او در آمد از ره دزدیده تفت
\\
جمله اعیان و مهان بر خاستند
&&
قصرها از بهر او آراستند
\\
گفت من از خودنمایی نامدم
&&
جز به خواری و گدایی نامدم
\\
نیستم در عزم قال و قیل من
&&
در به در گردم به کف زنبیل من
\\
بنده فرمانم که امرست از خدا
&&
که گدا باشم گدا باشم گدا
\\
در گدایی لفظ نادر ناورم
&&
جز طریق خس گدایان نسپرم
\\
تا شوم غرقهٔ مذلت من تمام
&&
تا سقطها بشنوم از خاص و عام
\\
امر حق جانست و من آن را تبع
&&
او طمع فرمود ذل من طمع
\\
چون طمع خواهد ز من سلطان دین
&&
خاک بر فرق قناعت بعد ازین
\\
او مذلت خواست کی عزت تنم
&&
او گدایی خواست کی میری کنم
\\
بعد ازین کد و مذلت جان من
&&
بیست عباس‌اند در انبان من
\\
شیخ بر می‌گشت زنبیلی به دست
&&
شیء لله خواجه توفیقیت هست
\\
برتر از کرسی و عرش اسرار او
&&
شیء لله شیء لله کار او
\\
انبیا هر یک همین فن می‌زنند
&&
خلق مفلس کدیه ایشان می‌کنند
\\
اقرضوا الله اقرضوا الله می‌زنند
&&
بازگون بر انصروا الله می‌تنند
\\
در به در این شیخ می‌آرد نیاز
&&
بر فلک صد در برای شیخ باز
\\
که آن گدایی که آن به جد می‌کرد او
&&
بهر یزدان بود نه از بهر گلو
\\
ور بکردی نیز از بهر گلو
&&
آن گلو از نور حق دارد غلو
\\
در حق او خورد نان و شهد و شیر
&&
به ز چله وز سه روزهٔ صد فقیر
\\
نور می‌نوشد مگو نان می‌خورد
&&
لاله می‌کارد به صورت می‌چرد
\\
چون شراری کو خورد روغن ز شمع
&&
نور افزاید ز خوردش بهر جمع
\\
نان‌خوری را گفت حق لاتسرفوا
&&
نور خوردن را نگفتست اکتفوا
\\
آن گلوی ابتلا بد وین گلو
&&
فارغ از اسراف و آمن از غلو
\\
امر و فرمان بود نه حرص و طمع
&&
آن چنان جان حرص را نبود تبع
\\
گر بگوید کیمیا مس را بده
&&
تو به من خود را طمع نبود فره
\\
گنجهای خاک تا هفتم طبق
&&
عرضه کرده بود پیش شیخ حق
\\
شیخ گفتا خالقا من عاشقم
&&
گر بجویم غیر تو من فاسقم
\\
هشت جنت گر در آرم در نظر
&&
ور کنم خدمت من از خوف سقر
\\
ممنی باشم سلامت‌جوی من
&&
زانک این هر دو بود حظ بدن
\\
عاشقی کز عشق یزدان خورد قوت
&&
صد بدن پیشش نیرزد تره‌توت
\\
وین بدن که دارد آن شیخ فطن
&&
چیز دگر گشت کم خوانش بدن
\\
عاشق عشق خدا وانگاه مزد
&&
جبرئیل مؤتمن وانگاه دزد
\\
عاشق آن لیلی کور و کبود
&&
ملک عالم پیش او یک تره بود
\\
پیش او یکسان شده بد خاک و زر
&&
زر چه باشد که نبد جان را خطر
\\
شیر و گرگ و دد ازو واقف شده
&&
هم‌چو خویشان گرد او گرد آمده
\\
کین شدست از خوی حیوان پاک پاک
&&
پر ز عشق و لحم و شحمش زهرناک
\\
زهر دد باشد شکرریز خرد
&&
زانک نیک نیک باشد ضد بد
\\
لحم عاشق را نیارد خورد دد
&&
عشق معروفست پیش نیک و بد
\\
ور خورد خود فی‌المثل دام و ددش
&&
گوشت عاشق زهر گردد بکشدش
\\
هر چه جز عشقست شد ماکول عشق
&&
دو جهان یک دانه پیش نول عشق
\\
دانه‌ای مر مرغ را هرگز خورد
&&
کاهدان مر اسپ را هرگز چرد
\\
بندگی کن تا شوی عاشق لعل
&&
بندگی کسبیست آید در عمل
\\
بنده آزادی طمع دارد ز جد
&&
عاشق آزادی نخواهد تا ابد
\\
بنده دایم خلعت و ادرارجوست
&&
خلعت عاشق همه دیدار دوست
\\
در نگنجد عشق در گفت و شنید
&&
عشق دریاییست قعرش ناپدید
\\
قطره‌های بحر را نتوان شمرد
&&
هفت دریا پیش آن بحرست خرد
\\
این سخن پایان ندارد ای فلان
&&
باز رو در قصهٔ شیخ زمان
\\
\end{longtable}
\end{center}
