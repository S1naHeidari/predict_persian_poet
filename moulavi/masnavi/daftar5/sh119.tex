\begin{center}
\section*{بخش ۱۱۹ - دانستن شیخ ضمیر سایل را بی گفتن و دانستن قدر وام وام‌داران  بی گفتن کی نشان آن باشد کی اخرج به صفاتی الی خلقی}
\label{sec:sh119}
\addcontentsline{toc}{section}{\nameref{sec:sh119}}
\begin{longtable}{l p{0.5cm} r}
حاجت خود گر نگفتی آن فقیر
&&
او بدادی و بدانستی ضمیر
\\
آنچ در دل داشتی آن پشت‌خم
&&
قدر آن دادی بدو نه بیش و کم
\\
پس بگفتندی چه دانستی که او
&&
این قدر اندیشه دارد ای عمو
\\
او بگفتی خانهٔ دل خلوتست
&&
خالی از کدیه مثال جنتست
\\
اندرو جز عشق یزدان کار نیست
&&
جز خیال وصل او دیار نیست
\\
خانه را من روفتم از نیک و بد
&&
خانه‌ام پرست از عشق احد
\\
هرچه بینم اندرو غیر خدا
&&
آن من نبود بود عکس گدا
\\
گر در آبی نخل یا عرجون نمود
&&
جز ز عکس نخلهٔ بیرون نبود
\\
در تگ آب ار ببینی صورتی
&&
عکس بیرون باشد آن نقش ای فتی
\\
لیک تا آب از قذی خالی شدن
&&
تنقیه شرطست در جوی بدن
\\
تا نماند تیرگی و خس درو
&&
تا امین گردد نماید عکس رو
\\
جز گلابه در تنت کو ای مقل
&&
آب صافی کن ز گل ای خصم دل
\\
تو بر آنی هر دمی کز خواب و خور
&&
خاک ریزی اندرین جو بیشتر
\\
\end{longtable}
\end{center}
