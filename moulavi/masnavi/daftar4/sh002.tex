\begin{center}
\section*{بخش ۲ - تمامی حکایت آن عاشق که از عسس گریخت در باغی مجهول خود معشوق را در باغ یافت و عسس را از شادی دعای خیر می‌کرد و می‌گفت کی عسی ان تکرهوا  شیا و هو خیر لکم}
\label{sec:sh002}
\addcontentsline{toc}{section}{\nameref{sec:sh002}}
\begin{longtable}{l p{0.5cm} r}
اندر آن بودیم کان شخص از عسس
&&
راند اندر باغ از خوفی فرس
\\
بود اندر باغ آن صاحب‌جمال
&&
کز غمش این در عنا بد هشت سال
\\
سایهٔ او را نبود امکان دید
&&
هم‌چو عنقا وصف او را می‌شنید
\\
جز یکی لقیه که اول از قضا
&&
بر وی افتاد و شد او را دلربا
\\
بعد از آن چندان که می‌کوشید او
&&
خود مجالش می‌نداد آن تندخو
\\
نه بلا به چاره بودش نه به مال
&&
چشم پر و بی‌طمع بود آن نهال
\\
عاشق هر پیشه‌ای و مطلبی
&&
حق بیالود اول کارش لبی
\\
چون بدان آسیب در جست آمدند
&&
پیش پاشان می‌نهد هر روز بند
\\
چون در افکندش بجست و جوی کار
&&
بعد از آن در بست که کابین بیار
\\
هم بر آن بو می‌تنند و می‌روند
&&
هر دمی راجی و آیس می‌شوند
\\
هر کسی را هست اومید بری
&&
که گشادندش در آن روزی دری
\\
باز در بستندش و آن درپرست
&&
بر همان اومید آتش پا شدست
\\
چون درآمد خوش در آن باغ آن جوان
&&
خود فرو شد پا به گنجش ناگهان
\\
مر عسس را ساخته یزدان سبب
&&
تا ز بیم او دود در باغ شب
\\
بیند آن معشوقه را او با چراغ
&&
طالب انگشتری در جوی باغ
\\
پس قرین می‌کرد از ذوق آن نفس
&&
با ثنای حق دعای آن عسس
\\
که زیان کردم عسس را از گریز
&&
بیست چندان سیم و زر بر وی بریز
\\
از عوانی مر ورا آزاد کن
&&
آنچنان که شادم او را شاد کن
\\
سعد دارش این جهان و آن جهان
&&
از عوانی و سگی‌اش وا رهان
\\
گرچه خوی آن عوان هست ای خدا
&&
که هماره خلق را خواهد بلا
\\
گر خبر آید که شه جرمی نهاد
&&
بر مسلمانان شود او زفت و شاد
\\
ور خبر آید که شه رحمت نمود
&&
از مسلمانان فکند آن را به جود
\\
ماتمی در جان او افتد از آن
&&
صد چنین ادبارها دارد عوان
\\
او عوان را در دعا در می‌کشید
&&
کز عوان او را چنان راحت رسید
\\
بر همه زهر و برو تریاق بود
&&
آن عوان پیوند آن مشتاق بود
\\
پس بد مطلق نباشد در جهان
&&
بد به نسبت باشد این را هم بدان
\\
در زمانه هیچ زهر و قند نیست
&&
که یکی را پا دگر را بند نیست
\\
مر یکی را پا دگر را پای‌بند
&&
مر یکی را زهر و بر دیگر چو قند
\\
زهر مار آن مار را باشد حیات
&&
نسبتش با آدمی باشد ممات
\\
خلق آبی را بود دریا چو باغ
&&
خلق خاکی را بود آن مرگ و داغ
\\
همچنین بر می‌شمر ای مرد کار
&&
نسبت این از یکی کس تا هزار
\\
زید اندر حق آن شیطان بود
&&
در حق شخصی دگر سلطان بود
\\
آن بگوید زید صدیق سنیست
&&
وین بگوید زید گبر کشتنیست
\\
گر تو خواهی کو ترا باشد شکر
&&
پس ورا از چشم عشاقش نگر
\\
منگر از چشم خودت آن خوب را
&&
بین به چشم طالبان مطلوب را
\\
چشم خود بر بند زان خوش‌چشم تو
&&
عاریت کن چشم از عشاق او
\\
بلک ازو کن عاریت چشم و نظر
&&
پس ز چشم او بروی او نگر
\\
تا شوی آمن ز سیری و ملال
&&
گفت کان الله له زین ذوالجلال
\\
چشم او من باشم و دست و دلش
&&
تا رهد از مدبریها مقبلش
\\
هر چه مکرو هست چون شد او دلیل
&&
سوی محبوبت حبیبست و خلیل
\\
\end{longtable}
\end{center}
