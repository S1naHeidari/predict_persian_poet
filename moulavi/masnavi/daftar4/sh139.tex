\begin{center}
\section*{بخش ۱۳۹ - نمودن جبرئیل علیه‌السلام خود را به مصطفی صلی‌الله علیه و  سلم به صورت خویش و از هفتصد پر او چون یک پر ظاهر شد افق را  بگرفت و آفتاب محجوب شد با همه شعاعش}
\label{sec:sh139}
\addcontentsline{toc}{section}{\nameref{sec:sh139}}
\begin{longtable}{l p{0.5cm} r}
مصطفی می‌گفت پیش جبرئیل
&&
که چنانک صورت تست ای خلیل
\\
مر مرا بنما تو محسوس آشکار
&&
تا ببینم مر ترا نظاره‌وار
\\
گفت نتوانی و طاقت نبودت
&&
حس ضعیفست و تنک سخت آیدت
\\
گفت بنما تا ببیند این جسد
&&
تا چد حد حس نازکست و بی‌مدد
\\
آدمی را هست حس تن سقیم
&&
لیک در باطن یکی خلقی عظیم
\\
بر مثال سنگ و آهن این تنه
&&
لیک هست او در صفت آتش‌زنه
\\
سنگ وآهن مولد ایجاد نار
&&
زاد آتش بر دو والد قهربار
\\
باز آتش دستکار وصف تن
&&
هست قاهر بر تن او و شعله‌زن
\\
باز در تن شعله ابراهیم‌وار
&&
که ازو مقهور گردد برج نار
\\
لاجرم گفت آن رسول ذو فنون
&&
رمز نحن الاخرون السابقون
\\
ظاهر این دو بسندانی زبون
&&
در صفت از کان آهنها فزون
\\
پس به صورت آدمی فرع جهان
&&
وز صفت اصل جهان این را بدان
\\
ظاهرش را پشه‌ای آرد به چرخ
&&
باطنش باشد محیط هفت چرخ
\\
چونک کرد الحاح بنمود اندکی
&&
هیبتی که که شود زومند کی
\\
شهپری بگرفته شرق و غرب را
&&
از مهابت گشت بیهش مصطفی
\\
چون ز بیم و ترس بیهوشش بدید
&&
جبرئیل آمد در آغوشش کشید
\\
آن مهابت قسمت بیگانگان
&&
وین تجمش دوستان را رایگان
\\
هست شاهان را زمان بر نشست
&&
هول سرهنگان و صارمها به دست
\\
دور باش و نیزه و شمشیرها
&&
که بلرزند از مهابت شیرها
\\
بانگ چاوشان و آن چوگانها
&&
که شود سست از نهیبش جانها
\\
این برای خاص وعام ره‌گذر
&&
که کندشان از شهنشاهی خبر
\\
از برای عام باشد این شکوه
&&
تا کلاه کبر ننهند آن گروه
\\
تا من و ماهای ایشان بشکند
&&
نفس خودبین فتنه و شر کم کند
\\
شهر از آن آمن شود کان شهریار
&&
دارد اندر قهر زخم و گیر و دار
\\
پس بمیرد آن هوسها در نفوس
&&
هیبت شه مانع آید زان نحوس
\\
باز چون آید به سوی بزم خاص
&&
کی بود آنجا مهابت یا قصاص
\\
حلم در حلمست و رحمتها به جوش
&&
نشنوی از غیر چنگ و ناخروش
\\
طبل و کوس هول باشد وقت جنگ
&&
وقت عشرت با خواص آواز چنگ
\\
هست دیوان محاسب عام را
&&
وان پری رویان حریف جام را
\\
آن زره وآن خود مر چالیش‌راست
&&
وین حریر و رود مر تعریش‌راست
\\
این سخن پایان ندارد ای جواد
&&
ختم کن والله اعلم بالرشاد
\\
اندر احمد آن حسی کو غاربست
&&
خفته این دم زیر خاک یثربست
\\
وآن عظیم الخلق او کان صفدرست
&&
بی‌تغیر مقعد صدق اندرست
\\
جای تغییرات اوصاف تنست
&&
روح باقی آفتابی روشنست
\\
بی ز تغییری که لا شرقیة
&&
بی ز تبدیلی که لا غربیة
\\
آفتاب از ذره کی مدهوش شد
&&
شمع از پروانه کی بیهوش شد
\\
جسم احمد را تعلق بد بدآن
&&
این تغیر آن تن باشد بدان
\\
هم‌چو رنجوری و هم‌چون خواب و درد
&&
جان ازین اوصاف باشد پاک و فرد
\\
روبهش گر یک دمی آشفته بود
&&
شیر جان مانا که آن دم خفته بود
\\
خفته بود آن شیر کز خوابست پاک
&&
اینت شیر نرمسار سهمناک
\\
خفته سازد شیر خود را آنچنان
&&
که تمامش مرده دانند این سگان
\\
ورنه در عالم کرا زهره بدی
&&
که ربودی از ضعیفی تربدی
\\
کف احمد زان نظر مخدوش گشت
&&
بحر او از مهر کف پرجوش گشت
\\
مه همه کفست معطی نورپاش
&&
ماه را گر کف نباشد گو مباش
\\
احمد ار بگشاید آن پر جلیل
&&
تا ابد بیهوش ماند جبرئیل
\\
چون گذشت احمد ز سدره و مرصدش
&&
وز مقام جبرئیل و از حدش
\\
گفت او را هین بپر اندر پیم
&&
گفت رو رو من حریف تو نیم
\\
باز گفت او را بیا ای پرده‌سوز
&&
من باوج خود نرفتستم هنوز
\\
گفت بیرون زین حد ای خوش‌فر من
&&
گر زنم پری بسوزد پر من
\\
حیرت اندر حیرت آمد این قصص
&&
بیهشی خاصگان اندر اخص
\\
بیهشیها جمله اینجا بازیست
&&
چند جان داری که جان پردازیست
\\
جبرئیلا گر شریفی و عزیز
&&
تو نه‌ای پروانه و نه شمع نیز
\\
شمع چون دعوت کند وقت فروز
&&
جان پروانه نپرهیزد ز سوز
\\
این حدیث منقلب را گور کن
&&
شیر را برعکس صید گور کن
\\
بند کن مشک سخن‌شاشیت را
&&
وا مکن انبان قلماشیت را
\\
آنک بر نگذشت اجزاش از زمین
&&
پیش او معکوس و قلماشیست این
\\
لا تخالفهم حبیبی دارهم
&&
یا غریبا نازلا فی دارهم
\\
اعط ما شائوا وراموا وارضهم
&&
یا ظعینا ساکنا فی‌ارضهم
\\
تا رسیدن در شه و در ناز خوش
&&
رازیا با مرغزی می‌ساز خویش
\\
موسیا در پیش فرعون زمن
&&
نرم باید گفت قولا لینا
\\
آب اگر در روغن جوشان کنی
&&
دیگدان و دیگ را ویران کنی
\\
نرم گو لیکن مگو غیر صواب
&&
وسوسه مفروش در لین الخطاب
\\
وقت عصر آمد سخن کوتاه کن
&&
ای که عصرت عصر را آگاه کن
\\
گو تو مر گل‌خواره را که قند به
&&
نرمی فاسد مکن طینش مده
\\
نطق جان را روضهٔ جانیستی
&&
گر ز حرف و صوت مستغنیستی
\\
این سر خر در میان قندزار
&&
ای بسا کس را که بنهادست خار
\\
ظن ببرد از دور کان آنست و بس
&&
چون قج مغلوب وا می‌رفت پس
\\
صورت حرف آن سر خر دان یقین
&&
در رز معنی و فردوس برین
\\
ای ضیاء الحق حسام الدین در آر
&&
این سر خر را در آن بطیخ‌زار
\\
تا سر خر چون بمرد از مسلخه
&&
نشو دیگر بخشدش آن مطبخه
\\
هین ز ما صورت‌گری و جان ز تو
&&
نه غلط هم این خود و هم آن ز تو
\\
بر فلک محمودی ای خورشید فاش
&&
بر زمین هم تا ابد محمود باش
\\
تا زمینی با سمایی بلند
&&
یک‌دل و یک‌قبله و یک‌خو شوند
\\
تفرقه برخیزد و شرک و دوی
&&
وحدتست اندر وجود معنوی
\\
چون شناسد جان من جان ترا
&&
یاد آرند اتحاد ماجری
\\
موسی و هارون شوند اندر زمین
&&
مختلط خوش هم‌چو شیر و انگبین
\\
چون شناسد اندک و منکر شود
&&
منکری‌اش پردهٔ ساتر شود
\\
پس شناسایی بگردانید رو
&&
خشم کرد آن مه ز ناشکری او
\\
زین سبب جان نبی را جان بد
&&
ناشناسا گشت و پشت پای زد
\\
این همه خواندی فرو خوان لم یکن
&&
تا بدانی لج این گبر کهن
\\
پیش از آنک نقش احمد فر نمود
&&
نعت او هر گبر را تعویذ بود
\\
کین چنین کس هست تا آید پدید
&&
از خیال روش دلشان می‌طپید
\\
سجده می‌کردند کای رب بشر
&&
در عیان آریش هر چه زودتر
\\
تا به نام احمد از یستفتحون
&&
یاغیانشان می‌شدندی سرنگون
\\
هر کجا حرب مهولی آمدی
&&
غوثشان کراری احمد بدی
\\
هر کجا بیماری مزمن بدی
&&
یاد اوشان داروی شافی شدی
\\
نقش او می‌گشت اندر راهشان
&&
در دل و در گوش و در افواهشان
\\
نقش او را کی بیابد هر شعال
&&
بلک فرع نقش او یعنی خیال
\\
نقش او بر روی دیوار ار فتد
&&
از دل دیوار خون دل چکد
\\
آنچنان فرخ بود نقشش برو
&&
که رهد در حال دیوار از دو رو
\\
گشته با یک‌رویی اهل صفا
&&
آن دورویی عیب مر دیوار را
\\
این همه تعظیم و تفخیم و وداد
&&
چون بدیدندش به صورت برد باد
\\
قلب آتش دید و در دم شد سیاه
&&
قلب را در قلب کی بودست راه
\\
قلب می‌زد لاف اشواق محک
&&
تا مریدان را دراندازد به شک
\\
افتد اندر دام مکرش ناکسی
&&
این گمان سر بر زند از هر خسی
\\
کین اگر نه نقد پاکیزه بدی
&&
کی به سنگ امتحان راغب شدی
\\
او محک می‌خواهد اما آنچنان
&&
که نگردد قلبی او زان عیان
\\
آن محک که او نهان دارد صفت
&&
نی محک باشد نه نور معرفت
\\
آینه کو عیب رو دارد نهان
&&
از برای خاطر هر قلتبان
\\
آینه نبود منافق باشد او
&&
این چنین آیینه تا توانی مجو
\\
\end{longtable}
\end{center}
