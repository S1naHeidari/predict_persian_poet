\begin{center}
\section*{بخش ۷۷ - اعتراض کردن معترضی بر رسول علیه‌السلام بر امیر کردن آن هذیلی}
\label{sec:sh077}
\addcontentsline{toc}{section}{\nameref{sec:sh077}}
\begin{longtable}{l p{0.5cm} r}
چون پیمبر سروری کرد از هذیل
&&
از برای لشکر منصور خیل
\\
بوالفضولی از حسد طاقت نداشت
&&
اعتراض و لانسلم بر فراشت
\\
خلق را بنگر که چون ظلمانی‌اند
&&
در متاع فانیی چون فانی‌اند
\\
از تکبر جمله اندر تفرقه
&&
مرده از جان زنده‌اند از مخرقه
\\
این عجب که جان به زندان اندرست
&&
وانگهی مفتاح زندانش به دست
\\
پای تا سر غرق سرگین آن جوان
&&
می‌زند بر دامنش جوی روان
\\
دایما پهلو به پهلو بی‌قرار
&&
پهلوی آرامگاه و پشت‌دار
\\
نور پنهانست و جست و جو گواه
&&
کز گزافه دل نمی‌جوید پناه
\\
گر نبودی حبس دنیا را مناص
&&
نه بدی وحشت نه دل جستی خلاص
\\
وحشتت هم‌چون موکل می‌کشد
&&
که بجو ای ضال منهاج رشد
\\
هست منهاج و نهان در مکمنست
&&
یافتش رهن گزافه جستنست
\\
تفرقه‌جویان جمع اندر کمین
&&
تو درین طالب رخ مطلوب بین
\\
مردگان باغ برجسته ز بن
&&
کان دهندهٔ زندگی را فهم کن
\\
چشم این زندانیان هر دم به در
&&
کی بدی گر نیستی کس مژده‌ور
\\
صد هزار آلودگان آب‌جو
&&
کی بدندی گر نبودی آب جو
\\
بر زمین پهلوت را آرام نیست
&&
دان که در خانه لحاف و بستریست
\\
بی‌مقرگاهی نباشد بی‌قرار
&&
بی‌خمار اشکن نباشد این خمار
\\
گفت نه نه یا رسول الله مکن
&&
سرور لشکر مگر شیخ کهن
\\
یا رسول الله جوان ار شیرزاد
&&
غیر مرد پیر سر لشکر مباد
\\
هم تو گفتستی و گفت تو گوا
&&
پیر باید پیر باید پیشوا
\\
یا رسول‌الله درین لشکر نگر
&&
هست چندین پیر و از وی پیشتر
\\
زین درخت آن برگ زردش را مبین
&&
سیبهای پختهٔ او را بچین
\\
برگهای زرد او خود کی تهیست
&&
این نشان پختگی و کاملیست
\\
برگ زرد ریش و آن موی سپید
&&
بهر عقل پخته می‌آرد نوید
\\
برگهای نو رسیدهٔ سبزفام
&&
شد نشان آنک آن میوه‌ست خام
\\
برگ بی‌برگی نشان عارفیست
&&
زردی زر سرخ رویی صارفیست
\\
آنک او گل عارضست ار نو خطست
&&
او به مکتب گاه مخبر نوخطست
\\
حرفهای خط او کژمژ بود
&&
مزمن عقلست اگر تن می‌دود
\\
پای پیر از سرعت ار چه باز ماند
&&
یافت عقل او دو پر بر اوج راند
\\
گر مثل خواهی به جعفر در نگر
&&
داد حق بر جای دست و پاش پر
\\
بگذر از زر کین سخت شد محتجب
&&
هم‌چو سیماب این دلم شد مضطرب
\\
ز اندرونم صدخموش خوش‌نفس
&&
دست بر لب می‌زند یعنی که بس
\\
خامشی بحرست و گفتن هم‌چو جو
&&
بحر می‌جوید ترا جو را مجو
\\
از اشارتهای دریا سر متاب
&&
ختم کن والله اعلم بالصواب
\\
هم‌چنین پیوسته کرد آن بی‌ادب
&&
پیش پیغامبر سخن زان سرد لب
\\
دست می‌دادش سخن او بی‌خبر
&&
که خبر هرزه بود پیش نظر
\\
این خبرها از نظر خود نایبست
&&
بهر حاضر نیست بهر غایبست
\\
هر که او اندر نظر موصول شد
&&
این خبرها پیش او معزول شد
\\
چونک با معشوق گشتی همنشین
&&
دفع کن دلالگان را بعد ازین
\\
هر که از طفلی گذشت و مرد شد
&&
نامه و دلاله بر وی سرد شد
\\
نامه خواند از پی تعلیم را
&&
حرف گوید از پی تفهیم را
\\
پیش بینایان خبر گفتن خطاست
&&
کان دلیل غفلت و نقصان ماست
\\
پیش بینا شد خموشی نفع تو
&&
بهر این آمد خطاب انصتوا
\\
گر بفرماید بگو بر گوی خوش
&&
لیک اندک گو دراز اندر مکش
\\
ور بفرماید که اندر کش دراز
&&
هم‌چنان شرمین بگو با امر ساز
\\
همچنین که من درین زیبا فسون
&&
با ضیاء الحق حسام‌الدین کنون
\\
چونک کوته می‌کنم من از رشد
&&
او به صد نوعم بگفتن می‌کشد
\\
ای حسام‌الدین ضیاء ذوالجلال
&&
چونک می‌بینی چه می‌جویی مقال
\\
این مگر باشد ز حب مشتهی
&&
اسقنی خمرا و قل لی انها
\\
بر دهان تست این دم جام او
&&
گوش می‌گوید که قسم گوش کو
\\
قسم تو گرمیست نک گرمی و مست
&&
گفت حرص من ازین افزون‌ترست
\\
\end{longtable}
\end{center}
