\begin{center}
\section*{بخش ۳ - حکایت آن واعظ کی هر آغاز تذکیر دعای ظالمان و سخت‌دلان و بی‌اعتقادان کردی}
\label{sec:sh003}
\addcontentsline{toc}{section}{\nameref{sec:sh003}}
\begin{longtable}{l p{0.5cm} r}
آن یکی واعظ چو بر تخت آمدی
&&
قاطعان راه را داعی شدی
\\
دست برمی‌داشت یا رب رحم ران
&&
بر بدان و مفسدان و طاغیان
\\
بر همه تسخرکنان اهل خیر
&&
برهمه کافردلان و اهل دیر
\\
می‌نکردی او دعا بر اصفیا
&&
می‌نکردی جز خبیثان را دعا
\\
مر ورا گفتند کین معهود نیست
&&
دعوت اهل ضلالت جود نیست
\\
گفت نیکویی ازینها دیده‌ام
&&
من دعاشان زین سبب بگزیده‌ام
\\
خبث و ظلم و جور چندان ساختند
&&
که مرا از شر به خیر انداختند
\\
هر گهی که رو به دنیا کردمی
&&
من ازیشان زخم و ضربت خوردمی
\\
کردمی از زخم آن جانب پناه
&&
باز آوردندمی گرگان به راه
\\
چون سبب‌ساز صلاح من شدند
&&
پس دعاشان بر منست ای هوشمند
\\
بنده می‌نالد به حق از درد و نیش
&&
صد شکایت می‌کند از رنج خویش
\\
حق همی گوید که آخر رنج و درد
&&
مر ترا لابه کنان و راست کرد
\\
این گله زان نعمتی کن کت زند
&&
از در ما دور و مطرودت کند
\\
در حقیقت هر عدو داروی تست
&&
کیمیا و نافع و دلجوی تست
\\
که ازو اندر گریزی در خلا
&&
استعانت جویی از لطف خدا
\\
در حقیقت دوستانت دشمن‌اند
&&
که ز حضرت دور و مشغولت کنند
\\
هست حیوانی که نامش اشغرست
&&
او به زخم چوب زفت و لمترست
\\
تا که چوبش می‌زنی به می‌شود
&&
او ز زخم چوب فربه می‌شود
\\
نفس مؤمن اشغری آمد یقین
&&
کو به زخم رنج زفتست و سمین
\\
زین سبب بر انبیا رنج و شکست
&&
از همه خلق جهان افزونترست
\\
تا ز جانها جانشان شد زفت‌تر
&&
که ندیدند آن بلا قوم دگر
\\
پوست از دارو بلاکش می‌شود
&&
چون ادیم طایفی خوش می‌شود
\\
ورنه تلخ و تیز مالیدی درو
&&
گنده گشتی ناخوش و ناپاک بو
\\
آدمی را پوست نامدبوغ دان
&&
از رطوبتها شده زشت و گران
\\
تلخ و تیز و مالش بسیار ده
&&
تا شود پاک و لطیف و با فره
\\
ور نمی‌توانی رضا ده ای عیار
&&
گر خدا رنجت دهد بی‌اختیار
\\
که بلای دوست تطهیر شماست
&&
علم او بالای تدبیر شماست
\\
چون صفا بیند بلا شیرین شود
&&
خوش شود دارو چو صحت‌بین شود
\\
برد بیند خویش را در عین مات
&&
پس بگوید اقتلونی یا ثقات
\\
این عوان در حق غیری سود شد
&&
لیک اندر حق خود مردود شد
\\
رحم ایمانی ازو ببریده شد
&&
کین شیطانی برو پیچیده شد
\\
کارگاه خشم گشت و کین‌وری
&&
کینه دان اصل ضلال و کافری
\\
\end{longtable}
\end{center}
