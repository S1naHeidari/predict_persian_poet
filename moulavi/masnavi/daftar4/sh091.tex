\begin{center}
\section*{بخش ۹۱ - بیان آنک هر حس مدرکی را از آدمی نیز مدرکاتی دیگرست کی از مدرکات آن  حس دگر بی‌خبرست چنانک هر پیشه‌ور استاد اعجمی کار آن استاد دگر پیشه‌ورست و بی‌خبری او از آنک وظیفهٔ او نیست دلیل نکند کی آن مدرکات نیست  اگر چه به حکم حال منکر بود آن را اما از منکری او اینجا جز بی‌خبری  نمی‌خواهیم درین مقام}
\label{sec:sh091}
\addcontentsline{toc}{section}{\nameref{sec:sh091}}
\begin{longtable}{l p{0.5cm} r}
چنبرهٔ دید جهان ادراک تست
&&
پردهٔ پاکان حس ناپاک تست
\\
مدتی حس را بشو ز آب عیان
&&
این چنین دان جامه‌شوی صوفیان
\\
چون شدی تو پاک پرده بر کند
&&
جان پاکان خویش بر تو می‌زند
\\
جمله عالم گر بود نور و صور
&&
چشم را باشد از آن خوبی خبر
\\
چشم بستی گوش می‌آری به پیش
&&
تا نمایی زلف و رخسارهٔ به تیش
\\
گوش گوید من به صورت نگروم
&&
صورت ار بانگی زند من بشنوم
\\
عالمم من لکی اندر فن خویش
&&
فن من جز حرف و صوتی نیست بیش
\\
هین بیا بینی ببین این خوب را
&&
نیست در خور بینی این مطلوب را
\\
گر بود مشک و گلابی بو برم
&&
فن من اینست و علم و مخبرم
\\
کی ببینم من رخ آن سیم‌ساق
&&
هین مکن تکلیف ما لیس یطاق
\\
باز حس کژ نبیند غیر کژ
&&
خواه کژ غژ پیش او یا راست غژ
\\
چشم احول از یکی دیدن یقین
&&
دانک معزولست ای خواجه معین
\\
تو که فرعونی همه مکری و زرق
&&
مر مرا از خود نمی‌دانی تو فرق
\\
منگر از خود در من ای کژباز تو
&&
تا یکی تو را نبینی تو دوتو
\\
بنگر اندر من ز من یک ساعتی
&&
تا ورای کون بینی ساحتی
\\
وا رهی از تنگی و از ننگ و نام
&&
عشق اندر عشق بینی والسلام
\\
پس بدانی چونک رستی از بدن
&&
گوش و بینی چشم می‌داند شدن
\\
راست گفتست آن شه شیرین‌زبان
&&
چشم گرد مو به موی عارفان
\\
چشم را چشمی نبود اول یقین
&&
در رحم بود او جنین گوشتین
\\
علت دیدن مدان پیه ای پسر
&&
ورنه خواب اندر ندیدی کس صور
\\
آن پری و دیو می‌بیند شبیه
&&
نیست اندر دیدگاه هر دو پیه
\\
نور را با پیه خود نسبت نبود
&&
نسبتش بخشید خلاق ودود
\\
آدمست از خاک کی ماند به خاک
&&
جنیست از نار بی‌هیچ اشتراک
\\
نیست مانندای آتش آن پری
&&
گر چه اصلش اوست چون می‌بنگری
\\
مرغ از بادست و کی ماند به باد
&&
نامناسب را خدا نسبت به داد
\\
نسبت این فرعها با اصلها
&&
هست بی‌چون ار چه دادش وصلها
\\
آدمی چون زادهٔ خاک هباست
&&
این پسر را با پدر نسبت کجاست
\\
نسبتی گر هست مخفی از خرد
&&
هست بی‌چون و خرد کی پی برد
\\
باد را بی چشم اگر بینش نداد
&&
فرق چون می‌کرد اندر قوم عاد
\\
چون همی دانست مؤمن از عدو
&&
چون همی دانست می را از کدو
\\
آتش نمرود را گر چشم نیست
&&
با خلیلش چون تجشم کردنیست
\\
گر نبودی نیل را آن نور و دید
&&
از چه قبطی را ز سبطی می‌گزید
\\
گرنه کوه و سنگ با دیدار شد
&&
پس چرا داود را او یار شد
\\
این زمین را گر نبودی چشم جان
&&
از چه قارون را فرو خورد آنچنان
\\
گر نبودی چشم دل حنانه را
&&
چون بدیدی هجر آن فرزانه را
\\
سنگ‌ریزه گر نبودی دیده‌ور
&&
چون گواهی دادی اندر مشت در
\\
ای خرد بر کش تو پر و بالها
&&
سوره بر خوان زلزلت زلزالها
\\
در قیامت این زمین بر نیک و بد
&&
کی ز نادیده گواهیها دهد
\\
که تحدث حالها و اخبارها
&&
تظهر الارض لنا اسرارها
\\
این فرستادن مرا پیش تو میر
&&
هست برهانی که بد مرسل خبیر
\\
کین چنین دارو چنین ناسور را
&&
هست درخور از پی میسور را
\\
واقعاتی دیده بودی پیش ازین
&&
که خدا خواهد مرا کردن گزین
\\
من عصا و نور بگرفته به دست
&&
شاخ گستاخ ترا خواهم شکست
\\
واقعات سهمگین از بهر این
&&
گونه گونه می‌نمودت رب دین
\\
در خور سر بد و طغیان تو
&&
تا بدانی کوست درخوردان تو
\\
تا بدانی کو حکیمست و خبیر
&&
مصلح امراض درمان‌ناپذیر
\\
تو به تاویلات می‌گشتی از آن
&&
کور و گر کین هست از خواب گران
\\
وآن طبیب و آن منجم در لمع
&&
دید تعبیرش بپوشید از طمع
\\
گفت دور از دولت و از شاهیت
&&
که درآید غصه در آگاهیت
\\
از غذای مختلف یا از طعام
&&
طبع شوریده همی‌بیند منام
\\
زانک دید او که نصیحت‌جو نه‌ای
&&
تند و خون‌خواری و مسکین‌خو نه‌ای
\\
پادشاهان خون کنند از مصلحت
&&
لیک رحمتشان فزونست از عنت
\\
شاه را باید که باشد خوی رب
&&
رحمت او سبق دارد بر غضب
\\
نه غضب غالب بود مانند دیو
&&
بی‌ضرورت خون کند از بهر ریو
\\
نه حلیمی مخنث‌وار نیز
&&
که شود زن روسپی زان و کنیز
\\
دیوخانه کرده بودی سینه را
&&
قبله‌ای سازیده بودی کینه را
\\
شاخ تیزت بس جگرها را که خست
&&
نک عصاام شاخ شوخت را شکست
\\
\end{longtable}
\end{center}
