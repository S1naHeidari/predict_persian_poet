\begin{center}
\section*{بخش ۴۰ - خبر یافتن جد مصطفی عبدالمطلب  از گم کردن حلیمه محمد  را علیه‌السلام و طالب شدن او گرد شهر و نالیدن او بر در کعبه و از حق درخواستن و یافتن او محمد را علیه‌السلام}
\label{sec:sh040}
\addcontentsline{toc}{section}{\nameref{sec:sh040}}
\begin{longtable}{l p{0.5cm} r}
چون خبر یابید جد مصطفی
&&
از حلیمه وز فغانش بر ملا
\\
وز چنان بانگ بلند و نعره‌ها
&&
که بمیلی می‌رسید از وی صدا
\\
زود عبدالمطلب دانست چیست
&&
دست بر سینه همی‌زد می‌گریست
\\
آمد از غم بر در کعبه بسوز
&&
کای خبیر از سر شب وز راز روز
\\
خویشتن را من نمی‌بینم فنی
&&
تا بود هم‌راز تو هم‌چون منی
\\
خویشتن را من نمی‌بینم هنر
&&
تا شوم مقبول این مسعود در
\\
یا سر و سجدهٔ مرا قدری بود
&&
یا باشکم دولتی خندان شود
\\
لیک در سیمای آن در یتیم
&&
دیده‌ام آثار لطفت ای کریم
\\
که نمی‌ماند به ما گرچه ز ماست
&&
ما همه مسیم و احمد کیمیاست
\\
آن عجایبها که من دیدم برو
&&
من ندیدم بر ولی و بر عدو
\\
آنک فضل تو درین طفلیش داد
&&
کس نشان ندهد به صد ساله جهاد
\\
چون یقین دیدم عنایتهای تو
&&
بر وی او دریست از دریای تو
\\
من هم او را می شفیع آرم به تو
&&
حال او ای حال‌دان با من بگو
\\
از درون کعبه آمد بانگ زود
&&
که هم‌اکنون رخ به تو خواهد نمود
\\
با دو صد اقبال او محظوظ ماست
&&
با دو صد طلب ملک محفوظ ماست
\\
ظاهرش را شهرهٔ گیهان کنیم
&&
باطنش را از همه پنهان کنیم
\\
زر کان بود آب و گل ما زرگریم
&&
که گهش خلخال و گه خاتم بریم
\\
گه حمایلهای شمشیرش کنیم
&&
گاه بند گردن شیرش کنیم
\\
گه ترنج تخت بر سازیم ازو
&&
گاه تاج فرقهای ملک‌جو
\\
عشقها داریم با این خاک ما
&&
زانک افتادست در قعدهٔ رضا
\\
گه چنین شاهی ازو پیدا کنیم
&&
گه هم او را پیش شه شیدا کنیم
\\
صد هزاران عاشق و معشوق ازو
&&
در فغان و در نفیر و جست و جو
\\
کار ما اینست بر کوری آن
&&
که به کار ما ندارد میل جان
\\
این فضیلت خاک را زان رو دهیم
&&
که نواله پیش بی‌برگان نهیم
\\
زانک دارد خاک شکل اغبری
&&
وز درون دارد صفات انوری
\\
ظاهرش با باطنش گشته به جنگ
&&
باطنش چون گوهر و ظاهر چو سنگ
\\
ظاهرش گوید که ما اینیم و بس
&&
باطنش گوید نکو بین پیش و پس
\\
ظاهرش منکر که باطن هیچ نیست
&&
باطنش گوید که بنماییم بیست
\\
ظاهرش با باطنش در چالش‌اند
&&
لاجرم زین صبر نصرت می‌کشند
\\
زین ترش‌رو خاک صورتها کنیم
&&
خندهٔ پنهانش را پیدا کنیم
\\
زانک ظاهر خاک اندوه و بکاست
&&
در درونش صد هزاران خنده‌هاست
\\
کاشف السریم و کار ما همین
&&
کین نهانها را بر آریم از کمین
\\
گرچه دزد از منکری تن می‌زند
&&
شحنه آن از عصر پیدا می‌کند
\\
فضلها دزدیده‌اند این خاکها
&&
تا مقر آریمشان از ابتلا
\\
بس عجب فرزند کو را بوده است
&&
لیک احمد بر همه افزوده است
\\
شد زمین و آسمان خندان و شاد
&&
کین چنین شاهی ز ما دو جفت زاد
\\
می‌شکافد آسمان از شادیش
&&
خاک چون سوسن شده ز آزادیش
\\
ظاهرت با باطنت ای خاک خوش
&&
چونک در جنگ‌اند و اندر کش‌مکش
\\
هر که با خود بهر حق باشد به جنگ
&&
تا شود معنیش خصم بو و رنگ
\\
ظلمتش با نور او شد در قتال
&&
آفتاب جانش را نبود زوال
\\
هر که کوشد بهر ما در امتحان
&&
پشت زیر پایش آرد آسمان
\\
ظاهرت از تیرگی افغان کنان
&&
باطن تو گلستان در گلستان
\\
قاصد او چون صوفیان روترش
&&
تا نیامیزند با هر نورکش
\\
عارفان روترش چون خارپشت
&&
عیش پنهان کرده در خار درشت
\\
باغ پنهان گرد باغ آن خار فاش
&&
کای عدوی دزد زین در دور باش
\\
خارپشتا خار حارس کرده‌ای
&&
سر چو صوفی در گریبان برده‌ای
\\
تا کسی دوچار دانگ عیش تو
&&
کم شود زین گلرخان خارخو
\\
طفل تو گرچه که کودک‌خو بدست
&&
هر دو عالم خود طفیل او بدست
\\
ما جهانی را بدو زنده کنیم
&&
چرخ را در خدمتش بنده کنیم
\\
گفت عبدالمطلب کین دم کجاست
&&
ای علیم السر نشان ده راه راست
\\
\end{longtable}
\end{center}
