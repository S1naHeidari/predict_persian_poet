\begin{center}
\section*{بخش ۳۲ - تهدید فرستادن سلیمان علیه‌السلام پیش بلقیس کی اصرار میندیش بر شرک و تاخیر مکن}
\label{sec:sh032}
\addcontentsline{toc}{section}{\nameref{sec:sh032}}
\begin{longtable}{l p{0.5cm} r}
هین بیا بلقیس ورنه بد شود
&&
لشکرت خصمت شود مرتد شود
\\
پرده‌دار تو درت را بر کند
&&
جان تو با تو به جان خصمی کند
\\
جمله ذرات زمین و آسمان
&&
لشکر حق‌اند گاه امتحان
\\
باد را دیدی که با عادان چه کرد
&&
آب را دیدی که در طوفان چه کرد
\\
آنچ بر فرعون زد آن بحر کین
&&
وآنچ با قارون نمودست این زمین
\\
وآنچ آن بابیل با آن پیل کرد
&&
وآنچ پشه کلهٔ نمرود خورد
\\
وآنک سنگ انداخت داودی بدست
&&
گشت شصد پاره و لشکر شکست
\\
سنگ می‌بارید بر اعدای لوط
&&
تا که در آب سیه خوردند غوط
\\
گر بگویم از جمادات جهان
&&
عاقلانه یاری پیغامبران
\\
مثنوی چندان شود که چل شتر
&&
گر کشد عاجز شود از بار پر
\\
دست بر کافر گواهی می‌دهد
&&
لشکر حق می‌شود سر می‌نهد
\\
ای نموده ضد حق در فعل درس
&&
در میان لشکر اویی بترس
\\
جزو جزوت لشکر از در وفاق
&&
مر ترا اکنون مطیع‌اند از نفاق
\\
گر بگوید چشم را کو را فشار
&&
درد چشم از تو بر آرد صد دمار
\\
ور به دندان گوید او بنما وبال
&&
پس ببینی تو ز دندان گوشمال
\\
باز کن طب را بخوان باب العلل
&&
تا ببینی لشکر تن را عمل
\\
چونک جان جان هر چیزی ویست
&&
دشمنی با جان جان آسان کیست
\\
خود رها کن لشکر دیو و پری
&&
کز میان جان کنندم صفدری
\\
ملک را بگذار بلقیس از نخست
&&
چون مرا یابی همه ملک آن تست
\\
خود بدانی چون بر من آمدی
&&
که تو بی من نقش گرمابه بدی
\\
نقش اگر خود نقش سلطان یا غنیست
&&
صورتست از جان خود بی چاشنیست
\\
زینت او از برای دیگران
&&
باز کرده بیهده چشم و دهان
\\
ای تو در بیگار خود را باخته
&&
دیگران را تو ز خود نشناخته
\\
تو به هر صورت که آیی بیستی
&&
که منم این والله آن تو نیستی
\\
یک زمان تنها بمانی تو ز خلق
&&
در غم و اندیشه مانی تا به حلق
\\
این تو کی باشی که تو آن اوحدی
&&
که خوش و زیبا و سرمست خودی
\\
مرغ خویشی صید خویشی دام خویش
&&
صدر خویشی فرش خویشی بام خویش
\\
جوهر آن باشد که قایم با خودست
&&
آن عرض باشد که فرع او شدست
\\
گر تو آدم‌زاده‌ای چون او نشین
&&
جمله ذریات را در خود ببین
\\
چیست اندر خم که اندر نهر نیست
&&
چیست اندر خانه کاندر شهر نیست
\\
این جهان خمست و دل چون جوی آب
&&
این جهان حجره‌ست و دل شهر عجاب
\\
\end{longtable}
\end{center}
