\begin{center}
\section*{بخش ۸۱ - بیان رسول علیه السلام سبب تفضیل و اختیار کردن او آن هذیلی را به امیری و سرلشکری بر پیران و کاردیدگان}
\label{sec:sh081}
\addcontentsline{toc}{section}{\nameref{sec:sh081}}
\begin{longtable}{l p{0.5cm} r}
حکم اغلب راست چون غالب بدند
&&
تیغ را از دست ره‌زن بستدند
\\
گفت پیغامبر کای ظاهرنگر
&&
تو مبین او را جوان و بی‌هنر
\\
ای بسا ریش سیاه و مرد پیر
&&
ای بسا ریش سپید و دل چو قیر
\\
عقل او را آزمودم بارها
&&
کرد پیری آن جوان در کارها
\\
پیر پیر عقل باشد ای پسر
&&
نه سپیدی موی اندر ریش و سر
\\
از بلیس او پیرتر خود کی بود
&&
چونک عقلش نیست او لاشی بود
\\
طفل گیرش چون بود عیسی نفس
&&
پاک باشد از غرور و از هوس
\\
آن سپیدی مو دلیل پختگیست
&&
پیش چشم بسته کش کوته‌تگیست
\\
آن مقلد چون نداند جز دلیل
&&
در علامت جوید او دایم سبیل
\\
بهر او گفتیم که تدبیر را
&&
چونک خواهی کرد بگزین پیر را
\\
آنک او از پردهٔ تقلید جست
&&
او به نور حق ببیند آنچ هست
\\
نور پاکش بی‌دلیل و بی‌بیان
&&
پوست بشکافد در آید در میان
\\
پیش ظاهربین چه قلب و چه سره
&&
او چه داند چیست اندر قوصره
\\
ای بسا زر سیه کرده بدود
&&
تا رهد از دست هر دزدی حسود
\\
ای بسا مس زر اندوده به زر
&&
تا فروشد آن به عقل مختصر
\\
ما که باطن‌بین جملهٔ کشوریم
&&
دل ببینیم و به ظاهر ننگریم
\\
قاضیانی که به ظاهر می‌تنند
&&
حکم بر اشکال ظاهر می‌کنند
\\
چون شهادت گفت و ایمانی نمود
&&
حکم او مؤمن کنند این قوم زود
\\
بس منافق کاندرین ظاهر گریخت
&&
خون صد مؤمن به پنهانی بریخت
\\
جهد کن تا پیر عقل و دین شوی
&&
تا چو عقل کل تو باطن‌بین شوی
\\
از عدم چون عقل زیبا رو گشاد
&&
خلعتش داد و هزارش نام داد
\\
کمترین زان نامهای خوش‌نفس
&&
این که نبود هیچ او محتاج کس
\\
گر به صورت وا نماید عقل رو
&&
تیره باشد روز پیش نور او
\\
ور مثال احمقی پیدا شود
&&
ظلمت شب پیش او روشن بود
\\
کو ز شب مظلم‌تر و تاری‌ترست
&&
لیک خفاش شقی ظلمت‌خرست
\\
اندک اندک خوی کن با نور روز
&&
ورنه خفاشی بمانی بیفروز
\\
عاشق هر جا شکال و مشکلیست
&&
دشمن هر جا چراغ مقبلیست
\\
ظلمت اشکال زان جوید دلش
&&
تا که افزون‌تر نماید حاصلش
\\
تا ترا مشغول آن مشکل کند
&&
وز نهاد زشت خود غافل کند
\\
\end{longtable}
\end{center}
