\begin{center}
\section*{بخش ۱۱۶ - بیان آنک روح حیوانی و عقل جز وی و وهم و خیال بر مثال دوغند و روح کی باقیست درین دوغ هم‌چون روغن پنهانست}
\label{sec:sh116}
\addcontentsline{toc}{section}{\nameref{sec:sh116}}
\begin{longtable}{l p{0.5cm} r}
جوهر صدقت خفی شد در دروغ
&&
هم‌چو طعم روغن اندر طعم دوغ
\\
آن دروغت این تن فانی بود
&&
راستت آن جان ربانی بود
\\
سالها این دوغ تن پیدا و فاش
&&
روغن جان اندرو فانی و لاش
\\
تا فرستد حق رسولی بنده‌ای
&&
دوغ را در خمره جنباننده‌ای
\\
تا بجنباند به هنجار و به فن
&&
تا بدانم من که پنهان بود من
\\
یا کلام بنده‌ای کان جزو اوست
&&
در رود در گوش او کو وحی جوست
\\
اذن مؤمن وحی ما را واعیست
&&
آنچنان گوشی قرین داعیست
\\
هم‌چنانک گوش طفل از گفت مام
&&
پر شود ناطق شود او درکلام
\\
ور نباشد طفل را گوش رشد
&&
گفت مادر نشنود گنگی شود
\\
دایما هر کر اصلی گنگ بود
&&
ناطق آنکس شد که از مادر شنود
\\
دانک گوش کر و گنگ از آفتیست
&&
که پذیرای دم و تعلیم نیست
\\
آنک بی‌تعلیم بد ناطق خداست
&&
که صفات او ز علتها جداست
\\
یا چو آدم کرده تلقینش خدا
&&
بی‌حجاب مادر و دایه و ازا
\\
یا مسیحی که به تعلیم ودود
&&
در ولادت ناطق آمد در وجود
\\
از برای دفع تهمت در ولاد
&&
که نزادست از زنا و از فساد
\\
جنبشی بایست اندر اجتهاد
&&
تا که دوغ آن روغن از دل باز داد
\\
روغن اندر دوغ باشد چون عدم
&&
دوغ در هستی برآورده علم
\\
آنک هستت می‌نماید هست پوست
&&
وآنک فانی می‌نماید اصل اوست
\\
دوغ روغن ناگرفتست و کهن
&&
تا بنگزینی بنه خرجش مکن
\\
هین بگردانش به دانش دست دست
&&
تا نماید آنچ پنهان کرده است
\\
زآنک این فانی دلیل باقیست
&&
لابهٔ مستان دلیل ساقیست
\\
\end{longtable}
\end{center}
