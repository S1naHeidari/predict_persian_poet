\begin{center}
\section*{بخش ۶۷ - دریافتن طبیبان الهی امراض دین و دل را در سیمای مرید و بیگانه و لحن گفتار او و رنگ چشم او و بی این همه نیز از راه دل کی انهم جواسیس القلوب فجالسوهم بالصدق}
\label{sec:sh067}
\addcontentsline{toc}{section}{\nameref{sec:sh067}}
\begin{longtable}{l p{0.5cm} r}
این طبیبان بدن دانش‌ورند
&&
بر سقام تو ز تو واقف‌ترند
\\
تا ز قاروره همی‌بینند حال
&&
که ندانی تو از آن رو اعتلال
\\
هم ز نبض و هم ز رنگ و هم ز دم
&&
بو برند از تو بهر گونه سقم
\\
پس طبیبان الهی در جهان
&&
چون ندانند از تو بی‌گفت دهان
\\
هم ز نبضت هم ز چشمت هم ز رنگ
&&
صد سقم بینند در تو بی‌درنگ
\\
این طبیبان نوآموزند خود
&&
که بدین آیاتشان حاجت بود
\\
کاملان از دور نامت بشنوند
&&
تا به قعر باد و بودت در دوند
\\
بلک پیش از زادن تو سالها
&&
دیده باشندت ترا با حالها
\\
\end{longtable}
\end{center}
