\begin{center}
\section*{بخش ۱۳۱ - لابه کردن قبطی سبطی را کی یک سبو بنیت خویش از نیل پر کن و بر لب من نه تا بخورم به حق دوستی و برادری کی سبو کی شما سبطیان بهر خود پر می‌کنید از نیل آب صاف است و سبوکی ما قبطیان پر می‌کنیم خون صاف است}
\label{sec:sh131}
\addcontentsline{toc}{section}{\nameref{sec:sh131}}
\begin{longtable}{l p{0.5cm} r}
من شنیدم که در آمد قبطیی
&&
از عطش اندر وثاق سبطیی
\\
گفت هستم یار و خویشاوند تو
&&
گشته‌ام امروز حاجتمند تو
\\
زانک موسی جادوی کرد و فسون
&&
تا که آب نیل ما را کرد خون
\\
سبطیان زو آب صافی می‌خورند
&&
پیش قبطی خون شد آب از چشم‌بند
\\
قبط اینک می‌مرند از تشنگی
&&
از پی ادبار خود یا بدرگی
\\
بهر خود یک طاس را پر آب کن
&&
تا خورد از آبت این یار کهن
\\
چون برای خود کنی آن طاس پر
&&
خون نباشد آب باشد پاک و حر
\\
من طفیل تو بنوشم آب هم
&&
که طفیلی در تبع به جهد ز غم
\\
گفت ای جان و جهان خدمت کنم
&&
پاس دارم ای دو چشم روشنم
\\
بر مراد تو روم شادی کنم
&&
بندهٔ تو باشم آزادی کنم
\\
طاس را از نیل او پر آب کرد
&&
بر دهان بنهاد و نیمی را بخورد
\\
طاس را کژ کرد سوی آب‌خواه
&&
که بخور تو هم شد آن خون سیاه
\\
باز ازین سو کرد کژ خون آب شد
&&
قبطی اندر خشم و اندر تاب شد
\\
ساعتی بنشست تا خشمش برفت
&&
بعد از آن گفتش کای صمصام زفت
\\
ای برادر این گره را چاره چیست
&&
گفت این را او خورد کو متقیست
\\
متقی آنست کو بیزار شد
&&
از ره فرعون و موسی‌وار شد
\\
قوم موسی شو بخور این آب را
&&
صلح کن با مه ببین مهتاب را
\\
صدهزاران ظلمتست از خشم تو
&&
بر عبادالله اندر چشم تو
\\
خشم بنشان چشم بگشا شاد شو
&&
عبرت از یاران بگیر استاد شو
\\
کی طفیل من شوی در اغتراف
&&
چون ترا کفریست هم‌چون کوه قاف
\\
کوه در سوراخ سوزن کی رود
&&
جز مگر که آن رشتهٔ یکتا شود
\\
کوه را که کن به استغفار و خوش
&&
جام مغفوران بگیر و خوش بکش
\\
تو بدین تزویر چون نوشی از آن
&&
چون حرامش کرد حق بر کافران
\\
خالق تزویر تزویر ترا
&&
کی خرد ای مفتری مفترا
\\
آل موسی شو که حیلت سود نیست
&&
حیله‌ات باد تهی پیمودنیست
\\
زهره دارد آب کز امر صمد
&&
گردد او با کافران آبی کند
\\
یا تو پنداری که تو نان می‌خوری
&&
زهر مار و کاهش جان می‌خوری
\\
نان کجا اصلاح آن جانی کند
&&
کو دل از فرمان جانان بر کند
\\
یا تو پنداری که حرف مثنوی
&&
چون بخوانی رایگانش بشنوی
\\
یا کلام حکمت و سر نهان
&&
اندر آید زغبه در گوش و دهان
\\
اندر آید لیک چون افسانه‌ها
&&
پوست بنماید نه مغز دانه‌ها
\\
در سر و رو در کشیده چادری
&&
رو نهان کرده ز چشمت دلبری
\\
شاه‌نامه یا کلیله پیش تو
&&
هم‌چنان باشد که قرآن از عتو
\\
فرق آنگه باشد از حق و مجاز
&&
که کند کحل عنایت چشم باز
\\
ورنه پشک و مشک پیش اخشمی
&&
هر دو یکسانست چون نبود شمی
\\
خویشتن مشغول کردن از ملال
&&
باشدش قصد از کلام ذوالجلال
\\
کاتش وسواس را و غصه را
&&
زان سخن بنشاند و سازد دوا
\\
بهر این مقدار آتش شاندن
&&
آب پاک و بول یکسان شدن به فن
\\
آتش وسواس را این بول و آب
&&
هر دو بنشانند هم‌چون وقت خواب
\\
لیک گر واقف شوی زین آب پاک
&&
که کلام ایزدست و روحناک
\\
نیست گردد وسوسه کلی ز جان
&&
دل بیابد ره به سوی گلستان
\\
زانک در باغی و در جویی پرد
&&
هر که از سر صحف بویی برد
\\
یا تو پنداری که روی اولیا
&&
آنچنان که هست می‌بینیم ما
\\
در تعجب مانده پیغامبر از آن
&&
چون نمی‌بینند رویم مؤمنان
\\
چون نمی‌بینند نور روم خلق
&&
که سبق بردست بر خورشید شرق
\\
ور همی‌بینند این حیرت چراست
&&
تا که وحی آمد که آن رو در خفاست
\\
سوی تو ماهست و سوی خلق ابر
&&
تا نبیند رایگان روی تو گبر
\\
سوی تو دانه‌ست و سوی خلق دام
&&
تا ننوشد زین شراب خاص عام
\\
گفت یزدان که تراهم ینظرون
&&
نقش حمامند هم لا یبصرون
\\
می‌نماید صورت ای صورت‌پرست
&&
که آن دو چشم مردهٔ او ناظرست
\\
پیش چشم نقش می‌آری ادب
&&
کو چرا پاسم نمی‌دارد عجب
\\
از چه پس بی‌پاسخست این نقش نیک
&&
که نمی‌گوید سلامم را علیک
\\
می‌نجنباند سر و سبلت ز جود
&&
پاس آنک کردمش من صد سجود
\\
حق اگر چه سر نجنباند برون
&&
پاس آن ذوقی دهد در اندرون
\\
که دو صد جنبیدن سر ارزد آن
&&
سر چنین جنباند آخر عقل و جان
\\
عقل را خدمت کنی در اجتهاد
&&
پاس عقل آنست که افزاید رشاد
\\
حق نجنباند به ظاهر سر ترا
&&
لیک سازد بر سران سرور ترا
\\
مر ترا چیزی دهد یزدان نهان
&&
که سجود تو کنند اهل جهان
\\
آنچنان که داد سنگی را هنر
&&
تا عزیز خلق شد یعنی که زر
\\
قطرهٔ آبی بیابد لطف حق
&&
گوهری گردد برد از زر سبق
\\
جسم خاکست و چو حق تابیش داد
&&
در جهان‌گیری چو مه شد اوستاد
\\
هین طلسمست این و نقش مرده است
&&
احمقان را چشمش از ره برده است
\\
می‌نماید او که چشمی می‌زند
&&
ابلهان سازیده‌اند او را سند
\\
\end{longtable}
\end{center}
