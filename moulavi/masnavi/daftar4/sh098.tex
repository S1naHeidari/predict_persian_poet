\begin{center}
\section*{بخش ۹۸ - تفسیر کنت کنزا مخفیا فاحببت ان اعرف}
\label{sec:sh098}
\addcontentsline{toc}{section}{\nameref{sec:sh098}}
\begin{longtable}{l p{0.5cm} r}
خانه بر کن کز عقیق این یمن
&&
صد هزاران خانه شاید ساختن
\\
گنج زیر خانه است و چاره نیست
&&
از خرابی خانه مندیش و مه‌ایست
\\
که هزاران خانه از یک نقد گنج
&&
توان عمارت کرد بی‌تکلیف و رنج
\\
عاقبت این خانه خود ویران شود
&&
گنج از زیرش یقین عریان شود
\\
لیک آن تو نباشد زانک روح
&&
مزد ویران کردنستش آن فتوح
\\
چون نکرد آن کار مزدش هست لا
&&
لییس للانسان الا ما سعی
\\
دست خایی بعد از آن تو کای دریغ
&&
این چنین ماهی بد اندر زیر میغ
\\
من نکردم آنچ گفتند از بهی
&&
گنج رفت و خانه و دستم تهی
\\
خانهٔ اجرت گرفتی و کری
&&
نیست ملک تو به بیعی یا شری
\\
این کری را مدت او تا اجل
&&
تا درین مدت کنی در وی عمل
\\
پاره‌دوزی می‌کنی اندر دکان
&&
زیر این دکان تو مدفون دو کان
\\
هست این دکان کرایی زود باش
&&
تیشه بستان و تکش را می‌تراش
\\
تا که تیشه ناگهان بر کان نهی
&&
از دکان و پاره‌دوزی وا رهی
\\
پاره‌دوزی چیست خورد آب و نان
&&
می‌زنی این پاره بر دلق گران
\\
هر زمان می‌درد این دلق تنت
&&
پاره بر وی می‌زنی زین خوردنت
\\
ای ز نسل پادشاه کامیار
&&
با خود آ زین پاره‌دوزی ننگ دار
\\
پاره‌ای بر کن ازین قعر دکان
&&
تا برآرد سر به پیش تو دو کان
\\
پیش از آن کین مهلت خانهٔ کری
&&
آخر آید تو نخورده زو بری
\\
پس ترا بیرون کند صاحب دکان
&&
وین دکان را بر کند از روی کان
\\
تو ز حسرت گاه بر سر می‌زنی
&&
گاه ریش خام خود بر می‌کنی
\\
کای دریغا آن من بود این دکان
&&
کور بودم بر نخوردم زین مکان
\\
ای دریغا بود ما را برد باد
&&
تا ابد یا حسرتا شد للعباد
\\
\end{longtable}
\end{center}
