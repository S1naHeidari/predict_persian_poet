\begin{center}
\section*{بخش ۱۳۷ - رفتن ذوالقرنین به کوه قاف و درخواست کردن کی ای کوه قاف از عظمت صفت حق ما را بگو و گفتن کوه قاف کی صفت عظمت او در گفت نیاید  کی پیش آنها ادراکها فدا شود و لابه کردن ذوالقرنین کی از صنایعش کی در خاطر داری و بر تو گفتن آن آسان‌تر بود بگوی}
\label{sec:sh137}
\addcontentsline{toc}{section}{\nameref{sec:sh137}}
\begin{longtable}{l p{0.5cm} r}
رفت ذوالقرنین سوی کوه قاف
&&
دید او را کز زمرد بود صاف
\\
گرد عالم حلقه گشته او محیط
&&
ماند حیران اندر آن خلق بسیط
\\
گفت تو کوهی دگرها چیستند
&&
که به پیش عظم تو بازیستند
\\
گفت رگهای من‌اند آن کوهها
&&
مثل من نبوند در حسن و بها
\\
من به هر شهری رگی دارم نهان
&&
بر عروقم بسته اطراف جهان
\\
حق چو خواهد زلزلهٔ شهری مرا
&&
گوید او من بر جهانم عرق را
\\
پس بجنبانم من آن رگ را بقهر
&&
که بدان رگ متصل گشتست شهر
\\
چون بگوید بس شود ساکن رگم
&&
ساکنم وز روی فعل اندر تگم
\\
هم‌چو مرهم ساکن و بس کارکن
&&
چون خرد ساکن وزو جنبان سخن
\\
نزد آنکس که نداند عقلش این
&&
زلزله هست از بخارات زمین
\\
\end{longtable}
\end{center}
