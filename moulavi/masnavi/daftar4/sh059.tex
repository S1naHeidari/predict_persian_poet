\begin{center}
\section*{بخش ۵۹ - نوشتن آن غلام قصهٔ شکایت نقصان اجری سوی پادشاه}
\label{sec:sh059}
\addcontentsline{toc}{section}{\nameref{sec:sh059}}
\begin{longtable}{l p{0.5cm} r}
قصه کوته کن برای آن غلام
&&
که سوی شه بر نوشتست او پیام
\\
قصه پر جنگ و پر هستی و کین
&&
می‌فرستد پیش شاه نازنین
\\
کالبد نامه‌ست اندر وی نگر
&&
هست لایق شاه را آنگه ببر
\\
گوشه‌ای رو نامه را بگشا بخوان
&&
بین که حرفش هست در خورد شهان
\\
گر نباشد درخور آن را پاره کن
&&
نامهٔ دیگر نویس و چاره کن
\\
لیک فتح نامهٔ تن زپ مدان
&&
ورنه هر کس سر دل دیدی عیان
\\
نامه بگشادن چه دشوارست و صعب
&&
کار مردانست نه طفلان کعب
\\
جمله بر فهرست قانع گشته‌ایم
&&
زانک در حرص و هوا آغشته‌ایم
\\
باشد آن فهرست دامی عامه را
&&
تا چنان دانند متن نامه را
\\
باز کن سرنامه را گردن متاب
&&
زین سخن والله اعلم بالصواب
\\
هست آن عنوان چو اقرار زبان
&&
متن نامهٔ سینه را کن امتحان
\\
که موافق هست با اقرار تو
&&
تا منافق‌وار نبود کار تو
\\
چون جوالی بس گرانی می‌بری
&&
زان نباید کم که در وی بنگری
\\
که چه داری در جوال از تلخ و خوش
&&
گر همی ارزد کشیدن را بکش
\\
ورنه خالی کن جوالت را ز سنگ
&&
باز خر خود را ازین بیگار و ننگ
\\
در جوال آن کن که می‌باید کشید
&&
سوی سلطانان و شاهان رشید
\\
\end{longtable}
\end{center}
