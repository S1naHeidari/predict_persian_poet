\begin{center}
\section*{بخش ۱۲۷ - بیان آنک عقل جزوی تا بگور بیش نبیند در باقی مقلد اولیا  و انبیاست}
\label{sec:sh127}
\addcontentsline{toc}{section}{\nameref{sec:sh127}}
\begin{longtable}{l p{0.5cm} r}
پیش‌بینی این خرد تا گور بود
&&
وآن صاحب دل به نفخ صور بود
\\
این خرد از گور و خاکی نگذرد
&&
وین قدم عرصهٔ عجایب نسپرد
\\
زین قدم وین عقل رو بیزار شو
&&
چشم غیبی جوی و برخوردار شو
\\
هم‌چو موسی نور کی یابد ز جیب
&&
سخرهٔ استاد و شاگردان کتاب
\\
زین نظر وین عقل ناید جز دوار
&&
پس نظر بگذار و بگزین انتظار
\\
از سخن‌گویی مجویید ارتفاع
&&
منتظر را به ز گفتن استماع
\\
منصب تعلیم نوع شهوتست
&&
هر خیال شهوتی در ره بتست
\\
گر بفضلش پی ببردی هر فضول
&&
کی فرستادی خدا چندین رسول
\\
عقل جزوی هم‌چو برقست و درخش
&&
در درخشی کی توان شد سوی وخش
\\
نیست نور برق بهر رهبری
&&
بلک امریست ابر را که می‌گری
\\
برق عقل ما برای گریه است
&&
تا بگرید نیستی در شوق هست
\\
عقل کودک گفت بر کتاب تن
&&
لیک نتواند به خود آموختن
\\
عقل رنجور آردش سوی طبیب
&&
لیک نبود در دوا عقلش مصیب
\\
نک شیاطین سوی گردون می‌شدند
&&
گوش بر اسرار بالا می‌زدند
\\
می‌ربودند اندکی زان رازها
&&
تا شهب می‌راندشان زود از سما
\\
که روید آنجا رسولی آمدست
&&
هر چه می‌خواهید زو آید به دست
\\
گر همی‌جویید در بی‌بها
&&
ادخلوا الابیات من ابوابها
\\
می‌زن آن حلقهٔ در و بر باب بیست
&&
از سوی بام فلکتان راه نیست
\\
نیست حاجتتان بدین راه دراز
&&
خاکیی را داده‌ایم اسرار راز
\\
پیش او آیید اگر خاین نیید
&&
نیشکر گردید ازو گرچه نیید
\\
سبزه رویاند ز خاکت آن دلیل
&&
نیست کم از سم اسپ جبرئیل
\\
سبزه گردی تازه گردی در نوی
&&
گر توخاک اسپ جبریلی شوی
\\
سبزهٔ جان‌بخش که آن را سامری
&&
کرد در گوساله تا شد گوهری
\\
جان گرفت و بانگ زد زان سبزه او
&&
آنچنان بانگی که شد فتنهٔ عدو
\\
گر امین آیید سوی اهل راز
&&
وا رهید از سر کله مانند باز
\\
سر کلاه چشم‌بند گوش‌بند
&&
که ازو بازست مسکین و نژند
\\
زان کله مر چشم بازان را سدست
&&
که همه میلش سوی جنس خودست
\\
چون برید از جنس با شه گشت یار
&&
بر گشاید چشم او را بازدار
\\
راند دیوان را حق از مرصاد خویش
&&
عقل جزوی را ز استبداد خویش
\\
که سری کم کن نه‌ای تو مستبد
&&
بلک شاگرد دلی و مستعد
\\
رو بر دل رو که تو جزو دلی
&&
هین که بندهٔ پادشاه عادلی
\\
بندگی او به از سلطانیست
&&
که انا خیر دم شیطانیست
\\
فرق بین و برگزین تو ای حبیس
&&
بندگی آدم از کبر بلیس
\\
گفت آنک هست خورشید ره او
&&
حرف طوبی هر که ذلت نفسه
\\
سایهٔ طوبی ببین وخوش بخسپ
&&
سر بنه در سایه بی‌سرکش بخسپ
\\
ظل ذلت نفسه خوش مضجعیست
&&
مستعد آن صفا و مهجعیست
\\
گر ازین سایه روی سوی منی
&&
زود طاغی گردی و ره گم کنی
\\
\end{longtable}
\end{center}
