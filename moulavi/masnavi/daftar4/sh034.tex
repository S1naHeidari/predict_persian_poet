\begin{center}
\section*{بخش ۳۴ - باقی قصهٔ ابراهیم ادهم قدس‌الله سره}
\label{sec:sh034}
\addcontentsline{toc}{section}{\nameref{sec:sh034}}
\begin{longtable}{l p{0.5cm} r}
بر سر تختی شنید آن نیک‌نام
&&
طقطقی و های و هویی شب ز بام
\\
گامهای تند بر بام سرا
&&
گفت با خود این چنین زهره کرا
\\
بانگ زد بر روزن قصر او که کیست
&&
این نباشد آدمی مانا پریست
\\
سر فرو کردند قومی بوالعجب
&&
ما همی گردیم شب بهر طلب
\\
هین چه می‌جویید گفتند اشتران
&&
گفت اشتر بام بر کی جست هان
\\
پس بگفتندش که تو بر تخت جاه
&&
چون همی جویی ملاقات اله
\\
خود همان بد دیگر او را کس ندید
&&
چون پری از آدمی شد ناپدید
\\
معنی‌اش پنهان و او در پیش خلق
&&
خلق کی بینند غیر ریش و دلق
\\
چون ز چشم خویش و خلقان دور شد
&&
هم‌چو عنقا در جهان مشهور شد
\\
جان هر مرغی که آمد سوی قاف
&&
جملهٔ عالم ازو لافند لاف
\\
چون رسید اندر سبا این نور شرق
&&
غلغلی افتاد در بلقیس و خلق
\\
روحهای مرده جمله پر زدند
&&
مردگان از گور تن سر بر زدند
\\
یک دگر را مژده می‌دادند هان
&&
نک ندایی می‌رسد از آسمان
\\
زان ندا دینها همی‌گردند گبز
&&
شاخ و برگ دل همی گردند سبز
\\
از سلیمان آن نفس چون نفخ صور
&&
مردگان را وا رهانید از قبور
\\
مر ترا بادا سعادت بعد ازین
&&
این گذشت الله اعلم بالیقین
\\
\end{longtable}
\end{center}
