\begin{center}
\section*{بخش ۱۰۹ - در بیان آنک شناسای قدرت حق نپرسد کی بهشت و دوزخ کجاست}
\label{sec:sh109}
\addcontentsline{toc}{section}{\nameref{sec:sh109}}
\begin{longtable}{l p{0.5cm} r}
هر کجا خدا دوزخ کند
&&
اوج را بر مرغ دام و فخ کند
\\
هم ز دندانت برآید دردها
&&
تا بگویی دوزخست و اژدها
\\
یا کند آب دهانت را عسل
&&
که بگویی که بهشتست و حلل
\\
از بن دندان برویاند شکر
&&
تا بدانی قوت حکم قدر
\\
پس به دندان بی‌گناهان را مگز
&&
فکر کن از ضربت نامحترز
\\
نیل را بر قبطیان حق خون کند
&&
سبطیان را از بلا محصون کند
\\
تا بدانی پیش حق تمییز هست
&&
در میان هوشیار راه و مست
\\
نیل تمییز از خدا آموختست
&&
که گشاد آن را و این را سخت بست
\\
لطف او عاقل کند مر نیل را
&&
قهر او ابله کند قابیل را
\\
در جمادات از کرم عقل آفرید
&&
عقل از عاقل به قهر خود برید
\\
در جماد از لطف عقلی شد پدید
&&
وز نکال از عاقلان دانش رمید
\\
عقل چون باران به امر آنجا بریخت
&&
عقل این سو خشم حق دید و گریخت
\\
ابر و خورشید و مه و نجم بلند
&&
جمله بر ترتیب آیند و روند
\\
هر یکی ناید مگر در وقت خویش
&&
که نه پس ماند ز هنگام و نه پیش
\\
چون نکردی فهم این را ز انبیا
&&
دانش آوردند در سنگ و عصا
\\
تا جمادات دگر را بی لباس
&&
چون عصا و سنگ داری از قیاس
\\
طاعت سنگ و عصا ظاهر شود
&&
وز جمادات دگر مخبر شود
\\
که ز یزدان آگهیم و طایعیم
&&
ما همه نی اتفاقی ضایعیم
\\
هم‌چو آب نیل دانی وقت غرق
&&
کو میان هر دو امت کرد فرق
\\
چون زمین دانیش دانا وقت خسف
&&
در حق قارون که قهرش کرد و نسف
\\
چون قمر که امر بشنید و شتافت
&&
پس دو نیمه گشت بر چرخ و شکافت
\\
چون درخت و سنگ کاندر هر مقام
&&
مصطفی را کرده ظاهرالسلام
\\
\end{longtable}
\end{center}
