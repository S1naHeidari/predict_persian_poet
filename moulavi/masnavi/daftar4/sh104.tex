\begin{center}
\section*{بخش ۱۰۴ - قصهٔ آن زن کی طفل او بر سر ناودان غیژید و خطر افتادن بود و از علی کرم‌الله وجهه چاره جست}
\label{sec:sh104}
\addcontentsline{toc}{section}{\nameref{sec:sh104}}
\begin{longtable}{l p{0.5cm} r}
یک زنی آمد به پیش مرتضی
&&
گفت شد بر ناودان طفلی مرا
\\
گرش می‌خوانم نمی‌آید به دست
&&
ور هلم ترسم که افتد او به پست
\\
نیست عاقل تا که دریابد چون ما
&&
گر بگویم کز خطر سوی من آ
\\
هم اشارت را نمی‌داند به دست
&&
ور بداند نشنود این هم بدست
\\
بس نمودم شیر و پستان را بدو
&&
او همی گرداند از من چشم و رو
\\
از برای حق شمایید ای مهان
&&
دستگیر این جهان و آن جهان
\\
زود درمان کن که می‌لرزد دلم
&&
که بدرد از میوهٔ دل بسکلم
\\
گفت طفلی را بر آور هم به بام
&&
تا ببیند جنس خود را آن غلام
\\
سوی جنس آید سبک زان ناودان
&&
جنس بر جنس است عاشق جاودان
\\
زن چنان کرد و چو دید آن طفل او
&&
جنس خود خوش خوش بدو آورد رو
\\
سوی بام آمد ز متن ناودان
&&
جاذب هر جنس را هم جنس دان
\\
غژغژان آمد به سوی طفل طفل
&&
وا رهید او از فتادن سوی سفل
\\
زان بود جنس بشر پیغامبران
&&
تا بجنسیت رهند از ناودان
\\
پس بشر فرمود خود را مثلکم
&&
تا به جنس آیید و کم گردید گم
\\
زانک جنسیت عجایب جاذبیست
&&
جاذبش جنسست هر جا طالبیست
\\
عیسی و ادریس بر گردون شدند
&&
با ملایک چونک هم‌جنس آمدند
\\
باز آن هاروت و ماروت از بلند
&&
جنس تن بودند زان زیر آمدند
\\
کافران هم جنس شیطان آمده
&&
جانشان شاگرد شیطانان شده
\\
صد هزاران خوی بد آموخته
&&
دیده‌های عقل و دل بر دوخته
\\
کمترین خوشان به زشتی آن حسد
&&
آن حسد که گردن ابلیس زد
\\
زان سگان آموخته حقد و حسد
&&
که نخواهد خلق را ملک ابد
\\
هر کرا دید او کمال از چپ و راست
&&
از حسد قولنجش آمد درد خاست
\\
زآنک هر بدبخت خرمن‌سوخته
&&
می‌نخواهد شمع کس افروخته
\\
هین کمالی دست آور تا تو هم
&&
از کمال دیگران نفتی به غم
\\
از خدا می‌خواه دفع این حسد
&&
تا خدایت وا رهاند از جسد
\\
مر ترا مشغولیی بخشد درون
&&
که نپردازی از آن سوی برون
\\
جرعهٔ می را خدا آن می‌دهد
&&
که بدو مست از دو عالم می‌دهد
\\
خاصیت بنهاده در کف حشیش
&&
کو زمانی می‌رهاند از خودیش
\\
خواب را یزدان بدان سان می‌کند
&&
کز دو عالم فکر را بر می‌کند
\\
کرد مجنون را ز عشق پوستی
&&
کو بنشناسد عدو از دوستی
\\
صد هزاران این چنین می‌دارد او
&&
که بر ادراکات تو بگمارد او
\\
هست میهای شقاوت نفس را
&&
که ز ره بیرون برد آن نحس را
\\
هست میهای سعادت عقل را
&&
که بیابد منزل بی‌نقل را
\\
خیمهٔ گردون ز سرمستی خویش
&&
بر کند زان سو بگیرد راه پیش
\\
هین بهر مستی دلا غره مشو
&&
هست عیسی مست حق خر مست جو
\\
این چنین می را بجو زین خنبها
&&
مستی‌اش نبود ز کوته دنبها
\\
زانک هر معشوق چون خنبیست پر
&&
آن یکی درد و دگر صافی چو در
\\
می‌شناسا هین بچش با احتیاط
&&
تا میی یابی منزه ز اختلاط
\\
هر دو مستی می‌دهندت لیک این
&&
مستی‌ات آرد کشان تا رب دین
\\
تا رهی از فکر و وسواس و حیل
&&
بی عقال این عقل در رقص‌الجمل
\\
انبیا چون جنس روحند و ملک
&&
مر ملک را جذب کردند از فلک
\\
باد جنس آتش است و یار او
&&
که بود آهنگ هر دو بر علو
\\
چون ببندی تو سر کوزهٔ تهی
&&
در میان حوض یا جویی نهی
\\
تا قیامت آن فرو ناید به پست
&&
که دلش خالیست و در وی باد هست
\\
میل بادش چون سوی بالا بود
&&
ظرف خود را هم سوی بالا کشد
\\
باز آن جانها که جنس انبیاست
&&
سوی‌ایشان کش کشان چون سایه‌هاست
\\
زانک عقلش غالبست و بی ز شک
&&
عقل جنس آمد به خلقت با ملک
\\
وان هوای نفس غالب بر عدو
&&
نفس جنس اسفل آمد شد بدو
\\
بود قبطی جنس فرعون ذمیم
&&
بود سبطی جنس موسی کلیم
\\
بود هامان جنس‌تر فرعون را
&&
برگزیدش برد بر صدر سرا
\\
لاجرم از صدر تا قعرش کشید
&&
که ز جنس دوزخ‌اند آن دو پلید
\\
هر دو سوزنده چو ذوزخ ضد نور
&&
هر دو چون دوزخ ز نور دل نفور
\\
زانک دوزخ گوید ای مؤمن تو زود
&&
برگذر که نورت آتش را ربود
\\
می‌رمد آن دوزخی از نور هم
&&
زانک طبع دوزخستش ای صنم
\\
دوزخ از مومن گریزد آنچنان
&&
که گریزد مومن از دوزخ به جان
\\
زانک جنس نار نبود نور او
&&
ضد نار آمد حقیقت نورجو
\\
در حدیث آمدی که مومن در دعا
&&
چون امان خواهد ز دوزخ از خدا
\\
دوزخ از وی هم امان خواهد به جان
&&
که خدایا دور دارم از فلان
\\
جاذبهٔ جنسیتست اکنون ببین
&&
که تو جنس کیستی از کفر و دین
\\
گر بهامان مایلی هامانیی
&&
ور به موسی مایلی سبحانیی
\\
ور بهر و مایلی انگیخته
&&
نفس و عقلی هر دوان آمیخته
\\
هر دو در جنگند هان و هان بکوش
&&
تا شود غالب معانی بر نقوش
\\
در جهان جنگ شادی این بسست
&&
که ببینی بر عدو هر دم شکست
\\
آن ستیزه‌رو بسختی عاقبت
&&
گفت با هامان برای مشورت
\\
وعده‌های آن کلیم‌الله را
&&
گفت و محرم ساخت آن گمراه را
\\
\end{longtable}
\end{center}
