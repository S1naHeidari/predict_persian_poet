\begin{center}
\section*{بخش ۹ - غرض از سمیع و بصیر گفتن خدا را}
\label{sec:sh009}
\addcontentsline{toc}{section}{\nameref{sec:sh009}}
\begin{longtable}{l p{0.5cm} r}
از پی آن گفت حق خود را بصیر
&&
که بود دید ویت هر دم نذیر
\\
از پی آن گفت حق خود را سمیع
&&
تا ببندی لب ز گفتار شنیع
\\
از پی آن گفت حق خود را علیم
&&
تا نیندیشی فسادی تو ز بیم
\\
نیست اینها بر خدا اسم علم
&&
که سیه کافور دارد نام هم
\\
اسم مشتقست و اوصاف قدیم
&&
نه مثال علت اولی سقیم
\\
ورنه تسخر باشد و طنز و دها
&&
کر را سامع ضریران را ضیا
\\
یا علم باشد حیی نام وقیح
&&
یا سیاه زشت را نام صبیح
\\
طفلک نوزاده را حاجی لقب
&&
یا لقب غازی نهی بهر نسب
\\
گر بگویند این لقبها در مدیح
&&
تا ندارد آن صفت نبود صحیح
\\
تسخر و طنزی بود آن یا جنون
&&
پاک حق عما یقول الظالمون
\\
من همی دانستمت پیش از وصال
&&
که نکورویی ولیکن بدخصال
\\
من همی دانستمت پیش از لقا
&&
کز ستیزه راسخی اندر شقا
\\
چونک چشمم سرخ باشد در غمش
&&
دانمش زان درد گر کم بینمش
\\
تو مرا چون بره دیدی بی شبان
&&
تو گمان بردی ندارم پاسبان
\\
عاشقان از درد زان نالیده‌اند
&&
که نظر ناجایگه مالیده‌اند
\\
بی‌شبان دانسته‌اند آن ظبی را
&&
رایگان دانسته‌اند آن سبی را
\\
تا ز غمزه تیر آمد بر جگر
&&
که منم حارس گزافه کم نگر
\\
کی کم از بره کم از بزغاله‌ام
&&
که نباشد حارس از دنباله‌ام
\\
حارسی دارم که ملکش می‌سزد
&&
داند او بادی که آن بر من وزد
\\
سرد بود آن باد یا گرم آن علیم
&&
نیست غافل نیست غایب ای سقیم
\\
نفس شهوانی ز حق کرست و کور
&&
من به دل کوریت می‌دیدم ز دور
\\
هشت سالت زان نپرسیدم به هیچ
&&
که پرت دیدم ز جهل پیچ پیچ
\\
خود چه پرسم آنک او باشد بتون
&&
که تو چونی چون بود او سرنگون
\\
\end{longtable}
\end{center}
