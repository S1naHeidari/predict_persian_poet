\begin{center}
\section*{بخش ۳۷ - چاره کردن سلیمان علیه‌السلام در احضار تخت بلقیس از سبا}
\label{sec:sh037}
\addcontentsline{toc}{section}{\nameref{sec:sh037}}
\begin{longtable}{l p{0.5cm} r}
گفت عفریتی که تختش را به فن
&&
حاضر آرم تا تو زین مجلس شدن
\\
گفت آصف من به اسم اعظمش
&&
حاضر آرم پیش تو در یک دمش
\\
گرچه عفریت اوستاد سحر بود
&&
لیک آن از نفخ آصف رو نمود
\\
حاضر آمد تخت بلقیس آن زمان
&&
لیک ز آصف نه از فن عفریتیان
\\
گفت حمدالله برین و صد چنین
&&
که بدیدستم ز رب العالمین
\\
پس نظر کرد آن سلیمان سوی تخت
&&
گفت آری گول‌گیری ای درخت
\\
پیش چوب و پیش سنگ نقش کند
&&
ای بسا گولان که سرها می‌نهند
\\
ساجد و مسجود از جان بی‌خبر
&&
دیده از جان جنبشی واندک اثر
\\
دیده در وقتی که شد حیران و دنگ
&&
که سخن گفت و اشارت کرد سنگ
\\
نرد خدمت چون بنا موضع بباخت
&&
شیر سنگین را شقی شیری شناخت
\\
از کرم شیر حقیقی کرد جود
&&
استخوانی سوی سگ انداخت زود
\\
گفت گرچه نیست آن سگ بر قوام
&&
لیک ما را استخوان لطفیست عام
\\
\end{longtable}
\end{center}
