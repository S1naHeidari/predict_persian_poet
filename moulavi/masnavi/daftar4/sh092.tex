\begin{center}
\section*{بخش ۹۲ - حمله بردن این جهانیان بر آن  جهانیان و تاختن بردن تا  سینور ذر و نسل کی سر حد غیب است و غفلت ایشان از کمین کی چون غازی به غزا نرود کافر تاختن آورد}
\label{sec:sh092}
\addcontentsline{toc}{section}{\nameref{sec:sh092}}
\begin{longtable}{l p{0.5cm} r}
حمله بردند اسپه جسمانیان
&&
جانب قلعه و دز روحانیان
\\
تا فرو گیرند بر دربند غیب
&&
تا کسی ناید از آن سو پاک‌جیب
\\
غازیان حملهٔ غزا چون کم برند
&&
کافران برعکس حمله آورند
\\
غازیان غیب چون از حلم خویش
&&
حمله ناوردند بر تو زشت‌کیش
\\
حمله بردی سوی دربندان غیب
&&
تا نیایند این طرف مردان غیب
\\
چنگ در صلب و رحمها در زدی
&&
تا که شارع را بگیری از بدی
\\
چون بگیری شه‌رهی که ذوالجلال
&&
بر گشادست از برای انتسال
\\
سد شدی دربندها را ای لجوج
&&
کوری تو کرد سرهنگی خروج
\\
نک منم سرهنگ هنگت بشکنم
&&
نک به نامش نام و ننگت بشکنم
\\
تو هلا در بندها را سخت بند
&&
چندگاهی بر سبال خود بخند
\\
سبلتت را بر کند یک یک قدر
&&
تا بدانی کالقدر یعمی الحذر
\\
سبلت تو تیزتر یا آن عاد
&&
که همی لرزید از دمشان بلاد
\\
تو ستیزه‌روتری یا آن ثمود
&&
که نیامد مثل ایشان در وجود
\\
صد ازینها گر بگویم تو کری
&&
بشنوی و ناشنوده آوری
\\
توبه کردم از سخن که انگیختم
&&
بی‌سخن من دارویت آمیختم
\\
که نهم بر ریش خامت تا پزد
&&
یا بسوزد ریش و ریشه‌ت تا ابد
\\
تا بدانی که خبیرست ای عدو
&&
می‌دهد هر چیز را درخورد او
\\
کی کژی کردی و کی کردی تو شر
&&
که ندیدی لایقش در پی اثر
\\
کی فرستادی دمی بر آسمان
&&
نیکیی کز پی نیامد مثل آن
\\
گر مراقب باشی و بیدار تو
&&
بینی هر دم پاسخ کردار تو
\\
چون مراقب باشی و گیری رسن
&&
حاجتت ناید قیامت آمدن
\\
آنک رمزی را بداند او صحیح
&&
حاجتش ناید که گویندش صریح
\\
این بلا از کودنی آید ترا
&&
که نکردی فهم نکته و رمزها
\\
از بدی چون دل سیاه و تیره شد
&&
فهم کن اینجا نشاید خیره شد
\\
ورنه خود تیری شود آن تیرگی
&&
در رسد در تو جزای خیرگی
\\
ور نیاید تیر از بخشایش است
&&
نه پی نادیدن آلایش است
\\
هین مراقب باش گر دل بایدت
&&
کز پی هر فعل چیزی زایدت
\\
ور ازین افزون ترا همت بود
&&
از مراقب کار بالاتر رود
\\
\end{longtable}
\end{center}
