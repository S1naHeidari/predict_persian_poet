\begin{center}
\section*{بخش ۱۳۰ - تصدیق کردن استر جوابهای شتر را و اقرار کردن بفضل او بر خود و ازو استعانت خواستن و بدو پناه گرفتن به صدق و نواختن شتر او را و ره نمودن و یاری دادن پدرانه و شاهانه}
\label{sec:sh130}
\addcontentsline{toc}{section}{\nameref{sec:sh130}}
\begin{longtable}{l p{0.5cm} r}
گفت استر راست گفتی ای شتر
&&
این بگفت و چشم کرد از اشک پر
\\
ساعتی بگریست و در پایش فتاد
&&
گفت ای بگزیدهٔ رب العباد
\\
چه زیان دارد گر از فرخندگی
&&
در پذیری تو مرا دربندگی
\\
گفت چون اقرار کردی پیش من
&&
رو که رستی تو ز آفات زمن
\\
دادی انصاف و رهیدی از بلا
&&
تو عدو بودی شدی ز اهل ولا
\\
خوی بد در ذات تو اصلی نبود
&&
کز بد اصلی نیاید جز جحود
\\
آن بد عاریتی باشد که او
&&
آرد اقرار و شود او توبه‌جو
\\
هم‌چو آدم زلتش عاریه بود
&&
لاجرم اندر زمان توبه نمود
\\
چونک اصلی بود جرم آن بلیس
&&
ره نبودش جانب توبهٔ نفیس
\\
رو که رستی از خود و از خوی بد
&&
واز زبانهٔ نار و از دندان دد
\\
رو که اکنون دست در دولت زدی
&&
در فکندی خود به بخت سرمدی
\\
ادخلی تو فی عبادی یافتی
&&
ادخلی فی جنتی در بافتی
\\
در عبادش راه کردی خویش را
&&
رفتی اندر خلد از راه خفا
\\
اهدنا گفتی صراط مستقیم
&&
دست تو بگرفت و بردت تا نعیم
\\
نار بودی نور گشتی ای عزیز
&&
غوره بودی گشتی انگور و مویز
\\
اختری بودی شدی تو آفتاب
&&
شاد باشد الله اعلم بالصواب
\\
ای ضیاء الحق حسام‌الدین بگیر
&&
شهد خویش اندر فکن در حوض شیر
\\
تا رهد آن شیر از تغییر طعم
&&
یابد از بحر مزه تکثیر طعم
\\
متصل گردد بدان بحر الست
&&
چونک شد دریا ز هر تغییر رست
\\
منفذی یابد در آن بحر عسل
&&
آفتی را نبود اندر وی عمل
\\
غره‌ای کن شیروار ای شیر حق
&&
تا رود آن غره بر هفتم طبق
\\
چه خبر جان ملول سیر را
&&
کی شناسد موش غرهٔ شیر را
\\
برنویس احولا خود با آب زر
&&
بهر هر دریادلی نیکوگهر
\\
آب نیلست این حدیث جان‌فزا
&&
یا ربش در چشم قبطی خون نما
\\
\end{longtable}
\end{center}
