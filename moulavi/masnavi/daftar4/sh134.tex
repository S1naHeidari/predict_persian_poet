\begin{center}
\section*{بخش ۱۳۴ - باقی قصهٔ موسی علیه‌السلام}
\label{sec:sh134}
\addcontentsline{toc}{section}{\nameref{sec:sh134}}
\begin{longtable}{l p{0.5cm} r}
که آمدش پیغام از وحی مهم
&&
که کژی بگذار اکنون فاستقم
\\
این درخت تن عصای موسیست
&&
که امرش آمد که بیندازش ز دست
\\
تا ببینی خیر او و شر او
&&
بعد از آن بر گیر او را ز امر هو
\\
پیش از افکندن نبود او غیر چوب
&&
چون به امرش بر گرفتی گشت خوب
\\
اول او بد برگ‌افشان بره را
&&
گشت معجز آن گروه غره را
\\
گشت حاکم بر سر فرعونیان
&&
آبشان خون کرد و کف بر سر زنان
\\
از مزارعشان برآمد قحط و مرگ
&&
از ملخهایی که می‌خوردند برگ
\\
تا بر آمد بی‌خود از موسی دعا
&&
چون نظر افتادش اندر منتها
\\
کین همه اعجاز و کوشیدن چراست
&&
چون نخواهند این جماعت گشت راست
\\
امر آمد که اتباع نوح کن
&&
ترک پایان‌بینی مشروح کن
\\
زان تغافل کن چو داعی رهی
&&
امر بلغ هست نبود آن تهی
\\
کمترین حکمت کزین الحاح تو
&&
جلوه گردد آن لجاج و آن عتو
\\
تا که ره بنمودن و اضلال حق
&&
فاش گردد بر همه اهل و فرق
\\
چونک مقصود از وجود اظهار بود
&&
بایدش از پند و اغوا آزمود
\\
دیو الحاح غوایت می‌کند
&&
شیخ‌الحاح هدایت می‌کند
\\
چون پیاپی گشت آن امر شجون
&&
نیل می‌آمد سراسر جمله خون
\\
تا بنفس خویش فرعون آمدش
&&
لابه می‌کردش دو تا گشته قدش
\\
کانچ ما کردیم ای سلطان مکن
&&
نیست ما را روی ایراد سخن
\\
پاره پاره گردمت فرمان‌پذیر
&&
من بعزت خوگرم سختم مگیر
\\
هین بجنبان لب به رحمت ای امین
&&
تا ببندد این دهانهٔ آتشین
\\
گفت یا رب می‌فریبد او مرا
&&
می‌فریبد او فریبندهٔ ترا
\\
بشنوم یا من دهم هم خدعه‌اش
&&
تا بداند اصل را آن فرع‌کش
\\
که اصل هر مکری و حیلت پیش ماست
&&
هر چه بر خاکست اصلش از سماست
\\
گفت حق آن سگ نیرزد هم به آن
&&
پیش سگ انداز از دور استخوان
\\
هین بجنبان آن عصا تا خاکها
&&
وا دهد هرچه ملخ کردش فنا
\\
وان ملخها در زمان گردد سیاه
&&
تا ببیند خلق تبدیل اله
\\
که سببها نیست حاجت مر مرا
&&
آن سبب بهر حجابست و غطا
\\
تا طبیعی خویش بر دارو زند
&&
تا منجم رو با ستاره کند
\\
تا منافق از حریصی بامداد
&&
سوی بازار آید از بیم کساد
\\
بندگی ناکرده و ناشسته روی
&&
لقمهٔ دوزخ بگشته لقمه‌جوی
\\
آکل و ماکول آمد جان عام
&&
هم‌چو آن برهٔ چرنده از حطام
\\
می‌چرد آن بره و قصاب شاد
&&
کو برای ما چرد برگ مراد
\\
کار دوزخ می‌کنی در خوردنی
&&
بهر او خود را تو فربه می‌کنی
\\
کار خود کن روزی حکمت بچر
&&
تا شود فربه دل با کر و فر
\\
خوردن تن مانع این خوردنست
&&
جان چو بازرگان و تن چون ره‌زنست
\\
شمع تاجر آنگهست افروخته
&&
که بود ره‌زن چو هیزم سوخته
\\
که تو آن هوشی و باقی هوش‌پوش
&&
خویشتن را گم مکن یاوه مکوش
\\
دانک هر شهوت چو خمرست و چو بنگ
&&
پردهٔ هوشست وعاقل زوست دنگ
\\
خمر تنها نیست سرمستی هوش
&&
هر چه شهوانیست بندد چشم و گوش
\\
آن بلیس از خمر خوردن دور بود
&&
مست بود او از تکبر وز جحود
\\
مست آن باشد که آن بیند که نیست
&&
زر نماید آنچ مس و آهنیست
\\
این سخن پایان ندارد موسیا
&&
لب بجنبان تا برون روژد گیا
\\
هم‌چنان کرد و هم اندر دم زمین
&&
سبز گشت از سنبل و حب ثمین
\\
اندر افتادند در لوت آن نفر
&&
قحط دیده مرده از جوع البقر
\\
چند روزی سیر خوردند از عطا
&&
آن دمی و آدمی و چارپا
\\
چون شکم پر گشت و بر نعمت زدند
&&
وآن ضرورت رفت پس طاغی شدند
\\
نفس فرعونیست هان سیرش مکن
&&
تا نیارد یاد از آن کفر کهن
\\
بی تف آتش نگردد نفس خوب
&&
تا نشد آهن چو اخگر هین مکوب
\\
بی‌مجاعت نیست تن جنبش‌کنان
&&
آهن سردیست می‌کوبی بدان
\\
گر بگرید ور بنالد زار زار
&&
او نخواهد شد مسلمان هوش دار
\\
او چو فرعونست در قحط آنچنان
&&
پیش موسی سر نهد لابه‌کنان
\\
چونک مستغنی شد او طاغی شود
&&
خر چو بار انداخت اسکیزه زند
\\
پس فراموشش شود چون رفت پیش
&&
کار او زان آه و زاریهای خویش
\\
سالها مردی که در شهری بود
&&
یک زمان که چشم در خوابی رود
\\
شهر دیگر بیند او پر نیک و بد
&&
هیچ در یادش نیاید شهر خود
\\
که من آنجا بوده‌ام این شهر نو
&&
نیست آن من درینجاام گرو
\\
بل چنان داند که خود پیوسته او
&&
هم درین شهرش به دست ابداع و خو
\\
چه عجب گر روح موطنهای خویش
&&
که بدستش مسکن و میلاد پیش
\\
می‌نیارد یاد کین دنیا چو خواب
&&
می‌فرو پوشد چو اختر را سحاب
\\
خاصه چندین شهرها را کوفته
&&
گردها از درک او ناروفته
\\
اجتهاد گرم ناکرده که تا
&&
دل شود صاف و ببیند ماجرا
\\
سر برون آرد دلش از بخش راز
&&
اول و آخر ببیند چشم باز
\\
\end{longtable}
\end{center}
