\begin{center}
\section*{بخش ۶ - قصهٔ آن صوفی کی زن خود را بیگانه‌ای بگرفت}
\label{sec:sh006}
\addcontentsline{toc}{section}{\nameref{sec:sh006}}
\begin{longtable}{l p{0.5cm} r}
صوفیی آمد به سوی خانه روز
&&
خانه یک در بود و زن با کفش‌دوز
\\
جفت گشته با رهی خویش زن
&&
اندر آن یک حجره از وسواس تن
\\
چون بزد صوفی به جد در چاشتگاه
&&
هر دو درماندند نه حیلت نه راه
\\
هیچ معهودش نبد کو آن زمان
&&
سوی خانه باز گردد از دکان
\\
قاصدا آن روز بی‌وقت آن مروع
&&
از خیالی کرد تا خانه رجوع
\\
اعتماد زن بر آن کو هیچ بار
&&
این زمان فا خانه نامد او ز کار
\\
آن قیاسش راست نامد از قضا
&&
گرچه ستارست هم بدهد سزا
\\
چونک بد کردی بترس آمن مباش
&&
زانک تخمست و برویاند خداش
\\
چند گاهی او بپوشاند که تا
&&
آیدت زان بد پشیمان و حیا
\\
عهد عمر آن امیر مؤمنان
&&
داد دزدی را به جلاد و عوان
\\
بانگ زد آن دزد کای میر دیار
&&
اولین بارست جرمم زینهار
\\
گفت عمر حاش لله که خدا
&&
بار اول قهر بارد در جزا
\\
بارها پوشد پی اظهار فضل
&&
باز گیرد از پی اظهار عدل
\\
تا که این هر دو صفت ظاهر شود
&&
آن مبشر گردد این منذر شود
\\
بارها زن نیز این بد کرده بود
&&
سهل بگذشت آن و سهلش می‌نمود
\\
آن نمی‌دانست عقل پای‌سست
&&
که سبو دایم ز جو ناید درست
\\
آنچنانش تنگ آورد آن قضا
&&
که منافق را کند مرگ فجا
\\
نه طریق و نه رفیق و نه امان
&&
دست کرده آن فرشته سوی جان
\\
آنچنان کین زن در آن حجره جفا
&&
خشک شد او و حریفش ز ابتلا
\\
گفت صوفی با دل خود کای دو گبر
&&
از شما کینه کشم لیکن به صبر
\\
لیک نادانسته آرم این نفس
&&
تا که هر گوشی ننوشد این جرس
\\
از شما پنهان کشد کینه محق
&&
اندک اندک هم‌چو بیماری دق
\\
مرد دق باشد چو یخ هر لحظه کم
&&
لیک پندارد بهر دم بهترم
\\
هم‌چو کفتاری که می‌گیرندش و او
&&
غرهٔ آن گفت کین کفتار کو
\\
هیچ پنهان‌خانه آن زن را نبود
&&
سمج و دهلیز و ره بالا نبود
\\
نه تنوری که در آن پنهان شود
&&
نه جوالی که حجاب آن شود
\\
هم‌چو عرصهٔ پهن روز رستخیز
&&
نه گو و نه پشته نه جای گریز
\\
گفت یزدان وصف این جای حرج
&&
بهر محشر لا تری فیها عوج
\\
\end{longtable}
\end{center}
