\begin{center}
\section*{بخش ۱۲۲ - در بیان آنک شه‌زاده آدمی بچه است خلیفهٔ خداست پدرش آدم صفی خلیفهٔ حق مسجود ملایک و آن کمپیر کابلی دنیاست کی آدمی‌بچه را از پدر ببرید به سحر و انبیا و اولیا آن طبیب تدارک کننده}
\label{sec:sh122}
\addcontentsline{toc}{section}{\nameref{sec:sh122}}
\begin{longtable}{l p{0.5cm} r}
ای برادر دانک شه‌زاده توی
&&
در جهان کهنه زاده از نوی
\\
کابلی جادو این دنیاست کو
&&
کرد مردان را اسیر رنگ و بو
\\
چون در افکندت دریغ آلوده روذ
&&
دم به دم می‌خوان و می‌دم قل اعوذ
\\
تا رهی زین جادوی و زین قلق
&&
استعاذت خواه از رب الفلق
\\
زان نبی دنیات را سحاره خواند
&&
کو به افسون خلق را در چه نشاند
\\
هین فسون گرم دارد گنده پیر
&&
کرده شاهان را دم گرمش اسیر
\\
در درون سینه نفاثات اوست
&&
عقده‌های سحر را اثبات اوست
\\
ساحرهٔ دنیا قوی دانا زنیست
&&
حل سحر او به پای عامه نیست
\\
ور گشادی عقد او را عقلها
&&
انبیا را کی فرستادی خدا
\\
هین طلب کن خوش‌دمی عقده‌گشا
&&
رازدان یفعل الله ما یشا
\\
هم‌چو ماهی بسته است او به شست
&&
شاه زاده ماند سالی و تو شصت
\\
شصت سال از شست او در محنتی
&&
نه خوشی نه بر طریق سنتی
\\
فاسقی بدبخت نه دنیات خوب
&&
نه رهیده از وبال و از ذنوب
\\
نفخ او این عقده‌ها را سخت کرد
&&
پس طلب کن نفخهٔ خلاق فرد
\\
تا نفخت فیه من روحی ترا
&&
وا رهاند زین و گوید برتر آ
\\
جز به نفخ حق نسوزد نفخ سحر
&&
نفخ قهرست این و آن دم نفح مهر
\\
رحمت او سابقست از قهر او
&&
سابقی خواهی برو سابق بجو
\\
تا رسی اندر نفوس زوجت
&&
کای شه مسحور اینک مخرجت
\\
با وجود زال ناید انحلال
&&
در شبیکه و در بر آن پر دلال
\\
نه بگفتست آن سراج امتان
&&
این جهان و آن جهان را ضرتان
\\
پس وصال این فراق آن بود
&&
صحت این تن سقام جان بود
\\
سخت می‌آید فراق این ممر
&&
پس فراق آن مقر دان سخت‌تر
\\
چون فراق نقش سخت آید ترا
&&
تا چه سخت آید ز نقاشش جدا
\\
ای که صبرت نیست از دنیای دون
&&
چونت صبرست از خدا ای دوست چون
\\
چونک صبرت نیست زین آب سیاه
&&
چون صبوری داری از چشمهٔ اله
\\
چونک بی این شرب کم داری سکون
&&
چون ز ابراری جدا وز یشربون
\\
گر ببینی یک نفس حسن ودود
&&
اندر آتش افکنی جان و وجود
\\
جیفه بینی بعد از آن این شرب را
&&
چون ببینی کر و فر قرب را
\\
هم‌چو شه‌زاده رسی در یار خویش
&&
پس برون آری ز پا تو خار خویش
\\
جهد کن در بی‌خودی خود را بیاب
&&
زودتر والله اعلم بالصواب
\\
هر زمانی هین مشو با خویش جفت
&&
هر زمان چون خر در آب و گل میفت
\\
از قصور چشم باشد آن عثار
&&
که نبیند شیب و بالا کور وار
\\
بوی پیراهان یوسف کن سند
&&
زانک بویش چشم روشن می‌کند
\\
صورت پنهان و آن نور جبین
&&
کرده چشم انبیا را دوربین
\\
نور آن رخسار برهاند ز نار
&&
هین مشو قانع به نور مستعار
\\
چشم را این نور حالی‌بین کند
&&
جسم و عقل و روح را گرگین کند
\\
صورتش نورست و در تحقیق نار
&&
گر ضیا خواهی دو دست از وی بدار
\\
دم به دم در رو فتد هر جا رود
&&
دیده و جانی که حالی‌بین بود
\\
دور بیند دوربین بی‌هنر
&&
هم‌چنانک دور دیدن خواب در
\\
خفته باشی بر لب جو خشک‌لب
&&
می‌دوی سوی سراب اندر طلب
\\
دور می‌بینی سراب و می‌دوی
&&
عاشق آن بینش خود می‌شوی
\\
می‌زنی در خواب با یاران تو لاف
&&
که منم بینادل و پرده‌شکاف
\\
نک بدان سو آب دیدم هین شتاب
&&
تا رویم آنجا و آن باشد سراب
\\
هر قدم زین آب تازی دورتر
&&
دو دوان سوی سراب با غرر
\\
عین آن عزمت حجاب این شده
&&
که به تو پیوسته است و آمده
\\
بس کسا عزمی به جایی می‌کند
&&
از مقامی کان غرض در وی بود
\\
دید و لاف خفته می‌ناید به کار
&&
جز خیالی نیست دست از وی بدار
\\
خوابناکی لیک هم بر راه خسپ
&&
الله الله بر ره الله خسپ
\\
تا بود که سالکی بر تو زند
&&
از خیالات نعاست بر کند
\\
خفته را گر فکر گردد هم‌چو موی
&&
او از آن دقت نیابد راه کوی
\\
فکر خفته گر دوتا و گر سه‌تاست
&&
هم خطا اندر خطا اندر خطاست
\\
موج بر وی می‌زند بی‌احتراز
&&
خفته پویان در بیابان دراز
\\
خفته می‌بیند عطشهای شدید
&&
آب اقرب منه من حبل الورید
\\
\end{longtable}
\end{center}
