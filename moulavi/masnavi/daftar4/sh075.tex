\begin{center}
\section*{بخش ۷۵ - قصهٔ آنک کسی به کسی مشورت می‌کرد گفتش  مشورت با دیگری کن کی من عدوی توم}
\label{sec:sh075}
\addcontentsline{toc}{section}{\nameref{sec:sh075}}
\begin{longtable}{l p{0.5cm} r}
مشورت می‌کرد شخصی با کسی
&&
کز تردد وا ردهد وز محبسی
\\
گفت ای خوش‌نام غیر من بجو
&&
ماجرای مشورت با او بگو
\\
من عدوم مر ترا با من مپیچ
&&
نبود از رای عدو پیروز هیچ
\\
رو کسی جو که ترا او هست دوست
&&
دوست بهر دوست لاشک خیرجوست
\\
من عدوم چاره نبود کز منی
&&
کژ روم با تو نمایم دشمنی
\\
حارسی از گرگ جستن شرط نیست
&&
جستن از غیر محل ناجستنیست
\\
من ترا بی‌هیچ شکی دشمنم
&&
من ترا کی ره نمایم ره زنم
\\
هر که باشد همنشین دوستان
&&
هست در گلخن میان بوستان
\\
هر که با دشمن نشیند در زمن
&&
هست او در بوستان در گولخن
\\
دوست را مازار از ما و منت
&&
تا نگردد دوست خصم و دشمنت
\\
خیر کن با خلق بهر ایزدت
&&
یا برای راحت جان خودت
\\
تا هماره دوست بینی در نظر
&&
در دلت ناید ز کین ناخوش صور
\\
چونک کردی دشمنی پرهیز کن
&&
مشورت با یار مهرانگیز کن
\\
گفت می‌دانم ترا ای بوالحسن
&&
که توی دیرینه دشمن‌دار من
\\
لیک مرد عاقلی و معنوی
&&
عقل تو نگذاردت که کژ روی
\\
طبع خواهد تا کشد از خصم کین
&&
عقل بر نفس است بند آهنین
\\
آید و منعش کند وا داردش
&&
عقل چون شحنه‌ست در نیک و بدش
\\
عقل ایمانی چو شحنهٔ عادلست
&&
پاسبان و حاکم شهر دلست
\\
هم‌چو گربه باشد او بیدارهوش
&&
دزد در سوراخ ماند هم‌چو موش
\\
در هر آنجا که برآرد موش دست
&&
نیست گربه یا که نقش گربه است
\\
گربهٔ چه شیر شیرافکن بود
&&
عقل ایمانی که اندر تن بود
\\
غرهٔ او حاکم درندگان
&&
نعرهٔ او مانع چرندگان
\\
شهر پر دزدست و پر جامه‌کنی
&&
خواه شحنه باش گو و خواه نی
\\
\end{longtable}
\end{center}
