\begin{center}
\section*{بخش ۱۰۰ - بیان این خبر کی کلموا الناس علی قدر عقولهم لا علی قدر عقولکم حتی لا یکذبوا الله و رسوله}
\label{sec:sh100}
\addcontentsline{toc}{section}{\nameref{sec:sh100}}
\begin{longtable}{l p{0.5cm} r}
چونک با کودک سر و کارم فتاد
&&
هم زبان کودکان باید گشاد
\\
که برو کتاب تا مرغت خرم
&&
یا مویز و جوز و فستق آورم
\\
جز شباب تن نمی‌دانی به کیر
&&
این جوانی را بگیر ای خر شعیر
\\
هیچ آژنگی نیفتد بر رخت
&&
تازه ماند آن شباب فرخت
\\
نه نژند پیریت آید برو
&&
نه قد چون سرو تو گردد دوتو
\\
نه شود زور جوانی از تو کم
&&
نه به دندانها خللها یا الم
\\
نه کمی در شهوت و طمث و بعال
&&
که زنان را آید از ضعفت ملال
\\
آنچنان بگشایدت فر شباب
&&
که گشود آن مژدهٔ عکاشه باب
\\
\end{longtable}
\end{center}
