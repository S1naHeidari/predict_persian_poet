\begin{center}
\section*{بخش ۱۰۶ - تزییف سخن هامان علیه‌اللعنه}
\label{sec:sh106}
\addcontentsline{toc}{section}{\nameref{sec:sh106}}
\begin{longtable}{l p{0.5cm} r}
دوست از دشمن همی نشناخت او
&&
نرد را کورانه کژ می‌باخت او
\\
دشمن تو جز تو نبود این لعین
&&
بی‌گناهان را مگو دشمن به کین
\\
پیش تو این حالت بد دولتست
&&
که دوادو اول و آخر لتست
\\
گر ازین دولت نتازی خز خزان
&&
این بهارت را همی آید خزان
\\
مشرق و مغرب چو تو بس دیده‌اند
&&
که سر ایشان ز تن ببریده‌اند
\\
مشرق و مغرب که نبود بر قرار
&&
چون کنند آخر کسی را پایدار
\\
تو بدان فخر آوری کز ترس و بند
&&
چاپلوست گشت مردم روز چند
\\
هر کرا مردم سجودی می‌کنند
&&
زهر اندر جان او می‌آکنند
\\
چونک بر گردد ازو آن ساجدش
&&
داند او کان زهر بود و موبدش
\\
ای خنک آن را که ذلت نفسه
&&
وای آنک از سرکشی شد چون که او
\\
این تکبر زهر قاتل دان که هست
&&
از می پر زهر شد آن گیج مست
\\
چون می پر زهر نوشد مدبری
&&
از طرب یکدم بجنباند سری
\\
بعد یک‌دم زهر بر جانش فتد
&&
زهر در جانش کند داد و ستد
\\
گر نذاری زهری‌اش را اعتقاد
&&
کو چه زهر آمد نگر در قوم عاد
\\
چونک شاهی دست یابد بر شهی
&&
بکشدش یا باز دارد در چهی
\\
ور بیابد خستهٔ افتاده را
&&
مرهمش سازد شه و بدهد عطا
\\
گر نه زهرست آن تکبر پس چرا
&&
کشت شه را بی‌گناه و بی‌خطا
\\
وین دگر را بی ز خدمت چون نواخت
&&
زین دو جنبش زهر را شاید شناخت
\\
راه‌زن هرگز گدایی را نزد
&&
گرگ گرگ مرده را هرگز گزد
\\
خضر کشتی را برای آن شکست
&&
تا تواند کشتی از فجار رست
\\
چون شکسته می‌رهد اشکسته شو
&&
امن در فقرست اندر فقر رو
\\
آن کهی کو داشت از کان نقد چند
&&
گشت پاره پاره از زخم کلند
\\
تیغ بهر اوست کو را گردنیست
&&
سایه که افکندست بر وی زخم نیست
\\
مهتری نفطست و آتش ای غوی
&&
ای برادر چون بر آذر می‌روی
\\
هر چه او هموار باشد با زمین
&&
تیرها را کی هدف گردد ببین
\\
سر بر آرد از زمین آنگاه او
&&
چون هدفها زخم یابد بی رفو
\\
نردبان خلق این ما و منیست
&&
عاقبت زین نردبان افتادنیست
\\
هر که بالاتر رود ابله‌ترست
&&
که استخوان او بتر خواهد شکست
\\
این فروعست و اصولش آن بود
&&
که ترفع شرکت یزدان بود
\\
چون نمردی و نگشتی زنده زو
&&
یاغیی باشی به شرکت ملک‌جو
\\
چون بدو زنده شدی آن خود ویست
&&
وحدت محضست آن شرکت کیست
\\
شرح این در آینهٔ اعمال جو
&&
که نیابی فهم آن از گفت و گو
\\
گر بگویم آنچ دارم در درون
&&
بس جگرها گردد اندر حال خون
\\
بس کنم خود زیرکان را این بس است
&&
بانگ دو کردم اگر در ده کس است
\\
حاصل آن هامان بدان گفتار بد
&&
این چنین راهی بر آن فرعون زد
\\
لقمهٔ دولت رسیده تا دهان
&&
او گلوی او بریده ناگهان
\\
خرمن فرعون را داد او به باد
&&
هیچ شه را این چنین صاحب مباد
\\
\end{longtable}
\end{center}
