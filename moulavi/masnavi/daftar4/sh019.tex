\begin{center}
\section*{بخش ۱۹ - قصهٔ آغاز خلافت عثمان رضی الله  عنه و خطبهٔ وی در بیان آنک ناصح  فعال به فعل به از ناصح قوال به قول}
\label{sec:sh019}
\addcontentsline{toc}{section}{\nameref{sec:sh019}}
\begin{longtable}{l p{0.5cm} r}
قصهٔ عثمان که بر منبر برفت
&&
چون خلافت یافت بشتابید تفت
\\
منبر مهتر که سه‌پایه بدست
&&
رفت بوبکر و دوم پایه نشست
\\
بر سوم پایه عمر در دور خویش
&&
از برای حرمت اسلام و کیش
\\
دور عثمان آمد او بالای تخت
&&
بر شد و بنشست آن محمودبخت
\\
پس سؤالش کرد شخصی بوالفضول
&&
که آن دو ننشستند بر جای رسول
\\
پس تو چون جستی ازیشان برتری
&&
چون برتبت تو ازیشان کمتری
\\
گفت اگر پایهٔ سوم را بسپرم
&&
وهم آید که مثال عمرم
\\
بر دوم پایه شوم من جای‌جو
&&
گویی بوبکرست و این هم مثل او
\\
هست این بالا مقام مصطفی
&&
وهم مثلی نیست با آن شه مرا
\\
بعد از آن بر جای خطبه آن ودود
&&
تا به قرب عصر لب‌خاموش بود
\\
زهره نه کس را که گوید هین بخوان
&&
یا برون آید ز مسجد آن زمان
\\
هیبتی بنشسته بد بر خاص و عام
&&
پر شده نور خدا آن صحن و بام
\\
هر که بینا ناظر نورش بدی
&&
کور زان خورشید هم گرم آمدی
\\
پس ز گرمی فهم کردی چشم کور
&&
که بر آمد آفتابی بی‌فتور
\\
لیک این گرمی گشاید دیده را
&&
تا ببیند عین هر بشنیده را
\\
گرمیش را ضجرتی و حالتی
&&
زان تبش دل را گشادی فسحتی
\\
کور چون شد گرم از نور قدم
&&
از فرح گوید که من بینا شدم
\\
سخت خوش مستی ولی ای بوالحسن
&&
پاره‌ای راهست تا بینا شدن
\\
این نصیب کور باشد ز آفتاب
&&
صد چنین والله اعلم بالصواب
\\
وآنک او آن نور را بینا بود
&&
شرح او کی کار بوسینا بود
\\
ور شود صد تو که باشد این زبان
&&
که بجنباند به کف پردهٔ عیان
\\
وای بر وی گر بساید پرده را
&&
تیغ اللهی کند دستش جدا
\\
دست چه بود خود سرش را بر کند
&&
آن سری کز جهل سرها می‌کند
\\
این به تقدیر سخن گفتم ترا
&&
ورنه خود دستش کجا و آن کجا
\\
خاله را خایه بدی خالو شدی
&&
این به تقدیر آمدست ار او بدی
\\
از زبان تا چشم کو پاک از شکست
&&
صد هزاران ساله گویم اندکست
\\
هین مشو نومید نور از آسمان
&&
حق چو خواهد می‌رسد در یک زمان
\\
صد اثر در کانها از اختران
&&
می‌رساند قدرتش در هر زمان
\\
اختر گردون ظلم را ناسخست
&&
اختر حق در صفاتش راسخست
\\
چرخ پانصد ساله راه ای مستعین
&&
در اثر نزدیک آمد با زمین
\\
سه هزاران سال و پانصد تا زحل
&&
دم بدم خاصیتش آرد عمل
\\
در همش آرد چو سایه در ایاب
&&
طول سایه چیست پیش آفتاب
\\
وز نفوس پاک اختروش مدد
&&
سوی اخترهای گردون می‌رسد
\\
ظاهر آن اختران قوام ما
&&
باطن ما گشته قوام سما
\\
\end{longtable}
\end{center}
