\begin{center}
\section*{بخش ۴۳ - مثل قانع شدن آدمی به دنیا و حرص او در طلب دنیا و غفلت او از دولت روحانیان کی ابنای جنس وی‌اند و نعره‌زنان کی  یا لیت قومی یعلمون}
\label{sec:sh043}
\addcontentsline{toc}{section}{\nameref{sec:sh043}}
\begin{longtable}{l p{0.5cm} r}
آن سگی در کو گدای کور دید
&&
حمله می‌آورد و دلقش می‌درید
\\
گفته‌ایم این را ولی باری دگر
&&
شد مکرر بهر تاکید خبر
\\
کور گفتش آخر آن یاران تو
&&
بر کهند این دم شکاری صیدجو
\\
قوم تو در کوه می‌گیرند گور
&&
در میان کوی می‌گیری تو کور
\\
ترک این تزویر گو شیخ نفور
&&
آب شوری جمع کرده چند کور
\\
کین مریدان من و من آب شور
&&
می‌خورند از من همی گردند کور
\\
آب خود شیرین کن از بحر لدن
&&
آب بد را دام این کوران مکن
\\
خیز شیران خدا بین گورگیر
&&
تو چو سگ چونی بزرقی کورگیر
\\
گور چه از صید غیر دوست دور
&&
جمله شیر و شیرگیر و مست نور
\\
در نظاره صید و صیادی شه
&&
کرده ترک صید و مرده در وله
\\
هم‌چو مرغ مرده‌شان بگرفته یار
&&
تا کند او جنس ایشان را شکار
\\
مرغ مرده مضطر اندر وصل و بین
&&
خوانده‌ای القلب بین اصبعین
\\
مرغ مرده‌ش را هر آنک شد شکار
&&
چون ببیند شد شکار شهریار
\\
هر که او زین مرغ مرده سر بتافت
&&
دست آن صیاد را هرگز نیافت
\\
گوید او منگر به مرداری من
&&
عشق شه بین در نگهداری من
\\
من نه مردارم مرا شه کشته است
&&
صورت من شبه مرده گشته است
\\
جنبشم زین پیش بود از بال و پر
&&
جنبشم اکنون ز دست دادگر
\\
جنبش فانیم بیرون شد ز پوست
&&
جنبشم باقیست اکنون چون ازوست
\\
هر که کژ جنبد به پیش جنبشم
&&
گرچه سیمرغست زارش می‌کشم
\\
هین مرا مرده مبین گر زنده‌ای
&&
در کف شاهم نگر گر بنده‌ای
\\
مرده زنده کرد عیسی از کرم
&&
من به کف خالق عیسی درم
\\
کی بمانم مرده در قبضهٔ خدا
&&
بر کف عیسی مدار این هم روا
\\
عیسی‌ام لیکن هر آنکو یافت جان
&&
از دم من او بماند جاودان
\\
شد ز عیسی زنده لیکن باز مرد
&&
شاد آنکو جان بدین عیسی سپرد
\\
من عصاام در کف موسی خویش
&&
موسیم پنهان و من پیدا به پیش
\\
بر مسلمانان پل دریا شوم
&&
باز بر فرعون اژدها شوم
\\
این عصا را ای پسر تنها مبین
&&
که عصا بی‌کف حق نبود چنین
\\
موج طوفان هم عصا بد کو ز درد
&&
طنطنهٔ جادوپرستان را بخورد
\\
گر عصاهای خدا را بشمرم
&&
زرق این فرعونیان را بر درم
\\
لیک زین شیرین گیای زهرمند
&&
ترک کن تا چند روزی می‌چرند
\\
گر نباشد جاه فرعون و سری
&&
از کجا یابد جهنم پروری
\\
فربهش کن آنگهش کش ای قصاب
&&
زانک بی‌برگ‌اند در دوزخ کلاب
\\
گر نبودی خصم و دشمن در جهان
&&
پس بمردی خشم اندر مردمان
\\
دوزخ آن خشمست خصمی بایدش
&&
تا زید ور نی رحیمی بکشدش
\\
پس بماندی لطف بی‌قهر و بدی
&&
پس کمال پادشاهی کی بدی
\\
ریش‌خندی کرده‌اند آن منکران
&&
بر مثلها و بیان ذاکران
\\
تو اگر خواهی بکن هم ریش‌خند
&&
چند خواهی زیست ای مردار چند
\\
شاد باشید ای محبان در نیاز
&&
بر همین در که شود امروز باز
\\
هر حویجی باشدش کردی دگر
&&
در میان باغ از سیر و کبر
\\
هر یکی با جنس خود در کرد خود
&&
از برای پختگی نم می‌خورد
\\
تو که کرد زعفرانی زعفران
&&
باش و آمیزش مکن با دیگران
\\
آب می‌خور زعفرانا تا رسی
&&
زعفرانی اندر آن حلوا رسی
\\
در مکن در کرد شلغم پوز خویش
&&
که نگردد با تو او هم‌طبع و کیش
\\
تو بکردی او بکردی مودعه
&&
زانک ارض الله آمد واسعه
\\
خاصه آن ارضی که از پهناوری
&&
در سفر گم می‌شود دیو و پری
\\
اندر آن بحر و بیابان و جبال
&&
منقطع می‌گردد اوهام و خیال
\\
این بیابان در بیابانهای او
&&
هم‌چو اندر بحر پر یک تای مو
\\
آب استاده که سیرستش نهان
&&
تازه‌تر خوشتر ز جوهای روان
\\
کو درون خویش چون جان و روان
&&
سیر پنهان دارد و پای روان
\\
مستمع خفتست کوته کن خطاب
&&
ای خطیب این نقش کم کن تو بر آب
\\
خیز بلقیسا که بازاریست تیز
&&
زین خسیسان کسادافکن گریز
\\
خیز بلقیسا کنون با اختیار
&&
پیش از آنک مرگ آرد گیر و دار
\\
بعد از آن گوشت کشد مرگ آنچنان
&&
که چو دزد آیی به شحنه جان‌کنان
\\
زین خران تا چند باشی نعل‌دزد
&&
گر همی دزدی بیا و لعل دزد
\\
خواهرانت یافته ملک خلود
&&
تو گرفته ملکت کور و کبود
\\
ای خنک آن را کزین ملکت بجست
&&
که اجل این ملک را ویران‌گرست
\\
خیز بلقیسا بیا باری ببین
&&
ملکت شاهان و سلطانان دین
\\
شسته در باطن میان گلستان
&&
ظاهر آحادی میان دوستان
\\
بوستان با او روان هر جا رود
&&
لیک آن از خلق پنهان می‌شود
\\
میوه‌ها لایه‌کنان کز من بچر
&&
آب حیوان آمده کز من بخور
\\
طوف می‌کن بر فلک بی‌پر و بال
&&
هم‌چو خورشید و چو بدر و چون هلال
\\
چون روان باشی روان و پای نی
&&
می‌خوری صد لوت و لقمه‌خای نی
\\
نی‌نهنگ غم زند بر کشتیت
&&
نی پدید آید ز مردم زشتیت
\\
هم تو شاه و هم تو لشکر هم تو تخت
&&
هم تو نیکوبخت باشی هم تو بخت
\\
گر تو نیکوبختی و سلطان زفت
&&
بخت غیر تست روزی بخت رفت
\\
تو بماندی چون گدایان بی‌نوا
&&
دولت خود هم تو باش ای مجتبی
\\
چون تو باشی بخت خود ای معنوی
&&
پس تو که بختی ز خود کی گم شوی
\\
تو ز خود کی گم شوی از خوش‌خصال
&&
چونک عین تو ترا شد ملک و مال
\\
\end{longtable}
\end{center}
