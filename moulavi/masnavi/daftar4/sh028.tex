\begin{center}
\section*{بخش ۲۸ - نیت کردن او کی این زر بدهم بدان  هیزم‌کش چون من روزی یافتم به کرامات مشایخ و رنجیدن آن هیزم‌کش از ضمیر و نیت او}
\label{sec:sh028}
\addcontentsline{toc}{section}{\nameref{sec:sh028}}
\begin{longtable}{l p{0.5cm} r}
آن یکی درویش هیزم می‌کشید
&&
خسته و مانده ز بیشه در رسید
\\
پس بگفتم من ز روزی فارغم
&&
زین سپس از بهر رزقم نیست غم
\\
میوهٔ مکروه بر من خوش شدست
&&
رزق خاصی جسم را آمد به دست
\\
چونک من فارغ شدستم از گلو
&&
حبه‌ای چندست این بدهم بدو
\\
بدهم این زر را بدین تکلیف‌کش
&&
تا دو سه روزک شود از قوت خوش
\\
خود ضمیرم را همی‌دانست او
&&
زانک سمعش داشت نور از شمع هو
\\
بود پیشش سر هر اندیشه‌ای
&&
چون چراغی در درون شیشه‌ای
\\
هیچ پنهان می‌نشد از وی ضمیر
&&
بود بر مضمون دلها او امیر
\\
پس همی منگید با خود زیر لب
&&
در جواب فکرتم آن بوالعجب
\\
که چنین اندیشی از بهر ملوک
&&
کیف تلقی الرزق ان لم یرزقوک
\\
من نمی‌کردم سخن را فهم لیک
&&
بر دلم می‌زد عتابش نیک نیک
\\
سوی من آمد به هیبت هم‌چو شیر
&&
تنگ هیزم را ز خود بنهاد زیر
\\
پرتو حالی که او هیزم نهاد
&&
لرزه بر هر هفت عضو من فتاد
\\
گفت یا رب گر ترا خاصان هی‌اند
&&
که مبارک‌دعوت و فرخ‌پی‌اند
\\
لطف تو خواهم که میناگر شود
&&
این زمان این تنگ هیزم زر شود
\\
در زمان دیدم که زر شد هیزمش
&&
هم‌چو آتش بر زمین می‌تافت خوش
\\
من در آن بی‌خود شدم تا دیرگه
&&
چونک با خویش آمدم من از وله
\\
بعد از آن گفت ای خداگر آن کبار
&&
بس غیورند و گریزان ز اشتهار
\\
باز این را بند هیزم ساز زود
&&
بی‌توقف هم بر آن حالی که بود
\\
در زمان هیزم شد آن اغصان زر
&&
مست شد در کار او عقل و نظر
\\
بعد از آن برداشت هیزم را و رفت
&&
سوی شهر از پیش من او تیز و تفت
\\
خواستم تا در پی آن شه روم
&&
پرسم از وی مشکلات و بشنوم
\\
بسته کرد آن هیبت او مر مرا
&&
پیش خاصان ره نباشد عامه را
\\
ور کسی را ره شود گو سر فشان
&&
کان بود از رحمت و از جذبشان
\\
پس غنیمت دار آن توفیق را
&&
چون بیابی صحبت صدیق را
\\
نه چو آن ابله که یابد قرب شاه
&&
سهل و آسان در فتد آن دم ز راه
\\
چون ز قربانی دهندش بیشتر
&&
پس بگوید ران گاوست این مگر
\\
نیست این از ران گاو ای مفتری
&&
ران گاوت می‌نماید از خری
\\
بذل شاهانه‌ست این بی رشوتی
&&
بخشش محضست این از رحمتی
\\
\end{longtable}
\end{center}
