\begin{center}
\section*{بخش ۹ - فرستادن پادشاه رسولان به سمرقند به آوردن زرگر}
\label{sec:sh009}
\addcontentsline{toc}{section}{\nameref{sec:sh009}}
\begin{longtable}{l p{0.5cm} r}
شه فرستاد آن طرف یک دو رسول
&&
حاذقان و کافیان بس عدول
\\
تا سمرقند آمدند آن دو امیر
&&
پیش آن زرگر ز شاهنشه بشیر
\\
کای لطیف استاد کامل معرفت
&&
فاش اندر شهرها از تو صفت
\\
نک فلان شه از برای زرگری
&&
اختیارت کرد زیرا مهتری
\\
اینک این خلعت بگیر و زر و سیم
&&
چون بیایی خاص باشی و ندیم
\\
مرد مال و خلعت بسیار دید
&&
غره شد از شهر و فرزندان برید
\\
اندر آمد شادمان در راه مرد
&&
بی‌خبر کان شاه قصد جانش کرد
\\
اسپ تازی برنشست و شاد تاخت
&&
خونبهای خویش را خلعت شناخت
\\
ای شده اندر سفر با صد رضا
&&
خود به پای خویش تا سؤ القضا
\\
در خیالش ملک و عز و مهتری
&&
گفت عزرائیل رو آری بری
\\
چون رسید از راه آن مرد غریب
&&
اندر آوردش به پیش شه طبیب
\\
سوی شاهنشاه بردندش بناز
&&
تا بسوزد بر سر شمع طراز
\\
شاه دید او را بسی تعظیم کرد
&&
مخزن زر را بدو تسلیم کرد
\\
پس حکیمش گفت کای سلطان مه
&&
آن کنیزک را بدین خواجه بده
\\
تا کنیزک در وصالش خوش شود
&&
آب وصلش دفع آن آتش شود
\\
شه بدو بخشید آن مه روی را
&&
جفت کرد آن هر دو صحبت جوی را
\\
مدت شش ماه می‌راندند کام
&&
تا به صحت آمد آن دختر تمام
\\
بعد از آن از بهر او شربت بساخت
&&
تا بخورد و پیش دختر می‌گداخت
\\
چون ز رنجوری جمال او نماند
&&
جان دختر در وبال او نماند
\\
چونک زشت و ناخوش و رخ زرد شد
&&
اندک‌اندک در دل او سرد شد
\\
عشقهایی کز پی رنگی بود
&&
عشق نبود عاقبت ننگی بود
\\
کاش کان هم ننگ بودی یکسری
&&
تا نرفتی بر وی آن بد داوری
\\
خون دوید از چشم همچون جوی او
&&
دشمن جان وی آمد روی او
\\
دشمن طاووس آمد پر او
&&
ای بسی شه را بکشته فر او
\\
گفت من آن آهوم کز ناف من
&&
ریخت این صیاد خون صاف من
\\
ای من آن روباه صحرا کز کمین
&&
سر بریدندش برای پوستین
\\
ای من آن پیلی که زخم پیلبان
&&
ریخت خونم از برای استخوان
\\
آنک کشتستم پی مادون من
&&
می‌نداند که نخسپد خون من
\\
بر منست امروز و فردا بر ویست
&&
خون چون من کس چنین ضایع کیست
\\
گر چه دیوار افکند سایهٔ دراز
&&
باز گردد سوی او آن سایه باز
\\
این جهان کوهست و فعل ما ندا
&&
سوی ما آید نداها را صدا
\\
این بگفت و رفت در دم زیر خاک
&&
آن کنیزک شد ز عشق و رنج پاک
\\
زانک عشق مردگان پاینده نیست
&&
زانک مرده سوی ما آینده نیست
\\
عشق زنده در روان و در بصر
&&
هر دمی باشد ز غنچه تازه‌تر
\\
عشق آن زنده گزین کو باقیست
&&
کز شراب جان‌فزایت ساقیست
\\
عشق آن بگزین که جمله انبیا
&&
یافتند از عشق او کار و کیا
\\
تو مگو ما را بدان شه بار نیست
&&
با کریمان کارها دشوار نیست
\\
\end{longtable}
\end{center}
