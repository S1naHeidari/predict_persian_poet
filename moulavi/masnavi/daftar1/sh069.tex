\begin{center}
\section*{بخش ۶۹ - قصهٔ آدم علیه‌السلام و بستن قضا نظر او را از مراعات صریح نهی و ترک تاویل}
\label{sec:sh069}
\addcontentsline{toc}{section}{\nameref{sec:sh069}}
\begin{longtable}{l p{0.5cm} r}
بوالبشر کو علم الاسما بگست
&&
صد هزاران علمش اندر هر رگست
\\
اسم هر چیزی چنان کان چیز هست
&&
تا به پایان جان او را داد دست
\\
هر لقب کو داد آن مبدل نشد
&&
آنک چستش خواند او کاهل نشد
\\
هر که اول مؤمنست اول بدید
&&
هر که آخر کافر او را شد پدید
\\
اسم هر چیزی تو از دانا شنو
&&
سر رمز علم الاسما شنو
\\
اسم هر چیزی بر ما ظاهرش
&&
اسم هر چیزی بر خالق سرش
\\
نزد موسی نام چوبش بد عصا
&&
نزد خالق بود نامش اژدها
\\
بد عمر را نام اینجا بت‌پرست
&&
لیک مؤمن بود نامش در الست
\\
آنک بد نزدیک ما نامش منی
&&
پیش حق این نقش بد که با منی
\\
صورتی بود این منی اندر عدم
&&
پیش حق موجود نه بیش و نه کم
\\
حاصل آن آمد حقیقت نام ما
&&
پیش حضرت کان بود انجام ما
\\
مرد را بر عاقبت نامی نهد
&&
نی بر آن کو عاریت نامی نهد
\\
چشم آدم چون به نور پاک دید
&&
جان و سر نامها گشتش پدید
\\
چون ملک انوار حق در وی بیافت
&&
در سجود افتاد و در خدمت شتافت
\\
مدح این آدم که نامش می‌برم
&&
قاصرم گر تا قیامت بشمرم
\\
این همه دانست و چون آمد قضا
&&
دانش یک نهی شد بر وی خطا
\\
کای عجب نهی از پی تحریم بود
&&
یا به تاویلی بد و توهیم بود
\\
در دلش تاویل چون ترجیح یافت
&&
طبع در حیرت سوی گندم شتافت
\\
باغبان را خار چون در پای رفت
&&
دزد فرصت یافت کالا برد تفت
\\
چون ز حیرت رست باز آمد به راه
&&
دید برده دزد رخت از کارگاه
\\
ربنا انا ظلمنا گفت و آه
&&
یعنی آمد ظلمت و گم گشت راه
\\
پس قضا ابری بود خورشیدپوش
&&
شیر و اژدرها شود زو همچو موش
\\
من اگر دامی نبینم گاه حکم
&&
من نه تنها جاهلم در راه حکم
\\
ای خنک آن کو نکوکاری گرفت
&&
زور را بگذاشت او زاری گرفت
\\
گر قضا پوشد سیه همچون شبت
&&
هم قضا دستت بگیرد عاقبت
\\
گر قضا صد بار قصد جان کند
&&
هم قضا جانت دهد درمان کند
\\
این قضا صد بار اگر راهت زند
&&
بر فراز چرخ خرگاهت زند
\\
از کرم دان این که می‌ترساندت
&&
تا به ملک ایمنی بنشاندت
\\
این سخن پایان ندارد گشت دیر
&&
گوش کن تو قصهٔ خرگوش و شیر
\\
\end{longtable}
\end{center}
