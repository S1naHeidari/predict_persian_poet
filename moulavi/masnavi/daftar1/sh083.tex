\begin{center}
\section*{بخش ۸۳ - در معنی آنک من اراد ان یجلس مع الله فلیجلس مع اهل التصوف}
\label{sec:sh083}
\addcontentsline{toc}{section}{\nameref{sec:sh083}}
\begin{longtable}{l p{0.5cm} r}
آن رسول از خود بشد زین یک دو جام
&&
نی رسالت یاد ماندش نی پیام
\\
واله اندر قدرت الله شد
&&
آن رسول اینجا رسید و شاه شد
\\
سیل چون آمد به دریا بحر گشت
&&
دانه چون آمد به مزرع گشت کشت
\\
چون تعلق یافت نان با بوالبشر
&&
نان مرده زنده گشت و با خبر
\\
موم و هیزم چون فدای نار شد
&&
ذات ظلمانی او انوار شد
\\
سنگ سرمه چونک شد در دیدگان
&&
گشت بینایی شد آنجا دیدبان
\\
ای خنک آن مرد کز خود رسته شد
&&
در وجود زنده‌ای پیوسته شد
\\
وای آن زنده که با مرده نشست
&&
مرده گشت و زندگی از وی بجست
\\
چون تو در قرآن حق بگریختی
&&
با روان انبیا آمیختی
\\
هست قرآن حالهای انبیا
&&
ماهیان بحر پاک کبریا
\\
ور بخوانی و نه‌ای قرآن‌پذیر
&&
انبیا و اولیا را دیده گیر
\\
ور پذیرایی چو بر خوانی قصص
&&
مرغ جانت تنگ آید در قفس
\\
مرغ کو اندر قفس زندانیست
&&
می‌نجوید رستن از نادانیست
\\
روحهایی کز قفسها رسته‌اند
&&
انبیاء رهبر شایسته‌اند
\\
از برون آوازشان آید ز دین
&&
که ره رستن ترا اینست این
\\
ما بذین رستیم زین تنگین قفس
&&
جز که این ره نیست چارهٔ این قفس
\\
خویش را رنجور سازی زار زار
&&
تا ترا بیرون کنند از اشتهار
\\
که اشتهار خلق بند محکمست
&&
در ره این از بند آهن کی کمست
\\
\end{longtable}
\end{center}
