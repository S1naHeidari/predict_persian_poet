\begin{center}
\section*{بخش ۴۹ - نگریستن عزرائیل بر مردی و گریختن آن مرد در سرای سلیمان و تقریر ترجیح توکل بر جهد و قلت فایدهٔ جهد}
\label{sec:sh049}
\addcontentsline{toc}{section}{\nameref{sec:sh049}}
\begin{longtable}{l p{0.5cm} r}
زاد مردی چاشتگاهی در رسید
&&
در سرا عدل سلیمان در دوید
\\
رویش از غم زرد و هر دو لب کبود
&&
پس سلیمان گفت ای خواجه چه بود
\\
گفت عزرائیل در من این چنین
&&
یک نظر انداخت پر از خشم و کین
\\
گفت هین اکنون چه می‌خواهی بخواه
&&
گفت فرما باد را ای جان پناه
\\
تا مرا زینجا به هندستان برد
&&
بوک بنده کان طرف شد جان برد
\\
نک ز درویشی گریزانند خلق
&&
لقمهٔ حرص و امل زانند خلق
\\
ترس درویشی مثال آن هراس
&&
حرص و کوشش را تو هندستان شناس
\\
باد را فرمود تا او را شتاب
&&
برد سوی قعر هندستان بر آب
\\
روز دیگر وقت دیوان و لقا
&&
پس سلیمان گفت عزرائیل را
\\
کان مسلمان را بخشم از بهر آن
&&
بنگریدی تا شد آواره ز خان
\\
گفت من از خشم کی کردم نظر
&&
از تعجب دیدمش در ره‌گذر
\\
که مرا فرمود حق کامروز هان
&&
جان او را تو بهندستان ستان
\\
از عجب گفتم گر او را صد پرست
&&
او به هندستان شدن دور اندرست
\\
تو همه کار جهان را همچنین
&&
کن قیاس و چشم بگشا و ببین
\\
از کی بگریزیم از خود ای محال
&&
از کی برباییم از حق ای وبال
\\
\end{longtable}
\end{center}
