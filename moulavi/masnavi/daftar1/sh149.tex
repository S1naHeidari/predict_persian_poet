\begin{center}
\section*{بخش ۱۴۹ - گفتن مهمان یوسف علیه‌السلام کی آینه‌ای آوردمت کی تا هر باری کی در وی نگری روی خوب خویش را بینی مرا یاد کنی}
\label{sec:sh149}
\addcontentsline{toc}{section}{\nameref{sec:sh149}}
\begin{longtable}{l p{0.5cm} r}
گفت یوسف هین بیاور ارمغان
&&
او ز شرم این تقاضا زد فغان
\\
گفت من چند ارمغان جستم ترا
&&
ارمغانی در نظر نامد مرا
\\
حبه‌ای را جانب کان چون برم
&&
قطره‌ای را سوی عمان چون برم
\\
زیره را من سوی کرمان آورم
&&
گر به پیش تو دل و جان آورم
\\
نیست تخمی کاندرین انبار نیست
&&
غیر حسن تو که آن را یار نیست
\\
لایق آن دیدم که من آیینه‌ای
&&
پیش تو آرم چو نور سینه‌ای
\\
تا ببینی روی خوب خود در آن
&&
ای تو چون خورشید شمع آسمان
\\
آینه آوردمت ای روشنی
&&
تا چو بینی روی خود یادم کنی
\\
آینه بیرون کشید او از بغل
&&
خوب را آیینه باشد مشتغل
\\
آینهٔ هستی چه باشد نیستی
&&
نیستی بر گر تو ابله نیستی
\\
هستی اندر نیستی بتوان نمود
&&
مال‌داران بر فقیر آرند جود
\\
آینهٔ صافی نان خود گرسنه‌ست
&&
سوخته هم آینهٔ آتش‌زنه‌ست
\\
نیستی و نقص هر جایی که خاست
&&
آینهٔ خوبی جمله پیشه‌هاست
\\
چونک جامه چست و دوزیده بود
&&
مظهر فرهنگ درزی چون شود
\\
ناتراشیده همی باید جذوع
&&
تا دروگر اصل سازد یا فروع
\\
خواجهٔ اشکسته‌بند آنجا رود
&&
کاندر آنجا پای اشکسته بود
\\
کی شود چون نیست رنجور نزار
&&
آن جمال صنعت طب آشکار
\\
خواری و دونی مسها بر ملا
&&
گر نباشد کی نماید کیمیا
\\
نقصها آیینهٔ وصف کمال
&&
و آن حقارت آینهٔ عز و جلال
\\
زانک ضد را ضد کند پیدا یقین
&&
زانک با سر که پدیدست انگبین
\\
هر که نقص خویش را دید و شناخت
&&
اندر استکمال خود ده اسپه تاخت
\\
زان نمی‌پرد به سوی ذوالجلال
&&
کو گمانی می‌برد خود را کمال
\\
علتی بتر ز پندار کمال
&&
نیست اندر جان تو ای ذو دلال
\\
از دل و از دیده‌ات بس خون رود
&&
تا ز تو این معجبی بیرون شود
\\
علت ابلیس انا خیری بدست
&&
وین مرض در نفس هر مخلوق هست
\\
گرچه خود را بس شکسته بیند او
&&
آب صافی دان و سرگین زیر جو
\\
چون بشوراند ترا در امتحان
&&
آب سرگین رنگ گردد در زمان
\\
در تگ جو هست سرگین ای فتی
&&
گرچه جو صافی نماید مر ترا
\\
هست پیر راه‌دان پر فطن
&&
باغهای نفس کل را جوی کن
\\
جوی خود را کی تواند پاک کرد
&&
نافع از علم خدا شد علم مرد
\\
کی تراشد تیغ دستهٔ خویش را
&&
رو به جراحی سپار این ریش را
\\
بر سر هر ریش جمع آمد مگس
&&
تا نبیند قبح ریش خویش کس
\\
آن مگس اندیشه‌ها وان مال تو
&&
ریش تو آن ظلمت احوال تو
\\
ور نهد مرهم بر آن ریش تو پیر
&&
آن زمان ساکن شود درد و نفیر
\\
تا که پندارد که صحت یافتست
&&
پرتو مرهم بر آنجا تافتست
\\
هین ز مرهم سر مکش ای پشت‌ریش
&&
و آن ز پرتو دان مدان از اصل خویش
\\
\end{longtable}
\end{center}
