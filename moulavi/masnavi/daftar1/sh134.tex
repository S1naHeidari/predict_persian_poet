\begin{center}
\section*{بخش ۱۳۴ - در بیان آنک عاشق دنیا بر مثال عاشق دیواریست کی بر و تاب آفتاب زند و جهد و جهاد نکرد تا فهم کند کی آن تاب و رونق از دیوار نیست از قرص آفتابست در آسمان چهارم لاجرم کلی دل بر دیوار نهاد چون پرتو آفتاب بفتاب پیوست او محروم ماند ابدا و حیل بینهم و بین ما یشتهون}
\label{sec:sh134}
\addcontentsline{toc}{section}{\nameref{sec:sh134}}
\begin{longtable}{l p{0.5cm} r}
عاشقان کل نه عشاق جزو
&&
ماند از کل آنک شد مشتاق جزو
\\
چونک جزوی عاشق جزوی شود
&&
زود معشوقش بکل خود رود
\\
ریش گاو و بندهٔ غیر آمد او
&&
غرقه شد کف در ضعیفی در زد او
\\
نیست حاکم تا کند تیمار او
&&
کار خواجهٔ خود کند یا کار او
\\
\end{longtable}
\end{center}
