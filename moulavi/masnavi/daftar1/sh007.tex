\begin{center}
\section*{بخش ۷ - خلوت طلبیدن آن ولی از پادشاه جهت دریافتن رنج کنیزک}
\label{sec:sh007}
\addcontentsline{toc}{section}{\nameref{sec:sh007}}
\begin{longtable}{l p{0.5cm} r}
گفت ای شه خلوتی کن خانه را
&&
دور کن هم خویش و هم بیگانه را
\\
کس ندارد گوش در دهلیزها
&&
تا بپرسم زین کنیزک چیزها
\\
خانه خالی ماند و یک دیار نه
&&
جز طبیب و جز همان بیمار نه
\\
نرم نرمک گفت شهر تو کجاست
&&
که علاج اهل هر شهری جداست
\\
واندر آن شهر از قرابت کیستت
&&
خویشی و پیوستگی با چیستت
\\
دست بر نبضش نهاد و یک بیک
&&
باز می‌پرسید از جور فلک
\\
چون کسی را خار در پایش جهد
&&
پای خود را بر سر زانو نهد
\\
وز سر سوزن همی جوید سرش
&&
ور نیابد می‌کند با لب ترش
\\
خار در پا شد چنین دشواریاب
&&
خار در دل چون بود وا ده جواب
\\
خار در دل گر بدیدی هر خسی
&&
دست کی بودی غمان را بر کسی
\\
کس به زیر دم خر خاری نهد
&&
خر نداند دفع آن بر می‌جهد
\\
بر جهد وان خار محکم‌تر زند
&&
عاقلی باید که خاری برکند
\\
خر ز بهر دفع خار از سوز و درد
&&
جفته می‌انداخت صد جا زخم کرد
\\
آن حکیم خارچین استاد بود
&&
دست می‌زد جابجا می‌آزمود
\\
زان کنیزک بر طریق داستان
&&
باز می‌پرسید حال دوستان
\\
با حکیم او قصه‌ها می‌گفت فاش
&&
از مقام و خواجگان و شهر و باش
\\
سوی قصه گقتنش می‌داشت گوش
&&
سوی نبض و جستنش می‌داشت هوش
\\
تا که نبض از نام کی گردد جهان
&&
او بود مقصود جانش در جهان
\\
دوستان و شهر او را برشمرد
&&
بعد از آن شهری دگر را نام برد
\\
گفت چون بیرون شدی از شهر خویش
&&
در کدامین شهر بودستی تو بیش
\\
نام شهری گفت و زان هم در گذشت
&&
رنگ روی و نبض او دیگر نگشت
\\
خواجگان و شهرها را یک به یک
&&
باز گفت از جای و از نان و نمک
\\
شهر شهر و خانه خانه قصه کرد
&&
نه رگش جنبید و نه رخ گشت زرد
\\
نبض او بر حال خود بد بی‌گزند
&&
تا بپرسید از سمرقند چو قند
\\
نبض جست و روی سرخ و زرد شد
&&
کز سمرقندی زرگر فرد شد
\\
چون ز رنجور آن حکیم این راز یافت
&&
اصل آن درد و بلا را باز یافت
\\
گفت کوی او کدامست در گذر
&&
او سر پل گفت و کوی غاتفر
\\
گفت دانستم که رنجت چیست زود
&&
در خلاصت سحرها خواهم نمود
\\
شاد باش و فارغ و آمن که من
&&
آن کنم با تو که باران با چمن
\\
من غم تو می‌خورم تو غم مخور
&&
بر تو من مشفق‌ترم از صد پدر
\\
هان و هان این راز را با کس مگو
&&
گرچه از تو شه کند بس جست و جو
\\
خانهٔ اسرار تو چون دل شود
&&
آن مرادت زودتر حاصل شود
\\
گفت پیغامبر که هر که سر نهفت
&&
زود گردد با مراد خویش جفت
\\
دانه چون اندر زمین پنهان شود
&&
سر او سرسبزی بستان شود
\\
زر و نقره گر نبودندی نهان
&&
پرورش کی یافتندی زیر کان
\\
وعده‌ها و لطفهای آن حکیم
&&
کرد آن رنجور را آمن ز بیم
\\
وعده‌ها باشد حقیقی دل‌پذیر
&&
وعده‌ها باشد مجازی تا سه گیر
\\
وعدهٔ اهل کرم گنج روان
&&
وعدهٔ نا اهل شد رنج روان
\\
\end{longtable}
\end{center}
