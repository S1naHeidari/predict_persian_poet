\begin{center}
\section*{بخش ۷۷ - آمدن رسول روم تا امیرالممنین عمر رضی‌الله عنه و دیدن او کرامات عمر را رضی‌الله عنه}
\label{sec:sh077}
\addcontentsline{toc}{section}{\nameref{sec:sh077}}
\begin{longtable}{l p{0.5cm} r}
تا عمر آمد ز قیصر یک رسول
&&
در مدینه از بیابان نغول
\\
گفت کو قصر خلیفه ای حشم
&&
تا من اسپ و رخت را آنجا کشم
\\
قوم گفتندش که او را قصر نیست
&&
مر عمر را قصر جان روشنیست
\\
گرچه از میری ورا آوازه‌ایست
&&
همچو درویشان مر او را کازه‌ایست
\\
ای برادر چون ببینی قصر او
&&
چونک در چشم دلت رستست مو
\\
چشم دل از مو و علت پاک آر
&&
وانگه آن دیدار قصرش چشم دار
\\
هر که را هست از هوسها جان پاک
&&
زود بیند حضرت و ایوان پاک
\\
چون محمد پاک شد زین نار و دود
&&
هر کجا رو کرد وجه الله بود
\\
چون رفیقی وسوسهٔ بدخواه را
&&
کی بدانی ثم وجه الله را
\\
هر که را باشد ز سینه فتح باب
&&
بیند او بر چرخ دل صد آفتاب
\\
حق پدیدست از میان دیگران
&&
همچو ماه اندر میان اختران
\\
دو سر انگشت بر دو چشم نه
&&
هیچ بینی از جهان انصاف ده
\\
گر نبینی این جهان معدوم نیست
&&
عیب جز ز انگشت نفس شوم نیست
\\
تو ز چشم انگشت را بر دار هین
&&
وانگهانی هرچه می‌خواهی ببین
\\
نوح را گفتند امت کو ثواب
&&
گفت او زان سوی واستغشوا ثیاب
\\
رو و سر در جامه‌ها پیچیده‌اید
&&
لاجرم با دیده و نادیده‌اید
\\
آدمی دیدست و باقی پوستست
&&
دید آنست آن که دید دوستست
\\
چونک دید دوست نبود کور به
&&
دوست کو باقی نباشد دور به
\\
چون رسول روم این الفاظ تر
&&
در سماع آورد شد مشتاق‌تر
\\
دیده را بر جستن عمر گماشت
&&
رخت را و اسپ را ضایع گذاشت
\\
هر طرف اندر پی آن مرد کار
&&
می‌شدی پرسان او دیوانه‌وار
\\
کین چنین مردی بود اندر جهان
&&
وز جهان مانند جان باشد نهان
\\
جست او را تاش چون بنده بود
&&
لاجرم جوینده یابنده بود
\\
دید اعرابی زنی او را دخیل
&&
گفت عمر نک به زیر آن نخیل
\\
زیر خرمابن ز خلقان او جدا
&&
زیر سایه خفته بین سایهٔ خدا
\\
\end{longtable}
\end{center}
