\begin{center}
\section*{بخش ۸۴ - قصهٔ بازرگان کی طوطی محبوس او او را پیغام داد به طوطیان هندوستان هنگام رفتن به تجارت}
\label{sec:sh084}
\addcontentsline{toc}{section}{\nameref{sec:sh084}}
\begin{longtable}{l p{0.5cm} r}
بود بازرگان و او را طوطیی
&&
در قفس محبوس زیبا طوطیی
\\
چونک بازرگان سفر را ساز کرد
&&
سوی هندستان شدن آغاز کرد
\\
هر غلام و هر کنیزک را ز جود
&&
گفت بهر تو چه آرم گوی زود
\\
هر یکی از وی مرادی خواست کرد
&&
جمله را وعده بداد آن نیک مرد
\\
گفت طوطی را چه خواهی ارمغان
&&
کارمت از خطهٔ هندوستان
\\
گفتش آن طوطی که آنجا طوطیان
&&
چون ببینی کن ز حال من بیان
\\
کان فلان طوطی که مشتاق شماست
&&
از قضای آسمان در حبس ماست
\\
بر شما کرد او سلام و داد خواست
&&
وز شما چاره و ره ارشاد خواست
\\
گفت می‌شاید که من در اشتیاق
&&
جان دهم اینجا بمیرم در فراق
\\
این روا باشد که من در بند سخت
&&
گه شما بر سبزه گاهی بر درخت
\\
این چنین باشد وفای دوستان
&&
من درین حبس و شما در گلستان
\\
یاد آرید ای مهان زین مرغ زار
&&
یک صبوحی درمیان مرغزار
\\
یاد یاران یار را میمون بود
&&
خاصه کان لیلی و این مجنون بود
\\
ای حریفان بت موزون خود
&&
من قدحها می‌خورم پر خون خود
\\
یک قدح می‌نوش کن بر یاد من
&&
گر نمی‌خواهی که بدهی داد من
\\
یا بیاد این فتادهٔ خاک‌بیز
&&
چونک خوردی جرعه‌ای بر خاک ریز
\\
ای عجب آن عهد و آن سوگند کو
&&
وعده‌های آن لب چون قند کو
\\
گر فراق بنده از بد بندگیست
&&
چون تو با بد بد کنی پس فرق چیست
\\
ای بدی که تو کنی در خشم و جنگ
&&
با طرب‌تر از سماع و بانگ چنگ
\\
ای جفای تو ز دولت خوب‌تر
&&
و انتقام تو ز جان محبوب‌تر
\\
نار تو اینست نورت چون بود
&&
ماتم این تا خود که سورت چون بود
\\
از حلاوتها که دارد جور تو
&&
وز لطافت کس نیابد غور تو
\\
نالم و ترسم که او باور کند
&&
وز کرم آن جور را کمتر کند
\\
عاشقم بر قهر و بر لطفش بجد
&&
بوالعجب من عاشق این هر دو ضد
\\
والله ار زین خار در بستان شوم
&&
همچو بلبل زین سبب نالان شوم
\\
این عجب بلبل که بگشاید دهان
&&
تا خورد او خار را با گلستان
\\
این چه بلبل این نهنگ آتشیست
&&
جمله ناخوشها ز عشق او را خوشیست
\\
عاشق کلست و خود کلست او
&&
عاشق خویشست و عشق خویش‌جو
\\
\end{longtable}
\end{center}
