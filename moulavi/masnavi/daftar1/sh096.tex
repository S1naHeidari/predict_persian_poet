\begin{center}
\section*{بخش ۹۶ - تفسیر ما شاء الله کان}
\label{sec:sh096}
\addcontentsline{toc}{section}{\nameref{sec:sh096}}
\begin{longtable}{l p{0.5cm} r}
این همه گفتیم لیک اندر بسیچ
&&
بی‌عنایات خدا هیچیم هیچ
\\
بی عنایات حق و خاصان حق
&&
گر ملک باشد سیاهستش ورق
\\
ای خدا ای فضل تو حاجت روا
&&
با تو یاد هیچ کس نبود روا
\\
این قدر ارشاد تو بخشیده‌ای
&&
تا بدین بس عیب ما پوشیده‌ای
\\
قطرهٔ دانش که بخشیدی ز پیش
&&
متصل گردان به دریاهای خویش
\\
قطرهٔ علمست اندر جان من
&&
وارهانش از هوا وز خاک تن
\\
پیش از آن کین خاکها خسفش کنند
&&
پیش از آن کین بادها نشفش کنند
\\
گر چه چون نشفش کند تو قادری
&&
کش ازیشان وا ستانی وا خری
\\
قطره‌ای کو در هوا شد یا که ریخت
&&
از خزینهٔ قدرت تو کی گریخت
\\
گر در آید در عدم یا صد عدم
&&
چون بخوانیش او کند از سر قدم
\\
صد هزاران ضد ضد را می‌کشد
&&
بازشان حکم تو بیرون می‌کشد
\\
از عدمها سوی هستی هر زمان
&&
هست یا رب کاروان در کاروان
\\
خاصه هر شب جمله افکار و عقول
&&
نیست گردد غرق در بحر نغول
\\
باز وقت صبح آن اللهیان
&&
بر زنند از بحر سر چون ماهیان
\\
در خزان آن صد هزاران شاخ و برگ
&&
از هزیمت رفته در دریای مرگ
\\
زاغ پوشیده سیه چون نوحه‌گر
&&
در گلستان نوحه کرده بر خضر
\\
باز فرمان آید از سالار ده
&&
مر عدم را کانچ خوردی باز ده
\\
آنچ خوردی وا ده ای مرگ سیاه
&&
از نبات و دارو و برگ و گیاه
\\
ای برادر عقل یکدم با خود آر
&&
دم بدم در تو خزانست و بهار
\\
باغ دل را سبز و تر و تازه بین
&&
پر ز غنچه و ورد و سرو و یاسمین
\\
ز انبهی برگ پنهان گشته شاخ
&&
ز انبهی گل نهان صحرا و کاخ
\\
این سخنهایی که از عقل کلست
&&
بوی آن گلزار و سرو و سنبلست
\\
بوی گل دیدی که آنجا گل نبود
&&
جوش مل دیدی که آنجا مل نبود
\\
بو قلاووزست و رهبر مر ترا
&&
می‌برد تا خلد و کوثر مر ترا
\\
بو دوای چشم باشد نورساز
&&
شد ز بویی دیدهٔ یعقوب باز
\\
بوی بد مر دیده را تاری کند
&&
بوی یوسف دیده را یاری کند
\\
تو که یوسف نیستی یعقوب باش
&&
همچو او با گریه و آشوب باش
\\
بشنو این پند از حکیم غزنوی
&&
تا بیابی در تن کهنه نوی
\\
ناز را رویی بباید همچو ورد
&&
چون نداری گرد بدخویی مگرد
\\
زشت باشد روی نازیبا و ناز
&&
سخت باشد چشم نابینا و درد
\\
پیش یوسف نازش و خوبی مکن
&&
جز نیاز و آه یعقوبی مکن
\\
معنی مردن ز طوطی بد نیاز
&&
در نیاز و فقر خود را مرده ساز
\\
تا دم عیسی ترا زنده کند
&&
همچو خویشت خوب و فرخنده کند
\\
از بهاران کی شود سرسبز سنگ
&&
خاک شو تا گل نمایی رنگ رنگ
\\
سالها تو سنگ بودی دل‌خراش
&&
آزمون را یک زمانی خاک باش
\\
\end{longtable}
\end{center}
