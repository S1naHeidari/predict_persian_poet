\begin{center}
\section*{بخش ۱۴۳ - امتحان کردن شیر گرگ را و گفتن کی پیش آی ای گرگ بخش کن صیدها را میان ما}
\label{sec:sh143}
\addcontentsline{toc}{section}{\nameref{sec:sh143}}
\begin{longtable}{l p{0.5cm} r}
گفت شیر ای گرگ این را بخش کن
&&
معدلت را نو کن ای گرگ کهن
\\
نایب من باش در قسمت‌گری
&&
تا پدید آید که تو چه گوهری
\\
گفت ای شه گاو وحشی بخش تست
&&
آن بزرگ و تو بزرگ و زفت و چست
\\
بز مرا که بز میانه‌ست و وسط
&&
روبها خرگوش بستان بی غلط
\\
شیر گفت ای گرگ چون گفتی بگو
&&
چونک من باشم تو گویی ما و تو
\\
گرگ خود چه سگ بود کو خویش دید
&&
پیش چون من شیر بی مثل و ندید
\\
گفت پیش آ ای خری کو خود خرید
&&
پیشش آمد پنجه زد او را درید
\\
چون ندیدش مغز و تدبیر رشید
&&
در سیاست پوستش از سر کشید
\\
گفت چون دید منت ز خود نبرد
&&
این چنین جان را بباید زار مرد
\\
چون نبودی فانی اندر پیش من
&&
فضل آمد مر ترا گردن زدن
\\
کل شیء هالک جز وجه او
&&
چون نه‌ای در وجه او هستی مجو
\\
هر که اندر وجه ما باشد فنا
&&
کل شیء هالک نبود جزا
\\
زانک در الاست او از لا گذشت
&&
هر که در الاست او فانی نگشت
\\
هر که بر در او من و ما می‌زند
&&
رد بابست او و بر لا می‌تند
\\
\end{longtable}
\end{center}
