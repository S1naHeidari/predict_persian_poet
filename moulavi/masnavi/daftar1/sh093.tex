\begin{center}
\section*{بخش ۹۳ - برون انداختن مرد تاجر طوطی را از قفص و پریدن طوطی مرده}
\label{sec:sh093}
\addcontentsline{toc}{section}{\nameref{sec:sh093}}
\begin{longtable}{l p{0.5cm} r}
بعد از آنش از قفس بیرون فکند
&&
طوطیک پرید تا شاخ بلند
\\
طوطی مرده چنان پرواز کرد
&&
کآفتاب شرق ترکی‌تاز کرد
\\
خواجه حیران گشت اندر کار مرغ
&&
بی‌خبر ناگه بدید اسرار مرغ
\\
روی بالا کرد و گفت ای عندلیب
&&
از بیان حال خودمان ده نصیب
\\
او چه کرد آنجا که تو آموختی
&&
ساختی مکری و ما را سوختی
\\
گفت طوطی کو به فعلم پند داد
&&
که رها کن لطف آواز و وداد
\\
زانک آوازت ترا در بند کرد
&&
خویشتن مرده پی این پند کرد
\\
یعنی ای مطرب شده با عام و خاص
&&
مرده شو چون من که تا یابی خلاص
\\
دانه باشی مرغکانت بر چنند
&&
غنچه باشی کودکانت بر کنند
\\
دانه پنهان کن بکلی دام شو
&&
غنچه پنهان کن گیاه بام شو
\\
هر که داد او حسن خود را در مزاد
&&
صد قضای بد سوی او رو نهاد
\\
جشمها و خشمها و رشکها
&&
بر سرش ریزد چو آب از مشکها
\\
دشمنان او را ز غیرت می‌درند
&&
دوستان هم روزگارش می‌برند
\\
آنک غافل بود از کشت و بهار
&&
او چه داند قیمت این روزگار
\\
در پناه لطف حق باید گریخت
&&
کو هزاران لطف بر ارواح ریخت
\\
تا پناهی یابی آنگه چون پناه
&&
آب و آتش مر ترا گردد سپاه
\\
نوح و موسی را نه دریا یار شد
&&
نه بر اعداشان بکین قهار شد
\\
آتش ابراهیم را نه قلعه بود
&&
تا برآورد از دل نمرود دود
\\
کوه یحیی را نه سوی خویش خواند
&&
قاصدانش را به زخم سنگ راند
\\
گفت ای یحیی بیا در من گریز
&&
تا پناهت باشم از شمشیر تیز
\\
\end{longtable}
\end{center}
