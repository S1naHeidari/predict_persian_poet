\begin{center}
\section*{بخش ۱۳۳ - پیش آمدن نقیبان و دربانان خلیفه از بهر اکرام اعرابی و پذیرفتن هدیهٔ او را}
\label{sec:sh133}
\addcontentsline{toc}{section}{\nameref{sec:sh133}}
\begin{longtable}{l p{0.5cm} r}
آن عرابی از بیابان بعید
&&
بر در دار الخلافه چون رسید
\\
پس نقیبان پیش او باز آمدند
&&
بس گلاب لطف بر جیبش زدند
\\
حاجت او فهمشان شد بی مقال
&&
کار ایشان بد عطا پیش از سئوال
\\
پس بدو گفتند یا وجه العرب
&&
از کجایی چونی از راه و تعب
\\
گفت وجهم گر مرا وجهی دهید
&&
بی وجوهم چون پس پشتم نهید
\\
ای که در روتان نشان مهتری
&&
فرتان خوشتر ز زر جعفری
\\
ای که یک دیدارتان دیدارها
&&
ای نثار دینتان دینارها
\\
ای همه ینظر بنور الله شده
&&
بهر بخشش از بر شه آمده
\\
تا زنید آن کیمیاهای نظر
&&
بر سر مسهای اشخاص بشر
\\
من غریبم از بیابان آمدم
&&
بر امید لطف سلطان آمدم
\\
بوی لطف او بیابانها گرفت
&&
ذره‌های ریگ هم جانها گرفت
\\
تا بدین جا بهر دینار آمدم
&&
چون رسیدم مست دیدار آمدم
\\
بهر نان شخصی سوی نانبا دوید
&&
داد جان چون حسن نانبا را بدید
\\
بهر فرجه شد یکی تا گلستان
&&
فرجهٔ او شد جمال باغبان
\\
همچو اعرابی که آب از چه کشید
&&
آب حیوان از رخ یوسف چشید
\\
رفت موسی کآتش آرد او بدست
&&
آتشی دید او که از آتش برست
\\
جست عیسی تا رهد از دشمنان
&&
بردش آن جستن به چارم آسمان
\\
دام آدم خوشهٔ گندم شده
&&
تا وجودش خوشهٔ مردم شده
\\
باز آید سوی دام از بهر خور
&&
ساعد شه یابد و اقبال و فر
\\
طفل شد مکتب پی کسب هنر
&&
بر امید مرغ با لطف پدر
\\
پس ز مکتب آن یکی صدری شده
&&
ماهگانه داده و بدری شده
\\
آمده عباس حرب از بهر کین
&&
بهر قمع احمد و استیز دین
\\
گشته دین را تا قیامت پشت و رو
&&
در خلافت او و فرزندان او
\\
من برین در طالب چیز آمدم
&&
صدر گشتم چون به دهلیز آمدم
\\
آب آوردم به تحفه بهر نان
&&
بوی نانم برد تا صدر جنان
\\
نان برون راند آدمی را از بهشت
&&
نان مرا اندر بهشتی در سرشت
\\
رستم از آب و ز نان همچون ملک
&&
بی‌غرض گردم برین در چون فلک
\\
بی‌غرض نبود بگردش در جهان
&&
غیر جسم و غیر جان عاشقان
\\
\end{longtable}
\end{center}
