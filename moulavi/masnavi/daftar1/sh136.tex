\begin{center}
\section*{بخش ۱۳۶ - سپردن عرب هدیه را یعنی سبو را به  غلامان خلیفه}
\label{sec:sh136}
\addcontentsline{toc}{section}{\nameref{sec:sh136}}
\begin{longtable}{l p{0.5cm} r}
آن سبوی آب را در پیش داشت
&&
تخم خدمت رادر آن حضرت بکاشت
\\
گفت این هدیه بدان سلطان برید
&&
سایل شه را ز حاجت وا خرید
\\
آب شیرین و سبوی سبز و نو
&&
ز آب بارانی که جمع آمد بگو
\\
خنده می‌آمد نقیبان را از آن
&&
لیک پذرفتند آن را همچو جان
\\
زانک لطف شاه خوب با خبر
&&
کرده بود اندر همه ارکان اثر
\\
خوی شاهان در رعیت جا کند
&&
چرخ اخضر خاک را خضرا کند
\\
شه چو حوضی دان حشم چون لوله‌ها
&&
آب از لوله روان در گوله‌ها
\\
چونک آب جمله از حوضیست پاک
&&
هر یکی آبی دهد خوش ذوقناک
\\
ور در آن حوض آب شورست و پلید
&&
هر یکی لوله همان آرد پدید
\\
زانک پیوستست هر لوله به حوض
&&
خوض کن در معنی این حرف خوض
\\
لطف شاهنشاه جان بی‌وطن
&&
چون اثر کردست اندر کل تن
\\
لطف عقل خوش‌نهاد خوش‌نسب
&&
چون همه تن را در آرد در ادب
\\
عشق شنگ بی‌قرار بی سکون
&&
چون در آرد کل تن را در جنون
\\
لطف آب بحر کو چون کوثرست
&&
سنگ‌ریزه‌ش جمله در و گوهرست
\\
هر هنر که استا بدان معروف شد
&&
جان شاگردان بدان موصوف شد
\\
پیش استاد اصولی هم اصول
&&
خواند آن شاگرد چست با حصول
\\
پیش استاد فقیه آن فقه‌خوان
&&
فقه خواند نه اصول اندر بیان
\\
پیش استادی که او نحوی بود
&&
جان شاگردش ازو نحوی شود
\\
باز استادی که او محو رهست
&&
جان شاگردش ازو محو شهست
\\
زین همه انواع دانش روز مرگ
&&
دانش فقرست ساز راه و برگ
\\
\end{longtable}
\end{center}
