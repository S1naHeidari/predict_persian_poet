\begin{center}
\section*{بخش ۴۷ - ترجیح نهادن شیر جهد را بر توکل}
\label{sec:sh047}
\addcontentsline{toc}{section}{\nameref{sec:sh047}}
\begin{longtable}{l p{0.5cm} r}
گفت شیر آری ولی رب العباد
&&
نردبانی پیش پای ما نهاد
\\
پایه پایه رفت باید سوی بام
&&
هست جبری بودن اینجا طمع خام
\\
پای داری چون کنی خود را تو لنگ
&&
دست داری چون کنی پنهان تو چنگ
\\
خواجه چون بیلی به دست بنده داد
&&
بی زبان معلوم شد او را مراد
\\
دست همچون بیل اشارتهای اوست
&&
آخراندیشی عبارتهای اوست
\\
چون اشارتهاش را بر جان نهی
&&
در وفای آن اشارت جان دهی
\\
پس اشارتهای اسرارت دهد
&&
بار بر دارد ز تو کارت دهد
\\
حاملی محمول گرداند ترا
&&
قابلی مقبول گرداند ترا
\\
قابل امر ویی قایل شوی
&&
وصل جویی بعد از آن واصل شوی
\\
سعی شکر نعمتش قدرت بود
&&
جبر تو انکار آن نعمت بود
\\
شکر قدرت قدرتت افزون کند
&&
جبر نعمت از کفت بیرون کند
\\
جبر تو خفتن بود در ره مخسپ
&&
تا نبینی آن در و درگه مخسپ
\\
هان مخسپ ای کاهل بی‌اعتبار
&&
جز به زیر آن درخت میوه‌دار
\\
تا که شاخ افشان کند هر لحظه باد
&&
بر سر خفته بریزد نقل و زاد
\\
جبر و خفتن درمیان ره‌زنان
&&
مرغ بی‌هنگام کی یابد امان
\\
ور اشارتهاش را بینی زنی
&&
مرد پنداری و چون بینی زنی
\\
این قدر عقلی که داری گم شود
&&
سر که عقل از وی بپرد دم شود
\\
زانک بی‌شکری بود شوم و شنار
&&
می‌برد بی‌شکر را در قعر نار
\\
گر توکل می‌کنی در کار کن
&&
کشت کن پس تکیه بر جبار کن
\\
\end{longtable}
\end{center}
