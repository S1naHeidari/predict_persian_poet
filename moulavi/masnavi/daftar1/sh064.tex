\begin{center}
\section*{بخش ۶۴ - عذر گفتن خرگوش}
\label{sec:sh064}
\addcontentsline{toc}{section}{\nameref{sec:sh064}}
\begin{longtable}{l p{0.5cm} r}
گفت خرگوش الامان عذریم هست
&&
گر دهد عفو خداوندیت دست
\\
گفت چه عذر ای قصور ابلهان
&&
این زمان آیند در پیش شهان
\\
مرغ بی‌وقتی سرت باید برید
&&
عذر احمق را نمی‌شاید شنید
\\
عذر احمق بتر از جرمش بود
&&
عذر نادان زهر هر دانش بود
\\
عذرت ای خرگوش از دانش تهی
&&
من نه خرگوشم که در گوشم نهی
\\
گفت ای شه ناکسی را کس شمار
&&
عذر استم دیده‌ای را گوش دار
\\
خاص از بهر زکات جاه خود
&&
گمرهی را تو مران از راه خود
\\
بحر کو آبی به هر جو می‌دهد
&&
هر خسی را بر سر و رو می‌نهد
\\
کم نخواهد گشت دریا زین کرم
&&
از کرم دریا نگردد بیش و کم
\\
گفت دارم من کرم بر جای او
&&
جامهٔ هر کس برم بالای او
\\
گفت بشنو گر نباشم جای لطف
&&
سر نهادم پیش اژدرهای عنف
\\
من بوقت چاشت در راه آمدم
&&
با رفیق خود سوی شاه آمدم
\\
با من از بهر تو خرگوشی دگر
&&
جفت و همره کرده بودند آن نفر
\\
شیری اندر راه قصد بنده کرد
&&
قصد هر دو همره آینده کرد
\\
گفتمش ما بنده شاهنشهیم
&&
خواجه تاشان که آن درگهیم
\\
گفت شاهنشه کی باشد شرم دار
&&
پیش من تو یاد هر ناکس میار
\\
هم ترا و هم شهت را بر درم
&&
گر تو با یارت بگردید از درم
\\
گفتمش بگذار تا بار دگر
&&
روی شه بینم برم از تو خبر
\\
گفت همره را گرو نه پیش من
&&
ور نه قربانی تو اندر کیش من
\\
لابه کردیمش بسی سودی نکرد
&&
یار من بستد مرا بگذاشت فرد
\\
یارم از زفتی دو چندان بد که من
&&
هم بلطف و هم بخوبی هم بتن
\\
بعد ازین زان شیر این ره بسته شد
&&
حال ما این بود و با تو گفته شد
\\
از وظیفه بعد ازین اومید بر
&&
حق همی گویم ترا والحق مر
\\
گر وظیفه بایدت ره پاک کن
&&
هین بیا و دفع آن بی‌باک کن
\\
\end{longtable}
\end{center}
