\begin{center}
\section*{بخش ۳۹ - کژ ماندن دهان آن مرد کی نام محمد را صلی‌الله علیه و سلم بتسخر خواند}
\label{sec:sh039}
\addcontentsline{toc}{section}{\nameref{sec:sh039}}
\begin{longtable}{l p{0.5cm} r}
آن دهان کژ کرد و از تسخر بخواند
&&
مر محمد را دهانش کژ بماند
\\
باز آمد کای محمد عفو کن
&&
ای ترا الطاف و علم من لدن
\\
من ترا افسوس می‌کردم ز جهل
&&
من بدم افسوس را منسوب و اهل
\\
چون خدا خواهد که پردهٔ کس درد
&&
میلش اندر طعنهٔ پاکان برد
\\
ور خدا خواهد که پوشد عیب کس
&&
کم زند در عیب معیوبان نفس
\\
چون خدا خواهد که‌مان یاری کند
&&
میل ما را جانب زاری کند
\\
ای خنک چشمی که آن گریان اوست
&&
وی همایون دل که آن بریان اوست
\\
آخر هر گریه آخر خنده‌ایست
&&
مرد آخربین مبارک بنده‌ایست
\\
هر کجا آب روان سبزه بود
&&
هر کجا اشکی روان رحمت شود
\\
باش چون دولاب نالان چشم تر
&&
تا ز صحن جانت بر روید خضر
\\
اشک خواهی رحم کن بر اشک‌بار
&&
رحم خواهی بر ضعیفان رحم آر
\\
\end{longtable}
\end{center}
