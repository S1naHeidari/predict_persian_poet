\begin{center}
\section*{بخش ۵۰ - باز ترجیح نهادن شیر جهد را بر توکل و فواید جهد را بیان کردن}
\label{sec:sh050}
\addcontentsline{toc}{section}{\nameref{sec:sh050}}
\begin{longtable}{l p{0.5cm} r}
شیر گفت آری ولیکن هم ببین
&&
جهدهای انبیا و مؤمنین
\\
حق تعالی جهدشان را راست کرد
&&
آنچ دیدند از جفا و گرم و سرد
\\
حیله‌هاشان جمله حال آمد لطیف
&&
کل شیء من ظریف هو ظریف
\\
دامهاشان مرغ گردونی گرفت
&&
نقصهاشان جمله افزونی گرفت
\\
جهد می‌کن تا توانی ای کیا
&&
در طریق انبیاء و اولیا
\\
با قضا پنجه زدن نبود جهاد
&&
زانک این را هم قضا بر ما نهاد
\\
کافرم من گر زیان کردست کس
&&
در ره ایمان و طاعت یک نفس
\\
سر شکسته نیست این سر را مبند
&&
یک دو روزک جهد کن باقی بخند
\\
بد محالی جست کو دنیا بجست
&&
نیک حالی جست کو عقبی بجست
\\
مکرها در کسب دنیا باردست
&&
مکرها در ترک دنیا واردست
\\
مکر آن باشد که زندان حفره کرد
&&
آنک حفره بست آن مکریست سرد
\\
این جهان زندان و ما زندانیان
&&
حفره‌کن زندان و خود را وا رهان
\\
چیست دنیا از خدا غافل بدن
&&
نه قماش و نقده و میزان و زن
\\
مال را کز بهر دین باشی حمول
&&
نعم مال صالح خواندش رسول
\\
آب در کشتی هلاک کشتی است
&&
آب اندر زیر کشتی پشتی است
\\
چونک مال و ملک را از دل براند
&&
زان سلیمان خویش جز مسکین نخواند
\\
کوزهٔ سربسته اندر آب زفت
&&
از دل پر باد فوق آب رفت
\\
باد درویشی چو در باطن بود
&&
بر سر آب جهان ساکن بود
\\
گر چه جملهٔ این جهان ملک ویست
&&
ملک در چشم دل او لاشی‌ست
\\
پس دهان دل ببند و مهر کن
&&
پر کنش از باد کبر من لدن
\\
جهد حقست و دوا حقست و درد
&&
منکر اندر نفی جهدش جهد کرد
\\
\end{longtable}
\end{center}
