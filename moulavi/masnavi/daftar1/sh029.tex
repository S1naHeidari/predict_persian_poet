\begin{center}
\section*{بخش ۲۹ - اعتراض مریدان در خلوت وزیر}
\label{sec:sh029}
\addcontentsline{toc}{section}{\nameref{sec:sh029}}
\begin{longtable}{l p{0.5cm} r}
جمله گفتند ای وزیر انکار نیست
&&
گفت ما چون گفتن اغیار نیست
\\
اشک دیده‌ست از فراق تو دوان
&&
آه آهست از میان جان روان
\\
طفل با دایه نه استیزد ولیک
&&
گرید او گر چه نه بد داند نه نیک
\\
ما چو چنگیم و تو زخمه می‌زنی
&&
زاری از ما نه تو زاری می‌کنی
\\
ما چو ناییم و نوا در ما ز تست
&&
ما چو کوهیم و صدا در ما ز تست
\\
ما چو شطرنجیم اندر برد و مات
&&
برد و مات ما ز تست ای خوش صفات
\\
ما که باشیم ای تو ما را جان جان
&&
تا که ما باشیم با تو درمیان
\\
ما عدمهاییم و هستیهای ما
&&
تو وجود مطلقی فانی‌نما
\\
ما همه شیران ولی شیر علم
&&
حمله‌شان از باد باشد دم‌بدم
\\
حمله‌شان پیداست و ناپیداست باد
&&
آنک ناپیداست هرگز گم مباد
\\
باد ما و بود ما از داد تست
&&
هستی ما جمله از ایجاد تست
\\
لذت هستی نمودی نیست را
&&
عاشق خود کرده بودی نیست را
\\
لذت انعام خود را وامگیر
&&
نقل و باده و جام خود را وا مگیر
\\
ور بگیری کیت جست و جو کند
&&
نقش با نقاش چون نیرو کند
\\
منگر اندر ما مکن در ما نظر
&&
اندر اکرام و سخای خود نگر
\\
ما نبودیم و تقاضامان نبود
&&
لطف تو ناگفتهٔ ما می‌شنود
\\
نقش باشد پیش نقاش و قلم
&&
عاجز و بسته چو کودک در شکم
\\
پیش قدرت خلق جمله بارگه
&&
عاجزان چون پیش سوزن کارگه
\\
گاه نقشش دیو و گه آدم کند
&&
گاه نقشش شادی و گه غم کند
\\
دست نه تا دست جنباند به دفع
&&
نطق نه تا دم زند در ضر و نفع
\\
تو ز قرآن بازخوان تفسیر بیت
&&
گفت ایزد ما رمیت اذ رمیت
\\
گر بپرانیم تیر آن نه ز ماست
&&
ما کمان و تیراندازش خداست
\\
این نه جبر این معنی جباریست
&&
ذکر جباری برای زاریست
\\
زاری ما شد دلیل اضطرار
&&
خجلت ما شد دلیل اختیار
\\
گر نبودی اختیار این شرم چیست
&&
وین دریغ و خجلت و آزرم چیست
\\
زجر شاگردان و استادان چراست
&&
خاطر از تدبیرها گردان چراست
\\
ور تو گویی غافلست از جبر او
&&
ماه حق پنهان کند در ابر رو
\\
هست این را خوش جواب ار بشنوی
&&
بگذری از کفر و در دین بگروی
\\
حسرت و زاری گه بیماریست
&&
وقت بیماری همه بیداریست
\\
آن زمان که می‌شوی بیمار تو
&&
می‌کنی از جرم استغفار تو
\\
می‌نماید بر تو زشتی گنه
&&
می‌کنی نیت که باز آیم به ره
\\
عهد و پیمان می‌کنی که بعد ازین
&&
جز که طاعت نبودم کاری گزین
\\
پس یقین گشت این که بیماری ترا
&&
می‌ببخشد هوش و بیداری ترا
\\
پس بدان این اصل را ای اصل‌جو
&&
هر که را دردست او بردست بو
\\
هر که او بیدارتر پر دردتر
&&
هر که او آگاه تر رخ زردتر
\\
گر ز جبرش آگهی زاریت کو
&&
بینش زنجیر جباریت کو
\\
بسته در زنجیر چون شادی کند
&&
کی اسیر حبس آزادی کند
\\
ور تو می‌بینی که پایت بسته‌اند
&&
بر تو سرهنگان شه بنشسته‌اند
\\
پس تو سرهنگی مکن با عاجزان
&&
زانک نبود طبع و خوی عاجز آن
\\
چون تو جبر او نمی‌بینی مگو
&&
ور همی بینی نشان دید کو
\\
در هر آن کاری که میلستت بدان
&&
قدرت خود را همی بینی عیان
\\
واندر آن کاری که میلت نیست و خواست
&&
خویش را جبری کنی کین از خداست
\\
انبیا در کار دنیا جبری‌اند
&&
کافران در کار عقبی جبری‌اند
\\
انبیا را کار عقبی اختیار
&&
جاهلان را کار دنیا اختیار
\\
زانک هر مرغی بسوی جنس خویش
&&
می‌پرد او در پس و جان پیش پیش
\\
کافران چون جنس سجین آمدند
&&
سجن دنیا را خوش آیین آمدند
\\
انبیا چون جنس علیین بدند
&&
سوی علیین جان و دل شدند
\\
این سخن پایان ندارد لیک ما
&&
باز گوییم آن تمام قصه را
\\
\end{longtable}
\end{center}
