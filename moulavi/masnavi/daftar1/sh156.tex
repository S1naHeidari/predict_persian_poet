\begin{center}
\section*{بخش ۱۵۶ - در بیان آنک حال خود و مستی خود پنهان باید داشت از جاهلان}
\label{sec:sh156}
\addcontentsline{toc}{section}{\nameref{sec:sh156}}
\begin{longtable}{l p{0.5cm} r}
بشنو الفاظ حکیم پرده‌ای
&&
سر همانجا نه که باده خورده‌ای
\\
چونک از میخانه مستی ضال شد
&&
تسخر و بازیچهٔ اطفال شد
\\
می‌فتد او سو به سو بر هر رهی
&&
در گل و می‌خنددش هر ابلهی
\\
او چنین و کودکان اندر پیش
&&
بی‌خبر از مستی و ذوق میش
\\
خلق اطفالند جز مست خدا
&&
نیست بالغ جز رهیده از هوا
\\
گفت دنیا لعب و لهوست و شما
&&
کودکیت و راست فرماید خدا
\\
از لعب بیرون نرفتی کودکی
&&
بی ذکات روح کی باشد ذکی
\\
چون جماع طفل دان این شهوتی
&&
که همی رانند اینجا ای فتی
\\
آن جماع طفل چه بود بازیی
&&
با جماع رستمی و غازیی
\\
جنگ خلقان همچو جنگ کودکان
&&
جمله بی‌معنی و بی‌مغز و مهان
\\
جمله با شمشیر چوبین جنگشان
&&
جمله در لا ینفعی آهنگشان
\\
جمله شان گشته سواره بر نیی
&&
کین براق ماست یا دلدل‌پیی
\\
حاملند و خود ز جهل افراشته
&&
راکب و محمول ره پنداشته
\\
باش تا روزی که محمولان حق
&&
اسپ‌تازان بگذرند از نه طبق
\\
تعرج الروح الیه و الملک
&&
من عروج الروح یهتز الفلک
\\
همچو طفلان جمله‌تان دامن‌سوار
&&
گوشهٔ دامن گرفته اسپ‌وار
\\
از حق ان الظن لا یغنی رسید
&&
مرکب ظن بر فلکها کی دوید
\\
اغلب الظنین فی ترجیح ذا
&&
لا تماری الشمس فی توضیحها
\\
آنگهی بینید مرکبهای خویش
&&
مرکبی سازیده‌ایت از پای خویش
\\
وهم و فکر و حس و ادراک شما
&&
همچو نی دان مرکب کودک هلا
\\
علمهای اهل دل حمالشان
&&
علمهای اهل تن احمالشان
\\
علم چون بر دل زند یاری شود
&&
علم چون بر تن زند باری شود
\\
گفت ایزد یحمل اسفاره
&&
بار باشد علم کان نبود ز هو
\\
علم کان نبود ز هو بی واسطه
&&
آن نپاید همچو رنگ ماشطه
\\
لیک چون این بار را نیکو کشی
&&
بار بر گیرند و بخشندت خوشی
\\
هین مکش بهر هوا آن بار علم
&&
تا ببینی در درون انبار علم
\\
تا که بر رهوار علم آیی سوار
&&
بعد از آن افتد ترا از دوش بار
\\
از هواها کی رهی بی جام هو
&&
ای ز هو قانع شده با نام هو
\\
از صفت وز نام چه زاید خیال
&&
و آن خیالش هست دلال وصال
\\
دیده‌ای دلال بی مدلول هیچ
&&
تا نباشد جاده نبود غول هیچ
\\
هیچ نامی بی حقیقت دیده‌ای
&&
یا ز گاف و لام گل گل چیده‌ای
\\
اسم خواندی رو مسمی را بجو
&&
مه به بالا دان نه اندر آب جو
\\
گر ز نام و حرف خواهی بگذری
&&
پاک کن خود را ز خود هین یکسری
\\
همچو آهن ز آهنی بی رنگ شو
&&
در ریاضت آینهٔ بی زنگ شو
\\
خویش را صافی کن از اوصاف خود
&&
تا ببینی ذات پاک صاف خود
\\
بینی اندر دل علوم انبیا
&&
بی کتاب و بی معید و اوستا
\\
گفت پیغامبر که هست از امتم
&&
کو بود هم گوهر و هم همتم
\\
مر مرا زان نور بیند جانشان
&&
که من ایشان را همی‌بینم بدان
\\
بی صحیحین و احادیث و روات
&&
بلک اندر مشرب آب حیات
\\
سر امسینا لکردیا بدان
&&
راز اصبحنا عرابیا بخوان
\\
ور مثالی خواهی از علم نهان
&&
قصه‌گو از رومیان و چینیان
\\
\end{longtable}
\end{center}
