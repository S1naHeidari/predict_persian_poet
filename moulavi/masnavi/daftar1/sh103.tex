\begin{center}
\section*{بخش ۱۰۳ - بقیهٔ قصهٔ پیر چنگی و بیان مخلص آن}
\label{sec:sh103}
\addcontentsline{toc}{section}{\nameref{sec:sh103}}
\begin{longtable}{l p{0.5cm} r}
مطربی کز وی جهان شد پر طرب
&&
رسته ز آوازش خیالات عجب
\\
از نوایش مرغ دل پران شدی
&&
وز صدایش هوش جان حیران شدی
\\
چون برآمد روزگار و پیر شد
&&
باز جانش از عجز پشه‌گیر شد
\\
پشت او خم گشت همچون پشت خم
&&
ابروان بر چشم همچون پالدم
\\
گشت آواز لطیف جان‌فزاش
&&
زشت و نزد کس نیرزیدی بلاش
\\
آن نوای رشک زهره آمده
&&
همچو آواز خر پیری شده
\\
خود کدامین خوش که او ناخوش نشد
&&
یا کدامین سقف کان مفرش نشد
\\
غیر آواز عزیزان در صدور
&&
که بود از عکس دمشان نفخ صور
\\
اندرونی کاندرونها مست ازوست
&&
نیستی کین هستهامان هست ازوست
\\
کهربای فکر و هر آواز او
&&
لذت الهام و وحی و راز او
\\
چونک مطرب پیرتر گشت و ضعیف
&&
شد ز بی کسبی رهین یک رغیف
\\
گفت عمر و مهلتم دادی بسی
&&
لطفها کردی خدایا با خسی
\\
معصیت ورزیده‌ام هفتاد سال
&&
باز نگرفتی ز من روزی نوال
\\
نیست کسب امروز مهمان توم
&&
چنگ بهر تو زنم کان توم
\\
چنگ را برداشت و شد الله‌جو
&&
سوی گورستان یثرب آه‌گو
\\
گفت خواهم از حق ابریشم‌بها
&&
کو به نیکویی پذیرد قلبها
\\
چونک زد بسیار و گریان سر نهاد
&&
چنگ بالین کرد و بر گوری فتاد
\\
خواب بردش مرغ جانش از حبس رست
&&
چنگ و چنگی را رها کرد و بجست
\\
گشت آزاد از تن و رنج جهان
&&
در جهان ساده و صحرای جان
\\
جان او آنجا سرایان ماجرا
&&
کاندرین جا گر بماندندی مرا
\\
خوش بدی جانم درین باغ و بهار
&&
مست این صحرا و غیبی لاله‌زار
\\
بی پر و بی پا سفر می‌کردمی
&&
بی لب و دندان شکر می‌خوردمی
\\
ذکر و فکری فارغ از رنج دماغ
&&
کردمی با ساکنان چرخ لاغ
\\
چشم بسته عالمی می‌دیدمی
&&
ورد و ریحان بی کفی می‌چیدمی
\\
مرغ آبی غرق دریای عسل
&&
عین ایوبی شراب و مغتسل
\\
که بدو ایوب از پا تا به فرق
&&
پاک شد از رنجها چون نور شرق
\\
مثنوی در حجم گر بودی چو چرخ
&&
در نگنجیدی درو زین نیم برخ
\\
کان زمین و آسمان بس فراخ
&&
کرد از تنگی دلم را شاخ شاخ
\\
وین جهانی کاندرین خوابم نمود
&&
از گشایش پر و بالم را گشود
\\
این جهان و راهش ار پیدا بدی
&&
کم کسی یک لحظه‌ای آنجا بدی
\\
امر می‌آمد که نه طامع مشو
&&
چون ز پایت خار بیرون شد برو
\\
مول مولی می‌زد آنجا جان او
&&
در فضای رحمت و احسان او
\\
\end{longtable}
\end{center}
