\begin{center}
\section*{بخش ۱۵۰ - مرتد شدن کاتب وحی به سبب آنک پرتو وحی برو زد آن آیت را پیش از پیغامبر صلی الله علیه و سلم بخواند گفت پس من هم محل وحیم}
\label{sec:sh150}
\addcontentsline{toc}{section}{\nameref{sec:sh150}}
\begin{longtable}{l p{0.5cm} r}
پیش از عثمان یکی نساخ بود
&&
کو به نسخ وحی جدی می‌نمود
\\
چون نبی از وحی فرمودی سبق
&&
او همان را وا نبشتی بر ورق
\\
پرتو آن وحی بر وی تافتی
&&
او درون خویش حکمت یافتی
\\
عین آن حکمت بفرمودی رسول
&&
زین قدر گمراه شد آن بوالفضول
\\
کانچ می‌گوید رسول مستنیر
&&
مر مرا هست آن حقیقت در ضمیر
\\
پرتو اندیشه‌اش زد بر رسول
&&
قهر حق آورد بر جانش نزول
\\
هم ز نساخی بر آمد هم ز دین
&&
شد عدو مصطفی و دین بکین
\\
مصطفی فرمود کای گبر عنود
&&
چون سیه گشتی اگر نور از تو بود
\\
گر تو ینبوع الهی بودیی
&&
این چنین آب سیه نگشودیی
\\
تا که ناموسش به پیش این و آن
&&
نشکند بر بست این او را دهان
\\
اندرون می‌سوختش هم زین سبب
&&
توبه کردن می‌نیارست این عجب
\\
آه می‌کرد و نبودش آه سود
&&
چون در آمد تیغ و سر را در ربود
\\
کرده حق ناموس را صد من حدید
&&
ای بسا بسته به بند ناپدید
\\
کبر و کفر آن سان ببست آن راه را
&&
که نیارد کرد ظاهر آه را
\\
گفت اغلالا فهم به مقمحون
&&
نیست آن اغلال بر ما از برون
\\
خلفهم سدا فاغشیناهم
&&
می‌نبیند بند را پیش و پس او
\\
رنگ صحرا دارد آن سدی که خاست
&&
او نمی‌داند که آن سد قضاست
\\
شاهد تو سد روی شاهدست
&&
مرشد تو سد گفت مرشدست
\\
ای بسا کفار را سودای دین
&&
بندشان ناموس و کبر آن و این
\\
بند پنهان لیک از آهن بتر
&&
بند آهن را کند پاره تبر
\\
بند آهن را توان کردن جدا
&&
بند غیبی را نداند کس دوا
\\
مرد را زنبور اگر نیشی زند
&&
طبع او آن لحظه بر دفعی تند
\\
زخم نیش اما چو از هستی تست
&&
غم قوی باشد نگردد درد سست
\\
شرح این از سینه بیرون می‌جهد
&&
لیک می‌ترسم که نومیدی دهد
\\
نی مشو نومید و خود را شاد کن
&&
پیش آن فریادرس فریاد کن
\\
کای محب عفو از ما عفو کن
&&
ای طبیب رنج ناسور کهن
\\
عکس حکمت آن شقی را یاوه کرد
&&
خود مبین تا بر نیارد از تو گرد
\\
ای برادر بر تو حکمت جاریه‌ست
&&
آن ز ابدالست و بر تو عاریه‌ست
\\
گرچه در خود خانه نوری یافتست
&&
آن ز همسایهٔ منور تافتست
\\
شکر کن غره مشو بینی مکن
&&
گوش دار و هیچ خودبینی مکن
\\
صد دریغ و درد کین عاریتی
&&
امتان را دور کرد از امتی
\\
من غلام آن که او در هر رباط
&&
خویش را واصل نداند بر سماط
\\
بس رباطی که بباید ترک کرد
&&
تا به مسکن در رسد یک روز مرد
\\
گرچه آهن سرخ شد او سرخ نیست
&&
پرتو عاریت آتش‌زنیست
\\
گر شود پر نور روزن یا سرا
&&
تو مدان روشن مگر خورشید را
\\
هر در و دیوار گوید روشنم
&&
پرتو غیری ندارم این منم
\\
پس بگوید آفتاب ای نارشید
&&
چونک من غارب شوم آید پدید
\\
سبزه‌ها گویند ما سبز از خودیم
&&
شاد و خندانیم و بس زیبا خدیم
\\
فصل تابستان بگوید ای امم
&&
خویش را بینید چون من بگذرم
\\
تن همی‌نازد به خوبی و جمال
&&
روح پنهان کرده فر و پر و بال
\\
گویدش ای مزبله تو کیستی
&&
یک دو روز از پرتو من زیستی
\\
غنج و نازت می‌نگنجد در جهان
&&
باش تا که من شوم از تو جهان
\\
گرم‌دارانت ترا گوری کنند
&&
طعمهٔ ماران و مورانت کنند
\\
بینی از گند تو گیرد آن کسی
&&
کو به پیش تو همی‌مردی بسی
\\
پرتو روحست نطق و چشم و گوش
&&
پرتو آتش بود در آب جوش
\\
آنچنانک پرتو جان بر تنست
&&
پرتو ابدال بر جان منست
\\
جان جان چو واکشد پا را ز جان
&&
جان چنان گردد که بی‌جان تن بدان
\\
سر از آن رو می‌نهم من بر زمین
&&
تا گواه من بود در روز دین
\\
یوم دین که زلزلت زلزالها
&&
این زمین باشد گواه حالها
\\
گو تحدث جهرة اخبارها
&&
در سخن آید زمین و خاره‌ها
\\
فلسفی منکر شود در فکر و ظن
&&
گو برو سر را بر آن دیوار زن
\\
نطق آب و نطق خاک و نطق گل
&&
هست محسوس حواس اهل دل
\\
فلسفی کو منکر حنانه است
&&
از حواس اولیا بیگانه است
\\
گوید او که پرتو سودای خلق
&&
بس خیالات آورد در رای خلق
\\
بلک عکس آن فساد و کفر او
&&
این خیال منکری را زد برو
\\
فلسفی مر دیو را منکر شود
&&
در همان دم سخرهٔ دیوی بود
\\
گر ندیدی دیو را خود را ببین
&&
بی جنون نبود کبودی بر جبین
\\
هر که را در دل شک و پیچانیست
&&
در جهان او فلسفی پنهانیست
\\
می‌نماید اعتقاد و گاه گاه
&&
آن رگ فلسف کند رویش سیاه
\\
الحذر ای مؤمنان کان در شماست
&&
در شما بس عالم بی‌منتهاست
\\
جمله هفتاد و دو ملت در توست
&&
وه که روزی آن بر آرد از تو دست
\\
هر که او را برگ آن ایمان بود
&&
همچو برگ از بیم این لرزان بود
\\
بر بلیس و دیو زان خندیده‌ای
&&
که تو خود را نیک مردم دیده‌ای
\\
چون کند جان بازگونه پوستین
&&
چند وا ویلی بر آید ز اهل دین
\\
بر دکان هر زرنما خندان شدست
&&
زانک سنگ امتحان پنهان شدست
\\
پرده‌ای ستار از ما بر مگیر
&&
باش اندر امتحان ما را مجیر
\\
قلب پهلو می‌زند با زر به شب
&&
انتظار روز می‌دارد ذهب
\\
با زبان حال زر گوید که باش
&&
ای مزور تا بر آید روز فاش
\\
صد هزاران سال ابلیس لعین
&&
بود ز ابدال و امیر المؤمنین
\\
پنجه زد با آدم از نازی که داشت
&&
گشت رسوا همچو سرگین وقت چاشت
\\
\end{longtable}
\end{center}
