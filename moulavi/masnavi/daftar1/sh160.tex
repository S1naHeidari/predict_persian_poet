\begin{center}
\section*{بخش ۱۶۰ - بقیهٔ قصه زید در جواب رسول صلی الله علیه و سلم}
\label{sec:sh160}
\addcontentsline{toc}{section}{\nameref{sec:sh160}}
\begin{longtable}{l p{0.5cm} r}
این سخن پایان ندارد خیز زید
&&
بر براق ناطقه بر بند قید
\\
ناطقه چون فاضح آمد عیب را
&&
می‌دراند پرده‌های غیب را
\\
غیب مطلوب حق آمد چند گاه
&&
این دهل زن را بران بر بند راه
\\
تگ مران درکش عنان مستور به
&&
هر کس از پندار خود مسرور به
\\
حق همی‌خواهد که نومیدان او
&&
زین عبادت هم نگردانند رو
\\
هم باومیدی مشرف می‌شوند
&&
چند روزی در رکابش می‌دوند
\\
خواهد آن رحمت بتابد بر همه
&&
بر بد و نیک از عموم مرحمه
\\
حق همی‌خواهد که هر میر و اسیر
&&
با رجا و خوف باشند و حذیر
\\
این رجا و خوف در پرده بود
&&
تا پس این پرده پرورده شود
\\
چون دریدی پرده کو خوف و رجا
&&
غیب را شد کر و فری بر ملا
\\
بر لب جو برد ظنی یک فتی
&&
که سلیمانست ماهی‌گیر ما
\\
گر ویست این از چه فردست و خفیست
&&
ورنه سیمای سلیمانیش چیست
\\
اندرین اندیشه می‌بود او دو دل
&&
تا سلیمان گشت شاه و مستقل
\\
دیو رفت از ملک و تخت او گریخت
&&
تیغ بختش خون آن شیطان بریخت
\\
کرد در انگشت خود انگشتری
&&
جمع آمد لشکر دیو و پری
\\
آمدند از بهر نظاره رجال
&&
در میانشان آنک بد صاحب‌خیال
\\
چون در انگشتش بدید انگشتری
&&
رفت اندیشه و گمانش یکسری
\\
وهم آنگاهست کان پوشیده است
&&
این تحری از پی نادیده است
\\
شد خیال غایب اندر سینه زفت
&&
چونک حاضر شد خیال او برفت
\\
گر سمای نور بی باریده نیست
&&
هم زمین تار بی بالیده نیست
\\
یمنون بالغیب می‌باید مرا
&&
زان ببستم روزن فانی سرا
\\
چون شکافم آسمان را در ظهور
&&
چون بگویم هل تری فیها فطور
\\
تا درین ظلمت تحری گسترند
&&
هر کسی رو جانبی می‌آورند
\\
مدتی معکوس باشد کارها
&&
شحنه را دزد آورد بر دارها
\\
تا که بس سلطان و عالی‌همتی
&&
بندهٔ بندهٔ خود آید مدتی
\\
بندگی در غیب آید خوب و گش
&&
حفظ غیب آید در استعباد خوش
\\
کو که مدح شاه گوید پیش او
&&
تا که در غیبت بود او شرم‌رو
\\
قلعه‌داری کز کنار مملکت
&&
دور از سلطان و سایهٔ سلطنت
\\
پاس دارد قلعه را از دشمنان
&&
قلعه نفروشد به مالی بی‌کران
\\
غایب از شه در کنار ثغرها
&&
همچو حاضر او نگه دارد وفا
\\
پیش شه او به بود از دیگران
&&
که به خدمت حاضرند و جان‌فشان
\\
پس بغیبت نیم ذره حفظ کار
&&
به که اندر حاضری زان صد هزار
\\
طاعت و ایمان کنون محمود شد
&&
بعد مرگ اندر عیان مردود شد
\\
چونک غیب و غایب و روپوش به
&&
پس لبان بر بند و لب خاموش به
\\
ای برادر دست وادار از سخن
&&
خود خدا پیدا کند علم لدن
\\
پس بود خورشید را رویش گواه
&&
ای شیء اعظم الشاهد اله
\\
نه بگویم چون قرین شد در بیان
&&
هم خدا و هم ملک هم عالمان
\\
یشهد الله و الملک و اهل العلوم
&&
انه لا رب الا من یدوم
\\
چون گواهی داد حق کی بود ملک
&&
تا شود اندر گواهی مشترک
\\
زانک شعشاع و حضور آفتاب
&&
بر نتابد چشم و دلهای خراب
\\
چون خفاشی کو تف خورشید را
&&
بر نتابد بسکلد اومید را
\\
پس ملایک را چو ما هم یار دان
&&
جلوه‌گر خورشید را بر آسمان
\\
کین ضیا ما ز آفتابی یافتیم
&&
چون خلیفه بر ضعیفان تافتیم
\\
چون مه نو یا سه روزه یا که بدر
&&
هر ملک دارد کمال و نور و قدر
\\
ز اجنحهٔ نور ثلاث او رباع
&&
بر مراتب هر ملک را آن شعاع
\\
همچو پرهای عقول انسیان
&&
که بسی فرقستشان اندر میان
\\
پس قرین هر بشر در نیک و بد
&&
آن ملک باشد که مانندش بود
\\
چشم اعمش چونک خور را بر نتافت
&&
اختر او را شمع شد تا ره بیافت
\\
\end{longtable}
\end{center}
