\begin{center}
\section*{بخش ۸۲ - سال کردن رسول روم از عمر رضی‌الله عنه از سبب ابتلای ارواح با این آب و گل جسم}
\label{sec:sh082}
\addcontentsline{toc}{section}{\nameref{sec:sh082}}
\begin{longtable}{l p{0.5cm} r}
گفت یا عمر چه حکمت بود و سر
&&
حبس آن صافی درین جای کدر
\\
آب صافی در گلی پنهان شده
&&
جان صافی بستهٔ ابدان شده
\\
گفت تو بحثی شگرفی می‌کنی
&&
معنیی را بند حرفی می‌کنی
\\
حبس کردی معنی آزاد را
&&
بند حرفی کرده‌ای تو یاد را
\\
از برای فایده این کرده‌ای
&&
تو که خود از فایده در پرده‌ای
\\
آنک از وی فایده زاییده شد
&&
چون نبیند آنچ ما را دیده شد
\\
صد هزاران فایده‌ست و هر یکی
&&
صد هزاران پیش آن یک اندکی
\\
آن دم نطقت که جزو جزوهاست
&&
فایده شد کل کل خالی چراست
\\
تو که جزوی کار تو با فایده‌ست
&&
پس چرا در طعن کل آری تو دست
\\
گفت را گر فایده نبود مگو
&&
ور بود هل اعتراض و شکر جو
\\
شکر یزدان طوق هر گردن بود
&&
نی جدال و رو ترش کردن بود
\\
گر ترش‌رو بودن آمد شکر و بس
&&
پس چو سرکه شکرگویی نیست کس
\\
سرکه را گر راه باید در جگر
&&
گو بشو سرکنگبین او از شکر
\\
معنی اندر شعر جز با خبط نیست
&&
چون قلاسنگست و اندر ضبط نیست
\\
\end{longtable}
\end{center}
