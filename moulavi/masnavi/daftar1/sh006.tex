\begin{center}
\section*{بخش ۶ - بردن پادشاه آن طبیب را بر بیمار تا حال او را ببیند}
\label{sec:sh006}
\addcontentsline{toc}{section}{\nameref{sec:sh006}}
\begin{longtable}{l p{0.5cm} r}
قصهٔ رنجور و رنجوری بخواند
&&
بعد از آن در پیش رنجورش نشاند
\\
رنگ روی و نبض و قاروره بدید
&&
هم علاماتش هم اسبابش شنید
\\
گفت هر دارو که ایشان کرده‌اند
&&
آن عمارت نیست ویران کرده‌اند
\\
بی‌خبر بودند از حال درون
&&
استعیذ الله مما یفترون
\\
دید رنج و کشف شد بروی نهفت
&&
لیک پنهان کرد وبا سلطان نگفت
\\
رنجش از صفرا و از سودا نبود
&&
بوی هر هیزم پدید آید ز دود
\\
دید از زاریش کو زار دلست
&&
تن خوشست و او گرفتار دلست
\\
عاشقی پیداست از زاری دل
&&
نیست بیماری چو بیماری دل
\\
علت عاشق ز علتها جداست
&&
عشق اصطرلاب اسرار خداست
\\
عاشقی گر زین سر و گر زان سرست
&&
عاقبت ما را بدان سر رهبرست
\\
هرچه گویم عشق را شرح و بیان
&&
چون به عشق آیم خجل باشم از آن
\\
گرچه تفسیر زبان روشنگرست
&&
لیک عشق بی‌زبان روشنترست
\\
چون قلم اندر نوشتن می‌شتافت
&&
چون به عشق آمد قلم بر خود شکافت
\\
عقل در شرحش چو خر در گل بخفت
&&
شرح عشق و عاشقی هم عشق گفت
\\
آفتاب آمد دلیل آفتاب
&&
گر دلیلت باید از وی رو متاب
\\
از وی ار سایه نشانی می‌دهد
&&
شمس هر دم نور جانی می‌دهد
\\
سایه خواب آرد ترا همچون سمر
&&
چون برآید شمس انشق القمر
\\
خود غریبی در جهان چون شمس نیست
&&
شمس جان باقیست کاو را امس نیست
\\
شمس در خارج اگر چه هست فرد
&&
می‌توان هم مثل او تصویر کرد
\\
شمس جان کو خارج آمد از اثیر
&&
نبودش در ذهن و در خارج نظیر
\\
در تصور ذات او را گنج کو
&&
تا در آید در تصور مثل او
\\
چون حدیث روی شمس الدین رسید
&&
شمس چارم آسمان سر در کشید
\\
واجب آید چونک آمد نام او
&&
شرح کردن رمزی از انعام او
\\
این نفس جان دامنم بر تافتست
&&
بوی پیراهان یوسف یافتست
\\
کز برای حق صحبت سالها
&&
بازگو حالی از آن خوش حالها
\\
تا زمین و آسمان خندان شود
&&
عقل و روح و دیده صد چندان شود
\\
لاتکلفنی فانی فی الفنا
&&
کلت افهامی فلا احصی ثنا
\\
کل شیء قاله غیرالمفیق
&&
ان تکلف او تصلف لا یلیق
\\
من چه گویم یک رگم هشیار نیست
&&
شرح آن یاری که او را یار نیست
\\
شرح این هجران و این خون جگر
&&
این زمان بگذار تا وقت دگر
\\
قال اطعمنی فانی جائع
&&
واعتجل فالوقت سیف قاطع
\\
صوفی ابن الوقت باشد ای رفیق
&&
نیست فردا گفتن از شرط طریق
\\
تو مگر خود مرد صوفی نیستی
&&
هست را از نسیه خیزد نیستی
\\
گفتمش پوشیده خوشتر سر یار
&&
خود تو در ضمن حکایت گوش‌دار
\\
خوشتر آن باشد که سر دلبران
&&
گفته آید در حدیث دیگران
\\
گفت مکشوف و برهنه بی‌غلول
&&
بازگو دفعم مده ای بوالفضول
\\
پرده بردار و برهنه گو که من
&&
می‌نخسپم با صنم با پیرهن
\\
گفتم ار عریان شود او در عیان
&&
نه تو مانی نه کنارت نه میان
\\
آرزو می‌خواه لیک اندازه خواه
&&
بر نتابد کوه را یک برگ کاه
\\
آفتابی کز وی این عالم فروخت
&&
اندکی گر پیش آید جمله سوخت
\\
فتنه و آشوب و خون‌ریزی مجوی
&&
بیش ازین از شمس تبریزی مگوی
\\
این ندارد آخر از آغاز گوی
&&
رو تمام این حکایت بازگوی
\\
\end{longtable}
\end{center}
