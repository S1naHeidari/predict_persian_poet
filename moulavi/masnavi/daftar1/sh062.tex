\begin{center}
\section*{بخش ۶۲ - هم در بیان مکر خرگوش}
\label{sec:sh062}
\addcontentsline{toc}{section}{\nameref{sec:sh062}}
\begin{longtable}{l p{0.5cm} r}
در شدن خرگوش بس تاخیر کرد
&&
مکر را با خویشتن تقریر کرد
\\
در ره آمد بعد تاخیر دراز
&&
تا به گوش شیر گوید یک دو راز
\\
تا چه عالمهاست در سودای عقل
&&
تا چه با پهناست این دریای عقل
\\
صورت ما اندرین بحر عذاب
&&
می‌دود چون کاسه‌ها بر روی آب
\\
تا نشد پر بر سر دریا چو طشت
&&
چونک پر شد طشت در وی غرق گشت
\\
عقل پنهانست و ظاهر عالمی
&&
صورت ما موج یا از وی نمی
\\
هر چه صورت می وسیلت سازدش
&&
زان وسیلت بحر دور اندازدش
\\
تا نبیند دل دهندهٔ راز را
&&
تا نبیند تیر دورانداز را
\\
اسپ خود را یاوه داند وز ستیز
&&
می‌دواند اسپ خود در راه تیز
\\
اسپ خود را یاوه داند آن جواد
&&
و اسپ خود او را کشان کرده چو باد
\\
در فغان و جست و جو آن خیره‌سر
&&
هر طرف پرسان و جویان در بدر
\\
کانک دزدید اسپ ما را کو و کیست
&&
این که زیر ران تست ای خواجه چیست
\\
آری این اسپست لیک این اسپ کو
&&
با خود آی ای شهسوار اسپ‌جو
\\
جان ز پیدایی و نزدیکیست گم
&&
چون شکم پر آب و لب خشکی چو خم
\\
کی ببینی سرخ و سبز و فور را
&&
تا نبینی پیش ازین سه نور را
\\
لیک چون در رنگ گم شد هوش تو
&&
شد ز نور آن رنگها روپوش تو
\\
چونک شب آن رنگها مستور بود
&&
پس بدیدی دید رنگ از نور بود
\\
نیست دید رنگ بی‌نور برون
&&
همچنین رنگ خیال اندرون
\\
این برون از آفتاب و از سها
&&
واندرون از عکس انوار علا
\\
نور نور چشم خود نور دلست
&&
نور چشم از نور دلها حاصلست
\\
باز نور نور دل نور خداست
&&
کو ز نور عقل و حس پاک و جداست
\\
شب نبد نور و ندیدی رنگها
&&
پس به ضد نور پیدا شد ترا
\\
دیدن نورست آنگه دید رنگ
&&
وین به ضد نور دانی بی‌درنگ
\\
رنج و غم را حق پی آن آفرید
&&
تا بدین ضد خوش‌دلی آید پدید
\\
پس نهانیها بضد پیدا شود
&&
چونک حق را نیست ضد پنهان بود
\\
که نظر پر نور بود آنگه برنگ
&&
ضد به ضد پیدا بود چون روم و زنگ
\\
پس به ضد نور دانستی تو نور
&&
ضد ضد را می‌نماید در صدور
\\
نور حق را نیست ضدی در وجود
&&
تا به ضد او را توان پیدا نمود
\\
لاجرم ابصار ما لا تدرکه
&&
و هو یدرک بین تو از موسی و که
\\
صورت از معنی چو شیر از بیشه دان
&&
یا چو آواز و سخن ز اندیشه دان
\\
این سخن و آواز از اندیشه خاست
&&
تو ندانی بحر اندیشه کجاست
\\
لیک چون موج سخن دیدی لطیف
&&
بحر آن دانی که باشد هم شریف
\\
چون ز دانش موج اندیشه بتاخت
&&
از سخن و آواز او صورت بساخت
\\
از سخن صورت بزاد و باز مرد
&&
موج خود را باز اندر بحر برد
\\
صورت از بی‌صورتی آمد برون
&&
باز شد که انا الیه راجعون
\\
پس ترا هر لحظه مرگ و رجعتیست
&&
مصطفی فرمود دنیا ساعتیست
\\
فکر ما تیریست از هو در هوا
&&
در هوا کی پاید آید تا خدا
\\
هر نفس نو می‌شود دنیا و ما
&&
بی‌خبر از نو شدن اندر بقا
\\
عمر همچون جوی نو نو می‌رسد
&&
مستمری می‌نماید در جسد
\\
آن ز تیزی مستمر شکل آمده‌ست
&&
چون شرر کش تیز جنبانی بدست
\\
شاخ آتش را بجنبانی بساز
&&
در نظر آتش نماید بس دراز
\\
این درازی مدت از تیزی صنع
&&
می‌نماید سرعت‌انگیزی صنع
\\
طالب این سر اگر علامه‌ایست
&&
نک حسام‌الدین که سامی نامه‌ایست
\\
\end{longtable}
\end{center}
