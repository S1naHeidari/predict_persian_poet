\begin{center}
\section*{بخش ۸۵ - صفت اجنحهٔ طیور عقول الهی}
\label{sec:sh085}
\addcontentsline{toc}{section}{\nameref{sec:sh085}}
\begin{longtable}{l p{0.5cm} r}
قصهٔ طوطی جان زین سان بود
&&
کو کسی کو محرم مرغان بود
\\
کو یکی مرغی ضعیفی بی‌گناه
&&
و اندرون او سلیمان با سپاه
\\
چون بنالد زار بی‌شکر و گله
&&
افتد اندر هفت گردون غلغله
\\
هر دمش صد نامه صد پیک از خدا
&&
یا ربی زو شصت لبیک از خدا
\\
زلت او به ز طاعت نزد حق
&&
پیش کفرش جمله ایمانها خلق
\\
هر دمی او را یکی معراج خاص
&&
بر سر تاجش نهد صد تاج خاص
\\
صورتش بر خاک و جان بر لامکان
&&
لامکانی فوق وهم سالکان
\\
لامکانی نه که در فهم آیدت
&&
هر دمی در وی خیالی زایدت
\\
بل مکان و لامکان در حکم او
&&
همچو در حکم بهشتی چار جو
\\
شرح این کوته کن و رخ زین بتاب
&&
دم مزن والله اعلم بالصواب
\\
باز می‌گردیم ما ای دوستان
&&
سوی مرغ و تاجر و هندوستان
\\
مرد بازرگان پذیرفت این پیام
&&
کو رساند سوی جنس از وی سلام
\\
\end{longtable}
\end{center}
