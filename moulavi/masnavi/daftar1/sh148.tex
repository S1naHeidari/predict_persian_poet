\begin{center}
\section*{بخش ۱۴۸ - آمدن مهمان پیش یوسف علیه‌السلام و تقاضا کردن یوسف علیه‌السلام ازو تحفه و ارمغان}
\label{sec:sh148}
\addcontentsline{toc}{section}{\nameref{sec:sh148}}
\begin{longtable}{l p{0.5cm} r}
آمد از آفاق یار مهربان
&&
یوسف صدیق را شد میهمان
\\
کاشنا بودند وقت کودکی
&&
بر وسادهٔ آشنایی متکی
\\
یاد دادش جور اخوان و حسد
&&
گفت کان زنجیر بود و ما اسد
\\
عار نبود شیر را از سلسله
&&
نیست ما را از قضای حق گله
\\
شیر را بر گردن ار زنجیر بود
&&
بر همه زنجیرسازان میر بود
\\
گفت چون بودی ز زندان و ز چاه
&&
گفت همچون در محاق و کاست ماه
\\
در محاق ار ماه نو گردد دوتا
&&
نی در آخر بدر گردد بر سما
\\
گرچه دردانه به هاون کوفتند
&&
نور چشم و دل شد و بیند بلند
\\
گندمی را زیر خاک انداختند
&&
پس ز خاکش خوشه‌ها بر ساختند
\\
بار دیگر کوفتندش ز آسیا
&&
قیمتش افزود و نان شد جان‌فزا
\\
باز نان را زیر دندان کوفتند
&&
گشت عقل و جان و فهم هوشمند
\\
باز آن جان چونک محو عشق گشت
&&
یعجب الزراع آمد بعد کشت
\\
این سخن پایان ندارد باز گرد
&&
تا که با یوسف چه گفت آن نیک مرد
\\
بعد قصه گفتنش گفت ای فلان
&&
هین چه آوردی تو ما را ارمغان
\\
بر در یاران تهی‌دست آمدن
&&
هست بی‌گندم سوی طاحون شدن
\\
حق تعالی خلق را گوید بحشر
&&
ارمغان کو از برای روز نشر
\\
جئتمونا و فرادی بی نوا
&&
هم بدان سان که خلقناکم کذا
\\
هین چه آوردید دست‌آویز را
&&
ارمغانی روز رستاخیز را
\\
یا امید بازگشتنتان نبود
&&
وعدهٔ امروز باطلتان نمود
\\
منکری مهمانیش را از خری
&&
پس ز مطبخ خاک و خاکستر بری
\\
ور نه‌ای منکر چنین دست تهی
&&
در در آن دوست چون پا می‌نهی
\\
اندکی صرفه بکن از خواب و خور
&&
ارمغان بهر ملاقاتش ببر
\\
شو قلیل النوم مما یهجعون
&&
باش در اسحار از یستغفرون
\\
اندکی جنبش بکن همچون جنین
&&
تا ببخشندت حواس نوربین
\\
وز جهان چون رحم بیرون روی
&&
از زمین در عرصهٔ واسع شوی
\\
آنک ارض الله واسع گفته‌اند
&&
عرصه‌ای دان انبیا را بس بلند
\\
دل نگردد تنگ زان عرصهٔ فراخ
&&
نخل تر آنجا نگردد خشک شاخ
\\
حاملی تو مر حواست را کنون
&&
کند و مانده می‌شوی و سرنگون
\\
چونک محمولی نه حامل وقت خواب
&&
ماندگی رفت و شدی بی رنج و تاب
\\
چاشنیی دان تو حال خواب را
&&
پیش محمولی حال اولیا
\\
اولیا اصحاب کهفند ای عنود
&&
در قیام و در تقلب هم رقود
\\
می‌کشدشان بی تکلف در فعال
&&
بی‌خبر ذات الیمین ذات الشمال
\\
چیست آن ذات الیمین فعل حسن
&&
چیست آن ذات الشکال اشغال تن
\\
می‌رود این هر دو کار از انبیا
&&
بی‌خبر زین هر دو ایشان چون صدا
\\
گر صدایت بشنواند خیر و شر
&&
ذات که باشد ز هر دو بی‌خبر
\\
\end{longtable}
\end{center}
