\begin{center}
\section*{بخش ۱۰۰ - تفسیر بیت حکم رضی‌الله عنه «آسمانهاست در ولایت جان   کارفرمای آسمان جهان» «در ره روح پست و بالاهاست  کوههای بلند و دریاهاست»}
\label{sec:sh100}
\addcontentsline{toc}{section}{\nameref{sec:sh100}}
\begin{longtable}{l p{0.5cm} r}
غیب را ابری و آبی دیگرست
&&
آسمان و آفتابی دیگرست
\\
ناید آن الا که بر خاصان پدید
&&
باقیان فی لبس من خلق جدید
\\
هست باران از پی پروردگی
&&
هست باران از پی پژمردگی
\\
نفع باران بهاران بوالعجب
&&
باغ را باران پاییزی چو تب
\\
آن بهاری نازپروردش کند
&&
وین خزانی ناخوش و زردش کند
\\
همچنین سرما و باد و آفتاب
&&
بر تفاوت دان و سررشته بیاب
\\
همچنین در غیب انواعست این
&&
در زیان و سود و در ربح و غبین
\\
این دم ابدال باشد زان بهار
&&
در دل و جان روید از وی سبزه‌زار
\\
فعل باران بهاری با درخت
&&
آید از انفاسشان در نیکبخت
\\
گر درخت خشک باشد در مکان
&&
عیب آن از باد جان‌افزا مدان
\\
باد کار خویش کرد و بر وزید
&&
آنک جانی داشت بر جانش گزید
\\
\end{longtable}
\end{center}
