\begin{center}
\section*{بخش ۱۲۱ - در بیان آنک موسی و فرعون هر دو مسخر مشیت‌اند چنانک زهر و پازهر و ظلمات و نور و مناجات کردن فرعون بخلوت تا ناموس نشکند}
\label{sec:sh121}
\addcontentsline{toc}{section}{\nameref{sec:sh121}}
\begin{longtable}{l p{0.5cm} r}
موسی و فرعون معنی را رهی
&&
ظاهر آن ره دارد و این بی‌رهی
\\
روز موسی پیش حق نالان شده
&&
نیمشب فرعون هم گریان بده
\\
کین چه غلست ای خدا بر گردنم
&&
ورنه غل باشد کی گوید من منم
\\
زانک موسی را منور کرده‌ای
&&
مر مرا زان هم مکدر کرده‌ای
\\
زانک موسی را تو مه‌رو کرده‌ای
&&
ماه جانم را سیه‌رو کرده‌ای
\\
بهتر از ماهی نبود استاره‌ام
&&
چون خسوف آمد چه باشد چاره‌ام
\\
نوبتم گر رب و سلطان می‌زنند
&&
مه گرفت و خلق پنگان می‌زنند
\\
می‌زنند آن طاس و غوغا می‌کنند
&&
ماه را زان زخمه رسوا می‌کنند
\\
من که فرعونم ز خلق ای وای من
&&
زخم طاس آن ربی الاعلای من
\\
خواجه‌تاشانیم اما تیشه‌ات
&&
می‌شکافد شاخ را در بیشه‌ات
\\
باز شاخی را موصل می‌کند
&&
شاخ دیگر را معطل می‌کند
\\
شاخ را بر تیشه دستی هست نی
&&
هیچ شاخ از دست تیشه جست نی
\\
حق آن قدرت که آن تیشه تراست
&&
از کرم کن این کژیها را تو راست
\\
باز با خود گفته فرعون ای عجب
&&
من نه دریا ربناام جمله شب
\\
در نهان خاکی و موزون می‌شوم
&&
چون به موسی می‌رسم چون می‌شوم
\\
رنگ زر قلب ده‌تو می‌شود
&&
پیش آتش چون سیه‌رو می‌شود
\\
نه که قلب و قالبم در حکم اوست
&&
لحظه‌ای مغزم کند یک لحظه پوست
\\
سبز گردم چونک گوید کشت باش
&&
زرد گردم چونک گوید زشت باش
\\
لحظه‌ای ماهم کند یک دم سیاه
&&
خود چه باشد غیر این کار اله
\\
پیش چوگانهای حکم کن فکان
&&
می‌دویم اندر مکان و لامکان
\\
چونک بی‌رنگی اسیر رنگ شد
&&
موسیی با موسیی در جنگ شد
\\
چون به بی‌رنگی رسی کان داشتی
&&
موسی و فرعون دارند آشتی
\\
گر ترا آید برین نکته سئوال
&&
رنگ کی خالی بود از قیل و قال
\\
این عجب کین رنگ از بی‌رنگ خاست
&&
رنگ با بی‌رنگ چون در جنگ خاست
\\
اصل روغن ز آب افزون می‌شود
&&
عاقبت با آب ضد چون میشود
\\
چونک روغن را ز آب اسرشته‌اند
&&
آب با روغن چرا ضد گشته‌اند
\\
چون گل از خارست و خار از گل چرا
&&
هر دو در جنگند و اندر ماجرا
\\
یا نه جنگست این برای حکمتست
&&
همچو جنگ خر فروشان صنعتست
\\
یا نه اینست و نه آن حیرانیست
&&
گنج باید جست این ویرانیست
\\
آنچ تو گنجش توهم می‌کنی
&&
زان توهم گنج را گم می‌کنی
\\
چون عمارت دان تو وهم و رایها
&&
گنج نبود در عمارت جایها
\\
در عمارت هستی و جنگی بود
&&
نیست را از هستها ننگی بود
\\
نه که هست از نیستی فریاد کرد
&&
بلک نیست آن هست را واداد کرد
\\
تو مگو که من گریزانم ز نیست
&&
بلک او از تو گریزانست بیست
\\
ظاهرا می‌خواندت او سوی خود
&&
وز درون می‌راندت با چوب رد
\\
نعلهای بازگونه‌ست ای سلیم
&&
نفرت فرعون می‌دان از کلیم
\\
\end{longtable}
\end{center}
