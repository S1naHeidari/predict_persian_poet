\begin{center}
\section*{بخش ۷۶ - معجزهٔ هود علیه‌السلام در تخلص ممنان امت به وقت نزول باد}
\label{sec:sh076}
\addcontentsline{toc}{section}{\nameref{sec:sh076}}
\begin{longtable}{l p{0.5cm} r}
مؤمنان از دست باد ضایره
&&
جمله بنشستند اندر دایره
\\
یاد طوفان بود و کشتی لطف هو
&&
بس چنین کشتی و طوفان دارد او
\\
پادشاهی را خدا کشتی کند
&&
تا به حرص خویش بر صفها زند
\\
قصد شه آن نه که خلق آمن شوند
&&
قصدش آنک ملک گردد پای‌بند
\\
آن خراسی می‌دود قصدش خلاص
&&
تا بیابد او ز زخم آن دم مناص
\\
قصد او آن نه که آبی بر کشد
&&
یاکه کنجد را بدان روغن کند
\\
گاو بشتابد ز بیم زخم سخت
&&
نه برای بردن گردون و رخت
\\
لیک دادش حق چنین خوف وجع
&&
تا مصالح حاصل آید در تبع
\\
هم‌چنان هر کاسبی اندر دکان
&&
بهر خود کوشد نه اصلاح جهان
\\
هر یکی بر درد جوید مرهمی
&&
در تبع قایم شده زین عالمی
\\
حق ستون این جهان از ترس ساخت
&&
هر یکی از ترس جان در کار باخت
\\
حمد ایزد را که ترسی را چنین
&&
کرد او معمار و اصلاح زمین
\\
این همه ترسنده‌اند از نیک و بد
&&
هیچ ترسنده نترسد خود ز خود
\\
پس حقیقت بر همه حاکم کسیست
&&
که قریبست او اگر محسوس نیست
\\
هست او محسوس اندر مکمنی
&&
لیک محسوس حس این خانه نی
\\
آن حسی که حق بر آن حس مظهرست
&&
نیست حس این جهان آن دیگرست
\\
حس حیوان گر بدیدی آن صور
&&
بایزید وقت بودی گاو و خر
\\
آنک تن را مظهر هر روح کرد
&&
وآنک کشتی را براق نوح کرد
\\
گر بخواهد عین کشتی را به خو
&&
او کند طوفان تو ای نورجو
\\
هر دمت طوفان و کشتی ای مقل
&&
با غم و شادیت کرد او متصل
\\
گر نبینی کشتی و دریا به پیش
&&
لرزها بین در همه اجزای خویش
\\
چون نبیند اصل ترسش را عیون
&&
ترس دارد از خیال گونه‌گون
\\
مشت بر اعمی زند یک جلف مست
&&
کور پندارد لگدزن اشترست
\\
زانک آن دم بانگ اشتر می‌شنید
&&
کور را گوشست آیینه نه دید
\\
باز گوید کور نه این سنگ بود
&&
یا مگر از قبهٔ پر طنگ بود
\\
این نبود و او نبود و آن نبود
&&
آنک او ترس آفرید اینها نمود
\\
ترس و لرزه باشد از غیری یقین
&&
هیچ کس از خود نترسد ای حزین
\\
آن حکیمک وهم خواند ترس را
&&
فهم کژ کردست او این درس را
\\
هیچ وهمی بی‌حقیقت کی بود
&&
هیچ قلبی بی‌صحیحی کی رود
\\
کی دروغی قیمت آرد بی ز راست
&&
در دو عالم هر دروغ از راست خاست
\\
راست را دید او رواجی و فروغ
&&
بر امید آن روان کرد او دروغ
\\
ای دروغی که ز صدقت این نواست
&&
شکر نعمت گو مکن انکار راست
\\
از مفلسف گویم و سودای او
&&
یا ز کشتیها و دریاهای او
\\
بل ز کشتیهاش کان پند دلست
&&
گویم از کل جزو در کل داخلست
\\
هر ولی را نوح و کشتیبان شناس
&&
صحبت این خلق را طوفان شناس
\\
کم گریز از شیر و اژدرهای نر
&&
ز آشنایان و ز خویشان کن حذر
\\
در تلاقی روزگارت می‌برند
&&
یادهاشان غایبی‌ات می‌چرند
\\
چون خر تشنه خیال هر یکی
&&
از قف تن فکر را شربت‌مکی
\\
نشف کرد از تو خیال آن وشات
&&
شبنمی که داری از بحر الحیات
\\
پس نشان نشف آب اندر غصون
&&
آن بود کان می‌نجنبد در رکون
\\
عضو حر شاخ تر و تازه بود
&&
می‌کشی هر سو کشیده می‌شود
\\
گر سبد خواهی توانی کردنش
&&
هم توانی کرد چنبر گردنش
\\
چون شد آن ناشف ز نشف بیخ خود
&&
ناید آن سویی که امرش می‌کشد
\\
پس بخوان قاموا کسالی از نبی
&&
چون نیابد شاخ از بیخش طبی
\\
آتشین است این نشان کوته کنم
&&
بر فقیر و گنج و احوالش زنم
\\
آتشی دیدی که سوزد هر نهال
&&
آتش جان بین کزو سوزد خیال
\\
نه خیال و نه حقیقت را امان
&&
زین چنین آتش که شعله زد ز جان
\\
خصم هر شیر آمد و هر روبه او
&&
کل شیء هالک الا وجهه
\\
در وجوه وجه او رو خرج شو
&&
چون الف در بسم در رو درج شو
\\
آن الف در بسم پنهان کرد ایست
&&
هست او در بسم و هم در بسم نیست
\\
هم‌چنین جملهٔ حروف گشته مات
&&
وقت حذف حرف از بهر صلات
\\
از صله‌ست و بی و سین زو وصل یافت
&&
وصل بی و سین الف را بر نتافت
\\
چونک حرفی برنتابد این وصال
&&
واجب آید که کنم کوته مقال
\\
چون یکی حرفی فراق سین و بیست
&&
خامشی اینجا مهمتر واجبیست
\\
چون الف از خود فنا شد مکتنف
&&
بی و سین بی او همی‌گویند الف
\\
ما رمیت اذ رمیت بی ویست
&&
هم‌چنین قال الله از صمتش بجست
\\
تا بود دارو ندارد او عمل
&&
چونک شد فانی کند دفع علل
\\
گر شود بیشه قلم دریا مداد
&&
مثنوی را نیست پایانی امید
\\
چارچوب خشت‌زن تا خاک هست
&&
می‌دهد تقطیع شعرش نیز دست
\\
چون نماند خاک و بودش جف کند
&&
خاک سازد بحر او چون کف کند
\\
چون نماند بیشه و سر در کشد
&&
بیشه‌ها از عین دریا سر کشد
\\
بهر این گفت آن خداوند فرج
&&
حدثوا عن بحرنا اذ لا حرج
\\
باز گرد از بحر و رو در خشک نه
&&
هم ز لعبت گو که کودک‌راست به
\\
تا ز لعبت اندک اندک در صبا
&&
جانش گردد با یم عقل آشنا
\\
عقل از آن بازی همی‌یابد صبی
&&
گرچه با عقلست در ظاهر ابی
\\
کودک دیوانه بازی کی کند
&&
جزو باید تا که کل را فی کند
\\
\end{longtable}
\end{center}
