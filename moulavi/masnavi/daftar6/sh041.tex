\begin{center}
\section*{بخش ۴۱ - قصهٔ درویشی کی از آن خانه هرچه می‌خواست می‌گفت نیست}
\label{sec:sh041}
\addcontentsline{toc}{section}{\nameref{sec:sh041}}
\begin{longtable}{l p{0.5cm} r}
سایلی آمد به سوی خانه‌ای
&&
خشک نانه خواست یا تر نانه‌ای
\\
گفت صاحب‌خانه نان اینجا کجاست
&&
خیره‌ای کی این دکان نانباست
\\
گفت باری اندکی پیهم بیاب
&&
گفت آخر نیست دکان قصاب
\\
گفت پارهٔ آرد ده ای کدخدا
&&
گفت پنداری که هست این آسیا
\\
گفت باری آب ده از مکرعه
&&
گفت آخر نیست جو یا مشرعه
\\
هر چه او درخواست از نان یا سبوس
&&
چربکی می‌گفت و می‌کردش فسوس
\\
آن گدا در رفت و دامن بر کشید
&&
اندر آن خانه بحسبت خواست رید
\\
گفت هی هی گفت تن زن ای دژم
&&
تا درین ویرانه خود فارغ کنم
\\
چون درینجا نیست وجه زیستن
&&
بر چنین خانه بباید ریستن
\\
چون نه‌ای بازی که گیری تو شکار
&&
دست آموز شکار شهریار
\\
نیستی طاوس با صد نقش بند
&&
که به نقشت چشمها روشن کنند
\\
هم نه‌ای طوطی که چون قندت دهند
&&
گوش سوی گفت شیرینت نهند
\\
هم نه‌ای بلبل که عاشق‌وار زار
&&
خوش بنالی در چمن یا لاله‌زار
\\
هم نه‌ای هدهد که پیکیها کنی
&&
نه چو لک‌لک که وطن بالا کنی
\\
در چه کاری تو و بهر چت خرند
&&
تو چه مرغی و ترا با چه خورند
\\
زین دکان با مکاسان برتر آ
&&
تا دکان فضل که الله اشتری
\\
کاله‌ای که هیچ خلقش ننگرید
&&
از خلاقت آن کریم آن را خرید
\\
هیچ قلبی پیش او مردود نیست
&&
زانک قصدش از خریدن سود نیست
\\
\end{longtable}
\end{center}
