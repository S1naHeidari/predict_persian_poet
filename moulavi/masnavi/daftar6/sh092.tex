\begin{center}
\section*{بخش ۹۲ - قصهٔ عبدالغوث و ربودن پریان او را و سالها میان پریان ساکن شدن او و بعد از سالها آمدن او به شهر و فرزندان خویش را باز ناشکیفتن او از آن پریان بحکم جنسیت و همدلی او با ایشان}
\label{sec:sh092}
\addcontentsline{toc}{section}{\nameref{sec:sh092}}
\begin{longtable}{l p{0.5cm} r}
بود عبدالغوث هم‌جنس پری
&&
چون پری نه سال در پنهان‌پری
\\
شد زنش را نسل از شوی دگر
&&
وآن یتیمانش ز مرگش در سمر
\\
که مرورا گرگ زد یا ره‌زنی
&&
یا فتاد اندر چهی یا مکمنی
\\
جمله فرزندانش در اشغال مست
&&
خود نگفتندی که بابایی بدست
\\
بعد نه سال آمد او هم عاریه
&&
گشت پیدا باز شد متواریه
\\
یک مهی مهمان فرزندان خویش
&&
بود و زان پس کس ندیدش رنگ بیش
\\
برد هم جنسی پریانش چنان
&&
که رباید روح را زخم سنان
\\
چون بهشتی جنس جنت آمدست
&&
هم ز جنسیت شود یزدان‌پرست
\\
نه نبی فرمود جود و محمده
&&
شاخ جنت دان به دنیا آمده
\\
مهرها را جمله جنس مهر خوان
&&
قهرها را جمله جنس قهر دان
\\
لاابالی لا ابالی آورد
&&
زانک جنس هم بوند اندر خرد
\\
بود جنسیت در ادریس از نجوم
&&
هشت سال او با زحل بد در قدوم
\\
در مشارق در مغارب یار او
&&
هم‌حدیث و محرم آثار او
\\
بعد غیبت چونک آورد او قدوم
&&
در زمین می‌گفت او درس نجوم
\\
پیش او استارگان خوش صف زده
&&
اختران در درس او حاضر شده
\\
آنچنان که خلق آواز نجوم
&&
می‌شنیدند از خصوص و از عموم
\\
جذب جنسیت کشیده تا زمین
&&
اختران را پیش او کرده مبین
\\
هر یکی نام خود و احوال خود
&&
باز گفته پیش او شرح رصد
\\
چیست جنسیت یکی نوع نظر
&&
که بدان یابند ره در هم‌دگر
\\
آن نظر که کرد حق در وی نهان
&&
چون نهد در تو تو گردی جنس آن
\\
هر طرف چه می‌کشد تن را نظر
&&
بی‌خبر را کی کشاند با خبر
\\
چونک اندر مرد خوی زن نهد
&&
او مخنث گردد و گان می‌دهد
\\
چون نهد در زن خدا خوی نری
&&
طالب زن گردد آن زن سعتری
\\
چون نهد در تو صفات جبرئیل
&&
هم‌چو فرخی بر هواجویی سبیل
\\
منتظر بنهاده دیده در هوا
&&
از زمین بیگانه عاشق بر سما
\\
چون نهد در تو صفت‌های خری
&&
صد پرت گر هست بر آخر پری
\\
از پی صورت نیامد موش خوار
&&
از خبیثی شد زبون موش‌خوار
\\
طعمه‌جوی و خاین و ظلمت‌پرست
&&
از پنیر و فستق و دوشاب مست
\\
باز اشهب را چو باشد خوی موش
&&
ننگ موشان باشد و عار وحوش
\\
خوی آن هاروت و ماروت ای پسر
&&
چون بگشت و دادشان خوی بشر
\\
در فتادند از لنحن الصافون
&&
در چه بابل ببسته سرنگون
\\
لوح محفوظ از نظرشان دور شد
&&
لوح ایشان ساحر و مسحور شد
\\
پر همان و سر همان هیکل همان
&&
موسیی بر عرش و فرعونی مهان
\\
در پی خو باش و با خوش‌خو نشین
&&
خوپذیری روغن گل را ببین
\\
خاک گور از مرد هم یابد شرف
&&
تا نهد بر گور او دل روی و کف
\\
خاک از همسایگی جسم پاک
&&
چون مشرف آمد و اقبال‌ناک
\\
پس تو هم الجار ثم الدار گو
&&
گر دلی داری برو دلدار جو
\\
خاک او هم‌سیرت جان می‌شود
&&
سرمهٔ چشم عزیزان می‌شود
\\
ای بسا در گور خفته خاک‌وار
&&
به ز صد احیا به نفع و انتشار
\\
سایه برده او و خاکش سایه‌مند
&&
صد هزاران زنده در سایهٔ ویند
\\
\end{longtable}
\end{center}
