\begin{center}
\section*{بخش ۶۴ - باقی قصهٔ فقیر روزی‌طلب بی‌واسطهٔ کسب}
\label{sec:sh064}
\addcontentsline{toc}{section}{\nameref{sec:sh064}}
\begin{longtable}{l p{0.5cm} r}
آن یکی بیچارهٔ مفلس ز درد
&&
که ز بی‌چیزی هزاران زهر خورد
\\
لابه کردی در نماز و در دعا
&&
کای خداوند و نگهبان رعا
\\
بی ز جهدی آفریدی مر مرا
&&
بی فن من روزیم ده زین سرا
\\
پنج گوهر دادیم در درج سر
&&
پنج حس دیگری هم مستتر
\\
لا یعد این داد و لا یحصی ز تو
&&
من کلیلم از بیانش شرم‌رو
\\
چونک در خلاقیم تنها توی
&&
کار رزاقیم تو کن مستوی
\\
سالها زو این دعا بسیار شد
&&
عاقبت زاری او بر کار شد
\\
هم‌چو آن شخصی که روزی حلال
&&
از خدا می‌خواست بی‌کسب و کلال
\\
گاو آوردش سعادت عاقبت
&&
عهد داود لدنی معدلت
\\
این متیم نیز زاریها نمود
&&
هم ز میدان اجابت گو ربود
\\
گاه بدظن می‌شدی اندر دعا
&&
از پی تاخیر پاداش و جزا
\\
باز ارجاء خداوند کریم
&&
در دلش بشار گشتی و زعیم
\\
چون شدی نومید در جهد از کلال
&&
از جناب حق شنیدی که تعال
\\
خافضست و رافعست این کردگار
&&
بی ازین دو بر نیاید هیچ کار
\\
خفض ارضی بین و رفع آسمان
&&
بی ازین دو نیست دورانش ای فلان
\\
خفض و رفع این زمین نوعی دگر
&&
نیم سالی شوره نیمی سبز و تر
\\
خفض و رفع روزگار با کرب
&&
نوع دیگر نیم روز و نیم شب
\\
خفض و رفع این مزاج ممترج
&&
گاه صحت گاه رنجوری مضج
\\
هم‌چنین دان جمله احوال جهان
&&
قحط و جدب و صلح و جنگ از افتتان
\\
این جهان با این دو پر اندر هواست
&&
زین دو جانها موطن خوف و رجاست
\\
تا جهان لرزان بود مانند برگ
&&
در شمال و در سموم بعث و مرگ
\\
تا خم یک‌رنگی عیسی ما
&&
بشکند نرخ خم صدرنگ را
\\
کان جهان هم‌چون نمکسار آمدست
&&
هر چه آنجا رفت بی‌تلوین شدست
\\
خاک را بین خلق رنگارنگ را
&&
می‌کند یک رنگ اندر گورها
\\
این نمکسار جسوم ظاهرست
&&
خود نمکسار معانی دیگرست
\\
آن نمکسار معانی معنویست
&&
از ازل آن تا ابد اندر نویست
\\
این نوی را کهنگی ضدش بود
&&
آن نوی بی ضد و بی ند و عدد
\\
آنچنان که از صقل نور مصطفی
&&
صد هزاران نوع ظلمت شد ضیا
\\
از جهود و مشرک و ترسا و مغ
&&
جملگی یک‌رنگ شد زان الپ الغ
\\
صد هزاران سایه کوتاه و دراز
&&
شد یکی در نور آن خورشید راز
\\
نه درازی ماند نه کوته نه پهن
&&
گونه گونه سایه در خورشید رهن
\\
لیک یک‌رنگی که اندر محشرست
&&
بر بد و بر نیک کشف و ظاهرست
\\
که معانی آن جهان صورت شود
&&
نقشهامان در خور خصلت شود
\\
گردد آنگه فکر نقش نامه‌ها
&&
این بطانه روی کار جامه‌ها
\\
این زمان سرها مثال گاو پیس
&&
دوک نطق اندر ملل صد رنگ ریس
\\
نوبت صدرنگیست و صددلی
&&
عالم یک رنگ کی گردد جلی
\\
نوبت زنگست رومی شد نهان
&&
این شبست و آفتاب اندر رهان
\\
نوبت گرگست و یوسف زیر چاه
&&
نوبت قبطست و فرعونست شاه
\\
تا ز رزق بی‌دریغ خیره‌خند
&&
این سگان را حصه باشد روز چند
\\
در درون بیشه شیران منتظر
&&
تا شود امر تعالوا منتشر
\\
پس برون آیند آن شیران ز مرج
&&
بی‌حجابی حق نماید دخل و خرج
\\
جوهر انسان بگیرد بر و بحر
&&
پیسه گاوان بسملان آن روز نحر
\\
روز نحر رستخیز سهمناک
&&
مؤمنان را عید و گاوان را هلاک
\\
جملهٔ مرغان آب آن روز نحر
&&
هم‌چو کشتیها روان بر روی بحر
\\
تا که یهلک من هلک عن بینه
&&
تا که ینجو من نجا واستیقنه
\\
تا که بازان جانب سلطان روند
&&
تا که زاغان سوی گورستان روند
\\
که استخوان و اجزاء سرگین هم‌چو نان
&&
نقل زاغان آمدست اندر جهان
\\
قند حکمت از کجا زاغ از کجا
&&
کرم سرگین از کجا باغ از کجا
\\
نیست لایق غزو نفس و مرد غر
&&
نیست لایق عود و مشک و کون خر
\\
چون غزا ندهد زنان را هیچ دست
&&
کی دهد آنک جهاد اکبرست
\\
جز بنادر در تن زن رستمی
&&
گشته باشد خفیه هم‌چون مریمی
\\
آنچنان که در تن مردان زنان
&&
خفیه‌اند و ماده از ضعف جنان
\\
آن جهان صورت شود آن مادگی
&&
هر که در مردی ندید آمادگی
\\
روز عدل و عدل داد در خورست
&&
کفش آن پا کلاه آن سرست
\\
تا به مطلب در رسد هر طالبی
&&
تا به غرب خود رود هر غاربی
\\
نیست هر مطلوب از طالب دریغ
&&
جفت تابش شمس و جفت آب میغ
\\
هست دنیا قهرخانهٔ کردگار
&&
قهر بین چون قهر کردی اختیار
\\
استخوان و موی مقهوران نگر
&&
تیغ قهر افکنده اندر بحر و بر
\\
پر و پای مرغ بین بر گرد دام
&&
شرح قهر حق کننده بی‌کلام
\\
مرد او بر جای خرپشته نشاند
&&
وآنک کهنه گشت هم پشته نماند
\\
هر کسی را جفت کرده عدل حق
&&
پیل را با پیل و بق را جنس بق
\\
مونس احمد به مجلس چار یار
&&
مونس بوجهل عتبه و ذوالخمار
\\
کعبهٔ جبریل و جانها سدره‌ای
&&
قبلهٔ عبدالبطون شد سفره‌ای
\\
قبلهٔ عارف بود نور وصال
&&
قبلهٔ عقل مفلسف شد خیال
\\
قبلهٔ زاهد بود یزدان بر
&&
قبلهٔ مطمع بود همیان زر
\\
قبلهٔ معنی‌وران صبر و درنگ
&&
قبلهٔ صورت‌پرستان نقش سنگ
\\
قبلهٔ باطن‌نشینان ذوالمنن
&&
قبلهٔ ظاهرپرستان روی زن
\\
هم‌چنین برمی‌شمر تازه و کهن
&&
ور ملولی رو تو کار خویش کن
\\
رزق ما در کاس زرین شد عقار
&&
وآن سگان را آب تتماج و تغار
\\
لایق آنک بدو خو داده‌ایم
&&
در خور آن رزق بفرستاده‌ایم
\\
خوی آن را عاشق نان کرده‌ایم
&&
خوی این را مست جانان کرده‌ایم
\\
چون به خوی خود خوشی و خرمی
&&
پس چه از درخورد خویت می‌رمی
\\
مادگی خوش آمدت چادر بگیر
&&
رستمی خوش آمدت خنجر بگیر
\\
این سخن پایان ندارد وآن فقیر
&&
گشته است از زخم درویشی عقیر
\\
\end{longtable}
\end{center}
