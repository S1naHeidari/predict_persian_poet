\begin{center}
\section*{بخش ۴۴ - رجوع به قصهٔ رنجور}
\label{sec:sh044}
\addcontentsline{toc}{section}{\nameref{sec:sh044}}
\begin{longtable}{l p{0.5cm} r}
باز گرد و قصهٔ رنجور گو
&&
با طبیب آگه ستارخو
\\
نبض او بگرفت و واقف شد ز حال
&&
که امید صحت او بد محال
\\
گفت هر چت دل بخواهد آن بکن
&&
تا رود از جسمت این رنج کهن
\\
هرچه خواهد خاطر تو وا مگیر
&&
تا نگردد صبر و پرهیزت زحیر
\\
صبر و پرهیز این مرض را دان زیان
&&
هرچه خواهد دل در آرش در میان
\\
این چنین رنجور را گفت ای عمو
&&
حق تعالی اعملوا ما شئتم
\\
گفت رو هین خیر بادت جان عم
&&
من تماشای لب جو می‌روم
\\
بر مراد دل همی‌گشت او بر آب
&&
تا که صحت را بیابد فتح باب
\\
بر لب جو صوفیی بنشسته بود
&&
دست و رو می‌شست و پاکی می‌فزود
\\
او قفااش دید چون تخییلیی
&&
کرد او را آرزوی سیلیی
\\
بر قفای صوفی حمزه‌پرست
&&
راست می‌کرد از برای صفع دست
\\
کارزو را گر نرانم تا رود
&&
آن طبیبم گفت کان علت شود
\\
سیلیش اندر برم در معرکه
&&
زانک لا تلقوا بایدی تهلکه
\\
تهلکه‌ست این صبر و پرهیز ای فلان
&&
خوش بکوبش تن مزن چون دیگران
\\
چون زدش سیلی برآمد یک طراق
&&
گفت صوفی هی هی ای قواد عاق
\\
خواست صوفی تا دو سه مشتش زند
&&
سبلت و ریشش یکایک بر کند
\\
خلق رنجور دق و بیچاره‌اند
&&
وز خداع دیو سیلی باره‌اند
\\
جمله در ایذای بی‌جرمان حریص
&&
در قفای همدگر جویان نقیص
\\
ای زننده بی‌گناهان را قفا
&&
در قفای خود نمی‌بینی جزا
\\
ای هوا را طب خود پنداشته
&&
بر ضعیفان صفع را بگماشته
\\
بر تو خندید آنک گفتت این دواست
&&
اوست که آدم را به گندم رهنماست
\\
که خورید این دانه او دو مستعین
&&
بهر دارو تا تکونا خالدین
\\
اوش لغزانید و او را زد قفا
&&
آن قفا وا گشت و گشت این را جزا
\\
اوش لغزانید سخت اندر زلق
&&
لیک پشت و دستگیرش بود حق
\\
کوه بود آدم اگر پر مار شد
&&
کان تریاقست و بی‌اضرار شد
\\
تو که تریاقی نداری ذره‌ای
&&
از خلاص خود چرایی غره‌ای
\\
آن توکل کو خلیلانه ترا
&&
وآن کرامت چون کلیمت از کجا
\\
تا نبرد تیغت اسمعیل را
&&
تا کنی شه‌راه قعر نیل را
\\
گر سعیدی از مناره اوفتید
&&
بادش اندر جامه افتاد و رهید
\\
چون یقینت نیست آن بخت ای حسن
&&
تو چرا بر باد دادی خویشتن
\\
زین مناره صد هزاران هم‌چو عاد
&&
در فتادند و سر و سر باد داد
\\
سرنگون افتادگان را زین منار
&&
می‌نگر تو صد هزار اندر هزار
\\
تو رسن‌بازی نمیدانی یقین
&&
شکر پاها گوی و می‌رو بر زمین
\\
پر مساز از کاغذ و از که مپر
&&
که در آن سودا بسی رفتست سر
\\
گرچه آن صوفی پر آتش شد ز خشم
&&
لیک او بر عاقبت انداخت چشم
\\
اول صف بر کسی ماندم به کام
&&
کو نگیرد دانه بیند بند دام
\\
حبذا دو چشم پایان بین راد
&&
که نگه دارند تن را از فساد
\\
آن ز پایان‌دید احمد بود کو
&&
دید دوزخ را همین‌جا مو به مو
\\
دید عرش و کرسی و جنات را
&&
تا درید او پردهٔ غفلات را
\\
گر همی‌خواهی سلامت از ضرر
&&
چشم ز اول بند و پایان را نگر
\\
تا عدمها ار ببینی جمله هست
&&
هستها را بنگری محسوس پست
\\
این ببین باری که هر کش عقل هست
&&
روز و شب در جست و جوی نیستست
\\
در گدایی طالب جودی که نیست
&&
بر دکانها طالب سودی که نیست
\\
در مزارع طالب دخلی که نیست
&&
در مغارس طالب نخلی که نیست
\\
در مدارس طالب علمی که نیست
&&
در صوامع طالب حلمی که نیست
\\
هستها را سوی پس افکنده‌اند
&&
نیستها را طالبند و بنده‌اند
\\
زانک کان و مخزن صنع خدا
&&
نیست غیر نیستی در انجلا
\\
پیش ازین رمزی بگفتستیم ازین
&&
این و آن را تو یکی بین دو مبین
\\
گفته شد که هر صناعت‌گر که رست
&&
در صناعت جایگاه نیست جست
\\
جست بنا موضعی ناساخته
&&
گشته ویران سقفها انداخته
\\
جست سقا کوزای کش آب نیست
&&
وان دروگر خانه‌ای کش باب نیست
\\
وقت صید اندر عدم بد حمله‌شان
&&
از عدم آنگه گریزان جمله‌شان
\\
چون امیدت لاست زو پرهیز چیست
&&
با انیس طمع خود استیز چیست
\\
چون انیس طمع تو آن نیستیست
&&
از فنا و نیست این پرهیز چیست
\\
گر انیس لا نه‌ای ای جان به سر
&&
در کمین لا چرایی منتظر
\\
زانک داری جمله دل برکنده‌ای
&&
شست دل در بحر لا افکنده‌ای
\\
پس گریز از چیست زین بحر مراد
&&
که بشستت صد هزاران صید داد
\\
از چه نام برگ را کردی تو مرگ
&&
جادوی بین که نمودت مرگ برگ
\\
هر دو چشمت بست سحر صنعتش
&&
تا که جان را در چه آمد رغبتش
\\
در خیال او ز مکر کردگار
&&
جمله صحرا فوق چه زهرست و مار
\\
لاجرم چه را پناهی ساختست
&&
تا که مرگ او را به چاه انداختست
\\
اینچ گفتم از غلطهات ای عزیز
&&
هم برین بشنو دم عطار نیز
\\
\end{longtable}
\end{center}
