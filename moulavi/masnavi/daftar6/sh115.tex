\begin{center}
\section*{بخش ۱۱۵ - حکایت امرء القیس کی پادشاه عرب بود و به صورت عظیم به جمال بود یوسف وقت خود بود و زنان عرب چون زلیخا مردهٔ او و او شاعر طبع قفا نبک من ذکری حبیب و منزل چون همه زنان او را به جان می‌جستند ای عجب غزل او و نالهٔ او بهر چه بود مگر دانست کی این‌ها همه تمثال صورتی‌اند کی بر تخته‌های خاک نقش کرده‌اند عاقبت این امرء القیس را حالی پیدا شد کی نیم‌شب از ملک و فرزند گریخت و خود را در دلقی پنهان کرد و از آن اقلیم به اقلیم دیگر رفت در طلب آن کس کی از اقلیم منزه است یختص برحمته من یشاء الی آخره}
\label{sec:sh115}
\addcontentsline{toc}{section}{\nameref{sec:sh115}}
\begin{longtable}{l p{0.5cm} r}
امرء القیس از ممالک خشک‌لب
&&
هم کشیدش عشق از خطهٔ عرب
\\
تا بیامد خشت می‌زد در تبوک
&&
با ملک گفتند شاهی از ملوک
\\
امرء القیس آمدست این‌جا به کد
&&
در شکار عشق و خشتی می‌زند
\\
آن ملک برخاست شب شد پیش او
&&
گفته او را ای ملیک خوب‌رو
\\
یوسف وقتی دو ملکت شد کمال
&&
مر ترا رام از بلاد و از جمال
\\
گشته مردان بندگان از تیغ تو
&&
وان زنان ملک مه بی‌میغ تو
\\
پیش ما باشی تو بخت ما بود
&&
جان ما از وصل تو صد جان شود
\\
هم من و هم ملک من مملوک تو
&&
ای به همت ملک‌ها متروک تو
\\
فلسفه گفتش بسی و او خموش
&&
ناگهان وا کرد از سر روی‌پوش
\\
تا چه گفتش او به گوش از عشق و درد
&&
هم‌چو خود در حال سرگردانش کرد
\\
دست او بگرفت و با او یار شد
&&
او هم از تخت و کمر بیزار شد
\\
تا بلاد دور رفتند این دو شه
&&
عشق یک کرت نکردست این گنه
\\
بر بزرگان شهد و بر طفلانست شیر
&&
او بهر کشتی بود من الاخیر
\\
غیر این دو بس ملوک بی‌شمار
&&
عشقشان از ملک بربود و تبار
\\
جان این سه شه‌بچه هم گرد چین
&&
هم‌چو مرغان گشته هر سو دانه‌چین
\\
زهره نی تا لب گشایند از ضمیر
&&
زانک رازی با خطر بود و خطیر
\\
صد هزاران سر بپولی آن زمان
&&
عشق خشم آلوده زه کرده کمان
\\
عشق خود بی‌خشم در وقت خوشی
&&
خوی دارد دم به دم خیره‌کشی
\\
این بود آن لحظه کو خشنود شد
&&
من چه گویم چونک خشم‌آلود شد
\\
لیک مرج جان فدای شیر او
&&
کش کشد این عشق و این شمشیر او
\\
کشتنی به از هزاران زندگی
&&
سلطنت‌ها مردهٔ این بندگی
\\
با کنایت رازها با هم‌دگر
&&
پست گفتندی به صد خوف و حذر
\\
راز را غیر خدا محرم نبود
&&
آه را جز آسمان هم‌دم نبود
\\
اصطلاحاتی میان هم‌دگر
&&
داشتندی بهر ایراد خبر
\\
زین لسان الطیر عام آموختند
&&
طمطراق و سروری اندوختند
\\
صورت آواز مرغست آن کلام
&&
غافلست از حال مرغان مرد خام
\\
کو سلیمانی که داند لحن طیر
&&
دیو گرچه ملک گیرد هست غیر
\\
دیو بر شبه سلیمان کرد ایست
&&
علم مکرش هست و علمناش نیست
\\
چون سلیمان از خدا بشاش بود
&&
منطق الطیری ز علمناش بود
\\
تو از آن مرغ هوایی فهم کن
&&
که ندیدستی طیور من لدن
\\
جای سیمرغان بود آن سوی قاف
&&
هر خیالی را نباشد دست‌باف
\\
جز خیالی را که دید آن اتفاق
&&
آنگهش بعدالعیان افتد فراق
\\
نه فراق قطع بهر مصلحت
&&
که آمنست از هر فراق آن منقبت
\\
بهر استبقاء آن روحی جسد
&&
آفتاب از برف یک‌دم درکشد
\\
بهر جان خویش جو زیشان صلاح
&&
هین مدزد از حرف ایشان اصطلاح
\\
آن زلیخا از سپندان تا به عود
&&
نام جمله چیز یوسف کرده بود
\\
نام او در نامها مکتوم کرد
&&
محرمان را سر آن معلوم کرد
\\
چون بگفتی موم ز آتش نرم شد
&&
این بدی کان یار با ما گرم شد
\\
ور بگفتی مه برآمد بنگرید
&&
ور بگفتی سبز شد آن شاخ بید
\\
ور بگفتی برگها خوش می‌طپند
&&
ور بگفتی خوش همی‌سوزد سپند
\\
ور بگفتی گل به بلبل راز گفت
&&
ور بگفتی شه سر شهناز گفت
\\
ور بگفتی چه همایونست بخت
&&
ور بگفتی که بر افشانید رخت
\\
ور بگفتی که سقا آورد آب
&&
ور بگفتی که بر آمد آفتاب
\\
ور بگفتی دوش دیگی پخته‌اند
&&
یا حوایج از پزش یک لخته‌اند
\\
ور بگفتی هست نانها بی‌نمک
&&
ور بگفتی عکس می‌گردد فلک
\\
ور بگفتی که به درد آمد سرم
&&
ور بگفتی درد سر شد خوشترم
\\
گر ستودی اعتناق او بدی
&&
ور نکوهیدی فراق او بدی
\\
صد هزاران نام گر بر هم زدی
&&
قصد او و خواه او یوسف بدی
\\
گرسنه بودی چو گفتی نام او
&&
می‌شدی او سیر و مست جام او
\\
تشنگیش از نام او ساکن شدی
&&
نام یوسف شربت باطن شدی
\\
ور بدی دردیش زان نام بلند
&&
درد او در حال گشتی سودمند
\\
وقت سرما بودی او را پوستین
&&
این کند در عشق نام دوست این
\\
عام می‌خوانند هر دم نام پاک
&&
این عمل نکند چو نبود عشقناک
\\
آنچ عیسی کرده بود از نام هو
&&
می‌شدی پیدا ورا از نام او
\\
چونک با حق متصل گردید جان
&&
ذکر آن اینست و ذکر اینست آن
\\
خالی از خود بود و پر از عشق دوست
&&
پس ز کوزه آن تلابد که دروست
\\
خنده بوی زعفران وصل داد
&&
گریه بوهای پیاز آن بعاد
\\
هر یکی را هست در دل صد مراد
&&
این نباشد مذهب عشق و وداد
\\
یار آمد عشق را روز آفتاب
&&
آفتاب آن روی را هم‌چون نقاب
\\
آنک نشناسد نقاب از روی یار
&&
عابد الشمس است دست از وی بدار
\\
روز او و روزی عاشق هم او
&&
دل همو دلسوزی عاشق هم او
\\
ماهیان را نقد شد از عین آب
&&
نان و آب و جامه و دارو و خواب
\\
هم‌چو طفلست او ز پستان شیرگیر
&&
او نداند در دو عالم غیر شیر
\\
طفل داند هم نداند شیر را
&&
راه نبود این طرف تدبیر را
\\
گیج کرد این گردنامه روح را
&&
تا بیابد فاتح و مفتوح را
\\
گیج نبود در روش بلک اندرو
&&
حاملش دریا بود نه سیل و جو
\\
چون بیابد او که یابد گم شود
&&
هم‌چو سیلی غرقهٔ قلزم شود
\\
دانه گم شد آنگهی او تین بود
&&
تا نمردی زر ندادم این بود
\\
\end{longtable}
\end{center}
