\begin{center}
\section*{بخش ۹۵ - رجوع کردن به حکایت آن شخص وام کرده و آمدن او به امید عنایت آن محتسب سوی تبریز}
\label{sec:sh095}
\addcontentsline{toc}{section}{\nameref{sec:sh095}}
\begin{longtable}{l p{0.5cm} r}
آن غریب ممتحن از بیم وام
&&
در ره آمد سوی آن دارالسلام
\\
شد سوی تبریز و کوی گلستان
&&
خفته اومیدش فراز گل ستان
\\
زد ز دارالملک تبریز سنی
&&
بر امیدش روشنی بر روشنی
\\
جانش خندان شد از آن روضهٔ رجال
&&
از نسیم یوسف و مصر وصال
\\
گفت یا حادی انخ لی ناقتی
&&
جاء اسعادی و طارت فاقتی
\\
ابرکی یا ناقتی طاب الامور
&&
ان تبریزا مناخات الصدور
\\
اسرحی یا ناقتی حول الریاض
&&
ان تبریزا لنا نعم المفاض
\\
ساربانا بار بگشا ز اشتران
&&
شهر تبریزست و کوی گلستان
\\
فر فردوسیست این پالیز را
&&
شعشعهٔ عرشیست این تبریز را
\\
هر زمانی نور روح‌انگیز جان
&&
از فراز عرش بر تبریزیان
\\
چون وثاق محتسب جست آن غریب
&&
خلق گفتندش که بگذشت آن حبیب
\\
او پریر از دار دنیا نقل کرد
&&
مرد و زن از واقعهٔ او روی‌زرد
\\
رفت آن طاوس عرشی سوی عرش
&&
چون رسید از هاتفانش بوی عرش
\\
سایه‌اش گرچه پناه خلق بود
&&
در نوردید آفتابش زود زود
\\
راند او کشتی ازین ساحل پریر
&&
گشته بود آن خواجه زین غم‌خانه سیر
\\
نعره‌ای زد مرد و بیهوش اوفتاد
&&
گوییا او نیز در پی جان بداد
\\
پس گلاب و آب بر رویش زدند
&&
همرهان بر حالتش گریان شدند
\\
تا به شب بی‌خویش بود و بعد از آن
&&
نیم مرده بازگشت از غیب جان
\\
\end{longtable}
\end{center}
