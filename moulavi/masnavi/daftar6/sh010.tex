\begin{center}
\section*{بخش ۱۰ - وا نمودن پادشاه به امرا و متعصبان در راه ایاز سبب فضیلت و مرتبت و قربت و جامگی او بریشان بر وجهی کی ایشان را حجت و اعتراض نماند}
\label{sec:sh010}
\addcontentsline{toc}{section}{\nameref{sec:sh010}}
\begin{longtable}{l p{0.5cm} r}
چون امیران از حسد جوشان شدند
&&
عاقبت بر شاه خود طعنه زدند
\\
کین ایاز تو ندارد سی خرد
&&
جامگی سی امیر او چون خورد
\\
شاه بیرون رفت با آن سی امیر
&&
سوی صحرا و کهستان صیدگیر
\\
کاروانی دید از دور آن ملک
&&
گفت امیری را برو ای مؤتفک
\\
رو بپرس آن کاروان را بر رصد
&&
کز کدامین شهر اندر می‌رسد
\\
رفت و پرسید و بیامد که ز ری
&&
گفت عزمش تا کجا درماند وی
\\
دیگری را گفت رو ای بوالعلا
&&
باز پرس از کاروان که تا کجا
\\
رفت و آمد گفت تا سوی یمن
&&
گفت رختش چیست هان ای موتمن
\\
ماند حیران گفت با میری دگر
&&
که برو وا پرس رخت آن نفر
\\
باز آمد گفت از هر جنس هست
&&
اغلب آن کاسه‌های رازیست
\\
گفت کی بیرون شدند از شهر ری
&&
ماند حیران آن امیر سست پی
\\
هم‌چنین تا سی امیر و بیشتر
&&
سست‌رای و ناقص اندر کر و فر
\\
گفت امیران را که من روزی جدا
&&
امتحان کردم ایاز خویش را
\\
که بپرس از کاروان تا از کجاست
&&
او برفت این جمله وا پرسید راست
\\
بی‌وصیت بی‌اشارت یک به یک
&&
حالشان دریافت بی ریبی و شک
\\
هر چه زین سی میر اندر سی مقام
&&
کشف شد زو آن به یکدم شد تمام
\\
\end{longtable}
\end{center}
