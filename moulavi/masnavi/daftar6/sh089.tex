\begin{center}
\section*{بخش ۸۹ - حکایت شب دزدان کی سلطان محمود شب در میان ایشان افتاد کی من یکی‌ام از شما و بر احوال ایشان مطلع شدن الی آخره}
\label{sec:sh089}
\addcontentsline{toc}{section}{\nameref{sec:sh089}}
\begin{longtable}{l p{0.5cm} r}
شب چو شه محمود برمی‌گشت فرد
&&
با گروهی قوم دزدان باز خورد
\\
پس بگفتندش کیی ای بوالوفا
&&
گفت شه من هم یکی‌ام از شما
\\
آن یکی گفت ای گروه مکر کیش
&&
تا بگوید هر یکی فرهنگ خویش
\\
تا بگوید با حریفان در سمر
&&
کو چه دارد در جبلت از هنر
\\
آن یکی گفت ای گروه فن‌فروش
&&
هست خاصیت مرا اندر دو گوش
\\
که بدانم سگ چه می‌گوید به بانگ
&&
قوم گفتندش ز دیناری دو دانگ
\\
آن دگر گفت ای گروه زرپرست
&&
جمله خاصیت مرا چشم اندرست
\\
هر که را شب بینم اندر قیروان
&&
روز بشناسم من او را بی‌گمان
\\
گفت یک خاصیتم در بازو است
&&
که زنم من نقبها با زور دست
\\
گفت یک خاصیتم در بینی است
&&
کار من در خاکها بوبینی است
\\
سرالناس معادن داد دست
&&
که رسول آن را پی چه گفته است
\\
من ز خاک تن بدانم کاندر آن
&&
چند نقدست و چه دارد او ز کان
\\
در یکی کان زر بی‌اندازه درج
&&
وان دگر دخلش بود کمتر ز خرج
\\
هم‌چو مجنون بو کنم من خاک را
&&
خاک لیلی را بیابم بی‌خطا
\\
بو کنم دانم ز هر پیراهنی
&&
گر بود یوسف و گر آهرمنی
\\
هم‌چو احمد که برد بو از یمن
&&
زان نصیبی یافت این بینی من
\\
که کدامین خاک همسایهٔ زرست
&&
یا کدامین خاک صفر و ابترست
\\
گفت یک نک خاصیت در پنجه‌ام
&&
که کمندی افکنم طول علم
\\
هم‌چو احمد که کمند انداخت جانش
&&
تا کمندش برد سوی آسمانش
\\
گفت حقش ای کمندانداز بیت
&&
آن ز من دان ما رمیت اذ رمیت
\\
پس بپرسیدند زان شه کای سند
&&
مر ترا خاصیت اندر چه بود
\\
گفت در ریشم بود خاصیتم
&&
که رهانم مجرمان را از نقم
\\
مجرمان را چون به جلادان دهند
&&
چون بجنبد ریش من زیشان رهند
\\
چون بجنبانم به رحمت ریش را
&&
طی کنند آن قتل و آن تشویش را
\\
قوم گفتندش که قطب ما توی
&&
که خلاص روز محنتمان شوی
\\
چون سگی بانگی بزد از سوی راست
&&
گفت می‌گوید که سلطان با شماست
\\
خاک بو کرد آن دگر از ربوه‌ای
&&
گفت این هست از وثاق بیوه‌ای
\\
پس کمند انداخت استاد کمند
&&
تا شدند آن سوی دیوار بلند
\\
جای دیگر خاک را چون بوی کرد
&&
گفت خاک مخزن شاهیست فرد
\\
نقب‌زن زد نقب در مخزن رسید
&&
هر یکی از مخزن اسبابی کشید
\\
بس زر و زربفت و گوهرهای زفت
&&
قوم بردند و نهان کردند تفت
\\
شه معین دید منزل‌گاهشان
&&
حلیه و نام و پناه و راهشان
\\
خویش را دزدید ازیشان بازگشت
&&
روز در دیوان بگفت آن سرگذشت
\\
پس روان گشتند سرهنگان مست
&&
تا که دزدان را گرفتند و ببست
\\
دست‌بسته سوی دیوان آمدند
&&
وز نهیب جان خود لرزان شدند
\\
چونک استادند پیش تخت شاه
&&
یار شبشان بود آن شاه چو ماه
\\
آنک چشمش شب بهرکه انداختی
&&
روز دیدی بی شکش بشناختی
\\
شاه را بر تخت دید و گفت این
&&
بود با ما دوش شب‌گرد و قرین
\\
آنک چندین خاصیت در ریش اوست
&&
این گرفت ما هم از تفتیش اوست
\\
عارف شه بود چشمش لاجرم
&&
بر گشاد از معرفت لب با حشم
\\
گفت و هو معکم این شاه بود
&&
فعل ما می‌دید و سرمان می‌شنود
\\
چشم من ره برد شب شه را شناخت
&&
جمله شب با روی ماهش عشق باخت
\\
امت خود را بخواهم من ازو
&&
کو نگرداند ز عارف هیچ رو
\\
چشم عارف دان امان هر دو کون
&&
که بدو یابید هر بهرام عون
\\
زان محمد شافع هر داغ بود
&&
که ز جز شه چشم او مازاغ بود
\\
در شب دنیا که محجوبست شید
&&
ناظر حق بود و زو بودش امید
\\
از الم نشرح دو چشمش سرمه یافت
&&
دید آنچ جبرئیل آن بر نتافت
\\
مر یتیمی را که سرمه حق کشد
&&
گردد او در یتیم با رشد
\\
نور او بر ذره‌ها غالب شود
&&
آن‌چنان مطلوب را طالب شود
\\
در نظر بودش مقامات العباد
&&
لاجرم نامش خدا شاهد نهاد
\\
آلت شاهد زبان و چشم تیز
&&
که ز شب‌خیزش ندارد سر گریز
\\
گر هزاران مدعی سر بر زند
&&
گوش قاضی جانب شاهد کند
\\
قاضیان را در حکومت این فنست
&&
شاهد ایشان را دو چشم روشنست
\\
گفت شاهد زان به جای دیده است
&&
کو بدیدهٔ بی‌غرض سر دیده است
\\
مدعی دیده‌ست اما با غرض
&&
پرده باشد دیدهٔ دل را غرض
\\
حق همی‌خواهد که تو زاهد شوی
&&
تا غرض بگذاری و شاهد شوی
\\
کین غرضها پردهٔ دیده بود
&&
بر نظر چون پرده پیچیده بود
\\
پس نبیند جمله را با طم و رم
&&
حبک الاشیاء یعمی و یصم
\\
در دلش خورشید چون نوری نشاند
&&
پیشش اختر را مقادیری نماند
\\
پس بدید او بی‌حجاب اسرار را
&&
سیر روح مؤمن و کفار را
\\
در زمین حق را و در چرخ سمی
&&
نیست پنهان‌تر ز روح آدمی
\\
باز کرد از رطب و یابس حق نورد
&&
روح را من امر ربی مهر کرد
\\
پس چو دید آن روح را چشم عزیز
&&
پس برو پنهان نماند هیچ چیز
\\
شاهد مطلق بود در هر نزاع
&&
بشکند گفتش خمار هر صداع
\\
نام حق عدلست و شاهد آن اوست
&&
شاهد عدلست زین رو چشم دوست
\\
منظر حق دل بود در دو سرا
&&
که نظر در شاهد آید شاه را
\\
عشق حق و سر شاهدبازیش
&&
بود مایهٔ جمله پرده‌سازیش
\\
پس از آن لولاک گفت اندر لقا
&&
در شب معراج شاهدباز ما
\\
این قضا بر نیک و بد حاکم بود
&&
بر قضا شاهد نه حاکم می‌شود
\\
شد اسیر آن قضا میر قضا
&&
شاد باش ای چشم‌تیز مرتضی
\\
عارف از معروف بس درخواست کرد
&&
کای رقیب ما تو اندر گرم و سرد
\\
ای مشیر ما تو اندر خیر و شر
&&
از اشارتهات دل‌مان بی‌خبر
\\
ای یرانا لانراه روز و شب
&&
چشم‌بند ما شده دید سبب
\\
چشم من از چشم‌ها بگزیده شد
&&
تا که در شب آفتابم دیده شد
\\
لطف معروف تو بود آن ای بهی
&&
پس کمال البر فی اتمامه
\\
یا رب اتمم نورنا فی الساهره
&&
وانجنا من مفضحات قاهره
\\
یار شب را روز مهجوری مده
&&
جان قربت‌دیده را دوری مده
\\
بعد تو مرگیست با درد و نکال
&&
خاصه بعدی که بود بعد الوصال
\\
آنک دیدستت مکن نادیده‌اش
&&
آب زن بر سبزهٔ بالیده‌اش
\\
من نکردم لا ابالی در روش
&&
تو مکن هم لاابالی در خلش
\\
هین مران از روی خود او را بعید
&&
آنک او یک‌باره آن روی تو دید
\\
دید روی جز تو شد غل گلو
&&
کل شیء ما سوی الله باطل
\\
باطل‌اند و می‌نمایندم رشد
&&
زانک باطل باطلان را می‌کشد
\\
ذره ذره کاندرین ارض و سماست
&&
جنس خود را هر یکی چون کهرباست
\\
معده نان را می‌کشد تا مستقر
&&
می‌کشد مر آب را تف جگر
\\
چشم جذاب بتان زین کویها
&&
مغز جویان از گلستان بویها
\\
زانک حس چشم آمد رنگ کش
&&
مغز و بینی می‌کشد بوهای خوش
\\
زین کششها ای خدای رازدان
&&
تو به جذب لطف خودمان ده امان
\\
غالبی بر جاذبان ای مشتری
&&
شاید ار درماندگان را وا خری
\\
رو به شه آورد چون تشنه به ابر
&&
آنک بود اندر شب قدر آن بدر
\\
چون لسان وجان او بود آن او
&&
آن او با او بود گستاخ‌گو
\\
گفت ما گشتیم چون جان بند طین
&&
آفتاب جان توی در یوم دین
\\
وقت آن شد ای شه مکتوم‌سیر
&&
کز کرم ریشی بجنبانی به خیر
\\
هر یکی خاصیت خود را نمود
&&
آن هنرها جمله بدبختی فزود
\\
آن هنرها گردن ما را ببست
&&
زان مناصب سرنگوساریم و پست
\\
آن هنر فی جیدنا حبل مسد
&&
روز مردن نیست زان فنها مدد
\\
جز همان خاصیت آن خوش‌حواس
&&
که به شب بد چشم او سلطان‌شناس
\\
آن هنرها جمله غول راه بود
&&
غیر چشمی کو ز شه آگاه بود
\\
شاه را شرم از وی آمد روز بار
&&
که به شب بر روی شه بودش نظار
\\
وان سگ آگاه از شاه وداد
&&
خود سگ کهفش لقب باید نهاد
\\
خاصیت در گوش هم نیکو بود
&&
کو به بانگ سگ ز شیر آگه شود
\\
سگ چو بیدارست شب چون پاسبان
&&
بی‌خبر نبود ز شبخیز شهان
\\
هین ز بدنامان نباید ننگ داشت
&&
هوش بر اسرارشان باید گماشت
\\
هر که او یک‌بار خود بدنام شد
&&
خود نباید نام جست و خام شد
\\
ای بسا زر که سیه‌تابش کنند
&&
تا شود آمن ز تاراج و گزند
\\
\end{longtable}
\end{center}
