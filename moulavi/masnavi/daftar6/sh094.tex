\begin{center}
\section*{بخش ۹۴ - آمدن جعفر رضی الله عنه به گرفتن قلعه به تنهایی و مشورت کردن ملک آن قلعه در دفع او و گفتن آن وزیر ملک را کی زنهار تسلیم کن و از  جهل تهور مکن کی این مرد میدست و از حق جمعیت عظیم دارد در جان خویش الی آخره}
\label{sec:sh094}
\addcontentsline{toc}{section}{\nameref{sec:sh094}}
\begin{longtable}{l p{0.5cm} r}
چونک جعفر رفت سوی قلعه‌ای
&&
قلعه پیش کام خشکش جرعه‌ای
\\
یک سواره تاخت تا قلعه بکر
&&
تا در قلعه ببستند از حذر
\\
زهره نه کس را که پیش آید به جنگ
&&
اهل کشتی را چه زهره با نهنگ
\\
روی آورد آن ملک سوی وزیر
&&
که چه چاره‌ست اندرین وقت ای مشیر
\\
گفت آنک ترک گویی کبر و فن
&&
پیش او آیی به شمشیر و کفن
\\
گفت آخر نه یکی مردیست فرد
&&
گفت منگر خوار در فردی مرد
\\
چشم بگشا قلعه را بنگر نکو
&&
هم‌چو سیمابست لرزان پیش او
\\
شسته در زین آن‌چنان محکم‌پیست
&&
گوییا شرقی و غربی با ویست
\\
چند کس هم‌چون فدایی تاختند
&&
خویشتن را پیش او انداختند
\\
هر یکی را او بگرزی می‌فکند
&&
سر نگوسار اندر اقدام سمند
\\
داده بودش صنع حق جمعیتی
&&
که همی‌زد یک تنه بر امتی
\\
چشم من چون دید روی آن قباد
&&
کثرت اعداد از چشمم فتاد
\\
اختران بسیار و خورشید ار یکیست
&&
پیش او بنیاد ایشان مندکیست
\\
گر هزاران موش پیش آرند سر
&&
گربه را نه ترس باشد نه حذر
\\
کی به پیش آیند موشان ای فلان
&&
نیست جمعیت درون جانشان
\\
هست جمعیت به صورتها فشار
&&
جمع معنی خواه هین از کردگار
\\
نیست جمعیت ز بسیاری جسم
&&
جسم را بر باد قایم دان چو اسم
\\
در دل موش ار بدی جمعیتی
&&
جمع گشتی چند موش از حمیتی
\\
بر زدندی چون فدایی حمله‌ای
&&
خویش را بر گربهٔ بی‌مهله‌ای
\\
آن یکی چشمش بکندی از ضراب
&&
وان دگر گوشش دریدی هم به ناب
\\
وان دگر سوراخ کردی پهلوش
&&
از جماعت گم شدی بیرون شوش
\\
لیک جمعیت ندارد جان موش
&&
بجهد از جانش به بانگ گربه هوش
\\
خشک گردد موش زان گربهٔ عیار
&&
گر بود اعداد موشان صد هزار
\\
از رمهٔ انبه چه غم قصاب را
&&
انبهی هش چه بندد خواب را
\\
مالک الملک است جمعیت دهد
&&
شیر را تا بر گلهٔ گوران جهد
\\
صد هزاران گور ده‌شاخ و دلیر
&&
چون عدم باشند پیش صول شیر
\\
مالک الملک است بدهد ملک حسن
&&
یوسفی را تا بود چون ماء مزن
\\
در رخی بنهد شعاع اختری
&&
که شود شاهی غلام دختری
\\
بنهد اندر روی دیگر نور خود
&&
که ببیند نیم‌شب هر نیک و بد
\\
یوسف و موسی ز حق بردند نور
&&
در رخ و رخسار و در ذات الصدور
\\
روی موسی بارقی انگیخته
&&
پیش رو او توبره آویخته
\\
نور رویش آن‌چنان بردی بصر
&&
که زمرد از دو دیدهٔ مار کر
\\
او ز حق در خواسته تا توبره
&&
گردد آن نور قوی را ساتره
\\
توبره گفت از گلیمت ساز هین
&&
کان لباس عارفی آمد امین
\\
کان کسا از نور صبری یافتست
&&
نور جان در تار و پودش تافتست
\\
جز چنین خرقه نخواهد شد صوان
&&
نور ما را بر نتابد غیر آن
\\
کوه قاف ار پیش آید بهرسد
&&
هم‌چو کوه طور نورش بر درد
\\
از کمال قدرت ابدان رجال
&&
یافت اندر نور بی‌چون احتمال
\\
آنچ طورش بر نتابد ذره‌ای
&&
قدرتش جا سازد از قاروره‌ای
\\
گشت مشکات و زجاجی جای نور
&&
که همی‌درد ز نور آن قاف و طور
\\
جسمشان مشکات دان دلشان زجاج
&&
تافته بر عرش و افلاک این سراج
\\
نورشان حیران این نور آمده
&&
چون ستاره زین ضحی فانی شده
\\
زین حکایت کرد آن ختم رسل
&&
از ملیک لا یزال و لم یزل
\\
که نگنجیدم در افلاک و خلا
&&
در عقول و در نفوس با علا
\\
در دل مؤمن بگنجیدم چو ضیف
&&
بی ز چون و بی چگونه بی ز کیف
\\
تا به دلالی آن دل فوق و تحت
&&
یابد از من پادشاهی‌ها و بخت
\\
بی‌چنین آیینه از خوبی من
&&
برنتابد نه زمین و نه زمن
\\
بر دو کون اسپ ترحم تاختیم
&&
پس عریض آیینه‌ای بر ساختیم
\\
هر دمی زین آینه پنجاه عرس
&&
بشنو آیینه ولی شرحش مپرس
\\
حاصل این کزلبس خویشش پرده ساخت
&&
که نفوذ آن قمر را می‌شناخت
\\
گر بدی پرده ز غیر لبس او
&&
پاره گشتی گر بدی کوه دوتو
\\
ز آهنین دیوارها نافذ شدی
&&
توبره با نور حق چه فن زدی
\\
گشته بود آن توبره صاحب تفی
&&
بود وقت شور خرقهٔ عارفی
\\
زان شود آتش رهین سوخته
&&
کوست با آتش ز پیش آموخته
\\
وز هوا و عشق آن نور رشاد
&&
خود صفورا هر دو دیده باد داد
\\
اولا بر بست یک چشم و بدید
&&
نور روی او و آن چشمش پرید
\\
بعد از آن صبرش نماند و آن دگر
&&
بر گشاد و کرد خرج آن قمر
\\
هم‌چنان مرد مجاهد نان دهد
&&
چون برو زد نور طاعت جان دهد
\\
پس زنی گفتش ز چشم عبهری
&&
که ز دستت رفت حسرت می‌خوری
\\
گفت حسرت می‌خورم که صد هزار
&&
دیده بودی تا همی‌کردم نثار
\\
روزن چشمم ز مه ویران شدست
&&
لیک مه چون گنج در ویران نشست
\\
کی گذارد گنج کین ویرانه‌ام
&&
یاد آرد از رواق و خانه‌ام
\\
نور روی یوسفی وقت عبور
&&
می‌فتادی در شباک هر قصور
\\
پس بگفتندی درون خانه در
&&
یوسفست این سو به سیران و گذر
\\
زانک بر دیوار دیدندی شعاع
&&
فهم کردندی پس اصحاب بقاع
\\
خانه‌ای را کش دریچه‌ست آن طرف
&&
دارد از سیران آن یوسف شرف
\\
هین دریچه سوی یوسف باز کن
&&
وز شکافش فرجه‌ای آغاز کن
\\
عشق‌ورزی آن دریچه کردنست
&&
کز جمال دوست سینه روشنست
\\
پس هماره روی معشوقه نگر
&&
این به دست تست بشنو ای پدر
\\
راه کن در اندرونها خویش را
&&
دور کن ادراک غیراندیش را
\\
کیمیا داری دوای پوست کن
&&
دشمنان را زین صناعت دوست کن
\\
چون شدی زیبا بدان زیبا رسی
&&
که رهاند روح را از بی‌کسی
\\
پرورش مر باغ جانها را نمش
&&
زنده کرده مردهٔ غم را دمش
\\
نه همه ملک جهان دون دهد
&&
صد هزاران ملک گوناگون دهد
\\
بر سر ملک جمالش داد حق
&&
ملکت تعبیر بی‌درس و سبق
\\
ملکت حسنش سوی زندان کشید
&&
ملکت علمش سوی کیوان کشید
\\
شه غلام او شد از علم و هنر
&&
ملک علم از ملک حسن استوده‌تر
\\
\end{longtable}
\end{center}
