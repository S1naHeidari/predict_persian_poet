\begin{center}
\section*{بخش ۸۲ - مثل}
\label{sec:sh082}
\addcontentsline{toc}{section}{\nameref{sec:sh082}}
\begin{longtable}{l p{0.5cm} r}
سوی جامع می‌شد آن یک شهریار
&&
خلق را می‌زد نقیب و چوبدار
\\
آن یکی را سر شکستی چوب‌زن
&&
و آن دگر را بر دریدی پیرهن
\\
در میانه بی‌دلی ده چوب خورد
&&
بی‌گناهی که برو از راه برد
\\
خون چکان رو کرد با شاه و بگفت
&&
ظلم ظاهر بین چه پرسی از نهفت
\\
خیر تو این است جامع می‌روی
&&
تا چه باشد شر و وزرت ای غوی
\\
یک سلامی نشنود پیر از خسی
&&
تا نپیچد عاقبت از وی بسی
\\
گرگ دریابد ولی را به بود
&&
زانک دریابد ولی را نفس بد
\\
زانک گرگ ارچه که بس استمگریست
&&
لیکش آن فرهنگ و کید و مکر نیست
\\
ورنه کی اندر فتادی او به دام
&&
مکر اندر آدمی باشد تمام
\\
گفت قج با گاو و اشتر ای رفاق
&&
چون چنین افتاد ما را اتفاق
\\
هر یکی تاریخ عمر ابدا کنید
&&
پیرتر اولیست باقی تن زنید
\\
گفت قج مرج من اندر آن عهود
&&
با قج قربان اسمعیل بود
\\
گاو گفتا بوده‌ام من سال‌خورد
&&
جفت آن گاوی کش آدم جفت کرد
\\
جفت آن گاوم که آدم جد خلق
&&
در زراعت بر زمین می‌کرد فلق
\\
چون شنید از گاو و قج اشتر شگفت
&&
سر فرود آورد و آن را برگرفت
\\
در هوا بر داشت آن بند قصیل
&&
اشتر بختی سبک بی‌قال و قیل
\\
که مرا خود حاجت تاریخ نیست
&&
کین چنین جسمی و عالی گردنیست
\\
خود همه کس داند ای جان پدر
&&
که نباشم از شما من خردتر
\\
داند این را هرکه ز اصحاب نهاست
&&
که نهاد من فزون‌تر از شماست
\\
جملگان دانند کین چرخ بلند
&&
هست صد چندان که این خاک نژند
\\
کو گشاد رقعه‌های آسمان
&&
کو نهاد بقعه‌های خاکدان
\\
\end{longtable}
\end{center}
