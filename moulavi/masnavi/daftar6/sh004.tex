\begin{center}
\section*{بخش ۴ - مناجات و پناه جستن به حق از فتنهٔ اختیار و از فتنهٔ اسباب اختیار کی سماوات و ارضین از اختیار و اسباب اختیار شکوهیدند و ترسیدند و خلقت آدمی مولع افتاد بر طلب اختیار و اسباب اختیار خویش چنانک بیمار باشد خود را اختیار کم بیند صحت خواهد کی  سبب اختیارست تا اختیارش بیفزاید و منصب خواهد تا اختیارش بیفزاید و  مهبط قهر حق در امم ماضیه فرط اختیار  و اسباب اختیار بوده است هرگز فرعون بی‌نوا کس ندیده است}
\label{sec:sh004}
\addcontentsline{toc}{section}{\nameref{sec:sh004}}
\begin{longtable}{l p{0.5cm} r}
اولم این جزر و مد از تو رسید
&&
ورنه ساکن بود این بحر ای مجید
\\
هم از آنجا کین تردد دادیم
&&
بی‌تردد کن مرا هم از کرم
\\
ابتلاام می‌کنی آه الغیاث
&&
ای ذکور از ابتلاات چون اناث
\\
تا بکی این ابتلا یا رب مکن
&&
مذهبی‌ام بخش و ده‌مذهب مکن
\\
اشتری‌ام لاغری و پشت ریش
&&
ز اختیار هم‌چو پالان‌شکل خویش
\\
این کژاوه گه شود این سو گران
&&
آن کژاوه گه شود آن سو کشان
\\
بفکن از من حمل ناهموار را
&&
تا ببینم روضهٔ ابرار را
\\
هم‌چو آن اصحاب کهف از باغ جود
&&
می‌چرم ایقاظ نی بل هم رقود
\\
خفته باشم بر یمین یا بر یسار
&&
برنگردم جز چو گو بی‌اختیار
\\
هم به تقلیب تو تا ذات الیمین
&&
یا سوی ذات الشمال ای رب دین
\\
صد هزاران سال بودم در مطار
&&
هم‌چو ذرات هوا بی‌اختیار
\\
گر فراموشم شدست آن وقت و حال
&&
یادگارم هست در خواب ارتحال
\\
می‌رهم زین چارمیخ چارشاخ
&&
می‌جهم در مسرح جان زین مناخ
\\
شیر آن ایام ماضیهای خود
&&
می‌چشم از دایهٔ خواب ای صمد
\\
جمله عالم ز اختیار و هست خود
&&
می‌گریزد در سر سرمست خود
\\
تا دمی از هوشیاری وا رهند
&&
ننگ خمر و زمر بر خود می‌نهند
\\
جمله دانسته کای این هستی فخ است
&&
فکر و ذکر اختیاری دوزخ است
\\
می‌گریزند از خودی در بیخودی
&&
یا به مستی یا به شغل ای مهتدی
\\
نفس را زان نیستی وا می‌کشی
&&
زانک بی‌فرمان شد اندر بیهشی
\\
لیس للجن و لا للانس ان
&&
ینفذوا من حبس اقطار الزمن
\\
لا نفوذ الا بسلطان الهدی
&&
من تجاویف السموات العلی
\\
لا هدی الا بسلطان یقی
&&
من حراس الشهب روح المتقی
\\
هیچ کس را تا نگردد او فنا
&&
نیست ره در بارگاه کبریا
\\
چیست معراج فلک این نیستی
&&
عاشقان را مذهب و دین نیستی
\\
پوستین و چارق آمد از نیاز
&&
در طریق عشق محراب ایاز
\\
گرچه او خود شاه را محبوب بود
&&
ظاهر و باطن لطیف و خوب بود
\\
گشته بی‌کبر و ریا و کینه‌ای
&&
حسن سلطان را رخش آیینه‌ای
\\
چونک از هستی خود او دور شد
&&
منتهای کار او محمود بد
\\
زان قوی‌تر بود تمکین ایاز
&&
که ز خوف کبر کردی احتراز
\\
او مهذب گشته بود و آمده
&&
کبر را و نفس را گردن زده
\\
یا پی تعلیم می‌کرد آن حیل
&&
یا برای حکمتی دور از وجل
\\
یا که دید چارقش زان شد پسند
&&
کز نسیم نیستی هستیست بند
\\
تا گشاید دخمه کان بر نیستیست
&&
تا بیاید آن نسیم عیش و زیست
\\
ملک و مال و اطلس این مرحله
&&
هست بر جان سبک‌رو سلسله
\\
سلسلهٔ زرین بدید و غره گشت
&&
ماند در سوراخ چاهی جان ز دشت
\\
صورتش جنت به معنی دوزخی
&&
افعیی پر زهر و نقشش گل رخی
\\
گرچه مؤمن را سقر ندهد ضرر
&&
لیک هم بهتر بود زانجا گذر
\\
گرچه دوزخ دور دارد زو نکال
&&
لیک جنت به ورا فی کل حال
\\
الحذر ای ناقصان زین گلرخی
&&
که بگاه صحبت آمد دوزخی
\\
\end{longtable}
\end{center}
