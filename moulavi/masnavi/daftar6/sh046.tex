\begin{center}
\section*{بخش ۴۶ - لیس للماضین هم الموت انما لهم حسره الموت}
\label{sec:sh046}
\addcontentsline{toc}{section}{\nameref{sec:sh046}}
\begin{longtable}{l p{0.5cm} r}
راست گفتست آن سپهدار بشر
&&
که هر آنک کرد از دنیا گذر
\\
نیستش درد و دریغ و غبن موت
&&
بلک هستش صد دریغ از بهر فوت
\\
که چرا قبله نکردم مرگ را
&&
مخزن هر دولت و هر برگ را
\\
قبله کردم من همه عمر از حول
&&
آن خیالاتی که گم شد در اجل
\\
حسرت آن مردگان از مرگ نیست
&&
زانست کاندر نقشها کردیم ایست
\\
ما ندیدیم این که آن نقش است و کف
&&
کف ز دریا جنبد و یابد علف
\\
چونک بحر افکند کفها را به بر
&&
تو بگورستان رو آن کفها نگر
\\
پس بگو کو جنبش و جولانتان
&&
بحر افکندست در بحرانتان
\\
تا بگویندت به لب نی بل به حال
&&
که ز دریا کن نه از ما این سؤال
\\
نقش چون کف کی بجنبد بی ز موج
&&
خاک بی بادی کجا آید بر اوج
\\
چون غبار نقش دیدی باد بین
&&
کف چو دیدی قلزم ایجاد بین
\\
هین ببین کز تو نظر آید به کار
&&
باقیت شحمی و لحمی پود و تار
\\
شحم تو در شمعها نفزود تاب
&&
لحم تو مخمور را نامد کباب
\\
در گداز این جمله تن را در بصر
&&
در نظر رو در نظر رو در نظر
\\
یک نظر دو گز همی‌بیند ز راه
&&
یک نظر دو کون دید و روی شاه
\\
در میان این دو فرقی بی‌شمار
&&
سرمه جو والله اعلم بالسرار
\\
چون شنیدی شرح بحر نیستی
&&
کوش دایم تا برین بحر ایستی
\\
چونک اصل کارگاه آن نیستیست
&&
که خلا و بی‌نشانست و تهیست
\\
جمله استادان پی اظهار کار
&&
نیستی جویند و جای انکسار
\\
لاجرم استاد استادان صمد
&&
کارگاهش نیستی و لا بود
\\
هر کجا این نیستی افزون‌ترست
&&
کار حق و کارگاهش آن سرست
\\
نیستی چون هست بالایین طبق
&&
بر همه بردند درویشان سبق
\\
خاصه درویشی که شد بی جسم و مال
&&
کار فقر جسم دارد نه سؤال
\\
سایل آن باشد که مال او گداخت
&&
قانع آن باشد که جسم خویش باخت
\\
پس ز درد اکنون شکایت بر مدار
&&
کوست سوی نیست اسپی راهوار
\\
این قدر گفتیم باقی فکر کن
&&
فکر اگر جامد بود رو ذکر کن
\\
ذکر آرد فکر را در اهتزاز
&&
ذکر را خورشید این افسرده ساز
\\
اصل خود جذبه است لیک ای خواجه‌تاش
&&
کار کن موقوف آن جذبه مباش
\\
زانک ترک کار چون نازی بود
&&
ناز کی در خورد جانبازی بود
\\
نه قبول اندیش نه رد ای غلام
&&
امر را و نهی را می‌بین مدام
\\
مرغ جذبه ناگهان پرد ز عش
&&
چون بدیدی صبح شمع آنگه بکش
\\
چشمها چون شد گذاره نور اوست
&&
مغزها می‌بیند او در عین پوست
\\
بیند اندر ذره خورشید بقا
&&
بیند اندر قطره کل بحر را
\\
\end{longtable}
\end{center}
