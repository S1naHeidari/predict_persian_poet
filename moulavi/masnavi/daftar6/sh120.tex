\begin{center}
\section*{بخش ۱۲۰ - رجوع کردن به قصهٔ آن شخص کی به او گنج نشان دادند به مصر و بیان تضرع او از درویشی به حضرت حق}
\label{sec:sh120}
\addcontentsline{toc}{section}{\nameref{sec:sh120}}
\begin{longtable}{l p{0.5cm} r}
مرد میراثی چو خورد و شد فقیر
&&
آمد اندر یا رب و گریه و نفیر
\\
خود کی کوبد این در رحمت‌نثار
&&
که نیابد در اجابت صد بهار
\\
خواب دید او هاتفی گفت او شنید
&&
که غنای تو به مصر آید پدید
\\
رو به مصر آنجا شود کار تو راست
&&
کرد کدیت را قبول او مرتجاست
\\
در فلان موضع یکی گنجی است زفت
&&
در پی آن بایدت تا مصر رفت
\\
بی‌درنگی هین ز بغداد ای نژند
&&
رو به سوی مصر و منبت‌گاه قند
\\
چون ز بغداد آمد او تا سوی مصر
&&
گرم شد پشتش چو دید او روی مصر
\\
بر امید وعدهٔ هاتف که گنج
&&
یابد اندر مصر بهر دفع رنج
\\
در فلان کوی و فلان موضع دفین
&&
هست گنجی سخت نادر بس گزین
\\
لیک نفقه‌ش بیش و کم چیزی نماند
&&
خواست دقی بر عوام‌الناس راند
\\
لیک شرم و همتش دامن گرفت
&&
خویش را در صبر افشردن گرفت
\\
باز نفسش از مجاعت بر طپید
&&
ز انتجاع و خواستن چاره ندید
\\
گفت شب بیرون روم من نرم نرم
&&
تا ز ظلمت نایدم در کدیه شرم
\\
هم‌چو شبکوکی کنم شب ذکر و بانگ
&&
تا رسد از بامهاام نیم دانگ
\\
اندرین اندیشه بیرون شد بکوی
&&
واندرین فکرت همی شد سو به سوی
\\
یک زمان مانع همی‌شد شرم و جاه
&&
یک زمانی جوع می‌گفتش بخواه
\\
پای پیش و پای پس تا ثلث شب
&&
که بخواهم یا بخسپم خشک‌لب
\\
\end{longtable}
\end{center}
