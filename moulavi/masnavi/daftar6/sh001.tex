\begin{center}
\section*{بخش ۱ - تمامت کتاب الموطد الکریم}
\label{sec:sh001}
\addcontentsline{toc}{section}{\nameref{sec:sh001}}
\begin{longtable}{l p{0.5cm} r}
ای حیات دل حسام‌الدین بسی
&&
میل می‌جوشد به قسم سادسی
\\
گشت از جذب چو تو علامه‌ای
&&
در جهان گردان حسامی نامه‌ای
\\
پیش‌کش می‌آرمت ای معنوی
&&
قسم سادس در تمام مثنوی
\\
شش جهت را نور ده زین شش صحف
&&
کی یطوف حوله من لم یطف
\\
عشق را با پنج و با شش کار نیست
&&
مقصد او جز که جذب یار نیست
\\
بوک فیما بعد دستوری رسد
&&
رازهای گفتنی گفته شود
\\
یا بیانی که بود نزدیکتر
&&
زین کنایات دقیق مستتر
\\
راز جز با رازدان انباز نیست
&&
راز اندر گوش منکر راز نیست
\\
لیک دعوت واردست از کردگار
&&
با قبول و ناقبول او را چه کار
\\
نوح نهصد سال دعوت می‌نمود
&&
دم به دم انکار قومش می‌فزود
\\
هیچ از گفتن عنان واپس کشید
&&
هیچ اندر غار خاموشی خزید
\\
گفت از بانگ و علالای سگان
&&
هیچ واگردد ز راهی کاروان
\\
یا شب مهتاب از غوغای سگ
&&
سست گردد بدر را در سیر تگ
\\
مه فشاند نور و سگ عو عو کند
&&
هر کسی بر خلقت خود می‌تند
\\
هر کسی را خدمتی داده قضا
&&
در خور آن گوهرش در ابتلا
\\
چونک نگذارد سگ آن نعرهٔ سقم
&&
من مهم سیران خود را چون هلم
\\
چونک سرکه سرکگی افزون کند
&&
پس شکر را واجب افزونی بود
\\
قهر سرکه لطف هم‌چون انگبین
&&
کین دو باشد رکن هر اسکنجبین
\\
انگبین گر پای کم آرد ز خل
&&
آیند آن اسکنجبین اندر خلل
\\
قوم بر وی سرکه‌ها می‌ریختند
&&
نوح را دریا فزون می‌ریخت قند
\\
قند او را بد مدد از بحر جود
&&
پس ز سرکهٔ اهل عالم می‌فزود
\\
واحد کالالف کی بود آن ولی
&&
بلک صد قرنست آن عبدالعلی
\\
خم که از دریا درو راهی شود
&&
پیش او جیحونها زانو زند
\\
خاصه این دریا که دریاها همه
&&
چون شنیدند این مثال و دمدمه
\\
شد دهانشان تلخ ازین شرم و خجل
&&
که قرین شد نام اعظم با اقل
\\
در قران این جهان با آن جهان
&&
این جهان از شرم می‌گردد جهان
\\
این عبارت تنگ و قاصر رتبتست
&&
ورنه خس را با اخص چه نسبتست
\\
زاغ در رز نعرهٔ زاغان زند
&&
بلبل از آواز خوش کی کم کند
\\
پس خریدارست هر یک را جدا
&&
اندرین بازار یفعل ما یشا
\\
نقل خارستان غذای آتش است
&&
بوی گل قوت دماغ سرخوش است
\\
گر پلیدی پیش ما رسوا بود
&&
خوک و سگ را شکر و حلوا بود
\\
گر پلیدان این پلیدیها کنند
&&
آبها بر پاک کردن می‌تنند
\\
گرچه ماران زهرافشان می‌کنند
&&
ورچه تلخان‌مان پریشان می‌کنند
\\
نحلها بر کو و کندو و شجر
&&
می‌نهند از شهد انبار شکر
\\
زهرها هرچند زهری می‌کنند
&&
زود تریاقاتشان بر می‌کنند
\\
این جهان جنگست کل چون بنگری
&&
ذره با ذره چو دین با کافری
\\
آن یکی ذره همی پرد به چپ
&&
وآن دگر سوی یمین اندر طلب
\\
ذره‌ای بالا و آن دیگر نگون
&&
جنگ فعلیشان ببین اندر رکون
\\
جنگ فعلی هست از جنگ نهان
&&
زین تخالف آن تخالف را بدان
\\
ذره‌ای کان محو شد در آفتاب
&&
جنگ او بیرون شد از وصف و حساب
\\
چون ز ذره محو شد نفس و نفس
&&
جنگش اکنون جنگ خورشیدست بس
\\
رفت از وی جنبش طبع و سکون
&&
از چه از انا الیه راجعون
\\
ما به بحر تو ز خود راجع شدیم
&&
وز رضاع اصل مسترضع شدیم
\\
در فروغ راه ای مانده ز غول
&&
لاف کم زن از اصول ای بی‌اصول
\\
جنگ ما و صلح ما در نور عین
&&
نیست از ما هست بین اصبعین
\\
جنگ طبعی جنگ فعلی جنگ قول
&&
در میان جزوها حربیست هول
\\
این جهان زن جنگ قایم می‌بود
&&
در عناصر در نگر تا حل شود
\\
چار عنصر چار استون قویست
&&
که بدیشان سقف دنیا مستویست
\\
هر ستونی اشکنندهٔ آن دگر
&&
استن آب اشکنندهٔ آن شرر
\\
پس بنای خلق بر اضداد بود
&&
لاجرم ما جنگییم از ضر و سود
\\
هست احوالم خلاف همدگر
&&
هر یکی با هم مخالف در اثر
\\
چونک هر دم راه خود را می‌زنم
&&
با دگر کس سازگاری چون کنم
\\
موج لشکرهای احوالم ببین
&&
هر یکی با دیگری در جنگ و کین
\\
می‌نگر در خود چنین جنگ گران
&&
پس چه مشغولی به جنگ دیگران
\\
یا مگر زین جنگ حقت وا خرد
&&
در جهان صلح یک رنگت برد
\\
آن جهان جز باقی و آباد نیست
&&
زانک آن ترکیب از اضداد نیست
\\
این تفانی از ضد آید ضد را
&&
چون نباشد ضد نبود جز بقا
\\
نفی ضد کرد از بهشت آن بی‌نظیر
&&
که نباشد شمس و ضدش زمهریر
\\
هست بی‌رنگی اصول رنگها
&&
صلحها باشد اصول جنگها
\\
آن جهانست اصل این پرغم وثاق
&&
وصل باشد اصل هر هجر و فراق
\\
این مخالف از چه‌ایم ای خواجه ما
&&
واز چه زاید وحدت این اعداد را
\\
زانک ما فرعیم و چار اضداد اصل
&&
خوی خود در فرع کرد ایجاد اصل
\\
گوهر جان چون ورای فصلهاست
&&
خوی او این نیست خوی کبریاست
\\
جنگها بین کان اصول صلحهاست
&&
چون نبی که جنگ او بهر خداست
\\
غالبست و چیر در هر دو جهان
&&
شرح این غالب نگنجد در دهان
\\
آب جیحون را اگر نتوان کشید
&&
هم ز قدر تشنگی نتوان برید
\\
گر شدی عطشان بحر معنوی
&&
فرجه‌ای کن در جزیرهٔ مثنوی
\\
فرجه کن چندانک اندر هر نفس
&&
مثنوی را معنوی بینی و بس
\\
باد که را ز آب جو چون وا کند
&&
آب یک‌رنگی خود پیدا کند
\\
شاخهای تازهٔ مرجان ببین
&&
میوه‌های رسته ز آب جان ببین
\\
چون ز حرف و صوت و دم یکتا شود
&&
آن همه بگذارد و دریا شود
\\
حرف‌گو و حرف‌نوش و حرفها
&&
هر سه جان گردند اندر انتها
\\
نان‌دهنده و نان‌ستان و نان‌پاک
&&
ساده گردند از صور گردند خاک
\\
لیک معنیشان بود در سه مقام
&&
در مراتب هم ممیز هم مدام
\\
خاک شد صورت ولی معنی نشد
&&
هر که گوید شد تو گویش نه نشد
\\
در جهان روح هر سه منتظر
&&
گه ز صورت هارب و گه مستقر
\\
امر آید در صور رو در رود
&&
باز هم از امرش مجرد می‌شود
\\
پس له الخلق و له الامرش بدان
&&
خلق صورت امر جان راکب بر آن
\\
راکب و مرکوب در فرمان شاه
&&
جسم بر درگاه وجان در بارگاه
\\
چونک خواهد که آب آید در سبو
&&
شاه گوید جیش جان را که ارکبوا
\\
باز جانها را چو خواند در علو
&&
بانگ آید از نقیبان که انزلوا
\\
بعد ازین باریک خواهد شد سخن
&&
کم کن آتش هیزمش افزون مکن
\\
تا نجوشد دیگهای خرد زود
&&
دیگ ادراکات خردست و فرود
\\
پاک سبحانی که سیبستان کند
&&
در غمام حرفشان پنهان کنند
\\
زین غمام بانگ و حرف و گفت و گوی
&&
پرده‌ای کز سیب ناید غیر بوی
\\
باری افزون کش تو این بو را به هوش
&&
تا سوی اصلت برد بگرفته گوش
\\
بو نگه‌دار و بپرهیز از زکام
&&
تن بپوش از باد و بود سرد عام
\\
تا نینداید مشامت را ز اثر
&&
ای هواشان از زمستان سردتر
\\
چون جمادند و فسرده و تن‌شگرف
&&
می‌جهد انفاسشان از تل برف
\\
چون زمین زین برف در پوشد کفن
&&
تیغ خورشید حسام‌الدین بزن
\\
هین بر آر از شرق سیف‌الله را
&&
گرم کن زان شرق این درگاه را
\\
برف را خنجر زند آن آفتاب
&&
سیلها ریزد ز کهها بر تراب
\\
زانک لا شرقیست و لا غربیست او
&&
با منجم روز و شب حربیست او
\\
که چرا جز من نجوم بی‌هدی
&&
قبله کردی از لئیمی و عمی
\\
تا خوشت ناید مقال آن امین
&&
در نبی که لا احب الا فلین
\\
از قزح در پیش مه بستی کمر
&&
زان همی رنجی ز وانشق القمر
\\
منکری این را که شمس کورت
&&
شمس پیش تست اعلی‌مرتبت
\\
از ستاره دیده تصریف هوا
&&
ناخوشت آید اذا النجم هوی
\\
خود مؤثرتر نباشد مه ز نان
&&
ای بسا نان که ببرد عرق جان
\\
خود مؤثرتر نباشد زهره زآب
&&
ای بسا آبا که کرد او تن خراب
\\
مهر آن در جان تست و پند دوست
&&
می‌زند بر گوش تو بیرون پوست
\\
پند ما در تو نگیرد ای فلان
&&
پند تو در ما نگیرد هم بدان
\\
جز مگر مفتاح خاص آید ز دوست
&&
که مقالید السموات آن اوست
\\
این سخن هم‌چون ستاره‌ست و قمر
&&
لیک بی‌فرمان حق ندهد اثر
\\
این ستارهٔ بی‌جهت تاثیر او
&&
می‌زند بر گوشهای وحی‌جو
\\
کی بیایید از جهت تا بی‌جهات
&&
تا ندراند شما را گرگ مات
\\
آنچنان که لمعهٔ درپاش اوست
&&
شمس دنیا در صفت خفاش اوست
\\
هفت چرخ ازرقی در رق اوست
&&
پیک ماه اندر تب و در دق اوست
\\
زهره چنگ مسئله در وی زده
&&
مشتری با نقد جان پیش آمده
\\
در هوای دستبوس او زحل
&&
لیک خود را می‌نبیند از محل
\\
دست و پا مریخ چندین خست ازو
&&
وآن عطارد صد قلم بشکست ازو
\\
با منجم این همه انجم به جنگ
&&
کای رها کرده تو جان بگزیده رنگ
\\
جان ویست و ما همه رنگ و رقوم
&&
کوکب هر فکر او جان نجوم
\\
فکر کو آنجا همه نورست پاک
&&
بهر تست این لفظ فکر ای فکرناک
\\
هر ستاره خانه دارد در علا
&&
هیچ خانه در نگنجد نجم ما
\\
جای سوز اندر مکان کی در رود
&&
نور نامحدود را حد کی بود
\\
لیک تمثیلی و تصویری کنند
&&
تا که در یابد ضعیفی عشقمند
\\
مثل نبود لیک باشد آن مثال
&&
تا کند عقل مجمد را گسیل
\\
عقل سر تیزست لیکن پای سست
&&
زانک دل ویران شدست و تن درست
\\
عقلشان در نقل دنیا پیچ پیچ
&&
فکرشان در ترک شهوت هیچ هیچ
\\
صدرشان در وقت دعوی هم‌چو شرق
&&
صبرشان در وقت تقوی هم‌چو برق
\\
عالمی اندر هنرها خودنما
&&
هم‌چو عالم بی‌وفا وقت وفا
\\
وقت خودبینی نگنجد در جهان
&&
در گلو و معده گم گشته چو نان
\\
این همه اوصافشان نیکو شود
&&
بد نماند چونک نیکوجو شود
\\
گر منی گنده بود هم‌چون منی
&&
چون به جان پیوست یابد روشنی
\\
هر جمادی که کند رو در نبات
&&
از درخت بخت او روید حیات
\\
هر نباتی کان به جان رو آورد
&&
خضروار از چشمهٔ حیوان خورد
\\
باز جان چون رو سوی جانان نهد
&&
رخت را در عمر بی‌پایان نهد
\\
\end{longtable}
\end{center}
