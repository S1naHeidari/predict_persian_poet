\begin{center}
\section*{بخش ۴۲ - رجوع به داستان آن کمپیر}
\label{sec:sh042}
\addcontentsline{toc}{section}{\nameref{sec:sh042}}
\begin{longtable}{l p{0.5cm} r}
چون عروسی خواست رفتن آن خریف
&&
موی ابرو پاک کرد آن مستخیف
\\
پیش رو آیینه بگرفت آن عجوز
&&
تا بیاراید رخ و رخسار و پوز
\\
چند گلگونه بمالید از بطر
&&
سفرهٔ رویش نشد پوشیده‌تر
\\
عشرهای مصحف از جا می‌برید
&&
می‌بچفسانید بر رو آن پلید
\\
تا که سفرهٔ روی او پنهان شود
&&
تا نگین حلقهٔ خوبان شود
\\
عشرها بر روی هر جا می‌نهاد
&&
چونک بر می‌بست چادر می‌فتاد
\\
باز او آن عشرها را با خدو
&&
می‌بچفسانید بر اطراف رو
\\
باز چادر راست کردی آن تکین
&&
عشرها افتادی از رو بر زمین
\\
چون بسی می‌کرد فن و آن می‌فتاد
&&
گفت صد لعنت بر آن ابلیس باد
\\
شد مصور آن زمان ابلیس زود
&&
گفت ای قحبهٔ قدید بی‌ورود
\\
من همه عمر این نیندیشیده‌ام
&&
نه ز جز تو قحبه‌ای این دیده‌ام
\\
تخم نادر در فضیحت کاشتی
&&
در جهان تو مصحفی نگذاشتی
\\
صد بلیسی تو خمیس اندر خمیس
&&
ترک من گوی ای عجوزهٔ دردبیس
\\
چند دزدی عشر از علم کتاب
&&
تا شود رویت ملون هم‌چو سیب
\\
چند دزدی حرف مردان خدا
&&
تا فروشی و ستانی مرحبا
\\
رنگ بر بسته ترا گلگون نکرد
&&
شاخ بر بسته فن عرجون نکرد
\\
عاقبت چون چادر مرگت رسد
&&
از رخت این عشرها اندر فتد
\\
چونک آید خیزخیزان رحیل
&&
گم شود زان پس فنون قال و قیل
\\
عالم خاموشی آید پیش بیست
&&
وای آنک در درون انسیش نیست
\\
صیقلی کن یک دو روزی سینه را
&&
دفتر خود ساز آن آیینه را
\\
که ز سایهٔ یوسف صاحب‌قران
&&
شد زلیخای عجوز از سر جوان
\\
می‌شود مبدل به خورشید تموز
&&
آن مزاح بارد برد العجوز
\\
می‌شود مبدل بسوز مریمی
&&
شاخ لب خشکی به نخلی خرمی
\\
ای عجوزه چند کوشی با قضا
&&
نقد جو اکنون رها کن ما مضی
\\
چون رخت را نیست در خوبی امید
&&
خواه گلگونه نه و خواهی مداد
\\
\end{longtable}
\end{center}
