\begin{center}
\section*{بخش ۶۹ - باز دادن شاه گنج‌نامه را به آن فقیر کی بگیر ما از سر این برخاستیم}
\label{sec:sh069}
\addcontentsline{toc}{section}{\nameref{sec:sh069}}
\begin{longtable}{l p{0.5cm} r}
چونک رقعهٔ گنج پر آشوب را
&&
شه مسلم داشت آن مکروب را
\\
گشت آمن او ز خصمان و ز نیش
&&
رفت و می‌پیچید در سودای خویش
\\
یار کرد او عشق درداندیش را
&&
کلب لیسد خویش ریش خویش را
\\
عشق را در پیچش خود یار نیست
&&
محرمش در ده یکی دیار نیست
\\
نیست از عاشق کسی دیوانه‌تر
&&
عقل از سودای او کورست و کر
\\
زآنک این دیوانگی عام نیست
&&
طب را ارشاد این احکام نیست
\\
گر طبیبی را رسد زین گون جنون
&&
دفتر طب را فرو شوید به خون
\\
طب جملهٔ عقلها منقوش اوست
&&
روی جمله دلبران روپوش اوست
\\
روی در روی خود آر ای عشق‌کیش
&&
نیست ای مفتون ترا جز خویش خویش
\\
قبله از دل ساخت آمد در دعا
&&
لیس للانسان الا ما سعی
\\
پیش از آن کو پاسخی بشنیده بود
&&
سالها اندر دعا پیچیده بود
\\
بی‌اجابت بر دعاها می‌تنید
&&
از کرم لبیک پنهان می‌شنید
\\
چونک بی‌دف رقص می‌کرد آن علیل
&&
ز اعتماد جود خلاق جلیل
\\
سوی او نه هاتف و نه پیک بود
&&
گوش اومیدش پر از لبیک بود
\\
بی‌زبان می‌گفت اومیدش تعال
&&
از دلش می‌روفت آن دعوت ملال
\\
آن کبوتر را که بام آموختست
&&
تو مخوان می‌رانش کان پر دوختست
\\
ای ضیاء الحق حسام‌الدین برانش
&&
کز ملاقات تو بر رستست جانش
\\
گر برانی مرغ جانش از گزاف
&&
هم بگرد بام تو آرد طواف
\\
چینه و نقلش همه بر بام تست
&&
پر زنان بر اوج مست دام تست
\\
گر دمی منکر شود دزدانه روح
&&
در ادای شکرت ای فتح و فتوح
\\
شحنهٔ عشق مکرر کینه‌اش
&&
طشت آتش می‌نهد بر سینه‌اش
\\
که بیا سوی مه و بگذر ز گرد
&&
شاه عشقت خواند زوتر باز گرد
\\
گرد این بام و کبوترخانه من
&&
چون کبوتر پر زنم مستانه من
\\
جبرئیل عشقم و سدره‌م توی
&&
من سقیمم عیسی مریم توی
\\
جوش ده آن بحر گوهربار را
&&
خوش بپرس امروز این بیمار را
\\
چون تو آن او شدی بحر آن اوست
&&
گرچه این دم نوبت بحران اوست
\\
این خود آن ناله‌ست کو کرد آشکار
&&
آنچ پنهانست یا رب زینهار
\\
دو دهان داریم گویا هم‌چو نی
&&
یک دهان پنهانست در لبهای وی
\\
یک دهان نالان شده سوی شما
&&
های هویی در فکنده در هوا
\\
لیک داند هر که او را منظرست
&&
که فغان این سری هم زان سرست
\\
دمدمهٔ این نای از دمهای اوست
&&
های هوی روح از هیهای اوست
\\
گر نبودی با لبش نی را سمر
&&
نی جهان را پر نکردی از شکر
\\
با کی خفتی وز چه پهلو خاستی
&&
که چنین پر جوش چون دریاستی
\\
یا ابیت عند ربی خواندی
&&
در دل دریای آتش راندی
\\
نعرهٔ یا نار کونی باردا
&&
عصمت جان تو گشت ای مقتدا
\\
ای ضیاء الحق حسام دین و دل
&&
کی توان اندود خورشیدی به گل
\\
قصد کردستند این گل‌پاره‌ها
&&
که بپوشانند خورشید ترا
\\
در دل که لعلها دلال تست
&&
باغها از خنده مالامال تست
\\
محرم مردیت را کو رستمی
&&
تا ز صد خرمن یکی جو گفتمی
\\
چون بخواهم کز سرت آهی کنم
&&
چون علی سر را فرو چاهی کنم
\\
چونک اخوان را دل کینه‌ورست
&&
یوسفم را قعر چه اولیترست
\\
مست گشتم خویش بر غوغا زنم
&&
چه چه باشد خیمه بر صحرا زنم
\\
بر کف من نه شراب آتشین
&&
وانگه آن کر و فر مستانه بین
\\
منتظر گو باش بی گنج آن فقیر
&&
زآنک ما غرقیم این دم در عصیر
\\
از خدا خواه ای فقیر این دم پناه
&&
از من غرقه شده یاری مخواه
\\
که مرا پروای آن اسناد نیست
&&
از خود و از ریش خویشم یاد نیست
\\
باد سبلت کی بگنجد و آب رو
&&
در شرابی که نگنجد تار مو
\\
در ده ای ساقی یکی رطلی گران
&&
خواجه را از ریش و سبلت وا رهان
\\
نخوتش بر ما سبالی می‌زند
&&
لیک ریش از رشک ما بر می‌کند
\\
مات او و مات او و مات او
&&
که همی‌دانیم تزویرات او
\\
از پس صد سال آنچ آید ازو
&&
پیر می‌بیند معین مو به مو
\\
اندر آیینه چه بیند مرد عام
&&
که نبیند پیر اندر خشت خام
\\
آنچ لحیانی به خانهٔ خود ندید
&&
هست بر کوسه یکایک آن پدید
\\
رو به دریایی که ماهی‌زاده‌ای
&&
هم‌چو خس در ریش چون افتاده‌ای
\\
خس نه‌ای دور از تو رشک گوهری
&&
در میان موج و بحر اولیتری
\\
بحر وحدانست جفت و زوج نیست
&&
گوهر و ماهیش غیر موج نیست
\\
ای محال و ای محال اشراک او
&&
دور از آن دریا و موج پاک او
\\
نیست اندر بحر شرک و پیچ پیچ
&&
لیک با احول چه گویم هیچ هیچ
\\
چونک جفت احولانیم ای شمن
&&
لازم آید مشرکانه دم زدن
\\
آن یکیی زان سوی وصفست و حال
&&
جز دوی ناید به میدان مقال
\\
یا چو احول این دوی را نوش کن
&&
یا دهان بر دوز و خوش خاموش کن
\\
یا به نوبت گه سکوت و گه کلام
&&
احولانه طبل می‌زن والسلام
\\
چون ببینی محرمی گو سر جان
&&
گل ببینی نعره زن چون بلبلان
\\
چون ببینی مشک پر مکر و مجاز
&&
لب ببند و خویشتن را خنب ساز
\\
دشمن آبست پیش او مجنب
&&
ورنه سنگ جهل او بشکست خنب
\\
با سیاستهای جاهل صبر کن
&&
خوش مدارا کن به عقل من لدن
\\
صبر با نااهل اهلان را جلاست
&&
صبر صافی می‌کند هر جا دلیست
\\
آتش نمرود ابراهیم را
&&
صفوت آیینه آمد در جلا
\\
جور کفر نوحیان و صبر نوح
&&
نوح را شد صیقل مرآت روح
\\
\end{longtable}
\end{center}
