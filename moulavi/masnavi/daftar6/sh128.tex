\begin{center}
\section*{بخش ۱۲۸ - آمدن نایب قاضی میان بازار و خریداری کردن صندوق را از جوحی الی آخره}
\label{sec:sh128}
\addcontentsline{toc}{section}{\nameref{sec:sh128}}
\begin{longtable}{l p{0.5cm} r}
نایب آمد گفت صندوقت به چند
&&
گفت نهصد بیشتر زر می‌دهند
\\
من نمی‌آیم فروتر از هزار
&&
گر خریداری گشا کیسه بیار
\\
گفت شرمی دار ای کوته‌نمد
&&
قیمت صندوق خود پیدا بود
\\
گفت بی‌ریت شری خود فاسدیست
&&
بیع ما زیر گلیم این راست نیست
\\
بر گشایم گر نمی‌ارزد مخر
&&
تا نباشد بر تو حیفی ای پدر
\\
گفت ای ستار بر مگشای راز
&&
سرببسته می‌خرم با من بساز
\\
ستر کن تا بر تو ستاری کنند
&&
تا نبینی آمنی بر کس مخند
\\
بس درین صندوق چون تو مانده‌اند
&&
خوش را اندر بلا بنشانده‌اند
\\
آنچ بر تو خواه آن باشد پسند
&&
بر دگر کس آن کن از رنج و گزند
\\
زانک بر مرصاد حق واندر کمین
&&
می‌دهد پاداش پیش از یوم دین
\\
آن عظیم العرش عرش او محیط
&&
تخت دادش بر همه جانها بسیط
\\
گوشهٔ عرشش به تو پیوسته است
&&
هین مجنبان جز بدین و داد دست
\\
تو مراقب باش بر احوال خویش
&&
نوش بین در داد و بعد از ظلم نیش
\\
گفت آری اینچ کردم استم است
&&
لیک هم می‌دان که بادی اظلم است
\\
گفت نایب یک به یک ما بادییم
&&
با سواد وجه اندر شادییم
\\
هم‌چو زنگی کو بود شادان و خوش
&&
او نبیند غیر او بیند رخش
\\
ماجرا بسیار شد در من یزید
&&
داد صد دینار و آن از وی خرید
\\
هر دمی صندوقیی ای بدپسند
&&
هاتفان و غیبیانت می‌خرند
\\
\end{longtable}
\end{center}
