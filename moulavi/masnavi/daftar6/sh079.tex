\begin{center}
\section*{بخش ۷۹ - آواز دادن هاتف مر طالب گنج را و اعلام کردن از حقیقت اسرار آن}
\label{sec:sh079}
\addcontentsline{toc}{section}{\nameref{sec:sh079}}
\begin{longtable}{l p{0.5cm} r}
اندرین بود او که الهام آمدش
&&
کشف شد این مشکلات از ایزدش
\\
کو بگفتت در کمان تیری بنه
&&
کی بگفتندت که اندر کش تو زه
\\
او نگفتت که کمان را سخت‌کش
&&
در کمان نه گفت او نه پر کنش
\\
از فضولی تو کمان افراشتی
&&
صنعت قواسیی بر داشتی
\\
ترک این سخته کمانی رو بگو
&&
در کمان نه تیر و پریدن مجو
\\
چون بیفتد بر کن آنجا می‌طلب
&&
زور بگذار و بزاری جو ذهب
\\
آنچ حقست اقرب از حبل الورید
&&
تو فکنده تیر فکرت را بعید
\\
ای کمان و تیرها بر ساخته
&&
صید نزدیک و تو دور انداخته
\\
هرکه دوراندازتر او دورتر
&&
وز چنین گنجست او مهجورتر
\\
فلسفی خود را از اندیشه بکشت
&&
گو بدو کوراست سوی گنج پشت
\\
گو بدو چندانک افزون می‌دود
&&
از مراد دل جداتر می‌شود
\\
جاهدوا فینا بگفت آن شهریار
&&
جاهدوا عنا نگفت ای بی‌قرار
\\
هم‌چو کنعان کو ز ننگ نوح رفت
&&
بر فراز قلهٔ آن کوه زفت
\\
هرچه افزون‌تر همی‌جست او خلاص
&&
سوی که می‌شد جداتر از مناص
\\
هم‌چو این درویش بهر گنج و کان
&&
هر صباحی سخت‌تر جستی کمان
\\
هر کمانی کو گرفتی سخت‌تر
&&
بود از گنج و نشان بدبخت‌تر
\\
این مثل اندر زمانه جانی است
&&
جان نادانان به رنج ارزانی است
\\
زانک جاهل ننگ دارد ز اوستاد
&&
لاجرم رفت و دکانی نو گشاد
\\
آن دکان بالای استاد ای نگار
&&
گنده و پر کزدمست و پر ز مار
\\
زود ویران کن دکان و بازگرد
&&
سوی سبزه و گلبنان و آب‌خورد
\\
نه چو کنعان کو ز کبر و ناشناخت
&&
از که عاصم سفینهٔ فوز ساخت
\\
علم تیراندازیش آمد حجاب
&&
وان مراد او را بده حاضر به جیب
\\
ای بسا علم و ذکاوات و فطن
&&
گشته ره‌رو را چو غول و راه‌زن
\\
بیشتر اصحاب جنت ابلهند
&&
تا ز شر فیلسوفی می‌رهند
\\
خویش را عریان کن از فضل و فضول
&&
تا کند رحمت به تو هر دم نزول
\\
زیرکی ضد شکستست و نیاز
&&
زیرکی بگذار و با گولی‌بساز
\\
زیرکی دان دام برد و طمع و گاز
&&
تا چه خواهد زیرکی را پاک‌باز
\\
زیرکان با صنعتی قانع شده
&&
ابلهان از صنع در صانع شده
\\
زانک طفل خرد را مادر نهار
&&
دست و پا باشد نهاده بر کنار
\\
\end{longtable}
\end{center}
