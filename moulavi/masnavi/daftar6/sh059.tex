\begin{center}
\section*{بخش ۵۹ - مثل}
\label{sec:sh059}
\addcontentsline{toc}{section}{\nameref{sec:sh059}}
\begin{longtable}{l p{0.5cm} r}
آن یکی می‌شد به ره سوی دکان
&&
پیش ره را بسته دید او از زنان
\\
پای او می‌سوخت از تعجیل و راه
&&
بسته از جوق زنان هم‌چو ماه
\\
رو به یک زن کرد و گفت ای مستهان
&&
هی چه بسیارید ای دخترچگان
\\
رو بدو کرد آن زن و گفت ای امین
&&
هیچ بسیاری ما منکر مبین
\\
بین که با بسیاری ما بر بساط
&&
تنگ می‌آید شما را انبساط
\\
در لواطه می‌فتید از قحط زن
&&
فاعل و مفعول رسوای زمن
\\
تو مبین این واقعات روزگار
&&
کز فلک می‌گردد اینجا ناگوار
\\
تو مبین تحشیر روزی و معاش
&&
تو مبین این قحط و خوف و ارتعاش
\\
بین که با این جمله تلخیهای او
&&
مردهٔ اویید و ناپروای او
\\
رحمتی دان امتحان تلخ را
&&
نقمتی دان ملک مرو و بلخ را
\\
آن براهیم از تلف نگریخت و ماند
&&
این براهیم از شرف بگریخت و راند
\\
آن نسوزد وین بسوزد ای عجب
&&
نعل معکوس است در راه طلب
\\
\end{longtable}
\end{center}
