\begin{center}
\section*{بخش ۹۸ - توزیع کردن پای‌مرد در جملهٔ شهر تبریز و جمع شدن اندک چیز و رفتن آن غریب به تربت محتسب به زیارت و این قصه را بر سر گور او گفتن به طریق نوحه الی آخره}
\label{sec:sh098}
\addcontentsline{toc}{section}{\nameref{sec:sh098}}
\begin{longtable}{l p{0.5cm} r}
واقعهٔ آن وام او مشهور شد
&&
پای مرد از درد او رنجور شد
\\
از پی توزیع گرد شهر گشت
&&
از طمع می‌گفت هر جا سرگذشت
\\
هیچ ناورد از ره کدیه به دست
&&
غیر صد دینار آن کدیه‌پرست
\\
پای مرد آمد بدو دستش گرفت
&&
شد بگور آن کریم بس شگفت
\\
گفت چون توفیق یابد بنده‌ای
&&
که کند مهمانی فرخنده‌ای
\\
مال خود ایثار راه او کند
&&
جاه خود ایثار جاه او کند
\\
شکر او شکر خدا باشد یقین
&&
چون به احسان کرد توفیقش قرین
\\
ترک شکرش ترک شکر حق بود
&&
حق او لا شک به حق ملحق بود
\\
شکر می‌کن مر خدا را در نعم
&&
نیز می‌کن شکر و ذکر خواجه هم
\\
رحمت مادر اگر چه از خداست
&&
خدمت او هم فریضه‌ست و سزاست
\\
زین سبب فرمود حق صلوا علیه
&&
که محمد بود محتال الیه
\\
در قیامت بنده را گوید خدا
&&
هین چه کردی آنچ دادم من ترا
\\
گوید ای رب شکر تو کردم به جان
&&
چون ز تو بود اصل آن روزی و نان
\\
گویدش حق نه نکردی شکر من
&&
چون نکردی شکر آن اکرام‌فن
\\
بر کریمی کرده‌ای ظلم و ستم
&&
نه ز دست او رسیدت نعمتم
\\
چون به گور آن ولی‌نعمت رسید
&&
گشت گریان زار و آمد در نشید
\\
گفت ای پشت و پناه هر نبیل
&&
مرتجی و غوث ابناء السبیل
\\
ای غم ارزاق ما بر خاطرت
&&
ای چو رزق عام احسان و برت
\\
ای فقیران را عشیره و والدین
&&
در خراج و خرج و در ایفاء دین
\\
ای چو بحر از بهر نزدیکان گهر
&&
داده و تحفه سوی دوران مطر
\\
پشت ما گرم از تو بود ای آفتاب
&&
رونق هر قصر و گنج هر خراب
\\
ای در ابرویت ندیده کس گره
&&
ای چو میکائیل راد و رزق‌ده
\\
ای دلت پیوسته با دریای غیب
&&
ای به قاف مکرمت عنقای غیب
\\
یاد ناورده که از مالم چه رفت
&&
سقف قصد همتت هرگز نکفت
\\
ای من و صد هم‌چو من در ماه و سال
&&
مر ترا چون نسل تو گشته عیال
\\
نقد ما و جنس ما و رخت ما
&&
نام ما و فخر ما و بخت ما
\\
تو نمردی ناز و بخت ما بمرد
&&
عیش ما و رزق مستوفی بمرد
\\
واحد کالالف در رزم و کرم
&&
صد چو حاتم گاه ایثار نعم
\\
حاتم ار مرده به مرده می‌دهد
&&
گردگان‌های شمرده می‌دهد
\\
تو حیاتی می‌دهی در هر نفس
&&
کز نفیسی می‌نگنجد در نفس
\\
تو حیاتی می‌دهی بس پایدار
&&
نقد زر بی‌کساد و بی‌شمار
\\
وارثی نا بوده یک خوی ترا
&&
ای فلک سجده کنان کوی ترا
\\
خلق را از گرگ غم لطفت شبان
&&
چون کلیم الله شبان مهربان
\\
گوسفندی از کلیم الله گریخت
&&
پای موسی آبله شد نعل ریخت
\\
در پی او تا به شب در جست و جو
&&
وان رمه غایب شده از چشم او
\\
گوسفند از ماندگی شد سست و ماند
&&
پس کلیم الله گرد از وی فشاند
\\
کف همی‌مالید بر پشت و سرش
&&
می‌نواخت از مهر هم‌چون مادرش
\\
نیم ذره طیرگی و خشم نی
&&
غیر مهر و رحم و آب چشم نی
\\
گفت گیرم بر منت رحمی نبود
&&
طبع تو بر خود چرا استم نمود
\\
با ملایک گفت یزدان آن زمان
&&
که نبوت را نمی‌زیبد فلان
\\
مصطفی فرمود خود که هر نبی
&&
کرد چوپانیش برنا یا صبی
\\
بی‌شبانی کردن و آن امتحان
&&
حق ندادش پیشوایی جهان
\\
گفت سایل هم تو نیز ای پهلوان
&&
گفت من هم بوده‌ام دهری شبان
\\
تا شود پیدا وقار و صبرشان
&&
کردشان پیش از نبوت حق شبان
\\
هر امیری کو شبانی بشر
&&
آن‌چنان آرد که باشد متمر
\\
حلم موسی‌وار اندر رعی خود
&&
او به جا آرد به تدبیر و خرد
\\
لاجرم حقش دهد چوپانیی
&&
بر فراز چرخ مه روحانیی
\\
آنچنان که انبیا را زین رعا
&&
بر کشید و داد رعی اصفیا
\\
خواجه باری تو درین چوپانیت
&&
کردی آنچ کور گردد شانیت
\\
دانم آنجا در مکافات ایزدت
&&
سروری جاودانه بخشدت
\\
بر امید کف چون دریای تو
&&
بر وظیفه دادن و ایفای تو
\\
وام کردم نه هزار از زر گزاف
&&
تو کجایی تا شود این درد صاف
\\
تو کجایی تا که خندان چون چمن
&&
گویی بستان آن و ده چندان ز من
\\
تو کجایی تا مرا خندان کنی
&&
لطف و احسان چون خداوندان کنی
\\
تو کجایی تا بری در مخزنم
&&
تا کنی از وام و فاقه آمنم
\\
من همی‌گویم بس و تو مفضلم
&&
گفته کین هم گیر از بهر دلم
\\
چون همی‌گنجد جهانی زیر طین
&&
چون بگنجد آسمانی در زمین
\\
حاش لله تو برونی زین جهان
&&
هم به وقت زندگی هم این زمان
\\
در هوای غیب مرغی می‌پرد
&&
سایهٔ او بر زمینی می‌زند
\\
جسم سایهٔ سایهٔ سایهٔ دلست
&&
جسم کی اندر خور پایهٔ دلست
\\
مرد خفته روح او چون آفتاب
&&
در فلک تابان و تن در جامه خواب
\\
جان نهان اندر خلا هم‌چون سجاف
&&
تن تقلب می‌کند زیر لحاف
\\
روح چون من امر ربی مختفیست
&&
هر مثالی که بگویم منتفیست
\\
ای عجب کو لعل شکربار تو
&&
وان جوابات خوش و اسرار تو
\\
ای عجب کو آن عقیق قندخا
&&
آن کلید قفل مشکل‌های ما
\\
ای عجب کو آن دم چون ذوالفقار
&&
آنک کردی عقل‌ها را بی‌قرار
\\
چند هم‌چون فاخته کاشانه‌جو
&&
کو و کو و کو و کو و کو و کو
\\
کو همان‌جا که صفات رحمتست
&&
قدرتست و نزهتست و فطنتست
\\
کو همان‌جا که دل و اندیشه‌اش
&&
دایم آن‌جا بد چو شیر و بیشه‌اش
\\
کو همان‌جا که امید مرد و زن
&&
می‌رود در وقت اندوه و حزن
\\
کو همان‌جا که به وقت علتی
&&
چشم پرد بر امید صحتی
\\
آن طرف که بهر دفع زشتیی
&&
باد جویی بهر کشت و کشتیی
\\
آن طرف که دل اشارت می‌کند
&&
چون زبان یا هو عبارت می‌کند
\\
او مع‌الله است بی کو کو همی
&&
کاش جولاهانه ماکو گفتمی
\\
عقل ما کو تا ببیند غرب و شرق
&&
روح‌ها را می‌زند صد گونه برق
\\
جزر و مدش بد به بحری در زبد
&&
منتهی شد جزر و باقی ماند مد
\\
نه هزارم وام و من بی دست‌رس
&&
هست صد دینار ازین توزیع و بس
\\
حق کشیدت ماندم در کش‌مکش
&&
می‌روم نومید ای خاک تو خوش
\\
همتی می‌دار در پر حسرتت
&&
ای همایون روی و دست و همتت
\\
آمدم بر چشمه و اصل عیون
&&
یافتم در وی به جای آب خون
\\
چرخ آن چرخست آن مهتاب نیست
&&
جوی آن جویست آب آن آب نیست
\\
محسنان هستند کو آن مستطاب
&&
اختران هستند کو آن آفتاب
\\
تو شدی سوی خدا ای محترم
&&
پس به سوی حق روم من نیز هم
\\
مجمع و پای علم ماوی القرون
&&
هست حق کل لدینا محضرون
\\
نقش‌ها گر بی‌خبر گر با خبر
&&
در کف نقاش باشد محتصر
\\
دم به دم در صفحهٔ اندیشه‌شان
&&
ثبت و محوی می‌کند آن بی‌نشان
\\
خشم می‌آرد رضا را می‌برد
&&
بخل می‌آرد سخا را می‌برد
\\
نیم لحظه مدرکاتم شام و غدو
&&
هیچ خالی نیست زین اثبات و محو
\\
کوزه‌گر با کوزه باشد کارساز
&&
کوزه از خود کی شود پهن و دراز
\\
چوب در دست دروگر معتکف
&&
ورنه چون گردد بریده و مؤتلف
\\
جامه اندر دست خیاطی بود
&&
ورنه از خود چون بدوزد یا درد
\\
مشک با سقا بود ای منتهی
&&
ورنه از خود چون شود پر یا تهی
\\
هر دمی پر می‌شوی تی می‌شوی
&&
پس بدانک در کف صنع ویی
\\
چشم‌بند از چشم روزی کی رود
&&
صنع از صانع چه سان شیدا شود
\\
چشم‌داری تو به چشم خود نگر
&&
منگر از چشم سفیهی بی‌خبر
\\
گوش داری تو به گوش خود شنو
&&
گوش گولان را چرا باشی گرو
\\
بی ز تقلیدی نظر را پیشه کن
&&
هم برای عقل خود اندیشه کن
\\
\end{longtable}
\end{center}
