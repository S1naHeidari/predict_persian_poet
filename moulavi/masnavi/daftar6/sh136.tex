\begin{center}
\section*{بخش ۱۳۶ - کرامات شیخ شیبان راعی قدس الله روحه العزیز}
\label{sec:sh136}
\addcontentsline{toc}{section}{\nameref{sec:sh136}}
\begin{longtable}{l p{0.5cm} r}
هم‌چو آن شیبان که از گرگ عنید
&&
وقت جمعه بر رعا خط می‌کشید
\\
تا برون ناید از آن خط گوسفند
&&
نه در آید گرگ و دزد با گزند
\\
بر مثال دایرهٔ تعویذ هود
&&
که اندر آن صرصر امان آل بود
\\
هشت روزی اندرین خط تن زنید
&&
وز برون مثله تماشا می‌کنید
\\
بر هوا بردی فکندی بر حجر
&&
تا دریدی لحم و عظم از هم‌دگر
\\
یک گره را بر هوا درهم زدی
&&
تا چو خشخاش استخوان ریزان شدی
\\
آن سیاست را که لرزید آسمان
&&
مثنوی اندر نگنجد شرح آن
\\
گر به طبع این می‌کنی ای باد سرد
&&
گرد خط و دایرهٔ آن هود گرد
\\
ای طبیعی فوق طبع این ملک بین
&&
یا بیا و محو کن از مصحف این
\\
مقریان را منع کن بندی بنه
&&
یا معلم را به مال و سهم ده
\\
عاجزی و خیره کن عجز از کجاست
&&
عجز تو تابی از آن روز جزاست
\\
عجزها داری تو در پیش ای لجوج
&&
وقت شد پنهانیان را نک خروج
\\
خرم آن کین عجز و حیرت قوت اوست
&&
در دو عالم خفته اندر ظل دوست
\\
هم در آخر عجز خود را او بدید
&&
مرده شد دین عجایز را گزید
\\
چون زلیخا یوسفش بر وی بتافت
&&
از عجوزی در جوانی راه یافت
\\
زندگی در مردن و در محنتست
&&
آب حیوان در درون ظلمتست
\\
\end{longtable}
\end{center}
