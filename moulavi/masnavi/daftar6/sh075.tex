\begin{center}
\section*{بخش ۷۵ - حکمت در انی جاعل فی الارض خلیفة}
\label{sec:sh075}
\addcontentsline{toc}{section}{\nameref{sec:sh075}}
\begin{longtable}{l p{0.5cm} r}
پس خلیفه ساخت صاحب‌سینه‌ای
&&
تا بود شاهیش را آیینه‌ای
\\
بس صفای بی‌حدودش داد او
&&
وانگه از ظلمت ضدش بنهاد او
\\
دو علم بر ساخت اسپید و سیاه
&&
آن یکی آدم دگر ابلیس راه
\\
در میان آن دو لشکرگاه زفت
&&
چالش و پیکار آنچ رفت رفت
\\
هم‌چنان دور دوم هابیل شد
&&
ضد نور پاک او قابیل شد
\\
هم‌چنان این دو علم از عدل و جور
&&
تا به نمرود آمد اندر دور دور
\\
ضد ابراهیم گشت و خصم او
&&
وآن دو لشکر کین‌گزار و جنگ‌جو
\\
چون درازی جنگ آمد ناخوشش
&&
فیصل آن هر دو آمد آتشش
\\
پس حکم کرد آتشی را و نکر
&&
تا شود حل مشکل آن دو نفر
\\
دور دور و قرن قرن این دو فریق
&&
تا به فرعون و به موسی شفیق
\\
سالها اندر میانشان حرب بود
&&
چون ز حد رفت و ملولی می‌فزود
\\
آب دریا را حکم سازید حق
&&
تا که ماند کی برد زین دو سبق
\\
هم‌چنان تا دور و طور مصطفی
&&
با ابوجهل آن سپهدار جفا
\\
هم نکر سازید از بهر ثمود
&&
صیحه‌ای که جانشان را در ربود
\\
هم نکر سازید بهر قوم عاد
&&
زود خیزی تیزرو یعنی که باد
\\
هم نکر سازید بر قارون ز کین
&&
در حلیمی این زمین پوشید کین
\\
تا حلیمی زمین شد جمله قهر
&&
برد قارون را و گنجش را به قعر
\\
لقمه‌ای را که ستون این تنست
&&
دفع تیغ جوع نان چون جوشنست
\\
چونک حق قهری نهد در نان تو
&&
چون خناق آن نان بگیرد در گلو
\\
این لباسی که ز سرما شد مجیر
&&
حق دهد او را مزاج زمهریر
\\
تا شود بر تنت این جبهٔ شگرف
&&
سرد هم‌چون یخ گزنده هم‌چو برف
\\
تا گریزی از وشق هم از حریر
&&
زو پناه آری به سوی زمهریر
\\
تو دو قله نیستی یک قله‌ای
&&
غافل از قصهٔ عذاب ظله‌ای
\\
امر حق آمد به شهرستان و ده
&&
خانه و دیوار را سایه مده
\\
مانع باران مباش و آفتاب
&&
تا بدان مرسل شدند امت شتاب
\\
که بمردیم اغلب ای مهتر امان
&&
باقیش از دفتر تفسیر خوان
\\
چون عصا را مار کرد آن چست‌دست
&&
گر ترا عقلیست آن نکته بس است
\\
تو نظر داری ولیک امعانش نیست
&&
چشمهٔ افسرده است و کرده ایست
\\
زین همی گوید نگارندهٔ فکر
&&
که بکن ای بنده امعان نظر
\\
آن نمی‌خواهد که آهن کوب سرد
&&
لیک ای پولاد بر داود گرد
\\
تن بمردت سوی اسرافیل ران
&&
دل فسردت رو به خورشید روان
\\
در خیال از بس که گشتی مکتسی
&&
نک بسوفسطایی بدظن رسی
\\
او خود از لب خرد معزول بود
&&
شد ز حس محروم و معزول از وجود
\\
هین سخن‌خا نوبت لب‌خایی است
&&
گر بگویی خلق را رسوایی است
\\
چیست امعان چشمه را کردن روان
&&
چون ز تن جان رست گویندش روان
\\
آن حکیمی را که جان از بند تن
&&
باز رست و شد روان اندر چمن
\\
دو لقب را او برین هر دو نهاد
&&
بهر فرق ای آفرین بر جانش باد
\\
در بیان آنک بر فرمان رود
&&
گر گلی را خار خواهد آن شود
\\
\end{longtable}
\end{center}
