\begin{center}
\section*{بخش ۱۱۲ - مقالت برادر بزرگین}
\label{sec:sh112}
\addcontentsline{toc}{section}{\nameref{sec:sh112}}
\begin{longtable}{l p{0.5cm} r}
آن بزرگین گفت ای اخوان خیر
&&
ما نه نر بودیم اندر نصح غیر
\\
از حشم هر که به ما کردی گله
&&
از بلا و فقر و خوف و زلزله
\\
ما همی‌گفتیم کم نال از حرج
&&
صبر کن کالصبر مفتاح الفرج
\\
این کلید صبر را اکنون چه شد
&&
ای عجب منسوخ شد قانون چه شد
\\
ما نمی‌گفتیم که اندر کش مکش
&&
اندر آتش هم‌چو زر خندید خوش
\\
مر سپه را وقت تنگاتنگ جنگ
&&
گفته ما که هین مگردانید رنگ
\\
آن زمان که بود اسپان را وطا
&&
جمله سرهای بریده زیر پا
\\
ما سپاه خویش را هی هی کنان
&&
که به پیش آیید قاهر چون سنان
\\
جمله عالم را نشان داده به صبر
&&
زانک صبر آمد چراغ و نور صدر
\\
نوبت ما شد چه خیره‌سر شدیم
&&
چون زنان زشت در چادر شدیم
\\
ای دلی که جمله را کردی تو گرم
&&
گرم کن خود را و از خود دار شرم
\\
ای زبان که جمله را ناصح بدی
&&
نوبت تو گشت از چه تن زدی
\\
ای خرد کو پند شکرخای تو
&&
دور تست این دم چه شد هیهای تو
\\
ای ز دلها برده صد تشویش را
&&
نوبت تو شد بجنبان ریش را
\\
از غری ریش ار کنون دزدیده‌ای
&&
پیش ازین بر ریش خود خندیده‌ای
\\
وقت پند دیگرانی های های
&&
در غم خود چون زنانی وای وای
\\
چون به درد دیگران درمان بدی
&&
درد مهمان تو آمد تن زدی
\\
بانگ بر لشکر زدن بد ساز تو
&&
بانگ بر زن چه گرفت آواز تو
\\
آنچ پنجه سال بافیدی به هوش
&&
زان نسیج خود بغلتانی بپوش
\\
از نوایت گوش یاران بود خوش
&&
دست بیرون آر و گوش خود بکش
\\
سر بدی پیوسته خود را دم مکن
&&
پا و دست و ریش و سبلت گم مکن
\\
بازی آن تست بر روی بساط
&&
خویش را در طبع آر و در نشاط
\\
\end{longtable}
\end{center}
