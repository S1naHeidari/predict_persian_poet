\begin{center}
\section*{بخش ۶۲ - حکایت در تقریر آنک صبر در رنج کار سهل‌تر از صبر در فراق یار بود}
\label{sec:sh062}
\addcontentsline{toc}{section}{\nameref{sec:sh062}}
\begin{longtable}{l p{0.5cm} r}
آن یکی زن شوی خود را گفت هی
&&
ای مروت را به یک ره کرده طی
\\
هیچ تیمارم نمی‌داری چرا
&&
تا بکی باشم درین خواری چرا
\\
گفت شو من نفقه چاره می‌کنم
&&
گرچه عورم دست و پایی می‌زنم
\\
نفقه و کسوه‌ست واجب ای صنم
&&
از منت این هر دو هست و نیست کم
\\
آستین پیرهن بنمود زن
&&
بس درشت و پر وسخ بد پیرهن
\\
گفت از سختی تنم را می‌خورد
&&
کس کسی را کسوه زین سان آورد
\\
گفت ای زن یک سالت می‌کنم
&&
مرد درویشم همین آمد فنم
\\
این درشتست و غلیظ و ناپسند
&&
لیک بندیش ای زن اندیشه‌مند
\\
این درشت و زشت‌تر یا خود طلاق
&&
این ترا مکروه‌تر یا خود فراق
\\
هم‌چنان ای خواجهٔ تشنیع زن
&&
از بلا و فقر و از رنج و محن
\\
لا شک این ترک هوا تلخی‌دهست
&&
لیک از تلخی بعد حق بهست
\\
گر جهاد و صوم سختست و خشن
&&
لیک این بهتر ز بعد ممتحن
\\
رنج کی ماند دمی که ذوالمنن
&&
گویدت چونی تو ای رنجور من
\\
ور نگوید کت نه آن فهم و فن است
&&
لیک آن ذوق تو پرسش کردنست
\\
آن ملیحان که طبیبان دل‌اند
&&
سوی رنجوران به پرسش مایل‌اند
\\
وز حذر از ننگ و از نامی کنند
&&
چاره‌ای سازند و پیغامی کنند
\\
ورنه در دلشان بود آن مفتکر
&&
نیست معشوقی ز عاشق بی‌خبر
\\
ای تو جویای نوادر داستان
&&
هم فسانهٔ عشق‌بازان را بخوان
\\
بس بجوشیدی درین عهد مدید
&&
ترک‌جوشی هم نگشتی ای قدید
\\
دیده‌ای عمری تو داد و داوری
&&
وانگه از نادیدگان ناشی‌تری
\\
هر که شاگردیش کرد استاد شد
&&
تو سپس‌تر رفته‌ای ای کور لد
\\
خود نبود از والدینت اختبار
&&
هم نبودت عبرت از لیل و نهار
\\
\end{longtable}
\end{center}
