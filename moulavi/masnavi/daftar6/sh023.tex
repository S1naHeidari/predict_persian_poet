\begin{center}
\section*{بخش ۲۳ - تشبیه مغفلی کی عمر ضایع کند و وقت مرگ در آن تنگاتنگ توبه و استغفار کردن گیرد به تعزیت داشتن شیعهٔ اهل حلب هر سالی در ایام عاشورا به دروازهٔ انطاکیه و رسیدن غریب شاعر از سفر و پرسیدن کی این غریو چه تعزیه است}
\label{sec:sh023}
\addcontentsline{toc}{section}{\nameref{sec:sh023}}
\begin{longtable}{l p{0.5cm} r}
روز عاشورا همه اهل حلب
&&
باب انطاکیه اندر تا به شب
\\
گرد آید مرد و زن جمعی عظیم
&&
ماتم آن خاندان دارد مقیم
\\
ناله و نوحه کنند اندر بکا
&&
شیعه عاشورا برای کربلا
\\
بشمرند آن ظلمها و امتحان
&&
کز یزید و شمر دید آن خاندان
\\
نعره‌هاشان می‌رود در ویل و وشت
&&
پر همی‌گردد همه صحرا و دشت
\\
یک غریبی شاعری از ره رسید
&&
روز عاشورا و آن افغان شنید
\\
شهر را بگذاشت و آن سوی رای کرد
&&
قصد جست و جوی آن هیهای کرد
\\
پرس پرسان می‌شد اندر افتقاد
&&
چیست این غم بر که این ماتم فتاد
\\
این رئیس زفت باشد که بمرد
&&
این چنین مجمع نباشد کار خرد
\\
نام او و القاب او شرحم دهید
&&
که غریبم من شما اهل دهید
\\
چیست نام و پیشه و اوصاف او
&&
تا بگویم مرثیه ز الطاف او
\\
مرثیه سازم که مرد شاعرم
&&
تا ازینجا برگ و لالنگی برم
\\
آن یکی گفتش که هی دیوانه‌ای
&&
تو نه‌ای شیعه عدو خانه‌ای
\\
روز عاشورا نمی‌دانی که هست
&&
ماتم جانی که از قرنی بهست
\\
پیش مؤمن کی بود این غصه خوار
&&
قدر عشق گوش عشق گوشوار
\\
پیش مؤمن ماتم آن پاک‌روح
&&
شهره‌تر باشد ز صد طوفان نوح
\\
\end{longtable}
\end{center}
