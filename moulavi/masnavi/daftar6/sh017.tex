\begin{center}
\section*{بخش ۱۷ - حکایت آن عاشق کی شب بیامد بر امید وعدهٔ معشوق بدان وثاقی کی اشارت کرده بود و بعضی از شب منتظر ماند و خوابش بربود معشوق آمد بهر انجاز وعده او را خفته یافت جیبش پر جوز کرد و او را خفته گذاشت و بازگشت}
\label{sec:sh017}
\addcontentsline{toc}{section}{\nameref{sec:sh017}}
\begin{longtable}{l p{0.5cm} r}
عاشقی بودست در ایام پیش
&&
پاسبان عهد اندر عهد خویش
\\
سالها در بند وصل ماه خود
&&
شاهمات و مات شاهنشاه خود
\\
عاقبت جوینده یابنده بود
&&
که فرج از صبر زاینده بود
\\
گفت روزی یار او که امشب بیا
&&
که بپختم از پی تو لوبیا
\\
در فلان حجره نشین تا نیم‌شب
&&
تا بیایم نیم‌شب من بی طلب
\\
مرد قربان کرد و نانها بخش کرد
&&
چون پدید آمد مهش از زیر گرد
\\
شب در آن حجره نشست آن گرمدار
&&
بر امید وعدهٔ آن یار غار
\\
بعد نصف اللیل آمد یار او
&&
صادق الوعدانه آن دلدار او
\\
عاشق خود را فتاده خفته دید
&&
اندکی از آستین او درید
\\
گردگانی چندش اندر جیب کرد
&&
که تو طفلی گیر این می‌باز نرد
\\
چون سحر از خواب عاشق بر جهید
&&
آستین و گردگانها را بدید
\\
گفت شاه ما همه صدق و وفاست
&&
آنچ بر ما می‌رسد آن هم ز ماست
\\
ای دل بی‌خواب ما زین ایمنیم
&&
چون حرس بر بام چوبک می‌زنیم
\\
گردگان ما درین مطحن شکست
&&
هر چه گوییم از غم خود اندکست
\\
عاذلا چند این صلای ماجرا
&&
پند کم ده بعد ازین دیوانه را
\\
من نخواهم عشوهٔ هجران شنود
&&
آزمودم چند خواهم آزمود
\\
هرچه غیر شورش و دیوانگیست
&&
اندرین ره دوری و بیگانگیست
\\
هین بنه بر پایم آن زنجیر را
&&
که دریدم سلسلهٔ تدبیر را
\\
غیر آن جعد نگار مقبلم
&&
گر دو صد زنجیر آری بگسلم
\\
عشق و ناموس ای برادر راست نیست
&&
بر رد ناموس ای عاشق مه‌ایست
\\
وقت آن آمد که من عریان شوم
&&
نقش بگذارم سراسر جان شوم
\\
ای عدو شرم و اندیشه بیا
&&
که دریدم پردهٔ شرم و حیا
\\
ای ببسته خواب جان از جادوی
&&
سخت‌دل یارا که در عالم توی
\\
هین گلوی صبر گیر و می‌فشار
&&
تا خنک گردد دل عشق ای سوار
\\
تا نسوزم کی خنگ گردد دلش
&&
ای دل ما خاندان و منزلش
\\
خانهٔ خود را همی‌سوزی بسوز
&&
کیست آن کس کو بگوید لایجوز
\\
خوش بسوز این خانه را ای شر مست
&&
خانهٔ عاشق چنین اولیترست
\\
بعد ازین این سوز را قبله کنم
&&
زانک شمعم من بسوزش روشنم
\\
خواب را بگذار امشب ای پدر
&&
یک شبی بر کوی بی‌خوابان گذر
\\
بنگر اینها را که مجنون گشته‌اند
&&
هم‌چو پروانه بوصلت کشته‌اند
\\
بنگر این کشتی خلقان غرق عشق
&&
اژدهایی گشت گویی حلق عشق
\\
اژدهایی ناپدید دلربا
&&
عقل هم‌چون کوه را او کهربا
\\
عقل هر عطار کاگه شد ازو
&&
طبله‌ها را ریخت اندر آب جو
\\
رو کزین جو برنیایی تا ابد
&&
لم یکن حقا له کفوا احد
\\
ای مزور چشم بگشای و ببین
&&
چند گویی می‌ندانم آن و این
\\
از وبای زرق و محرومی بر آ
&&
در جهان حی و قیومی در آ
\\
تا نمی‌بینم همی‌بینم شود
&&
وین ندانمهات می‌دانم بود
\\
بگذر از مستی و مستی‌بخش باش
&&
زین تلون نقل کن در استواش
\\
چند نازی تو بدین مستی بس است
&&
بر سر هر کوی چندان مست هست
\\
گر دو عالم پر شود سرمست یار
&&
جمله یک باشند و آن یک نیست خوار
\\
این ز بسیاری نیابد خواریی
&&
خوار کی بود تن‌پرستی ناریی
\\
گر جهان پر شد ز نور آفتاب
&&
کی بود خوار آن تف خوش‌التهاب
\\
لیک با این جمله بالاتر خرام
&&
چونک ارض الله واسع بود و رام
\\
گرچه این مستی چو باز اشهبست
&&
برتر از وی در زمین قدس هست
\\
رو سرافیلی شو اندر امتیاز
&&
در دمندهٔ روح و مست و مست‌ساز
\\
مست را چون دل مزاح اندیشه شد
&&
این ندانم و آن ندانم پیشه شد
\\
این ندانم وان ندانم بهر چیست
&&
تا بگویی آنک می‌دانیم کیست
\\
نفی بهر ثبت باشد در سخن
&&
نفی بگذار و ز ثبت آغاز کن
\\
نیست این و نیست آن هین واگذار
&&
آنک آن هستست آن را پیش آر
\\
نفی بگذار و همان هستی پرست
&&
این در آموز ای پدر زان ترک مست
\\
\end{longtable}
\end{center}
