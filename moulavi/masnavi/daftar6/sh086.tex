\begin{center}
\section*{بخش ۸۶ - تدبیر کردن موش به چغز کی من نمی‌توانم بر تو آمدن به وقت حاجت در آب میان ما وصلتی باید کی چون من بر لب جو آیم ترا توانم خبر کردن و تو چون بر سر سوراخ موش‌خانه آیی مرا توانی خبر کردن الی آخره}
\label{sec:sh086}
\addcontentsline{toc}{section}{\nameref{sec:sh086}}
\begin{longtable}{l p{0.5cm} r}
این سخن پایان ندارد گفت موش
&&
چغز را روزی کای مصباح هوش
\\
وقتها خواهم که گویم با تو راز
&&
تو درون آب داری ترک‌تاز
\\
بر لب جو من ترا نعره‌زنان
&&
نشنوی در آب نالهٔ عاشقان
\\
من بدین وقت معین ای دلیر
&&
می‌نگردم از محاکات تو سیر
\\
پنج وقت آمد نماز و رهنمون
&&
عاشقان را فی صلاة دائمون
\\
نه به پنج آرام گیرد آن خمار
&&
که در آن سرهاست نی پانصد هزار
\\
نیست زر غبا وظیفهٔ عاشقان
&&
سخت مستسقیست جان صادقان
\\
نیست زر غبا وظیفهٔ ماهیان
&&
زانک بی‌دریا ندارند انس جان
\\
آب این دریا که هایل بقعه‌ایست
&&
با خمار ماهیان خود جرعه‌ایست
\\
یک دم هجران بر عاشق چو سال
&&
وصل سالی متصل پیشش خیال
\\
عشق مستسقیست مستسقی‌طلب
&&
در پی هم این و آن چون روز و شب
\\
روز بر شب عاشقست و مضطرست
&&
چون ببینی شب برو عاشق‌ترست
\\
نیستشان از جست‌وجو یک لحظه‌ایست
&&
از پی همشان یکی دم ایست نیست
\\
این گرفته پای آن آن گوش این
&&
این بر آن مدهوش و آن بی‌هوش این
\\
در دل معشوق جمله عاشق است
&&
در دل عذرا همیشه وامق است
\\
در دل عاشق به جز معشوق نیست
&&
در میانشان فارق و فاروق نیست
\\
بر یکی اشتر بود این دو درا
&&
پس چه زر غبا بگنجد این دو را
\\
هیچ کس با خویش زر غبا نمود
&&
هیچ کس با خود به نوبت یار بود
\\
آن یکیی نه که عقلش فهم کرد
&&
فهم این موقوف شد بر مرگ مرد
\\
ور به عقل ادراک این ممکن بدی
&&
قهر نفس از بهر چه واجب شدی
\\
با چنان رحمت که دارد شاه هش
&&
بی‌ضرورت چون بگوید نفس کش
\\
\end{longtable}
\end{center}
