\begin{center}
\section*{بخش ۹۹ - دیدن خوارزمشاه رحمه الله در سیران در موکب خود اسپی بس نادر و تعلق دل شاه به حسن و چستی آن اسپ و سرد کردن  عمادالملک آن اسپ را در دل شاه و گزیدن شاه گفت او را بر دید  خویش چنانک حکیم رحمةالله علیه  در الهی‌نامه فرمود چون زبان حسد شود نخاس  یوسفی یابی از گزی کرباس  از دلالی برادران یوسف حسودانه در دل مشتریان آن چندان حسن  پوشیده شد و زشت نمودن گرفت کی و کانوا فیه من الزاهدین}
\label{sec:sh099}
\addcontentsline{toc}{section}{\nameref{sec:sh099}}
\begin{longtable}{l p{0.5cm} r}
بود امیری را یکی اسپی گزین
&&
در گلهٔ سلطان نبودش یک قرین
\\
او سواره گشت در موکب به گاه
&&
ناگهان دید اسپ را خوارزمشاه
\\
چشم شه را فر و رنگ او ربود
&&
تا به رجعت چشم شه با اسپ بود
\\
بر هر آن عضوش که افکندی نظر
&&
هر یکش خوشتر نمودی زان دگر
\\
غیر چستی و گشی و روحنت
&&
حق برو افکنده بد نادر صفت
\\
پس تجسس کرد عقل پادشاه
&&
کین چه باشد که زند بر عقل راه
\\
چشم من پرست و سیرست و غنی
&&
از دو صد خورشید دارد روشنی
\\
ای رخ شاهان بر من بیذقی
&&
نیم اسپم در رباید بی حقی
\\
جادوی کردست جادو آفرین
&&
جذبه باشد آن نه خاصیات این
\\
فاتحه خواند و بسی لا حول کرد
&&
فاتحه‌ش در سینه می‌افزود درد
\\
زانک او را فاتحه خود می‌کشید
&&
فاتحه در جر و دفع آمد وحید
\\
گر نماید غیر هم تمویه اوست
&&
ور رود غیر از نظر تنبیه اوست
\\
پس یقین گشتش که جذبه زان سریست
&&
کار حق هر لحظه نادر آوریست
\\
اسپ سنگین گاو سنگین ز ابتلا
&&
می‌شود مسجود از مکر خدا
\\
پیش کافر نیست بت را ثانیی
&&
نیست بت را فر و نه روحانیی
\\
چست آن جاذب نهان اندر نهان
&&
در جهان تابیده از دیگر جهان
\\
عقل محجوبست و جان هم زین کمین
&&
من نمی‌بینم تو می‌توانی ببین
\\
چونک خوارمشه ز سیران باز گشت
&&
با خواص ملک خود هم‌راز گشت
\\
پس به سرهنگان بفرمود آن زمان
&&
تا بیارند اسپ را زان خاندان
\\
هم‌چو آتش در رسیدند آن گروه
&&
هم‌چو پشمی گشت امیر هم‌چو کوه
\\
جانش از درد و غبین تا لب رسید
&&
جز عمادالملک زنهاری ندید
\\
که عمادالملک بد پای علم
&&
بهر هر مظلوم و هر مقتول غم
\\
محترم‌تر خود نبد زو سروری
&&
پیش سلطان بود چون پیغامبری
\\
بی‌طمع بود او اصیل و پارسا
&&
رایض و شب‌خیز و حاتم در سخا
\\
بس همایون‌رای و با تدبیر و راد
&&
آزموده رای او در هر مراد
\\
هم به بذل جان سخی و هم به مال
&&
طالب خورشید غیب او چون هلال
\\
در امیری او غریب و محتبس
&&
در صفات فقر وخلت ملتبس
\\
بوده هر محتاج را هم‌چون پدر
&&
پیش سلطان شافع و دفع ضرر
\\
مر بدان را ستر چون حلم خدا
&&
خلق او بر عکس خلقان و جدا
\\
بارها می‌شد به سوی کوه فرد
&&
شاه با صد لابه او را دفع کرد
\\
هر دم ار صد جرم را شافع شدی
&&
چشم سلطان را ازو شرم آمدی
\\
رفت او پیش عماد الملک راد
&&
سر برهنه کرد و بر خاک اوفتاد
\\
که حرم با هر چه دارم گو بگیر
&&
تا بگیرد حاصلم را هر مغیر
\\
این یکی اسپست جانم رهن اوست
&&
گر برد مردم یقین ای خیردوست
\\
گر برد این اسپ را از دست من
&&
من یقین دانم نخواهم زیستن
\\
چون خدا پیوستگیی داده است
&&
بر سرم مال ای مسیحا زود دست
\\
از زن و زر و عقارم صبر هست
&&
این تکلف نیست نی تزویریست
\\
اندرین گر می‌نداری باورم
&&
امتحان کن امتحان گفت و قدم
\\
آن عمادالملک گریان چشم‌مال
&&
پیش سلطان در دوید آشفته‌حال
\\
لب ببست و پیش سلطان ایستاد
&&
راز گویان با خدا رب العباد
\\
ایستاده راز سلطان می‌شنید
&&
واندرون اندیشه‌اش این می‌تنید
\\
کای خداگر آن جوان کژ رفت راه
&&
که نشاید ساختن جز تو پناه
\\
تو از آن خود بکن از وی مگیر
&&
گرچه او خواهد خلاص از هر اسیر
\\
زانک محتاجند این خلقان همه
&&
از گدایی گیر تا سلطان همه
\\
با حضور آفتاب با کمال
&&
رهنمایی جستن از شمع و ذبال
\\
با حضور آفتاب خوش‌مساغ
&&
روشنایی جستن از شمع و چراغ
\\
بی‌گمان ترک ادب باشد ز ما
&&
کفر نعمت باشد و فعل هوا
\\
لیک اغلب هوش‌ها در افتکار
&&
هم‌چو خفاشند ظلمت دوستدار
\\
در شب ار خفاش کرمی می‌خورد
&&
کرم را خورشید جان می‌پرورد
\\
در شب ار خفاش از کرمیست مست
&&
کرم از خورشید جنبنده شدست
\\
آفتابی که ضیا زو می‌زهد
&&
دشمن خود را نواله می‌دهد
\\
لیک شهبازی که او خفاش نیست
&&
چشم بازش راست‌بین و روشنیست
\\
گر به شب جوید چو خفاش او نمو
&&
در ادب خورشید مالد گوش او
\\
گویدش گیرم که آن خفاش لد
&&
علتی دارد ترا باری چه شد
\\
مالشت بدهم به زجر از اکتیاب
&&
تا نتابی سر دگر از آفتاب
\\
\end{longtable}
\end{center}
