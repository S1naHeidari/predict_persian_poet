\begin{center}
\section*{بخش ۱۰۹ - حکایت آن دو برادر یکی کوسه و یکی امرد در عزب خانه‌ای خفتند شبی اتفاقا امرد خشت‌ها بر مقعد خود انبار کرد عاقبت دباب دب آورد و آن خشت‌ها را به حیله و نرمی از پس او برداشت کودک بیدار شد به جنگ کی این خشت‌ها کو کجا بردی و چرا بردی او گفت تو این خشت‌ها را چرا نهادی الی آخره}
\label{sec:sh109}
\addcontentsline{toc}{section}{\nameref{sec:sh109}}
\begin{longtable}{l p{0.5cm} r}
امردی و کوسه‌ای در انجمن
&&
آمدند و مجمعی بد در وطن
\\
مشتغل ماندند قوم منتجب
&&
روز رفت و شد زمانه ثلث شب
\\
زان عزب‌خانه نرفتند آن دو کس
&&
هم بخفتند آن سو از بیم عسس
\\
کوسه را بد بر زنخدان چار مو
&&
لیک هم‌چون ماه بدرش بود رو
\\
کودک امرد به صورت بود زشت
&&
هم نهاد اندر پس کون بیست خشت
\\
لوطیی دب برد شب در انبهی
&&
خشتها را نقل کرد آن مشتهی
\\
دست چون بر وی زد او از جا بجست
&&
گفت هی تو کیستی ای سگ‌پرست
\\
گفت این سی خشت چون انباشتی
&&
گفت تو سی خشت چون بر داشتی
\\
کودک بیمارم و از ضعف خود
&&
کردم اینجا احتیاط و مرتقد
\\
گفت اگر داری ز رنجوری تفی
&&
چون نرفتی جانب دار الشفا
\\
یا به خانهٔ یک طبیبی مشفقی
&&
که گشادی از سقامت مغلقی
\\
گفت آخر من کجا دانم شدن
&&
که بهرجا می‌روم من ممتحن
\\
چون تو زندیقی پلیدی ملحدی
&&
می بر آرد سر به پیشم چون ددی
\\
خانقاهی که بود بهتر مکان
&&
من ندیدم یک دمی در وی امان
\\
رو به من آرند مشتی حمزه‌خوار
&&
چشم‌ها پر نطفه کف خایه‌فشار
\\
وانک ناموسیست خود از زیر زیر
&&
غمزه دزدد می‌دهد مالش به کیر
\\
خانقه چون این بود بازار عام
&&
چون بود خر گله و دیوان خام
\\
خر کجا ناموس و تقوی از کجا
&&
خر چه داند خشیت و خوف و رجا
\\
عقل باشد آمنی و عدل‌جو
&&
بر زن و بر مرد اما عقل کو
\\
ور گریزم من روم سوی زنان
&&
هم‌چو یوسف افتم اندر افتتان
\\
یوسف از زن یافت زندان و فشار
&&
من شوم توزیع بر پنجاه دار
\\
آن زنان از جاهلی بر من تنند
&&
اولیاشان قصد جان من کنند
\\
نه ز مردان چاره دارم نه از زنان
&&
چون کنم که نی ازینم نه از آن
\\
بعد از آن کودک به کوسه بنگریست
&&
گفت او با آن دو مو از غم بریست
\\
فارغست از خشت و از پیکار خشت
&&
وز چو تو مادرفروش کنک زشت
\\
بر زنخ سه چار مو بهر نمون
&&
بهتر از سی خشت گرداگرد کون
\\
ذره‌ای سایهٔ عنایت بهترست
&&
از هزاران کوشش طاعت‌پرست
\\
زانک شیطان خشت طاعت بر کند
&&
گر دو صد خشتست خود را ره کند
\\
خشت اگر پرست بنهادهٔ توست
&&
آن دو سه مو از عطای آن سوست
\\
در حقیقت هر یکی مو زان کهیست
&&
کان امان‌نامهٔ صلهٔ شاهنشهیست
\\
تو اگر صد قفل بنهی بر دری
&&
بر کند آن جمله را خیره‌سری
\\
شحنه‌ای از موم اگر مهری نهد
&&
پهلوانان را از آن دل بشکهد
\\
آن دو سه تار عنایت هم‌چو کوه
&&
سد شد چون فر سیما در وجوه
\\
خشت را مگذار ای نیکوسرشت
&&
لیک هم آمن مخسپ از دیو زشت
\\
رو دو تا مو زان کرم با دست آر
&&
وانگهان آمن بخسپ و غم مدار
\\
نوم عالم از عبادت به بود
&&
آنچنان علمی که مستنبه بود
\\
آن سکون سابح اندر آشنا
&&
به ز جهد اعجمی با دست و پا
\\
اعجمی زد دست و پا و غرق شد
&&
می‌رود سباح ساکن چون عمد
\\
علم دریاییست بی‌حد و کنار
&&
طالب علمست غواص بحار
\\
گر هزاران سال باشد عمر او
&&
او نگردد سیر خود از جست و جو
\\
کان رسول حق بگفت اندر بیان
&&
اینک منهومان هما لا یشبعان
\\
\end{longtable}
\end{center}
