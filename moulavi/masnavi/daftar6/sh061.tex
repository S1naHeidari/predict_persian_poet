\begin{center}
\section*{بخش ۶۱ - جواب دادن قاضی صوفی را}
\label{sec:sh061}
\addcontentsline{toc}{section}{\nameref{sec:sh061}}
\begin{longtable}{l p{0.5cm} r}
گفت قاضی گر نبودی امر مر
&&
ور نبودی خوب و زشت و سنگ و در
\\
ور نبودی نفس و شیطان و هوا
&&
ور نبودی زخم و چالیش و وغا
\\
پس به چه نام و لقب خواندی ملک
&&
بندگان خویش را ای منهتک
\\
چون بگفتی ای صبور و ای حلیم
&&
چون بگفتی ای شجاع و ای حکیم
\\
صابرین و صادقین و منفقین
&&
چون بدی بی ره‌زن و دیو لعین
\\
رستم و حمزه و مخنث یک بدی
&&
علم و حکمت باطل و مندک بدی
\\
علم و حکمت بهر راه و بی‌رهیست
&&
چون همه ره باشد آن حکمت تهیست
\\
بهر این دکان طبع شوره‌آب
&&
هر دو عالم را روا داری خراب
\\
من همی‌دانم که تو پاکی نه خام
&&
وین سؤالت هست از بهر عوام
\\
جور دوران و هر آن رنجی که هست
&&
سهل‌تر از بعد حق و غفلتست
\\
زآنک اینها بگذرند آن نگذرد
&&
دولت آن دارد که جان آگه برد
\\
\end{longtable}
\end{center}
