\begin{center}
\section*{بخش ۶۳ - مثل}
\label{sec:sh063}
\addcontentsline{toc}{section}{\nameref{sec:sh063}}
\begin{longtable}{l p{0.5cm} r}
عارفی پرسید از آن پیر کشیش
&&
که توی خواجه مسن‌تر یا که ریش
\\
گفت نه من پیش ازو زاییده‌ام
&&
بی ز ریشی بس جهان را دیده‌ام
\\
گفت ریشت شد سپید از حال گشت
&&
خوی زشت تو نگردیدست وشت
\\
او پس از تو زاد و از تو بگذرید
&&
تو چنین خشکی ز سودای ثرید
\\
تو بر آن رنگی که اول زاده‌ای
&&
یک قدم زان پیش‌تر ننهاده‌ای
\\
هم‌چنان دوغی ترش در معدنی
&&
خود نگردی زو مخلص روغنی
\\
هم خمیری خمر طینه دری
&&
گرچه عمری در تنور آذری
\\
چون حشیشی پا به گل بر پشته‌ای
&&
گرچه از باد هوس سرگشته‌ای
\\
هم‌چو قوم موسی اندر حر تیه
&&
مانده‌ای بر جای چل سال ای سفیه
\\
می‌روی هر روز تا شب هروله
&&
خویش می‌بینی در اول مرحله
\\
نگذری زین بعد سیصد ساله تو
&&
تا که داری عشق آن گوساله تو
\\
تا خیال عجل از جانشان نرفت
&&
بد بریشان تیه چون گرداب زفت
\\
غیر این عجلی کزو یابیده‌ای
&&
بی‌نهایت لطف و نعمت دیده‌ای
\\
گاو طبعی زان نکوییهای زفت
&&
از دلت در عشق این گوساله رفت
\\
باری اکنون تو ز هر جزوت بپرس
&&
صد زبان دارند این اجزای خرس
\\
ذکر نعمتهای رزاق جهان
&&
که نهان شد آن در اوراق زمان
\\
روز و شب افسانه‌جویانی تو چست
&&
جزو جزو تو فسانه‌گوی تست
\\
جزو جزوت تا برستست از عدم
&&
چند شادی دیده‌اند و چند غم
\\
زانک بی‌لذت نروید هیچ جزو
&&
بلک لاغر گردد از هی پیچ جزو
\\
جزو ماند و آن خوشی از یاد رفت
&&
بل نرفت آن خفیه شد از پنج و هفت
\\
هم‌چو تابستان که از وی پنبه‌زاد
&&
ماند پنبه رفت تابستان ز یاد
\\
یا مثال یخ که زاید از شتا
&&
شد شتا پنهان و آن یخ پیش ما
\\
هست آن یخ زان صعوبت یادگار
&&
یادگار صیف در دی این ثمار
\\
هم‌چنان هر جزو جزوت ای فتی
&&
در تنت افسانه گوی نعمتی
\\
چون زنی که بیست فرزندش بود
&&
هر یکی حاکی حال خوش بود
\\
حمل نبود بی ز مستی و ز لاغ
&&
بی بهاری کی شود زاینده باغ
\\
حاملان و بچگانشان بر کنار
&&
شد دلیل عشق‌بازی با بهار
\\
هر درختی در رضاع کودکان
&&
هم‌چو مریم حامل از شاهی نهان
\\
گرچه صد در آب آتشی پوشیده شد
&&
صد هزاران کف برو جوشیده شد
\\
گرچه آتش سخت پنهان می‌تند
&&
کف بده انگشت اشارت می‌کند
\\
هم‌چنین اجزای مستان وصال
&&
حامل از تمثالهای حال و قال
\\
در جمال حال وا مانده دهان
&&
چشم غایب گشته از نقش جهان
\\
آن موالید از زه این چار نیست
&&
لاجرم منظور این ابصار نیست
\\
آن موالید از تجلی زاده‌اند
&&
لاجرم مستور پردهٔ ساده‌اند
\\
زاده گفتیم و حقیقت زاد نیست
&&
وین عبارت جز پی ارشاد نیست
\\
هین خمش کن تا بگوید شاه قل
&&
بلبلی مفروش با این جنس گل
\\
این گل گویاست پر جوش و خروش
&&
بلبلا ترک زبان کن باش گوش
\\
هر دو گون تمثال پاکیزه‌مثال
&&
شاهد عدل‌اند بر سر وصال
\\
هر دو گون حسن لطیف مرتضی
&&
شاهد احبال و حشر ما مضی
\\
هم‌چو یخ کاندر تموز مستجد
&&
هر دم افسانهٔ زمستان می‌کند
\\
ذکر آن اریاح سرد و زمهریر
&&
اندر آن ازمان و ایام عسیر
\\
هم‌چو آن میوه که در وقت شتا
&&
می‌کند افسانهٔ لطف خدا
\\
قصهٔ دور تبسمهای شمس
&&
وآن عروسان چمن را لمس و طمس
\\
حال رفت و ماند جزوت یادگار
&&
یا ازو واپرس یا خود یاد آر
\\
چون فرو گیرد غمت گر چستیی
&&
زان دم نومید کن وا جستیی
\\
گفتییش ای غصهٔ منکر به حال
&&
راتبهٔ انعامها را زان کمال
\\
گر بهر دم نت بهار و خرمیست
&&
هم‌چو چاش گل تنت انبار چیست
\\
چاش گل تن فکر تو هم‌چون گلاب
&&
منکر گل شد گلاب اینت عجاب
\\
از کپی‌خویان کفران که دریغ
&&
بر نبی‌خویان نثار مهر و میغ
\\
آن لجاج کفر قانون کپیست
&&
وآن سپاس و شکر منهاج نبیست
\\
با کپی‌خویان تهتکها چه کرد
&&
با نبی‌رویان تنسکها چه کرد
\\
در عمارتها سگانند و عقور
&&
در خرابیهاست گنج عز و نور
\\
گر نبودی این بزوغ اندر خسوف
&&
گم نکردی راه چندین فیلسوف
\\
زیرکان و عاقلان از گمرهی
&&
دیده بر خرطوم داغ ابلهی
\\
\end{longtable}
\end{center}
