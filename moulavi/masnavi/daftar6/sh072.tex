\begin{center}
\section*{بخش ۷۲ - جواب گفتن مرید و زجر کردن مرید آن طعانه را از کفر و بیهوده گفتن}
\label{sec:sh072}
\addcontentsline{toc}{section}{\nameref{sec:sh072}}
\begin{longtable}{l p{0.5cm} r}
بانگ زد بر وی جوان و گفت بس
&&
روز روشن از کجا آمد عسس
\\
نور مردان مشرق و مغرب گرفت
&&
اسمانها سجده کردند از شگفت
\\
آفتاب حق بر آمد از حمل
&&
زیر چادر رفت خورشید از خجل
\\
ترهات چون تو ابلیسی مرا
&&
کی بگرداند ز خاک این سرا
\\
من به بادی نامدم هم‌چون سحاب
&&
تا بگردی باز گردم زین جناب
\\
عجل با آن نور شد قبلهٔ کرم
&&
قبله بی آن نور شد کفر و صنم
\\
هست اباحت کز هوای آمد ضلال
&&
هست اباحت کز خدا آمد کمال
\\
کفر ایمان گشت و دیو اسلام یافت
&&
آن طرف کان نور بی‌اندازه تافت
\\
مظهر عزست و محبوب به حق
&&
از همه کروبیان برده سبق
\\
سجده آدم را بیان سبق اوست
&&
سجده آرد مغز را پیوست پوست
\\
شمع حق را پف کنی تو ای عجوز
&&
هم تو سوزی هم سرت ای گنده‌پوز
\\
کی شود دریا ز پوز سگ نجس
&&
کی شود خورشید از پف منطمس
\\
حکم بر ظاهر اگر هم می‌کنی
&&
چیست ظاهرتر بگو زین روشنی
\\
جمله ظاهرها به پیش این ظهور
&&
باشد اندر غایت نقص و قصور
\\
هر که بر شمع خدا آرد پف او
&&
شمع کی میرد بسوزد پوز او
\\
چون تو خفاشان بسی بینند خواب
&&
کین جهان ماند یتیم از آفتاب
\\
موجهای تیز دریاهای روح
&&
هست صد چندان که بد طوفان نوح
\\
لیک اندر چشم کنعان موی رست
&&
نوح و کشتی را بهشت و کوه جست
\\
کوه و کنعان را فرو برد آن زمان
&&
نیم موجی تا به قعر امتهان
\\
مه فشاند نور و سگ وع وع کند
&&
سگ ز نور ماه کی مرتع کند
\\
شب روان و همرهان مه بتگ
&&
ترک رفتن کی کنند از بانگ سگ
\\
جزو سوی کل دوان مانند تیر
&&
کی کند وقف از پی هر گنده‌پیر
\\
جان شرع و جان تقوی عارفست
&&
معرفت محصول زهد سالفست
\\
زهد اندر کاشتن کوشیدنست
&&
معرفت آن کشت را روییدنست
\\
پس چو تن باشد جهاد و اعتقاد
&&
جان این کشتن نباتست و حصاد
\\
امر معروف او و هم معروف اوست
&&
کاشف اسرار و هم مکشوف اوست
\\
شاه امروزینه و فردای ماست
&&
پوست بندهٔ مغز نغزش دایماست
\\
چون انا الحق گفت شیخ و پیش برد
&&
پس گلوی جمله کوران را فشرد
\\
چون انای بنده لا شد از وجود
&&
پس چه ماند تو بیندیش ای جحود
\\
گر ترا چشمیست بگشا در نگر
&&
بعد لا آخر چه می‌ماند دگر
\\
ای بریده آن لب و حلق و دهان
&&
که کند تف سوی مه یا آسمان
\\
تف برویش باز گردد بی شکی
&&
تف سوی گردون نیابد مسلکی
\\
تا قیامت تف برو بارد ز رب
&&
هم‌چو تبت بر روان بولهب
\\
طبل و رایت هست ملک شهریار
&&
سگ کسی که خواند او را طبل‌خوار
\\
آسمانها بندهٔ ماه وی‌اند
&&
شرق و مغرب جمله نانخواه وی‌اند
\\
زانک لولاکست بر توقیع او
&&
جمله در انعام و در توزیع او
\\
گر نبودی او نیابیدی فلک
&&
گردش و نور و مکانی ملک
\\
گر نبودی او نیابیدی به حار
&&
هیبت و ماهی و در شاهوار
\\
گر نبودی او نیابیدی زمین
&&
در درونه گنج و بیرون یاسمین
\\
رزقها هم رزق‌خواران وی‌اند
&&
میوه‌ها لب‌خشک باران وی‌اند
\\
هین که معکوس است در امر این گره
&&
صدقه‌بخش خویش را صدقه بده
\\
از فقیرستت همه زر و حریر
&&
هین غنی راده زکاتی ای فقیر
\\
چون تو ننگی جفت آن مقبول‌روح
&&
چون عیال کافر اندر عقد نوح
\\
گر نبودی نسبت تو زین سرا
&&
پاره‌پاره کردمی این دم ترا
\\
دادمی آن نوح را از تو خلاص
&&
تا مشرف گشتمی من در قصاص
\\
لیک با خانهٔ شهنشاه زمن
&&
این چنین گستاخیی ناید ز من
\\
رو دعا کن که سگ این موطنی
&&
ورنه اکنون کردمی من کردنی
\\
\end{longtable}
\end{center}
