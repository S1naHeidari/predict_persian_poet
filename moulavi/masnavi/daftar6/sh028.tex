\begin{center}
\section*{بخش ۲۸ - باز گردانیدن صدیق رضی الله عنه واقعهٔ بلال را رضی الله عنه و ظلم جهودان را بر وی و احد احد گفتن او و افزون شدن کینهٔ جهودان و قصه کردن آن قضیه پیش مصطفی علیه‌السلام و مشورت در خریدن او}
\label{sec:sh028}
\addcontentsline{toc}{section}{\nameref{sec:sh028}}
\begin{longtable}{l p{0.5cm} r}
بعد از آن صدیق پیش مصطفی
&&
گفت حال آن بلال با وفا
\\
کان فلک‌پیمای میمون‌بال چست
&&
این زمان در عشق و اندر دام تست
\\
باز سلطانست زان جغدان برنج
&&
در حدث مدفون شدست آن زفت‌گنج
\\
جغدها بر باز استم می‌کنند
&&
پر و بالش بی‌گناهی می‌کنند
\\
جرم او اینست کو بازست و بس
&&
غیر خوبی جرم یوسف چیست پس
\\
جغد را ویرانه باشد زاد و بود
&&
هستشان بر باز زان زخم جهود
\\
که چرا می یاد آری زان دیار
&&
یا ز قصر و ساعد آن شهریار
\\
در ده جغدان فضولی می‌کنی
&&
فتنه و تشویش در می‌افکنی
\\
مسکن ما را که شد رشک اثیر
&&
تو خرابه خوانی و نام حقیر
\\
شید آوردی که تا جغدان ما
&&
مر ترا سازند شاه و پیشوا
\\
وهم و سودایی دریشان می‌تنی
&&
نام این فردوس ویران می‌کنی
\\
بر سرت چندان زنیم ای بد صفات
&&
که بگویی ترک شید و ترهات
\\
پیش مشرق چارمیخش می‌کنند
&&
تن برهنه شاخ خارش می‌زنند
\\
از تنش صد جای خون بر می‌جهد
&&
او احد می‌گوید و سر می‌نهد
\\
پندها دادم که پنهان دار دین
&&
سر بپوشان از جهودان لعین
\\
عاشق است او را قیامت آمدست
&&
تا در توبه برو بسته شدست
\\
عاشقی و توبه یا امکان صبر
&&
این محالی باشد ای جان بس سطبر
\\
توبه کردم و عشق هم‌چون اژدها
&&
توبه وصف خلق و آن وصف خدا
\\
عشق ز اوصاف خدای بی‌نیاز
&&
عاشقی بر غیر او باشد مجاز
\\
زانک آن حسن زراندود آمدست
&&
ظاهرش نور اندرون دود آمدست
\\
چون رود نور و شود پیدا دخان
&&
بفسرد عشق مجازی آن زمان
\\
وا رود آن حسن سوی اصل خود
&&
جسم ماند گنده و رسوا و بد
\\
نور مه راجع شود هم سوی ماه
&&
وا رود عکسش ز دیوار سیاه
\\
پس بماند آب و گل بی آن نگار
&&
گردد آن دیوار بی مه دیووار
\\
قلب را که زر ز روی او بجست
&&
بازگشت آن زر بکان خود نشست
\\
پس مس رسوا بماند دود وش
&&
زو سیه‌روتر بماند عاشقش
\\
عشق بینایان بود بر کان زر
&&
لاجرم هر روز باشد بیشتر
\\
زانک کان را در زری نبود شریک
&&
مرحبا ای کان زر لاشک فیک
\\
هر که قلبی را کند انباز کان
&&
وا رود زر تا بکان لامکان
\\
عاشق و معشوق مرده ز اضطراب
&&
مانده ماهی رفته زان گرداب آب
\\
عشق ربانیست خورشید کمال
&&
امر نور اوست خلقان چون ظلال
\\
مصطفی زین قصه چون خوش برشکفت
&&
رغبت افزون گشت او را هم بگفت
\\
مستمع چون یافت هم‌چون مصطفی
&&
هر سر مویش زبانی شد جدا
\\
مصطفی گفتش که اکنون چاره چیست
&&
گفت این بنده مر او را مشتریست
\\
هر بها که گوید او را می‌خرم
&&
در زیان و حیف ظاهر ننگرم
\\
کو اسیر الله فی الارض آمدست
&&
سخرهٔ خشم عدو الله شدست
\\
\end{longtable}
\end{center}
