\begin{center}
\section*{بخش ۳۰ - خندیدن جهود و پنداشتن کی صدیق مغبونست درین عقد}
\label{sec:sh030}
\addcontentsline{toc}{section}{\nameref{sec:sh030}}
\begin{longtable}{l p{0.5cm} r}
قهقهه زد آن جهود سنگ‌دل
&&
از سر افسوس و طنز و غش و غل
\\
گفت صدیقش که این خنده چه بود
&&
در جواب پرسش او خنده فزود
\\
گفت اگر جدت نبودی و غرام
&&
در خریداری این اسود غلام
\\
من ز استیزه نمی‌جوشیدمی
&&
خود به عشر اینش بفروشیدمی
\\
کو به نزد من نیرزد نیم دانگ
&&
تو گران کردی بهایش را به بانگ
\\
پس جوابش داد صدیق ای غبی
&&
گوهری دادی به جوزی چون صبی
\\
کو به نزد من همی‌ارزد دو کون
&&
من به جانش ناظرستم تو بلون
\\
زر سرخست او سیه‌تاب آمده
&&
از برای رشک این احمق‌کده
\\
دیدهٔ این هفت رنگ جسمها
&&
در نیابد زین نقاب آن روح را
\\
گر مکیسی کردیی در بیع بیش
&&
دادمی من جمله ملک و مال خویش
\\
ور مکاس افزودیی من ز اهتمام
&&
دامنی زر کردمی از غیر وام
\\
سهل دادی زانک ارزان یافتی
&&
در ندیدی حقه را نشکافتی
\\
حقه سربسته جهل تو بداد
&&
زود بینی که چه غبنت اوفتاد
\\
حقهٔ پر لعل را دادی به باد
&&
هم‌چو زنگی در سیه‌رویی تو شاد
\\
عاقبت وا حسرتا گویی بسی
&&
بخت ودولت را فروشد خود کسی
\\
بخت با جامهٔ غلامانه رسید
&&
چشم بدبختت به جز ظاهر ندید
\\
او نمودت بندگی خویشتن
&&
خوی زشتت کرد با او مکر و فن
\\
این سیه‌اسرار تن‌اسپید را
&&
بت‌پرستانه بگیر ای ژاژخا
\\
این ترا و آن مرا بردیم سود
&&
هین لکم دین ولی دین ای جهود
\\
خود سزای بت‌پرستان این بود
&&
جلش اطلس اسپ او چوبین بود
\\
هم‌چو گور کافران پر دود و نار
&&
وز برون بر بسته صد نقش و نگار
\\
هم‌چو مال ظالمان بیرون جمال
&&
وز درونش خون مظلوم و وبال
\\
چون منافق از برون صوم و صلات
&&
وز درون خاک سیاه بی‌نبات
\\
هم‌چو ابری خالیی پر قر و قر
&&
نه درو نفع زمین نه قوت بر
\\
هم‌چو وعدهٔ مکر و گفتار دروغ
&&
آخرش رسوا و اول با فروغ
\\
بعد از آن بگرفت او دست بلال
&&
آن ز زخم ضرس محنت چون خلال
\\
شد خلالی در دهانی راه یافت
&&
جانب شیرین‌زبانی می‌شتافت
\\
چون بدید آن خسته روی مصطفی
&&
خر مغشیا فتاد او بر قفا
\\
تا بدیری بی‌خود و بی‌خویش ماند
&&
چون به خویش آمد ز شادی اشک راند
\\
مصطفی‌اش در کنار خود کشید
&&
کس چه داند بخششی کو را رسید
\\
چون بود مسی که بر اکسیر زد
&&
مفلسی بر گنج پر توفیر زد
\\
ماهی پژمرده در بحر اوفتاد
&&
کاروان گم شده زد بر رشاد
\\
آن خطاباتی که گفت آن دم نبی
&&
گر زند بر شب بر آید از شبی
\\
روز روشن گردد آن شب چون صباح
&&
من نتوانم باز گفت آن اصطلاح
\\
خود تو دانی که آفتابی در حمل
&&
تا چه گوید با نبات و با دقل
\\
خود تو دانی هم که آن آب زلال
&&
می چه گوید با ریاحین و نهال
\\
صنع حق با جمله اجزای جهان
&&
چون دم و حرفست از افسون‌گران
\\
جذب یزدان با اثرها و سبب
&&
صد سخن گوید نهان بی‌حرف و لب
\\
نه که تاثیر از قدر معمول نیست
&&
لیک تاثیرش ازو معقول نیست
\\
چون مقلد بود عقل اندر اصول
&&
دان مقلد در فروعش ای فضول
\\
گر بپرسد عقل چون باشد مرام
&&
گو چنانک تو ندانی والسلام
\\
\end{longtable}
\end{center}
