\begin{center}
\section*{بخش ۲۵ - تمثیل مرد حریص نابیننده رزاقی حق را و خزاین و رحمت او را به موری کی در خرمنگاه بزرگ با دانهٔ گندم می‌کوشد و می‌جوشد و می‌لرزد و به تعجیل می‌کشد و سعت آن خرمن را نمی‌بیند}
\label{sec:sh025}
\addcontentsline{toc}{section}{\nameref{sec:sh025}}
\begin{longtable}{l p{0.5cm} r}
مور بر دانه بدان لرزان شود
&&
که ز خرمنهای خوش اعمی بود
\\
می‌کشد آن دانه را با حرص و بیم
&&
که نمی‌بیند چنان چاش کریم
\\
صاحب خرمن همی‌گوید که هی
&&
ای ز کوری پیش تو معدوم شی
\\
تو ز خرمنهای ما آن دیده‌ای
&&
که در آن دانه به جان پیچیده‌ای
\\
ای به صورت ذره کیوان را ببین
&&
مور لنگی رو سلیمان را ببین
\\
تو نه‌ای این جسم تو آن دیده‌ای
&&
وا رهی از جسم گر جان دیده‌ای
\\
آدمی دیده‌ست باقی گوشت و پوست
&&
هرچه چشمش دیده است آن چیز اوست
\\
کوه را غرقه کند یک خم ز نم
&&
منفذش چون باز باشد سوی یم
\\
چون به دریا راه شد از جان خم
&&
خم با جیحون برآرد اشتلم
\\
زان سبب قل گفتهٔ دریا بود
&&
هرچه نطق احمدی گویا بود
\\
گفتهٔ او جمله در بحر بود
&&
که دلش را بود در دریا نفوذ
\\
داد دریا چون ز خم ما بود
&&
چه عجب در ماهیی دریا بود
\\
چشم حس افسرد بر نقش ممر
&&
تش ممر می‌بینی و او مستقر
\\
این دوی اوصاف دید احولست
&&
ورنه اول آخر آخر اولست
\\
هی ز چه معلوم گردد این ز بعث
&&
بعث را جو کم کن اندر بعث بحث
\\
شرط روز بعث اول مردنست
&&
زانک بعث از مرده زنده کردنست
\\
جمله عالم زین غلط کردند راه
&&
کز عدم ترسند و آن آمد پناه
\\
از کجا جوییم علم از ترک علم
&&
از کجا جوییم سلم از ترک سلم
\\
از کجا جوییم هست از ترک هست
&&
از کجا جوییم سیب از ترک دست
\\
هم تو تانی کرد یا نعم المعین
&&
دیدهٔ معدوم‌بین را هست بین
\\
دیده‌ای کو از عدم آمد پدید
&&
ذات هستی را همه معدوم دید
\\
این جهان منتظم محشر شود
&&
گر دو دیده مبدل و انور شود
\\
زان نماید این حقایق ناتمام
&&
که برین خامان بود فهمش حرام
\\
نعمت جنات خوش بر دوزخی
&&
شد محرم گرچه حق آمد سخی
\\
در دهانش تلخ آید شهد خلد
&&
چون نبود از وافیان در عهد خلد
\\
مر شما را نیز در سوداگری
&&
دست کی جنبد چو نبود مشتری
\\
کی نظاره اهل بخریدن بود
&&
آن نظاره گول گردیدن بود
\\
پرس پرسان کین به چند و آن به چند
&&
از پی تعبیر وقت و ریش‌خند
\\
از ملولی کاله می‌خواهد ز تو
&&
نیست آن کس مشتری و کاله‌جو
\\
کاله را صد بار دید و باز داد
&&
جامه کی پیمود او پیمود باد
\\
کو قدوم و کر و فر مشتری
&&
کو مزاح گنگلی سرسری
\\
چونک در ملکش نباشد حبه‌ای
&&
جز پی گنگل چه جوید جبه‌ای
\\
در تجارت نیستش سرمایه‌ای
&&
پس چه شخص زشت او چه سایه‌ای
\\
مایه در بازار این دنیا زرست
&&
مایه آنجا عشق و دو چشم ترست
\\
هر که او بی‌مایهٔ بازار رفت
&&
عمر رفت و بازگشت او خام تفت
\\
هی کجا بودی برادر هیچ جا
&&
هی چه پختی بهر خوردن هیچ با
\\
مشتری شو تا بجنبد دست من
&&
لعل زاید معدن آبست من
\\
مشتری گرچه که سست و باردست
&&
دعوت دین کن که دعوت واردست
\\
باز پران کن حمام روح گیر
&&
در ره دعوت طریق نوح گیر
\\
خدمتی می‌کن برای کردگار
&&
با قبول و رد خلقانت چه کار
\\
\end{longtable}
\end{center}
