\begin{center}
\section*{بخش ۱۰۷ - دیدن ایشان در قصر این قلعهٔ ذات الصور نقش روی دختر شاه چین را و بیهوش شدن هر سه و در فتنه افتادن و تفحص کردن کی این صورت کیست}
\label{sec:sh107}
\addcontentsline{toc}{section}{\nameref{sec:sh107}}
\begin{longtable}{l p{0.5cm} r}
این سخن پایان ندارد آن گروه
&&
صورتی دیدند با حسن و شکوه
\\
خوب‌تر زان دیده بودند آن فریق
&&
لیک زین رفتند در بحر عمیق
\\
زانک افیونشان درین کاسه رسید
&&
کاسه‌ها محسوس و افیون ناپدید
\\
کرد فعل خویش قلعهٔ هش‌ربا
&&
هر سه را انداخت در چاه بلا
\\
تیر غمزه دوخت دل را بی‌کمان
&&
الامان و الامان ای بی‌امان
\\
قرنها را صورت سنگین بسوخت
&&
آتشی در دین و دلشان بر فروخت
\\
چونک روحانی بود خود چون بود
&&
فتنه‌اش هر لحظه دیگرگون بود
\\
عشق صورت در دل شه‌زادگان
&&
چون خلش می‌کرد مانند سنان
\\
اشک می‌بارید هر یک هم‌چو میغ
&&
دست می‌خایید و می‌گفت ای دریغ
\\
ما کنون دیدیم شه ز آغاز دید
&&
چندمان سوگند داد آن بی‌ندید
\\
انبیا را حق بسیارست از آن
&&
که خبر کردند از پایانمان
\\
کاینچ می‌کاری نروید جز که خار
&&
وین طرف پری نیابی زو مطار
\\
تخم از من بر که تا ریعی دهد
&&
با پر من پر که تیر آن سو جهد
\\
تو ندانی واجبی آن و هست
&&
هم تو گویی آخر آن واجب بدست
\\
او توست اما نه این تو آن توست
&&
که در آخر واقف بیرون‌شوست
\\
توی آخر سوی توی اولت
&&
آمدست از بهر تنبیه و صلت
\\
توی تو در دیگری آمد دفین
&&
من غلام مرد خودبینی چنین
\\
آنچ در آیینه می‌بیند جوان
&&
پیر اندر خشت بیند بیش از آن
\\
ز امر شاه خویش بیرون آمدیم
&&
با عنایات پدر یاغی شدیم
\\
سهل دانستیم قول شاه را
&&
وان عنایت‌های بی اشباه را
\\
نک در افتادیم در خندق همه
&&
کشته و خستهٔ بلا بی ملحمه
\\
تکیه بر عقل خود و فرهنگ خویش
&&
بودمان تا این بلا آمد به پیش
\\
بی‌مرض دیدیم خویش و بی ز رق
&&
آنچنان که خویش را بیمار دق
\\
علت پنهان کنون شد آشکار
&&
بعد از آنک بند گشتیم و شکار
\\
سایهٔ رهبر بهست از ذکر حق
&&
یک قناعت به که صد لوت و طبق
\\
چشم بینا بهتر از سیصد عصا
&&
چشم بشناسد گهر را از حصا
\\
در تفحص آمدند از اندهان
&&
صورت کی بود عجب این در جهان
\\
بعد بسیاری تفحص در مسیر
&&
کشف کرد آن راز را شیخی بصیر
\\
نه از طریق گوش بل از وحی هوش
&&
رازها بد پیش او بی روی‌پوش
\\
گفت نقش رشک پروینست این
&&
صورت شه‌زادهٔ چینست این
\\
هم‌چو جان و چون جنین پنهانست او
&&
در مکتم پرده و ایوانست او
\\
سوی او نه مرد ره دارد نه زن
&&
شاه پنهان کرد او را از فتن
\\
غیرتی دارد ملک بر نام او
&&
که نپرد مرغ هم بر بام او
\\
وای آن دل کش چنین سودا فتاد
&&
هیچ کس را این چنین سودا مباد
\\
این سزای آنک تخم جهل کاشت
&&
وآن نصیحت را کساد و سهل داشت
\\
اعتمادی کرد بر تدبیر خویش
&&
که برم من کار خود با عقل پیش
\\
نیم ذره زان عنایت به بود
&&
که ز تدبیر خرد سیصد رصد
\\
ترک مکر خویشتن گیر ای امیر
&&
پا بکش پیش عنایت خوش بمیر
\\
این به قدر حیلهٔ معدود نیست
&&
زین حیل تا تو نمیری سود نیست
\\
\end{longtable}
\end{center}
