\begin{center}
\section*{بخش ۱ - سر آغاز}
\label{sec:sh001}
\addcontentsline{toc}{section}{\nameref{sec:sh001}}
\begin{longtable}{l p{0.5cm} r}
مدتی این مثنوی تاخیر شد
&&
مهلتی بایست تا خون شیر شد
\\
تا نزاید بخت تو فرزند نو
&&
خون نگردد شیر شیرین خوش شنو
\\
چون ضیاء الحق حسام الدین عنان
&&
باز گردانید ز اوج آسمان
\\
چون به معراج حقایق رفته بود
&&
بی‌بهارش غنچه‌ها ناکفته بود
\\
چون ز دریا سوی ساحل بازگشت
&&
چنگ شعر مثنوی با ساز گشت
\\
مثنوی که صیقل ارواح بود
&&
باز گشتش روز استفتاح بود
\\
مطلع تاریخ این سودا و سود
&&
سال اندر ششصد و شصت و دو بود
\\
بلبلی زینجا برفت و بازگشت
&&
بهر صید این معانی بازگشت
\\
ساعد شه مسکن این باز باد
&&
تا ابد بر خلق این در باز باد
\\
آفت این در هوا و شهوتست
&&
ورنه اینجا شربت اندر شربتست
\\
این دهان بر بند تا بینی عیان
&&
چشم‌بند آن جهان حلق و دهان
\\
ای دهان تو خود دهانهٔ دوزخی
&&
وی جهان تو بر مثال برزخی
\\
نور باقی پهلوی دنیای دون
&&
شیر صافی پهلوی جوهای خون
\\
چون درو گامی زنی بی احتیاط
&&
شیر تو خون می‌شودر از اختلاط
\\
یک قدم زد آدم اندر ذوق نفس
&&
شد فراق صدر جنت طوق نفس
\\
همچو دیو از وی فرشته می‌گریخت
&&
بهر نانی چند آب چشم ریخت
\\
گرچه یک مو بد گنه کو جسته بود
&&
لیک آن مو در دو دیده رسته بود
\\
بود آدم دیدهٔ نور قدیم
&&
موی در دیده بود کوه عظیم
\\
گر در آن آدم بکردی مشورت
&&
در پشیمانی نگفتی معذرت
\\
زانک با عقلی چو عقلی جفت شد
&&
مانع بد فعلی و بد گفت شد
\\
نفس با نفس دگر چون یار شد
&&
عقل جزوی عاطل و بی‌کار شد
\\
چون ز تنهایی تو نومیدی شوی
&&
زیر سایهٔ یار خورشیدی شوی
\\
رو بجو یار خدایی را تو زود
&&
چون چنان کردی خدا یار تو بود
\\
آنک در خلوت نظر بر دوختست
&&
آخر آن را هم ز یار آموختست
\\
خلوت از اغیار باید نه ز یار
&&
پوستین بهر دی آمد نه بهار
\\
عقل با عقل دگر دوتا شود
&&
نور افزون گشت و ره پیدا شود
\\
نفس با نفس دگر خندان شود
&&
ظلمت افزون گشت و ره پنهان شود
\\
یار چشم تست ای مرد شکار
&&
از خس و خاشاک او را پاک دار
\\
هین بجاروب زبان گردی مکن
&&
چشم را از خس ره‌آوردی مکن
\\
چون که مؤمن آینهٔ مؤمن بود
&&
روی او ز آلودگی ایمن بود
\\
یار آیینست جان را در حزن
&&
در رخ آیینه ای جان دم مزن
\\
تا نپوشد روی خود را در دمت
&&
دم فرو خوردن بباید هر دمت
\\
کم ز خاکی چونک خاکی یار یافت
&&
از بهاری صد هزار انوار یافت
\\
آن درختی کو شود با یار جفت
&&
از هوای خوش ز سر تا پا شکفت
\\
در خزان چون دید او یار خلاف
&&
در کشید او رو و سر زیر لحاف
\\
گفت یار بد بلا آشفتنست
&&
چونک او آمد طریقم خفتنست
\\
پس بخسپم باشم از اصحاب کهف
&&
به ز دقیانوس آن محبوس لهف
\\
یقظه‌شان مصروف دقیانوس بود
&&
خوابشان سرمایهٔ ناموس بود
\\
خواب بیداریست چون با دانشست
&&
وای بیداری که با نادان نشست
\\
چونک زاغان خیمه بر بهمن زدند
&&
بلبلان پنهان شدند و تن زدند
\\
زانک بی گلزار بلبل خامشست
&&
غیبت خورشید بیداری‌کشست
\\
آفتابا ترک این گلشن کنی
&&
تا که تحت الارض را روشن کنی
\\
آفتاب معرفت را نقل نیست
&&
مشرق او غیر جان و عقل نیست
\\
خاصه خورشید کمالی کان سریست
&&
روز و شب کردار او روشن‌گریست
\\
مطلع شمس آی گر اسکندری
&&
بعد از آن هرجا روی نیکو فری
\\
بعد از آن هر جا روی مشرق شود
&&
شرقها بر مغربت عاشق شود
\\
حس خفاشت سوی مغرب دوان
&&
حس درپاشت سوی مشرق روان
\\
راه حس راه خرانست ای سوار
&&
ای خران را تو مزاحم شرم دار
\\
پنج حسی هست جز این پنج حس
&&
آن چو زر سرخ و این حسها چو مس
\\
اندر آن بازار کاهل محشرند
&&
حس مس را چون حس زر کی خرند
\\
حس ابدان قوت ظلمت می‌خورد
&&
حس جان از آفتابی می‌چرد
\\
ای ببرده رخت حسها سوی غیب
&&
دست چون موسی برون آور ز جیب
\\
ای صفاتت آفتاب معرفت
&&
و آفتاب چرخ بند یک صفت
\\
گاه خورشیدی و گه دریا شوی
&&
گاه کوه قاف و گه عنقا شوی
\\
تو نه این باشی نه آن در ذات خویش
&&
ای فزون از وهمها وز بیش بیش
\\
روح با علمست و با عقلست یار
&&
روح را با تازی و ترکی چه کار
\\
از تو ای بی نقش با چندین صور
&&
هم مشبه هم موحد خیره‌سر
\\
گه مشبه را موحد می‌کند
&&
گه موحد را صور ره می‌زند
\\
گه ترا گوید ز مستی بوالحسن
&&
یا صغیر السن یا رطب البدن
\\
گاه نقش خویش ویران می‌کند
&&
آن پی تنزیه جانان می‌کند
\\
چشم حس را هست مذهب اعتزال
&&
دیدهٔ عقلست سنی در وصال
\\
سخرهٔ حس‌اند اهل اعتزال
&&
خویش را سنی نمایند از ضلال
\\
هر که بیرون شد ز حس سنی ویست
&&
اهل بینش چشم عقل خوش‌پیست
\\
گر بدیدی حس حیوان شاه را
&&
پس بدیدی گاو و خر الله را
\\
گر نبودی حس دیگر مر ترا
&&
جز حس حیوان ز بیرون هوا
\\
پس بنی‌آدم مکرم کی بدی
&&
کی به حس مشترک محرم شدی
\\
نامصور یا مصور گفتنت
&&
باطل آمد بی ز صورت رفتنت
\\
نامصور یا مصور پیش اوست
&&
کو همه مغزست و بیرون شد ز پوست
\\
گر تو کوری نیست بر اعمی حرج
&&
ورنه رو کالصبر مفتاح الفرج
\\
پرده‌های دیده را داروی صبر
&&
هم بسوزد هم بسازد شرح صدر
\\
آینهٔ دل چون شود صافی و پاک
&&
نقشها بینی برون از آب و خاک
\\
هم ببینی نقش و هم نقاش را
&&
فرش دولت را و هم فراش را
\\
چون خلیل آمد خیال یار من
&&
صورتش بت معنی او بت‌شکن
\\
شکر یزدان را که چون او شد پدید
&&
در خیالش جان خیال خود بدید
\\
خاک درگاهت دلم را می‌فریفت
&&
خاک بر وی کو ز خاکت می‌شکیفت
\\
گفتم ار خوبم پذیرم این ازو
&&
ورنه خود خندید بر من زشت‌رو
\\
چاره آن باشد که خود را بنگرم
&&
ورنه او خندد مرا من کی خرم
\\
او جمیلست و محب للجمال
&&
کی جوان نو گزیند پیر زال
\\
خوب خوبی را کند جذب این بدان
&&
طیبات و طیبین بر وی بخوان
\\
در جهان هر چیز چیزی جذب کرد
&&
گرم گرمی را کشید و سرد سرد
\\
قسم باطل باطلان را می‌کشند
&&
باقیان از باقیان هم سرخوشند
\\
ناریان مر ناریان را جاذب‌اند
&&
نوریان مر نوریان را طالب‌اند
\\
چشم چون بستی ترا جان کند نیست
&&
چشم را از نور روزن صبر نیست
\\
چشم چون بستی ترا تاسه گرفت
&&
نور چشم از نور روزن کی شکفت
\\
تاسهٔ تو جذب نور چشم بود
&&
تا بپیوندد به نور روز زود
\\
چشم باز ار تاسه گیرد مر ترا
&&
دانک چشم دل ببستی بر گشا
\\
آن تقاضای دو چشم دل شناس
&&
کو همی‌جوید ضیای بی‌قیاس
\\
چون فراق آن دو نور بی‌ثبات
&&
تاسه آوردت گشادی چشمهات
\\
پس فراق آن دو نور پایدار
&&
تا سه می‌آرد مر آن را پاس دار
\\
او چو می‌خواند مرا من بنگرم
&&
لایق جذبم و یا بد پیکرم
\\
گر لطیفی زشت را در پی کند
&&
تسخری باشد که او بر وی کند
\\
کی ببینم روی خود را ای عجب
&&
تا چه رنگم همچو روزم یا چو شب
\\
نقش جان خویش من جستم بسی
&&
هیچ می‌ننمود نقشم از کسی
\\
گفتم آخر آینه از بهر چیست
&&
تا بداند هر کسی کو چیست و کیست
\\
آینهٔ آهن برای پوستهاست
&&
آینهٔ سیمای جان سنگی‌بهاست
\\
آینهٔ جان نیست الا روی یار
&&
روی آن یاری که باشد زان دیار
\\
گفتم ای دل آینهٔ کلی بجو
&&
رو به دریا کار بر ناید بجو
\\
زین طلب بنده به کوی تو رسید
&&
درد مریم را به خرمابن کشید
\\
دیدهٔ تو چون دلم را دیده شد
&&
شد دل نادیده غرق دیده شد
\\
آینهٔ کلی ترا دیدم ابد
&&
دیدم اندر چشم تو من نقش خود
\\
گفتم آخر خویش را من یافتم
&&
در دو چشمش راه روشن یافتم
\\
گفت وهمم کان خیال تست هان
&&
ذات خود را از خیال خود بدان
\\
نقش من از چشم تو آواز داد
&&
که منم تو تو منی در اتحاد
\\
کاندرین چشم منیر بی زوال
&&
از حقایق راه کی یابد خیال
\\
در دو چشم غیر من تو نقش خود
&&
گر ببینی آن خیالی دان و رد
\\
زانک سرمهٔ نیستی در می‌کشد
&&
باده از تصویر شیطان می‌چشد
\\
چشمشان خانهٔ خیالست و عدم
&&
نیستها را هست بیند لاجرم
\\
چشم من چون سرمه دید از ذوالجلال
&&
خانهٔ هستیست نه خانهٔ خیال
\\
تا یکی مو باشد از تو پیش چشم
&&
در خیالت گوهری باشد چو یشم
\\
یشم را آنگه شناسی از گهر
&&
کز خیال خود کنی کلی عبر
\\
یک حکایت بشنو ای گوهر شناس
&&
تا بدانی تو عیان را از قیاس
\\
\end{longtable}
\end{center}
