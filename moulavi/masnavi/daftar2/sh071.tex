\begin{center}
\section*{بخش ۷۱ - شکایت قاضی از آفت قضا و جواب گفتن نایب او را}
\label{sec:sh071}
\addcontentsline{toc}{section}{\nameref{sec:sh071}}
\begin{longtable}{l p{0.5cm} r}
قاضیی بنشاندند و می‌گریست
&&
گفت نایب قاضیا گریه ز چیست
\\
این نه وقت گریه و فریاد تست
&&
وقت شادی و مبارک‌باد تست
\\
گفت اه چون حکم راند بی‌دلی
&&
در میان آن دو عالم جاهلی
\\
آن دو خصم از واقعهٔ خود واقفند
&&
قاضی مسکین چه داند زان دو بند
\\
جاهلست و غافلست از حالشان
&&
چون رود در خونشان و مالشان
\\
گفت خصمان عالم‌اند و علتی
&&
جاهلی تو لیک شمع ملتی
\\
زانک تو علت نداری در میان
&&
آن فراغت هست نور دیدگان
\\
وان دو عالم را غرضشان کور کرد
&&
علمشان را علت اندر گور کرد
\\
جهل را بی‌علتی عالم کند
&&
علم را علت کژ و ظالم کند
\\
تا تو رشوت نستدی بیننده‌ای
&&
چون طمع کردی ضریر و بنده‌ای
\\
از هوا من خوی را وا کرده‌ام
&&
لقمه‌های شهوتی کم خورده‌ام
\\
چاشنی‌گیر دلم شد با فروغ
&&
راست را داند حقیقت از دروغ
\\
\end{longtable}
\end{center}
