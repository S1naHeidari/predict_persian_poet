\begin{center}
\section*{بخش ۱۱۴ - قصهٔ بط بچگان کی مرغ خانگی پروردشان}
\label{sec:sh114}
\addcontentsline{toc}{section}{\nameref{sec:sh114}}
\begin{longtable}{l p{0.5cm} r}
تخم بطی گر چه مرغ خانه‌ات
&&
کرد زیر پر چو دایه تربیت
\\
مادر تو بط آن دریا بدست
&&
دایه‌ات خاکی بد و خشکی‌پرست
\\
میل دریا که دل تو اندرست
&&
آن طبیعت جانت را از مادرست
\\
میل خشکی مر ترا زین دایه است
&&
دایه را بگذار کو بدرایه است
\\
دایه را بگذار در خشک و بران
&&
اندر آ در بحر معنی چون بطان
\\
گر ترا مادر بترساند ز آب
&&
تو مترس و سوی دریا ران شتاب
\\
تو بطی بر خشک و بر تر زنده‌ای
&&
نی چو مرغ خانه خانه‌گنده‌ای
\\
تو ز کرمنا بنی آدم شهی
&&
هم به خشکی هم به دریا پا نهی
\\
که حملناهم علی البحر بجان
&&
از حملناهم علی البر پیش ران
\\
مر ملایک را سوی بر راه نیست
&&
جنس حیوان هم ز بحر آگاه نیست
\\
تو بتن حیوان بجانی از ملک
&&
تا روی هم بر زمین هم بر فلک
\\
تا بظاهر مثلکم باشد بشر
&&
با دل یوحی الیه دیده‌ور
\\
قالب خاکی فتاده بر زمین
&&
روح او گردان برین چرخ برین
\\
ما همه مرغابیانیم ای غلام
&&
بحر می‌داند زبان ما تمام
\\
پس سلیمان بحر آمد ما چو طیر
&&
در سلیمان تا ابد داریم سیر
\\
با سلیمان پای در دریا بنه
&&
تا چو داود آب سازد صد زره
\\
آن سلیمان پیش جمله حاضرست
&&
لیک غیرت چشم‌بند و ساحرست
\\
تا ز جهل و خوابناکی و فضول
&&
او بپیش ما و ما از وی ملول
\\
تشنه را درد سر آرد بانگ رعد
&&
چون نداند کو کشاند ابر سعد
\\
چشم او ماندست در جوی روان
&&
بی‌خبر از ذوق آب آسمان
\\
مرکب همت سوی اسباب راند
&&
از مسبب لاجرم محجوب ماند
\\
آنک بیند او مسبب را عیان
&&
کی نهد دل بر سببهای جهان
\\
\end{longtable}
\end{center}
