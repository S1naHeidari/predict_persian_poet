\begin{center}
\section*{بخش ۶۵ - باز تقریر کردن معاویه با ابلیس مکر او را}
\label{sec:sh065}
\addcontentsline{toc}{section}{\nameref{sec:sh065}}
\begin{longtable}{l p{0.5cm} r}
گفت امیر او را که اینها راستست
&&
لیک بخش تو ازینها کاستست
\\
صد هزاران را چو من تو ره زدی
&&
حفره کردی در خزینه آمدی
\\
آتشی از تو نسوزم چاره نیست
&&
کیست کز دست تو جامه‌ش پاره نیست
\\
طبعت ای آتش چو سوزانیدنیست
&&
تا نسوزانی تو چیزی چاره نیست
\\
لعنت این باشد که سوزانت کند
&&
اوستاد جمله دزدانت کند
\\
با خدا گفتی شنیدی روبرو
&&
من چه باشم پیش مکرت ای عدو
\\
معرفتهای تو چون بانگ صفیر
&&
بانگ مرغانست لیکن مرغ گیر
\\
صد هزاران مرغ را آن ره زدست
&&
مرغ غره کشنایی آمدست
\\
در هوا چون بشنود بانگ صفیر
&&
از هوا آید شود اینجا اسیر
\\
قوم نوح از مکر تو در نوحه‌اند
&&
دل کباب و سینه شرحه شرحه‌اند
\\
عاد را تو باد دادی در جهان
&&
در فکندی در عذاب و اندهان
\\
از تو بود آن سنگسار قوم لوط
&&
در سیاهابه ز تو خوردند غوط
\\
مغز نمرود از تو آمد ریخته
&&
ای هزاران فتنه‌ها انگیخته
\\
عقل فرعون ذکی فیلسوف
&&
کور گشت از تو نیابید او وقوف
\\
بولهب هم از تو نااهلی شده
&&
بوالحکم هم از تو بوجهلی شده
\\
ای برین شطرنج بهر یاد را
&&
مات کرده صد هزار استاد را
\\
ای ز فرزین‌بندهای مشکلت
&&
سوخته دلها سیه گشته دلت
\\
بحر مکری تو خلایق قطره‌ای
&&
تو چو کوهی وین سلیمان ذره‌ای
\\
کی رهد از مکر تو ای مختصم
&&
غرق طوفانیم الا من عصم
\\
بس ستارهٔ سعد از تو محترق
&&
بس سپاه و جمع از تو مفترق
\\
\end{longtable}
\end{center}
