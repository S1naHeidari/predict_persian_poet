\begin{center}
\section*{بخش ۸۶ - قصد کردن غزان بکشتن یک مردی تا آن دگر بترسد}
\label{sec:sh086}
\addcontentsline{toc}{section}{\nameref{sec:sh086}}
\begin{longtable}{l p{0.5cm} r}
آن غزان ترک خون‌ریز آمدند
&&
بهر یغما بر دهی ناگه زدند
\\
دو کس از اعیان آن ده یافتند
&&
در هلاک آن یکی بشتافتند
\\
دست بستندش که قربانش کنند
&&
گفت ای شاهان و ارکان بلند
\\
در چه مرگم چرا می‌افکنید
&&
از چه آخر تشنهٔ خون منید
\\
چیست حکمت چه غرض در کشتنم
&&
چون چنین درویشم و عریان‌تنم
\\
گفت تا هیبت برین یارت زند
&&
تا بترسد او و زر پیدا کند
\\
گفت آخر او ز من مسکین‌ترست
&&
گفت قاصد کرده است او را زرست
\\
گفت چون وهمست ما هر دو یکیم
&&
در مقام احتمال و در شکیم
\\
خود ورا بکشید اول ای شهان
&&
تا بترسم من دهم زر را نشان
\\
پس کرمهای الهی بین که ما
&&
آمدیم آخر زمان در انتها
\\
آخرین قرنها پیش از قرون
&&
در حدیثست آخرون السابقون
\\
تا هلاک قوم نوح و قوم هود
&&
عارض رحمت بجان ما نمود
\\
کشت ایشان را که ما ترسیم ازو
&&
ور خود این بر عکس کردی وای تو
\\
\end{longtable}
\end{center}
