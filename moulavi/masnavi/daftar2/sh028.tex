\begin{center}
\section*{بخش ۲۸ - فهم کردن مریدان کی ذاالنون دیوانه نشد قاصد کرده است}
\label{sec:sh028}
\addcontentsline{toc}{section}{\nameref{sec:sh028}}
\begin{longtable}{l p{0.5cm} r}
دوستان در قصهٔ ذاالنون شدند
&&
سوی زندان و در آن رایی زدند
\\
کین مگر قاصد کند یا حکمتیست
&&
او درین دین قبله‌ای و آیتیست
\\
دور دور از عقل چون دریای او
&&
تا جنون باشد سفه‌فرمای او
\\
حاش لله از کمال جاه او
&&
کابر بیماری بپوشد ماه او
\\
او ز شر عامه اندر خانه شد
&&
او ز ننگ عاقلان دیوانه شد
\\
او ز عار عقل کند تن‌پرست
&&
قاصدا رفتست و دیوانه شدست
\\
که ببندیدم قوی وز ساز گاو
&&
بر سر و پشتم بزن وین را مکاو
\\
تا ز زخم لخت یابم من حیات
&&
چون قتیل از گاو موسی ای ثقات
\\
تا ز زخم لخت گاوی خوش شوم
&&
همچو کشته و گاو موسی گش شوم
\\
زنده شد کشته ز زخم دم گاو
&&
همچو مس از کیمیا شد زر ساو
\\
کشته بر جست و بگفت اسرار را
&&
وا نمود آن زمرهٔ خون‌خوار را
\\
گفت روشن کین جماعت کشته‌اند
&&
کین زمان در خصمیم آشفته‌اند
\\
چونک کشته گردد این جسم گران
&&
زنده گردد هستی اسراردان
\\
جان او بیند بهشت و نار را
&&
باز داند جملهٔ اسرار را
\\
وا نماید خونیان دیو را
&&
وا نماید دام خدعه و ریو را
\\
گاو کشتن هست از شرط طریق
&&
تا شود از زخم دمش جان مفیق
\\
گاو نفس خویش را زوتر بکش
&&
تا شود روح خفی زنده و بهش
\\
\end{longtable}
\end{center}
