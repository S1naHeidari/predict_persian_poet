\begin{center}
\section*{بخش ۷۹ - اندیشیدن یکی از صحابه بانکار کی رسول چرا ستاری نمی‌کند}
\label{sec:sh079}
\addcontentsline{toc}{section}{\nameref{sec:sh079}}
\begin{longtable}{l p{0.5cm} r}
تا یکی یاری ز یاران رسول
&&
در دلش انکار آمد زان نکول
\\
که چنین پیران با شیب و وقار
&&
می‌کندشان این پیمبر شرمسار
\\
کو کرم کو سترپوشی کو حیا
&&
صد هزاران عیب پوشند انبیا
\\
باز در دل زود استغفار کرد
&&
تا نگردد ز اعتراض او روی‌زرد
\\
شومی یاری اصحاب نفاق
&&
کرد مؤمن را چو ایشان زشت و عاق
\\
باز می‌زارید کای علام سر
&&
مر مرا مگذار بر کفران مصر
\\
دل به دستم نیست همچون دید چشم
&&
ورنه دل را سوزمی این دم ز خشم
\\
اندرین اندیشه خوابش در ربود
&&
مسجد ایشانش پر سرگین نمود
\\
سنگهاش اندر حدث جای تباه
&&
می‌دمید از سنگها دود سیاه
\\
دود در حلقش شد و حلقش بخست
&&
از نهیب دود تلخ از خواب جست
\\
در زمان در رو فتاد و می‌گریست
&&
کای خدا اینها نشان منکریست
\\
خلم بهتر از چنین حلم ای خدا
&&
که کند از نور ایمانم جدا
\\
گر بکاوی کوشش اهل مجاز
&&
تو بتو گنده بود همچون پیاز
\\
هر یکی از یکدگر بی مغزتر
&&
صادقان را یک ز دیگر نغزتر
\\
صد کمر آن قوم بسته بر قبا
&&
بهر هدم مسجد اهل قبا
\\
همچو آن اصحاب فیل اندر حبش
&&
کعبه‌ای کردند حق آتش زدش
\\
قصد کعبه ساختند از انتقام
&&
حالشان چون شد فرو خوان از کلام
\\
مر سیه‌رویان دین را خود جهاز
&&
نیست الا حیلت و مکر و ستیز
\\
هر صحابی دید زان مسجد عیان
&&
واقعه تا شد یقینشان سر آن
\\
واقعات ار باز گویم یک بیک
&&
پس یقین گردد صفا بر اهل شک
\\
لیک می‌ترسم ز کشف رازشان
&&
نازنینانند و زیبد نازشان
\\
شرع بی تقلید می‌پذرفته‌اند
&&
بی محک آن نقد را بگرفته‌اند
\\
حکمت قرآن چو ضالهٔمؤمنست
&&
هر کسی در ضالهٔ خود موقنست
\\
\end{longtable}
\end{center}
