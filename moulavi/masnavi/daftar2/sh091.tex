\begin{center}
\section*{بخش ۹۱ - قصهٔ تیراندازی و ترسیدن او از سواری کی در بیشه می‌رفت}
\label{sec:sh091}
\addcontentsline{toc}{section}{\nameref{sec:sh091}}
\begin{longtable}{l p{0.5cm} r}
یک سواری با سلاح و بس مهیب
&&
می‌شد اندر بیشه بر اسپی نجیب
\\
تیراندازی بحکم او را بدید
&&
پس ز خوف او کمان را در کشید
\\
تا زند تیری سوارش بانگ زد
&&
من ضعیفم گرچه زفتستم جسد
\\
هان و هان منگر تو در زفتی من
&&
که کمم در وقت جنگ از پیرزن
\\
گفت رو که نیک گفتی ورنه نیش
&&
بر تو می‌انداختم از ترس خویش
\\
بس کسان را کلت پیگار کشت
&&
بی رجولیت چنان تیغی به مشت
\\
گر بپوشی تو سلاح رستمان
&&
رفت جانت چون نباشی مرد آن
\\
جان سپر کن تیغ بگذار ای پسر
&&
هر که بی سر بود ازین شه برد سر
\\
آن سلاحت حیله و مکر توست
&&
هم ز تو زایید و هم جان تو خست
\\
چون نکردی هیچ سودی زین حیل
&&
ترک حیلت کن که پیش آید دول
\\
چون یکی لحظه نخوردی بر ز فن
&&
ترک فن گو می‌طلب رب المنن
\\
چون مبارک نیست بر تو این علوم
&&
خویشتن گولی کن و بگذر ز شوم
\\
چون ملایک گو که لا علم لنا
&&
یا الهی غیر ما علمتنا
\\
\end{longtable}
\end{center}
