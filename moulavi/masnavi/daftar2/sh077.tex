\begin{center}
\section*{بخش ۷۷ - قصهٔ منافقان و مسجد ضرار ساختن ایشان}
\label{sec:sh077}
\addcontentsline{toc}{section}{\nameref{sec:sh077}}
\begin{longtable}{l p{0.5cm} r}
یک مثال دیگر اندر کژروی
&&
شاید ار از نقل قرآن بشنوی
\\
این چنین کژ بازیی در جفت و طاق
&&
با نبی می‌باختند اهل نفاق
\\
کز برای عز دین احمدی
&&
مسجدی سازیم و بود آن مرتدی
\\
این چنین کژ بازیی می‌باختند
&&
مسجدی جز مسجد او ساختند
\\
سقف و فرش و قبه‌اش آراسته
&&
لیک تفریق جماعت خواسته
\\
نزد پیغامبر بلابه آمدند
&&
همچو اشتر پیش او زانو زدند
\\
کای رسول حق برای محسنی
&&
سوی آن مسجد قدم رنجه کنی
\\
تا مبارک گردد از اقدام تو
&&
تا قیامت تازه بادا نام تو
\\
مسجد روز گلست و روز ابر
&&
مسجد روز ضرورت وقت فقر
\\
تا غریبی یابد آنجا خیر و جا
&&
تا فراوان گردد این خدمت‌سرا
\\
تا شعار دین شود بسیار و پر
&&
زانک با یاران شود خوش کار مر
\\
ساعتی آن جایگه تشریف ده
&&
تزکیه‌مان کن ز ما تعریف ده
\\
مسجد و اصحاب مسجد را نواز
&&
تو مهی ما شب دمی با ما بساز
\\
تا شود شب از جمالت همچو روز
&&
ای جمالت آفتاب جان‌فروز
\\
ای دریغا کان سخن از دل بدی
&&
تا مراد آن نفر حاصل شدی
\\
لطف کاید بی دل و جان در زبان
&&
همچو سبزهٔ تون بود ای دوستان
\\
هم ز دورش بنگر و اندر گذر
&&
خوردن و بو را نشاید ای پسر
\\
سوی لطف بی وفایان هین مرو
&&
کان پل ویران بود نیکو شنو
\\
گر قدم را جاهلی بر وی زند
&&
بشکند پل و آن قدم را بشکند
\\
هر کجا لشکر شکسته میشود
&&
از دو سه سست مخنث می‌بود
\\
در صف آید با سلاح او مردوار
&&
دل برو بنهند کاینک یار غار
\\
رو بگرداند چو بیند زخم را
&&
رفتن او بشکند پشت ترا
\\
این درازست و فراوان می‌شود
&&
وآنچ مقصودست پنهان می‌شود
\\
\end{longtable}
\end{center}
