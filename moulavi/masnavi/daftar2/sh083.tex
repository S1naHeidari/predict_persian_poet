\begin{center}
\section*{بخش ۸۳ - شرح فایدهٔ حکایت آن شخص شتر جوینده}
\label{sec:sh083}
\addcontentsline{toc}{section}{\nameref{sec:sh083}}
\begin{longtable}{l p{0.5cm} r}
اشتری گم کرده‌ای ای معتمد
&&
هر کسی ز اشتر نشانت می‌دهد
\\
تو نمی‌دانی که آن اشتر کجاست
&&
لیک دانی کین نشانیها خطاست
\\
وانک اشتر گم نکرد او از مری
&&
همچو آن گم کرده جوید اشتری
\\
که بلی من هم شتر گم کرده‌ام
&&
هر که یابد اجرتش آورده‌ام
\\
تا در اشتر با تو انبازی کند
&&
بهر طمع اشتر این بازی کند
\\
او نشان کژ بشناسد ز راست
&&
لیک گفتت آن مقلد را عصاست
\\
هرچه را گویی خطا بود آن نشان
&&
او به تقلید تو می‌گوید همان
\\
چون نشان راست گویند و شبیه
&&
پس یقین گردد ترا لا ریب فیه
\\
آن شفای جان رنجورت شود
&&
رنگ روی و صحت و زورت شود
\\
چشم تو روشن شود پایت دوان
&&
جسم تو جان گردد و جانت روان
\\
پس بگویی راست گفتی ای امین
&&
این نشانیها بلاغ آمد مبین
\\
فیه آیات ثقات بینات
&&
این براتی باشد و قدر نجات
\\
این نشان چون داد گویی پیش رو
&&
وقت آهنگست پیش‌آهنگ شو
\\
پی روی تو کنم ای راست‌گو
&&
بوی بردی ز اشترم بنما که کو
\\
پیش آنکس که نه صاحب اشتریست
&&
کو درین جست شتر بهر مریست
\\
زین نشان راست نفزودش یقین
&&
جز ز عکس ناقه‌جوی راستین
\\
بوی برد از جد و گرمیهای او
&&
که گزافه نیست این هیهای او
\\
اندرین اشتر نبودش حق ولی
&&
اشتری گم کرده است او هم بلی
\\
طمع ناقهٔ غیر روپوشش شده
&&
آنچ ازو گم شد فراموشش شده
\\
هر کجا او می‌دود این می‌دود
&&
از طمع هم‌درد صاحب می‌شود
\\
کاذبی با صادقی چون شد روان
&&
آن دروغش راستی شد ناگهان
\\
اندر آن صحرا که آن اشتر شتافت
&&
اشتر خود نیز آن دیگر بیافت
\\
چون بدیدش یاد آورد آن خویش
&&
بی طمع شد ز اشتر آن یار و خویش
\\
آن مقلد شد محقق چون بدید
&&
اشتر خود را که آنجا می‌چرید
\\
او طلب‌کار شتر آن لحظه گشت
&&
می‌نجستش تا ندید او را بدشت
\\
بعد از آن تنهاروی آغاز کرد
&&
چشم سوی ناقهٔ خود باز کرد
\\
گفت آن صادق مرا بگذاشتی
&&
تا باکنون پاس من می‌داشتی
\\
گفت تا اکنون فسوسی بوده‌ام
&&
وز طمع در چاپلوسی بوده‌ام
\\
این زمان هم درد تو گشتم که من
&&
در طلب از تو جدا گشتم بتن
\\
از تو می‌دزدیدمی وصف شتر
&&
جان من دید آن خود شد چشم‌پر
\\
تا نیابیدم نبودم طالبش
&&
مس کنون مغلوب شد زر غالبش
\\
سیتم شد همه طاعات شکر
&&
هزل شد فانی و جد اثبات شکر
\\
سیتم چون وسیلت شد بحق
&&
پس مزن بر سیتم هیچ دق
\\
مر ترا صدق تو طالب کرده بود
&&
مر مرا جد و طلب صدقی گشود
\\
صدق تو آورد در جستن ترا
&&
جستنم آورد در صدقی مرا
\\
تخم دولت در زمین می‌کاشتم
&&
سخره و بیگار می‌پنداشتم
\\
آن نبد بیگار کسبی بود چست
&&
هر یکی دانه که کشتم صد برست
\\
دزد سوی خانه‌ای شد زیر دست
&&
چون در آمد دید کان خانهٔ خودست
\\
گرم باش ای سرد تا گرمی رسد
&&
با درشتی ساز تا نرمی رسد
\\
آن دو اشتر نیست آن یک اشترست
&&
تنگ آمد لفظ معنی بس پرست
\\
لفظ در معنی همیشه نارسان
&&
زان پیمبر گفت قد کل لسان
\\
نطق اصطرلاب باشد در حساب
&&
چه قدر داند ز چرخ و آفتاب
\\
خاصه چرخی کین فلک زو پره‌ایست
&&
آفتاب از آفتابش ذره‌ایست
\\
\end{longtable}
\end{center}
