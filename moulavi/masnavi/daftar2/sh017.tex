\begin{center}
\section*{بخش ۱۷ - شکایت کردن اهل زندان پیش وکیل قاضی از دست آن مفلس}
\label{sec:sh017}
\addcontentsline{toc}{section}{\nameref{sec:sh017}}
\begin{longtable}{l p{0.5cm} r}
با وکیل قاضی ادراک‌مند
&&
اهل زندان در شکایت آمدند
\\
که سلام ما به قاضی بر کنون
&&
بازگو آزار ما زین مرد دون
\\
کندرین زندان بماند او مستمر
&&
یاوه‌تاز و طبل‌خوارست و مضر
\\
چون مگس حاضر شود در هر طعام
&&
از وقاحت بی صلا و بی سلام
\\
پیش او هیچست لوت شصت کس
&&
کر کند خود را اگر گوییش بس
\\
مرد زندان را نیاید لقمه‌ای
&&
ور به صد حیلت گشاید طعمه‌ای
\\
در زمان پیش آید آن دوزخ گلو
&&
حجتش این که خدا گفتا کلوا
\\
زین چنین قحط سه‌ساله داد داد
&&
ظل مولانا ابد پاینده باد
\\
یا ز زندان تا رود این گاومیش
&&
یا وظیفه کن ز وقفی لقمه‌ایش
\\
ای ز تو خوش هم ذکور و هم اناث
&&
داد کن المستغاث المستغاث
\\
سوی قاضی شد وکیل با نمک
&&
گفت با قاضی شکایت یک بیک
\\
خواند او را قاضی از زندان به پیش
&&
پس تفحص کرد از اعیان خویش
\\
گشت ثابت پیش قاضی آن همه
&&
که نمودند از شکایت آن رمه
\\
گفت قاضی خیز ازین زندان برو
&&
سوی خانهٔ مردریگ خویش شو
\\
گفت خان و مان من احسان تست
&&
همچو کافر جنتم زندان تست
\\
گر ز زندانم برانی تو برد
&&
خود بمیرم من ز تقصیری و کد
\\
همچو ابلیسی که می‌گفت ای سلام
&&
رب انظرنی الی یوم القیام
\\
کاندرین زندان دنیا من خوشم
&&
تا که دشمن‌زادگان را می‌کشم
\\
هر که او را قوت ایمانی بود
&&
وز برای زاد ره نانی بود
\\
می‌ستانم گه به مکر و گه به ریو
&&
تا بر آرند از پشیمانی غریو
\\
گه به درویشی کنم تهدیدشان
&&
گه به زلف و خال بندم دیدشان
\\
قوت ایمانی درین زندان کمست
&&
وانک هست از قصد این سگ در خمست
\\
از نماز و صوم و صد بیچارگی
&&
قوت ذوق آید برد یکبارگی
\\
استعیذ الله من شیطانه
&&
قد هلکنا آه من طغیانه
\\
یک سگست و در هزاران می‌رود
&&
هر که در وی رفت او او می‌شود
\\
هر که سردت کرد می‌دان کو دروست
&&
دیو پنهان گشته اندر زیر پوست
\\
چون نیابد صورت آید در خیال
&&
تا کشاند آن خیالت در وبال
\\
گه خیال فرجه و گاهی دکان
&&
گه خیال علم و گاهی خان و مان
\\
هان بگو لا حولها اندر زمان
&&
از زبان تنها نه بلک از عین جان
\\
\end{longtable}
\end{center}
