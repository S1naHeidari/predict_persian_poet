\begin{center}
\section*{بخش ۹۴ - آغاز منور شدن عارف بنور غیب‌بین}
\label{sec:sh094}
\addcontentsline{toc}{section}{\nameref{sec:sh094}}
\begin{longtable}{l p{0.5cm} r}
چون یکی حس در روش بگشاد بند
&&
ما بقی حسها همه مبدل شوند
\\
چون یکی حس غیر محسوسات دید
&&
گشت غیبی بر همه حسها پدید
\\
چون ز جو جست از گله یک گوسفند
&&
پس پیاپی جمله زان سو برجهند
\\
گوسفندان حواست را بران
&&
در چرا از اخرج المرعی چران
\\
تا در آنجا سنبل و ریحان چرند
&&
تا به گلزار حقایق ره برند
\\
هر حست پیغامبر حسها شود
&&
تا یکایک سوی آن جنت رود
\\
حسها با حس تو گویند راز
&&
بی حقیقت بی زبان و بی مجاز
\\
کین حقیقت قابل تاویلهاست
&&
وین توهم مایه تخییلهاست
\\
آن حقیقت را که باشد از عیان
&&
هیچ تاویلی نگنجد در میان
\\
چونک هر حس بندهٔ حس تو شد
&&
مر فلکها را نباشد از تو بد
\\
چونک دعویی رود در ملک پوست
&&
مغز آن کی بود قشر آن اوست
\\
چون تنازع در فتد در تنگ کاه
&&
دانه آن کیست آن را کن نگاه
\\
پس فلک قشرست و نور روح مغز
&&
این پدیدست آن خفی زین رو ملغز
\\
جسم ظاهر روح مخفی آمدست
&&
جسم همچون آستین جان همچو دست
\\
باز عقل از روح مخفی‌تر پرد
&&
حس سوی روح زوتر ره برد
\\
جنبشی بینی بدانی زنده است
&&
این ندانی که ز عقل آکنده است
\\
تا که جنبشهای موزون سر کند
&&
جنبش مس را به دانش زر کند
\\
زان مناسب آمدن افعال دست
&&
فهم آید مر ترا که عقل هست
\\
روح وحی از عقل پنهان‌تر بود
&&
زانک او غیبیست او زان سر بود
\\
عقل احمد از کسی پنهان نشد
&&
روح وحیش مدرک هر جان نشد
\\
روح وحیی را مناسبهاست نیز
&&
در نیابد عقل کان آمد عزیز
\\
گه جنون بیند گهی حیران شود
&&
زانک موقوفست تا او آن شود
\\
چون مناسبهای افعال خضر
&&
عقل موسی بود در دیدش کدر
\\
نامناسب می‌نمود افعال او
&&
پیش موسی چون نبودش حال او
\\
عقل موسی چون شود در غیب بند
&&
عقل موشی خود کیست ای ارجمند
\\
علم تقلیدی بود بهر فروخت
&&
چون بیابد مشتری خوش بر فروخت
\\
مشتری علم تحقیقی حقست
&&
دایما بازار او با رونقست
\\
لب ببسته مست در بیع و شری
&&
مشتری بی حد که الله اشتری
\\
درس آدم را فرشته مشتری
&&
محرم درسش نه دیوست و پری
\\
آدم انبئهم باسما درس گو
&&
شرح کن اسرار حق را مو بمو
\\
آنچنان کس را که کوته‌بین بود
&&
در تلون غرق و بی تمکین بود
\\
موش گفتم زانک در خاکست جاش
&&
خاک باشد موش را جای معاش
\\
راهها داند ولی در زیر خاک
&&
هر طرف او خاک را کردست چاک
\\
نفس موشی نیست الا لقمه‌رند
&&
قدر حاجت موش را عقلی دهند
\\
زانک بی حاجت خداوند عزیز
&&
می‌نبخشد هیچ کس را هیچ چیز
\\
گر نبودی حاجت عالم زمین
&&
نافریدی هیچ رب العالمین
\\
وین زمین مضطرب محتاج کوه
&&
گر نبودی نافریدی پر شکوه
\\
ور نبودی حاجت افلاک هم
&&
هفت گردون ناوریدی از عدم
\\
آفتاب و ماه و این استارگان
&&
جز بحاجت کی پدید آمد عیان
\\
پس کمند هستها حاجت بود
&&
قدر حاجت مرد را آلت دهد
\\
پس بیفزا حاجت ای محتاج زود
&&
تا بجوشد در کرم دریای جود
\\
این گدایان بر ره و هر مبتلا
&&
حاجت خود می‌نماید خلق را
\\
کوری و شلی و بیماری و درد
&&
تا ازین حاجت بجنبد رحم مرد
\\
هیچ گوید نان دهید ای مردمان
&&
که مرا مالست و انبارست و خوان
\\
چشم ننهادست حق در کورموش
&&
زانک حاجت نیست چشمش بهر نوش
\\
می‌تواند زیست بی چشم و بصر
&&
فارغست از چشم او در خاک تر
\\
جز بدزدی او برون ناید ز خاک
&&
تا کند خالق از آن دزدیش پاک
\\
بعد از آن پر یابد و مرغی شود
&&
چون ملایک جانب گردون رود
\\
هر زمان در گلشن شکر خدا
&&
او بر آرد همچو بلبل صد نوا
\\
کای رهاننده مرا از وصف زشت
&&
ای کننده دوزخی را تو بهشت
\\
در یکی پیهی نهی تو روشنی
&&
استخوانی را دهی سمع ای غنی
\\
چه تعلق آن معانی را به جسم
&&
چه تعلق فهم اشیا را به اسم
\\
لفظ چون وکرست و معنی طایرست
&&
جسم جوی و روح آب سایرست
\\
او روانست و تو گویی واقفست
&&
او دوانست و تو گویی عاکفست
\\
گر نبینی سیر آب از چاکها
&&
چیست بر وی نو بنو خاشاکها
\\
هست خاشاک تو صورتهای فکر
&&
نو بنو در می‌رسد اشکال بکر
\\
روی آب و جوی فکر اندر روش
&&
نیست بی خاشاک محبوب و وحش
\\
قشرها بر روی این آب روان
&&
از ثمار باغ غیبی شد دوان
\\
قشرها را مغز اندر باغ جو
&&
زانک آب از باغ می‌آید به جو
\\
گر نبینی رفتن آب حیات
&&
بنگر اندر جوی و این سیر نبات
\\
آب چون انبه‌تر آید در گذر
&&
زو کند قشر صور زوتر گذر
\\
چون بغایت تیز شد این جو روان
&&
غم نپاید در ضمیر عارفان
\\
چون بغایت ممتلی بود و شتاب
&&
پس نگنجید اندرو الا که آب
\\
\end{longtable}
\end{center}
