\begin{center}
\section*{بخش ۸ - التزام کردن خادم تعهد بهیمه را و تخلف نمودن}
\label{sec:sh008}
\addcontentsline{toc}{section}{\nameref{sec:sh008}}
\begin{longtable}{l p{0.5cm} r}
حلقهٔ آن صوفیان مستفید
&&
چونک در وجد و طرب آخر رسید
\\
خوان بیاوردند بهر میهمان
&&
از بهیمه یاد آورد آن زمان
\\
گفت خادم را که در آخر برو
&&
راست کن بهر بهیمه کاه و جو
\\
گفت لا حول این چه افزون گفتنست
&&
از قدیم این کارها کار منست
\\
گفت تر کن آن جوش را از نخست
&&
کان خر پیرست و دندانهاش سست
\\
گفت لا حول این چه می‌گویی مها
&&
از من آموزند این ترتیبها
\\
گفت پالانش فرو نه پیش پیش
&&
داروی منبل بنه بر پشت ریش
\\
گفت لا حول آخر ای حکمت‌گزار
&&
جنس تو مهمانم آمد صد هزار
\\
جمله راضی رفته‌اند از پیش ما
&&
هست مهمان جان ما و خویش ما
\\
گفت آبش ده ولیکن شیر گرم
&&
گفت لا حول از توم بگرفت شرم
\\
گفت اندر جو تو کمتر کاه‌کن
&&
گفت لا حول این سخن کوتاه کن
\\
گفت جایش را بروب از سنگ و پشک
&&
ور بود تر ریز بر وی خاک خشک
\\
گفت لا حول ای پدر لا حول کن
&&
با رسول اهل کمتر گو سخن
\\
گفت بستان شانه پشت خر بخار
&&
گفت لا حول ای پدر شرمی بدار
\\
خادم این گفت و میان را بست چست
&&
گفت رفتم کاه و جو آرم نخست
\\
رفت و از آخر نکرد او هیچ یاد
&&
خواب خرگوشی بدان صوفی بداد
\\
رفت خادم جانب اوباش چند
&&
کرد بر اندرز صوفی ریش‌خند
\\
صوفی از ره مانده بود و شد دراز
&&
خوابها می‌دید با چشم فراز
\\
کان خرش در چنگ گرگی مانده بود
&&
پاره‌ها از پشت و رانش می‌ربود
\\
گفت لا حول این چه مالیخولیاست
&&
ای عجب آن خادم مشفق کجاست
\\
باز می‌دید آن خرش در راه‌رو
&&
گه به چاهی می‌فتاد و گه بگو
\\
گونه‌گون می‌دید ناخوش واقعه
&&
فاتحه می‌خواند او والقارعه
\\
گفت چاره چیست یاران جسته‌اند
&&
رفته‌اند و جمله درها بسته‌اند
\\
باز می‌گفت ای عجب آن خادمک
&&
نه که با ما گشت هم‌نان و نمک
\\
من نکردم با وی الا لطف و لین
&&
او چرا با من کند برعکس کین
\\
هر عداوت را سبب باید سند
&&
ورنه جنسیت وفا تلقین کند
\\
باز می‌گفت آدم با لطف و جود
&&
کی بر آن ابلیس جوری کرده بود
\\
آدمی مر مار و کزدم را چه کرد
&&
کو همی‌خواهد مرورا مرگ و درد
\\
گرگ را خود خاصیت بدریدنست
&&
این حسد در خلق آخر روشنست
\\
باز می‌گفت این گمان بد خطاست
&&
بر برادر این چنین ظنم چراست
\\
باز گفتی حزم سؤ الظن تست
&&
هر که بدظن نیست کی ماند درست
\\
صوفی اندر وسوسه وان خر چنان
&&
که چنین بادا جزای دشمنان
\\
آن خر مسکین میان خاک و سنگ
&&
کژ شده پالان دریده پالهنگ
\\
کشته از ره جملهٔ شب بی علف
&&
گاه در جان کندن و گه در تلف
\\
خر همه شب ذکر می‌کرد ای اله
&&
جو رها کردم کم از یک مشت کاه
\\
با زبان حال می‌گفت ای شیوخ
&&
رحمتی که سوختم زین خام شوخ
\\
آنچ آن خر دید از رنج و عذاب
&&
مرغ خاکی بیند اندر سیل آب
\\
بس به پهلو گشت آن شب تا سحر
&&
آن خر بیچاره از جوع البقر
\\
روز شد خادم بیامد بامداد
&&
زود پالان جست بر پشتش نهاد
\\
خر فروشانه دو سه زخمش بزد
&&
کرد با خر آنچ زان سگ می‌سزد
\\
خر جهنده گشت از تیزی نیش
&&
کو زبان تا خر بگوید حال خویش
\\
\end{longtable}
\end{center}
