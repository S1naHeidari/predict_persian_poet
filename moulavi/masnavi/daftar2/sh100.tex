\begin{center}
\section*{بخش ۱۰۰ - کشیدن موش مهار شتر را و معجب  شدن موش در خود}
\label{sec:sh100}
\addcontentsline{toc}{section}{\nameref{sec:sh100}}
\begin{longtable}{l p{0.5cm} r}
موشکی در کف مهار اشتری
&&
در ربود و شد روان او از مری
\\
اشتر از چستی که با او شد روان
&&
موش غره شد که هستم پهلوان
\\
بر شتر زد پرتو اندیشه‌اش
&&
گفت بنمایم ترا تو باش خوش
\\
تا بیامد بر لب جوی بزرگ
&&
کاندرو گشتی زبون پیل سترگ
\\
موش آنجا ایستاد و خشک گشت
&&
گفت اشتر ای رفیق کوه و دشت
\\
این توقف چیست حیرانی چرا
&&
پا بنه مردانه اندر جو در آ
\\
تو قلاوزی و پیش‌آهنگ من
&&
درمیان ره مباش و تن مزن
\\
گفت این آب شگرفست و عمیق
&&
من همی‌ترسم ز غرقاب ای رفیق
\\
گفت اشتر تا ببینم حد آب
&&
پا درو بنهاد آن اشتر شتاب
\\
گفت تا زانوست آب ای کور موش
&&
از چه حیران گشتی و رفتی ز هوش
\\
گفت مور تست و ما را اژدهاست
&&
که ز زانو تا به زانو فرقهاست
\\
گر ترا تا زانو است ای پر هنر
&&
مر مرا صد گز گذشت از فرق سر
\\
گفت گستاخی مکن بار دگر
&&
تا نسوزد جسم و جانت زین شرر
\\
تو مری با مثل خود موشان بکن
&&
با شتر مر موش را نبود سخن
\\
گفت توبه کردم از بهر خدا
&&
بگذران زین آب مهلک مر مرا
\\
رحم آمد مر شتر را گفت هین
&&
برجه و بر کودبان من نشین
\\
این گذشتن شد مسلم مر مرا
&&
بگذرانم صد هزاران چون ترا
\\
چون پیمبر نیستی پس رو به راه
&&
تا رسی از چاه روزی سوی جاه
\\
تو رعیت باش چون سلطان نه‌ای
&&
خود مران چون مرد کشتیبان نه‌ای
\\
چون نه‌ای کامل دکان تنها مگیر
&&
دست‌خوش می‌باش تا گردی خمیر
\\
انصتوا را گوش کن خاموش باش
&&
چون زبان حق نگشتی گوش باش
\\
ور بگویی شکل استفسار گو
&&
با شهنشاهان تو مسکین‌وار گو
\\
ابتدای کبر و کین از شهوتست
&&
راسخی شهوتت از عادتست
\\
چون ز عادت گشت محکم خوی بد
&&
خشم آید بر کسی کت واکشد
\\
چونک تو گل‌خوار گشتی هر ک او
&&
واکشد از گل ترا باشد عدو
\\
بت‌پرستان چونک گرد بت تنند
&&
مانعان راه خود را دشمن‌اند
\\
چونک کرد ابلیس خو با سروری
&&
دید آدم را حقیر او از خری
\\
که به از من سروری دیگر بود
&&
تا که او مسجود چون من کس شود
\\
سروری زهرست جز آن روح را
&&
کو بود تریاق‌لانی ز ابتدا
\\
کوه اگر پر مار شد باکی مدار
&&
کو بود اندر درون تریاق‌زار
\\
سروری چون شد دماغت را ندیم
&&
هر که بشکستت شود خصم قدیم
\\
چون خلاف خوی تو گوید کسی
&&
کینه‌ها خیزد ترا با او بسی
\\
که مرا از خوی من بر می‌کند
&&
خویش را بر من چو سرور می‌کند
\\
چون نباشد خوی بد سرکش درو
&&
کی فروزد از خلاف آتش درو
\\
با مخالف او مدارایی کند
&&
در دل او خویش را جایی کند
\\
زانک خوی بد نگشتست استوار
&&
مور شهوت شد ز عادت همچو مار
\\
مار شهوت را بکش در ابتلا
&&
ورنه اینک گشت مارت اژدها
\\
لیک هر کس مور بیند مار خویش
&&
تو ز صاحب‌دل کن استفسار خویش
\\
تا نشد زر مس نداند من مسم
&&
تا نشد شه دل نداند مفلسم
\\
خدمت اکسیر کن مس‌وار تو
&&
جور می‌کش ای دل از دلدار تو
\\
کیست دلدار اهل دل نیکو بدان
&&
که چو روز و شب جهانند از جهان
\\
عیب کم گو بندهٔ الله را
&&
متهم کم کن به دزدی شاه را
\\
\end{longtable}
\end{center}
