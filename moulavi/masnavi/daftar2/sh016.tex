\begin{center}
\section*{بخش ۱۶ - تعریف کردن منادیان قاضی مفلس را گرد شهر}
\label{sec:sh016}
\addcontentsline{toc}{section}{\nameref{sec:sh016}}
\begin{longtable}{l p{0.5cm} r}
بود شخصی مفلسی بی خان و مان
&&
مانده در زندان و بند بی امان
\\
لقمهٔ زندانیان خوردی گزاف
&&
بر دل خلق از طمع چون کوه قاف
\\
زهره نه کس را که لقمهٔ نان خورد
&&
زانک آن لقمه‌ربا گاوش برد
\\
هر که دور از دعوت رحمان بود
&&
او گداچشمست اگر سلطان بود
\\
مر مروت را نهاده زیر پا
&&
گشته زندان دوزخی زان نان‌ربا
\\
گر گریزی بر امید راحتی
&&
زان طرف هم پیشت آید آفتی
\\
هیچ کنجی بی دد و بی دام نیست
&&
جز بخلوتگاه حق آرام نیست
\\
کنج زندان جهان ناگزیر
&&
نیست بی پامزد و بی دق الحصیر
\\
والله ار سوراخ موشی در روی
&&
مبتلای گربه چنگالی شوی
\\
آدمی را فربهی هست از خیال
&&
گر خیالاتش بود صاحب‌جمال
\\
ور خیالاتش نماید ناخوشی
&&
می‌گذارد همچو موم از آتشی
\\
در میان مار و کزدم گر ترا
&&
با خیالات خوشان دارد خدا
\\
مار و کزدم مر ترا مونس بود
&&
کان خیالت کیمیای مس بود
\\
صبر شیرین از خیال خوش شدست
&&
کان خیالات فرج پیش آمدست
\\
آن فرج آید ز ایمان در ضمیر
&&
ضعف ایمان ناامیدی و زحیر
\\
صبر از ایمان بیابد سر کله
&&
حیث لا صبر فلا ایمان له
\\
گفت پیغامبر خداش ایمان نداد
&&
هر که را صبری نباشد در نهاد
\\
آن یکی در چشم تو باشد چو مار
&&
هم وی اندر چشم آن دیگر نگار
\\
زانک در چشمت خیال کفر اوست
&&
وان خیال مؤمنی در چشم دوست
\\
کاندرین یک شخص هر دو فعل هست
&&
گاه ماهی باشد او و گاه شست
\\
نیم او مؤمن بود نیمیش گبر
&&
نیم او حرص‌آوری نیمیش صبر
\\
گفت یزدانت فمنکم مؤمن
&&
باز منکم کافر گبر کهن
\\
همچو گاوی نیمهٔ چپش سیاه
&&
نیمهٔ دیگر سپید همچو ماه
\\
هر که این نیمه ببیند رد کند
&&
هر که آن نیمه ببیند کد کند
\\
یوسف اندر چشم اخوان چون ستور
&&
هم وی اندر چشم یعقوبی چو حور
\\
از خیال بد مرورا زشت دید
&&
چشم فرع و چشم اصلی ناپدید
\\
چشم ظاهر سایهٔ آن چشم دان
&&
هرچه آن بیند بگردد این بدان
\\
تو مکانی اصل تو در لامکان
&&
این دکان بر بند و بگشا آن دکان
\\
شش جهت مگریز زیرا در جهات
&&
ششدره‌ست و ششدره ماتست مات
\\
\end{longtable}
\end{center}
