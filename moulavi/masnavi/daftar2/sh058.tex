\begin{center}
\section*{بخش ۵۸ - خواندن محتسب مست خراب افتاده را به زندان}
\label{sec:sh058}
\addcontentsline{toc}{section}{\nameref{sec:sh058}}
\begin{longtable}{l p{0.5cm} r}
محتسب در نیم شب جایی رسید
&&
در بن دیوار مستی خفته دید
\\
گفت هی مستی چه خوردستی بگو
&&
گفت ازین خوردم که هست اندر سبو
\\
گفت آخر در سبو واگو که چیست
&&
گفت از آنک خورده‌ام گفت این خفیست
\\
گفت آنچ خورده‌ای آن چیست آن
&&
گفت آنک در سبو مخفیست آن
\\
دور می‌شد این سؤال و این جواب
&&
ماند چون خر محتسب اندر خلاب
\\
گفت او را محتسب هین آه کن
&&
مست هوهو کرد هنگام سخن
\\
گفت گفتم آه کن هو می‌کنی
&&
گفت من شاد و تو از غم منحنی
\\
آه از درد و غم و بیدادیست
&&
هوی هوی می‌خوران از شادیست
\\
محتسب گفت این ندانم خیز خیز
&&
معرفت متراش و بگذار این ستیز
\\
گفت رو تو از کجا من از کجا
&&
گفت مستی خیز تا زندان بیا
\\
گفت مست ای محتسب بگذار و رو
&&
از برهنه کی توان بردن گرو
\\
گر مرا خود قوت رفتن بدی
&&
خانهٔ خود رفتمی وین کی شدی
\\
من اگر با عقل و با امکانمی
&&
همچو شیخان بر سر دکانمی
\\
\end{longtable}
\end{center}
