\begin{center}
\section*{بخش ۸۷ - بیان حال خودپرستان و ناشکران در نعمت وجود انبیا و اولیا علیهم السلام}
\label{sec:sh087}
\addcontentsline{toc}{section}{\nameref{sec:sh087}}
\begin{longtable}{l p{0.5cm} r}
هر که زیشان گفت از عیب و گناه
&&
وز دل چون سنگ وز جان سیاه
\\
وز سبک‌داری فرمانهای او
&&
وز فراغت از غم فردای او
\\
وز هوس وز عشق این دنیای دون
&&
چون زنان مر نفس را بودن زبون
\\
وان فرار از نکته‌های ناصحان
&&
وان رمیدن از لقای صالحان
\\
با دل و با اهل دل بیگانگی
&&
با شهان تزویر و روبه‌شانگی
\\
سیر چشمان را گدا پنداشتن
&&
از حسدشان خفیه دشمن داشتن
\\
گر پذیرد چیز تو گویی گداست
&&
ورنه گویی زرق و مکرست و دغاست
\\
گر در آمیزد تو گویی طامعست
&&
ورنی گویی در تکبر مولعست
\\
یا منافق‌وار عذر آری که من
&&
مانده‌ام در نفقهٔ فرزند و زن
\\
نه مرا پروای سر خاریدنست
&&
نه مرا پروای دین ورزیدنست
\\
ای فلان ما را بهمت یاد دار
&&
تا شویم از اولیا پایان کار
\\
این سخن نی هم ز درد و سوز گفت
&&
خوابناکی هرزه گفت و باز خفت
\\
هیچ چاره نیست از قوت عیال
&&
از بن دندان کنم کسپ حلال
\\
چه حلال ای گشته از اهل ضلال
&&
غیر خون تو نمی‌بینم حلال
\\
از خدا چاره‌ستش و از قوت نی
&&
چاره‌ش است از دین و از طاغوت نی
\\
ای که صبرت نیست از دنیای دون
&&
صبر چون داری ز نعم الماهدون
\\
ای که صبرت نیست از ناز و نعیم
&&
صبر چون داری از الله کریم
\\
ای که صبرت نیست از پاک و پلید
&&
صبر چون داری از آن کین آفرید
\\
کو خلیلی کو برون آمد ز غار
&&
گفت هذا رب هان کو کردگار
\\
من نخواهم در دو عالم بنگریست
&&
تا نبینم این دو مجلس آن کیست
\\
بی تماشای صفتهای خدا
&&
گر خورم نان در گلو ماند مرا
\\
چون گوارد لقمه بی دیدار او
&&
بی تماشای گل و گلزار او
\\
جز بر اومید خدا زین آب و خور
&&
کی خورد یک لحظه غیر گاو و خر
\\
آنک کالانعام بد بل هم اضل
&&
گرچه پر مکرست آن گنده‌بغل
\\
مکر او سرزیر و او سرزیر شد
&&
روزگارک برد و روزش دیر شد
\\
فکرگاهش کند شد عقلش خرف
&&
عمر شد چیزی ندارد چون الف
\\
آنچ می‌گوید درین اندیشه‌ام
&&
آن هم از دستان آن نفسست هم
\\
وآنچ می‌گوید غفورست و رحیم
&&
نیست آن جز حیلهٔ نفس لیم
\\
ای ز غم مرده که دست از نان تهیست
&&
چون غفورست و رحیم این ترس چیست
\\
\end{longtable}
\end{center}
