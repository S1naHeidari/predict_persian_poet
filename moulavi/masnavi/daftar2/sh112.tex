\begin{center}
\section*{بخش ۱۱۲ - منازعت چهار کس جهت انگور کی هر یکی به نام دیگر فهم کرده بود آن را}
\label{sec:sh112}
\addcontentsline{toc}{section}{\nameref{sec:sh112}}
\begin{longtable}{l p{0.5cm} r}
چار کس را داد مردی یک درم
&&
آن یکی گفت این بانگوری دهم
\\
آن یکی دیگر عرب بد گفت لا
&&
من عنب خواهم نه انگور ای دغا
\\
آن یکی ترکی بد و گفت این بنم
&&
من نمی‌خواهم عنب خواهم ازم
\\
آن یکی رومی بگفت این قیل را
&&
ترک کن خواهیم استافیل را
\\
در تنازع آن نفر جنگی شدند
&&
که ز سر نامها غافل بدند
\\
مشت بر هم می‌زدند از ابلهی
&&
پر بدند از جهل و از دانش تهی
\\
صاحب سری عزیزی صد زبان
&&
گر بدی آنجا بدادی صلحشان
\\
پس بگفتی او که من زین یک درم
&&
آرزوی جمله‌تان را می‌دهم
\\
چونک بسپارید دل را بی دغل
&&
این درمتان می‌کند چندین عمل
\\
یک درمتان می‌شود چار المراد
&&
چار دشمن می‌شود یک ز اتحاد
\\
گفت هر یکتان دهد جنگ و فراق
&&
گفت من آرد شما را اتفاق
\\
پس شما خاموش باشید انصتوا
&&
تا زبانتان من شوم در گفت و گو
\\
گر سخنتان می‌نماید یک نمط
&&
در اثر مایهٔ نزاعست و سخط
\\
گرمی عاریتی ندهد اثر
&&
گرمی خاصیتی دارد هنر
\\
سرکه را گر گرم کردی ز آتش آن
&&
چون خوری سردی فزاید بی گمان
\\
زانک آن گرمی او دهلیزیست
&&
طبع اصلش سردیست و تیزیست
\\
ور بود یخ‌بسته دوشاب ای پسر
&&
چون خوری گرمی فزاید در جگر
\\
پس ریای شیخ به ز اخلاص ماست
&&
کز بصیرت باشد آن وین از عماست
\\
از حدیث شیخ جمعیت رسد
&&
تفرقه آرد دم اهل جسد
\\
چون سلیمان کز سوی حضرت بتاخت
&&
کو زبان جمله مرغان را شناخت
\\
در زمان عدلش آهو با پلنگ
&&
انس بگرفت و برون آمد ز جنگ
\\
شد کبوتر آمن از چنگال باز
&&
گوسفند از گرگ ناورد احتراز
\\
او میانجی شد میان دشمنان
&&
اتحادی شد میان پرزنان
\\
تو چو موری بهر دانه می‌دوی
&&
هین سلیمان جو چه می‌باشی غوی
\\
دانه‌جو را دانه‌اش دامی شود
&&
و آن سلیمان‌جوی را هر دو بود
\\
مرغ جانها را درین آخر زمان
&&
نیستشان از همدگر یک دم امان
\\
هم سلیمان هست اندر دور ما
&&
کو دهد صلح و نماند جور ما
\\
قول ان من امة را یاد گیر
&&
تا به الا و خلا فیها نذیر
\\
گفت خود خالی نبودست امتی
&&
از خلیفهٔ حق و صاحب‌همتی
\\
مرغ جانها را چنان یکدل کند
&&
کز صفاشان بی غش و بی غل کند
\\
مشفقان گردند همچون والده
&&
مسلمون را گفت نفس واحده
\\
نفس واحد از رسول حق شدند
&&
ور نه هر یک دشمن مطلق بدند
\\
\end{longtable}
\end{center}
