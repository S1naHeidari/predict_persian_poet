\begin{center}
\section*{بخش ۲۱ - امتحان پادشاه به آن دو غلام کی نو خریده بود}
\label{sec:sh021}
\addcontentsline{toc}{section}{\nameref{sec:sh021}}
\begin{longtable}{l p{0.5cm} r}
پادشاهی دو غلام ارزان خرید
&&
با یکی زان دو سخن گفت و شنید
\\
یافتش زیرک‌دل و شیرین جواب
&&
از لب شکر چه زاید شکرآب
\\
آدمی مخفیست در زیر زبان
&&
این زبان پرده‌ست بر درگاه جان
\\
چونک بادی پرده را در هم کشید
&&
سر صحن خانه شد بر ما پدید
\\
کاندر آن خانه گهر یا گندمست
&&
گنج زر یا جمله مار و کزدمست
\\
یا درو گنجست و ماری بر کران
&&
زانک نبود گنج زر بی پاسبان
\\
بی تامل او سخن گفتی چنان
&&
کز پس پانصد تامل دیگران
\\
گفتیی در باطنش دریاستی
&&
جمله دریا گوهر گویاستی
\\
نور هر گوهر کزو تابان شدی
&&
حق و باطل را ازو فرقان شدی
\\
نور فرقان فرق کردی بهر ما
&&
ذره ذره حق و باطل را جدا
\\
نور گوهر نور چشم ما شدی
&&
هم سؤال و هم جواب از ما بدی
\\
چشم کژ کردی دو دیدی قرص ماه
&&
چون سؤالست این نظر در اشتباه
\\
راست گردان چشم را در ماهتاب
&&
تا یکی بینی تو مه را نک جواب
\\
فکرتت که کژ مبین نیکو نگر
&&
هست هم نور و شعاع آن گهر
\\
هر جوابی کان ز گوش آید بدل
&&
چشم گفت از من شنو آن را بهل
\\
گوش دلاله‌ست و چشم اهل وصال
&&
چشم صاحب حال و گوش اصحاب قال
\\
در شنود گوش تبدیل صفات
&&
در عیان دیده‌ها تبدیل ذات
\\
ز آتش ار علمت یقین شد از سخن
&&
پختگی جو در یقین منزل مکن
\\
تا نسوزی نیست آن عین الیقین
&&
این یقین خواهی در آتش در نشین
\\
گوش چون نافذ بود دیده شود
&&
ورنه قل در گوش پیچیده شود
\\
این سخن پایان ندارد باز گرد
&&
تا که شه با آن غلامانش چه کرد
\\
\end{longtable}
\end{center}
