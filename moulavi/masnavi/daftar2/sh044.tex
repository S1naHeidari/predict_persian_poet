\begin{center}
\section*{بخش ۴۴ - ترک کردن آن مرد ناصح بعد از مبالغهٔ پند مغرور خرس را}
\label{sec:sh044}
\addcontentsline{toc}{section}{\nameref{sec:sh044}}
\begin{longtable}{l p{0.5cm} r}
آن مسلمان ترک ابله کرد و تفت
&&
زیر لب لاحول گویان باز رفت
\\
گفت چون از جد و بندم وز جدال
&&
در دل او پیش می‌زاید خیال
\\
پس ره پند و نصیحت بسته شد
&&
امر اعرض عنهم پیوسته شد
\\
چون دوایت می‌فزاید درد پس
&&
قصه با طالب بگو بر خوان عبس
\\
چونک اعمی طالب حق آمدست
&&
بهر فقر او را نشاید سینه خست
\\
تو حریصی بر رشاد مهتران
&&
تا بیاموزند عام از سروران
\\
احمدا دیدی که قومی از ملوک
&&
مستمع گشتند گشتی خوش که بوک
\\
این رئیسان یار دین گردند خوش
&&
بر عرب اینها سرند و بر حبش
\\
بگذرد این صیت از بصره و تبوک
&&
زانک الناس علی دین الملوک
\\
زین سبب تو از ضریر مهتدی
&&
رو بگردانیدی و تنگ آمدی
\\
کندرین فرصت کم افتد این مناخ
&&
تو ز یارانی و وقت تو فراخ
\\
مزدحم می‌گردیم در وقت تنگ
&&
این نصیحت می‌کنم نه از خشم و جنگ
\\
احمدا نزد خدا این یک ضریر
&&
بهتر از صد قیصرست و صد وزیر
\\
یاد الناس معادن هین بیار
&&
معدنی باشد فزون از صد هزار
\\
معدن لعل و عقیق مکتنس
&&
بهترست از صد هزاران کان مس
\\
احمدا اینجا ندارد مال سود
&&
سینه باید پر ز عشق و درد و دود
\\
اعمیی روشن‌دل آمد در مبند
&&
پند او را ده که حق اوست پند
\\
گر دو سه ابله ترا منکر شدند
&&
تلخ کی گردی چو هستی کان قند
\\
گر دو سه ابله ترا تهمت نهد
&&
حق برای تو گواهی می‌دهد
\\
گفت از اقرار عالم فارغم
&&
آنک حق باشد گواه او را چه غم
\\
گر خفاشی را ز خورشیدی خوریست
&&
آن دلیل آمد که آن خورشید نیست
\\
نفرت خفاشکان باشد دلیل
&&
که منم خورشید تابان جلیل
\\
گر گلابی را جعل راغب شود
&&
آن دلیل ناگلابی می‌کند
\\
گر شود قلبی خریدار محک
&&
در محکی‌اش در آید نقص و شک
\\
دزد شب خواهد نه روز این را بدان
&&
شب نیم روزم که تابم در جهان
\\
فارقم فاروقم و غلبیروار
&&
تا که که از من نمی‌یابد گذار
\\
آرد را پیدا کنم من از سبوس
&&
تا نمایم کین نقوشست آن نفوس
\\
من چو میزان خدایم در جهان
&&
وا نمایم هر سبک را از گران
\\
گاو را داند خدا گوساله‌ای
&&
خر خریداری و در خور کاله‌ای
\\
من نه گاوم تا که گوسالم خرد
&&
من نه خارم که اشتری از من چرد
\\
او گمان دارد که با من جور کرد
&&
بلک از آیینهٔ من روفت گرد
\\
\end{longtable}
\end{center}
