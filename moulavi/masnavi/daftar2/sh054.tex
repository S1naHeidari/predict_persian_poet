\begin{center}
\section*{بخش ۵۴ - دانستن پیغامبر علیه السلام کی سبب رنجوری آن شخص گستاخی بوده است در دعا}
\label{sec:sh054}
\addcontentsline{toc}{section}{\nameref{sec:sh054}}
\begin{longtable}{l p{0.5cm} r}
چون پیمبر دید آن بیمار را
&&
خوش نوازش کرد یار غار را
\\
زنده شد او چون پیمبر را بدید
&&
گوییا آن دم مر او را آفرید
\\
گفت بیماری مرا این بخت داد
&&
کآمد این سلطان بر من بامداد
\\
تا مرا صحت رسید و عافیت
&&
از قدوم این شه بی حاشیت
\\
ای خجسته رنج و بیماری و تب
&&
ای مبارک درد و بیداری شب
\\
نک مرا در پیری از لطف و کرم
&&
حق چنین رنجوریی داد و سقم
\\
درد پشتم داد هم تا من ز خواب
&&
بر جهم هر نیمشب لا بد شتاب
\\
تا نخسپم جمله شب چون گاومیش
&&
دردها بخشید حق از لطف خویش
\\
زین شکست آن رحم شاهان جوش کرد
&&
دوزخ از تهدید من خاموش کرد
\\
رنج گنج آمد که رحمتها دروست
&&
مغز تازه شد چو بخراشید پوست
\\
ای برادر موضع تاریک و سرد
&&
صبر کردن بر غم و سستی و درد
\\
چشمهٔ حیوان و جام مستی است
&&
کان بلندیها همه در پستی است
\\
آن بهاران مضمرست اندر خزان
&&
در بهارست آن خزان مگریز از آن
\\
همره غم باش و با وحشت بساز
&&
می‌طلب در مرگ خود عمر دراز
\\
آنچ گوید نفس تو کاینجا بدست
&&
مشنوش چون کار او ضد آمدست
\\
تو خلافش کن که از پیغامبران
&&
این چنین آمد وصیت در جهان
\\
مشورت در کارها واجب شود
&&
تا پشیمانی در آخر کم بود
\\
حیله‌ها کردند بسیار انبیا
&&
تا که گردان شد برین سنگ آسیا
\\
نفس می‌خواهد که تا ویران کند
&&
خلق را گمراه و سرگردان کند
\\
گفت امت مشورت با کی کنیم
&&
انبیا گفتند با عقل امام
\\
گفت گر کودک در آید یا زنی
&&
کو ندارد عقل و رای روشنی
\\
گفت با او مشورت کن وانچ گفت
&&
تو خلاف آن کن و در راه افت
\\
نفس خود را زن شناس از زن بتر
&&
زانک زن جزویست نفست کل شر
\\
مشورت با نفس خود گر می‌کنی
&&
هرچه گوید کن خلاف آن دنی
\\
گر نماز و روزه می‌فرمایدت
&&
نفس مکارست مکری زایدت
\\
مشورت با نفس خویش اندر فعال
&&
هرچه گوید عکس آن باشد کمال
\\
برنیایی با وی و استیز او
&&
رو بر یاری بگیر آمیز او
\\
عقل قوت گیرد از عقل دگر
&&
نیشکر کامل شود از نیشکر
\\
من ز مکر نفس دیدم چیزها
&&
کو برد از سحر خود تمییزها
\\
وعده‌ها بدهد ترا تازه به دست
&&
که هزاران بار آنها را شکست
\\
عمر اگر صد سال خود مهلت دهد
&&
اوت هر روزی بهانهٔ نو نهد
\\
گرم گوید وعده‌های سرد را
&&
جادوی مردی ببندد مرد را
\\
ای ضیاء الحق حسام الدین بیا
&&
که نروید بی تو از شوره گیا
\\
از فلک آویخته شد پرده‌ای
&&
از پی نفرین دل آزرده‌ای
\\
این قضا را هم قضا داند علاج
&&
عقل خلقان در قضا گیجست گیج
\\
اژدها گشتست آن مار سیاه
&&
آنک کرمی بود افتاده به راه
\\
اژدها و مار اندر دست تو
&&
شد عصا ای جان موسی مست تو
\\
حکم خذها لا تخف دادت خدا
&&
تا به دستت اژدها گردد عصا
\\
هین ید بیضا نما ای پادشاه
&&
صبح نو بگشا ز شبهای سیاه
\\
دوزخی افروخت بر وی دم فسون
&&
ای دم تو از دم دریا فزون
\\
بحر مکارست بنموده کفی
&&
دوزخست از مکر بنموده تفی
\\
زان نماید مختصر در چشم تو
&&
تا زبون بینیش جنبد خشم تو
\\
همچنانک لشکر انبوه بود
&&
مر پیمبر را به چشم اندک نمود
\\
تا بریشان زد پیمبر بی خطر
&&
ور فزون دیدی از آن کردی حذر
\\
آن عنایت بود و اهل آن بدی
&&
احمدا ورنه تو بد دل می‌شدی
\\
کم نمود او را و اصحاب ورا
&&
آن جهاد ظاهر و باطن خدا
\\
تا میسر کرد یسری را برو
&&
تا ز عسری او بگردانید رو
\\
کم نمودن مر ورا پیروز بود
&&
که حقش یار و طریق‌آموز بود
\\
آنک حق پشتش نباشد از ظفر
&&
وای اگر گربه‌ش نماید شیر نر
\\
وای اگر صد را یکی بیند ز دور
&&
تا به چالش اندر آید از غرور
\\
زان نماید ذوالفقاری حربه‌ای
&&
زان نماید شیر نر چون گربه‌ای
\\
تا دلیر اندر فتد احمق به جنگ
&&
واندر آردشان بدین حیلت به چنگ
\\
تا به پای خویش باشند آمده
&&
آن فلیوان جانب آتشکده
\\
کاه برگی می‌نماید تا تو زود
&&
پف کنی کو را برانی از وجود
\\
هین که آن که کوهها بر کنده است
&&
زو جهان گریان و او در خنده است
\\
می‌نماید تا بکعب این آب جو
&&
صد چو عاج ابن عنق شد غرق او
\\
می‌نماید موج خونش تل مشک
&&
می‌نماید قعر دریا خاک خشک
\\
خشک دید آن بحر را فرعون کور
&&
تا درو راند از سر مردی و زور
\\
چون در آید در تک دریا بود
&&
دیدهٔ فرعون کی بینا بود
\\
دیده بینا از لقای حق شود
&&
حق کجا همراز هر احمق شود
\\
قند بیند خود شود زهر قتول
&&
راه بیند خود بود آن بانگ غول
\\
ای فلک در فتنهٔ آخر زمان
&&
تیز می‌گردی بده آخر زمان
\\
خنجر تیزی تو اندر قصد ما
&&
نیش زهرآلوده‌ای در فصد ما
\\
ای فلک از رحم حق آموز رحم
&&
بر دل موران مزن چون مار زخم
\\
حق آنک چرخهٔ چرخ ترا
&&
کرد گردان بر فراز این سرا
\\
که دگرگون گردی و رحمت کنی
&&
پیش از آن که بیخ ما را بر کنی
\\
حق آنک دایگی کردی نخست
&&
تا نهال ما ز آب و خاک رست
\\
حق آن شه که ترا صاف آفرید
&&
کرد چندان مشعله در تو پدید
\\
آنچنان معمور و باقی داشتت
&&
تا که دهری از ازل پنداشتت
\\
شکر دانستیم آغاز ترا
&&
انبیا گفتند آن راز ترا
\\
آدمی داند که خانه حادثست
&&
عنکبوتی نه که در وی عابشست
\\
پشه کی داند که این باغ از کیست
&&
کو بهاران زاد و مرگش در دیست
\\
کرم کاندر چوب زاید سست‌حال
&&
کی بداند چوب را وقت نهال
\\
ور بداند کرم از ماهیتش
&&
عقل باشد کرم باشد صورتش
\\
عقل خود را می‌نماید رنگها
&&
چون پری دورست از آن فرسنگها
\\
از ملک بالاست چه جای پری
&&
تو مگس‌پری بپستی می‌پری
\\
گرچه عقلت سوی بالا می‌پرد
&&
مرغ تقلیدت بپستی می‌چرد
\\
علم تقلیدی وبال جان ماست
&&
عاریه‌ست و ما نشسته کان ماست
\\
زین خرد جاهل همی باید شدن
&&
دست در دیوانگی باید زدن
\\
هرچه بینی سود خود زان می‌گریز
&&
زهر نوش و آب حیوان را بریز
\\
هر که بستاید ترا دشنام ده
&&
سود و سرمایه به مفلس وام ده
\\
ایمنی بگذار و جای خوف باش
&&
بگذر از ناموس و رسوا باش و فاش
\\
آزمودم عقل دور اندیش را
&&
بعد ازین دیوانه سازم خویش را
\\
\end{longtable}
\end{center}
