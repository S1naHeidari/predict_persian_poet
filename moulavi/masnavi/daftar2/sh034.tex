\begin{center}
\section*{بخش ۳۴ - انکار فلسفی بر قرائت ان اصبح ماکم غورا}
\label{sec:sh034}
\addcontentsline{toc}{section}{\nameref{sec:sh034}}
\begin{longtable}{l p{0.5cm} r}
مقریی می‌خواند از روی کتاب
&&
ماؤکم غورا ز چشمه بندم آب
\\
آب را در غورها پنهان کنم
&&
چشمه‌ها را خشک و خشکستان کنم
\\
آب را در چشمه کی آرد دگر
&&
جز من بی مثل و با فضل و خطر
\\
فلسفی منطقی مستهان
&&
می‌گذشت از سوی مکتب آن زمان
\\
چونک بشنید آیت او از ناپسند
&&
گفت آریم آب را ما با کلند
\\
ما به زخم بیل و تیزی تبر
&&
آب را آریم از پستی زبر
\\
شب بخفت و دید او یک شیرمرد
&&
زد طبانچه هر دو چشمش کور کرد
\\
گفت زین دو چشمهٔ چشم ای شقی
&&
با تبر نوری بر آر ار صادقی
\\
روز بر جست و دو چشم کور دید
&&
نور فایض از دو چشمش ناپدید
\\
گر بنالیدی و مستغفر شدی
&&
نور رفته از کرم ظاهر شدی
\\
لیک استغفار هم در دست نیست
&&
ذوق توبه نقل هر سرمست نیست
\\
زشتی اعمال و شومی جحود
&&
راه توبه بر دل او بسته بود
\\
از نیاز و اعتقاد آن خلیل
&&
گشت ممکن امر صعب و مستحیل
\\
همچنین بر عکس آن انکار مرد
&&
مس کند زر را و صلحی را نبرد
\\
دل بسختی همچو روی سنگ گشت
&&
چون شکافد توبه آن را بهر کشت
\\
چون شعیبی کو که تا او از دعا
&&
بهر کشتن خاک سازد کوه را
\\
یا بدریوزه مقوقس از رسول
&&
سنگ‌لاخی مزرعی شد با اصول
\\
کهربای مسخ آمد این دغا
&&
خاک قابل را کند سنگ و حصا
\\
هر دلی را سجده هم دستور نیست
&&
مزد رحمت قسم هر مزدور نیست
\\
هین به پشت آن مکن جرم و گناه
&&
که کنم توبه در آیم در پناه
\\
می‌بباید تاب و آبی توبه را
&&
شرط شد برق و سحابی توبه را
\\
آتش و آبی بباید میوه را
&&
واجب آید ابر و برق این شیوه را
\\
تا نباشد برق دل و ابر دو چشم
&&
کی نشیند آتش تهدید و خشم
\\
کی بروید سبزهٔ ذوق وصال
&&
کی بجوشد چشمه‌ها ز آب زلال
\\
کی گلستان راز گوید با چمن
&&
کی بنفشه عهد بندد با سمن
\\
کی چناری کف گشاید در دعا
&&
کی درختی سر فشاند در هوا
\\
کی شکوفه آستین پر نثار
&&
بر فشاندن گیرد ایام بهار
\\
کی فروزد لاله را رخ همچو خون
&&
کی گل از کیسه بر آرد زر برون
\\
کی بیاید بلبل و گل بو کند
&&
کی چو طالب فاخته کوکو کند
\\
کی بگوید لک‌لک آن لک‌لک بجان
&&
لک چه باشد ملک تست ای مستعان
\\
کی نماید خاک اسرار ضمیر
&&
کی شود بی آسمان بستان منیر
\\
از کجا آورده‌اند آن حله‌ها
&&
من کریم من رحیم کلها
\\
آن لطافتها نشان شاهدیست
&&
آن نشان پای مرد عابدیست
\\
آن شود شاد از نشان کو دید شاه
&&
چون ندید او را نباشد انتباه
\\
روح آنکس کو بهنگام الست
&&
دید رب خویش و شد بی‌خویش مست
\\
او شناسد بوی می کو می بخورد
&&
چون نخورد او می چه داند بوی کرد
\\
زانک حکمت همچو ناقهٔ ضاله است
&&
همچو دلاله شهان را داله است
\\
تو ببینی خواب در یک خوش‌لقا
&&
کو دهد وعده و نشانی مر ترا
\\
که مراد تو شود و اینک نشان
&&
که به پیش آید ترا فردا فلان
\\
یک نشانی آن که او باشد سوار
&&
یک نشانی که ترا گیرد کنار
\\
یک نشانی که بخندد پیش تو
&&
یک نشان که دست بندد پیش تو
\\
یک نشانی آنک این خواب از هوس
&&
چون شود فردا نگویی پیش کس
\\
زان نشان هم زکریا را بگفت
&&
که نیایی تا سه روز اصلا بگفت
\\
تا سه شب خامش کن از نیک و بدت
&&
این نشان باشد که یحی آیدت
\\
دم مزن سه روز اندر گفت و گو
&&
کین سکوتست آیت مقصود تو
\\
هین میاور این نشان را تو بگفت
&&
وین سخن را دار اندر دل نهفت
\\
این نشانها گویدش همچون شکر
&&
این چه باشد صد نشانی دگر
\\
این نشان آن بود کان ملک و جاه
&&
که همی‌جویی بیابی از اله
\\
آنک می‌گریی بشبهای دراز
&&
وانک می‌سوزی سحرگه در نیاز
\\
آنک بی آن روز تو تاریک شد
&&
همچو دوکی گردنت باریک شد
\\
وآنچ دادی هرچه داری در زکات
&&
چون زکات پاک‌بازان رختهات
\\
رختها دادی و خواب و رنگ رو
&&
سر فدا کردی و گشتی همچو مو
\\
چند در آتش نشستی همچو عود
&&
چند پیش تیغ رفتی همچو خود
\\
زین چنین بیچارگیها صد هزار
&&
خوی عشاقست و ناید در شمار
\\
چونک شب این خواب دیدی روز شد
&&
از امیدش روز تو پیروز شد
\\
چشم گردان کرده‌ای بر چپ و راست
&&
کان نشان و آن علامتها کجاست
\\
بر مثال برگ می‌لرزی که وای
&&
گر رود روز و نشان ناید بجای
\\
می‌دوی در کوی و بازار و سرا
&&
چون کسی کو گم کند گوساله را
\\
خواجه خیرست این دوادو چیستت
&&
گم شده اینجا که داری کیستت
\\
گوییش خیرست لیکن خیر من
&&
کس نشاید که بداند غیر من
\\
گر بگویم نک نشانم فوت شد
&&
چون نشان شد فوت وقت موت شد
\\
بنگری در روی هر مرد سوار
&&
گویدت منگر مرا دیوانه‌وار
\\
گوییش من صاحبی گم کرده‌ام
&&
رو به جست و جوی او آورده‌ام
\\
دولتت پاینده بادا ای سوار
&&
رحم کن بر عاشقان معذور دار
\\
چون طلب کردی بجد آمد نظر
&&
جد خطا نکند چنین آمد خبر
\\
ناگهان آمد سواری نیکبخت
&&
پس گرفت اندر کنارت سخت سخت
\\
تو شدی بیهوش و افتادی بطاق
&&
بی‌خبر گفت اینت سالوس و نفاق
\\
او چه می‌بیند درو این شور چیست
&&
او نداند کان نشان وصل کیست
\\
این نشان در حق او باشد که دید
&&
آن دگر را کی نشان آید پدید
\\
هر زمان کز وی نشانی می‌رسید
&&
شخص را جانی بجانی می‌رسید
\\
ماهی بیچاره را پیش آمد آب
&&
این نشانها تلک آیات الکتاب
\\
پس نشانیها که اندر انبیاست
&&
خاص آن جان را بود کو آشناست
\\
این سخن ناقص بماند و بی‌قرار
&&
دل ندارم بی‌دلم معذور دار
\\
ذره‌ها را کی تواند کس شمرد
&&
خاصه آن کو عشق از وی عقل برد
\\
می‌شمارم برگهای باغ را
&&
می‌شمارم بانگ کبک و زاغ را
\\
در شمار اندر نیاید لیک من
&&
می‌شمارم بهر رشد ممتحن
\\
نحس کیوان یا که سعد مشتری
&&
ناید اندر حصر گرچه بشمری
\\
لیک هم بعضی ازین هر دو اثر
&&
شرح باید کرد یعنی نفع و ضر
\\
تا شود معلوم آثار قضا
&&
شمه‌ای مر اهل سعد و نحس را
\\
طالع آنکس که باشد مشتری
&&
شاد گردد از نشاط و سروری
\\
وانک را طالع زحل از هر شرور
&&
احتیاطش لازم آید در امور
\\
اذکروا الله شاه ما دستور داد
&&
اندر آتش دید ما را نور داد
\\
گفت اگرچه پاکم از ذکر شما
&&
نیست لایق مر مرا تصویرها
\\
لیک هرگز مست تصویر و خیال
&&
در نیابد ذات ما را بی مثال
\\
ذکر جسمانه خیال ناقصست
&&
وصف شاهانه از آنها خالصست
\\
شاه را گوید کسی جولاه نیست
&&
این چه مدحست این مگر آگاه نیست
\\
\end{longtable}
\end{center}
