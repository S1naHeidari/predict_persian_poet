\begin{center}
\section*{بخش ۱۴۷ - حکایت مندیل در تنور پر آتش انداختن  انس رضی الله عنه و ناسوختن}
\label{sec:sh147}
\addcontentsline{toc}{section}{\nameref{sec:sh147}}
\begin{longtable}{l p{0.5cm} r}
از انس فرزند مالک آمدست
&&
که به مهمانی او شخصی شدست
\\
او حکایت کرد کز بعد طعام
&&
دید انس دستارخوان را زردفام
\\
چرکن و آلوده گفت ای خادمه
&&
اندر افکن در تنورش یک‌دمه
\\
در تنور پر ز آتش در فکند
&&
آن زمان دستارخوان را هوشمند
\\
جمله مهمانان در آن حیران شدند
&&
انتظار دود کندوری بدند
\\
بعد یکساعت بر آورد از تنور
&&
پاک و اسپید و از آن اوساخ دور
\\
قوم گفتند ای صحابی عزیز
&&
چون نسوزید و منقی گشت نیز
\\
گفت زانک مصطفی دست و دهان
&&
بس بمالید اندرین دستارخوان
\\
ای دل ترسنده از نار و عذاب
&&
با چنان دست و لبی کن اقتراب
\\
چون جمادی را چنین تشریف داد
&&
جان عاشق را چه‌ها خواهد گشاد
\\
مر کلوخ کعبه را چون قبله کرد
&&
خاک مردان باش ای جان در نبرد
\\
بعد از آن گفتند با آن خادمه
&&
تو نگویی حال خود با این همه
\\
چون فکندی زود آن از گفت وی
&&
گیرم او بردست در اسرار پی
\\
این‌چنین دستارخوان قیمتی
&&
چون فکندی اندر آتش ای ستی
\\
گفت دارم بر کریمان اعتماد
&&
نیستم ز اکرام ایشان ناامید
\\
میزری چه بود اگر او گویدم
&&
در رو اندر عین آتش بی ندم
\\
اندر افتم از کمال اعتماد
&&
از عباد الله دارم بس امید
\\
سر در اندازم نه این دستارخوان
&&
ز اعتماد هر کریم رازدان
\\
ای برادر خود برین اکسیر زن
&&
کم نباید صدق مرد از صدق زن
\\
آن دل مردی که از زن کم بود
&&
آن دلی باشد که کم ز اشکم بود
\\
\end{longtable}
\end{center}
