\begin{center}
\section*{بخش ۱۹۶ - مکرر کردن عاذلان پند را بر آن مهمان آن مسجد مهمان کش}
\label{sec:sh196}
\addcontentsline{toc}{section}{\nameref{sec:sh196}}
\begin{longtable}{l p{0.5cm} r}
گفت پیغامبر که ان فی البیان
&&
سحرا و حق گفت آن خوش پهلوان
\\
هین مکن جلدی برو ای بوالکرم
&&
مسجد و ما را مکن زین متهم
\\
که بگوید دشمنی از دشمنی
&&
آتشی در ما زند فردا دنی
\\
که بتاسانید او را ظالمی
&&
بر بهانهٔ مسجد او بد سالمی
\\
تا بهانهٔ قتل بر مسجد نهد
&&
چونک بدنامست مسجد او جهد
\\
تهمتی بر ما منه ای سخت‌جان
&&
که نه‌ایم آمن ز مکر دشمنان
\\
هین برو جلدی مکن سودا مپز
&&
که نتان پیمود کیوان را بگز
\\
چون تو بسیاران بلافیده ز بخت
&&
ریش خود بر کنده یک یک لخت لخت
\\
هین برو کوتاه کن این قیل و قال
&&
خویش و ما را در میفکن در وبال
\\
\end{longtable}
\end{center}
