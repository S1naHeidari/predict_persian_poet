\begin{center}
\section*{بخش ۲۸ - حکایت}
\label{sec:sh028}
\addcontentsline{toc}{section}{\nameref{sec:sh028}}
\begin{longtable}{l p{0.5cm} r}
همچنان کاینجا مغول حیله‌دان
&&
گفت می‌جویم کسی از مصریان
\\
مصریان را جمع آرید این طرف
&&
تا در آید آنک می‌باید بکف
\\
هر که می‌آمد بگفتا نیست این
&&
هین در آ خواجه در آن گوشه نشین
\\
تا بدین شیوه همه جمع آمدند
&&
گردن ایشان بدین حیلت زدند
\\
شومی آنک سوی بانگ نماز
&&
داعی الله را نبردندی نیاز
\\
دعوت مکارشان اندر کشید
&&
الحذر از مکر شیطان ای رشید
\\
بانگ درویشان و محتاجان بنوش
&&
تا نگیرد بانگ محتالیت گوش
\\
گر گدایان طامع‌اند و زشت‌خو
&&
در شکم‌خواران تو صاحب‌دل بجو
\\
در تگ دریا گهر با سنگهاست
&&
فخرها اندر میان ننگهاست
\\
پس بجوشیدند اسرائیلیان
&&
از پگه تا جانب میدان دوان
\\
چون بحیلتشان به میدان برد او
&&
روی خود ننمودشان بس تازه‌رو
\\
کرد دلداری و بخششها بداد
&&
هم عطا هم وعده‌ها کرد آن قباد
\\
بعد از آن گفت از برای جانتان
&&
جمله در میدان بخسپید امشبان
\\
پاسخش دادند که خدمت کنیم
&&
گر تو خواهی یک مه اینجا ساکنیم
\\
\end{longtable}
\end{center}
