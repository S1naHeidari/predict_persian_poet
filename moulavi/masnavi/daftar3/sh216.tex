\begin{center}
\section*{بخش ۲۱۶ - نظرکردن پیغامبر علیه السلام به اسیران و تبسم کردن و گفتن کی عجبت من قوم یجرون الی الجنة بالسلاسل و الاغلال}
\label{sec:sh216}
\addcontentsline{toc}{section}{\nameref{sec:sh216}}
\begin{longtable}{l p{0.5cm} r}
دید پیغامبر یکی جوقی اسیر
&&
که همی‌بردند و ایشان در نفیر
\\
دیدشان در بند آن آگاه شیر
&&
می نظر کردند در وی زیر زیر
\\
تا همی خایید هر یک از غضب
&&
بر رسول صدق دندانها و لب
\\
زهره نه با آن غضب که دم زنند
&&
زانک در زنجیر قهر ده‌منند
\\
می‌کشاندشان موکل سوی شهر
&&
می‌برد از کافرستانشان به قهر
\\
نه فدایی می‌ستاند نه زری
&&
نه شفاعت می‌رسد از سروری
\\
رحمت عالم همی‌گویند و او
&&
عالمی را می‌برد حلق و گلو
\\
با هزار انکار می‌رفتند راه
&&
زیر لب طعنه‌زنان بر کار شاه
\\
چاره‌ها کردیم و اینجا چاره نیست
&&
خود دل این مرد کم از خاره نیست
\\
ما هزاران مرد شیر الپ ارسلان
&&
با دو سه عریان سست نیم‌جان
\\
این چنین درمانده‌ایم از کژرویست
&&
یا ز اخترهاست یا خود جادویست
\\
بخت ما را بر درید آن بخت او
&&
تخت ما شد سرنگون از تخت او
\\
کار او از جادوی گر گشت زفت
&&
جادوی کردیم ما هم چون نرفت
\\
\end{longtable}
\end{center}
