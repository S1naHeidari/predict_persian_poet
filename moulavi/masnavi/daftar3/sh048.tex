\begin{center}
\section*{بخش ۴۸ - جمع آمدن ساحران از مداین پیش فرعون و تشریفها یافتن و دست بر سینه زدن در قهر  خصم او کی این بر ما نویس}
\label{sec:sh048}
\addcontentsline{toc}{section}{\nameref{sec:sh048}}
\begin{longtable}{l p{0.5cm} r}
تا بفرعون آمدند آن ساحران
&&
دادشان تشریفهای بس گران
\\
وعده‌هاشان کرد و پیشین هم بداد
&&
بندگان و اسپان و نقد و جنس و زاد
\\
بعد از آن می‌گفت هین ای سابقان
&&
گر فزون آیید اندر امتحان
\\
برفشانم بر شما چندان عطا
&&
که بدرد پردهٔ جود و سخا
\\
پس بگفتندش به اقبال تو شاه
&&
غالب آییم و شود کارش تباه
\\
ما درین فن صفدریم و پهلوان
&&
کس ندارد پای ما اندر جهان
\\
ذکر موسی بند خاطرها شدست
&&
کین حکایتهاست که پیشین بدست
\\
ذکر موسی بهر روپوشست لیک
&&
نور موسی نقد تست ای مرد نیک
\\
موسی و فرعون در هستی تست
&&
باید این دو خصم را در خویش جست
\\
تا قیامت هست از موسی نتاج
&&
نور دیگر نیست دیگر شد سراج
\\
این سفال و این پلیته دیگرست
&&
لیک نورش نیست دیگر زان سرست
\\
گر نظر در شیشه داری گم شوی
&&
زانک از شیشه‌ست اعداد دوی
\\
ور نظر بر نور داری وا رهی
&&
از دوی واعداد جسم منتهی
\\
از نظرگاهست ای مغز وجود
&&
اختلاف مؤمن و گبر و جهود
\\
\end{longtable}
\end{center}
