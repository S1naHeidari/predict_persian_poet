\begin{center}
\section*{بخش ۲۱۵ - فسخ عزایم و نقضها جهت با خبر کردن آدمی را از آنک مالک و قاهر اوست و گاه گاه عزم او را فسخ ناکردن و نافذ  داشتن تا طمع او را بر عزم کردن  دارد تا باز عزمش را بشکند  تا تنبیه بر تنبیه بود}
\label{sec:sh215}
\addcontentsline{toc}{section}{\nameref{sec:sh215}}
\begin{longtable}{l p{0.5cm} r}
عزمها و قصدها در ماجرا
&&
گاه گاهی راست می‌آید ترا
\\
تا به طمع آن دلت نیت کند
&&
بار دیگر نیتت را بشکند
\\
ور بکلی بی‌مرادت داشتی
&&
دل شدی نومید امل کی کاشتی
\\
ور بکاریدی امل از عوریش
&&
کی شدی پیدا برو مقهوریش
\\
عاشقان از بی‌مرادیهای خویش
&&
باخبر گشتند از مولای خویش
\\
بی‌مرادی شد قلاوز بهشت
&&
حفت الجنه شنو ای خوش سرشت
\\
که مراداتت همه اشکسته‌پاست
&&
پس کسی باشد که کام او رواست
\\
پس شدند اشکسته‌اش آن صادقان
&&
لیک کو خود آن شکست عاشقان
\\
عاقلان اشکسته‌اش از اضطرار
&&
عاشقان اشکسته با صد اختیار
\\
عاقلانش بندگان بندی‌اند
&&
عاشقانش شکری و قندی‌اند
\\
ائتیا کرها مهار عاقلان
&&
ائتیا طوعا بهار بی‌دلان
\\
\end{longtable}
\end{center}
