\begin{center}
\section*{بخش ۵۳ - داستان مشغول شدن عاشقی به عشق‌نامه خواندن  و مطالعه کردن عشق‌نامه درحضور معشوق خویش و معشوق آن را ناپسند داشتن کی طلب الدلیل عند حضور المدلول  قبیح والاشتغال بالعلم بعد الوصول الی المعلوم مذموم}
\label{sec:sh053}
\addcontentsline{toc}{section}{\nameref{sec:sh053}}
\begin{longtable}{l p{0.5cm} r}
آن یکی را یار پیش خود نشاند
&&
نامه بیرون کرد و پیش یار خواند
\\
بیتها در نامه و مدح و ثنا
&&
زاری و مسکینی و بس لابه‌ها
\\
گفت معشوق این اگر بهر منست
&&
گاه وصل این عمر ضایع کردنست
\\
من به پیشت حاضر و تو نامه خوان
&&
نیست این باری نشان عاشقان
\\
گفت اینجا حاضری اما ولیک
&&
من نمی‌یایم نصیب خویش نیک
\\
آنچ می‌دیدم ز تو پارینه سال
&&
نیست این دم گرچه می‌بینم وصال
\\
من ازین چشمه زلالی خورده‌ام
&&
دیده و دل ز آب تازه کرده‌ام
\\
چشمه می‌بینم ولیکن آب نی
&&
راه آبم را مگر زد ره‌زنی
\\
گفت پس من نیستم معشوق تو
&&
من به بلغار و مرادت در قتو
\\
عاشقی تو بر من و بر حالتی
&&
حالت اندر دست نبود یا فتی
\\
پس نیم کلی مطلوب تو من
&&
جزو مقصودم ترا اندرز من
\\
خانهٔ معشوقه‌ام معشوق نی
&&
عشق بر نقدست بر صندوق نی
\\
هست معشوق آنک او یکتو بود
&&
مبتدا و منتهاات او بود
\\
چون بیابی‌اش نمانی منتظر
&&
هم هویدا او بود هم نیز سر
\\
میر احوالست نه موقوف حال
&&
بندهٔ آن ماه باشد ماه و سال
\\
چون بگوید حال را فرمان کند
&&
چون بخواهد جسمها را جان کند
\\
منتها نبود که موقوفست او
&&
منتظر بنشسته باشد حال‌جو
\\
کیمیای حال باشد دست او
&&
دست جنباند شود مس مست او
\\
گر بخواهد مرگ هم شیرین شود
&&
خار و نشتر نرگس و نسرین شود
\\
آنک او موقوف حالست آدمیست
&&
کو بحال افزون و گاهی در کمیست
\\
صوفی ابن الوقت باشد در منال
&&
لیک صافی فارغست از وقت و حال
\\
حالها موقوف عزم و رای او
&&
زنده از نفخ مسیح‌آسای او
\\
عاشق حالی نه عاشق بر منی
&&
بر امید حال بر من می‌تنی
\\
آنک یک دم کم دمی کامل بود
&&
نیست معبود خلیل آفل بود
\\
وانک آفل باشد و گه آن و این
&&
نیست دلبر لا احب افلین
\\
آنک او گاهی خوش و گه ناخوشست
&&
یک زمانی آب و یک دم آتشست
\\
برج مه باشد ولیکن ماه نه
&&
نقش بت باشد ولی آگاه نه
\\
هست صوفی صفاجو ابن وقت
&&
وقت را همچون پدر بگرفته سخت
\\
هست صافی غرق عشق ذوالجلال
&&
ابن کس نه فارغ از اوقات و حال
\\
غرقهٔ نوری که او لم یولدست
&&
لم یلد لم یولد آن ایزدست
\\
رو چنین عشقی بجو گر زنده‌ای
&&
ورنه وقت مختلف را بنده‌ای
\\
منگر اندر نقش زشت و خوب خویش
&&
بنگر اندر عشق و در مطلوب خویش
\\
منگر آنک تو حقیری یا ضعیف
&&
بنگر اندر همت خود ای شریف
\\
تو به هر حالی که باشی می‌طلب
&&
آب می‌جو دایما ای خشک‌لب
\\
کان لب خشکت گواهی می‌دهد
&&
کو بخر بر سر منبع رسد
\\
خشکی لب هست پیغامی ز آب
&&
که بمات آرد یقین این اضطراب
\\
کین طلب‌کاری مبارک جنبشیست
&&
این طلب در راه حق مانع کشیست
\\
این طلب مفتاح مطلوبات تست
&&
این سپاه و نصرت رایات تست
\\
این طلب همچون خروسی در صیاح
&&
می‌زند نعره که می‌آید صباح
\\
گرچه آلت نیستت تو می‌طلب
&&
نیست آلت حاجت اندر راه رب
\\
هر که را بینی طلب‌کار ای پسر
&&
یار او شو پیش او انداز سر
\\
کز جوار طالبان طالب شوی
&&
وز ظلال غالبان غالب شوی
\\
گر یکی موری سلیمانی بجست
&&
منگر اندر جستن او سست سست
\\
هرچه داری تو ز مال و پیشه‌ای
&&
نه طلب بود اول و اندیشه‌ای
\\
\end{longtable}
\end{center}
