\begin{center}
\section*{بخش ۱ - سر آغاز}
\label{sec:sh001}
\addcontentsline{toc}{section}{\nameref{sec:sh001}}
\begin{longtable}{l p{0.5cm} r}
ای ضیاء الحق حسام الدین بیار
&&
این سوم دفتر که سنت شد سه بار
\\
بر گشا گنجینهٔ اسرار را
&&
در سوم دفتر بهل اعذار را
\\
قوتت از قوت حق می‌زهد
&&
نه از عروقی کز حرارت می‌جهد
\\
این چراغ شمس کو روشن بود
&&
نه از فتیل و پنبه و روغن بود
\\
سقف گردون کو چنین دایم بود
&&
نه از طناب و استنی قایم بود
\\
قوت جبریل از مطبخ نبود
&&
بود از دیدار خلاق وجود
\\
همچنان این قوت ابدال حق
&&
هم ز حق دان نه از طعام و از طبق
\\
جسمشان را هم ز نور اسرشته‌اند
&&
تا ز روح و از ملک بگذشته‌اند
\\
چونک موصوفی باوصاف جلیل
&&
ز آتش امراض بگذر چون خلیل
\\
گردد آتش بر تو هم برد و سلام
&&
ای عناصر مر مزاجت را غلام
\\
هر مزاجی را عناصر مایه‌است
&&
وین مزاجت برتر از هر پایه است
\\
این مزاجت از جهان منبسط
&&
وصف وحدت را کنون شد ملتقط
\\
ای دریغا عرصهٔ افهام خلق
&&
سخت تنگ آمد ندارد خلق حلق
\\
ای ضیاء الحق بحذق رای تو
&&
حلق بخشد سنگ را حلوای تو
\\
کوه طور اندر تجلی حلق یافت
&&
تا که می نوشید و می را بر نتافت
\\
صار دکا منه وانشق الجبل
&&
هل رایتم من جبل رقص الجمل
\\
لقمه‌بخشی آید از هر کس به کس
&&
حلق‌بخشی کار یزدانست و بس
\\
حلق بخشد جسم را و روح را
&&
حلق بخشد بهر هر عضوت جدا
\\
این گهی بخشد که اجلالی شوی
&&
وز دغا و از دغل خالی شوی
\\
تا نگویی سر سلطان را به کس
&&
تا نریزی قند را پیش مگس
\\
گوش آنکس نوشد اسرار جلال
&&
کو چو سوسن صدزبان افتاد و لال
\\
حلق بخشد خاک را لطف خدا
&&
تا خورد آب و بروید صد گیا
\\
باز خاکی را ببخشد حلق و لب
&&
تا گیاهش را خورد اندر طلب
\\
چون گیاهش خورد حیوان گشت زفت
&&
گشت حیوان لقمهٔ انسان و رفت
\\
باز خاک آمد شد اکال بشر
&&
چون جدا شد از بشر روح و بصر
\\
ذره‌ها دیدم دهانشان جمله باز
&&
گر بگویم خوردشان گردد دراز
\\
برگها را برگ از انعام او
&&
دایگان را دایه لطف عام او
\\
رزقها را رزقها او می‌دهد
&&
زانک گندم بی غذایی چون زهد
\\
نیست شرح این سخن را منتهی
&&
پاره‌ای گفتم بدانی پاره‌ها
\\
جمله عالم آکل و ماکول دان
&&
باقیان را مقبل و مقبول دان
\\
این جهان و ساکنانش منتشر
&&
وان جهان و سالکانش مستمر
\\
این جهان و عاشقانش منقطع
&&
اهل آن عالم مخلد مجتمع
\\
پس کریم آنست کو خود را دهد
&&
آب حیوانی که ماند تا ابد
\\
باقیات الصالحات آمد کریم
&&
رسته از صد آفت و اخطار و بیم
\\
گر هزارانند یک کس بیش نیست
&&
چون خیالاتی عدد اندیش نیست
\\
آکل و ماکول را حلقست و نای
&&
غالب و مغلوب را عقلست و رای
\\
حلق بخشید او عصای عدل را
&&
خورد آن چندان عصا و حبل را
\\
واندرو افزون نشد زان جمله اکل
&&
زانک حیوانی نبودش اکل و شکل
\\
مر یقین را چون عصا هم حلق داد
&&
تا بخورد او هر خیالی را که زاد
\\
پس معانی را چو اعیان حلقهاست
&&
رازق حلق معانی هم خداست
\\
پس ز مه تا ماهی هیچ از خلق نیست
&&
که بجذب مایه او را حلق نیست
\\
حلق جان از فکر تن خالی شود
&&
آنگهان روزیش اجلالی شود
\\
شرط تبدیل مزاج آمد بدان
&&
کز مزاج بد بود مرگ بدان
\\
چون مزاج آدمی گل‌خوار شد
&&
زرد و بدرنگ و سقیم و خوار شد
\\
چون مزاج زشت او تبدیل یافت
&&
رفت زشتی از رخش چون شمع تافت
\\
دایه‌ای کو طفل شیرآموز را
&&
تا بنعمت خوش کند پدفوز را
\\
گر ببندد راه آن پستان برو
&&
برگشاید راه صد بستان برو
\\
زانک پستان شد حجاب آن ضعیف
&&
از هزاران نعمت و خوان و رغیف
\\
پس حیات ماست موقوف فطام
&&
اندک اندک جهد کن تم الکلام
\\
چون جنین بد آدمی بد خون غذا
&&
از نجس پاکی برد مؤمن کذا
\\
از فطام خون غذااش شیر شد
&&
وز فطام شیر لقمه‌گیر شد
\\
وز فطام لقمه لقمانی شود
&&
طالب اشکار پنهانی شود
\\
گر جنین را کس بگفتی در رحم
&&
هست بیرون عالمی بس منتظم
\\
یک زمینی خرمی با عرض و طول
&&
اندرو صد نعمت و چندین اکول
\\
کوهها و بحرها و دشتها
&&
بوستانها باغها و کشتها
\\
آسمانی بس بلند و پر ضیا
&&
آفتاب و ماهتاب و صد سها
\\
از جنوب و از شمال و از دبور
&&
باغها دارد عروسیها و سور
\\
در صفت ناید عجایبهای آن
&&
تو درین ظلمت چه‌ای در امتحان
\\
خون خوری در چارمیخ تنگنا
&&
در میان حبس و انجاس و عنا
\\
او بحکم حال خود منکر بدی
&&
زین رسالت معرض و کافر شدی
\\
کین محالست و فریبست و غرور
&&
زانک تصویری ندارد وهم کور
\\
جنس چیزی چون ندید ادراک او
&&
نشنود ادراک منکرناک او
\\
همچنانک خلق عام اندر جهان
&&
زان جهان ابدال می‌گویندشان
\\
کین جهان چاهیست بس تاریک و تنگ
&&
هست بیرون عالمی بی بو و رنگ
\\
هیچ در گوش کسی زیشان نرفت
&&
کین طمع آمد حجاب ژرف و زفت
\\
گوش را بندد طمع از استماع
&&
چشم را بندد غرض از اطلاع
\\
همچنانک آن جنین را طمع خون
&&
کان غذای اوست در اوطان دون
\\
از حدیث این جهان محجوب کرد
&&
غیر خون او می‌نداند چاشت خورد
\\
\end{longtable}
\end{center}
