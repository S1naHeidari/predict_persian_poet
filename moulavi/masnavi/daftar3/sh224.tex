\begin{center}
\section*{بخش ۲۲۴ - امرکردن سلیمان علیه السلام پشهٔ متظلم را به احضار خصم به دیوان حکم}
\label{sec:sh224}
\addcontentsline{toc}{section}{\nameref{sec:sh224}}
\begin{longtable}{l p{0.5cm} r}
پس سلیمان گفت ای زیبادوی
&&
امر حق باید که از جان بشنوی
\\
حق به من گفتست هان ای دادور
&&
مشنو از خصمی تو بی خصمی دگر
\\
تانیاید هر دو خصم اندر حضور
&&
حق نیاید پیش حاکم در ظهور
\\
خصم تنها گر بر آرد صد نفیر
&&
هان و هان بی خصم قول او مگیر
\\
من نیارم رو ز فرمان تافتن
&&
خصم خود را رو بیاور سوی من
\\
گفت قول تست برهان و درست
&&
خصم من بادست و او در حکم تست
\\
بانگ زد آن شه که ای باد صبا
&&
پشه افغان کرد از ظلمت بیا
\\
هین مقابل شو تو و خصم و بگو
&&
پاسخ خصم و بکن دفع عدو
\\
باد چون بشنید آمد تیز تیز
&&
پشه بگرفت آن زمان راه گریز
\\
پس سلیمان گفت ای پشه کجا
&&
باش تا بر هر دو رانم من قضا
\\
گفت ای شه مرگ من از بود اوست
&&
خود سیاه این روز من از دود اوست
\\
او چو آمد من کجا یابم قرار
&&
کو بر آرد از نهاد من دمار
\\
همچنین جویای درگاه خدا
&&
چون خدا آمد شود جوینده لا
\\
گرچه آن وصلت بقا اندر بقاست
&&
لیک ز اول آن بقا اندر فناست
\\
سایه‌هایی که بود جویای نور
&&
نیست گردد چون کند نورش ظهور
\\
عقل کی ماند چو باشد سرده او
&&
کل شیء هالک الا وجهه
\\
هالک آید پیش وجهش هست و نیست
&&
هستی اندر نیستی خود طرفه‌ایست
\\
اندرین محضر خردها شد ز دست
&&
چون قلم اینجا رسیده شد شکست
\\
\end{longtable}
\end{center}
