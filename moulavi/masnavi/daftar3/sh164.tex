\begin{center}
\section*{بخش ۱۶۴ - حکایت آن زنی کی فرزندش نمی‌زیست بنالید جواب آمد کی آن عوض ریاضت تست و به جای جهاد مجاهدانست ترا}
\label{sec:sh164}
\addcontentsline{toc}{section}{\nameref{sec:sh164}}
\begin{longtable}{l p{0.5cm} r}
آن زنی هر سال زاییدی پسر
&&
بیش از شش مه نبودی عمرور
\\
یاسه مه یا چار مه گشتی تباه
&&
ناله کرد آن زن که افغان ای اله
\\
نه مهم بارست و سه ماهم فرح
&&
نعمتم زوتر رو از قوس قزح
\\
پیش مردان خدا کردی نفیر
&&
زین شکایت آن زن از درد نذیر
\\
بیست فرزند این‌چنین در گور رفت
&&
آتشی در جانشان افتاد تفت
\\
تا شبی بنمود او را جنتی
&&
باقیی سبزی خوشی بی ضنتی
\\
باغ گفتم نعمت بی‌کیف را
&&
کاصل نعمتهاست و مجمع باغها
\\
ورنه لا عین رات چه جای باغ
&&
گفت نور غیب را یزدان چراغ
\\
مثل نبود آن مثال آن بود
&&
تا برد بوی آنک او حیران بود
\\
حاصل آن زن دید آن را مست شد
&&
زان تجلی آن ضعیف از دست شد
\\
دید در قصری نبشته نام خویش
&&
آن خود دانستش آن محبوب‌کیش
\\
بعد از آن گفتند کین نعمت وراست
&&
کو بجان بازی به جز صادق نخاست
\\
خدمت بسیار می‌بایست کرد
&&
مر ترا تا بر خوری زین چاشت‌خورد
\\
چون تو کاهل بودی اندر التجا
&&
آن مصیبتها عوض دادت خدا
\\
گفت یا رب تا به صد سال و فزون
&&
این چنینم ده بریز از من تو خون
\\
اندر آن باغ او چو آمد پیش پیش
&&
دید در وی جمله فرزندان خویش
\\
گفت از من کم شد از تو گم نشد
&&
بی دو چشم غیب کس مردم نشد
\\
تو نکردی فصد و از بینی دوید
&&
خون افزون تا ز تب جانت رهید
\\
مغز هر میوه بهست از پوستش
&&
پوست دان تن را و مغز آن دوستش
\\
مغز نغزی دارد آخر آدمی
&&
یکدمی آن را طلب گر زان دمی
\\
\end{longtable}
\end{center}
