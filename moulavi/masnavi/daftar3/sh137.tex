\begin{center}
\section*{بخش ۱۳۷ - مکرر کردن قوم اعتراض ترجیه بر انبیا علیهم‌السلام}
\label{sec:sh137}
\addcontentsline{toc}{section}{\nameref{sec:sh137}}
\begin{longtable}{l p{0.5cm} r}
قوم گفتند از شما سعد خودیت
&&
نحس مایید و ضدیت و مرتدیت
\\
جان ما فارغ بد از اندیشه‌ها
&&
در غم افکندید ما را و عنا
\\
ذوق جمعیت که بود و اتفاق
&&
شد ز فال زشتتان صد افتراق
\\
طوطی نقل شکر بودیم ما
&&
مرغ مرگ‌اندیش گشتیم از شما
\\
هر کجا افسانهٔ غم‌گستریست
&&
هر کجا آوازهٔ مستنکریست
\\
هر کجا اندر جهان فال بذست
&&
هر کجا مسخی نکالی ماخذست
\\
در مثال قصه و فال شماست
&&
در غم‌انگیزی شما را مشتهاست
\\
\end{longtable}
\end{center}
