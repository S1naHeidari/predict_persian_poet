\begin{center}
\section*{بخش ۸۱ - صبرکردن لقمان چون دید کی داود حلقه‌ها می‌ساخت از سال کردن با این نیت کی صبر از سال موجب فرج باشد}
\label{sec:sh081}
\addcontentsline{toc}{section}{\nameref{sec:sh081}}
\begin{longtable}{l p{0.5cm} r}
رفت لقمان سوی داود صفا
&&
دید کو می‌کرد ز آهن حلقه‌ها
\\
جمله را با همدگر در می‌فکند
&&
ز آهن پولاد آن شاه بلند
\\
صنعت زراد او کم دیده بود
&&
درعجب می‌ماند وسواسش فزود
\\
کین چه شاید بود وا پرسم ازو
&&
که چه می‌سازی ز حلقه تو بتو
\\
باز با خود گفت صبر اولیترست
&&
صبر تا مقصود زوتر رهبرست
\\
چون نپرسی زودتر کشفت شود
&&
مرغ صبر از جمله پران‌تر بود
\\
ور بپرسی دیرتر حاصل شود
&&
سهل از بی صبریت مشکل شود
\\
چونک لقمان تن بزد هم در زمان
&&
شد تمام از صنعت داود آن
\\
پس زره سازید و در پوشید او
&&
پیش لقمان کریم صبرخو
\\
گفت این نیکو لباسست ای فتی
&&
درمصاف و جنگ دفع زخم را
\\
گفت لقمان صبر هم نیکو دمیست
&&
که پناه و دافع هر جا غمیست
\\
صبر را با حق قرین کرد ای فلان
&&
آخر والعصر را آگه بخوان
\\
صد هزاران کیمیا حق آفرید
&&
کیمیایی همچو صبر آدم ندید
\\
\end{longtable}
\end{center}
