\begin{center}
\section*{بخش ۲۱۰ - تفسیر آیت واجلب علیهم بخیلک و رجلک}
\label{sec:sh210}
\addcontentsline{toc}{section}{\nameref{sec:sh210}}
\begin{longtable}{l p{0.5cm} r}
تو چو عزم دین کنی با اجتهاد
&&
دیو بانگت بر زند اندر نهاد
\\
که مرو زان سو بیندیش ای غوی
&&
که اسیر رنج و درویشی شوی
\\
بی‌نوا گردی ز یاران وابری
&&
خوار گردی و پشیمانی خوری
\\
تو ز بیم بانگ آن دیو لعین
&&
وا گریزی در ضلالت از یقین
\\
که هلا فردا و پس فردا مراست
&&
راه دین پویم که مهلت پیش ماست
\\
مرگ بینی باز کو از چپ و راست
&&
می‌کشد همسایه را تا بانگ خاست
\\
باز عزم دین کنی از بیم جان
&&
مرد سازی خویشتن را یک زمان
\\
پس سلح بر بندی از علم و حکم
&&
که من از خوفی نیارم پای کم
\\
باز بانگی بر زند بر تو ز مکر
&&
که بترس و باز گرد از تیغ فقر
\\
باز بگریزی ز راه روشنی
&&
آن سلاح علم و فن را بفکنی
\\
سالها او را به بانگی بنده‌ای
&&
در چنین ظلمت نمد افکنده‌ای
\\
هیبت بانگ شیاطین خلق را
&&
بند کردست و گرفته حلق را
\\
تا چنان نومید شد جانشان ز نور
&&
که روان کافران ز اهل قبور
\\
این شکوه بانگ آن ملعون بود
&&
هیبت بانگ خدایی چون بود
\\
هیبت بازست بر کبک نجیب
&&
مر مگس را نیست زان هیبت نصیب
\\
زانک نبود باز صیاد مگس
&&
عنکبوتان می مگس گیرند و بس
\\
عنکبوت دیو بر چون تو ذباب
&&
کر و فر دارد نه بر کبک و عقاب
\\
بانگ دیوان گله‌بان اشقیاست
&&
بانگ سلطان پاسبان اولیاست
\\
تا نیامیزد بدین دو بانگ دور
&&
قطره‌ای از بحر خوش با بحر شور
\\
\end{longtable}
\end{center}
