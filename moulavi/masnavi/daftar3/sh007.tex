\begin{center}
\section*{بخش ۷ - بیان آنک الله گفتن نیازمند عین لبیک گفتن حق است}
\label{sec:sh007}
\addcontentsline{toc}{section}{\nameref{sec:sh007}}
\begin{longtable}{l p{0.5cm} r}
آن یکی الله می‌گفتی شبی
&&
تا که شیرین می‌شد از ذکرش لبی
\\
گفت شیطان آخر ای بسیارگو
&&
این همه الله را لبیک کو
\\
می‌نیاید یک جواب از پیش تخت
&&
چند الله می‌زنی با روی سخت
\\
او شکسته‌دل شد و بنهاد سر
&&
دید در خواب او خضر را در خضر
\\
گفت هین از ذکر چون وا مانده‌ای
&&
چون پشیمانی از آن کش خوانده‌ای
\\
گفت لبیکم نمی‌آید جواب
&&
زان همی‌ترسم که باشم رد باب
\\
گفت آن الله تو لبیک ماست
&&
و آن نیاز و درد و سوزت پیک ماست
\\
حیله‌ها و چاره‌جوییهای تو
&&
جذب ما بود و گشاد این پای تو
\\
ترس و عشق تو کمند لطف ماست
&&
زیر هر یا رب تو لبیکهاست
\\
جان جاهل زین دعا جز دور نیست
&&
زانک یا رب گفتنش دستور نیست
\\
بر دهان و بر دلش قفلست و بند
&&
تا ننالد با خدا وقت گزند
\\
داد مر فرعون را صد ملک و مال
&&
تا بکرد او دعوی عز و جلال
\\
در همه عمرش ندید او درد سر
&&
تا ننالد سوی حق آن بدگهر
\\
داد او را جمله ملک این جهان
&&
حق ندادش درد و رنج و اندهان
\\
درد آمد بهتر از ملک جهان
&&
تا بخوانی مر خدا را در نهان
\\
خواندن بی درد از افسردگیست
&&
خواندن با درد از دل‌بردگیست
\\
آن کشیدن زیر لب آواز را
&&
یاد کردن مبدا و آغاز را
\\
آن شده آواز صافی و حزین
&&
ای خدا وی مستغاث و ای معین
\\
نالهٔ سگ در رهش بی جذبه نیست
&&
زانک هر راغب اسیر ره‌زنیست
\\
چون سگ کهفی که از مردار رست
&&
بر سر خوان شهنشاهان نشست
\\
تا قیامت می‌خورد او پیش غار
&&
آب رحمت عارفانه بی تغار
\\
ای بسا سگ‌پوست کو را نام نیست
&&
لیک اندر پرده بی آن جام نیست
\\
جان بده از بهر این جام ای پسر
&&
بی جهاد و صبر کی باشد ظفر
\\
صبر کردن بهر این نبود حرج
&&
صبر کن کالصبر مفتاح الفرج
\\
زین کمین بی صبر و حزمی کس نرست
&&
حزم را خود صبر آمد پا و دست
\\
حزم کن از خورد کین زهرین گیاست
&&
حزم کردن زور و نور انبیاست
\\
کاه باشد کو به هر بادی جهد
&&
کوه کی مر باد را وزنی نهد
\\
هر طرف غولی همی‌خواند ترا
&&
کای برادر راه خواهی هین بیا
\\
ره نمایم همرهت باشم رفیق
&&
من قلاووزم درین راه دقیق
\\
نه قلاوزست و نه ره داند او
&&
یوسفا کم رو سوی آن گرگ‌خو
\\
حزم این باشد که نفریبد ترا
&&
چرب و نوش و دامهای این سرا
\\
که نه چربش دارد و نه نوش او
&&
سحر خواند می‌دمد در گوش او
\\
که بیا مهمان ما ای روشنی
&&
خانه آن تست و تو آن منی
\\
حزم آن باشد که گویی تخمه‌ام
&&
یا سقیمم خستهٔ این دخمه‌ام
\\
یا سرم دردست درد سر ببر
&&
یا مرا خواندست آن خالو پسر
\\
زانک یک نوشت دهد با نیشها
&&
که بکارد در تو نوشش ریشها
\\
زر اگر پنجاه اگر شصتت دهد
&&
ماهیا او گوشت در شستت دهد
\\
گر دهد خود کی دهد آن پر حیل
&&
جوز پوسیدست گفتار دغل
\\
ژغژغ آن عقل و مغزت را برد
&&
صد هزاران عقل را یک نشمرد
\\
یار تو خرجین تست و کیسه‌ات
&&
گر تو رامینی مجو جز ویسه‌ات
\\
ویسه و معشوق تو هم ذات تست
&&
وین برونیها همه آفات تست
\\
حزم آن باشد که چون دعوت کنند
&&
تو نگویی مست و خواهان منند
\\
دعوت ایشان صفیر مرغ دان
&&
که کند صیاد در مکمن نهان
\\
مرغ مرده پیش بنهاده که این
&&
می‌کند این بانگ و آواز و حنین
\\
مرغ پندارد که جنس اوست او
&&
جمع آید بر دردشان پوست او
\\
جز مگر مرغی که حزمش داد حق
&&
تا نگردد گیج آن دانه و ملق
\\
هست بی حزمی پشیمانی یقین
&&
بشنو این افسانه را در شرح این
\\
\end{longtable}
\end{center}
