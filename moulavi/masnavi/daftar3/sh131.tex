\begin{center}
\section*{بخش ۱۳۱ - وخامت کار آن مرغ کی ترک حزم کرد از حرص و هوا}
\label{sec:sh131}
\addcontentsline{toc}{section}{\nameref{sec:sh131}}
\begin{longtable}{l p{0.5cm} r}
باز مرغی فوق دیواری نشست
&&
دیده سوی دانه دامی ببست
\\
یک نظر او سوی صحرا می‌کند
&&
یک نظر حرصش به دانه می‌کشد
\\
این نظر با آن نظر چالیش کرد
&&
ناگهانی از خرد خالیش کرد
\\
باز مرغی کان تردد را گذاشت
&&
زان نظر بر کند و بر صحرا گماشت
\\
شاد پر و بال او بخا له
&&
تا امام جمله آزادان شد او
\\
هر که او را مقتدا سازد برست
&&
در مقام امن و آزادی نشست
\\
زانک شاه حازمان آمد دلش
&&
تا گلستان و چمن شد منزلش
\\
حزم ازو راضی و او راضی ز حزم
&&
این چنین کن گر کنی تدبیر و عزم
\\
بارها در دام حرص افتاده‌ای
&&
حلق خود را در بریدن داده‌ای
\\
بازت آن تواب لطف آزاد کرد
&&
توبه پذرفت و شما را شاد کرد
\\
گفت ان عدتم کذا عدنا کذا
&&
نحن زوجنا الفعال بالجزا
\\
چونک جفتی را بر خود آورم
&&
آید آن را جفتش دوانه لاجرم
\\
جفت کردیم این عمل را با اثر
&&
چون رسد جفتی رسد جفتی دگر
\\
چون رباید غارتی از جفت شوی
&&
جفت می‌آید پس او شوی‌جوی
\\
بار دیگر سوی این دام آمدیت
&&
خاک اندر دیدهٔ توبه زدیت
\\
بازتان تواب بگشاد از گره
&&
گفت هین بگریز روی این سو منه
\\
باز چون پروانهٔ نسیان رسید
&&
جانتان را جانب آتش کشید
\\
کم کن ای پروانه نسیان و شکی
&&
در پر سوزیده بنگر تو یکی
\\
چون رهیدی شکر آن باشد که هیچ
&&
سوی آن دانه نداری پیچ پیچ
\\
تا ترا چون شکر گویی بخشد او
&&
روزیی بی دام و بی خوف عدو
\\
شکر آن نعمت که‌تان آزاد کرد
&&
نعمت حق را بباید یاد کرد
\\
چند اندر رنجها و در بلا
&&
گفتی از دامم رها ده ای خدا
\\
تا چنین خدمت کنم احسان کنم
&&
خاک اندر دیدهٔ شیطان زنم
\\
\end{longtable}
\end{center}
