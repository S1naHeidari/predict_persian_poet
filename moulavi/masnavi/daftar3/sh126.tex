\begin{center}
\section*{بخش ۱۲۶ - بیان آنک هر کس را نرسد مثل آوردن  خاصه در کار الهی}
\label{sec:sh126}
\addcontentsline{toc}{section}{\nameref{sec:sh126}}
\begin{longtable}{l p{0.5cm} r}
کی رسدتان این مثلها ساختن
&&
سوی آن درگاه پاک انداختن
\\
آن مثل آوردن آن حضرتست
&&
که بعلم سر و جهر او آیتست
\\
تو چه دانی سر چیزی تا تو کل
&&
یا به زلفی یا به رخ آری مثل
\\
موسیی آن را عصا دید و نبود
&&
اژدها بد سر او لب می‌گشود
\\
چون چنان شاهی نداند سر چوب
&&
تو چه دانی سر این دام و حبوب
\\
چون غلط شد چشم موسی در مثل
&&
چون کند موشی فضولی مدخل
\\
آن مثالت را چو اژدرها کند
&&
تا به پاسخ جزو جزوت بر کند
\\
این مثال آورد ابلیس لعین
&&
تا که شد ملعون حق تا یوم دین
\\
این مثال آورد قارون از لجاج
&&
تا فرو شد در زمین با تخت و تاج
\\
این مثالت را چو زاغ و بوم دان
&&
که ازیشان پست شد صد خاندان
\\
\end{longtable}
\end{center}
