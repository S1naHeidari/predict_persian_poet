\begin{center}
\section*{بخش ۱۵۱ - بیان آنک حق تعالی هرچه داد و آفرید از سماوات و ارضین و اعیان و اعراض همه باستدعاء حاجت آفرید خود را محتاج چیزی باید کردن تا بدهد کی امن یجیب المضطر اذا دعاه اضطرار گواه استحقاقست}
\label{sec:sh151}
\addcontentsline{toc}{section}{\nameref{sec:sh151}}
\begin{longtable}{l p{0.5cm} r}
آن نیاز مریمی بودست و درد
&&
که چنان طفلی سخن آغاز کرد
\\
جزو او بی او برای او بگفت
&&
جزو جزوت گفت دارد در نهفت
\\
دست و پا شاهد شوندت ای رهی
&&
منکری را چند دست و پا نهی
\\
ور نباشی مستحق شرح و گفت
&&
ناطقهٔ ناطق ترا دید و بخفت
\\
هر چه رویید از پی محتاج رست
&&
تا بیابد طالبی چیزی که جست
\\
حق تعالی گر سماوات آفرید
&&
از برای دفع حاجات آفرید
\\
هر کجا دردی دوا آنجا رود
&&
هر کجا فقری نوا آنجا رود
\\
هر کجا مشکل جواب آنجا رود
&&
هر کجا کشتیست آب آنجا رود
\\
آب کم جو تشنگی آور بدست
&&
تا بجوشد آب از بالا و پست
\\
تا نزاید طفلک نازک گلو
&&
کی روان گردد ز پستان شیر او
\\
رو بدین بالا و پستیها بدو
&&
تا شوی تشنه و حرارت را گرو
\\
بعد از آن بانگ زنبور هوا
&&
بانگ آب جو بنوشی ای کیا
\\
حاجت تو کم نباشد از حشیش
&&
آب را گیری سوی او می‌کشیش
\\
گوش گیری آب را تو می‌کشی
&&
سوی زرع خشک تا یابد خوشی
\\
زرع جان را کش جواهر مضمرست
&&
ابر رحمت پر ز آب کوثرست
\\
تا سقاهم ربهم آید خطاب
&&
تشنه باش الله اعلم بالصواب
\\
\end{longtable}
\end{center}
