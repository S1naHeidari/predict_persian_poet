\begin{center}
\section*{بخش ۲۰ - چرب کردن مرد لافی لب و سبلت خود را هر بامداد به پوست دنبه و بیرون آمدن میان  حریفان کی من چنین خورده‌ام و چنان}
\label{sec:sh020}
\addcontentsline{toc}{section}{\nameref{sec:sh020}}
\begin{longtable}{l p{0.5cm} r}
پوست دنبه یافت شخصی مستهان
&&
هر صباحی چرب کردی سبلتان
\\
در میان منعمان رفتی که من
&&
لوت چربی خورده‌ام در انجمن
\\
دست بر سبلت نهادی در نوید
&&
رمز یعنی سوی سبلت بنگرید
\\
کین گواه صدق گفتار منست
&&
وین نشان چرب و شیرین خوردنست
\\
اشکمش گفتی جواب بی‌طنین
&&
که اباد الله کید الکاذبین
\\
لاف تو ما را بر آتش بر نهاد
&&
کان سبال چرب تو بر کنده باد
\\
گر نبودی لاف زشتت ای گدا
&&
یک کریمی رحم افکندی به ما
\\
ور نمودی عیب و کژ کم باختی
&&
یک طبیبی داروی او ساختی
\\
گفت حق که کژ مجنبان گوش و دم
&&
ینفعن الصادقین صدقهم
\\
گفت اندر کژ مخسپ ای محتلم
&&
آنچ داری وا نما و فاستقم
\\
ور نگویی عیب خود باری خمش
&&
از نمایش وز دغل خود را مکش
\\
گر تو نقدی یافتی مگشا دهان
&&
هست در ره سنگهای امتحان
\\
سنگهای امتحان را نیز پیش
&&
امتحانها هست در احوال خویش
\\
گفت یزدان از ولادت تا بحین
&&
یفتنون کل عام مرتین
\\
امتحان در امتحانست ای پدر
&&
هین به کمتر امتحان خود را مخر
\\
\end{longtable}
\end{center}
