\begin{center}
\section*{بخش ۱۲۴ - حکایت خرگوشان کی خرگوشی راپیش پیل  فرستادند کی بگو کی من رسول ماه آسمانم پیش تو کی ازین چشمه آب حذر کن چنانک در کتاب کلیله تمام گفته است}
\label{sec:sh124}
\addcontentsline{toc}{section}{\nameref{sec:sh124}}
\begin{longtable}{l p{0.5cm} r}
این بدان ماند که خرگوشی بگفت
&&
من رسول ماهم و با ماه جفت
\\
کز رمهٔ پیلان بر آن چشمهٔ زلال
&&
جمله نخجیران بدند اندر وبال
\\
جمله محروم و ز خوف از چشمه دور
&&
حیله‌ای کردند چون کم بود زور
\\
از سر که بانگ زد خرگوش زال
&&
سوی پیلان در شب غرهٔ هلال
\\
که بیا رابع عشر ای شاه‌پیل
&&
تا درون چشمه یابی این دلیل
\\
شاه‌پیلا من رسولم پیش بیست
&&
بر رسولان بند و زجر و خشم نیست
\\
ماه می‌گوید که ای پیلان روید
&&
چشمه آن ماست زین یکسو شوید
\\
ورنه منتان کور گردانم ستم
&&
گفتم از گردن برون انداختم
\\
ترک این چشمه بگویید و روید
&&
تا ز زخم تیغ مه ایمن شوید
\\
نک نشان آنست کاندر چشمه ماه
&&
مضطرب گردد ز پیل آب‌خواه
\\
آن فلان شب حاضر آ ای شاه‌پیل
&&
تا درون چشمه یابی زین دلیل
\\
چونک هفت و هشت از مه بگذرید
&&
شاه‌پیل آمد ز چشمه می‌چرید
\\
چونک زد خرطوم پیل آن شب درآب
&&
مضطرب شد آب ومه کرد اضطراب
\\
پیل باور کرد از وی آن خطاب
&&
چون درون چشمه مه کرد اضطراب
\\
مانه زان پیلان گولیم ای گروه
&&
که اضطراب ماه آردمان شکوه
\\
انبیا گفتند آوه پند جان
&&
سخت‌تر کرد ای سفیهان بندتان
\\
\end{longtable}
\end{center}
