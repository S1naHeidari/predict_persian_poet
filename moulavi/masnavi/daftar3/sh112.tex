\begin{center}
\section*{بخش ۱۱۲ - عزم کردن داود علیه السلام به خواندن خلق بدان صحرا کی راز آشکارا کند و  حجتها را همه قطع کند}
\label{sec:sh112}
\addcontentsline{toc}{section}{\nameref{sec:sh112}}
\begin{longtable}{l p{0.5cm} r}
گفت ای یاران زمان آن رسید
&&
کان سر مکتوم او گردد پدید
\\
جمله برخیزید تا بیرون رویم
&&
تا بر آن سر نهان واقف شویم
\\
در فلان صحرا درختی هست زفت
&&
شاخهااش انبه و بسیار و چفت
\\
سخت راسخ خیمه‌گاه و میخ او
&&
بوی خون می‌آیدم از بیخ او
\\
خون شدست اندر بن آن خوش درخت
&&
خواجه راکشتست این منحوس‌بخت
\\
تا کنون حلم خدا پوشید آن
&&
آخر از ناشکری آن قلتبان
\\
که عیال خواجه را روزی ندید
&&
نه بنوروز و نه موسمهای عید
\\
بی‌نوایان را به یک لقمه نجست
&&
یاد ناورد او ز حقهای نخست
\\
تا کنون از بهر یک گاو این لعین
&&
می‌زند فرزند او را در زمین
\\
او بخود برداشت پرده از گناه
&&
ورنه می‌پوشید جرمش را اله
\\
کافر و فاسق درین دور گزند
&&
پرده خود را بخود بر می‌درند
\\
ظلم مستورست در اسرار جان
&&
می‌نهد ظالم بپیش مردمان
\\
که ببینیدم که دارم شاخها
&&
گاو دوزخ را ببینید از ملا
\\
\end{longtable}
\end{center}
