\begin{center}
\section*{بخش ۱۰ - جمع آمدن اهل آفت هر صباحی بر در صومعهٔ  عیسی علیه السلام جهت طلب شفا به دعای او}
\label{sec:sh010}
\addcontentsline{toc}{section}{\nameref{sec:sh010}}
\begin{longtable}{l p{0.5cm} r}
صومعهٔ عیسیست خوان اهل دل
&&
هان و هان ای مبتلا این در مهل
\\
جمع گشتندی ز هر اطراف خلق
&&
از ضریر و لنگ و شل و اهل دلق
\\
بر در آن صومعهٔ عیسی صباح
&&
تا بدم اوشان رهاند از جناح
\\
او چو فارغ گشتی از اوراد خویش
&&
چاشتگه بیرون شدی آن خوب‌کیش
\\
جوق جوقی مبتلا دیدی نزار
&&
شسته بر در در امید و انتظار
\\
گفتی ای اصحاب آفت از خدا
&&
حاجت این جملگانتان شد روا
\\
هین روان گردید بی رنج و عنا
&&
سوی غفاری و اکرام خدا
\\
جملگان چون اشتران بسته‌پای
&&
که گشایی زانوی ایشان برای
\\
خوش دوان و شادمانه سوی خان
&&
از دعای او شدندی پا دوان
\\
آزمودی تو بسی آفات خویش
&&
یافتی صحت ازین شاهان کیش
\\
چند آن لنگی تو رهوار شد
&&
چند جانت بی غم و آزار شد
\\
ای مغفل رشته‌ای بر پای بند
&&
تا ز خود هم گم نگردی ای لوند
\\
ناسپاسی و فراموشی تو
&&
یاد ناورد آن عسل‌نوشی تو
\\
لاجرم آن راه بر تو بسته شد
&&
چون دل اهل دل از تو خسته شد
\\
زودشان در یاب و استغفار کن
&&
همچو ابری گریه‌های زار کن
\\
تا گلستانشان سوی تو بشکفد
&&
میوه‌های پخته بر خود وا کفد
\\
هم بر آن در گرد کم از سگ مباش
&&
با سگ کهف ار شدستی خواجه‌تاش
\\
چون سگان هم مر سگان را ناصح‌اند
&&
که دل اندر خانهٔ اول ببند
\\
آن در اول که خوردی استخوان
&&
سخت گیر و حق گزار آن را ممان
\\
می‌گزندش تا ز ادب آنجا رود
&&
وز مقام اولین مفلح شود
\\
می‌گزندش کای سگ طاغی برو
&&
با ولی نعمتت یاغی مشو
\\
بر همان در همچو حلقه بسته باش
&&
پاسبان و چابک و برجسته باش
\\
صورت نقض وفای ما مباش
&&
بی‌وفایی را مکن بیهوده فاش
\\
مر سگان را چون وفا آمد شعار
&&
رو سگان را ننگ و بدنامی میار
\\
بی‌وفایی چون سگان را عار بود
&&
بی‌وفایی چون روا داری نمود
\\
حق تعالی فخر آورد از وفا
&&
گفت من اوفی بعهد غیرنا
\\
بی‌وفایی دان وفا با رد حق
&&
بر حقوق حق ندارد کس سبق
\\
حق مادر بعد از آن شد کان کریم
&&
کرد او را از جنین تو غریم
\\
صورتی کردت درون جسم او
&&
داد در حملش ورا آرام و خو
\\
همچو جزو متصل دید او ترا
&&
متصل را کرد تدبیرش جدا
\\
حق هزاران صنعت و فن ساختست
&&
تا که مادر بر تو مهر انداختست
\\
پس حق حق سابق از مادر بود
&&
هر که آن حق را نداند خر بود
\\
آنک مادر آفرید و ضرع و شیر
&&
با پدر کردش قرین آن خود مگیر
\\
ای خداوند ای قدیم احسان تو
&&
آنک دانم وانک نه هم آن تو
\\
تو بفرمودی که حق را یاد کن
&&
زانک حق من نمی‌گردد کهن
\\
یاد کن لطفی که کردم آن صبوح
&&
با شما از حفظ در کشتی نوح
\\
پیله بابایانتان را آن زمان
&&
دادم از طوفان و از موجش امان
\\
آب آتش خو زمین بگرفته بود
&&
موج او مر اوج که را می‌ربود
\\
حفظ کردم من نکردم ردتان
&&
در وجود جد جد جدتان
\\
چون شدی سر پشت پایت چون زنم
&&
کارگاه خویش ضایع چون کنم
\\
چون فدای بی‌وفایان می‌شوی
&&
از گمان بد بدان سو می‌روی
\\
من ز سهو و بی‌وفاییها بری
&&
سوی من آیی گمان بد بری
\\
این گمان بد بر آنجا بر که تو
&&
می‌شوی در پیش همچون خود دوتو
\\
بس گرفتی یار و همراهان زفت
&&
گر ترا پرسم که کو گویی که زفت
\\
یار نیکت رفت بر چرخ برین
&&
یار فسقت رفت در قعر زمین
\\
تو بماندی در میانه آنچنان
&&
بی‌مدد چون آتشی از کاروان
\\
دامن او گیر ای یار دلیر
&&
کو منزه باشد از بالا و زیر
\\
نه چو عیسی سوی گردون بر شود
&&
نه چو قارون در زمین اندر رود
\\
با تو باشد در مکان و بی‌مکان
&&
چون بمانی از سرا و از دکان
\\
او بر آرد از کدورتها صفا
&&
مر جفاهای ترا گیرد وفا
\\
چون جفا آری فرستد گوشمال
&&
تا ز نقصان وا روی سوی کمال
\\
چون تو وردی ترک کردی در روش
&&
بر تو قبضی آید از رنج و تبش
\\
آن ادب کردن بود یعنی مکن
&&
هیچ تحویلی از آن عهد کهن
\\
پیش از آن کین قبض زنجیری شود
&&
این که دلگیریست پاگیری شود
\\
رنج معقولت شود محسوس و فاش
&&
تا نگیری این اشارت را بلاش
\\
در معاصی قبضها دلگیر شد
&&
قبضها بعد از اجل زنجیر شد
\\
نعط من اعرض هنا عن ذکرنا
&&
عیشة ضنک و نجزی بالعمی
\\
دزد چون مال کسان را می‌برد
&&
قبض و دلتنگی دلش را می‌خلد
\\
او همی‌گوید عجب این قبض چیست
&&
قبض آن مظلوم کز شرت گریست
\\
چون بدین قبض التفاتی کم کند
&&
باد اصرار آتشش را دم کند
\\
قبض دل قبض عوان شد لاجرم
&&
گشت محسوس آن معانی زد علم
\\
غصه‌ها زندان شدست و چارمیخ
&&
غصه بیخست و بروید شاخ بیخ
\\
بیخ پنهان بود هم شد آشکار
&&
قبض و بسط اندرون بیخی شمار
\\
چونک بیخ بد بود زودش بزن
&&
تا نروید زشت‌خاری در چمن
\\
قبض دیدی چارهٔ آن قبض کن
&&
زانک سرها جمله می‌روید ز بن
\\
بسط دیدی بسط خود را آب ده
&&
چون بر آید میوه با اصحاب ده
\\
\end{longtable}
\end{center}
