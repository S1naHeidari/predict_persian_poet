\begin{center}
\section*{بخش ۱۷۱ - بیان آنک هرچه غفلت و غم و کاهلی و تاریکیست همه از تنست کی ارضی است و سفلی}
\label{sec:sh171}
\addcontentsline{toc}{section}{\nameref{sec:sh171}}
\begin{longtable}{l p{0.5cm} r}
غفلت از تن بود چون تن روح شد
&&
بیند او اسرار را بی هیچ بد
\\
چون زمین برخاست از جو فلک
&&
نه شب و نه سایه باشد نه دلک
\\
هر کجا سایه‌ست و شب یا سایگه
&&
از زمین باشد نه از افلاک و مه
\\
دود پیوسته هم از هیزم بود
&&
نه ز آتشهای مستنجم بود
\\
وهم افتد در خطا و در غلط
&&
عقل باشد در اصابتها فقط
\\
هر گرانی و کسل خود از تنست
&&
جان ز خفت جمله در پریدنست
\\
روی سرخ از غلبه خونها بود
&&
روی زرد از جنبش صفرا بود
\\
رو سپید از قوت بلغم بود
&&
باشد از سودا که رو ادهم بود
\\
در حقیقت خالق آثار اوست
&&
لیک جز علت نبیند اهل پوست
\\
مغز کو از پوستها آواره نیست
&&
از طبیب و علت او را چاره نیست
\\
چون دوم بار آدمی‌زاده بزاد
&&
پای خود بر فرق علتها نهاد
\\
علت اولی نباشد دین او
&&
علت جزوی ندارد کین او
\\
می‌پرد چون آفتاب اندر افق
&&
با عروس صدق و صورت چون تتق
\\
بلک بیرون از افق وز چرخها
&&
بی مکان باشد چو ارواح و نهی
\\
بل عقول ماست سایه‌های او
&&
می‌فتد چون سایه‌ها در پای او
\\
مجتهد هر گه که باشد نص‌شناس
&&
اندر آن صورت نیندیشد قیاس
\\
چون نیابد نص اندر صورتی
&&
از قیاس آنجا نماید عبرتی
\\
\end{longtable}
\end{center}
