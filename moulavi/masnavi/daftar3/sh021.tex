\begin{center}
\section*{بخش ۲۱ - آمن بودن بلعم باعور کی امتحانها کرد حضرت او را و از آنها روی سپید آمده بود}
\label{sec:sh021}
\addcontentsline{toc}{section}{\nameref{sec:sh021}}
\begin{longtable}{l p{0.5cm} r}
بلعم باعور و ابلیس لعین
&&
ز امتحان آخرین گشته مهین
\\
او بدعوی میل دولت می‌کند
&&
معده‌اش نفرین سبلت می‌کند
\\
کانچ پنهان می‌کند پیدایش کن
&&
سوخت ما را ای خدا رسواش کن
\\
جمله اجزای تنش خصم ویند
&&
کز بهاری لافد ایشان در دیند
\\
لاف وا داد کرمها می‌کند
&&
شاخ رحمت را ز بن بر می‌کند
\\
راستی پیش آر یا خاموش کن
&&
وانگهان رحمت ببین و نوش کن
\\
آن شکم خصم سبال او شده
&&
دست پنهان در دعا اندر زده
\\
کای خدا رسوا کن این لاف لئام
&&
تا بجنبد سوی ما رحم کرام
\\
مستجاب آمد دعای آن شکم
&&
شورش حاجت بزد بیرون علم
\\
گفت حق گر فاسقی و اهل صنم
&&
چون مرا خوانی اجابتها کنم
\\
تو دعا را سخت گیر و می‌شخول
&&
عاقبت برهاندت از دست غول
\\
چون شکم خود را به حضرت در سپرد
&&
گربه آمد پوست آن دنبه ببرد
\\
از پس گربه دویدند او گریخت
&&
کودک از ترس عتابش رنگ ریخت
\\
آمد اندر انجمن آن طفل خرد
&&
آب روی مرد لافی را ببرد
\\
گفت آن دنبه که هر صبحی بدان
&&
چرب می‌کردی لبان و سبلتان
\\
گربه آمد ناگهانش در ربود
&&
بس دویدیم و نکرد آن جهد سود
\\
خنده آمد حاضران را از شگفت
&&
رحمهاشان باز جنبیدن گرفت
\\
دعوتش کردند و سیرش داشتند
&&
تخم رحمت در زمینش کاشتند
\\
او چو ذوق راستی دید از کرام
&&
بی تکبر راستی را شد غلام
\\
\end{longtable}
\end{center}
