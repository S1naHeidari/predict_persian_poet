\begin{center}
\section*{بخش ۲۰۰ - عذر گفتن کدبانو با نخود و حکمت در  جوش داشتن کدبانو نخود را}
\label{sec:sh200}
\addcontentsline{toc}{section}{\nameref{sec:sh200}}
\begin{longtable}{l p{0.5cm} r}
آن ستی گوید ورا که پیش ازین
&&
من چو تو بودم ز اجزای زمین
\\
چون بنوشیدم جهاد آذری
&&
پس پذیرا گشتم و اندر خوری
\\
مدتی جوشیده‌ام اندر زمن
&&
مدتی دیگر درون دیگ تن
\\
زین دو جوشش قوت حسها شدم
&&
روح گشتم پس ترا استا شدم
\\
در جمادی گفتمی زان می‌دوی
&&
تا شوی علم و صفات معنوی
\\
چون شدم من روح پس بار دگر
&&
جوش دیگر کن ز حیوانی گذر
\\
از خدا می‌خواه تا زین نکته‌ها
&&
در نلغزی و رسی در منتها
\\
زانک از قرآن بسی گمره شدند
&&
زان رسن قومی درون چه شدند
\\
مر رسن را نیست جرمی ای عنود
&&
چون ترا سودای سربالا نبود
\\
\end{longtable}
\end{center}
