\begin{center}
\section*{بخش ۵۲ - حکایت}
\label{sec:sh052}
\addcontentsline{toc}{section}{\nameref{sec:sh052}}
\begin{longtable}{l p{0.5cm} r}
در صحابه کم بدی حافظ کسی
&&
گرچه شوقی بود جانشان را بسی
\\
زانک چون مغزش در آگند و رسید
&&
پوستها شد بس رقیق و واکفید
\\
قشر جوز و فستق و بادام هم
&&
مغز چون آگندشان شد پوست کم
\\
مغز علم افزود کم شد پوستش
&&
زانک عاشق را بسوزد دوستش
\\
وصف مطلوبی چو ضد طالبیست
&&
وحی و برق نور سوزندهٔ نبیست
\\
چون تجلی کرد اوصاف قدیم
&&
پس بسوزد وصف حادث را گلیم
\\
ربع قرآن هر که را محفوظ بود
&&
جل فینا از صحابه می‌شنود
\\
جمع صورت با چنین معنی ژرف
&&
نیست ممکن جز ز سلطانی شگرف
\\
در چنین مستی مراعات ادب
&&
خود نباشد ور بود باشد عجب
\\
اندر استغنا مراعات نیاز
&&
جمع ضدینست چون گرد و دراز
\\
خود عصا معشوق عمیان می‌بود
&&
کور خود صندوق قرآن می‌بود
\\
گفت کوران خود صنادیقند پر
&&
از حروف مصحف و ذکر و نذر
\\
باز صندوقی پر از قرآن به است
&&
زانک صندوقی بود خالی بدست
\\
باز صندوقی که خالی شد ز بار
&&
به ز صندوقی که پر موشست و مار
\\
حاصل اندر وصل چون افتاد مرد
&&
گشت دلاله به پیش مرد سرد
\\
چون به مطلوبت رسیدی ای ملیح
&&
شد طلب کاری علم اکنون قبیح
\\
چون شدی بر بامهای آسمان
&&
سرد باشد جست وجوی نردبان
\\
جز برای یاری و تعلیم غیر
&&
سرد باشد راه خیر از بعد خیر
\\
آینهٔ روشن که شد صاف و ملی
&&
جهل باشد بر نهادن صیقلی
\\
پیش سلطان خوش نشسته در قبول
&&
زشت باشد جستن نامه و رسول
\\
\end{longtable}
\end{center}
