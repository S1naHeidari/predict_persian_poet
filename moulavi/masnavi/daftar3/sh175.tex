\begin{center}
\section*{بخش ۱۷۵ - فرق میان دانستن چیزی به مثال و تقلید و میان دانستن ماهیت آن چیز}
\label{sec:sh175}
\addcontentsline{toc}{section}{\nameref{sec:sh175}}
\begin{longtable}{l p{0.5cm} r}
ظاهرست آثار و میوهٔ رحمتش
&&
لیک کی داند جز او ماهیتش
\\
هیچ ماهیات اوصاف کمال
&&
کس نداند جز بثار و مثال
\\
طفل ماهیت نداند طمث را
&&
جز که گویی هست چون حلوا ترا
\\
کی بود ماهیت ذوق جماع
&&
مثل ماهیات حلوا ای مطاع
\\
لیک نسبت کرد از روی خوشی
&&
با تو آن عاقل چو تو کودک‌وشی
\\
تا بداند کودک آن را از مثال
&&
گر نداند ماهیت یا عین حال
\\
پس اگر گویی بدانم دور نیست
&&
ور ندانم گفت کذب و زور نیست
\\
گر کسی گوید که دانی نوح را
&&
آن رسول حق و نور روح را
\\
گر بگویی چون ندانم کان قمر
&&
هست از خورشید و مه مشهورتر
\\
کودکان خرد در کتابها
&&
و آن امامان جمله در محرابها
\\
نام او خوانند در قرآن صریح
&&
قصه‌اش گویند از ماضی فصیح
\\
راست‌گو دانیش تو از روی وصف
&&
گرچه ماهیت نشد از نوح کشف
\\
ور بگویی من چه دانم نوح را
&&
همچو اویی داند او را ای فتی
\\
مور لنگم من چه دانم فیل را
&&
پشه‌ای کی داند اسرافیل را
\\
این سخن هم راستست از روی آن
&&
که بماهیت ندانیش ای فلان
\\
عجز از ادراک ماهیت عمو
&&
حالت عامه بود مطلق مگو
\\
زانک ماهیات و سر سر آن
&&
پیش چشم کاملان باشد عیان
\\
در وجود از سر حق و ذات او
&&
دورتر از فهم و استبصار کو
\\
چونک آن مخفی نماند از محرمان
&&
ذات و وصفی چیست کان ماند نهان
\\
عقل بحثی گوید این دورست و گو
&&
بی ز تاویل محالی کم شنو
\\
قطب گوید مر ترا ای سست‌حال
&&
آنچ فوق حال تست آید محال
\\
واقعاتی که کنونت بر گشود
&&
نه که اول هم محالت می‌نمود
\\
چون رهانیدت ز ده زندان کرم
&&
تیه را بر خود مکن حبس ستم
\\
\end{longtable}
\end{center}
