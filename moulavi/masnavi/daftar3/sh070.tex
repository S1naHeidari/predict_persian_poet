\begin{center}
\section*{بخش ۷۰ - بقیهٔ قصهٔ آن زاهد کوهی کی نذر کرده بود کی میوهٔ کوهی از درخت باز نکنم و درخت نفشانم و کسی را نگویم صریح و کنایت کی بیفشان آن خورم کی باد افکنده باشد  از درخت}
\label{sec:sh070}
\addcontentsline{toc}{section}{\nameref{sec:sh070}}
\begin{longtable}{l p{0.5cm} r}
اندر آن که بود اشجار و ثمار
&&
بس مرودی کوهی آنجا بی‌شمار
\\
گفت آن درویش یا رب با تو من
&&
عهد کردم زین نچینم در زمن
\\
جز از آن میوه که باد انداختش
&&
من نچینم از درخت منتعش
\\
مدتی بر نذر خود بودش وفا
&&
تا در آمد امتحانات قضا
\\
زین سبب فرمود استثنا کنید
&&
گر خدا خواهد به پیمان بر زنید
\\
هر زمان دل را دگر میلی دهم
&&
هرنفس بر دل دگر داغی نهم
\\
کل اصباح لنا شان جدید
&&
کل شیء عن مرادی لا یحید
\\
در حدیث آمد که دل همچون پریست
&&
در بیابانی اسیر صرصریست
\\
باد پر را هر طرف راند گزاف
&&
گه چپ و گه راست با صد اختلاف
\\
در حدیث دیگر این دل دان چنان
&&
کب جوشان ز آتش اندر قازغان
\\
هر زمان دل را دگر رایی بود
&&
آن نه از وی لیک از جایی بود
\\
پس چرا آمن شوی بر رای دل
&&
عهد بندی تا شوی آخر خجل
\\
این هم از تاثیر حکمست و قدر
&&
چاه می‌بیینی و نتوانی حذر
\\
نیست خود ازمرغ پران این عجب
&&
که نبیند دام و افتد در عطب
\\
این عجب که دام بیند هم وتد
&&
گر بخواهد ور نخواهد می‌فتد
\\
چشم باز و گوش باز و دام پیش
&&
سوی دامی می‌پرد با پر خویش
\\
\end{longtable}
\end{center}
