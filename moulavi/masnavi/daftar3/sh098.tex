\begin{center}
\section*{بخش ۹۸ - اقتدا کردن قوم از پس دقوقی}
\label{sec:sh098}
\addcontentsline{toc}{section}{\nameref{sec:sh098}}
\begin{longtable}{l p{0.5cm} r}
پیش در شد آن دقوقی در نماز
&&
قوم همچون اطلس آمد او طراز
\\
اقتدا کردند آن شاهان قطار
&&
در پی آن مقتدای نامدار
\\
چونک با تکبیرها مقرون شدند
&&
همچو قربان از جهان بیرون شدند
\\
معنی تکبیر اینست ای امام
&&
کای خدا پیش تو ما قربان شدیم
\\
وقت ذبح الله اکبر می‌کنی
&&
همچنین در ذبح نفس کشتنی
\\
تن چو اسمعیل و جان همچون خلیل
&&
کرد جان تکبیر بر جسم نبیل
\\
گشت کشته تن ز شهوتها و آز
&&
شد به بسم الله بسمل در نماز
\\
چون قیامت پیش حق صفها زده
&&
در حساب و در مناجات آمده
\\
ایستاده پیش یزدان اشک‌ریز
&&
بر مثال راست‌خیز رستخیز
\\
حق همی‌گوید چه آوردی مرا
&&
اندرین مهلت که دادم من ترا
\\
عمر خود را در چه پایان برده‌ای
&&
قوت و قوت در چه فانی کرده‌ای
\\
گوهر دیده کجا فرسوده‌ای
&&
پنج حس را در کجا پالوده‌ای
\\
چشم و هوش و گوش و گوهرهای عرش
&&
خرج کردی چه خریدی تو ز فرش
\\
دست و پا دادمت چون بیل و کلند
&&
من ببخشیدم ز خود آن کی شدند
\\
همچنین پیغامهای دردگین
&&
صد هزاران آید از حضرت چنین
\\
در قیام این کفتها دارد رجوع
&&
وز خجالت شد دوتا او در رکوع
\\
قوت استادن از خجلت نماند
&&
در رکوع از شرم تسبیحی بخواند
\\
باز فرمان می‌رسد بردار سر
&&
از رکوع و پاسخ حق بر شمر
\\
سر بر آرد از رکوع آن شرمسار
&&
باز اندر رو فتد آن خام‌کار
\\
باز فرمان آیدش بردار سر
&&
از سجود و وا ده از کرده خبر
\\
سر بر آرد او دگر ره شرمسار
&&
اندر افتد باز در رو همچو مار
\\
باز گوید سر بر آر و باز گو
&&
که بخواهم جست از تو مو بمو
\\
قوت پا ایستادن نبودش
&&
که خطاب هیبتی بر جان زدش
\\
پس نشیند قعده زان بار گران
&&
حضرتش گوید سخن گو با بیان
\\
نعمتت دادم بگو شکرت چه بود
&&
دادمت سرمایه هین بنمای سود
\\
رو بدست راست آرد در سلام
&&
سوی جان انبیا و آن کرام
\\
یعنی ای شاهان شفاعت کین لئیم
&&
سخت در گل ماندش پای و گلیم
\\
\end{longtable}
\end{center}
