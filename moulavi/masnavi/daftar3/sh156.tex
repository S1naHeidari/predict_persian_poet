\begin{center}
\section*{بخش ۱۵۶ - وحی آمدن از حق تعالی به موسی کی بیاموزش چیزی کی استدعا کند یا بعضی از آن}
\label{sec:sh156}
\addcontentsline{toc}{section}{\nameref{sec:sh156}}
\begin{longtable}{l p{0.5cm} r}
گفت یزدان تو بده بایست او
&&
برگشا در اختیار آن دست او
\\
اختیار آمد عبادت را نمک
&&
ورنه می‌گردد بناخواه این فلک
\\
گردش او را نه اجر و نه عقاب
&&
که اختیار آمد هنر وقت حساب
\\
جمله عالم خود مسبح آمدند
&&
نیست آن تسبیح جبری مزدمند
\\
تیغ در دستش نه از عجزش بکن
&&
تا که غازی گردد او یا راه‌زن
\\
زانک کرمنا شد آدم ز اختیار
&&
نیم زنبور عسل شد نیم مار
\\
مومنان کان عسل زنبوروار
&&
کافران خود کان زهری همچو مار
\\
زانک مؤمن خورد بگزیده نبات
&&
تا چو نحلی گشت ریق او حیات
\\
باز کافر خورد شربت از صدید
&&
هم ز قوتش زهر شد در وی پدید
\\
اهل الهام خدا عین الحیات
&&
اهل تسویل هوا سم الممات
\\
در جهان این مدح و شاباش و زهی
&&
ز اختیارست و حفاظ آگهی
\\
جمله رندان چونک در زندان بوند
&&
متقی و زاهد و حق‌خوان شوند
\\
چونک قدرت رفت کاسد شد عمل
&&
هین که تا سرمایه نستاند اجل
\\
قدرتت سرمایهٔ سودست هین
&&
وقت قدرت را نگه دار و ببین
\\
آدمی بر خنگ کرمنا سوار
&&
در کف درکش عنان اختیار
\\
باز موسی داد پند او را بمهر
&&
که مرادت زرد خواهد کرد چهر
\\
ترک این سودا بگو وز حق بترس
&&
دیو دادستت برای مکر درس
\\
\end{longtable}
\end{center}
