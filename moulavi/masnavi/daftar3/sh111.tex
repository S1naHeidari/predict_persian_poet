\begin{center}
\section*{بخش ۱۱۱ - حکم کردن داود بر صاحب گاو کی جمله مال خود را به وی ده}
\label{sec:sh111}
\addcontentsline{toc}{section}{\nameref{sec:sh111}}
\begin{longtable}{l p{0.5cm} r}
بعد از آن داود گفتش کای عنود
&&
جمله مال خویش او را بخش زود
\\
ورنه کارت سخت گردد گفتمت
&&
تا نگردد ظاهر از وی استمت
\\
خاک بر سر کرد و جامه بر درید
&&
که بهر دم می‌کنی ظلمی مزید
\\
یک‌دمی دیگر برین تشنیع راند
&&
باز داودش به پیش خویش خواند
\\
گفت چون بختت نبود ای بخت‌کور
&&
ظلمت آمد اندک اندک در ظهور
\\
ریده‌ای آنگاه صدر و پیشگاه
&&
ای دریغ از چون تو خر خاشاک و کاه
\\
رو که فرزندان تو با جفت تو
&&
بندگان او شدند افزون مگو
\\
سنگ بر سینه همی‌زد با دو دست
&&
می‌دوید از جهل خود بالا و پست
\\
خلق هم اندر ملامت آمدند
&&
کز ضمیر کار او غافل بدند
\\
ظالم از مظلوم کی داند کسی
&&
کو بود سخرهٔ هوا همچون خسی
\\
ظالم از مظلوم آنکس پی برد
&&
کو سر نفس ظلوم خود برد
\\
ورنه آن ظالم که نفس است از درون
&&
خصم هر مظلوم باشد از جنون
\\
سگ هماره حمله بر مسکین کند
&&
تا تواند زخم بر مسکین زند
\\
شرم شیران راست نه سگ را بدان
&&
که نگیرد صید از همسایگان
\\
عامهٔ مظلوم‌کش ظالم‌پرست
&&
از کمین سگشان سوی داود جست
\\
روی در داود کردند آن فریق
&&
کای نبی مجتبی بر ما شفیق
\\
این نشاید از تو کین ظلمیست فاش
&&
قهر کردی بی‌گناهی را بلاش
\\
\end{longtable}
\end{center}
