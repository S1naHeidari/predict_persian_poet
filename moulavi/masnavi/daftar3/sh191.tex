\begin{center}
\section*{بخش ۱۹۱ - ملامت کردن اهل مسجد مهمان عاشق را از شب خفتن در آنجا و تهدید کردن مرورا}
\label{sec:sh191}
\addcontentsline{toc}{section}{\nameref{sec:sh191}}
\begin{longtable}{l p{0.5cm} r}
قوم گفتندش که هین اینجا مخسپ
&&
تا نکوبد جانستانت همچو کسپ
\\
که غریبی و نمی‌دانی ز حال
&&
کاندرین جا هر که خفت آمد زوال
\\
اتفاقی نیست این ما بارها
&&
دیده‌ایم و جمله اصحاب نهی
\\
هر که آن مسجد شبی مسکن شدش
&&
نیم‌شب مرگ هلاهل آمدش
\\
از یکی ما تابه صد این دیده‌ایم
&&
نه به تقلید از کسی بشنیده‌ایم
\\
گفت الدین نصیحه آن رسول
&&
آن نصیحت در لغت ضد غلول
\\
این نصیحت راستی در دوستی
&&
در غلولی خاین و سگ‌پوستی
\\
بی خیانت این نصیحت از وداد
&&
می‌نماییمت مگرد از عقل و داد
\\
\end{longtable}
\end{center}
