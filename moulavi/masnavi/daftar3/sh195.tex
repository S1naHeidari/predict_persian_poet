\begin{center}
\section*{بخش ۱۹۵ - گفتن شیطان قریش را کی به جنگ احمد آیید کی من یاریها کنم وقبیلهٔ خود را بیاری خوانم و وقت ملاقات صفین گریختن}
\label{sec:sh195}
\addcontentsline{toc}{section}{\nameref{sec:sh195}}
\begin{longtable}{l p{0.5cm} r}
همچو شیطان در سپه شد صد یکم
&&
خواند افسون که اننی جار لکم
\\
چون قریش از گفت او حاضر شدند
&&
هر دو لشکر در ملاقان آمدند
\\
دید شیطان از ملایک اسپهی
&&
سوی صف مؤمنان اندر رهی
\\
آن جنودا لم تروها صف زده
&&
گشت جان او ز بیم آتشکده
\\
پای خود وا پس کشیده می‌گرفت
&&
که همی‌بینم سپاهی من شگفت
\\
ای اخاف الله ما لی منه عون
&&
اذهبوا انی اری ما لاترون
\\
گفت حارث ای سراقه شکل هین
&&
دی چرا تو می‌نگفتی اینچنین
\\
گفت این دم من همی‌بینم حرب
&&
گفت می‌بینی جعاشیش عرب
\\
می‌نبینی غیر این لیک ای تو ننگ
&&
آن زمان لاف بود این وقت جنگ
\\
دی همی‌گفتی که پایندان شدم
&&
که بودتان فتح و نصرت دم‌بدم
\\
دی زعیم الجیش بودی ای لعین
&&
وین زمان نامرد و ناچیز و مهین
\\
تا بخوردیم آن دم تو و آمدیم
&&
تو بتون رفتی و ما هیزم شدیم
\\
چونک حارث با سراقه گفت این
&&
از عتابش خشمگین شد آن لعین
\\
دست خود خشمین ز دست او کشید
&&
چون ز گفت اوش درد دل رسید
\\
سینه‌اش را کوفت شیطان و گریخت
&&
خون آن بیچارگان زین مکر ریخت
\\
چونک ویران کرد چندین عالم او
&&
پس بگفت این بری منکم
\\
کوفت اندر سینه‌اش انداختش
&&
پس گریزان شد چو هیبت تاختش
\\
نفس و شیطان هر دو یک تن بوده‌اند
&&
در دو صورت خویش را بنموده‌اند
\\
چون فرشته و عقل کایشان یک بدند
&&
بهر حکمتهاش دو صورت شدند
\\
دشمنی داری چنین در سر خویش
&&
مانع عقلست و خصم جان و کیش
\\
یکنفس حمله کند چون سوسمار
&&
پس بسوراخی گریزد در فرار
\\
در دل او سوراخها دارد کنون
&&
سر ز هر سوراخ می‌آرد برون
\\
نام پنهان گشتن دیو از نفوس
&&
واندر آن سوراخ رفتن شد خنوس
\\
که خنوسش چون خنوس قنفذست
&&
چون سر قنفذ ورا آمد شذست
\\
که خدا آن دیو را خناس خواند
&&
کو سر آن خارپشتک را بماند
\\
می نهان گردد سر آن خارپشت
&&
دم‌بدم از بیم صیاد درشت
\\
تا چو فرصت یافت سر آرد برون
&&
زین چنین مکری شود مارش زبون
\\
گرنه نفس از اندرون راهت زدی
&&
ره‌زنان را بر تو دستی کی بدی
\\
زان عوان مقتضی که شهوتست
&&
دل اسیر حرص و آز و آفتست
\\
زان عوان سر شدی دزد و تباه
&&
تا عوانان را به قهر تست راه
\\
در خبر بشنو تو این پند نکو
&&
بیم جنبیکم لکم اعدی عدو
\\
طمطراق این عدو مشنو گریز
&&
کو چو ابلیسست در لج و ستیز
\\
بر تو او از بهر دنیا و نبرد
&&
آن عذاب سرمدی را سهل کرد
\\
چه عجب گر مرگ را آسان کند
&&
او ز سحر خویش صد چندان کند
\\
سحر کاهی را به صنعت که کند
&&
باز کوهی را چو کاهی می‌تند
\\
زشتها را نغز گرداند به فن
&&
نغزها را زشت گرداند به ظن
\\
کار سحر اینست کو دم می‌زند
&&
هر نفس قلب حقایق می‌کند
\\
آدمی را خر نماید ساعتی
&&
آدمی سازد خری را وآیتی
\\
این چنین ساحر درون تست و سر
&&
ان فی الوسواس سحرا مستتر
\\
اندر آن عالم که هست این سحرها
&&
ساحران هستند جادویی‌گشا
\\
اندر آن صحرا که رست این زهر تر
&&
نیز روییدست تریاق ای پسر
\\
گویدت تریاق از من جو سپر
&&
که ز زهرم من به تو نزدیکتر
\\
گفت او سحرست و ویرانی تو
&&
گفت من سحرست و دفع سحر او
\\
\end{longtable}
\end{center}
