\begin{center}
\section*{بخش ۱۴۸ - قصهٔ فریاد رسیدن رسول علیه السلام کاروان عرب را کی از تشنگی و بی‌آبی در مانده بودند و دل بر مرگ نهاده شتران و خلق زبان برون انداخته}
\label{sec:sh148}
\addcontentsline{toc}{section}{\nameref{sec:sh148}}
\begin{longtable}{l p{0.5cm} r}
اندر آن وادی گروهی از عرب
&&
خشک شد از قحط بارانشان قرب
\\
در میان آن بیابان مانده
&&
کاروانی مرگ خود بر خوانده
\\
ناگهانی آن مغیث هر دو کون
&&
مصطفی پیدا شد از ره بهر عون
\\
دید آنجا کاروانی بس بزرگ
&&
بر تف ریگ و ره صعب و سترگ
\\
اشترانشان را زبان آویخته
&&
خلق اندر ریگ هر سو ریخته
\\
رحمش آمد گفت هین زوتر روید
&&
چند یاری سوی آن کثبان دوید
\\
گر سیاهی بر شتر مشک آورد
&&
سوی میر خود به زودی می‌برد
\\
آن شتربان سیه را با شتر
&&
سوی من آرید با فرمان مر
\\
سوی کثبان آمدند آن طالبان
&&
بعد یکساعت بدیدند آنچنان
\\
بنده‌ای می‌شد سیه با اشتری
&&
راویه پر آب چون هدیه‌بری
\\
پس بدو گفتند می‌خواند ترا
&&
این طرف فخر البشر خیر الوری
\\
گفت من نشناسم او را کیست او
&&
گفت او آن ماه‌روی قندخو
\\
نوعها تعریف کردندش که هست
&&
گفت مانا او مگر آن شاعرست
\\
که گروهی را زبون کرد او بسحر
&&
من نیایم جانب او نیم شبر
\\
کش‌کشانش آوریدند آن طرف
&&
او فغان برداشت در تشنیع و تف
\\
چون کشیدندش به پیش آن عزیز
&&
گفت نوشید آب و بردارید نیز
\\
جمله را زان مشک او سیراب کرد
&&
اشتران و هر کسی زان آب خورد
\\
راویه پر کرد و مشک از مشک او
&&
ابر گردون خیره ماند از رشک او
\\
این کسی دیدست کز یک راویه
&&
سرد گردد سوز چندان هاویه
\\
این کسی دیدست کز یک مشک آب
&&
گشت چندین مشک پر بی اضطراب
\\
مشک خود روپوش بود و موج فضل
&&
می‌رسید از امر او از بحر اصل
\\
آب از جوشش همی‌گردد هوا
&&
و آن هوا گردد ز سردی آبها
\\
بلک بی علت و بیرون زین حکم
&&
آب رویانید تکوین از عدم
\\
تو ز طفلی چون سببها دیده‌ای
&&
در سبب از جهل بر چفسیده‌ای
\\
با سببها از مسبب غافلی
&&
سوی این روپوشها زان مایلی
\\
چون سببها رفت بر سر می‌زنی
&&
ربنا و ربناها می‌کنی
\\
رب می‌گوید برو سوی سبب
&&
چون ز صنعم یاد کردی ای عجب
\\
گفت زین پس من ترا بینم همه
&&
ننگرم سوی سبب و آن دمدمه
\\
گویدش ردوا لعادوا کار تست
&&
ای تو اندر توبه و میثاق سست
\\
لیک من آن ننگرم رحمت کنم
&&
رحمتم پرست بر رحمت تنم
\\
ننگرم عهد بدت بدهم عطا
&&
از کرم این دم چو می‌خوانی مرا
\\
قافله حیران شد اندر کار او
&&
یا محمد چیست این ای بحر خو
\\
کرده‌ای روپوش مشک خرد را
&&
غرقه کردی هم عرب هم کرد را
\\
\end{longtable}
\end{center}
